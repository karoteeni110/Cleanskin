%!TEX root =  HuangS16_QIP17.extended_abstract.tex
\noindent
Cryptographic hash functions are fundamental primitives widely used in practice.
For such a function $f:\{0, 1\}^n\to\{0, 1\}^m$,
it is nearly impossible for an adversary to produce the hash $f(x)$ without knowing the secret message 
$x\in\{0, 1\}^n$. Unfortunately, all hash functions are vulnerable under the side-channel attack, which is a
grave concern for information security in practice. This is because typically $m\ll n$ and an adversary needs only $m$ bits of information to pass the verification test.

In sharp contrast, we show that when quantum states are used, the leakage allowed can be almost the entire secret.  More precisely, we call a function that maps $n$ bits to $m$ qubits a quantum cryptographic function
if the maximum fidelity between two distinct hashes is negligible in $n$.
We show that for any $k=n-\omega(\log n)$, all quantum cryptographic hash functions remain 
cryptographically secure when leaking $k$ bits of information.
By the quantum fingerprinting constructions of Buhrman et al.~({\em Phys. Rev. Lett.} 87, 167902), 
for all $m=\omega(\log n)$, there exist such quantum cryptographic hash functions. We also show that one only needs $\omega(\log^2 n)$ qubits to verify a quantum cryptographic hash, rather than the whole classical information needed to generate one.

Our result also shows that to approximately produce a small amount of quantum side information on a classical secret, it may require almost the full information of the secret. 
This large gap represents a significant barrier for proving quantum security of classical-proof extractors.

