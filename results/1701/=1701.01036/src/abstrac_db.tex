\begin{abstract}
%Although Deep Convolutional Neural Networks (CNNs) have liberated their power in various computer vision tasks, the most important components of CNN, convolutional layers and fully connected layers, are still limited to linear transformations. In this paper, we propose a novel Factorized Bilinear (FB) layer to model the pairwise feature interactions by considering the quadratic terms in the transformations. Compared with existing methods that tried to incorporate complex non-linearity structures into CNNs, the factorized parameterization makes our FB layer only require a linear increase of parameters and affordable computational cost. To further reduce the risk of overfitting of the FB layer, a specific remedy called DropFactor is devised during the training process. We also analyze the connection between FB layer and some existing models, and show FB layer is a generalization to them. Finally, we validate the effectiveness of FB layer on several widely adopted datasets including CIFAR-10, CIFAR-100 and ImageNet, and demonstrate superior results compared with various state-of-the-art deep models.

Neural Style Transfer~\cite{neuralart} has recently demonstrated very exciting results which catches eyes in both academia and industry. Despite the amazing results, the principle of neural style transfer, especially why the Gram matrices could represent style remains unclear. In this paper, we propose a novel interpretation of neural style transfer by treating it as a domain adaptation problem. Specifically, we theoretically show that matching the Gram matrices of feature maps is equivalent to minimize the Maximum Mean Discrepancy (MMD) with the second order polynomial kernel. Thus, we argue that the essence of neural style transfer is to match the feature distributions between the style images and the generated images. To further support our standpoint, we experiment with several other distribution alignment methods, and achieve appealing results. We believe this novel interpretation connects these two important research fields, and could enlighten future researches.

\end{abstract}