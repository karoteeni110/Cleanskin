\documentclass[oneside,english]{amsart}

\usepackage[latin1]{inputenc}
\pagestyle{plain}
%\usepackage{setspace}
%\onehalfspacing
\usepackage{amssymb}

\usepackage{amssymb}
\usepackage[]{epsf,epsfig,amsmath,amssymb,amsfonts,latexsym}

\newtheorem{thm}{Theorem}[section]
\newtheorem{lem}[thm]{Lemma}
\newtheorem{prop}[thm]{Proposition}
\newtheorem{cor}[thm]{Corollary}

%\theoremstyle{remark}
\theoremstyle{definition}
\newtheorem{defn}[thm]{Definition}
\newtheorem{remark}[thm]{Remark}
\newtheorem{example}[thm]{Example}
\newtheorem{question}[thm]{Question}
\newtheorem{problem}[thm]{Problem}


\newcommand{\AAA}{\mathcal{A}}
\newcommand{\ZZ}{\mathbb{Z}}
\newcommand{\NN}{\mathbb{N}}
\newcommand{\RR}{\mathbb{R}}
\newcommand{\ZD}{\mathbb{Z}^d}
\newcommand{\GG}{\mathbb{G}}
\newcommand{\HH}{\mathbb{H}}
\newcommand{\FF}{\mathbb{F}}
\newcommand{\LL}{\mathcal{L}}
\newcommand{\rank}{\mathit{rank}}
\newcommand{\Aut}{\mathit{Aut}}
\newcommand{\Mrk}[1]{\mathit{Mrk}({#1})}
\newcommand{\Mon}[1]{\mathit{Mon}({#1})}
\newcommand{\Per}[1]{\mathit{Per}{#1}}
\newcommand{\PA}[1]{\mathit{PA}(#1)}
\newcommand{\IRS}{\mathit{IRS}}
\newcommand{\stab}{\mathit{stab}}
\newcommand{\qstab}{\mathit{qstab}}
\newcommand{\supp}{\mathit{supp}}
\newcommand{\T}{\mathfrak{T}}
\newcommand{\SV}{\mathit{SV}}
\newcommand{\dT}{\rho_\infty}
\newcommand{\act}[2]{{#1} \curvearrowright {#2}}
\newcommand{\Act}[2]{\mathit{Act}({#1},{#2})}
\newcommand{\Alg}[2]{\mathit{Act}_{\mathit{alg}}({#1},{#2})}
\newcommand{\ActG}[2]{\mathit{Act}_{\mathit{grp}}({#1},{#2})}
\newcommand{\subG}[2]{\mathit{Sub}_{{#1}}({#2})}

\newcommand{\DCC}{decending chain condition}
\newcommand{\POT}{pseudo-orbit tracing}
\newcommand{\cov}{\mathit{cov}}
\newcommand{\sep}{\mathit{sep}}
\newcommand{\SFT}[2]{({#1})^{(#2)}}

\title{Pseudo-Orbit Tracing and Algebraic actions of countable amenable groups}


\author{Tom Meyerovitch}
\address{Tom Meyerovitch\\
Department of Mathematics\\
Ben-Gurion University of the Negev}
\email{mtom@math.bgu.ac.il}
\thanks{The research leading to these results has received funding from the People Programme (Marie Curie Actions) of the European Union's Seventh Framework Programme (FP7/2007-2013) under REA grant agreement no. 333598
 and from the Israel  Science Foundation (grant no. 626/14)}
\keywords{Algebraic actions, expansiveness, group actions, pseudo-orbit tracing property, subshift of finite type}
\renewcommand{\subjclassname}{MSC 2000}
\subjclass[2000]{22D40,37B05, 37B40}

\begin{document}

\maketitle

\begin{abstract}
Consider a countable amenable group acting by  homeomorphisms on a compact metrizable space.
Chung and Li asked if expansiveness and positive entropy of the action imply existence of an off-diagonal asymptotic pair.  For algebraic actions of polycyclic-by-finite groups, Chung and Li proved it does.
We provide examples showing that Chung and Li's result is near-optimal in the sense that  the conclusion fails for some non-algebraic action generated by a single homeomorphism, and for some algebraic actions of non-finitely generated abelian groups.
On the other hand,
we prove that every expansive action of an amenable group with positive entropy that has  the \POT~   property must admit  off-diagonal
asymptotic pairs.
 Using  Chung and Li's  algebraic characterization of expansiveness,  we prove  the \POT~   property for  a class of expansive algebraic actions.
This class includes every expansive principal algebraic action of an arbitrary countable group.
%We also construct  expansive homeomorphisms  and   expansive algebraic  actions  of %(non-finitely generated) abelian groups having positive topological entropy and no off-%diagonal
%asymptotic pairs. %and finitely generated solvable groups
%This complements a result and answers a question of Chung and Li.

\end{abstract}

\section{Introduction}
This paper is partly motivated by relatively recent work of Chung and Li \cite{MR3314515}
%,  an important contribution in the understanding of
about the dynamics of countable subgroups of automorphisms for compact groups, and algebraic actions in particular.
Part of the paper \cite{MR3314515} is an investigation of  the relation between topological entropy
 and asymptotic behavior of orbits for actions $\act{\Gamma}{X}$ where $X$ is a compact metrizable group and $\Gamma$ acts by group automorphisms \cite{MR3314515}.
The following question  was posed  and left open by Chung and Li  \cite{MR3314515}:

\begin{question}\label{ques:Chung_Li} %[ \cite{MR3314515},Question $1.1$]
Let  a countable amenable group $\Gamma$  act by  homeomorphisms  on  a compact
metric space $X$. Suppose the action   $\act{\Gamma}{X}$ is expansive and has positive topological entropy.  Must there be an off-diagonal
asymptotic pair in $X$?
\end{question}
Under the additional assumptions that  $\Gamma$ is  a  polycyclic-by-finite group, that $X$ is a compact abelian  group and  that $\act{\Gamma}{X}$ is an expansive action of $\Gamma$  by automorphisms,  Chung and Li obtained an affirmative answer to Question \ref{ques:Chung_Li} \cite[Theorem $1.2$]{MR3314515}.
On the other hand, as remarked in \cite{MR3314515}, there is a much older example due to Lind and Schmidt    of non-expansive $\mathbb{Z}$-actions (in fact, toral automorphisms) with positive entropy where  all asymptotic pairs are on the diagonal \cite[Expample $3.4$]{MR1678035}.

In addition to expansiveness, the two main assumptions in \cite[Theorem $1.2$]{MR3314515} concern  the algebraic nature of the action (action by automorphisms on  a compact group) and the group theoretic condition on $\Gamma$ (polycyclic-by-finite).
We show that these assumptions are not just artifacts of the proof method. Specifically, we prove:

\begin{thm}\label{thm:alg_exp_pos_ent_no_asymp}
There exists a totally disconnected compact abelian group $X$ and a countable amenable subgroup  $\Gamma \subset \Aut(X)$ so that $\act{\Gamma}{X}$ is expansive, has positive topological entropy and no off-diagonal
asymptotic pairs. In particular,  the group $\Gamma$ can be an abelian group (for instance $\bigoplus_{n \in \mathbb{Z}}(\mathbb{Z}/2\mathbb{Z})$).
%,or a finitely generated solvable group (for instance the lamplighter group  $\mathbb{Z} \wr (\mathbb{Z}/2\mathbb{Z})$).
\end{thm}


\begin{thm}\label{thm:exapnsive_pos_entropy_no_asymp}
There exists an expansive homeomorphism $T:X \to X$ of a totally disconnected compact metrizable space $X$ with positive topological entropy and no off-diagonal
asymptotic pair.
\end{thm}
Theorem \ref{thm:exapnsive_pos_entropy_no_asymp} might be considered a distant cousin of a striking result due to Ornstein and Weiss  stating that every transformation is bilaterally deterministic \cite{MR0382600}.



In attempt to understand when Chung and Li's question has an affirmative answer, we are led to  explore the \POT~ property. This fundamental dynamical property turns out to  ensure an affirmative answer to Question \ref{ques:Chung_Li}.
The \POT~ property was first introduced and studied by R. Bowen \cite{MR0482842} for $\mathbb{Z}$-actions, motivated by the study of Axiom A maps.
Walters and others continued this study and obtained further consequences of \POT~  \cite{MR518563}.
%: ``Rufus Bowen has stated that the tracing of pseudo orbits is the
%most important dynamical property of Axiom A maps''.
Chung and Lee recently considered the pseudo-orbit tracing property for actions of (finitely generated) countable amenable groups and showed that topological stability and other important consequences of  the \POT~ property  hold  in this more general setting \cite{1611.08994}.
In relation with Question \ref{ques:Chung_Li} above we have the following:

\begin{thm}\label{prop:finite_type_entropy_off_diagonal}
Let $\Gamma$ be a countable amenable group.  % and let $X$  be a compact metrizable space.
% If $\alpha \in \Act{\Gamma}{X}$  is expansive,
Every expansive $\Gamma$-action on a compact  metrizable  space that satisfies the \POT~ property and has positive topological entropy admits an off-diagonal  asymptotic pair. % in $X\times X$.
\end{thm}

%As an application to algebraic actions, we show the following:
The \POT~ property is of interest in the context of algebraic actions.
A particular instance of one of our result applies to \emph{principal algebraic actions}, a well-known class of algebraic actions:
\begin{thm}\label{thm:priniciple_alg_pot}
Let $\Gamma$ be a countable group. Every expansive principal algebraic $\Gamma$-action satisfies the \POT~ property.
\end{thm}

In the paper \cite{MR3314515} Chung and Li also provided an  affirmative answer to Question \ref{ques:Chung_Li} for principal algebraic actions for any countable amenable group $\Gamma$ (in fact they proved much more, see Remark \ref{rem:princ_alg} below).
This is also a  conclusion of Theorem \ref{thm:priniciple_alg_pot} combined with Theorem \ref{prop:finite_type_entropy_off_diagonal}.

The organization of the paper is as follows: In section \ref{sec:pot} we recall the \POT~ property, subshifts and subshifts of finite type in particular. We also prove Theorem \ref{prop:finite_type_entropy_off_diagonal}. In section \ref{sec:pot_alg} we discuss the pseudo-orbit tracing for algebraic actions and derive some consequences, in particular the proof of  Theorem \ref{thm:priniciple_alg_pot}. In section \ref{sec:algebraic_exp_no_asymp}
 we prove Theorem \ref{thm:alg_exp_pos_ent_no_asymp}, providing a negative answer to Question \ref{ques:Chung_Li} within the class of algebraic actions. In section \ref{sec:positive_exp_no_asymp}
we prove Theorem \ref{thm:exapnsive_pos_entropy_no_asymp}.

\textbf{Acknowledgements:} I thank Nishant Chandgotia, Nhan-Phu Chung and  Hanfeng Li for valuable comments on an early version of this paper.

\section{The \POT~ property for actions of countable groups}\label{sec:pot}

Throughout this paper, $\Gamma$ will be a countable group and $X$ will be a compact metrizable space, equipped with a metric $d:X\times X \to \mathbb{R}_+$.
We denote by $\Act{\Gamma}{X}$ the space of all continuous actions  $\act{\Gamma}{X}$. The space $\Act{\Gamma}{X}$ inherits a Polish topology  from $\mathit{Homeo}(X)^\Gamma$. Specifically,  given an enumeration of $\Gamma = \{g_1,g_2 \ldots\}$, we have the  following metric on $\Act{\Gamma}{X}$:
$$ \rho(\alpha,\beta) := \sum_{n=1}^\infty \frac{1}{2^n} \sup_{x \in X}d\left(\alpha_{g_n}(x),\beta_{g_n}(x)\right).$$
% being a closed  subset of $\mathit{Homeo}(X)^\Gamma$, is  a polish space *** Check! ***.
In this paper  when the action $\alpha \in \Act{\Gamma}{X}$ is clear from the context we will sometimes  write $g\cdot x$ instead of $\alpha_g(x)$.


\begin{defn}(Compare with \cite[Definition $2.5$]{1611.08994})
Fix $S \subset \Gamma$ and $\delta >0$.
A \emph{$(S,\delta)$ pseudo-orbit}  for $\alpha \in \Act{\Gamma}{X}$ is a $\Gamma$-sequence $(x_g)_{g \in \Gamma}$ in $X$ such that $d(\alpha_s(x_{g}),x_{sg})< \delta$ for all $s \in S$ and $g \in \Gamma$.
We say that a  pseudo-orbit $(x_g)_{g \in \Gamma}$ is \emph{$\epsilon$-traced} by $x \in X$ if $d(\alpha_g(x),x_g) <\epsilon$ for all $g \in \Gamma$.
\end{defn}

\begin{defn}
An action $\alpha \in \Act{\Gamma}{X}$ has the \emph{\POT~ property (abbreviated by p.o.t.)} if for every $\epsilon >0$ there exists $\delta >0$ and a finite set $S \subset \Gamma$ such that every $(S,\delta)$ pseudo-orbit is $\epsilon$-traced by some point $x \in X$.
\end{defn}

\begin{defn}(Compare with \cite[Definition $2.1$]{1611.08994})
We say that $\alpha \in \Act{\Gamma}{X}$ is \emph{topologically stable} if
for every $\epsilon >0$
there exists an open neighborhood $\mathcal{U} \subset \Act{\Gamma}{X}$ of $\alpha$ such that for every $\beta \in \mathcal{U}$ there exists a continuous $f:X \to X$ so that $\alpha_g \circ f= f \circ \beta_g$ for every $g \in \Gamma$ and
$$\sup_{x \in X}d(f(x),x) \le \epsilon.$$
\end{defn}

Chung and Lee \cite[Theorem $2.8$]{1611.08994} proved the following:

\begin{thm}\label{thm:pot_stable}
If an action $\alpha \in \Act{\Gamma}{X}$ is expansive and satisfies p.o.t, then it is topologically stable.
Moreover if $\eta >0$ is an expansive constant for $\alpha$ then:
\begin{enumerate}
\item  For every $0 < \epsilon < \eta$ there exists an open neighborhood $\alpha \in \mathcal{U} \subset \Act{\Gamma}{X}$ so that for every $\beta \in \mathcal{U}$ there exists a unique map $f:X \to X$ so that $\alpha_g \circ f = f \circ \beta_g$ for every $g \in \Gamma$ and $\sup_{x \in X}d(f(x),x) \le \epsilon$.
\item If furthermore $\beta$ as above is expansive with expansive constant  $2\epsilon$, then the conjugating map $f$ above is injective.
\end{enumerate}
\end{thm}

Here is a quick sketch of  proof for Theorem \ref{thm:pot_stable}: If $\beta \in \Act{\Gamma}{X}$ is sufficiently close to $\alpha$ then every $\beta$-orbit is an $(S,\delta)$ pseudo-orbit for $\alpha$. By p.o.t this pseudo-orbit is  $\epsilon$-traced by the $\alpha$-orbit of some point $y$. If $\epsilon$ is sufficiently small, expansiveness of $\alpha$ implies that $y$ as above is unique.  Furthermore, expansiveness implies that the function $f:X \to X$ sending a point $x \in X$ to the unique point $y$ whose $\alpha$-orbit traces the $\beta$-orbit of $x$ is continuous.
Uniqueness of the map $f$  implies it is $\Gamma$-equivariant.
If $2\epsilon$ is an expansive constant for $\beta$, then it is impossible for a single $y$ to $\epsilon$-trace  the  $\beta$-orbits of two distinct  points, so $f$ is injective.

\begin{remark}
Strictly speaking,
 Chung and Lee restricted attention to finitely generated groups in \cite{1611.08994}.
The extension to general countable groups does not require any new ideas.
Note that the following simplification  occurs in the finitely generated case: If $\Gamma$ is generated by a finite set $S$, $\alpha \in \Act{\Gamma}{X}$ satisfies p.o.t if  for every $\epsilon >0$ there exists $\delta >0$   such that every $(S,\delta)$ pseudo-orbit is $\epsilon$-traced by some point $x \in X$.
\end{remark}



\begin{defn}
Given $\alpha \in \Act{\Gamma}{X}$,
 $(x,y) \in X \times X$ is called a \emph{$\alpha$-asymptotic pair} if $\lim_{\Gamma \ni g \to \infty}d(\alpha_g(x),\alpha_g(y))=0$. In other words, if for every $\epsilon>0$ there are at most finitely many  $g \in \Gamma$ so that $d(\alpha_g(x),\alpha_g(y)) >  \epsilon$.
\end{defn}

\begin{proof}[Proof of Theorem \ref{prop:finite_type_entropy_off_diagonal}]
Suppose  $\alpha \in \Act{\Gamma}{X}$  is expansive, satisfies p.o.t and has positive topological entropy.

%Let $\mathcal{U}$ be a cover of finite type for $\act{\Gamma}{X}$ with window $W \subset \Gamma$ as in Definition \ref{def:gen_finite_type}.
Fix  $\epsilon >0$ so that $2\epsilon$ is an expansive constant for $\alpha$. Because $\alpha$ satisfies p.o.t there exists $\delta >0$ and a finite set $S \subset \Gamma$ so that every $(\delta,S)$ pseudo-orbit is $\epsilon/2$-traced by some $x \in X$. By further increasing $S$ we can safely assume that $S=S^{-1}$ and that $S$ contains the identity.


Let
$(F_n)_{n=1}^\infty$ be a left-F{\o}lner sequence in $\Gamma$. Denote
$$\partial_S F_n:= \left\{ g \in \Gamma~:~ Sg \cap F_n \ne \emptyset \mbox{ and } Sg \cap F_n^c \ne \emptyset\right\}.$$
Because $(F_n)_{n=1}^\infty$ is a left-F{\o}lner sequence in $\Gamma$, it follows that
\begin{equation}\label{eq:partial_F_n}
\lim_{n \to \infty}\frac{|\partial_S F_n|}{|F_n|} = 0.
\end{equation}


Given a finite $F \subset \Gamma$ and $\delta >0$, a set $Y \subset X$ is  \emph{$(F,\delta)$-separated} if
$$\max_{g\in F}d(\alpha_g(x),\alpha_g(y)) \ge \delta \mbox{ for every distinct } x,y \in Y.$$
Let $\sep_{\delta,F}(X,d)$ denote the maximal cardinality of an $(F,\delta)$-separated set in $X$.

Standard argument give that for every finite $F \subset \Gamma$ and every $\delta>0$ the following holds:
$$ \log \sep_{\delta,F}(X,d) \le |F| \log \sep_{\delta/2,\{1\}}(X,d).$$

Thus by \eqref{eq:partial_F_n}:
\begin{equation}\label{eq:sep_bd_small_o}
\log \sep_{\delta,\partial_S F_n}(X,d) = o(|F_n|),
\end{equation}

For every $n >0$, let $X_n \subset X$ be an $(2\epsilon,F_n)$-separated set of  maximal cardinality.
Because $2\epsilon$ is an expansive constant for $\alpha$, the topological entropy of $\alpha$ is equal to
\begin{equation}\label{eq:h_sep_pos}
h(\alpha) =\lim_{n \to \infty}\frac{1}{|F_n|}\log|X_n| >0.
\end{equation}

By \eqref{eq:sep_bd_small_o} and \eqref{eq:h_sep_pos},
$$  \sep_{\delta,\partial_S F_n}(X,d) = o(|X_n|).$$
In particular, for large enough $n$ there exists distinct $x,x' \in X_n$  so that
\begin{equation}
\max_{g \in \partial_S F_n}d(\alpha_g(x),\alpha_g(x')) < \delta.
\end{equation}

Define $(y_g)_{g \in \Gamma} \in X^\Gamma$ as follows:

\begin{equation}
y_g = \begin{cases}
\alpha_g(x') & g \in F_n\\
\alpha_g(x) &  g \in \Gamma \setminus F_n
\end{cases}
\end{equation}

Then $(y_g)_{g \in \Gamma}$ is a $(S,\delta)$ pseudo-orbit for $\alpha$, and by p.o.t it is $\epsilon$-traced by some $y \in X$.
Thus
$d(\alpha_g(x),\alpha_g(y)) < \epsilon$ for every $g \in \Gamma \setminus F_n$.
This implies that $(x,y)$ is an $\alpha$-asymptotic pair.
Also
$$\max_{F \in F_n}d(\alpha_g(x'),\alpha_g(y)) < \epsilon,$$
Because $\{x,x'\}$ are $(F_n,2\epsilon)$-separated, it follows that
there exists $g \in F_n$ so that $d(\alpha_g(x'),\alpha_g(x)) > 2\epsilon$.
By the triangle inequality, there exists $g \in \Gamma$ so that $d(\alpha_g(x),\alpha_g(y))>\epsilon$, and in particular $x \ne y$.
\end{proof}

%The standard definition of subshifts of finite type uses a finite list of forbidden patterns. It is obvious from this definition that there are at most countably many subshift of finite type for %each countable group $\Gamma$. Again, the same result holds outside the symbolic setting:
%\begin{prop}\label{prop:POT_countable}
%Let $X$ be a compact metrizable space then the set of expansive $\alpha \in \Act{\Gamma}{X}$ that satisfy p.o.t  is at most countable.
%\end{prop}


%Our main result concerns principal algebraic actions:
%\begin{thm}\label{thm:priniciple_alg_pot}
%Let $\Gamma$ be a countable group. Every expansive principal algebraic $\Gamma$-action satisfies p.o.t.
%\end{thm}


%\section{Symbolic dynamical systems, shifts of finite type and locally maximal %subsystems}
We now recall a class of $\Gamma$-actions called  \emph{$\Gamma$-subshifts}. These are also referred to as \emph{symbolic dynamical systems}.% \cite{}.

\begin{defn}
Let $\AAA$ be a discrete finite set. Consider $\AAA^\Gamma$ as a (metrizable) topological space with the product topology.
The (left) shift action $\sigma \in \Act{\Gamma}{\AAA^\Gamma}$ is given by: %
%For ease of notation, we will consider the following $G$ action  $\act{\Gamma}{\AAA^\Gamma}$ from the right:
\begin{equation}
\sigma_g \cdot (x)_h = x_{g^{-1}h}.
\end{equation}

%, either from the right or from the left (these actions are isomorphic to each other).
%The corresponding left-action is requires taking inverses, and the two are clearly isomorphic.

The pair $(\AAA^\Gamma,\sigma)$ is called  the \emph{full shift} with alphabet $\AAA$ over the group $\Gamma$.
A \emph{$\Gamma$-subshift} is a subsystem of a full shift. In other words, the dynamical systems $(X,\sigma_X)$ where action $\sigma_X := \sigma \mid_X \in \Act{\Gamma}{X}$,
 where $X \subset \AAA^\Gamma$ is closed $\sigma$-invariant subset of $\AAA^\Gamma$.
\end{defn}

Evidently, every $\Gamma$-subshift is expansive. It is also well known that  every expansive action $\act{\Gamma}{X}$ on a totally disconnected compact metrizable space $X$ is isomorphic to a $\Gamma$-subshift. An action  $\act{\Gamma}{X}$ that is isomorphic to a $\Gamma$-subshift is sometimes called a \emph{symbolic dynamical system}. From this abstract point of view, symbolic dynamics is the study of expansive actions on a totally disconnected compact metrizable space.

 Let us recall a class of systems called \emph{subshifts of finite type},
%Subshifts of finite type are
arguably the most important class of systems in symbolic dynamics.  %We recall the definition:
\begin{defn}\label{def:SFT}
 A \emph{$\Gamma$-subshift of finite type} (\emph{$\Gamma$-SFT}) is a subshift $(X,\sigma_x)$ of the  form: %with the following property:
\begin{equation}\label{eq:SFT}
X = \left\{x \in \AAA^\Gamma~:~ (g \cdot x)\mid_F \in L \mbox{ for every } g \in \Gamma \right\},
\end{equation}
where  $F \subset \Gamma$ is a finite set and $L \subset \AAA^F$.
\end{defn}

Subshifts of finite type over a countable group $\Gamma$ can be characterized as the set of  expansive $\alpha \in \Act{\Gamma}{X}$   that satisfy p.o.t, where $X$ is a  totally disconnected compact metrizable space \cite{1611.08994,MR2353915,MR518563}.
Another characterization of  subshifts of finite type is given  in terms  of a certain descending chain condition that is reminiscent of the definition of Noetherian rings \cite{MR3493309}.
A variant of this descending chain condition appeared back in the early work of Kitchens and Schmidt on automorphisms of compact groups \cite{MR1036904}.
\begin{remark}
Theorem \ref{prop:finite_type_entropy_off_diagonal} is a direct extension of the classical observation that every subshift of finite type with positive entropy has an off-diagonal  asymptotic pair \cite[Proposition $2.1$]{MR1359979}.
\end{remark}
As we will now see,
in many respects, expansive actions on a compact metrizable space $X$ that satisfy p.o.t can be thought of as ``systems of finite type'', even when $X$ is not totally disconnected.

\begin{defn}
Suppose $X$ is a compact metric space and  $\alpha \in \Act{\Gamma}{X}$. Let
\begin{equation}\label{eq:subG}
\subG{\alpha}{X} := \left\{ Y \subset X ~:~ Y \mbox{ is closed and } \alpha_g(Y) =  Y \mbox{ for every } g \in \Gamma \right\}.
\end{equation}
The Hausdorff metric on the closed subsets of $X$ induces a topology on  $\subG{\alpha}{X}$ that makes it a compact metric space.
We say that $Y \in \subG{\alpha}{X}$ is a \emph{local maximum} if there exists an open neighborhood $\mathcal{U} \subset \subG{\alpha}{X}$  of $Y$ such that $Z \subseteq Y$ for every $Z \in \mathcal{U}$.
\end{defn}


\begin{prop}\label{prop:finite_type}
Let $\alpha \in \Act{\Gamma}{X}$ be expansive and satisfy p.o.t.
If $\beta \in \Act{\Gamma}{Y}$ is expansive and   $\Phi:X \to Y$ is continuous, $\Gamma$-equivariant and  injective then $\Phi(X)\in \subG{\beta}{Y}$ is a local maximum in $\subG{\beta}{Y}$.
%Then the following equivalent properties hold:
%\begin{enumerate}
%\item \label{en:SFT_1} The action $T$ satisfies the stable intersection property.
%  superis of finite type in the sense of Definition \ref{def:exp_SFT}.*** Maybe this and $(4)$ are weaker than $(2)$ and $(3)$? ***
%\item  For every compact metric space $Y$ and every expansive  $S \in \Act{\Gamma}{Y}$, if  $\Phi:X \to Y$ is $\Gamma$-equivariant and  injective, $\Phi(X)$ is a local maximum in %$\subG{Y}$.
%\item \label{en:SFT_3} There exists a  finite type topological generator $\mathcal{U}$ for $X %\curvearrowleft G$, as in Definition \ref{def:gen_finite_type}.
%\end{enumerate}
\end{prop}



\begin{proof}
Let $\alpha \in \Act{\Gamma}{X}$, $\beta \in \Act{\Gamma}{Y}$ and $\Phi:X \to Y$ be as above.
Then $\beta\mid_{\Phi(X)} \in \Act{\Gamma}{\Phi(X)}$ is isomorphic to $\alpha$ and in particular satisfies p.o.t.
Let $\epsilon$ be an expansive constant for $\beta$. Choose  a finite $S \subset \Gamma$ and $\delta  \in (0,\epsilon)$ so that every $(S,\delta)$ pseudo-orbit for $\beta\mid_{\Phi(X)}$ is $\epsilon/2$-traced by some $y \in \Phi(X)$.
Assume with out loss of generality that $ 1 \in S$.
Now let
\begin{equation}
\mathcal{U}:= \left\{Z \in \subG{\beta}{Y}:~ \sup_{z \in Z}\inf_{ x \in X}\max_{g \in S}d(\beta_{g}(z),\beta_g(\Phi(x))) < \delta/2 \right\}.
\end{equation}

Then $\mathcal{U}$ is an open neighborhood of $\Phi(X)$ in $\subG{\beta}{Y}$.
Now if $Z \in \mathcal{U}$, then by definition of $\mathcal{U}$,
for every $z \in Z$ there is $(y_g)_{g \in \Gamma} \in \Phi(X)^\Gamma$  so that
$$d(\beta_{hg}(z),\beta_{h}(y_g)) < \delta/2 \mbox{ for every } g \in \Gamma,~ h \in S.$$
It follows that  $d(\beta_s(y_{g}),y_{sg})< \delta$ for all $s \in S$ and $g \in \Gamma$, so $(y_g)_{g \in \Gamma}$ is an $(S,\delta)$ pseudo-orbit for $\beta\mid_{\Phi(X)}$.
Thus it is $\epsilon/2$ traced by some  $y \in \Phi(X)$. But for every $g \in \Gamma$ we have
$$d(\beta_g(y),\beta_g(z))< d(\beta_g(y),y_g)+ d(y_g,\beta_g(z)) < \epsilon.$$
It follows that $z = y$, so $z \in \Phi(X)$. It follows that $Z \subseteq \Phi(X)$.

\end{proof}

\begin{remark}\label{rem:stable_intersection}
A subsystem $Z \in \subG{\alpha}{Y}$ is a local maximum if and only if  it satisfies the following \emph{stable intersection property
}: For every decreasing sequence $(Y_n)_{n=1}^\infty \in \subG{\alpha}{Y}^{\mathbb{N}}$
\begin{equation}\label{eq:dec_chain}
Y \supseteq Y_1 \ldots \supseteq Y_n \supseteq Y_{n+1} \supseteq \ldots
% \ldots \subseteq Y_{n+1} \subseteq Y_n \subseteq \ldots \subset Y_1 \subseteq Y
\end{equation}
such that $Z = \bigcap_{n=1}^\infty Y_n$,
 there exists $N \in\mathbb{N}$ so that $Y_{n}=Y_N$ for all $n \ge N$.
The equivalence of these two conditions follows because
the relation $\subseteq$ is closed in $\subG{\alpha}{Y} \times \subG{\alpha}{Y}$.
The observation that the stable intersection property characterizes subshifts of finite type is due to K. Schmidt \cite{MR3493309}.
\end{remark}

\section{Pseudo-orbit tracing  for  algebraic actions}\label{sec:pot_alg}
In this section we discuss the pseudo-orbit tracing for algebraic actions and some consequences.
We derive Theorem \ref{thm:priniciple_alg_pot}  as a particular case of  Theorem \ref{thm:X_A_SFT} below, thus establishing \POT~ for a class of algebraic actions.

Let $X$ be a  compact metrizable  \emph{abelian} group.
We  denote by $\Alg{\Gamma}{X} \subset \Act{\Gamma}{X}$  the collection of  $\Gamma$-actions on $X$  by continuous group automorphisms.
%The elements of
Every  $\alpha \in \Alg{\Gamma}{X}$ is  called an \emph{algebraic action}.
%We use the notation and conventions
We recall some notation, essentially following \cite{MR3314515}:

Denote by $\mathbb{Z}\Gamma$  the group ring of $\Gamma$. Denote by $\ell^\infty(\Gamma)$ the Banach space of all bounded $\mathbb{R}$-valued functions
on $\Gamma$, equipped with the $\|\cdot\|_\infty$-norm. Also,  denote  by $\ell^1(\Gamma)$  the Banach
algebra of all absolutely summable $\mathbb{R}$-valued functions on $\Gamma$, equipped with the
$\ell^1$-norm $\|\cdot\|_1$ and the involution $f \mapsto f^*$ defined by $\left(\sum_{s \in \Gamma}f_ss \right)^* := \sum_{s \in \Gamma}f_s s^{-1}$.

For $k \in \mathbb{N}$ and $p \in[1,\infty]$, we write $\ell^p(\Gamma,\mathbb{R}^k):= (\ell^p(\Gamma))^k$, equipped with the suitable $\|\cdot\|_p$-norm.
We denote by $\ell^p(\Gamma,\ZZ^k)$ the integer valued elements of $\ell^p(\Gamma,\mathbb{R}^k)$.
For $k \in \mathbb{N}$, let $M_k(\ell^1(\Gamma))$ denote the Banach algebra of $k \times k$ matrices with $\ell^1(\Gamma)$-entries, with the norm
$$ \| (f_{i,j})_{ 1\le i,j \le n}\|_1 := \sum_{1\le i,j \le n} \| f_{i,j}\|_1.$$
%Each $f \in M_k(\ell^1(\Gamma))$ is naturally identified with an $M_k(\mathbb{R})$-valued function on $\Gamma$ $g \mapsto f_g$ so that  the projection onto every coordinate is in %$\ell^1(\Gamma)$.
%, equipped with the natural $\ell^1$-norm that we also denote by $\|\cdot \|_1$ and
The involution on $\ell^1(\Gamma)$ also extends naturally to an isometric linear involution on $M_k(\ell^1(\Gamma))$ given by
$$ (f_{i,j})^*_{1 \le i \le k,\; 1 \le i \le k} :=  (f_{j,i}^*)_{1 \le i \le k,\; 1 \le i \le k}.$$

The following is a classical and crucial fact in the theory of algebraic actions:
Pontryagin
duality yields a natural one-to-one correspondence between algebraic actions of $\Gamma$ and (discrete, countable) $\mathbb{Z}\Gamma$-modules.
Thus, to each  $\mathbb{Z}\Gamma$-module $\mathcal{M}$  corresponds  an algebraic action  $\alpha^{(\mathcal{M})}\in \Alg{\Gamma}{\widehat{\mathcal{M}}}$. The dynamics of an algebraic action $\alpha^{(\mathcal{M})}$ are completely determined  by the algebraic properties of the dual as a $\mathbb{Z}\Gamma$-module.  Over the years, a significant number of important dynamical properties have found beautiful
algebraic interpretations in terms of the dual module. For $f \in \mathbb{Z}\Gamma$, we let $X_f$ denote the dual group of the $\mathbb{Z}\Gamma$-module $\mathbb{Z}\Gamma / \mathbb{Z}\Gamma f$. The corresponding algebraic action $\alpha^{(f)} \in \Alg{\Gamma}{X_f}$ is called the \emph{principal algebraic action } associated with $f$.


\begin{defn}
If $X$ is a compact group with identity element $1 \in X$, and  $\alpha \in \Alg{\Gamma}{X}$, a point $x \in X$ is called \emph{homoclinic} with respect to the action $\alpha$ if $(x,1)$ is an $\alpha$-asymptotic pair. In this case the \emph{homoclinic group}, denoted by $\Delta(X)$, is the set of all homoclinic points in $X$.
\end{defn}
It is straightforward to check that  $\Delta(X)$ is a $\Gamma$-invariant subgroup and $(x,y)$ is a  an $\alpha$-asymptotic pair if and only if $xy^{-1} \in \Delta(X)$.

We set up some more notation:

Let
\begin{equation}
P:(\ell^\infty(\Gamma))^k \to ((\mathbb{R}/\mathbb{Z})^k)^\Gamma
\end{equation}
denote the canonical projection map,
and  let $\dT$ be the metric on $(\mathbb{R} / \mathbb{Z})^k$ given by
\begin{equation}\label{eq_dT}
\dT \left(v+\mathbb{Z}^k ,w+\mathbb{Z}^k\right):=
\min_{m \in \mathbb{Z}^k}\| v-w -m\|_\infty,~  v,w \in \mathbb{R}^k
\end{equation}

%The following two lemmas % about lifting from $(\RR/\ZZ)^k$ to $\RR^k$
%will be used in  the proof of Proposition \ref{prop:X_A_SFT} below:
\begin{lem}\label{lem:lift_approx}
For  any $\delta < 1/2$ and $\tilde a,\tilde b \in (\RR/\ZZ)^k$ satisfying $\rho_\infty(\tilde a,\tilde b) < \delta$, the following holds: Let   $a \in \RR^k$ be the unique
element of $[-\frac{1}{2},\frac{1}{2})^k$ such that $\tilde a = a + \ZZ^k$. Then there exists a unique $b \in [-1,1]^k$ so that $\tilde b = b + \ZZ^k$ and
$\|a - b\|_{\infty} < \delta$.
\end{lem}
\begin{proof}
Existence and uniqueness of $a$ as above follows from the fact that
$\RR^k= \biguplus_{n \in \ZZ^k}\left([-\frac{1}{2},\frac{1}{2})^k+n\right)$.
Similarly, there exists a unique $b \in [-\frac{1}{2},\frac{1}{2})^k + a$ so that
$b = \tilde b + \ZZ^k$.
Note that $[-\frac{1}{2},\frac{1}{2})^k +  a \subseteq [-1,1]^k$ because $ a \in [-\frac{1}{2},\frac{1}{2})^k$, so $b \in [-1,1]^k$.
From the definition of $\rho_\infty$ in \eqref{eq_dT},
there exists $n \in \ZZ^k$ so that $\|a - (b + n)\|_\infty < \delta$. %, so $( b + n) \in [-\delta,+\delta]^k + a$.
It follows that
$$\|n\|_\infty =\|b -(b - n)\|_\infty < \|b- a\|_\infty + \|a- (b+n)\|_\infty \le \frac{1}{2} + \delta < 1.$$
It follows that  $n=0$, so $\|b-a\|_\infty < \delta$.
\end{proof}

\begin{lem}\label{lem:lift_compt}
Suppose $\delta \in (0,1/2)$ and that $K,W \subset \Gamma$ are finite subsets that contain the identity element of $\Gamma$. %such that $1 \in K$.
%$\delta  < \frac{1}{4}\min\{ |K| \cdot |W|,1\}$,
Assume
$(x^{(g)})_{g \in \Gamma} \in  ((\RR/\ZZ)^k)^\Gamma$ satisfy  % \eqref{eq:finite_type_dist}.
%Then by the choice of $\epsilon$ we have
\begin{equation}\label{eq:dT}
\dT (x^{(g)}_{w^{-1}f},x^{(wg)}_f)< \delta \mbox{ for every } g \in \Gamma,~ f \in K  \mbox{ and }w \in K^{-1}W.
\end{equation}

%inherited from the norm $\|\cdot \|_\infty$ on  $\mathbb{R}^k$.
%\eqref{eq:finite_type_dist}.

Then there exists $y^{(g)} \in ([-1,1]^k)^\Gamma \subset (\ell^\infty(\Gamma))^k$ with $P(y^{(g)})=x^{(g)}$ for every $g \in \Gamma$ so that:
\begin{equation}\label{eq:ys_compat}
\|y^{(g)}_{h^{-1}f}- y^{(hg)}_f\|_\infty < 2\delta \mbox{ for every } g \in \Gamma,~ f \in K  \mbox{ and }h \in W.
\end{equation}
\end{lem}

\begin{proof}
Let $(x^{(g)})_{g \in \Gamma} \in ( (\RR/\ZZ)^k)^\Gamma$  satisfy \eqref{eq:dT}.
Define $y^{(g)} \in ([-1,1]^k)^\Gamma \subset (\ell^\infty(\Gamma))^k$  as follows:

First, for every $g \in \Gamma$ let $y^{(g)}_1$ be the unique $a \in [-\frac{1}{2},\frac{1}{2})^k$ so that $P(a)= x^{(g)}_1$.
Next, for every $g \in \Gamma$, $h \in W$ and $f \in K \setminus \{h\}$, let $y^{(g)}_{h^{-1}f}$
be the unique element of $[-1,1]^k$ satisfying
$$\|y^{(g)}_{h^{-1}f}- y^{(f^{-1}hg)}_1\|_\infty < \delta.$$
Existence and uniqueness of such elements follow by applying  Lemma \ref{lem:lift_approx} above with $\tilde b =x^{(g)}_{h^{-1}f}$ and $\tilde a =x^{(f^{-1}hg)}_1$, noting that  $\dT(x^{(g)}_{h^{-1}f},x^{(f^{-1}hg)}_1)< \delta$ by \eqref{eq:dT} because $1 \in K$.
%Such $b \in [-1,1]^k$ exists and is uniquely defined by Lemma \ref{lem:lift_approx} above (observing that $.
% Note that this is well defined, in the sense that it only depends on the product %$h^{-1}f$.
Finally, for every $g \in \Gamma$, $h \in \Gamma \setminus W^{-1}K$, let
$y^{(g)}_h$ be the unique $a \in [-\frac{1}{2},\frac{1}{2})^k$ so that $P(a)= x^{(g)}_h$.
It now follows that for every $g \in \Gamma$, $f \in K$ and $h \in W$,
$$\|y^{(g)}_{h^{-1}f}- y^{(hg)}_f\| \le \|y^{(g)}_{h^{-1}f}- y^{(f^{-1}hg)}_1\| + \|y^{(f^{-1}hg)}_1 - y^{(hg)}_f\| < 2\delta.$$
\end{proof}
For $A \in M_k(\mathbb{Z}\Gamma)$, denote $\displaystyle X_A := \widehat{(\mathbb{Z}\Gamma)^k / ( \mathbb{Z}\Gamma)^kA}$.
Explicitly:
\begin{equation}\label{eq:def_X_A}
X_A := \left\{ x \in ((\mathbb{R}/\mathbb{Z})^k)^\Gamma~:~ (xA^*)_g = \mathbb{Z}^k \mbox{ for every } g \in \Gamma\right\}
\end{equation}
Let $\alpha^{(A)} \in \Alg{\Gamma}{X_A}$ denote the canonical  algebraic action on $X_A$.

Theorem \ref{thm:priniciple_alg_pot}  is a particular instance of the  following more general result, inspired by \cite{MR3314515}:
%, and the proof of Lemma $3.7$ of that paper in particular.
%provides a source of examples for expansive algebraic systems of finite type:
\begin{thm}\label{thm:X_A_SFT}
Let % $G$ be a countable group and $f \in \mathbb{Z}G$
$k \in \mathbb{N}$ and $A \in M_k(\mathbb{Z}\Gamma)$ be invertible in $M_k(\ell^1(\Gamma))$. Then the canonical action
$ \alpha_A \in \Alg{\Gamma}{X_A}$ is expansive and satisfies p.o.t.
%that is invertible in $\ell^1(G)$, then the canonical algebraic action of $G$ on the dual of $\widehat X = \mathbb{Z} G/ (\mathbb{Z} G f)$ is of finite type.
\end{thm}

\begin{proof}
To avoid some subscripts, in this  proof we let
\begin{equation}
X:= X_A \mbox{ and }
\alpha:= \alpha_A \in \Alg{\Gamma}{X}.
\end{equation}
By \cite[Lemma $3.7$]{MR3314515}, $\alpha \in  \Alg{\Gamma}{X}$ is expansive.
Fix a metric  $d$ on $X$.

Let   $\epsilon >0$ be arbitrary.
 By definition of p.o.t,
we need to show that there exists  $\delta' >0$ and a finite set $W \subset \Gamma$ so that every $(W,\delta')$ pseudo-orbit is $\epsilon$-traced by some point $x \in X$.


Choose $\delta >0$ small enough so that
\begin{equation}\label{eq:epsilon_small_A}
 \delta \le \min\left\{\frac{1}{4}\|A\|_1^{-1},\frac{1}{4}, \epsilon\right\}.
\end{equation}

From the fact that $\alpha$ is expansive, it follows that by possibly making $\delta$ even smaller we have
\begin{equation}\label{eq:delta_exp}
 \sup_{ g \in \Gamma}\dT( x_{g},\tilde x_{g}) < 2\delta  \mbox{ implies } x=\tilde x.
\end{equation}
In fact, the proof of \cite[Lemma $3.7$]{MR3314515} shows that \eqref{eq:epsilon_small_A} already implies \eqref{eq:delta_exp}, but we will not need this.

Since A is invertible in $M_k(\ell^1(\Gamma))$ so is $A^*$. Let $B \in M_k(\ell^1(\Gamma)) \cong \ell^1(\Gamma, M_k)$ denote the inverse of $A^*$.
Using the natural  identification of $M_k(\ell^1(\Gamma))$ as a subset of the $M_k(\mathbb{R})$-valued functions on $\Gamma$, write $B= \sum_{g \in \Gamma} B_g$ with $B_g \in M_k(\mathbb{R})$. %, so $\sum_{g \in G} \|B_g\| < \infty$.
There exists a  finite symmetric set  $F\subset \Gamma$  containing the identity with the property that
\begin{equation}\label{eq:F_1_approx}
\sum_{g \in \Gamma \setminus F} \|B_g\| < \frac{\delta}{2}\|A^*\|^{-1}_1.
\end{equation}


% for every $y \in \ell^\infty(G,\mathbb{R}^k)$ with $\| y\|_\infty < \|A\|_1$ such that $y_g =0$ for every $g \in F_1$ we have $ (By)_1 \in [-\epsilon,\epsilon]^k$.
%Let $S \subset G$ denote the support of $A$, and

% $K$ and $\delta$  *** Check if it should be $F_2F_1$ or maybe $F_2^{-1}F_1$ or whatever ****.
Choose a finite symmetric set $S \subset \Gamma$  containing the identity that supports  $A^*$ in the sense that there exists $(A^*_s)_{s \in S} \in (M_k(\mathbb{Z}))^S$ so that
$$(yA^*)_g = \sum_{s \in S}y_{gs^{-1}}A^*_s \mbox{ for every } y \in (\ell^\infty(\Gamma))^k \mbox{ and } g \in  \Gamma,$$
and let
\begin{equation}
K := FS^{-1} \subset \Gamma.
\end{equation}

Choose $\delta'>0$ small enough so that
\begin{equation}\label{eq:delta_prime}
d(x,y) < \delta' \mbox{ implies } \max_{f \in K}\dT( x_f,y_f) < \delta.
\end{equation}

%****and
%let  $\epsilon >0$ be an expansive constant with respect to the metric $d$.
%Furthermore, choose $\epsilon >0$ small enough to have  the property that %a finite set $\tilde K \subset \Gamma$ with the property that for every $x, \tilde x \in X_A$
%$d( x,\tilde x) < \epsilon$ implies   $\dT(x_g,\tilde x_g)< \delta$ for every $g \in K$.
%As in the proof of \cite[Lemma $3.7$]{MR3314515},
%using the fact that $A$ is invertible in $M_k(\ell^1(\Gamma))$,
% it follows that for every $\delta>0$ satisfying \eqref{eq:epsilon_small_A}  and every $\epsilon >0$


By compactness of $X$ and \eqref{eq:delta_exp} it follows that
there exists a finite set $W \subset \Gamma$ with the property that:

\begin{equation}\label{eq:W_delta_prime}
\max_{ g \in W}\dT( x_{g^{-1}},\tilde x_{g^{-1}}) < 2\delta \mbox{ implies } d(x,\tilde x) < \min\{\delta',\epsilon\} \mbox{ for all } x,\tilde x \in X.
\end{equation}

%*** repeat the argument! ***
Let $W \subset \Gamma$ be such a finite set, so that in addition $F^{-1} \subset W$.
%We will  show that $\act{\Gamma}{X_A}$ has the $(\epsilon,W)$-tracing property for $\epsilon>0$ and $W \subset \Gamma$ as above. % for every set $W \subset \Gamma$ containing the identity.

%Let $K \subset \Gamma$ and $\delta >0$ as in  Lemma \ref{lem:mod1_cont}.
 Suppose $(x^{(g)})_{g \in \Gamma} \in X^\Gamma$ is a $(K^{-1}W,\delta')$ pseudo-orbit.
Then by  \eqref{eq:delta_prime}, it follows that  \eqref{eq:dT} holds.
%we have
%\begin{equation}\label{eq:dT}
%\dT (x^{(g)}_{h^{-1}f},x^{(hg)}_f)< \delta \mbox{ for every } g \in \Gamma,~ f \in K  \mbox{ and }h \in W,
%\end{equation}

%inherited from the norm $\|\cdot \|_\infty$ on  $\mathbb{R}^k$.
%\eqref{eq:finite_type_dist}.

Let   $y^{(g)} \in ([-1,1]^k)^\Gamma \subset (\ell^\infty(\Gamma))^k$ be as in the conclusion of  Lemma \ref{lem:lift_compt}.
%with $P(y^{(g)})=x^{(g)}$ for every $g \in \Gamma$ so that the following holds:
%\begin{equation}\label{eq:ys_compat}
%\|y^{(g)}_{h^{-1}f}- y^{(hg)}_f\| < 2\delta \mbox{ for every } g \in \Gamma,~ f \in K  \mbox{ and }h \in W.
%\end{equation}
Let
\begin{equation}
z^{(g)}:= y^{(g)}A^* \in \ell^\infty(\Gamma,\mathbb{Z}^k).
\end{equation}
Then \eqref{eq:ys_compat} together with \eqref{eq:epsilon_small_A} imply
\begin{equation}\label{eq:z_approx}
\|z^{(g)}_{h^{-1}f} - z^{(hg)}_f\|_\infty <1 \mbox{ for every } g \in \Gamma~,~  f \in K \mbox{ and } h \in W.
\end{equation}
But $z^{(g)}_{h^{-1}f} ,z^{(hg)}_f \in\ZZ^k$ so

\begin{equation}\label{eq:z_g_h_f}
z^{(g)}_{h^{-1}f} =  z^{(hg)}_f \mbox{ for every } g \in \Gamma~,~  f \in K \mbox{ and } h \in W.
\end{equation}
Also note that

\begin{equation}\label{eq:z_infty_norm}
\|z^{(g)}\|_\infty \le \|y^{(g)}\|_\infty \|A^*\|_1 \le \|A^*\|_1.
\end{equation}

%Because $(z^{(g)})_h$ and $(z^{(gh^{-1})})_{gh^{-1}}$ have integer coordinates, it follows that
%\begin{equation}\label{eq:z_exact}
%(z^{(gh^{-1})})_{fh}=(z^{(g)})_f \mbox{ for every } g \in G~,~  h \in F_2 \mbox{ and } f \in F.
%\end{equation}
Define $z \in  \ell^\infty(\Gamma,\mathbb{Z}^k)$ by
\begin{equation}\label{eq:z_def}
z_{g^{-1}} := z^{(g)}_1 \mbox{ for every } g\in \Gamma.
\end{equation}

Using \eqref{eq:z_g_h_f} and the fact that $F^{-1} \subset W$ it follows that
\begin{equation}\label{eq:g_z_F}
 \alpha_g(z)\mid_F = z^{(g)}\mid_F.
\end{equation}
By \eqref{eq:F_1_approx}, \eqref{eq:z_infty_norm} and \eqref{eq:g_z_F}:
%$$z_g = A^*(y)_g = (A^*B z)_g = \sum_{g_2 \in F_2}\sum_{g_1 \in G}
\begin{equation}\label{eq:z_B_trace}
 \| (\alpha_g (z) B)_1 - (z^{(g)} B)_1 \| < \delta \mbox{ for every } g \in \Gamma.
\end{equation}
Let
\begin{equation}
y:=  z B \in (\ell^\infty(\Gamma))^k \mbox{ and } x:= P(y).
\end{equation}
It follows that

$$y A^*= z\in  \ell^\infty(G,\mathbb{Z}^k).$$
Thus $x \in X_A$, recalling the definition of $X_A$ in \eqref{eq:def_X_A}.

It follows that from \eqref{eq:z_B_trace} and \eqref{eq:dT} that
$$\rho_{\infty}((\alpha_g(x))_{f^{-1}},x^{(g)}_{f^{-1}}) < 2\delta \mbox{ for every } g \in \Gamma \mbox{ and } f \in W.$$
By  \eqref{eq:W_delta_prime}  this implies
$$d(\alpha_g(x),x ^{(g)}) < \epsilon \mbox{ for every } g \in \Gamma.$$
We found $x \in X_A$  that  $\epsilon$-traces $(x^{(g)})_{g \in \Gamma}$, so the proof is complete.
%the distance between $x_g$ and $x^{(g)}_1$ is at most $\epsilon$ for every $g \in G$.
\end{proof}

Theorem \ref{thm:priniciple_alg_pot} follows from Theorem \ref{thm:X_A_SFT} above by letting $k=1$, so $A \in M_k(\mathbb{Z}\Gamma) \cong \mathbb{Z}\Gamma$.

Combining Theorem \ref{thm:priniciple_alg_pot} with Theorem \ref{prop:finite_type_entropy_off_diagonal} we recover the following result:
\begin{cor}\label{cor:princ_alg_homoc}
Let $\Gamma$ be a countable amenable group. Every expansive principal algebraic $\Gamma$-action with positive topological entropy admits a non-diagonal asymptotic pair.
Equivalently, it has a non-trivial homoclinic group.
\end{cor}

\begin{remark}\label{rem:princ_alg} By \cite[Corollary $7.9$]{MR3314515} every non-trivial expansive principal algebraic action of an amenable group has positive entropy.
By  \cite[Lemma $5.4$]{MR3314515}  algebraic actions of the  form $\alpha_A \in \Act{\Gamma}{X_A}$ as in the statement of Theorem \ref{thm:X_A_SFT} have a dense set of homolicnic points. In particular, this proves Corollary \ref{cor:princ_alg_homoc} holds. I thank Nhan-Phu Chung for pointing out this out to me.
\end{remark}

Combining Theorem \ref{thm:priniciple_alg_pot} with Theorem \ref{thm:pot_stable} we get:
\begin{cor}\label{cor:princ_alg_stbl}
Every expansive principal algebraic $\Gamma$-action  is topologically stable.
\end{cor}

Using an algebraic characterizations of expansive algebraic actions due to Chung and Li we have:

\begin{cor}
Let $\Gamma$ be a countable group.
Suppose $\alpha \in \Alg{\Gamma}{X}$ is expansive. Then
$\alpha$ is algebraically  conjugate to a subsystem  of an algebraic action that is expansive and  satisfies p.o.t.
In other words,
there exists an expansive $\beta \in \Alg{\Gamma}{Y}$ that satisfies p.o.t and a $\Gamma$-equivariant  continuous monomorphism $\Phi:X \to Y$.
\end{cor}
\begin{proof}
Suppose $\alpha \in \Alg{\Gamma}{X}$ is expansive. As in \cite{MR3314515}, we can identify the left $\mathbb{Z}\Gamma$-module $\widehat{X}$ with $(\mathbb{Z}\Gamma)^k/J$ for some $k \in \mathbb{N}$ and some left  $\mathbb{Z}\Gamma$-submodule $J$.
By \cite[Theorem $3.1$]{MR3314515} there exists $A \in M_k(\mathbb{Z}\Gamma)$ invertible in $M_k(\ell^1(\Gamma))$ so that the rows of $A$ are contained in $J$.
Then $\alpha$ is algebraically  conjugate to a subsystem  of the natural algebraic $\Gamma$-action on the dual of  $(\mathbb{Z}\Gamma)^k/(\mathbb{Z}\Gamma)^kA$.
By Theorem \ref{thm:priniciple_alg_pot}  this $\Gamma$-action satisfies p.o.t.
\end{proof}

As we mentioned, for algebraic actions every dynamical property corresponds to some algebraic property of $\mathbb{Z}\Gamma$-modules.
Is there a natural algebraic interpretation for p.o.t in terms of the  Pontryagin dual?
Theorem \ref{thm:X_A_SFT} is a sufficient condition for algebraic actions to satisfy p.o.t. The following gives a natural necessary condition for expansive actions to satisfy p.o.t:

%\begin{prop}
%**
%\end{prop}

Given a ring $\mathcal{R}$, recall that a left $\mathcal{R}$-module $\mathcal{M}$ is \emph{finitely presented} if it is isomorphic to $\mathcal{R}^k/J$, where $J \subset \mathcal{R}^k$ is a finitely generated left ideal.
%** Recall that a ring $\mathcal{R}$ is  \emph{left-Noetherian} if every increasing chain of left-ideals  in $\mathcal{R}$ stabilizes.

\begin{prop}\label{thm:finitely_presented_finite_type}
Let $\Gamma$ be a countable group and $\act{\Gamma}{X}$ an algebraic action.
If $\alpha \in \Alg{\Gamma}{X}$  is expansive and satisfies p.o.t, then the Pontryagin dual $\widehat X$ of $X$ is a finitely presented left $\ZZ\Gamma$-module.

\end{prop}

\begin{proof}
Let $\alpha \in \Alg{\Gamma}{X}$  be expansive. % and satisfy p.o.t.
In particular it  follows that the dual module $\widehat{X}$ is finitely generated as $\mathbb{Z}\Gamma$-module \cite{MR1036904}.
% \cite[Theorem $3.1$]{MR3314515} %$\widehat{X}$ is finitely generated as a %$\mathbb{Z}\Gamma$-module.
Furthermore, by a characterization of expansive algebraic actions given in \cite{MR3314515}, there exists $k \in \mathbb{N}$, a left $\mathbb{Z}\Gamma$-submodule $J$ of $(\mathbb{Z}\Gamma)^k$, and a $A \in M_k(\ZZ\Gamma)$  invertible in $M_k(\ell^1(\Gamma))$ such that the rows of $A$ are contained in $J$
so that $\widehat{X}$ is isomorphic as a $\ZZ\Gamma$-module to $(\mathbb{Z}\Gamma)^k / J$.
%If $J$ is a finitely generated $\ZZ\Gamma$-module, then $\widehat{X}$ is finitely presented and we are done.
Suppose $J$ is not a finitely generated $\ZZ\Gamma$-module. Let $J_1$ be the $\ZZ\Gamma$-module generated by the rows of $A$. Because $J$ is not finitely generated, there exists a strictly increasing sequence of left $\mathbb{Z}\Gamma$-modules
$$J_1 \subset J_2 \subset J_3 \subset \ldots,$$
so that the row of $A$ are contained in $J_1$ and $J = \bigcup_{n=1}^\infty J_n$. It follows that $X_A$ is expansive and that $X \in  \subG{\alpha}{X_A}$ is a strictly decreasing intersection of the systems $\widehat{(\mathbb{Z}\Gamma)^k)/J_n}$. By Proposition \ref{prop:finite_type} and Remark \ref{rem:stable_intersection}, it follows that $\alpha$ does not  satisfy p.o.t.
\end{proof}

In the case $X$ is a  disconnected compact metrizable abelian group then every expansive $\alpha \in \Alg{\Gamma}{X}$ is isomorphic to an abelian \emph{group shift}.
In this case it was shown in \cite{MR1345152} that if $\Gamma$ is polycyclic-by-finite then  $\alpha$ as above is topologically conjugate to a shift of finite type.
In fact, the arguments above and those of  \cite{MR1345152} easily imply that in the totally disconnected case an expansive algebraic action satisfies p.o.t if and only if the dual module is finitely presented.
%More generally, when $X$ is a  disconnected compact metrizable abelian group,  $\alpha \in \Alg{\Gamma}{X}$ satisfies p.o.t ??

%\begin{prop}
%Let $X$ be a totally disconnected compact metrizable abelian group and $\alpha \in \Alg{\Gamma}{X}$.
%Then $(X,\alpha)$ is topologically conjugate to a subshift of finite type if and only if
 %the Pontryagin dual $\widehat X$ of $X$ is a finitely presented left $\ZZ\Gamma$-module.
%\end{prop}
%\begin{proof}
%*************** Check if this is Schmidt... ****
%\end{proof}

In view of the above, one might try to guess that an expansive algebraic $\Gamma$-action satisfies p.o.t if and only if he Pontryagin  is a finitely presented left $\ZZ\Gamma$-module.
%This is indeed the case when $X$ is a  \emph{totally disconnected} compact metrizable abelian group. An expansive $\Gamma$-actions by automorphisms of a totally disconnected %compact metrizable abelian group is isomorphic to a \emph{group-shift}:  A subshift that also has a compact group structure. This follows from

The following example shows that for some expansive algebraic actions, having a  finitely presented Pontryagin dual does not imply p.o.t:
\begin{example}
Let $\Gamma= \langle a,b \rangle$ be the free group generated by two elements $a,b$. Consider the natural algebraic $\Gamma$-action  on the  Pontryagin dual of the following finitely presented  left $\ZZ\Gamma$-module:
$$\widehat{X} = \mathbb{Z}\Gamma / \langle a-2,b-2 \rangle.$$
So
$$X = \left\{ x\in (\mathbb{R}/\mathbb{Z})^\Gamma~:~  x_{ga} = 2 x_{g} \mbox { and } x_{gb} = 2 x_{g} \mbox{ for every } g \in \Gamma \right\}.$$
%It is clear from the definition that $\widehat{X}$ is a finitely presented  left $\ZZ\Gamma$-%module.
%On the other hand,
 Let
$$Y =  \left\{ x\in (\mathbb{R}/\mathbb{Z})^\Gamma~:~  x_{ga} = 2 x_{g}\right\}.$$
Then $\act{\Gamma}{Y}$ is expansive. Now let
$$ X_n = \left\{ x\in (\mathbb{R}/\mathbb{Z})^\Gamma~:~  x_{ga} = 2 x_{g} \mbox { and } \rho_\infty(x_{gb},2 x_{g})  \le  \frac{1}{n} \mbox{ for every } g \in \Gamma \right\}.$$
Then $X= \bigcap_{n=1}^\infty X_n$ is a strictly decreasing intersection of expansive $\Gamma$-systems, so it does not satisfy p.o.t., by Proposition \ref{prop:finite_type} and Remark \ref{rem:stable_intersection}.
\end{example}
%Now suppose  $\act{\Gamma}{X}$ is an expansive algebraic action and that the Pontryagin dual $\widehat X$ of $X$ is a finitely presented left $\ZZ\Gamma$-module.
%Identify $\hat{X}$ with $\widehat{(\mathbb{Z}\Gamma)^k / J}$, where $J$ is a finitely generated $\mathbb{Z}\Gamma$-submodule of $(\mathbb{Z}\Gamma)^k$.
%As before, by the algebraic characterization of expansive algebraic actions of \cite{MR3314515} above, there exists $A \in M_k(\ZZ\Gamma)$   being invertible in $M_k(\ell^1(\Gamma))$ %such that the rows of $A$ are contained in $J$. In particular  $\act{\Gamma}{X}$ naturally embeds in $\act{\Gamma}{X_A}$.  By Lemma \ref{} and Proposition \ref{} the proof will be %complete one we show that $\act{\Gamma}{X}$ is of finite type relatively to $\act{\Gamma}{X_A}$.****


%**** I'm a bit stuck on this part ! It amounts to the following:
%Suppose $A \in M_k(\ZZ \Gamma)$ is invertible in $M_k(\ell^1(\Gamma))$ and $f =(f_1,\ldots,f_k) \in (\ZZ\Gamma)^k$, then there exists $\epsilon >0$ so that if $y \in %([-1,1]^k)^\Gamma$ s.t $yA^* \in \ell^\infty(\ZZ^k,\Gamma)$ and
%$(y f)_g$ is $\epsilon$-close to an integer for every $g \in \Gamma$ then $yf$ has all integer coordinates.
%****
%****
%\end{proof}

\begin{question}\label{ques:noetherian_pot}
Let $\Gamma$ be a  countable group and suppose $\ZZ \Gamma$ is left Noetherian (for example, suppose $\Gamma$ is polycyclic-by-finite). Does every expansive $\alpha \in \Alg{\Gamma}{X}$ satisfy p.o.t?
\end{question}

An affirmative answer to Question \ref{ques:noetherian_pot} would in particular recover \cite[Corollary 9.12]{MR3314515}, an affirmative answer to Question \ref{ques:Chung_Li} for expansive algebraic $\Gamma$-actions when $\ZZ \Gamma$ is left Noetherian.
%Combining the above with Proposition \ref{prop:finite_type_entropy_off_diagonal} we recover \cite[Corollary 9.12]{MR3314515}:
%\begin{cor}
%Let $\Gamma$ be a countably infinite amenable group  and suppose that $\ZZ \Gamma$ is left Noetherian.
% Let $\Gamma$ act on a compact abelian group $X$ expansively by automorphisms. Then $\act{X}{\Gamma}$ has positive entropy if and only if $\Delta(X)$ is nontrivial.
%\end{cor}

\section{An expansive algebraic action with positive entropy and trivial homoclinic group}\label{sec:algebraic_exp_no_asymp}
In this section we prove Theorem \ref{thm:alg_exp_pos_ent_no_asymp}, providing a negative answer to Question \ref{ques:Chung_Li} within the class of algebraic actions.
By \cite{MR3314515},  if  $\alpha \in \Alg{\Gamma}{X}$ is an algebraic counterexample,  the acting group $\Gamma$ cannot be polycyclic-by-finite.
The construction makes crucial use of the fact that $\mathbb{Z}\Gamma$ is not left Noetherian for the group $\Gamma$ we use.
%Theorem \ref{thm:alg_exp_pos_ent_no_asymp} shows this is not always the case when $\Gamma$ is abelian (but not finitely generated) or when  $\Gamma$ is a finitely-generated solvable group. The reader can compare the construction below with \cite[Example $3.9$ (1), Page $23$]{MR1345152}.

%In this section $\Gamma$ will denote a countably \emph{infinite locally} finite group.
%Recall that a group $\Gamma$ is locally finite if and only if every finitely generated subgroup of $\Gamma$ is finite.
%A locally finite group is automatically amenable.

%Some examples of locally finite groups are the group $S_\infty = \overrightarrow{\lim}_{n \to \infty}S_n$ of finitely supported permutations of $\mathbb{N}$ and the direct sum %$\bigoplus_{n \in \mathbb{N}}\Gamma_n$ of finite groups $(\Gamma_n)_{n=1}^\infty$.
Let $(\Gamma_n)_{n=1}^\infty$ be a sequence of finite groups, and let $\Gamma := \bigoplus_{n=1}^\infty \Gamma_n$ be the direct sum of these groups.
Namely, $\Gamma$ is the countable subgroup of $\prod_{n=1}^\infty \Gamma_n$
\begin{equation}
\Gamma =\bigcup_{N=1}^\infty\left\{ (g_n)_{n=1}^\infty\in  \prod_{n=1}^\infty \Gamma_n ~:~ g_n=1_{\Gamma_n} \mbox{ for all } n \ge N\right\}.
\end{equation}

%\begin{lem}\label{lem:locally_fin_seq}
%Let $\Gamma$ be a locally finite group.
%then there exists a sequence $(\Gamma_n)_{n=1}^\infty$ of finite subgroups $\Gamma_n < \Gamma$ so that:
%\begin{enumerate}
%item
%The set $\bigcup_{n=1}^\infty \Gamma_n$ generates the group $\Gamma$.
%\item  For every finite set $F \subset \Gamma$ there exists $N \in \mathbb{N}$ so that for every $n \ge N$ $|F \cap \Gamma_n| \le 1$.
%\item For every $n \ge 1$ $|\Gamma_{n+1}| \ge n^2 |\Gamma_n|$.
%\end{enumerate}
%\end{lem}
%\begin{proof}
%*****
%\end{proof}

%Let $(\Gamma_n)_{n=1}^\infty$ be a sequence of finite subgroups as given by  Lemma \ref{lem:locally_fin_seq}.
For every $n \in \mathbb{N}$, we naturally identify $\Gamma_n$ with the following subgroup of $\Gamma$:
$$\Gamma_n \cong \left\{ (g_k)_{k=1}^\infty  \in \Gamma ~:~ g_k = 1_{\Gamma_k} \mbox{ for all } k \ne n \right\}$$

The reader can keep in mind the case $\Gamma_n = (\mathbb{Z}/2\mathbb{Z})^{a_n}$, where $a_n$ is some rapidly increasing sequence of integers.
In this case $\Gamma$ will be isomorphic to the countable abelian $2$-torsion group $\bigoplus_{\mathbb{N}}(\mathbb{Z}/2\mathbb{Z})$, also isomorphic to the additive group of polynomials over the finite field of size $2$.


For every $n \in \mathbb{N}$ let
\begin{equation}
\tilde \Gamma_n := \left\{(g_n)_{k=1}^\infty \in \Gamma~:~ g_k =1_{\Gamma_k} \mbox{ for all } k > n \right\}.
\end{equation}
%$\tilde \Gamma_n$ is the smallest group that contains $\Gamma_k$ for all $k \le n$.
 Clearly   $\tilde \Gamma_n \cong \bigoplus_{k=1}^{n}\Gamma_k$ is a finite subgroup of $\Gamma$. %, because it is finitely generated.
 Also, the sequence $(\tilde \Gamma_n)_{n=1}^\infty$ is a left-F{\o}lner sequence for $\Gamma$.

For $n \ge 1$ let $f_n \in \mathbb{Z}\Gamma$ be given by
\begin{equation}
 f_n := \sum_{g \in \Gamma_n}g,
\end{equation}
Let $J$ be the left $\mathbb{Z}\Gamma$-ideal generated by the element $2 \in \mathbb{Z}\Gamma$ and by $\{f_n\}_{n=1}^\infty$,
and let
$$ X :=  \widehat{\mathbb{Z} \Gamma / J}.$$
Then $X$ is a totally disconnected compact abelian group.  We will identify $X$ with the following $\Gamma$-subshift:
$$ X = \left\{ x \in (\mathbb{Z}/2\mathbb{Z})^\Gamma ~:~ x\cdot f_n = 0 \mbox{ for every } n \in \mathbb{N} \right\}=$$
$$=\left\{ x \in (\mathbb{Z}/2\mathbb{Z})^\Gamma ~:~ \sum_{h \in \Gamma_n}x_{gh} =0 \mbox{ for every } n \in \mathbb{N} \mbox{ and } g \in \Gamma\right\}.$$
For every $n \in \mathbb{N}$, choose $\gamma_n \in \Gamma_n$ so that $\gamma_n \ne 1$ for all but finitely many $n$'s.
Let
\begin{equation}
E = \{ (g_n)_{n=1}^\infty \in \Gamma~:~ g_n \ne \gamma_n \mbox{ for every } n\ge 1\}.
\end{equation}

\begin{lem}\label{lem:X_E_ext}
For every $w \in \{0,1\}^E$ there exists $x \in X$ such that $x\mid_E = w$.
\end{lem}
\begin{proof}
For $g \in \Gamma$ let
\begin{equation}
I(g) :=  \left\{ n\in \mathbb{N}~:~ g_n =\gamma_n\right\}
\end{equation}
and
\begin{equation}
T(g):= \left\{ (h_n)_{n \in I(g)}:~ h_n \in \Gamma_{n} \setminus \{\gamma_n\}\right\}.
\end{equation}
Fix $w \in \{0,1\}^E$. We define $x \in \{0,1\}^\Gamma$  as follows:
%For $g \in E$  let $x_g = w_g$. For $g \in \Gamma \setminus E$, let
\begin{equation}
x_g := \begin{cases}
\omega_g & g \in E\\
\sum_{(h_n)_{n \in I(g)} \in T(g)}\omega_{g\prod_{n \in I(g)}h_n} & \mbox{otherwise}.
\end{cases}
\end{equation}
Note that the definition is independent of the order of the product $\prod_{n \in I(g)}h_n$, because $\Gamma_n$ and  $\Gamma_m$ are commuting subgroups of $\Gamma$ for $n \ne m$.
It is clear that $x\mid_E = \omega$.
We will now verify that $x \in X$. Equivalently, we need to show that for every $g \in \Gamma$, and every $n \in \mathbb{N}$,
\begin{equation}\label{eq:sum_x_g}
\sum_{h \in \Gamma_n} x_{gh} = 0 \mod 2.
\end{equation}

Fix $g \in \Gamma$ and $n \in \mathbb{N}$.
%First consider the case that $g_n \ne \gamma_n$.
We have
$$\sum_{h \in \Gamma_n} x_{gh} = \sum_{h \in \Gamma_n} x_{gg_n^{-1}h} =
 \sum_{h \in  \Gamma_n \setminus  \{\gamma_n\}} x_{gg_n^{-1}h} +  x_{g g_n^{-1} \gamma_n}.$$
The first equality follows because $g_n \in \Gamma_n$ so $g_n\Gamma_n= \Gamma_n$,  so the summands in the sum on the left hand side of the equality are a permutation of the summands on the right.

Let
$$I(g)\setminus \{n\} = \left\{n_1,\ldots,n_k \right\}.$$
Then we have
$$I(g g_n^{-1}\gamma_n) = \{n\} \uplus \left\{n_1,\ldots,n_k \right\}.$$
By definition of $x$ it this follows that
$$ x_{g g_n^{-1} \gamma_n} = \sum_{h \in \Gamma_n \setminus \{\gamma_n\}} \sum_{h_1 \in \Gamma_{n_1}\setminus \{\gamma_{n_1}\}}\ldots\sum_{h_k \in \Gamma_{n_k}\setminus \{\gamma_{n_k}\}} \omega_{gg_n^{-1}\gamma_n h h_1\ldots h_k}.$$
Similarly for $h \in \Gamma_n \setminus  \{\gamma_n\}$ we have
$$I(gg_n^{-1}h) = \left\{n_1,\ldots,n_k \right\}.$$ Thus, for every $h \in \Gamma_n \setminus  \{\gamma_n\}$ we have:
$$  x_{gg_n^{-1}h} = \sum_{h_1 \in \Gamma_{n_1}\setminus \{\gamma_{n_1}\}}\ldots\sum_{h_k \in \Gamma_{n_k}\setminus \{\gamma_{n_k}\}} \omega_{gg_n^{-1}\gamma_n h h_1\ldots h_k}.$$
Thus,
$$ x_{g g_n^{-1} \gamma_n}=  \sum_{h \in  \Gamma_n \setminus  \{\gamma_n\}} x_{gg_n^{-1}h}.$$

% = \sum_{h \in  \Gamma_n \setminus  \{\gamma_n\}}  x_{gg_n^{-1}h}.$$
%We have
%$$ \sum_{h \in \Gamma_n} x_{gh} = \sum_{h \in \Gamma_n \setminus \{g_n^{-1} \gamma_n}} x_{gh} + x_{g g_n^{-1} \gamma_n}=$$
%$$ = \sum_{h \in \Gamma_n \setminus \{g_n^{-1} \gamma_n}}
%By definition, the term
%Thus in this case, $ \sum_{h \in \Gamma_n} x_{gh} =0 \mod 2$.
%It follows that for every $n \in \mathbb{N}$ and every $g \in \Gamma$,
%$$ \sum_{h \in \Gamma_n} x_{gh} =0 \mod 2.$$
We have thus shown that  \eqref{eq:sum_x_g} holds. So $x \in X$ and $x\mid_E = \omega$.
\end{proof}
\begin{lem}\label{lem:X_alg_lower_bound}
\begin{equation}\label{eq:h_lower_bound}
h(\act{\Gamma}{X}) = \prod_{n=1}^\infty(1 - |\Gamma_n|^{-1})\log 2.
\end{equation}
\end{lem}
\begin{proof}
Let
$$ N_n := \# \left\{ x\mid_{\tilde \Gamma_n} ~:~ x \in X\right\}.$$
From  Lemma \ref{lem:X_E_ext}, it follows that
$$ N_n = 2^{\prod_{k=1}^n(\Gamma_k \setminus \{\gamma_k\})}.$$
Because $\{\tilde \Gamma_n\}$ is a left-F{\o}lner sequence, it follows that
$$h(\act{\Gamma}{x})  =  \lim_{n \to \infty} \frac{\log N_n}{|\tilde \Gamma_n|}.$$
Thus,
$$h(\act{\Gamma}{x}) =\lim_{n \to \infty}\frac{\prod_{k=1}^n\left|\Gamma_k \setminus \{\gamma_k\} \right|}{\prod_{k=1}^n |\Gamma_k|} \log 2 .$$
From this we immediately get  \eqref{eq:h_lower_bound}.
\end{proof}


\begin{lem}\label{lem:X_alg_no_asymp}
The action $\act{\Gamma}{X}$ has no off-diagonal asymptotic pairs.
\end{lem}
\begin{proof}
We will show that the only homoclinic point $x \in X$ is the identity element.
In other words, we will show that if $x \in X$ and $F \subset \Gamma$ is a finite set such that
$x_g = 0$ for all $g \in \Gamma \setminus F$, then $x_g =0$ for all $g \in \Gamma$.
Indeed, because $F$ is finite, there exists $n \in \mathbb{N}$ such that $g_n = 1$ for all $g \in F$. Suppose $g \in F$.
Then for every $h \in \Gamma_n \setminus \{1\}$, $gh \not\in F$.
It follows that
$$x_g = \sum_{h \in \Gamma_n} x_{gh} =0.$$\
This proves that $x_g =0$ for all $g \in F$, thus $x$ is the identity element of $X$.
\end{proof}



The above construction with  $\Gamma_n = (\mathbb{Z}/2\mathbb{Z})^{a_n}$ completes the proof %of the first part
 of Theorem \ref{thm:alg_exp_pos_ent_no_asymp}.

%A minor modification of the  construction  can produce an algebraic action of a finitely generated solvable group with positive entropy and no asymptotic pairs.
%%The details are very similar to the above case.
%We sketch the required modifications below:
%
%The \emph{lamplighter group} is defined by
%$$\Gamma := \mathbb{Z} \wr (\mathbb{Z}/2\mathbb{Z}) \cong  \{ (x^n,h) ~:~ h \in \mathbb{F}_2[x,x^{-1}] ~,~ n \in \mathbb{Z}\},$$
%Where $\mathbb{F}_2[x,x^{-1}]$ denotes the ring of Laurent polynomials over the finite field with two elements. The group  multiplication rule in $\Gamma$ is given by:
%$$ (x^n,h) \cdot (x^m,g) = (x^{n+m},h+x^n\cdot g).$$
%
%The lamplighter group is finitely generated, two-step solvable and in particular amenable. Let
%$$F_n := \left\{ (x^m,\sum_{k=-b_n}^{b_n} c_k x^k ) \in \Gamma:~ |m| \le n,~c_k \in \mathbb{F}_2 \mbox{ for } k=-b_n\ldots,b_n \right\}$$
%Then $(F_n)_{n=1}^\infty$ is a  left-F{\o}lner sequence for $\Gamma$ assuming $\lim_{n \to \infty}\frac{b_n}{n} 0$.
%Choose sequences $(a_n)_{n=1}^\infty$ and $(b_n)_{n=1}^\infty$ of integers so that
%$$ a_1 \le b_1 < a_2 \le b_2 \le \ldots < a_n \le b_n \le \ldots$$
%and let
%$$\Gamma_n := \left\{ (0, \sum_{k=0}^{b_n} c_{k} x^{ka_n} ) \in \Gamma~:~ c_k \in \mathbb{F}_2 \mbox{ for } k=0\ldots,b_n \right\}.$$
%Then $\Gamma_n$  is a finite subgroup of $\Gamma$.
%As before, define
%$f_n \in \mathbb{Z}\Gamma$ by
%$$f_n = \sum_{g \in \Gamma_n} g,$$
%Let $J$ be the left $\mathbb{Z}\Gamma$-ideal generated by the element $2 \in \mathbb{Z}\Gamma$ and by $\{f_n\}_{n=1}^\infty$, and $X= \widehat{ \mathbb{Z}\Gamma / J}$.
%Let $\pi_n:\Gamma \to \Gamma_n$ be given by
%$$\pi_n(x^n,\sum_{-\infty}^\infty c_nx^n) = (0, \sum_{k=a_n}^{b_n} c_k x^k )$$
%and
%let
%$$\gamma_n = (0,\sum_{k=a_n}^{b_n} x^k )\in \Gamma$$
%\begin{equation}
%E = \left\{g \in \Gamma:~ \pi_n(g) \ne \gamma_n \mbox{ for all } n \ge 1 \right\}
%\end{equation}
%Lemma \ref{lem:X_alg_lower_bound} holds true with this choice of $E$, essentially with the same proof. ****
%As in the proof of Lemma \ref{lem:X_alg_no_asymp},
%it follows that
% with routine modifications show that $\act{\Gamma}{X}$ always has a trivial homoclinic group and positive entropy if the sequence $(b_n-a_n-1)_{n=1}^\infty$ grows sufficiently fast.

Using the characterization of completely positive entropy for from \cite{MR3314515}, we can further show that the construction above has completely positive entropy:

Let $\mathcal{B}_X$ denote the Borel $\sigma$-algebra on $X$, and let $\mu_X$ denote Haar measure on $X$, normalized to be a probability measure.
Then $\act{\Gamma}{(X,\mathcal{B}_X,\mu_X)}$ is a probability preserving $\Gamma$-action.

\begin{prop}
If $ \prod_{n=1}^\infty(1 - |\Gamma_n|^{-1})>0$ then the measure preserving action $\act{\Gamma}{(X,\mathcal{B}_X,\mu_X)}$ has completely positive entropy.
\end{prop}
\begin{proof}
By \cite[Corollary $8.4$]{MR3314515} we need to show that $\mathit{IE}(X)=X$. Equivalently (see \cite[Definition $2.3$]{MR3314515})  we need to show that for any pair of non-empty open sets $U_0,U_1$ with $0 \in U_0$  there exists a finite $K \subset \Gamma$  and $c,\epsilon >0$ such that for any finite set $F \subset \Gamma$ with $|K F \setminus F| < \epsilon |F|$, the pair $(U_0,U_1)$ has an independence set $F' \subset F$ with $|F'| \ge c |F|$. In our case it suffices to prove the above holds for open sets  $U_0,U_1$ of the form:
$$U_0 = [0]_{\tilde \Gamma_n}\mbox { and } U_1 = [x]_{\tilde \Gamma_n},~ x \in X\; n \in \mathbb{N}.$$
%Let $K :=\tilde \Gamma_n$, $\epsilon=1$ and
The choice of $K$ and $\epsilon$ will not be relevant for us.
Let $$c = \frac{1}{2}|\tilde \Gamma_n|^{-1} \prod_{k=n+1}^\infty (1-|\Gamma_k|^{-1}).$$

%Suppose $F \subset \Gamma$ is a finite set so that $|KF\setminus F| < \epsilon |F|$.  *** Do I need $K$ and $\epsilon$? and this condition? ***
Choose any finite set $F \subset \Gamma$.
For any choice of   $\tilde \gamma_k \in \Gamma_k$ for $1 \le k \le n$, let
$$F_{\tilde \gamma_1,\ldots,\tilde \gamma_n} := \left\{(g_k)_{k=1}^\infty \in  F~:~ g_k = \gamma_n \mbox{ for all } k\le n  \right\}.$$
It follows that  the sets $F_{\tilde \gamma_1,\ldots,\tilde \gamma_n}$ are a partition of $F$ into $|\tilde \Gamma_n|$ sets, so there exists a choice of $\tilde \gamma_1,\ldots,\tilde \gamma_n$ as above so that
$|F_{\tilde \gamma_1,\ldots,\tilde \gamma_n} | \ge |\tilde \Gamma_n|^{-1} |F|$.
Next, for every $k > n$ define $F_k \subset \Gamma$ and $\gamma_k \in \Gamma_k$ by induction as follows:
To start the induction, set $F_n := F_{\tilde \gamma_1,\ldots,\tilde \gamma_n}$.  Assume $F_k$ has been defined. Choose $\gamma_{k+1} \in \Gamma_{k+1}$ so that
$$\left|  \left\{(g_j)_{j=1}^\infty \in  F_k~:~ g_{k+1}=\gamma_{k+1}\right\}\right| \le |\Gamma_{k+1}|^{-1} |F_k|,$$
and let
$$F_{k+1} :=   \left\{(g_j)_{j=1}^\infty \in  F_k~:~ g_{k+1}\ne\gamma_{k+1}\right\}.$$
Let $F' = \bigcap_{k=n+1}^\infty F_k$.
Note that  the decreasing sequence $\ldots \subset F_{k+1} \subset F_k$ stabilizes because $F$ is finite. Also note that $\gamma_n \ne 1$ for all but finitely many $n$'s.
From the construction and choice of $c$ it is clear that
$|F'| \ge c |F|$. An application of  Lemma \ref{lem:X_E_ext} with the corresponding $\gamma_k$'s so that $\gamma_k \ne \tilde \gamma_k$ for $k \le n$, gives that $F'$ is an independence set for $(U_0,U_1)$.
\end{proof}

%*** Todo: Modify the above example to an algebraic action of the lamplighter group with positive entropy and no off-diagonal asymptotic pairs. ****
\section{Positive entropy subshifts without off-diagonal  asymptotic pairs}\label{sec:positive_exp_no_asymp}

%**** Check: Changed $G$ to $\Gamma$ and move to left action throughout this section! ***

In this section we prove Theorem \ref{thm:exapnsive_pos_entropy_no_asymp}.
We actually prove a slightly more general result.
%, where the group  $\mathbb{Z}$ can by a large class of groups.
This extra generalization does not seem to complicate the construction much and we hope it  helps isolate the main idea.
Throughout this section $\Gamma$ will be  a countably infinite  amenable group that is \emph{residually finite}.

Recall that a group $\Gamma$ is residually finite if and only if there exist a decreasing sequence of normal subgroups so that
\begin{equation}\label{eq:Res_fin}
\ldots \Gamma_{n+1} \lhd \Gamma_n \lhd \ldots \Gamma_1 \lhd \Gamma_0 = \Gamma, \mbox{ and } \bigcap_{n=1}^\infty \Gamma_n= \{1\}.
\end{equation}
and
\begin{equation}\label{eq:Res_fin_2}
[\Gamma:\Gamma_n] < \infty \mbox{ for every } n \ge 1.
\end{equation}

%The easiest examples for residually finite amenable groups are $\mathbb{Z}^d$ for %every $d \ge 1$.
%Non-abelian free groups are the basic example of non-amenable groups that are %residually finite.

%As announced in the introduction, in this section we describe a $G$-subshift that has positive entropy and no off-diagonal asymptotic pairs.
Let
\begin{equation}
\AAA= \{0,1,2\} \cong \ZZ/3\ZZ.
\end{equation}
%Note that $\AAA$ is a finite field. We will not make essential use of that,  but t
%We will use that $\AAA$ is a   group.
%structure on $\AAA$ will be useful for our construction.
Fix a strictly decreasing sequence of finite-index normal  subgroups $(\Gamma_n)_{n=1}^\infty$ with trivial intersection as in \eqref{eq:Res_fin}. For every $n \ge 1$, let
\begin{equation}
b_n := [\Gamma:\Gamma_n] \mbox{ and } a_n := [\Gamma_{n-1}:\Gamma_{n}].
\end{equation}
For every $n \ge 1$, choose  a set of representatives  $T_n \subset \Gamma_{n-1}$ for $\Gamma_{n}$-cosets in $\Gamma_{n-1}$. Namely, the following holds:
\begin{equation}\label{eq:T_n_g_N}
\Gamma_{n-1} = \biguplus_{t \in T_n}\Gamma_{n}t=\biguplus_{t \in T_n}t\Gamma_{n}.
\end{equation}
Furthermore, assume that  $1 \in  T_n$ for every $n \ge 1$.
Note that $|a_n| > 1$ for every $n$, and so by passing to a subsequence we can make sure the sequence $(a_n)_{n=1}^\infty$ grows sufficiently fast in order to satisfy some conditions that will be stated later.
%asymptotically faster than any prescribed sequence.


It follows that $|T_n|= [\Gamma_{n-1}:\Gamma_{n}]= a_n$.
Also, let
\begin{equation}\label{eq:E_n_def}
E_{n} := T_n\cdot \ldots \cdot T_1 = \left\{ t_n\cdot\ldots\cdot t_1~:~ t_i \in T_i ~ i =1,\ldots,n\right\}.
\end{equation}

Note that
\begin{equation}\label{eq:G_n_E_n}
 \Gamma= \biguplus_{g \in E_n} \Gamma_{n}g= \biguplus_{g \in E_n} g\Gamma_{n},
\end{equation}
and $|E_n| = [\Gamma:\Gamma_{n}] = b_{n}$.
It will be useful to define
\begin{equation}
\Gamma_0 := \Gamma,~ T_0:= \{1\}\mbox{  and } a_0:= b_0 := 1.
\end{equation}
Note that $b_n = a_n \cdot b_{n-1}$.

For $E \subset \Gamma$, $w \in \AAA^E$ and $g \in \Gamma$, let $g \cdot w \in \AAA^{gE}$ be given by
\begin{equation}
 (g \cdot w)_{ g f}:= w_f \mbox{ for } f \in E.
\end{equation}

%We identify  $\{0,1,2\}$ with $\FF_3 = \ZZ/3\ZZ$.
%The construction takes as input  a sequence of integers $(a_n)_{n=1}^\infty$ with $a_n  > 2$ for all $n \in \NN$. We will need the sequence $(a_n)_{n=1}^\infty$  to be rapidly growing enough (the precise %conditions will appear later). Let $b_n= \prod_{k=1}^n a_k$. It will be useful to define $a_0= b_0 = 1$, so $b_n = a_n \cdot b_{n-1}$.
%By induction, construct sequences of words $w_n,s_n \in \FF_3^{b_n}$ and $A_n \subset \FF_3^{b_n}$ as follows:

The first step in our construction is to  inductively define  sequences $(w^{(n)})_{n=0}^\infty$ and $(s^{(n)})_{n=1}^\infty$  with $w^{(n)},s^{(n)} \in \AAA^{E_n}$ together with  sequences $A_n \subset \AAA^{E_n}$ so that $w^{(n)} \in A_n$ as follows:

To start the induction define
$w^{(0)} \in \AAA^{T_0} \cong \AAA$  by $w^{(0)}_1 := 0$, and  define $A_0 := \AAA^{T_0}$.

%Let $A_{n} \subset \FF_3^{b_n}$ be the set of  words which start with $w_{n-1}$ followed by a concatenation of $a_n-1$ words  %$m_1,\ldots,m_{a_n-1} \in A_{n-1}$
%so that:
%\begin{enumerate}
%\item $m_j \ne w_{n-1}$ for $j=1,\ldots, a_n-2$.
%\item $ w_{n-1}\oplus m_{1}\oplus \ldots \oplus  m_{a_n-1} = 0^{b_{n-1}}$, where $\oplus$ represents coordinate-wise addition %in $\FF_3^{b_{n-1}}$.
%\end{enumerate}
%Let
%$$ B_n := \left\{m_1m_2m_2\ldots m_{a_n} \in \FF_3^{b_n} ~:~  m_1 = w_1 \mbox{ and } m_2,\ldots,m_{a_n} \in A_{n-1 } \setminus \{w_{n-1}\}\right\}.$$

Suppose $n \ge 1$, that $w^{(n-1)}$ and $A_{n-1}$ have been defined.


Let
\begin{equation}
 B_n' := \left\{ w \in \AAA^{E_n} ~:~   t^{-1} \cdot  w \mid_{E_{n-1}} \in  A_{n-1} \setminus \{w^{(n-1)}\} ~ \forall t \in T_n \setminus \{1\}\right\},
\end{equation}
and
\begin{equation}
 B_n = \left\{ w \in B_n' ~:~ w \mid_{E_{n-1}} = w^{(n-1)}\right\}.
\end{equation}

%$B_n$ is the set of words of length $b_n$ over the alphabet $A$ that are concatenation of words in $A_{n-1}$.

For $s \in \AAA^{E_{n-1}}$ let
\begin{equation}\label{eq:C_s_n_def}
 C_{s,n} := \left\{ w \in B_n ~:~ \sum_{t \in T_n} (t^{-1} \cdot w)\mid_{E_{n-1}} = s \right\}.
\end{equation}

%and define $f_n:\FF_3^{E_{n-1}} \to \NN$ by
%$$f_n(s) = |C_{s,n}|,$$
In \eqref{eq:C_s_n_def} above and elsewhere we sum elements of $\AAA^{E_{n-1}}$ pointwise, as $\AAA$-valued functions, recalling that $\AAA = \mathbb{Z}/3\mathbb{Z}$.
Thus $B_n = \biguplus_{s \in \AAA^{E_{n-1}}}C_{s,n}$.
Choose  $s^{(n-1)} \in \AAA^{E_{n-1}}$ so that
\begin{equation}
|C_{s^{(n-1)},n}| = \max\left\{|C_{s,n}|~:~ s \in \AAA^{E_{n-1}}\right\}.
\end{equation}

Now define
$$A_n := C_{s^{(n-1)},n},$$
and choose $w_n$ to be an arbitrary element of $A_n$.


\begin{example}
Consider the case $\Gamma = \ZZ$. Let $\Gamma_1 = 4 \ZZ$, $\Gamma_2 = 12 \ZZ$, $E_1 = T_1 = \{0,1,2,3\}$, $T_2= \{0,4,8\}$, $E_2 = \{0,1,2,\ldots,11\}$. We have
 $a_1 = 4$ and $a_2 =3$. Because $w^{(0)} = 0 $, it follows that
$A_1 = \{1,2\}^{\{0,1,2,3\}}$.
Considering elements of $A_1$ as strings of length $4$ over the alphabet $A_1$, we have
$$ C_{0,1} = \left\{ 0111, 0222\right\}, C_{1,1} = \{0121,0211,0112\}, C_{2,1}=\{0221,0122,0212\} .$$
So  at this point the procedure above allows to choose  either  $s^{(1)} =1$ or $s^{(1)}=2$. Suppose we choose $s^{(1)} =1$. Then
$$A_1= C_{1,1} = \{0121,0211,0112\},$$
and we can choose for example $w^{(1)} = 0121$.
\end{example}

We note  that if for some $n$ it happens that $|A_n| \le 2$, then $|A_{n+1}| = 1$ and $|A_{n+2}|= \emptyset$, so we will not be able to continue the construction.
The following lemma shows that this does not happen if the sequence $(a_n)_{n=1}^\infty$ grows rapidly.
\begin{lem}\label{lem:A_n_big}
If $A_n \ne \emptyset$ then
\begin{equation}\label{eq:A_n_big}
|A_{n+1}| \ge \frac{1}{3^{b_{n}}}\left( |A_n| -1\right)^{a_{n+1} -1}.
\end{equation}

\end{lem}
\begin{proof}
Because $B_{n+1} = \biguplus_{ s\in \AAA^{E_{n}}}C_{s,n+1}$, it follows that
\begin{equation}\label{eq:b_n_ineq}
|B_{n+1}| = \sum_{s \in \AAA^{E_{n}}} |C_{s,n+1}|\le 3^{b_{n}} \max\{C_{s,n+1}~:~s \in \AAA^{E_{n}}\} =  3^{b_{n}} |A_{n+1}|.
\end{equation}
It is clear that $B_{n+1}$ is in bijection with $(A_n \setminus \{w_n\})^{a_{n+1}-1}$, so
\begin{equation}\label{eq:B_n_card}
|B_{n+1}| = \left( |A_n| -1\right)^{a_{n+1} -1}.
\end{equation}
The inequality \eqref{eq:A_n_big} follows from \eqref{eq:b_n_ineq} and \eqref{eq:B_n_card}.
\end{proof}

\begin{lem}\label{lem:A_n_not_empty}
Suppose
\begin{equation}\label{eq:a_n_big_1}
a_{n+1} \ge 2b_{n} +3 \mbox{ for every } n \ge 1.
\end{equation}
 Then for every $n \ge 1$:
\begin{equation}\label{eq:A_n_LB}
|A_n| \ge 3^{b_{n-1}}+1
\end{equation}
%\begin{enumerate}
%\item[(i)] $|A_n| \ge 3^{b_{n-1}}+1$  for every $n \ge 1$.
% \item[(ii)] $|A_n| \ge 3^{n-1}$ for every $n \ge 2$.
%\end{enumerate}
In particular  $A_n \ne \emptyset$ for every $n \ge 0$.
\end{lem}
\begin{proof}
Proceed  by induction:
For $n=0$, $b_0=1$ and $A_0 = \AAA$ so $|A_0| = 3 > 3^0+1=3^{b_0}+1$, so \eqref{eq:A_n_LB} is true for $n=0$.
%If  \eqref{eq:a_n_big_1} holds for then $|A_{n+1}| \ge 3^{b_{n}}+1$.
Assume by induction that \eqref{eq:A_n_LB} holds for fixed $n$.
In particular  $|A_n| \ge 3$. Thus by Lemma \ref{lem:A_n_big} and \eqref{eq:a_n_big_1}
$$ |A_{n+1}| \ge  \frac{1}{3^{b_{n}}}3^{b_{n-1}(a_{n+1}-1)} \ge 3^{-b_{n}}\cdot 3^{b_{n-1}(2b_{n} +2)} \ge 3^{(2b_{n-1}-1)b_n}\cdot 3^{2n}\ge 3^{b_{n}} +1.$$
%This proves $(i)$.
%The assertion (ii) follows from (i) by observing that $\{b_n\}_{n=0}^\infty$ is a strictly i%increasing sequence of natural numbers so $b_{n-1} \ge n -1$.
\end{proof}

From now on assume \eqref{eq:a_n_big_1} holds, and so by Lemma \ref{lem:A_n_not_empty} $A_n \ne \emptyset$.
For $n \in \mathbb{N}$, define $R_n \subset \AAA^\Gamma$ as follows:
\begin{equation}\label{eq:R_n_def}
R_n := \left\{ x \in \AAA^\Gamma:~ (g \cdot x)\mid_{E_n} \in A_n\mbox{ for every } g \in \Gamma_n\right\}.
\end{equation}

%Because  $gE_n \cap g'E_n$ for every pair of distinct $g$,$g'$ in $\Gamma_n$, it follows that $R_n$ is naturally  homeomorphic  to $A_n^{\Gamma_n}$. In particular $R_n \ne \emptyset$ as $A_n \ne %\emptyset$.
%$R_n$ is a non-empty compact subset of $\AAA^\Gamma$ and that $g \cdot R_n = R_n$ for all $g \in \Gamma_n$.
Clearly $R_n$ is a compact $\Gamma_n$-invariant subset of $\AAA^\Gamma$. By \eqref{eq:G_n_E_n},  the action $\act{\Gamma_n}{R_n}$  is trivially isomorphic to the shift action $\act{\Gamma_n}{A_n^{\Gamma_n}}$.

\begin{lem}\label{lem:R_n_nested}
For every $n \ge 0$
$$R_{n+1} \subset R_{n}.$$
\end{lem}
\begin{proof}
Suppose $x \in R_{n+1}$.  Choose $g \in \Gamma_{n}$. By \eqref{eq:T_n_g_N}, $\Gamma_{n} =  \Gamma_{n+1}T_{n+1}$. Thus there exists $g' \in \Gamma_{n+1}$ and $t \in T_{n+1}$ so that $g= g't$.
By definition of $R_{n+1}$, $w := ((g')^{-1} \cdot x )\mid_{E_{n+1}} \in A_{n+1}$. By definition of $A_{n+1}$, it follows that $(t^{-1} \cdot w) \mid_{E_{n}} \in A_{n}$. But
$(t^{-1} \cdot w ) \mid_{E_{n}} = (g^{-1} \cdot x )\mid_{E_{n}}$, so $x \in R_{n}$.
\end{proof}

\begin{lem}\label{lem:R_n_trans_disjoint}
%For every $n \ge 1$,
%\begin{equation}\label{eq:R_n_trans_distjoint}
%R_n \cap \sigma^j R_n = \emptyset.
% x_{k+1}\ldots x_{k+b_n} \in A_n
%\end{equation}
For every $n \ge 1$ , $x \in R_n$ we have
\begin{equation}\label{eq:R_n_trans_distjoint}
(g \cdot x )\mid_{E_n} \in A_n \mbox{ if and only if } g \in \Gamma_n.
\end{equation}
% if and only if $k \in b_n \ZZ$.
% for every  $j \in \ZZ \setminus b_j \ZZ$.
\end{lem}

\begin{proof}
We prove this by induction on $n$.
%Fix $x \in R_n$.
To start the induction, note that if $x \in R_1$ then $x_{g}=0$ if and only if $g \in \Gamma_1$ so $x \cdot g \in R_1 $ if and only if $g \in \Gamma_1$.

For the induction step, fix $x \in R_n$. From the definition of $R_n$ it is clear that if $g \in \Gamma_n$ then $(g \cdot x)\mid_{E_n} \in A_n$.  By Lemma \ref{lem:R_n_nested} $R_{n} \subset R_{n-1}$, so by induction hypothesis, for every $g \in \Gamma \setminus  \Gamma_{n-1}$  we have
$ (g \cdot x)\mid_{E_{n-1}} \not\in A_{n-1}$.
%Choose    $g \in \Gamma \setminus  \Gamma_{n-1}$.% then $g \in \Gamma %\setminus  \Gamma_{n-1}$. Let $v := (g \cdot x )\mid_{E_n}$
%It follows that $v \mid_{E_{n-1}} \ne v^{(n-1)} \in A_{n-1}$, so $v
 %\not\in A_n$, so $x \not \in R_{n-1}$.


It remains to show that $g^{-1} \cdot x  \not \in R_{n-1}$ for every  $g \in \Gamma_{n-1} \setminus \Gamma_n$.
%Because $R_{n} = R_{n+1}\cdot g $ for every $g \in G_n$, it is enough to show  \eqref{eq:R_n_trans_distjoint} for $g \in T_n \setminus \{1\}$.
Choose $g \in \Gamma_{n-1} \setminus \Gamma_n$. Because $\Gamma_{n-1} =  \biguplus_{t \in T_{n-1}} \Gamma_{n} t$,  there exists  $g' \in \Gamma_n$ and $t \in T_{n-1} \setminus \{1\}$ so that $g = g't$.
Let $v : = ((g')^{-1} \cdot x)\mid_{E_n}$.
Because $x \in R_n$, it follows from the definition of $R_n$ that $ v \in A_{n}$,
so
$(t^{-1} \cdot v)\mid_{E_{n-1}} \in A_{n-1}  \setminus \{w^{(n-1)}\}$.
But  $(t^{-1} \cdot v)\mid_{E_{n-1}} = (g^{-1} \cdot x )\mid_{E_{n-1}}$, so  $g^{-1} \cdot x \not \in R_{n}$.
\end{proof}

Define
\begin{equation}\label{X_n_def}
X_n := \bigcup_{g \in E_n}g \cdot R_n.
\end{equation}

$X_n$ is a finite union of  compact subsets so it is compact.
%To see that $X_n$ is $G$-invariant, fix $h \in G$,
%and note that $G = \biguplus_{g \in E_n}G_n \cdot h$.
%$$X_n h =\left( \bigcup_{g \in E_n} R_n \cdot gh \right)$$
%Also note that if $h \in gG_n$ then $R_n \cdot h = R_n \cdot g$.
%It follows that $X_nh = X_n$.
Also, because $R_n$ is $\Gamma_n$-invariant, it follows that $X_n$ is $\Gamma$-invariant.
%Furthermore,   $X_n$ is a subshift of finite type, although we will not use this fact.

\begin{lem}\label{lem:X_n_nested}
For every $n \ge 1$, $X_{n+1} \subset X_n$.
\end{lem}
\begin{proof}
By Lemma \ref{lem:R_n_nested} $R_{n+1} \subset R_n$ so
$$X_{n+1} = \bigcup_{g \in E_{n+1}}g\cdot R_{n+1}   \subset \bigcup_{g \in E_{n+1}}g\cdot R_{n} $$
Because $E_{n+1}= T_{n+1} E_n $,
$$\bigcup_{g \in E_{n+1}}g \cdot R_{n}  = \bigcup_{t \in T_{n+1}} t \cdot \left(  \bigcup_{g \in E_{n}} g\cdot R_n \right)  =\bigcup_{t \in T_{n+1}} t\cdot  X_n.$$
Note that  $t \cdot X_n  = X_n$ for $t \in T_{n+1}$.
We conclude that indeed $X_{n+1} \subset X_n$.
\end{proof}


%*********
%The first requirement assures that two words $u,v \in A_n$ can not  partially overlap each other.



We now define:
\begin{equation}
X:= \bigcap_{n=1}^\infty X_n.
\end{equation}

%We first claim that  the subshift $X$ has no off-diagonal asymptotic pairs (for any sequence $(a_n)_{n=1}^\infty$ as above).



%The claim follows from the lemma below:


\begin{lem}\label{lem:X_n_assympt}
Suppose $(x,y) \in X_{n+1} \times X_{n+1}$ and $F \subset \Gamma$ are such that
\begin{equation}\label{eq:x_y_sim}
x_g=y_g \mbox{ for all  } g \in \Gamma \setminus F.
\end{equation}
In addition suppose that
$g_1\Gamma_n \ne g_2\Gamma_n$ for every pair of distinct elements $g_1,g_2 \in F$.
%*** should $E_{n-1}$ to $E_n$? ***
Then $x=y$.
\end{lem}
\begin{proof}
Let $(x,y) \in   X_{n+1} \times X_{n+1}$ and $F \subset \Gamma$ be as above. By \eqref{X_n_def} there exists $f,\tilde f \in E_{n+1}$ so that  $f \cdot x \in  R_{n+1} $ and $\tilde f \cdot y \in  R_{n+1} $.
Because $f \cdot x \in R_{n+1}$, it follows that
$(g\cdot f \cdot x)\mid_{E_{n+1}} \in A_{n+1}$ for all $g \in \Gamma_{n+1}$.


By \eqref{eq:x_y_sim}
\begin{equation}\label{eq:x_y_t_sim}
( f \cdot x)_g=(f\cdot y )_g \mbox{ for all  } g \in \Gamma \setminus (f F)
\end{equation}
Because $\Gamma_{n+1}$ is infinite and $F$ is finite, %Choose $g^{-1} \in %\Gamma_{n+1} \setminus fE_{n}$.
by \eqref{eq:x_y_t_sim} there are infinitely many $g \in \Gamma_{n+1}$ so that %$$(y \cdot \tilde e \cdot \tilde e^{-1} e \cdot g) =
$$ (g^{-1} \cdot f \cdot y)\mid_{E_{n+1}} = (g^{-1} \cdot f \cdot x)\mid_{E_{n+1}} \in A_{n+1}.$$
By Lemma \ref{lem:R_n_trans_disjoint} it follows that %$\tilde e^{-1}e g \in G_{n+1}$  so $\tilde e^{-1}e \in G_{n+1}$ so
$f \cdot y\in  R_{n+1}$.
%We claim that for every $g \in  G$
%Indeed, by \eqref{eq:x_y_t_sim}

Because $f \cdot x,f \cdot y \in R_{n+1}$ it follows that for every $g \in \Gamma_{n+1}$,
 $$(g^{-1} \cdot f \cdot x)\mid_{E_{n+1}},(g^{-1} \cdot f \cdot y)\mid_{E_{n+1}} \in A_{n+1}.$$
Thus for every $r \in E_{n+1}$
\begin{equation}\label{eq:sum_eq_u_v}
 \sum_{t \in T_{n+1}} (t^{-1} \cdot g^{-1} \cdot f \cdot x)_r =\sum_{ t \in T_{n+1}}  (t^{-1} \cdot g^{-1} \cdot f \cdot y)_r = s^{(n)}_r.
\end{equation}
Equivalently,
\begin{equation}\label{eq:sum_eq_u_v}
 \sum_{t \in T_{n+1}} x_{f^{-1} \cdot g \cdot t \cdot r} =\sum_{ t \in T_{n+1}}  y_{f^{-1} \cdot g \cdot t \cdot r} = s^{(n)}_r.
\end{equation}



We claim that there exists at most one $t \in T_{n+1}$ so that $f^{-1} \cdot g \cdot t \cdot r \in F$. Indeed, because $T_{n+1} \subset \Gamma_n$, and because $\Gamma_n$ is normal, the set
$\left\{ f^{-1} \cdot g \cdot t \cdot r :~  t \in T_{n+1} \right\}$ is contained in a single $\Gamma_n$-coset.

It follows that there exists at most one $t \in T_{n+1}$ so that   $x_{f^{-1} \cdot g \cdot t \cdot r} \ne y_{f^{-1} \cdot g \cdot t \cdot r}$.
By \eqref{eq:sum_eq_u_v}, it follows that in fact
\begin{equation}\label{eq:u_eq_v}
x_{f^{-1}\cdot g\cdot   t \cdot  r} = y_{f^{-1} \cdot g \cdot t\cdot r} \mbox{ for every } g\in \Gamma_{n+1}~,~ t \in T_{n+1} \mbox{ and } r \in E_{n},
\end{equation}
%In particular we can apply \eqref{eq:u_eq_v} with $t=1$.
 Every $\tilde g \in \Gamma$ can be written as $\tilde g = f^{-1} g t r$ for some $r \in E_{n+1}$, $t \in T_{n+1}$ and $g \in \Gamma_{n+1}$,
so $x_{\tilde g} = y_{\tilde g}$ for every $\tilde g \in \Gamma$.
\end{proof}


Our next goal is to  show that if the sequence $(a_n)_{n=1}^\infty$ grows sufficiently fast, then $h(X)>0$.

To be more specific:

\begin{lem}\label{lem:h_X_n_lower_bound}
If $(a_n)_{n=1}^\infty$ satisfies the assumption in the statement of Lemma \ref{lem:A_n_not_empty} then
\begin{equation}\label{h_X_n_lower_bound}
 h(\act{\Gamma}{X_n})\ge \frac{a_1-1}{a_1}\log(2)- \frac{1}{a_1}\log(3) - \sum_{k=2}^n \left(\frac{2}{3^{b_{k-2}}+1}+\frac{2}{a_k}\right)\cdot \log(3).
% \log3 - (2\log 3)\sum_{n=1}^\infty \frac{1}{a_n}+\sum_{n=2}^\infty \log\left( 1- \frac{1}{3^{b_{n-2}}+1}\right) + \log(2/3).
\end{equation}
\end{lem}
\begin{proof}
Because $\act{\Gamma_n}{tR_n}$  is isomorphic to the full-shift over the alphabet $A_n$ for every $n \ge1 $ and every $t \in E_n$,
 it follows %by Proposition \ref{prop:entropy_full_shift}
that
$$h(\act{\Gamma_n}{tR_n}) = \log |A_n|.$$
Because $X_n=  \bigcup_{t \in E_n}t R_n$ is a finite union of $\Gamma_n$-invariant sets it follows that
$$h(\act{\Gamma_n}{X_n}) = \max_{t \in E_n} h(\act{\Gamma_n}{t R_n}) = \log |A_n|.$$
Thus, by the subgroup formula for entropy  (see \cite[Theorem $2.16$]{MR1878075} for a stronger result):
%(Proposition \ref{prop:entropy_subgroup} above):
$$h(\act{\Gamma}{X_n})   = [\Gamma:\Gamma_n]^{-1} \log |A_n|= b_n^{-1} \log |A_n|$$

Taking logs in  \eqref{eq:A_n_big} we get:
\begin{equation}\label{eq:log_A_n}
\log |A_n| \ge (a_{n} -1) \log\left( |A_{n-1}|- 1\right) - b_{n-1} \log 3 .
\end{equation}
Denote
\begin{equation}
h_n := h(\act{\Gamma}{X_n})= b_n^{-1} \log |A_n| .
\end{equation}
Recall that $|A_0|=3$ and $b_1=a_1$. So substituting  $n=1$ in \eqref{eq:log_A_n} we get:
\begin{equation}\label{eq_h1}
h_1 = \frac{1}{a_1}\log(|A_1|) \ge \frac{a_1-1}{a_1}\log(2)- \frac{1}{a_1}\log(3).
\end{equation}
% By  Lemma \ref{lem:A_n_not_empty} $|A_n| \ge 3$ so $|A_n|-  1 \ge |A_n|/2$.
%Thus
Dividing \eqref{eq:log_A_n} by $b_n$ we get:
$$\frac{1}{b_n}\log |A_n| \ge \frac{(a_n -1)\cdot b_{n-1}}{b_n} \cdot \frac{\log \left(|A_{n-1}|-1 \right)}{b_{n-1}}  -\frac{ b_{n-1}}{b_n} \log 3.$$
Using the relation $b_n = a_n b_{n-1}$ it follows that:
$$h_n \ge \left(1-  \frac{1}{a_n}\right)\left[h_{n-1}\cdot\frac{\log(|A_{n-1}|-1)}{\log |A_{n-1}|}\right] - \frac{1}{a_n}\log 3 .$$
Apply  the estimate:
$$\frac{\log(x-1)}{\log(x)} = 1 -\frac{\log(x)-\log(x-1)}{\log(x)} \ge 1- \frac{1}{(x-1)\log(x)} \ge 1- \frac{2}{x} \mbox{ for } x\ge 2,$$
with $x= |A_{n-1}|$ to get:
$$h_n\ge \left(1-  \frac{1}{a_n}\right)\left[h_{n-1}\left(1-\frac{2}{|A_{n-1}|}\right)\right] - \frac{1}{a_n}\log 3 .$$
Equivalently:
$$h_n - h_{n-1} \ge -\left(\frac{1}{a_n}+\frac{2}{|A_{n-1}|} -\frac{1}{a_n|A_{n-1}|}\right)h_{n-1}- \frac{1}{a_n}\log 3.$$
By  Lemma \ref{lem:A_n_not_empty}, for  every $n \ge 2$:
$$|A_{n-1}| \ge 3^{b_{n-2}}+1.$$
Note that $X_0= (\ZZ/3\ZZ)^\Gamma$ so $h_0= h(\act{\Gamma}{X_0}) = \log 3$. Because the sequence $\{h_n\}_{n=1}^\infty$ is monotone non-increasing $h_n \le \log 3$ for every $n \ge 1$.
We thus have:
\begin{equation}\label{eq:h_diff}
h_k - h_{k-1} \ge  -\left(\frac{2}{3^{b_{k-2}}+1}+\frac{2}{a_k}\right)\cdot \log(3) \mbox{ for every } k \ge 2.
\end{equation}
%It follows that
%$$h(\act{\Gamma}{X_n}) \ge h(\act{\Gamma}{X_{n-1}}) -\frac{2}{a_n}\log 3 + \log(1-\frac{1}{|A_{n-1}|}).$$
%By Lemma \ref{lem:A_n_not_empty},
%$$ \log(1-\frac{1}{|A_{n-1}|}) \ge  \log(1 - \frac{1}{3^{b_{n-2}}+1}) \mbox{ for every } n \ge 2.$$
%Also $b_n \ge  a_n$ so $\frac{\log 2}{b_n}  \le \frac{\log 3}{a_n}$.
%We thus have
%$$h(\act{\Gamma}{X_n}) \ge h(\act{\Gamma}{X_{n-1}}) - (2\log 3) \cdot \frac{1}{a_n}+\log(1 - \frac{1}{3^{b_{n-2}}+1}) \mbox{ for every } n \ge 2,$$
and
%$$h(\act{\Gamma}{X_1}) \ge \log(3) - (2\log 3) \cdot \frac{1}{a_2}+\log(2/3).$$
Summing \eqref{eq:h_diff} over $2 \le k \le n$ we get
$$ h_n \ge h_1 - \sum_{k=2}^n \left(\frac{2}{3^{b_{k-2}}+1}+\frac{2}{a_k}\right)\cdot \log(3).$$
The estimate \eqref{h_X_n_lower_bound} follows using  the estimate \eqref{eq_h1} for $h_1$.
\end{proof}

To conclude our construction:
\begin{prop}
If %$\sum_{n=1}^\infty \frac{1}{a_n} < \frac{1}{3}$ and
 $(a_n)_{n=1}^\infty$ satisfies the assumption in the statement of Lemma \ref{lem:A_n_not_empty}  and $a_1,a_2,\ldots $ are sufficiently big then
 $h(\act{\Gamma}{X})>0 $  and $X$ has no off-diagonal asymptotic pairs.
\end{prop}
\begin{proof}
Because $X = \bigcap_{n=1}^\infty X_n$, and $X_{n+1} \subset X_n$ it follows from  %Proposition \ref{entropy_decreasing_intersection}
upper semi-continuity of topological entropy (see \cite[Appendix $A$]{ETS:10144219})
that
$$h(\act{\Gamma}{X})= \inf_n  h(\act{\Gamma}{X_n}).$$
By  Lemma \ref{lem:h_X_n_lower_bound},
$$\inf_n h(\act{\Gamma}{X_n}) \ge \frac{a_1-1}{a_1}\log(2)- \frac{1}{a_1}\log(3) - \sum_{k=2}^\infty \left(\frac{2}{3^{b_{k-1}}+1}+\frac{2}{a_k}\right)\cdot \log(3).$$
If $a_1$ is sufficiently big than $ \frac{a_1-1}{a_1}\log(2)- \frac{1}{a_1}\log(3)  > \log(2) -\epsilon$. Also, by choosing $a_2,a_3,\ldots$ to grow sufficiently fast
the series $\sum_{k=2}^\infty \left(\frac{2}{3^{b_{k-2}}+1}+\frac{2}{a_k}\right)$ converges and the sum can be made smaller than $\frac{1}{2}\log(3) + \epsilon$ for any $\epsilon >0$.
This proves $\act{\Gamma}{X}$ has  positive entropy as soon as $a_1,a_2,\ldots $ are sufficiently big.

Let us show that $X$ has no  off-diagonal asymptotic pairs.
Suppose $(x,y) \in X\times X$ is an asymptotic pair.   If follows
that $(x,y)$  satisfy \eqref{eq:x_y_sim} for some finite $F \subset \Gamma$.
Because $\bigcap_n \Gamma_n = \{1\}$ it follows that there exists $n$ so that the map $g \mapsto g\Gamma_n$ is injective of $F$.
 Because $(x,y) \in X \times X \subset X_{n+1} \times X_{n+1}$ it follows from  Lemma \ref{lem:X_n_assympt} that $x=y$.
\end{proof}


\bibliographystyle{abbrv}
\bibliography{asymptotic_pairs}
\end{document}

\begin{lem}\label{lem:T_n_G_n_E_n}
For every $g,\tilde g \in \Gamma$
there exists at most one
$ t \in T_{n+1}$ so that
\begin{equation}\label{eq:T_n_G_n_E_n}
g \cdot  t \cdot \tilde g \in E_{n}.
\end{equation}
\end{lem}
\begin{proof}
Let $$q = g \cdot t \cdot \tilde g\in E_n.$$
We can uniquely write
$$g =  h \cdot s \cdot f \mbox{ with } f\in E_n,~s \in T_{n+1}  \mbox{ and }  h \in \Gamma_{n+1}.$$
 $$\tilde g = \tilde h \cdot \tilde s \cdot \tilde f \mbox{ with } \tilde f \in E_n,~\tilde s \in T_{n+1}  \mbox{ and } \tilde h \in \Gamma_{n+1}.$$
Write  \eqref{eq:T_n_G_n_E_n}  as: %because $E_n= there exists $t_1,\ldots,t_{n-1}$ with $t_i \in T_i$ so that
\begin{equation}\label{eq:mod_G_n}
 h \cdot s \cdot f \cdot t \cdot \tilde h \cdot \tilde s \cdot \tilde f \in E_{n}.
\end{equation}

Because $T_{n+1} \subset \Gamma_{n}$ , $\Gamma_{n+1} \lhd \Gamma_n$ and $\Gamma_n$ is a  normal subgroup of $\Gamma$, it follows % from \eqref{eq:mod_G_n}
 that
\begin{equation}\label{eq:e_r_t}
q \Gamma_{n} = f \cdot \tilde  f \Gamma_{n},
\end{equation}
Because in every $\Gamma_n$-coset there is a unique element of $E_n$, and $q \in E_n$, by \eqref{eq:e_r_t}, $f\cdot \tilde f$ determines $q$ uniquely. In particular, $q \in E_n$ is uniquely determined by $g$ and $\tilde g$.
It follows that
$t = g^{-1} \cdot q \cdot \tilde g^{-1}$ is also uniquely determined by $g$ and $\tilde g$ and the condition $q \in E_n$.
\end{proof}

More generally, if $\alpha \in \Alg{\Gamma}{X}$  is expansive and satisfies p.o.t, then there exists $k \in \mathbb{N}$ and a
 left $\mathbb{Z}\Gamma$-submodule $J$ of $(\mathbb{Z}\Gamma)^k$ so that
 the  Pontryagin dual  is isomorphic as a $\ZZ\Gamma$-module to $(\mathbb{Z}\Gamma)^k / J$, and $J$ has the following additional property:
If    $J' \subset J$ is a left $\mathbb{Z}\Gamma$-submodule  and the dual of $(\mathbb{Z}\Gamma)^k / J'$ is expansive then $J'$ is finitely generated.
%there exists $k \in \mathbb{N}$ and a finite set $W \subset \mathbb{Z}G$ so that every element of $W$ is invertible in $\ell^1(G)$  the Pontryagin
%dual $\widehat X$ of $X$ is isomorphic to  $(\mathbb{Z}G)^k / I_W$, where $I_W$ is the left $\mathbb{Z}G$-module generated by $W$. ***
