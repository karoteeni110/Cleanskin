

The estimator of the SMC asymptotic variance that we propose is a natural modification of the CLE introduced in \cite{chan:lai:2013}. As in \cite{kitagawa:sato:2001,olsson:cappe:douc:moulines:2006}, the main idea is to reduce the degree of genealogical tracing, which has a devastating effect on the CLE's numerical stability, at the cost of a small bias, which may be controlled using the forgetting properties of the model. That this measure stabilises numerically the estimator in the long run is confirmed by our theoretical results in Section~\ref{sec:theoretical:results}, which are obtained under---what we believe---minimal model assumptions being satisfied also for many models with possibly non-compact state space. The fact that the bias can be shown to decrease geometrically fast as the lag increases indicates that tight control of the bias is possible also for moderately large particle sample sizes. This is approved by our numerical experiments in Section~\ref{sec:numerical:study}, which report, in the examples under consideration, a negligible bias already for some thousands of particles.

Needless to say, the success of our approach depends highly on the interplay between the forgetting properties of the model, the particle sample size, and the choice of the lag. Adaptive lag design is hence a natural direction for future research. Moreover, as our estimator provides numerically stable estimates of the asymptotic variance, it should be highly useful for online SMC sample size adaptation. Here one natural approach could be to estimate the variance of the next time step using a part of the particle population (\emph{pilot sampling}) and then ``refuel'' the particle system at time steps of high variance (here the techniques developed in \cite{lee:whiteley:2016}, where the authors consider the SMC sample allocation problem in the batch mode, should be useful for the theoretical analysis). Finally, casting, using the results obtained in \cite{lindsten:shoen:olsson:2011}, our estimator and analysis into the framework of \emph{Rao-Blackwellised SMC algorithms}, should be of high relevance for high-dimensional applications. 
