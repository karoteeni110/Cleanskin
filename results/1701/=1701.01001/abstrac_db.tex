This paper discusses variance estimation in sequential Monte Carlo methods, alternatively termed particle filters. The variance estimator that we propose is a natural modification of that suggested by H.~P.~Chan and T.~L.~Lai [A general theory of particle filters in hidden Markov models and some applications. \emph{Ann. Statist.}, 41(6):2877--2904, 2013], which allows the variance to be estimated in a single run of the particle filter by tracing the genealogical history of the particles. However, due particle lineage degeneracy, the estimator of the mentioned work becomes numerically unstable as the number of sequential particle updates increases. Thus, by tracing only a part of the particles' genealogy rather than the full one, our estimator gains long-term numerical stability at the cost of a bias. The scope of the genealogical tracing is regulated by a lag, and under mild, easily checked model assumptions, we prove that the bias tends to zero geometrically fast as the lag increases. As confirmed by our numerical results, this allows the bias to be tightly controlled also for moderate particle sample sizes. 