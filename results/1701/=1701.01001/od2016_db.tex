\documentclass[preprint]{imsart}
%\documentclass[bj]{imsart}

\usepackage{lineno,hyperref}


\usepackage{aliascnt}
\usepackage{amsmath}
\usepackage{amssymb}
\usepackage{amsthm}
\usepackage{bibentry}
\usepackage{bbm}
\usepackage{booktabs}
\usepackage{color}
\usepackage{enumerate}
\usepackage{float}
\usepackage{graphicx}
\usepackage{hyperref}
\usepackage{ifthen}
\usepackage{mathtools}
\usepackage[numbers]{natbib}
\usepackage[linesnumbered,vlined]{algorithm2e}
\usepackage{stmaryrd}
\usepackage{ushort}
\usepackage{xargs}


\RequirePackage[colorlinks,citecolor=blue,urlcolor=blue]{hyperref}

% provide arXiv number if available:
%\arxiv{arXiv:0000.0000}

% put your definitions there:
\startlocaldefs


%\providecommand*{\definitionautorefname}{Definition}
%\providecommand*{\lemmaautorefname}{Lemma}
%\providecommand*{\exampleautorefname}{Example}
\providecommand*{\propositionautorefname}{Proposition}
%\providecommand*{\exerciseautorefname}{Exercise}
%\providecommand*{\corollaryautorefname}{Corollary}
%\providecommand*{\chapterautorefname}{Chapter}
%\providecommand*{\partautorefname}{Part}
%\providecommand*{\remarkautorefname}{Remark}
%\renewcommand*\sectionautorefname{Section} 
%
%
%
\newtheorem{theorem}{Theorem}
\newaliascnt{proposition}{theorem}
\newtheorem{proposition}[proposition]{Proposition}
\aliascntresetthe{proposition}
%
\newaliascnt{lemma}{theorem}
\newtheorem{lemma}[lemma]{Lemma}
\aliascntresetthe{lemma}
%
%\newaliascnt{corollary}{theorem}
%\newtheorem{corollary}[corollary]{Corollary}
%\aliascntresetthe{corollary}
%
\newaliascnt{definition}{theorem}
\newtheorem{definition}[definition]{Definition}
\aliascntresetthe{definition}
%
\newaliascnt{example}{theorem}
\newtheorem{example}[example]{Example}
\aliascntresetthe{example}
%
\newaliascnt{remark}{theorem}
\newtheorem{remark}[remark]{Remark}
\aliascntresetthe{remark}

% A

\newcommandx\A[2][1=]{
\ifthenelse{\equal{#1}{}}
{\hspace{-1mm}(\textbf{A\ref{#2}})\hspace{-1mm}}
{\hspace{-1mm}(\textbf{A\ref{#1}--\ref{#2}})\hspace{-1mm}}
}

% B

\newcommandx\B[2][1=]{
\ifthenelse{\equal{#1}{}}
{\hspace{-1mm}(\textbf{S})\hspace{-1mm}}
{\hspace{-1mm}(\textbf{S\ref{#1}--\ref{#2}})\hspace{-1mm}}
}
%\newcommandx\B[2][1=]{
%\ifthenelse{\equal{#1}{}}
%{\hspace{-1mm}(\textbf{S\ref{#2}})\hspace{-1mm}}
%{\hspace{-1mm}(\textbf{S\ref{#1}--\ref{#2}})\hspace{-1mm}}
%}
\newcommand{\bd}{c}
\newcommand{\bias}[2]{\beta_{#2} \langle #1 \rangle} 
\newcommand{\biasfilt}[2]{\bar{\beta}_{#2} \langle #1 \rangle} 
\newcommand{\binset}[1]{\mathsf{I}_{#1}}
\newcommand{\Binsp}[1]{\mathsf{B}_{#1}}
\newcommand{\bmf}[1]{\mathbb{F}(#1)}


% C

\newcommand{\cat}{\mathsf{Cat}}
\newcommand{\chunk}[3]{{#1}_{#2}^{#3}}

% D 

\newcommand{\DDelta}[3]{\Delta_{#1}\langle #2\rangle(#3)}
\newcommand{\dlim}{\stackrel{\mathcal D}{\longrightarrow}}

% E

\newcommand{\E}{\mathbb{E}}
\newcommand{\ed}{g}
\newcommand{\Efd}{\mathcal{E}}
\newcommand{\enoch}[3]{E_{#1,#2}^{#3}}
\newcommand{\epart}[2]{\xi_{#1}^{#2}}
\newcommand{\eqdef}{\vcentcolon=}
\newcommand{\Esp}{\mathsf{E}}
\newcommandx{\eve}[3][1=]{\ifthenelse{\equal{#1}{}}{E_{#2}^{#3}}{E_{#1,#2}^{#3}}}

% F

\newcommandx{\filt}[1][1=]{\ifthenelse{\equal{#1}{}}{\filtsymb}{\filtsymb \langle #1 \rangle}}
\newcommand{\filtpart}[1][1=]{\ifthenelse{\equal{#1}{}}{\filtsymb_\N}{\filtsymb_\N \langle #1 \rangle}}
\newcommand{\filtsymb}{\bar{\eta}}
\newcommandx{\filtvariance}[3][1=,3=]{\bar{\sigma}^{#1}_{#3} \langle #2 \rangle}

% G

\newcommandx{\gen}[1][1=]{\ifthenelse{\equal{#1}{1}}{G}{G'}} 
\newcommand{\genkernel}{\kernel{K}}

% H

\newcommand{\hk}{\kernel{M}}


% I

\newcommand{\init}{\chi}
\newcommand{\ind}[2]{I_{#1}^{#2}}
\newcommand{\intvect}[2]{\llbracket #1, #2 \rrbracket}

% K 

\newcommand{\kernel}[1]{\mathbf{#1}}

% L

\newcommand{\lag}{\lambda}
\newcommand{\lagtime}[2]{#1(#2)}
%\newcommand{\lagtime}[2]{\langle #1 \rangle_{#2}}
\newcommandx{\likeli}[3][1=]{\pi_{#1} \langle #2 \rangle(#3)}
\newcommand{\limitfunc}[1]{\pi \langle #1 \rangle}

% M

\newcommand{\md}{m}
\newcommand{\mdlow}{\ushort{\varepsilon}}
\newcommand{\mdup}{\bar{\varepsilon}}
%\newcommand{\mdr}{\mathcal{M}}
\newcommand{\mdr}{\mathbb{M}}
\newcommand{\me}{\mathrm{e}}
\newcommand{\mk}{\kernel{M}}
\newcommand{\mklow}{\ushort{\varepsilon}}
\newcommand{\mkup}{\bar{\varepsilon}}
\newcommand{\mumeas}[2]{\mu_{#1} \langle #2 \rangle}

% N

\newcommand{\N}{N}
\newcommand{\nset}{\mathbb{N}}
\newcommand{\nsetpos}{\mathbb{N}^\ast}

% O

\newcommand{\1}{\mathbbm{1}}
\newcommand{\ordo}{\mathcal{O}}

% P 

\newcommand{\p}{p}
\newcommand{\partfd}[1]{\mathcal{F}_{#1}}
\newcommand{\per}{\zeta}
\newcommand{\perblock}{\bar{\zeta}}
\newcommand{\plim}{\stackrel{\prob}{\longrightarrow}}
\newcommandx{\pot}[1][1=]{\ifthenelse{\equal{#1}{}}{g}{g \langle #1 \rangle}}
\newcommand{\potlow}{\ushort{\delta}}
\newcommand{\potup}{\bar{\delta}}
\newcommand{\predsymb}{\eta}
\newcommandx{\pred}[1][1=]{\ifthenelse{\equal{#1}{}}{\predsymb}{\predsymb \langle #1 \rangle}}
\newcommand{\predpart}[1][1=]{\ifthenelse{\equal{#1}{}}{\predsymb_\N}{\predsymb_\N \langle #1 \rangle}}
\newcommand{\prob}{\mathbb{P}}
\newcommand{\probmeas}[1]{\mathbb{M}(#1)}
\newcommand{\probdoeblin}[2]{\mu_{#1} \langle #2 \rangle}
\newcommand{\rmd}{\mathrm{d}}

% R

\newcommand{\rate}{\rho}
\newcommand{\refm}{\nu}
\newcommand{\rset}{\mathbb{R}}
\newcommand{\rsetpos}{\mathbb{R}^\ast_+}

% S

% T

\newcommand{\tbw}{\emph{(To be written.)}}
\newcommand{\term}[3][]{\upsilon_{#2,#3} \langle #1 \rangle}
%\newcommand{\thickhline}{%
%    \noalign {\ifnum 0=`}\fi \hrule height 2pt
%    \futurelet \reserved@a \@xhline
%}
%\newcolumntype{w}{@{\hskip\tabcolsep\vrule width 2pt\hskip\tabcolsep}}
%\makeatother

% U 

\newcommandx{\uk}[1][1=]{\ifthenelse{\equal{#1}{}}{\kernel{Q}}{\kernel{Q} \langle #1 \rangle}}
\newcommand{\unitstr}[2]{1_{#1}}
\newcommand{\unpredsymbol}{\gamma}
\newcommandx{\unpred}[1][1=]{\ifthenelse{\equal{#1}{}}{\unpredsymb}{\unpredsymb \langle #1 \rangle}}
\newcommand{\unpredpart}[1][1=]{\ifthenelse{\equal{#1}{}}{\unpredsymbol_\N}{\unpredsymbol_\N \langle #1 \rangle}}

% V 

\newcommandx{\varest}[3][1=,3=]{\ifthenelse{\equal{#3}{}}{\sigma^{#1}_\N \langle #2 \rangle}{\sigma^{#1}_{\N, #3} \langle #2 \rangle}}
\newcommandx{\varestfilt}[3][1=,3=]{\ifthenelse{\equal{#3}{}}{\bar{\sigma}^{#1}_\N \langle #2 \rangle}{\bar{\sigma}^{#1}_{\N, #3} \langle #2 \rangle}}
\newcommandx{\variance}[3][1=,3=]{\sigma^{#1}_{#3} \langle #2 \rangle}


% W 

\newcommand{\wgt}[2]{\omega_{#1}^{#2}}
\newcommand{\wgtsum}[1]{\Omega_{#1}}

% X 

\newcommand{\Xsp}{\mathsf{X}}
\newcommand{\Xfd}{\mathcal{X}}

% Y

\newcommand{\Ysp}{\mathsf{Y}}
\newcommand{\Yfd}{\mathcal{Y}}

% Z

\newcommand{\zerostr}[1]{0_{#1}}
\newcommand{\Zsp}{\mathsf{Z}}
\newcommand{\Zfd}{\mathcal{Z}}
\newcommand{\zset}{\mathbb{Z}}

% Hypotheses

\newcounter{hypA}
\newenvironment{hypA}{\refstepcounter{hypA}\begin{itemize}
  \item[({\bf A\arabic{hypA}})]}{\end{itemize}}
%\newenvironment{hypA}{\begin{sf}\refstepcounter{hypA}\begin{itemize}
%  \item[({\bf A\arabic{hypA}})]}{\end{itemize}\end{sf}}

\newcounter{hypB}
%\newenvironment{hypB}{\refstepcounter{hypB}\begin{itemize}
%  \item[({\bf S\arabic{hypB}})]}{\end{itemize}}
\newenvironment{hypB}{\refstepcounter{hypB}\begin{itemize}
  \item[({\bf S})]}{\end{itemize}}



\endlocaldefs

\begin{document}

\begin{frontmatter}

% "Title of the Paper"

\title{Numerically stable online estimation of variance in particle filters}

\runtitle{Estimation of variance in particle filters}

\begin{aug}
\author{\fnms{Jimmy} \snm{Olsson}\thanksref{a,t1}\ead[label=e1]{jimmyol@kth.se}}
\and
\author{\fnms{Randal} \snm{Douc}\thanksref{b}\ead[label=e2]{randal.douc@it-sudparis.eu}}

\address[a]{Department of Mathematics \\
KTH Royal Institute of Technology \\
SE-100 44  Stockholm, Sweden \\
\printead{e1}}

\address[b]{D\'epartement CITI \\ 
TELECOM SudParis \\
9 rue Charles Fourier, 91000 EVRY \\
\printead{e2}}

\thankstext{t1}{J.~Olsson is supported by the Swedish Research Council, Grant 2011-5577.}

\runauthor{J. Olsson and R. Douc}

\affiliation{KTH Royal Institute of Technology and Institut T\'el\'ecom/T\'el\'ecom SudParis}

\end{aug}

\begin{abstract}
\input{abstract}
\end{abstract}

\begin{keyword}[class=MSC]
\kwd[Primary ]{62M09}
%\kwd{}
\kwd[; secondary ]{62F12}
\end{keyword}


\begin{keyword}
\kwd{Asymptotic variance}
\kwd{Feynman-Kac models}
\kwd{hidden Markov models}
\kwd{particle filters}
\kwd{sequential Monte Carlo methods}
\kwd{state-space models}
\kwd{variance estimation}
\end{keyword}

% history:
% \received{\smonth{1} \syear{0000}}

%\tableofcontents

\end{frontmatter}

% Introduction
\section{Introduction}
\label{sec:introduction}
%\begin{figure}\center
  %\missingfigure[figheight=.10\textheight, figwidth=\textwidth]{Graphical Abstract}
%  \includegraphics[height=.15\textheight]{graphical_abstract-crop}
%  \caption{Scheme of analyses involving the core structural connectivity matrix.\label{fig:process-illustration}}
%\end{figure}

Isolating the common brain connectivity network from a population is a main problem in current neuroscience~\cite{Bullmore2009,Gong2009,Wassermann2016}. Recent evidence suggests that there's a common and densely connected brain connectome across humans~\cite{Bassett2013}. In this work we present a new approach for selecting these common connections, combining recent topological hypotheses~\cite{Bassett2013}  and  current methods~\cite{Gong2009,Wassermann2016}.

Finding the common brain connectome across subjects has the potential to increase our understanding of the relationship between function and structure in the brain. This relationship is one of the main open questions in neuroscience~\cite{Bullmore2009,Donahue2016}. Moreover, knowledge about the most common connections in a population will facilitate clinical and cognitive Diffusion MRI analyses by reducing the number of surveyed connections, increasing the statistical power of those analyses. Finding the common connectome will also allow us to increase our knowledge about the brain structure by comparing core networks across different populations.

We formalize the problem of selecting the common connections combining graph theory and statistics. Then, we prove that the problem is \NP-Hard and propose a polynomial-time algorithm to find approximate solutions. To do this, we develop an exact polynomial-time algorithm for a relaxed version of the problem and prove the algorithm's correctness and complexity.

Currently, the most used algorithm to extract a population's core structural connectivity network (CSNC)~\cite{Gong2009} uses an statistical approach: first, compute a connectivity matrix for each subject; then, analize each connection separately with a hypothesis test, using as null hypothesis that that edge is not present in the population; finally, construct a binary graph with the edges for which the null hypothesis was rejected. The main problem of Gong et al.'s~\cite{Gong2009} algorithm is that the resulting graph can be a set of disconnected subgraphs. Moreover, recent studies have shown that the brain has a \emph{core} network tightly connected and a sparsely connected \emph{outer} one~\cite{Bassett2013}. In other words, this approach ignores the resulting network's topology. Performing statistical analyses in a feature set chosen by hypothesis testing incurs in the double dipping problem~\cite{Kriegeskorte2009}.

A newer approach to solve the CSNC problem, designed by Wassermann et al.~\cite{Wassermann2016}, uses graph theory to get a connected CSCN: first, compute a binary connectivity graph for each subject using a threshold;  for each possible connection compute the ``cost'' of including or excluding it from the common graph by evaluating in how many subjects that connection is present; finally, construct the binary graph with all the edges that is ``cheaper'' to include than to exclude and connect the resulting graph if it's disconnected, using the minimum possible cost. This algorithm guarantees that the resulting graph is connected, but the connection binarization discards significant information for the resulting common network. In other words, it discards information of the probability of each connection being in the brain. This is problematic because the resulting graph may include edges for which tractography assigned a very low existence probability across subjects. Also, the outer part of the brain, the connections which do not result in the core network, should also be sparsely connected~\cite{Bassett2013}, which this algorithm does not enforce.

In this work we propose, for the first time, a polynomial-time algorithm to obtain the CSCN of a population  addressing the issues listed above. Our algorithm combines the recent graph-theoretical approach~\cite{Wassermann2016} with the statistical awareness of the most popular one~\cite{Gong2009}. We start by formalizing the problem, which allow us to prove that it's \NP-Hard. Then, we propose a first algorithm that solves a relaxed version of the problem in an exact way, giving the best possible core graph for our formalization. Then, we adapt it to guarantee a connected result, agreeing with recent evidence on structural connectivity network topology \cite[e.g.]{Bassett2013}. Finally, we validate our approach using 300 subjects from the HCP database and comparing the performance of the networks obtained by our new approach, Wassermann et al.'s~\cite{Wassermann2016} and Gong et al.'s~\cite{Gong2009} predicting connectivity values from handedness in the core network.

% Preliminaries
\section{Preliminaries}
\label{sec:preliminaries}
\subsection{Some notation and conventions}

We assume that all random variables are defined on a common probability space $(\Omega, \mathcal{F}, \prob)$. The set of natural numbers is denoted by $\nset = \{0, 1, 2, \ldots\}$, and we let $\nsetpos = \nset \setminus \{�0\}$ be the positive ones. For all $(m, n) \in \nset^2$, we set $\intvect{m}{n} \eqdef \{m, m + 1, \ldots, n\}$. The set of nonnegative real numbers is denoted by $\rset_+$. For any quantities $\{ a_\ell \}_{\ell = 1}^m$, vectors are denoted by $\chunk{a}{\ell}{m} \eqdef (a_\ell, \ldots, a_m)$. 

We introduce some measure and kernel notation. Given some state space $(\Esp, \Efd)$, we denote by $\bmf{\Efd}$ and $\probmeas{\Efd}$ the spaces of bounded measurable functions and probability measures on $(\Esp, \Efd)$, respectively. For any functions $(h, h') \in \bmf{\Efd}^2$ we define the product function $h \varotimes h' : \Esp^2 \ni (x, x') \mapsto h(x) h'(x')$. The identity function $x \mapsto x$ is denoted by $\operatorname{id}$. Let $\mu$ be a measure on $(\Esp, \Efd)$; then for any $\mu$-integrable function $h$, we denote by
$$
\mu h \eqdef \int h(x) \, \mu(\rmd x)
$$
the Lebesgue integral of $h$ w.r.t. $\mu$. In addition, let $(\Esp', \Efd')$ be some other measurable space and $\genkernel$ some possibly unnormalised transition kernel $\genkernel : \Esp \times \Efd' \rightarrow \rset_+$. The kernel $\genkernel$ induces two integral operators, one acting on functions and the other on measures. More specifically, given a measure $\nu$ on $(\Esp, \Efd)$ and a measurable function $h$ on $(\Esp', \Efd')$, we define the measure 
$$
    \nu \genkernel : \Efd' \ni A \mapsto \int \genkernel(x, A) \, \nu(\rmd x) 
$$
and the function 
$$
    \genkernel h : \Esp \ni x \mapsto \int h(y) \, \genkernel(x, \rmd y),
$$
whenever these quantities are well defined. 


\subsection{Randomly perturbed Feynman-Kac models} 
\label{sec:Feynman:Kac:models}
Let $(\Xsp, \Xfd)$ and $(\Zsp, \Zfd)$ be a pair of general measurable spaces. Moreover, let $\kernel{K}$ and $\init$ be a Markov transition kernel and a probability measure on $(\Xsp, \Xfd)$, respectively, and $\{ \pot[z] : z \in \Zsp \}$ a family of real-valued, positive, and measurable \emph{potential functions} on $(\Xsp, \Xfd)$. For  all vectors $\chunk{z}{k}{m} \in \Zsp^{m - k + 1}$, we define unnormalised transition kernels
$$
    \uk[\chunk{z}{k}{m}] : \Xsp \times \Xfd \ni (x_k, A) 
    \mapsto \idotsint \1_A(x_{m + 1}) \prod_{\ell = k}^m 
    \pot[z_\ell](x_\ell) \, \mk(x_\ell, \rmd x_{\ell + 1}),
$$
with the convention $\uk[\chunk{z}{k}{m}](x, A) = \delta_x(A)$ if $m < k$ (where $\delta_x$ denotes the Dirac mass located at $x$),  
and probability measures 
\begin{equation} \label{eq:def:pred}
    \pred[\chunk{z}{k}{m}] : \Xfd \ni A 
    \mapsto \frac{\init \uk[\chunk{z}{k}{m}] \1_A}
    {\init \uk[\chunk{z}{k}{m}] \1_\Xsp}. 
\end{equation}
Using these definitions we may, given a sequence $\{ z_n \}_{n \in \nset}$ of \emph{perturbations} in $\Zsp$, express the \emph{Feynman-Kac distribution flow} $\{ \pred[\chunk{z}{0}{n}] \}_{n \in \nset}$ recursively as 
\begin{equation} \label{eq:pred:rec}
    \pred[\chunk{z}{0}{n}] 
    = \frac{\pred[\chunk{z}{0}{n - 1}] \uk[z_n]}{\pred[\chunk{z}{0}{n - 1}]  \uk[z_n] \1_\Xsp}, \quad n \in \nset 
\end{equation}
(where, by the previous convention, $\pred[\chunk{z}{0}{- 1}] = \init$). Even though the previous model may be applied in a non-temporal context, we will often refer to the index $n$ as ``time''. 

\begin{example}[partially dominated state-space models] \label{example:state:space:model}
Let $(\Xsp, \Xfd)$ be a measurable space, $\hk : \Xsp \times \Xfd \rightarrow [0, 1]$ a Markov transition kernel, and $\init$ a probability measure on $(\Xsp, \Xfd)$ (the latter being referred to as the \emph{initial distribution}). In addition, let $(\Ysp, \Yfd)$ be another measurable space and $\ed : \Xsp \times \Ysp \rightarrow \rset_+$ a Markov transition density with respect to some reference measure $\refm$ on $(\Ysp, \Yfd)$. By a general state-space model we mean the canonical version of the bivariate Markov chain $\{ (X_n, Y_n) \}_{n \in \nset}$ having transition kernel 
$$
    \Xsp \times \Ysp \times \Xfd \varotimes \Yfd \ni ((x, y), A) \mapsto \iint \1_A(x', y') \ed(x', y') \, \refm(\rmd y') \, \hk(x, \rmd x')
$$
and initial distribution 
$$
    \Xfd \varotimes \Yfd \ni A \mapsto \iint \1_A(x, y) \ed(x, y) \, \refm(\rmd y) \, \init(\rmd x).  
$$ 
Here the marginal process $\{ X_n \}_{n \in \nset}$, referred to as the \emph{state process}, is only partially observed through the \emph{observation process} $\{ Y_n \}_{n \in \nset}$. For the model  $\{ (X_n, Y_n) \}_{n \in \nset}$ defined in this way, 
\begin{itemize}
    \item[(i)] the state process is a Markov chain with transition kernel $\mk$ and initial distribution $\init$, 
    \item[(ii)] the observations are, given the states, conditionally independent and such that the marginal conditional distribution of each $Y_n$ depends on $X_n$ only and has density $\pot(X_n, \cdot)$
\end{itemize}
(we refer to \cite[Section~2.2]{cappe:moulines:ryden:2005} for details). When operating on a well-specified state-space model, a key ingredient is typically the computation of the flow of \emph{predictor distributions}, where the predictor $\pred[\chunk{y}{0}{n - 1}]$ at time $n \in \nset$ is defined as the conditional distribution of the state $X_n$� given the record $\chunk{y}{0}{n - 1} \in \Ysp^n$ of realised historical observations up to time $n - 1$. Using Bayes' formula (see, e.g., \cite[Section~3.2.2]{cappe:moulines:ryden:2005} for details), it is straightforwardly shown that the predictor flow satisfies a perturbed Feynman-Kac recursion \eqref{eq:pred:rec} with $(\Xsp, \Xfd)$, $\hk$, and $\init$ given above, the observations $\{ Y_n \}_{n \in \nset}$ playing the role of perturbations (i.e., $\Zsp \gets \Ysp$ and $\Zfd \gets \Yfd$), and the local likelihood functions $\{�\ed(\cdot, y) : y \in \Ysp \}$ playing the role of potential functions $\{�\pot[y] : y \in \Ysp\}$. We will return to this framework in Section~\ref{sec:numerical:study}. 
\end{example}

\subsection{Sequential Monte Carlo methods}
SMC methods approximate online the Feynman-Kac flow generated by \eqref{eq:pred:rec} and a given sequence $\{ z_n \}_{n \in \nset}$ of perturbations by propagating recursively a random sample  $\{ \epart{n}{i} \}_{i = 1}^\N$ of $\Xsp$-valued \emph{particles}. More specifically, given a particle sample $\{ \epart{n}{i} \}_{i = 1}^\N$ \emph{targeting} $\pred[\chunk{z}{0}{n - 1}]$ in the sense that for all $h \in \bmf{\Xfd}$, $\predpart[\chunk{z}{0}{n - 1}] h \backsimeq \pred[\chunk{z}{0}{n - 1}] h$ as $\N$ tends to infinity, where  
$$
    \predpart[\chunk{z}{0}{n - 1}]: \Xfd \ni A \mapsto \frac{1}{\N} \sum_{i = 1}^\N \1_A(\epart{n}{i})
$$
denotes the empirical measure associated with the particles, an updated particle sample $\{ \epart{n + 1}{i} \}_{i = 1}^\N$ approximating $\pred[\chunk{z}{0}{n}]$ is, as the perturbation $z_n$ becomes accessible, formed by Algorithm~\ref{alg:SMC}. 

\bigskip
\begin{algorithm}[H] \label{alg:SMC}
    \KwData{$\{ \epart{n}{i} \}_{i = 1}^\N$, $z_n$}
    \KwResult{$\{ \epart{n + 1}{i} \}_{i = 1}^\N$}
    set $\wgtsum{n} \gets 0$\;
    \For{$i = 1 \to \N$}{
        set $\wgt{n}{i} \gets \pot[z_n](\epart{n}{i})$\;
        set $\wgtsum{n} \gets \wgtsum{n} + \wgt{n}{i}$\;
    }
    \For {$i = 1 \to \N$}{
        draw $\ind{n + 1}{i} \sim \cat(\{ \wgt{n}{\ell} / \wgtsum{n} \}_{\ell = 1}^N)$\;
        draw $\epart{n + 1}{i} \sim \mk(\epart{n}{\ind{n + 1}{i}}, \cdot)$\;
    }
    \caption{SMC particle update}
\end{algorithm}
\bigskip
(In the algorithm above, $\cat( \{ \wgt{n}{\ell} / \wgtsum{n} \}_{\ell = 1}^N)$ denotes the categorical distribution induced by the normalised particle weights $\{ \wgt{n}{\ell} / \wgtsum{n} \}_{\ell = 1}^N$.) Algorithm~\ref{alg:SMC} is initialised at time $n = 0$ by drawing $\{ \epart{0}{i} \}_{i = 1}^\N \sim \init^{\varotimes \N}$. For all $n \in \nset$ and all $h \in \bmf{\Xfd}$, the convergence, as $\N$ tends to infinity, of $\predpart[\chunk{z}{0}{n - 1}] h$ to $\pred[\chunk{z}{0}{n - 1}] h$ �can be established in several probabilistic senses. In particular, the first CLT for SMC methods was provided by \cite{delmoral:guionnet:1999}, establishing that 
\begin{equation} \label{eq:CLT}
\sqrt{\N} \left( \predpart[\chunk{z}{0}{n - 1}] h - \pred[\chunk{z}{0}{n - 1}] h \right) \dlim \variance{\chunk{z}{0}{n - 1}}(h) Z,  
\end{equation}
where $Z$ is standard normally distributed and the asymptotic variance is given by $\variance{\chunk{z}{0}{n - 1}} \eqdef \variance{\chunk{z}{0}{n - 1}}[0]$ with 
\begin{equation} \label{eq:def:as:var}
\variance[2]{\chunk{z}{0}{n - 1}}[\ell] : \bmf{\Xfd} \ni h \mapsto \sum_{m = \ell}^n \frac{\pred[\chunk{z}{0}{m - 1}] \{ \uk[\chunk{z}{m}{n - 1}](h - \pred[\chunk{z}{0}{n - 1}] h) \}^2 }{(\pred[\chunk{z}{0}{m - 1}] \uk[\chunk{z}{m}{n - 1}] \1_{\Xsp})^2}
\end{equation}
(see also \cite{chopin:2004,kuensch:2005,douc:moulines:2008} for similar results). The fact that we in \eqref{eq:def:as:var} define a truncated version of the variance with only $n - \ell + 1$ terms will be clear later on. In the coming section we propose a lag-based, numerically stable estimator of the sequence $\{ \variance[2]{\chunk{z}{0}{n - 1}} \}_{n \in \nset}$ of asymptotic variances. The estimator approximates $\{�\variance[2]{\chunk{z}{0}{n - 1}} \}_{n \in \nset}$ online, as $n$ increases, under constant computational complexity and memory requirements. Importantly, the estimator is obtained as a by-product of the particle filter output and does not require additional simulations. The numerical stability is obtained at the price of a small bias, which may be controlled under weak assumptions on the mixing properties of the model. 


% The algorithm
\section{A lag-based variance estimator}
\label{sec:estimator}
\subsection{The variance estimator proposed in \cite{chan:lai:2013}}
\label{sec:the:Lai:estimator}
Since Algorithm~\ref{alg:SMC} resamples the particles at each time step, the particle cloud may be associated with a tree describing the genealogical lineages of the particles. The estimators proposed in \cite{chan:lai:2013} and \cite{lee:whiteley:2016} are based on the particles' \emph{Eve indices} $\{ \eve{n}{i} \}_{i = 1}^\N$ (the terminology is adopted from \cite{lee:whiteley:2016}), which are, for all $n \in \nset$, defined as the indices of the time-zero ancestors of the particles $\{ \epart{n}{i} \}_{i = 1}^\N$. More specifically, the Eve indices may, for all $i \in \intvect{1}{\N}$, be computed recursively in Algorithm~\ref{alg:SMC} (just after Line~6) by letting 
$$
    \eve{n}{i} \eqdef
    \begin{cases}
        i & \mbox{for } n = 0, \\
        \eve{n - 1}{\ind{n}{i}} & \mbox{for } n \in \nsetpos. 
    \end{cases}
$$
Using the Eve indices, H.~P. Chan and T.~L.~Lai proposed, in \cite{chan:lai:2013}, for all $n \in \nset$, $\varest[2]{\chunk{z}{0}{n - 1}}(h)$, with
\begin{equation} \label{eq:Lai:estimator}
    \varest[2]{\chunk{z}{0}{n - 1}} : \bmf{\Xfd} \ni h \mapsto \frac{1}{\N} \sum_{i = 1}^\N \left( \sum_{j : \eve{n}{j} = i} \left\{ h(\epart{n}{j}) - \predpart[\chunk{z}{0}{n - 1}] h \right\} \right)^2,
\end{equation}
as an estimator of $\variance[2]{\chunk{z}{0}{n - 1}}(h)$ for all $h \in \bmf{\Xfd}$. As mentioned in the introduction, we will refer to this estimator as the CLE. (More precisely, in \cite{chan:lai:2013}, focus was set on the \emph{updated} distribution flows discussed in Section~\ref{sec:updated:measures} below; the adaptation is however straightforward.) In \cite{lee:whiteley:2016}, a generalisation of the CLE, allowing the particle population size $\N$ to vary between SMC iterations, is presented. As the main result of \cite{chan:lai:2013}, the consistency, as $\N$ tends to infinity, of the CLE is established; see also \cite[Theorem~1 and Corollary~1]{lee:whiteley:2016} for a generalisation. 

The CLE is indeed remarkable, as it allows the variance to be estimated online in a single run of the particle filter with no further simulation. Nevertheless, as explained in the introduction, the previous estimator has a serious flaw which is related to the well-known \emph{particle path depletion phenomenon} of SMC algorithms. More specifically, resampling the particles systematically at each time  leads without exception to a random time point before which all the genealogical traces coincide; we refer again to \cite{jacob:murray:rubenthaler:2015}, which provides a time uniform $\ordo(\N \log \N)$ bound on the expected number of generations back in time to this most recent common ancestor. Thus, as $n$ increases, the sets $\{ j \in \intvect{1}{\N} : \eve{n}{j} = i \}$ will eventually be empty for all indices $i \in \intvect{1}{\N}$ except one, say, $i_0$, for which $\{ j \in \intvect{1}{\N} : \eve{n}{j} = i_0 \} = \intvect{1}{\N}$. As a consequence, eventually, $\varest[2]{\chunk{z}{0}{n - 1}}(h) = 0$ for all $h \in \bmf{\Xfd}$, which makes the estimator impractical. In the next section, we propose a simple modification of the CLE that stabilises numerically the same at the cost of a negligible, controllable bias. 

\subsection{Our estimator}
\label{sec:our:estimator}

The estimator that we propose is based on the simple idea of stabilising numerically the CLE by tracing, backwards in time, only a few generations of the particle genealogy, rather than tracing the history all the way back to the time-zero ancestors. In our approach, the Eve indices will be replaced by
\emph{Enoch indices}\footnote{Two figures named Enoch appear in the 2nd as well as the 6th generations of the Genealogies of Genesis, as the son of Cain and the son-son-son-son-son of Seth, respectively.} defined, for all $i \in \intvect{1}{\N}$ and $m \in \nset$, recursively as  
\begin{equation} \label{eq:def:Enoch}
\enoch{m}{n}{i} \eqdef 
\begin{cases}
i & \mbox{for } n = m, \\
\enoch{m}{n - 1}{\ind{n}{i}}  & \mbox{for } n > m.  
\end{cases}
\end{equation}
In other words, for all $n \in \nset$, $m \in \intvect{1}{n}$, and $i \in \intvect{1}{\N}$, $\epart{m}{\enoch{m}{n}{i}}$ is the ancestor of $\epart{n}{i}$ at time $m$. Now, let $\lag \in \nset$ be some fixed number, referred to as the lag, and define $\lagtime{n}{\lag} \eqdef (n - \lambda) \vee 0$; then, we propose $\varest[2]{\chunk{z}{0}{n - 1}}[\lambda](h)$, with  
\begin{equation} \label{eq:estimator}
\varest[2]{\chunk{z}{0}{n - 1}}[\lambda] : \bmf{\Xfd} \ni h \mapsto \frac{1}{\N} \sum_{i = 1}^\N \left( \sum_{j : \enoch{\lagtime{n}{\lambda}}{n}{j} = i} \left\{ h(\epart{n}{j}) - \predpart[\chunk{z}{0}{n - 1}] h \right\} \right)^2,
\end{equation}
as an estimator of the variance $\variance[2]{\chunk{z}{0}{n - 1}}(h)$ for all $n \in \nset$, $\chunk{z}{0}{n - 1} \in \Zsp^n$, and $h \in \bmf{\Xfd}$. 
Online computation of the Enoch indices $\{ \enoch{\lagtime{n}{\lambda}}{n}{i} \}_{i =�1}^\N$ requires the propagation of a window $\{�\enoch{\lagtime{n}{\lag}}{n}{i}, \ldots,  \enoch{n}{n}{i} \}_{i = 1}^\N$ of indices; see Algorithm~\ref{alg:fixed-lag:SMC} for a pseudo-code.  As the length of the window is bounded by $\lag + 1$, the memory demand of the estimator is $\ordo(\lag \N)$ independently of $n$. Moreover, since genealogical tracing has a linear complexity in $\N$, the total complexity of the estimator is $\ordo(\lag \N)$, again independently of $n$. 

\bigskip
\begin{algorithm}[H] \label{alg:fixed-lag:SMC}
    \KwData{$\{ \epart{n}{i} \}_{i = 1}^\N$, $\{�\enoch{\lagtime{n}{\lag}}{n}{i}, \ldots,  \enoch{n}{n}{i} \}_{i = 1}^\N$, $z_n$}
    \KwResult{$\{ \epart{n + 1}{i} \}_{i = 1}^\N$, $\{�\enoch{\lagtime{(n + 1)}{\lag}}{n + 1}{i}, \ldots,  \enoch{n + 1}{n + 1}{i} \}_{i = 1}^\N$}
    set $\wgtsum{n} \gets 0$\;
    \For{$i = 1 \to \N$}{
        set $\wgt{n}{i} \gets \pot[z_n](\epart{n}{i})$\;
        set $\wgtsum{n} \gets \wgtsum{n} + \wgt{n}{i}$\;
    }
    \For {$i = 1 \to \N$}{
        draw $\ind{n + 1}{i} \sim \cat(\{ \wgt{n}{\ell} / \wgtsum{n} \}_{\ell = 1}^N)$\;
        draw $\epart{n + 1}{i} \sim \mk(\epart{n}{\ind{n + 1}{i}}, \cdot)$\;
        \For{$m = \lagtime{(n + 1)}{\lambda} \to n$}{
            set $\enoch{m}{n + 1}{i} \gets \enoch{m}{n}{\ind{n + 1}{i}}$\;
        }
        set $\enoch{n + 1}{n + 1}{i} \gets i$\;
    }
    \caption{SMC particle and Enoch-index update}
\end{algorithm}
\bigskip

For $n = 0$, Algorithm~\ref{alg:fixed-lag:SMC} is initialised by drawing $\{ \epart{0}{i} \}_{i = 1}^\N \sim \init^{\varotimes \N}$ and setting $\enoch{0}{0}{i} \gets i$ for all $i \in \intvect{1}{\N}$. At the end of the algorithm, after the second \textbf{for}-loop, an estimate 
$$
    \varest[2]{\chunk{z}{0}{n}}[\lambda](h) = \frac{1}{\N} \sum_{i = 1}^\N \left( \sum_{j : \enoch{\lagtime{(n + 1)}{\lambda}}{n + 1}{j} = i} \{ h(\epart{n + 1}{j}) - \predpart[\chunk{z}{0}{n}] h \} \right)^2
$$
of $\variance[2]{\chunk{z}{0}{n}}[\lambda](h)$ may be formed for all $h \in \bmf{\Xfd}$. 

\subsection{Variance estimators for flows of updated distributions}
\label{sec:updated:measures}

Some applications involve approximation of the \emph{updated} measures 
\begin{equation} \label{eq:def:filt}
    \filt[\chunk{z}{k}{m}] : \Xfd \ni A 
    \mapsto \frac{\init \uk[\chunk{z}{k}{m - 1}] (\pot[z_m] \1_A)}
    {\init \uk[\chunk{z}{k}{m - 1}] (\pot[z_m] \1_\Xsp)},
\end{equation}
for $\chunk{z}{k}{m} \in \Zsp^{m - k + 1}$, rather than the measures defined by \eqref{eq:def:pred}.   
\begin{example}[partially dominated state-space models, revisited]
In the case of the partially dominated state-space models discussed in Example~\ref{example:state:space:model}, the updated measures $\{ \filt[\chunk{y}{0}{n}] \}_{n \in \nset}$ defined through \eqref{eq:def:filt} are the \emph{filter distributions}; more precisely, in this context, for all $n \in \nset$, $\filt[\chunk{y}{0}{n}]$ is the conditional distribution of the state $X_n$ given the realised observations $\chunk{y}{0}{n} \in \Ysp^{n + 1}$ up to time $n$ (i.e., \emph{including} the last observation $y_n$). 
\end{example}
Since for all $h \in \bmf{\Xfd}$, by normalisation, 
$$
    \filt[\chunk{z}{k}{m}] h = \frac{\pred[\chunk{z}{k}{m - 1}](\pot[z_m] h)}{\pred[\chunk{z}{k}{m - 1}] \pot[z_m]},
$$
the flow $\{ \filt[\chunk{z}{0}{n}] \}_{n \in \nset}$ of updated distributions is naturally approximated by the flow of weighted empirical measures 
\begin{equation} \label{eq:def:particle:filter}
    \filtpart[\chunk{z}{0}{n}] : A \ni \Xfd \mapsto \frac{\predpart[\chunk{z}{k}{m - 1}](\pot[z_m] \1_A)}{\predpart[\chunk{z}{k}{m - 1}] \pot[z_m]} = \sum_{i = 1}^\N \frac{\wgt{n}{i}}{\wgtsum{n}} \1_A(\epart{n}{i}),
\end{equation}
for some given sequence $\{ z_n \}_{n \in \nset}$ of perturbations, where the weights $\{ \wgt{n}{i} \}_{i = 1}^\N$ and the weight sum $\wgtsum{n}$ are computed in Algorithm~\ref{alg:SMC}. By the normality \eqref{eq:CLT} and the consistency \eqref{eq:particle:filter:consistency} one obtains, using Slutsky's theorem, for all $\chunk{z}{0}{n} \in \Zsp^{n + 1}$, the central limit theorem 
\begin{equation} \label{eq:CLT:updated:measures}
    \sqrt{\N} \left( \filt[\chunk{z}{0}{n}] h - \filt[\chunk{z}{0}{n}] h \right) \dlim \filtvariance{\chunk{z}{0}{n}}(h) Z,  
\end{equation}
as $\N$ tends to infinity, where $Z$ is standard normally distributed and the asymptotic variance is given by $\filtvariance[2]{\chunk{z}{0}{n}}(h) = \filtvariance[2]{\chunk{z}{0}{n}}[0](h)$ with 
\begin{equation} \label{eq:def:as:var:updated:measures}
    \filtvariance[2]{\chunk{z}{0}{n}}[\ell] : \bmf{\Xfd} \ni h \mapsto \frac{\variance[2]{\chunk{z}{0}{n - 1}}[\ell](\pot[z_n] \{ h - \filt[\chunk{z}{0}{n}] h \})}{(\pred[\chunk{z}{0}{n - 1}] \pot[z_n])^2}
\end{equation}
(where $\variance{\chunk{z}{0}{n - 1}}[\ell]$ is defined in \eqref{eq:def:as:var} for the original Feynman-Kac particle model). In the case $\ell = 0$, the expression \eqref{eq:def:as:var:updated:measures} is found also in \cite[Eqn.~(17)]{douc:moulines:olsson:2014}. In the light of \eqref{eq:def:as:var:updated:measures}, casting our fixed-lag approach into the framework of updated Feynman-Kac models yields the estimator 
\begin{multline} \label{eq:def:var:est:updated:measures}
    \varestfilt[2]{\chunk{z}{0}{n}}[\lambda] : \bmf{\Xfd} \ni h \mapsto \frac{\varest[2]{\chunk{z}{0}{n - 1}}[\lambda](\pot[z_n] \{h - \filtpart[\chunk{z}{0}{n}] h\})}{(\predpart[\chunk{z}{0}{n - 1}] \pot[z_n])^2} \\ 
    = \N \sum_{i = 1}^\N \left( \sum_{j : \enoch{\lagtime{n}{\lambda}}{n}{j} = i} \frac{\wgt{n}{i}}{\wgtsum{n}} \left\{ h(\epart{n}{j}) - \filtpart[\chunk{z}{0}{n}] h \right\} \right)^2
\end{multline}
for some suitable lag $\lag \in \nset$ (where the equality stems from the fact that $\predpart[\chunk{z}{0}{n - 1}] \{�\pot[z_n](h -  \filtpart[\chunk{z}{0}{n}] h) \} = 0$).  







% Theoretical results
\section{Theoretical results}
\label{sec:theoretical:results}
\input{theory}

% Numerical study 
\section{Application to state-space models}
\label{sec:numerical:study}
\input{numerics} 

% Conclusion 
\section{Conclusion}
\label{sec:conclusion}
We presented for the first time a polynomial algorithm to extract the core structural connectivity network of a population combining a graph-theoretical approach with statistic relevance of the connections, observing the recent evidence of the structural network topology.

Our results show that our algorithm outperforms, in the prediction experiment, the most used technique~\cite{Gong2009} as well as latest approaches~\cite{Wassermann2016}. In Table~\ref{table:number_of_features} we can see that our algorithm preserves, in average, more connections correlated with the handedness of the subjects. We can also see that despite being less stable than Wassermann et al.'s it is stabler than Gong et al.'s. Finally, Fig.~\ref{fig:prediction_performance} shows that, in the handedness prediction experiment, our method outperforms  Gong et al.'s and Wassermann et al's: the number of cases with lower AIC and MSE is larger in our case. Hence, our CSCN is better as linear model relating connectivity with handedness in terms of model fitting and prediction.


In terms of theoretical contributions, we formalized the problem, proved its difficulty and gave a novel algorithm for dealing with it. We then validated our approach by showing its power as feature selector for getting connections related to handedness with 300 real subjects' data. The experiment shows our method performs better than the currently available. Moreover, our method avoids the double dipping problem by not choosing the feature set with hypothesis testing.

\appendix

\section{Proofs}
\label{sec:proofs}
\subsection{Proof of Proposition~\ref{prop:consistency:fixed:lag}}
\label{sec:proof:consistency:fixed:lag}

The proof of Proposition~\ref{prop:consistency:fixed:lag} relies on the machinery developed in \cite{lee:whiteley:2016}, from which we adopt the following definitions. Throughout this section, 
let $n \in \nset$ and $\chunk{z}{0}{n - 1} \in \Zsp^n$ be picked arbitrarily.  
\begin{itemize}
\item Denote by $\Binsp{n} \eqdef \{0, 1\}^{n + 1}$ the space of binary strings of length $n + 1$. The \emph{zero string} of length $n + 1$ is denoted by $\zerostr{n}$ and for $m \in \intvect{0}{n}$, $\unitstr{m}{n}$ denotes a \emph{unit string} of length $n + 1$ with $1$ on position $m$ (with positions indexed from $0$) and zeros everywhere else.  
\item For a given string $\chunk{b}{0}{n} \in \Binsp{n}$, a Markov chain $\{�(X_m, X_m') \}_{m = 0}^n$ on $(\Xsp^2, \Xfd^{\varotimes 2})$ is defined as follows. If $b_0 = 0$, then $(X_0, X_0') \sim \init^{\varotimes 2}$; otherwise, if $b_0 = 1$, $X_0' = X_0 \sim \init$ (the initial distribution). After this, if $b_{m + 1} = 0$, $X_{m + 1} \sim \mk(X_m, \cdot)$� and $X_{m + 1}' \sim \mk(X_m', \cdot)$ conditionally independently; otherwise, if $b_{m + 1} = 1$, $X_{m + 1}' = X_{m + 1} \sim \mk(X_m, \cdot)$. 
\item With $\E_{\chunk{b}{0}{n}}$ denoting the expectation under the law of $\{�(X_m, X_m') \}_{m = 0}^n$, we define, for all $\chunk{b}{0}{n} \in \Binsp{n}$, the measures 
$$
\mu_{\chunk{b}{0}{n}} \langle \chunk{z}{0}{n - 1} \rangle : \Xfd^{\varotimes 2} \ni A \mapsto 
\E_{\chunk{b}{0}{n}} \left[ \1_A(X_n, X_n') \prod_{m = 0}^{n - 1} \pot[z_m](X_m) \pot[z_m](X_m') \right]. 
$$ 
Note that for all $h \in \bmf{\Xfd}$ it holds that $\mu_{\zerostr{n}} \langle \chunk{z}{0}{n - 1} \rangle h^{\varotimes 2} = (\init \uk[ \chunk{z}{0}{n - 1}] h)^2$ and $\mu_{\unitstr{m}{n}} \langle \chunk{z}{0}{n - 1} \rangle h^{\varotimes 2} = \init \uk[\chunk{z}{0}{m - 1}] \1_\Xsp \times \init \uk[\chunk{z}{0}{m - 1}](\uk[\chunk{z}{m}{n - 1}] h)^2$, and defining 
$$
\term[\chunk{z}{0}{n - 1}]{m}{n} : \bmf{\Xfd} \ni h \mapsto \frac{\mu_{\unitstr{m}{n}} \langle \chunk{z}{0}{n - 1} \rangle h^{\varotimes 2} - \mu_{\zerostr{n}} \langle \chunk{z}{0}{n - 1} \rangle h^{\varotimes 2}}{(\init \uk[\chunk{z}{0}{n - 1}] \1_\Xsp)^2}
$$
yields for all $h \in \bmf{\Xfd}$, 
$$
\term[\chunk{z}{0}{n - 1}]{m}{n}(h) = \frac{\pred[\chunk{z}{0}{m - 1}] (\uk[\chunk{z}{m}{n - 1}] h)^2}{(\pred[\chunk{z}{0}{m - 1}] \uk[\chunk{z}{m}{n - 1}] \1_{\Xsp})^2} - (\pred[\chunk{z}{0}{n - 1}] h)^2 
$$
and, consequently, for all $\ell \in \intvect{0}{n}$, 
\begin{equation} \label{eq:variance:alt:expression}
\variance[2]{\chunk{z}{0}{n - 1}}[\ell](h) = \sum_{m = \ell}^n \term[\chunk{z}{0}{n - 1}]{m}{n} (h - \pred[\chunk{z}{0}{n - 1}] h). 
\end{equation}
\item For all $\N \in \nsetpos$, let �$\partfd{n} \eqdef \sigma( \{ \epart{0}{i} \}_{i = 1}^\N, \{�\epart{m}{i}, \ind{m}{i} \}_{i = 1}^\N ; m \in \intvect{1}{n} )$ be the $\sigma$-field generated by the output of Algorithm~\ref{alg:SMC} during the first $n$ iterations. Conditionally on $\partfd{n}$, a genealogical trace $\chunk{\gen[1]}{0}{n}$ is formed backwards in time by, first, drawing $\gen[1]_n$ uniformly over $\intvect{1}{\N}$ and, second, setting $\gen[1]_m = \ind{m + 1}{\gen[1]_{m + 1}}$ for all $m \in \intvect{0}{n - 1}$. In addition, a parallel trace $\chunk{\gen[2]}{0}{n}$ is formed by, first, drawing $\gen[2]_n$ uniformly over $\intvect{1}{\N}$ and, second, letting $\gen[2]_m = \ind{m + 1}{\gen[2]_{m + 1}}$ if $\gen[2]_{m + 1} \neq \gen[1]_{m + 1}$ or $\gen[2]_m \sim \cat(\{ \wgt{m}{i} / \wgtsum{m} \}_{i = 1}^\N)$ otherwise. 
\end{itemize}

The proof of the following lemma follows closely that of \cite[Lemma~4]{lee:whiteley:2016} and is hence omitted. Define for all $\ell \in \intvect{0}{n}$ and $\chunk{b}{0}{\ell} \in \Binsp{\ell}$, 
$$
\binset{\ell}(\chunk{b}{0}{\ell}) \eqdef \left \{�(\chunk{k}{0}{\ell}, \chunk{k'}{0}{\ell}) \in \intvect{1}{\N}^{2(\ell + 1)} : \mbox{for all } \ell' \in \intvect{0}{\ell}, \ k_{\ell'} = k'_{\ell'} \Leftrightarrow b_{\ell'} = 1 \right \}. 
$$
\begin{lemma} \label{lemma:equiv:sets}
For all $\N \in \nsetpos$ and $m \in \intvect{0}{n}$, 
$$
\left \{ \enoch{m}{n}{\gen[1]_n} \neq \enoch{m}{n}{\gen[2]_n} \right \} = \left \{ (\chunk{\gen[1]}{m}{n}, \chunk{\gen[2]}{m}{n}) \in \binset{n - m}(0_{n - m}) \right \}. 
$$ 
\end{lemma}

In addition, define, for all $\N \in \nsetpos$, the measures
\begin{equation} \label{eq:def:part:gamma}
\unpredpart[\chunk{z}{0}{n - 1}] : \Xfd \ni A \mapsto \frac{1}{\N^{n + 1}} \left( \prod_{m = 0}^{n - 1} \wgtsum{m} \right) \sum_{i = 1}^\N \1_A(\epart{n}{i})
\end{equation}
and 
\begin{multline} \label{eq:mu:meas}
\mumeas{\N, \chunk{b}{0}{n}}{\chunk{z}{0}{n - 1}} :  \Xfd^{\varotimes 2} \ni A \mapsto 
\N^{\#_1(\chunk{b}{0}{n})} \left( \frac{\N}{\N - 1} \right)^{\#_0(\chunk{b}{0}{n})}
\left( \unpredpart[\chunk{z}{0}{n - 1}] \1_\Xsp \right)^2 \\ 
\times \E \left[ \1_A \left( \epart{n}{\gen[1]_n}, \epart{n}{\gen[2]_n} \right) \1 \left \{�(\chunk{\gen[1]}{0}{n}, \chunk{\gen[2]}{0}{n}) \in \binset{n}(\chunk{b}{0}{n}) \right \} \mid \partfd{n} \right],  
\end{multline}
where $\#_1(\chunk{b}{0}{n}) \eqdef \sum_{m = 0}^n b_m$ and $\#_0(\chunk{b}{0}{n}) \eqdef n + 1 - \#_1(\chunk{b}{0}{n})$ denote the numbers of ones and zeros in $\chunk{b}{0}{n}$, respectively. Note that \eqref{eq:def:part:gamma} implies that for all $h \in \bmf{\Xfd}$, $\predpart[\chunk{z}{0}{n - 1}] h = \unpredpart[\chunk{z}{0}{n - 1}] h / \unpredpart[\chunk{z}{0}{n - 1}] \1_\Xsp$. 

The following lemma, where first part is established in \cite[Theorem~2]{lee:whiteley:2016} and the last part is a standard result (see, e.g, \cite{douc:moulines:2008} for results on the weak consistency of SMC), will be instrumental. 

\begin{lemma} \label{lemma:mu:convergence}
For all $\chunk{b}{0}{n} \in \Binsp{n}$ and $h \in \bmf{\Xfd^{\varotimes 2}}$, as $\N \rightarrow \infty$, 
$$
\mumeas{\N, \chunk{b}{0}{n}}{\chunk{z}{0}{n - 1}} h \plim \mumeas{\chunk{b}{0}{n}}{\chunk{z}{0}{n - 1}} h. 
$$
In addition, for all $h \in \bmf{\Xfd}$, 
$$
\unpredpart[\chunk{z}{0}{n - 1}] h \plim \init \uk[\chunk{z}{0}{n - 1}] h. 
$$
\end{lemma}

\begin{proof}[Proof of Proposition~\ref{prop:consistency:fixed:lag}]
Fix $\ell \in \intvect{0}{n}$ and define for all $\N \in \nsetpos$,
\begin{multline*}
\varphi_{\N, \ell} \langle \chunk{z}{0}{n - 1} \rangle : \Xfd^{\varotimes 2} \ni A \mapsto  \left( \unpredpart[\chunk{z}{0}{n - 1}] \1_\Xsp \right)^2 \\�
\times \E \left[ \1_A \left( \epart{n}{\gen[1]_n}, \epart{n}{\gen[2]_n} \right) \1 \left \{�(\chunk{\gen[1]}{\ell}{n}, \chunk{\gen[2]}{\ell}{n}) \in \binset{n - \ell}(0_{n - \ell}) \right \} \mid \partfd{n} \right]. 
\end{multline*}
First, note that by Lemma~\ref{lemma:equiv:sets}, since $\gen[1]_n$ and $\gen[2]_n$ are conditionally independent and uniformly distributed over $\intvect{1}{\N}$, for all $h \in \bmf{\Xfd}$,  
\begin{equation} \label{eq:estimator:alt:expression}
\begin{split} 
\frac{1}{\N^2} \sum_{i = 1}^\N \left( \sum_{j : \enoch{\ell}{n}{j} = i} h(\epart{n}{j}) \right)^2 
&= \frac{1}{\N^2} \sum_{(i, j) : \enoch{\ell}{n}{i} = \enoch{\ell}{n}{j}} h(\epart{n}{i}) h(\epart{n}{j}) \\
&= (\predpart[\chunk{z}{0}{n - 1}] h)^2 - \frac{1}{\N^2} \sum_{(i, j) : \enoch{\ell}{n}{i} \neq \enoch{\ell}{n}{j}} h(\epart{n}{i}) h(\epart{n}{j}) \\
&= (\predpart[\chunk{z}{0}{n - 1}] h)^2 - \frac{\varphi_{\N, \ell} \langle \chunk{z}{0}{n - 1} \rangle h^{\varotimes 2}}{(\unpredpart[\chunk{z}{0}{n - 1}] \1_\Xsp)^2}.  
\end{split}
\end{equation}
It is hence enough to prove that for all $\N \in \nsetpos$ and $h \in \bmf{\Xfd}$, 
\begin{multline} \label{eq:sufficient:condition}
\N \left \{ (\unpredpart[\chunk{z}{0}{n - 1}] h)^2 - \varphi_{\N, \ell} \langle \chunk{z}{0}{n - 1} \rangle h^{\varotimes 2} \right\} \\�
= %\frac{1}{\N} 
\sum_{m = \ell}^n \left( \mumeas{\N, \unitstr{m}{n}}{\chunk{z}{0}{n - 1}} h^{\varotimes 2} - \mumeas{\N, \zerostr{n}}{\chunk{z}{0}{n - 1}} h^{\varotimes 2} \right) + (n - \ell + 1) (\unpredpart[\chunk{z}{0}{n - 1}] h)^2 \\�
+ \| h \|_\infty^2 \ordo(\N^{-1}),
\end{multline}
where the $\ordo(\N^{-2})$ term does not depend on $h$; indeed, along the lines of the proof of \cite[Theorem~1]{lee:whiteley:2016}, Lemma~\ref{lemma:mu:convergence} implies that for all $\chunk{b}{0}{n} \in \Binsp{n}$, as $\N \rightarrow \infty$, 
$$
\mumeas{\N, \chunk{b}{0}{n}}{\chunk{z}{0}{n - 1}} \{�h - \predpart[\chunk{z}{0}{n - 1}] h \}^{\varotimes 2} \plim \mumeas{\chunk{b}{0}{n}}{\chunk{z}{0}{n - 1}} \{�h - \pred[\chunk{z}{0}{n - 1}] h \}^{\varotimes 2}, 
$$
and \eqref{eq:sufficient:condition} hence yields, with $\ell = \lagtime{n}{\lag}$ �and when combined with \eqref{eq:estimator:alt:expression} and \eqref{eq:variance:alt:expression}, again as $\N \rightarrow \infty$, 
\begin{multline} \label{eq:critical:identity}
\varest[2]{\chunk{z}{0}{n - 1}}[\lag](h) \\
= \sum_{m = \lagtime{n}{\lag}}^n \frac{\mumeas{\N, \unitstr{m}{n}}{\chunk{z}{0}{n - 1}} \{�h - \predpart[\chunk{z}{0}{n - 1}] h \}^{\varotimes 2} - \mumeas{\N, 0_n}{\chunk{z}{0}{n - 1}} \{�h - \predpart[\chunk{z}{0}{n - 1}] h \}^{\varotimes 2}}{(\unpredpart[\chunk{z}{0}{n - 1}] \1_\Xsp)^2} \\
+ \| h \|_\infty^2 \ordo(\N^{-1}) \plim \variance[2]{\chunk{z}{0}{n - 1}}[\lagtime{n}{\lag}](h). 
\end{multline}
In order to establish \eqref{eq:sufficient:condition}, write, using that $\gen[1]_n$ and $\gen[2]_n$ are, given $ \partfd{n}$, conditionally independent and uniformly distributed over $\intvect{1}{\N}$,   
\begin{align} \label{eq:estimator:difference:form}
\lefteqn{(\unpredpart[\chunk{z}{0}{n - 1}] h)^2 - \varphi_{\N, \ell} \langle \chunk{z}{0}{n - 1} \rangle h^{\varotimes 2}} \nonumber \\ 
&= (\unpredpart[\chunk{z}{0}{n - 1}] \1_\Xsp)^2 \sum_{\chunk{b}{0}{n} \in \Binsp{n}} \E \left[ h\big( \epart{n}{\gen[1]_n} \big) h \big( \epart{n}{\gen[2]_n} \big) \1 \left \{�(\chunk{\gen[1]}{0}{n}, \chunk{\gen[2]}{0}{n}) \in \binset{n}(\chunk{b}{0}{n}) \right \} \mid \partfd{n} \right] \nonumber \\ 
&- (\unpredpart[\chunk{z}{0}{n - 1}] \1_\Xsp)^2 \sum_{\chunk{b}{0}{n} \in \Binsp{n} : \chunk{b}{\ell}{n} = 0_{n - \ell}} \E \left[ h\big( \epart{n}{\gen[1]_n} \big) h \big( \epart{n}{\gen[2]_n} \big) \1 \left \{�(\chunk{\gen[1]}{0}{n}, \chunk{\gen[2]}{0}{n}) \in \binset{n}(\chunk{b}{0}{n}) \right \} \mid \partfd{n} \right]
\end{align}
and note that, by definition \eqref{eq:mu:meas}, 
\begin{align} \label{eq:estimator:first:term}
\lefteqn{(\unpredpart[\chunk{z}{0}{n - 1}] \1_\Xsp)^2 \sum_{\chunk{b}{0}{n} \in \Binsp{n}} \E \left[ h \big( \epart{n}{\gen[1]_n} \big) h \big( \epart{n}{\gen[2]_n} \big) \1 \left \{�(\chunk{\gen[1]}{0}{n}, \chunk{\gen[2]}{0}{n}) \in \binset{n}(\chunk{b}{0}{n}) \right \} \mid \partfd{n} \right]} \nonumber \\�
&= \frac{1}{\N} \sum_{m = 0}^n \mumeas{\N, \unitstr{m}{n}}{\chunk{z}{0}{n - 1}} h^{\varotimes 2} + \left( 1 - \frac{1}{\N} \right)^{n + 1} \mumeas{\N, 0_n}{\chunk{z}{0}{n - 1}} h^{\varotimes 2} + \| h \|_\infty^2 \ordo(\N^{-2}) \nonumber \\
&= \frac{1}{\N} \sum_{m = 0}^n \left( \mumeas{\N, \unitstr{m}{n}}{\chunk{z}{0}{n - 1}} h^{\varotimes 2}  - \mumeas{\N, \zerostr{n}}{\chunk{z}{0}{n - 1}} h^{\varotimes 2} \right) + \mumeas{\N, 0_n}{\chunk{z}{0}{n - 1}} h^{\varotimes 2} + \| h \|_\infty^2 \ordo(\N^{-2}). 
\end{align}
Similarly, 
\begin{multline} \label{eq:estimator:second:term}
\lefteqn{(\unpredpart[\chunk{z}{0}{n - 1}] \1_\Xsp)^2 \sum_{\chunk{b}{0}{n} \in \Binsp{n} : \chunk{b}{\ell}{n} = 0_{n - \ell}} \E \left[ h\big( \epart{n}{\gen[1]_n} \big) h \big( \epart{n}{\gen[2]_n} \big) \1 \left \{�(\chunk{\gen[1]}{0}{n}, \chunk{\gen[2]}{0}{n}) \in \binset{n}(\chunk{b}{0}{n}) \right \} \mid \partfd{n} \right]} \\
= \frac{1}{\N} \sum_{m = 0}^{\ell - 1} \left( \mumeas{\N, \unitstr{m}{n}}{\chunk{z}{0}{n - 1}} h^{\varotimes 2}  - \mumeas{\N, \zerostr{n}}{\chunk{z}{0}{n - 1}} h^{\varotimes 2} \right) + \mumeas{\N, 0_n}{\chunk{z}{0}{n - 1}} h^{\varotimes 2} \\�
- \frac{n - \ell + 1}{\N} \mumeas{\N, 0_n}{\chunk{z}{0}{n - 1}} h^{\varotimes 2} + \| h \|_\infty^2 \ordo(\N^{-2}), 
\end{multline}
and combining \eqref{eq:estimator:difference:form}, \eqref{eq:estimator:first:term}, \eqref{eq:estimator:second:term}, and the fact that $\mumeas{0_n}{\chunk{z}{0}{n - 1}} h^{\varotimes 2} = (\init \uk[\chunk{z}{0}{n - 1}] h)^2$ yields \eqref{eq:sufficient:condition}. This completes the proof. 
\end{proof}

\subsection{Proof of Theorem~\ref{thm:tightness:bias}}

In the proof of Theorem~\ref{thm:tightness:bias}, the asymptotic bias is bounded using the time-shift approach taken in \cite[Theorem~10]{douc:moulines:olsson:2014}. Even though the theoretical analysis provided in \cite{douc:moulines:olsson:2014} is cast into the framework of general state-space models, it never makes use of the fact that $\pot$ is a normalised transition density. As stressed in \cite[Remark~1]{douc:moulines:olsson:2014}, it is hence directly applicable to the framework of randomly perturbed Feynman-Kac models in Section~\ref{sec:Feynman:Kac:models}. 

\begin{proof}
Pick arbitrarily $n \in \nset$, $\lag \in \nset$, $h \in \bmf{\Xfd}$, and $\init \in \mdr(D, r)$. 
By defining, for all $m \in \nset$, $\ell \in \intvect{0}{m - 1}$, $\chunk{z}{\ell}{m - 1} \in \Zsp^{m - \ell}$, and measures $(\mu, \mu') \in \probmeas{\Xfd}^2$, 
\begin{multline} \label{eq:def-Delta}
\Delta_{\mu, \mu'} \langle \chunk{z}{\ell}{m - 1} \rangle : \bmf{\Xfd}^2 \ni (h, h') \mapsto 
\mu \uk[\chunk{z}{\ell}{m - 1}] h \times \mu' \uk[\chunk{z}{\ell}{m - 1}] h'
\\ - \mu \uk[\chunk{z}{\ell}{m - 1}] h' \times \mu' \uk[\chunk{z}{\ell}{m - 1}] h, 
\end{multline}
we may, using the identity 
$$
\pred[\chunk{\per}{0}{m - 1}] \uk[\chunk{\per}{m}{n - 1}] \1_{\Xsp} = \frac{\init\uk[\chunk{\per}{0}{n - 1}] \1_{\Xsp}}{\init\uk[\chunk{\per}{0}{m - 1}] \1_{\Xsp}} = \prod_{\ell = m}^{n - 1} \frac{\init\uk[\chunk{\per}{0}{\ell}] \1_{\Xsp}}{\init\uk[\chunk{\per}{0}{\ell - 1}] \1_{\Xsp}} = \prod_{\ell = m}^{n - 1} \pred[\chunk{\per}{0}{\ell - 1}]  \pot[\per_\ell],
$$
for all $m \in \intvect{0}{n}$,
write the asymptotic bias at time $n$ as
\begin{equation} \label{bias:alternative:form}
\bias{\chunk{\per}{0}{n - 1}}{\lag}(h) 
= \sum_{m = 0}^{\lagtime{n}{\lag} - 1} \int \pred[\chunk{\per}{0}{m - 1}](\rmd x) \left( \frac{\DDelta{\delta_x,\pred[\chunk{\per}{0}{m - 1}]}{\chunk{\per}{m}{n - 1}}{h, \1_{\Xsp}}}
{[\prod_{\ell = m}^{n - 1} \pred[\chunk{\per}{0}{\ell - 1}]  \pot[\per_\ell]]^2} \right)^2. 
\end{equation}
In addition, under the assumptions of the theorem, \cite[Proposition~1]{douc:moulines:2012} provides the existence of a function $\pi: \Zsp^\infty \to \rset$ such that for all initial distributions $\init \in \mdr(D, r)$,
$$
\lim_{m \to \infty} \pred[\chunk{\per}{-m}{- 1}] \pot[\per_0] = \limitfunc{\chunk{\per}{- \infty}{0}}, \quad \prob\mbox{-a.s.}
$$
Since the perturbations $\{ \per_n \}_{n \in \zset}$ are stationary, the distribution of $\bias{\chunk{\per}{0}{n - 1}}{\lag}(h)$ coincides with that of the time-shifted bias $\bias{\chunk{\per}{-n}{- 1}}{\lag}(h)$, and a key step in the present proof is to express, via \eqref{bias:alternative:form}, the latter as 
\begin{multline*}
\bias{\chunk{\per}{-n}{- 1}}{\lag}(h) = \\�
\sum_{m = 0}^{\lagtime{n}{\lag} - 1} \int \pred[\chunk{\per}{-n}{- n + \lagtime{n}{\lag} - m - 2}](\rmd x) \left( \frac{\DDelta{\delta_x,\pred[\chunk{\per}{-n}{- n + \lagtime{n}{\lag} - m - 2}]}{\chunk{\per}{- n + \lagtime{n}{\lag} - m - 1}{-1}}{h, \1_{\Xsp}}}{[\prod_{\ell = 1}^{n - \lagtime{n}{\lag} + m + 1} \pred[\chunk{\per}{-n}{- \ell - 1}]  \pot[\per_{- \ell}]]^2 } \right)^2.  
\end{multline*}
We hence obtain the bound 
\begin{equation} \label{eq:fundamental:bias:bound}
\bias{\chunk{\per}{-n}{-1}}{\lag}(h) \leq A_n \times B_n,
\end{equation}
where
\begin{align*} 
A_n &\eqdef \left(\sup_{(k, m) \in \zset^2 : \, -n \leq k \leq m} \prod_{\ell = k}^{m - 1} \frac{\limitfunc{\chunk{\per}{-\infty}{\ell}}}{\pred[\chunk{\per}{-n}{\ell - 1}] \pot[\per_\ell]} \right)^4, \\ 
B_n &\eqdef \sum_{m = 0}^{\lagtime{n}{\lag} - 1} \left( \frac{\sup_{x \in \Xsp} |\DDelta{\delta_x, \pred[\chunk{\per}{-n}{- n + \lagtime{n}{\lag} - m - 2}]}{\chunk{\per}{- n + \lagtime{n}{\lag} - m - 1}{-1}}{h, \1_\Xsp}|}{[\prod_{\ell = 1}^{n - \lagtime{n}{\lag} + m + 1} \limitfunc{\chunk{\per}{-\infty}{- \ell}}]^2} \right)^2.
\end{align*}
To bound uniformly the sequence $\{ B_n \}_{n \in \nset}$, decompose each term according to 
\begin{multline} \label{eq:termwise:decomposition:Bn}
\frac{\sup_{x \in \Xsp} |\DDelta{\delta_x, \pred[\chunk{\per}{-n}{- n + \lagtime{n}{\lag} - m - 2}]}{\chunk{\per}{- n + \lagtime{n}{\lag} - m - 1}{-1}}{h, \1_\Xsp}|}{[\prod_{\ell = 1}^{n - \lagtime{n}{\lag} + m + 1} \limitfunc{\chunk{\per}{-\infty}{- \ell}}]^2} \\
= \left( \frac{\| \uk[\chunk{\per}{- n + \lagtime{n}{\lag} - m - 1}{-1}] \1_\Xsp \|_\infty}{\prod_{\ell = 1}^{n - \lagtime{n}{\lag} + m + 1} \limitfunc{\chunk{\per}{-\infty}{- \ell}}} \right)^2 \\
\times \frac{\sup_{x \in \Xsp} |\DDelta{\delta_x, \pred[\chunk{\per}{-n}{- n + \lagtime{n}{\lag} - m - 2}]}{\chunk{\per}{- n + \lagtime{n}{\lag} - m - 1}{-1}}{h, \1_\Xsp}|}{\| \uk[\chunk{\per}{- n + \lagtime{n}{\lag} - m - 1}{-1}] \1_\Xsp \|_\infty^2}. 
\end{multline}
We consider separately the two factors of \eqref{eq:termwise:decomposition:Bn}. First,
\begin{equation} \label{eq:first:factor:Bn:deomposition}
\left( \frac{\| \uk[\chunk{\per}{- n + \lagtime{n}{\lag} - m - 1}{-1}] \1_\Xsp \|_\infty}{\prod_{\ell = 1}^{n - \lagtime{n}{\lag} + m + 1} \limitfunc{\chunk{\per}{-\infty}{- \ell}}} \right)^2 = \exp\{�(n - \lagtime{n}{\lag} + m + 1) \varepsilon_{n - \lagtime{n}{\lag} + m + 1} \}, 
\end{equation}
with
$$
\varepsilon_k \eqdef \frac{2}{k} \left( \ln \| \uk[\chunk{\per}{-k}{-1}] \1_\Xsp \|_\infty - \sum_{\ell = 1}^k \ln  \limitfunc{\chunk{\per}{-\infty}{- \ell}} \right)
$$
being independent of $\init$ for all $k \in \nsetpos$.  
By \cite[Lemma~17]{douc:moulines:olsson:2014}, $\varepsilon_k \to 0$, $\prob$-a.s., as $k \to \infty$, which implies that \eqref{eq:first:factor:Bn:deomposition} grows at most subgeometrically fast with $m$.  In addition, by \cite[Proposition~16(iii)]{douc:moulines:olsson:2014} there exists a constant $\rho \in (0, 1)$ and a $\prob$-a.s. finite random variable $D$ such that for all $n$ and $m$, all $h \in \bmf{\Xfd}$, and all $\init \in \probmeas{\Xfd}$, $\prob$-a.s,
$$
\frac{\sup_{x \in \Xsp} |\DDelta{\delta_x, \pred[\chunk{\per}{-n}{- n + \lagtime{n}{\lag} - m - 2}]}{\chunk{\per}{- n + \lagtime{n}{\lag} - m - 1}{-1}}{h, \1_\Xsp}|}{\| \uk[\chunk{\per}{- n + \lagtime{n}{\lag} - m - 1}{-1}] \1_\Xsp \|_\infty^2} \leq D \rho^{n - \lagtime{n}{\lag} + m + 1} \| h \|_\infty. 
$$
Thus, $\prob$-a.s,
$$
B_n \leq D^2 \| h \|_\infty^2 \sum_{m = 0}^{\lagtime{n}{\lag} - 1} \rho^{2(n - \lagtime{n}{\lag} + m + 1)} \exp\{�2 (n - \lagtime{n}{\lag} + m + 1) \varepsilon_{n - \lagtime{n}{\lag} + m + 1} \}.  
$$
If $n \leq \lag$, then $\lagtime{n}{\lag} = 0$, and the bias vanishes. Thus, we assume in the following that $\lag < n$, which means that $\lagtime{n}{\lag} = n - \lag$. Then, by the Cauchy-Schwartz inequality, $\prob$-a.s,
\begin{align}
B_n &\leq D^2  \| h \|_\infty^2 \sum_{m = \lag + 1}^\infty \rho^{2m} \exp(2m \varepsilon_m) \nonumber \\
&\leq D^2 \| h \|_\infty^2 \left( \sum_{m = \lag + 1}^\infty \rho^{2 m} \right)^{1/2} \left( \sum_{m = \lag + 1}^\infty \rho^{2 m} \exp(4 m \varepsilon_m) \right)^{1/2} \nonumber \\
&\leq D^2 \| h \|_\infty^2 \rho^{\lag + 1} \left( \sum_{m = 0}^\infty \rho^{2 m} \right)^{1/2} \left( \sum_{m = 0}^\infty \rho^{2 m} \exp(4 m \varepsilon_m) \right)^{1/2} \nonumber \\ 
&= C \| h \|_\infty^2 \rho^{\lag + 1}, \label{eq:bias:cauchy:bound}
\end{align}
where random variable  
$$
C \eqdef D^2 \left( \sum_{m = 0}^\infty \rho^{2 m} \right)^{1/2} \left( \sum_{m = 0}^\infty \rho^{2 m} \exp(4 m \varepsilon_m) \right)^{1/2}
$$
is $\prob$-a.s. finite and independent of $\lag$, $h$, and $\init$. For $c \in \rset_+$, write, using \eqref{eq:fundamental:bias:bound} and  \eqref{eq:bias:cauchy:bound}, 
$$
\prob \left( \frac{\bias{\chunk{\per}{0}{n - 1}}{\lag}(h)}{\rate^{\lambda + 1} \| h \|_{\infty}^2} > c \right) = \prob \left( \frac{\bias{\chunk{\per}{-n}{- 1}}{\lag}(h)}{\rate^{\lambda + 1} \| h \|_{\infty}^2} > c \right) \leq \prob \left( A_n C > c \right),  
$$
where the probability on the right hand side is again independent of $\lag$, $h$, or $\init$. Thus, using the stationarity of $\{ A_n \}_{n \in \nset}$, 
$$
\prob \left( \frac{\bias{\chunk{\per}{0}{n - 1}}{\lag}(h)}{\rate^{\lambda + 1} \| h \|_{\infty}^2} > c \right) \leq \prob \left( A_0 > c^{1/2} \right)  + \prob \left( C > c^{1/2} \right), 
$$ 
where the right hand side does not depend on $n$. Now, the $\prob$-a.s. finiteness of $A_0$� was established as a part of the proof of \cite[Theorem~10]{douc:moulines:olsson:2014}. Consequently, as also $C$ is $\prob$-a.s. finite, there exists, for all $k \in \nsetpos$, a constant $c_k \in \rset_+$, independent of $\lag$, $h$, and $\init$, such that the probabilities $\prob( A_0 > c^{1/2}_k)$� and $\prob( C > c^{1/2}_k)$ are both bounded by $1/(2 k)$. This  completes the proof. 
\end{proof}

% Proof in the case of flows of updated measures

\subsection{Proof of Proposition~\ref{prop:consistency:fixed:lag:updated:case}} 

\begin{proof}
The proof consists mainly in combining some of the equalities in the proof of Proposition~\ref{prop:consistency:fixed:lag} with the identity     
\begin{multline} \label{eq:function:prod:identity}
\{�\pot[z_n] (h - \filtpart[\chunk{z}{0}{n}] h) \}^{\varotimes 2} = (\pot[z_n] h)^{\varotimes 2} - \{ (\pot[z_n] h) \varotimes \pot[z_n] \} \filtpart[\chunk{z}{0}{n}] h \\�
- \{ \pot[z_n] \varotimes (\pot[z_n] h) \} \filtpart[\chunk{z}{0}{n}] h + \pot[z_n]^{\varotimes 2} (\filtpart[\chunk{z}{0}{n}] h)^2. 
\end{multline}
More specifically, as it holds that 
\begin{equation} \label{eq:zero:identity}
\predpart[\chunk{z}{0}{n - 1}](\pot[z_n] \{ h - \filtpart[\chunk{z}{0}{n}] h \})= 0,
\end{equation}   
by reusing the equality in \eqref{eq:critical:identity},
\begin{multline} \label{eq:filtering:variance:numerator}
\varest[2]{\chunk{z}{0}{n - 1}}[\lambda](\pot[z_n] \{ h - \filtpart[\chunk{z}{0}{n}] h\}) = \\
\sum_{m = \lagtime{n}{\lag}}^n \frac{\mumeas{\N, \unitstr{m}{n}}{\chunk{z}{0}{n - 1}} \{�\pot[z_n] (h - \filtpart[\chunk{z}{0}{n}] h) \}^{\varotimes 2} - \mumeas{\N, 0_n}{\chunk{z}{0}{n - 1}} \{�\pot[z_n] (h - \filtpart[\chunk{z}{0}{n}] h) \}^{\varotimes 2}}{(\unpredpart[\chunk{z}{0}{n - 1}] \1_\Xsp)^2} \\
+ \| \pot[z_n] \|_\infty^2 \| h \|_\infty^2 \ordo(\N^{-1}). 
\end{multline}
Now, write, using \eqref{eq:function:prod:identity}, for all $\chunk{b}{0}{n} \in \Binsp{n}$, 
\begin{multline} \label{eq:termwise:decomposition:updated:case}
\mumeas{\N, \chunk{b}{0}{n}}{\chunk{z}{0}{n - 1}} \{�\pot[z_n] (h - \filtpart[\chunk{z}{0}{n}] h) \}^{\varotimes 2} =
\mumeas{\N, \chunk{b}{0}{n}}{\chunk{z}{0}{n - 1}} (\pot[z_n] h)^{\varotimes 2} \\
- \mumeas{\N, \chunk{b}{0}{n}}{\chunk{z}{0}{n - 1}} \{ (\pot[z_n] h) \varotimes \pot[z_n] \} \filtpart[\chunk{z}{0}{n}] h 
- \mumeas{\N, \chunk{b}{0}{n}}{\chunk{z}{0}{n - 1}} \{ \pot[z_n] \varotimes (\pot[z_n] h) \} \filtpart[\chunk{z}{0}{n}] h \\
+ \mumeas{\N, \chunk{b}{0}{n}}{\chunk{z}{0}{n - 1}} \pot[z_n]^{\varotimes 2} (\filtpart[\chunk{z}{0}{n}] h)^2. 
\end{multline}
Applying Lemma~\ref{lemma:mu:convergence} to each term of \eqref{eq:termwise:decomposition:updated:case} (note that the second part of Lemma~\ref{lemma:mu:convergence} implies the consistency of the updated particle measures, as 
\begin{equation} \label{eq:particle:filter:consistency}
\filtpart[\chunk{z}{0}{n}] h = \frac{\predpart[\chunk{z}{0}{n - 1}] (\pot[z_n] h)}{\predpart[\chunk{z}{0}{n - 1}] \pot[z_n]} \plim \frac{\init \uk[\chunk{z}{0}{n - 1}] (\pot[z_n] h)}{\init \uk[\chunk{z}{0}{n - 1}] \pot[z_n]} = \filt[\chunk{z}{0}{n}] h
\end{equation} 
when $\N$ tends to infinity, a now classical result) yields, using again \eqref{eq:function:prod:identity}, 
\begin{multline*}
\mumeas{\N, \chunk{b}{0}{n}}{\chunk{z}{0}{n - 1}} \{�\pot[z_n] (h - \filtpart[\chunk{z}{0}{n}] h) \}^{\varotimes 2} \plim 
\mumeas{\chunk{b}{0}{n}}{\chunk{z}{0}{n - 1}} (\pot[z_n] h)^{\varotimes 2} \\
- \mumeas{\chunk{b}{0}{n}}{\chunk{z}{0}{n - 1}} \{ (\pot[z_n] h) \varotimes \pot[z_n] \} \filt[\chunk{z}{0}{n}] h 
- \mumeas{\chunk{b}{0}{n}}{\chunk{z}{0}{n - 1}} \{ \pot[z_n] \varotimes (\pot[z_n] h) \} \filt[\chunk{z}{0}{n}] h \\
+ \mumeas{\chunk{b}{0}{n}}{\chunk{z}{0}{n - 1}} \pot[z_n]^{\varotimes 2} (\filt[\chunk{z}{0}{n}] h)^2
= \mumeas{\chunk{b}{0}{n}}{\chunk{z}{0}{n - 1}} \{�\pot[z_n] (h - \filt[\chunk{z}{0}{n}] h) \}^{\varotimes 2}, 
\end{multline*}
 as $\N$ tends to infinity. Now, applying the previous limit to \eqref{eq:filtering:variance:numerator} and using \eqref{eq:variance:alt:expression}, \eqref{eq:zero:identity}, and the second part of Lemma~\ref{lemma:mu:convergence} yields 
\begin{multline*}
\varestfilt[2]{\chunk{z}{0}{n}}[\lambda](h) = \frac{\varest[2]{\chunk{z}{0}{n - 1}}[\lambda](\pot[z_n] \{h - \filtpart[\chunk{z}{0}{n}] h\})}{(\predpart[\chunk{z}{0}{n - 1}] \pot[z_n])^2} \\�
\plim  \frac{\variance[2]{\chunk{z}{0}{n - 1}}[\lambda](\pot[z_n] \{ h - \filt[\chunk{z}{0}{n}] h \})}{(\pred[\chunk{z}{0}{n - 1}] \pot[z_n])^2} = \filtvariance[2]{\chunk{z}{0}{n}}[\lambda](h)
\end{multline*}
as $\N$ tends to infinity, which completes the proof. 
\end{proof}

\subsection{Proof of Theorem~\ref{thm:tightness:bias:filter}}

\begin{proof}
Pick arbitrarily $n \in \nset$, $\lag \in \nset$, and $h \in \bmf{\Xfd}$. Using the expression of $\filtvariance[2]{\chunk{\per}{0}{n}}(h)$� derived in the proof of \cite[Theorem~11]{douc:moulines:olsson:2014}, one may express the bias $\biasfilt{\chunk{\per}{0}{n}}{\lag}(h)$ as 
\begin{equation*} 
\biasfilt{\chunk{\per}{0}{n}}{\lag}(h) = \sum_{m = 0}^{\lagtime{n}{\lag} - 1} \int \pred[\chunk{\per}{0}{m - 1}](\rmd x) \left( \frac{\DDelta{\delta_x,\pred[\chunk{\per}{0}{m - 1}]}{\chunk{\per}{m}{n - 1}}{\pot[\per_n] h, \pot[\per_n]}}
{(\pred[\chunk{\per}{0}{m - 1}] \uk[\chunk{\per}{m}{n - 1}] \1_{\Xsp} )^2} \right)^2,  
\end{equation*}
where $\DDelta{\delta_x,\pred[\chunk{\per}{0}{m - 1}]}{\chunk{\per}{m}{n - 1}}{\pot[\per_n] h, \pot[\per_n]}$ is given by \eqref{eq:def-Delta}. Thus, the proof is finalised by following closely the lines of the proof of Theorem~\ref{thm:tightness:bias} and noting that the statement of \cite[Proposition~16(iii)]{douc:moulines:olsson:2014} still holds true when $h$� and $\1_\Xsp$ are replaced by $\pot[\per_n] h$ and $\pot[\per_n]$, respectively. 
\end{proof}

\subsection{Proof of Theorem~\ref{thm:strong:bias:bound}}
\label{sec:proof:strong:bias:bound}

\begin{proof}
If $n \leq \lag$, the bias vanishes by definition; we thus assume that $n > \lag$. Write, for $m \in \intvect{0}{n - \lag}$ and $x \in \Xsp$, 
\begin{multline} \label{eq:strong:bound:decomposition}
\frac{\uk[\chunk{z}{m}{n - 1}](h - \pred[\chunk{z}{0}{n - 1}] h)(x)}{\pred[\chunk{z}{0}{m - 1}] \uk[\chunk{z}{m}{n - 1}] \1_{\Xsp}} \\�
= \frac{\delta_x \uk[\chunk{z}{m}{n - 1}] \1_{\Xsp}}{\pred[\chunk{z}{0}{m - 1}] \uk[\chunk{z}{m}{n - 1}] \1_{\Xsp}} \left( \frac{\delta_x \uk[\chunk{z}{m}{n - 1}] h}{\delta_x \uk[\chunk{z}{m}{n - 1}] \1_{\Xsp}} - \frac{\pred[\chunk{z}{0}{m - 1}] \uk[\chunk{z}{m}{n - 1}] h}{\pred[\chunk{z}{0}{m - 1}] \uk[\chunk{z}{m}{n - 1}] \1_{\Xsp}} \right),   
\end{multline}
where \cite[Proposition~4.3.4]{delmoral:2004} bounds uniformly the second factor of \eqref{eq:strong:bound:decomposition} according to 
\begin{equation} \label{eq:strong:bound:second:term}
\left| \frac{\delta_x \uk[\chunk{z}{m}{n - 1}] h}{\delta_x \uk[\chunk{z}{m}{n - 1}] \1_{\Xsp}} - \frac{\pred[\chunk{z}{0}{m - 1}] \uk[\chunk{z}{m}{n - 1}] h}{\pred[\chunk{z}{0}{m - 1}] \uk[\chunk{z}{m}{n - 1}] \1_{\Xsp}} \right| \leq \varrho^{n - m} \| h \|_\infty. 
\end{equation}
To bound the first factor of \eqref{eq:strong:bound:decomposition}, note that
$$
\pred[\chunk{z}{0}{m - 1}] \uk[\chunk{z}{m}{n - 1}] \1_{\Xsp} = \filt[\chunk{z}{0}{m - 1}] \mk (\pot[z_m] \mk \uk[\chunk{z}{m + 1}{n - 1}] \1_{\Xsp}) \\
\geq \potlow \mdlow \mu \uk[\chunk{z}{m + 1}{n - 1}] \1_{\Xsp}
$$
and 
$$
\delta_x \uk[\chunk{z}{m}{n - 1}] \1_{\Xsp} = \pot[z_m](x) \mk \uk[\chunk{z}{m + 1}{n - 1}] \1_{\Xsp}(x) \leq \potup \mdup \mu \uk[\chunk{z}{m + 1}{n - 1}] \1_{\Xsp},
$$
where $(\mdlow, \mdup)$ and $(\potlow, \potup)$ are given in \B{ass:strong:mixing}(\ref{ass:strong:mixing-2}), implying that  
\begin{equation} \label{eq:strong:bound:first:term}
\frac{\delta_x \uk[\chunk{z}{m}{n - 1}] \1_{\Xsp}}{\pred[\chunk{z}{0}{m - 1}] \uk[\chunk{z}{m}{n - 1}] \1_{\Xsp}} \leq \frac{ \potup \mdup}{ \potlow \mdlow}.\end{equation}
Now, using \eqref{eq:strong:bound:second:term} and \eqref{eq:strong:bound:first:term}, proceed like  
\begin{multline*}
\bias{\chunk{z}{0}{n - 1}}{\lag}(h) = \sum_{m = 0}^{n - \lag - 1} \frac{\pred[\chunk{z}{0}{m - 1}] \{ \uk[\chunk{z}{m}{n - 1}](h - \pred[\chunk{z}{0}{n - 1}] h) \}^2 }{(\pred[\chunk{z}{0}{m - 1}] \uk[\chunk{z}{m}{n - 1}] \1_{\Xsp})^2} \\ 
\leq \| h \|_{\infty}^2 \left( \frac{ \potup \mdup}{ \potlow \mdlow} \right)^2  \sum_{m = 0}^{n - \lag - 1} \varrho^{2(n - m)} \leq c \| h \|_{\infty}^2 \varrho^{2(\lag + 1)}, 
\end{multline*}
with $c \eqdef (\potup \mdup / \potlow \mdlow)^2 / (1 - \varrho^2)$, and the proof is complete. 
\end{proof}












\bibliographystyle{plain}
\bibliography{mybibfile}


\end{document}
