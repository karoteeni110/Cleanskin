\section{A class of higher dimensional examples}
\label{secExamples}

In this section we will construct a general class of examples of K\"ahler subgroups of direct products of surface groups arising as kernels of homomorphisms onto $\ZZ^{2k}$ for any $k\geq 1$. A subgroup $H\leq G_1 \times \dots \times G_r$ of a direct product of $r$ groups $G_i$ is called \textit{full} if all intersections $G_i\cap H:= \left(1\times \dots \times 1 \times G_i \times 1 \times \dots \times 1\right) \cap H$ are non-trivial, and \textit{subdirect} if $p_i(H)= G_i$ for all $1\leq i \leq r$, with $p_i: G_1 \times \dots \times G_r\to G_i$ the projection. We will give a more general discussion of full subdirect products of surface groups and their finiteness properties in Section \ref{secNotProd}.

Let $E=\CC/\Lambda$ be an elliptic curve, let $r\geq 3$ and let
\[
 \alpha _i: S_{\g_i}\rightarrow E
\]
be branched holomorphic coverings for $1\leq i \leq r$. 

Our groups will be the fundamental groups of the fibres of holomorphic surjective maps from the direct product $S_{\g_1}\times \cdots \times S_{\g_r}$ onto the $k$-fold direct product $E^{\times k}$ of $E$ with itself. For vectors $w_1,\cdots w_n\in \ZZ^k$ we will use the notation $\left(w_1\mid \cdots \mid w_n\right)$ to denote the $k\times n$-matrix with columns $w_i$. To construct these maps we make use of the following result

\begin{lemma}
Let $v_1=\left(v_{1,1},\cdots,v_{k,1}\right)^t,\cdots, v_r=\left(v_{1,r},\cdots,v_{k,r}\right)^t\in \ZZ^k$. Then the $\CC$-linear map $B=(v_1\mid v_2\mid \cdots \mid v_r)\in \ZZ^{k\times r}\subset \CC^{k\times r}$ descends to a holomorphic map
\[
 \overline{B}: E^{\times r}\rightarrow E^{\times k}.
\]
If, in addition, $r=k$ and $B\in \rm{GL}(k,\CC)\cap \ZZ^{k\times k}$ then $\overline{B}$ is a regular covering map. In particular, $\overline{B}$ is a biholomorphic automorphism of $E^{\times k}$ if $B\in \rm{GL}(k,\ZZ)$.
\label{lemAutkEll}
\end{lemma}

\begin{proof}
 It suffices to prove that $B$ preserves maps $\Lambda^{\times r}$ into $\Lambda^{\times k}$. For this let $\lambda_1,\lambda_2\in \CC$ be a $\ZZ$-basis for $\Lambda$ and denote by $\lambda_{1,i},\lambda_{2,i}$ the corresponding $\ZZ$-basis of the ith factor of $\Lambda ^ {\times r}$ and by $\lambda'_{1,j},\lambda'_{2,j}$ the corresponding $\ZZ$-basis of the jth factor of $\Lambda^{\times k}$. Then we have
 \[
  B \lambda_{i,j} = \sum _{l=1}^k v_{l,j} \lambda'_{i,l} \in \Lambda^{\times k} \mbox{    for   } 1\leq i \leq 2,\mbox{  }1\leq j \leq r
 \]
It follows that $B$ descends to a holomorphic map $\overline{B}:E^{\times r}\rightarrow E^{\times k}$. If $B\in \rm{GL}(k,\CC)\cap \ZZ^{k\times k}$ then $\overline{B}$ is a regular covering map, since $B$ is a local homeomorphism. If, in addition, $B\in \rm{GL}(k,\ZZ)$, then it is immediate that $\overline{B}$ and $\overline{B^{-1}}$ are mutual inverses.
\end{proof}



We say that a set of vectors $\mathcal{C}=\left\{v_1,\cdots, v_r\right\}\subset\ZZ^k$ has property 
\begin{enumerate}
\item[(P)] if there is $1\leq i_1 < \dots <i_k \leq r$ such that $\left\{v_{i_1},\dots,v_{i_k}\right\}$ is a $\ZZ$-basis for $\ZZ^k$ and any choice of $k$ vectors in $\mathcal{C}$ is linearly independent; and
\item[(P')] if $\mathcal{C}$ has property (P), and in addition $i_j=j$ and $\left\{v_1,\dots v_k\right\}$ is the standard basis for $\ZZ^k$.
\end{enumerate}

By Lemma \ref{lemAutkEll} for any set $\mathcal{C}=\left\{v_1,\cdots,v_r\right\}\subset \ZZ^k$ and $B=\left(v_1\mid \cdots \mid v_r\right)$ we can define a holomorphic map 
\[
h=\overline{B}\circ \left(\alpha_1,\cdots,\alpha_r\right)=\sum \limits_{i=1}^r v_i \cdot \alpha_i :S_{\g_1}\times \cdots \times S_{\g_r}\rightarrow E^{\times k}.
\]

We will be interested in maps $h$ for which the set $\mathcal{C}$ has property (P). Note that, after reordering factors and adjusting by a biholomorphic automorphism of $E^{\times k}$, say $A\in \rm{GL}(k,\ZZ)$, we may in fact assume that $\mathcal{C}$ has property (P') if it has property (P). 

The following result shows that such maps exist.

\begin{lemma}
 For all positive integers $r,k$ there is a set $\mathcal{C}=\left\{v_1,\cdots,v_r\right\}\subset \ZZ^k$ such that for all $1\leq i_1 <\dots <i_k\leq r$ the set $\left\{v_{i_1},\dots, v_{i_k}\right\}$ is linearly independent. Moreover, if $r\geq k$ we may assume that $\mathcal{C}$ satisfies property (P').
\label{lemLinIndep}
\end{lemma}

\begin{proof}
The proof is by induction on $r$. For $r=1$ the statement is trivial. Assume that for a positive integer $r$ we have a set $\mathcal{C}=\left\{v_1,\cdots, v_r\right\}\subset \ZZ^k$ of $r$ vectors with the property that for any $1\leq i_1<i_2<\cdots <i_k\leq r$ the subset $\left\{v_{i_1},\cdots,v_{i_k}\right\}\subset \mathcal{C}$ is linearly independent.

 Let $I$ be the set of all $(k-1)$-tuples $\mathbf{i}=(i_1,\cdots, i_{k-1})$ of integers $1\leq i_1<i_2<\cdots <i_{k-1}\leq r$. For $\mathbf{i}\in I$ denote by $W_{\mathbf{i}}=\mathrm{span}_{\CC}\left\{v_{i_1},\cdots, v_{i_{k-1}}\right\}$ the $\CC$-span of the linearly independent set $\left\{v_{i_1},\cdots, v_{i_{k-1}}\right\}$. Then $W=\bigcup\limits_{\mathbf{i}\in I} W_{\textbf{i}}$ is a finite union of complex hyperplanes in $\CC^k$ such that for any vector $v_{r+1}\in \ZZ^k\setminus W$ the set $\mathcal{C}\cup \left\{v_{r+1}\right\}$ has the desired property. The set $\ZZ^k\setminus W$ is nonempty, because $\ZZ^k\subset \CC^k$ is Zariski-dense in $\CC^k$.
 
 For $r\geq k$, choosing $\left\{v_1,\cdots,v_k\right\}$ to be the standard basis of $\CC^k$ ensures that property (P') is satisfied.
\end{proof}

\begin{remark}
 The proof of Lemma \ref{lemLinIndep} shows that we can choose any $m\leq r$ explicit vectors $v_1,\dots, v_m$ such that $\left\{v_1,\dots,v_m\right\}$ has property (P') and then complete to a set $\left\{v_1,\dots,v_r\right\}$ with property (P').
\end{remark}

Note that the proof of Lemma \ref{lemLinIndep} shows that the property (P') is in some sense a generic property.

The main result of this section is:

\begin{theorem}
 Let $1\leq k \leq r-2$ and let $\mathcal{C}\subset \ZZ^k$ be as defined above. Assume that $\mathcal{C}$ satisfies property (P'), and that $\alpha_1,\cdots,\alpha_k$ are surjective on fundamental groups. Then the smooth generic fibre $\overline{H}$ of $h$ is connected and its fundamental group fits into a short exact sequence
 \[
  1\rightarrow \pi_1 \overline{H} \rightarrow \pi_1 S_{\g_1}\times \cdots \times \pi_1 S_{\g_r}\stackrel{h_{\ast}}{\rightarrow} \pi_1 E^{\times k} = \ZZ^{2k}\rightarrow 1.
 \]
 Furthermore, $\pi_1 \overline{H}$ is an irreducible K\"ahler group of type $\mathcal{F}_{r-k}$ but not of type $\mathcal{F}_{r-k+1}$. In fact $\pi_j \overline{H} = 0$ for $2\leq j \leq r-k-1$.
 \label{thmExsGenClass}
\end{theorem}

Denote by $\pi_l : E^{\times k}\rightarrow \left\{0\right\} \times E^{\times l}$ the canonical projection onto the last $l$ factors and for a map $h$ satisfying the conditions of Theorem \ref{thmExsGenClass} let $h_l=\pi_l\circ h$. Due to the assumptions on $\mathcal{C}$ the map $h_l$ factors as $h_l= f_l\circ g_l$ for $1\leq l \leq k$ with
 \[
  f_l=\pi_l \circ \left(v_{k-l+1} \mid \cdots  \mid v_r\right)\circ(\a _{k-l+1},\cdots, \a _r):  S_{\g_{k-l+1}}\times \cdots \times S_{\g_r}\rightarrow Y^k/Y^{k-l}= \left\{0\right\} \times E^{\times l}  
 \]
 and
 \[
 g_l: S_{\g_1}\times \cdots \times S_{\g_r}\rightarrow  S_{\g_{k-l+1}} \times \cdots \times S_{\g_r}
 \]
 the canonical projection with fibre $F_l:= S_{\g_1}\times \cdots \times S_{\g_{k-l}}$, a product of closed hyperbolic surfaces (It follows from the fact that $v_1,\cdots, v_k\in \ZZ^k$ is a standard basis of $\ZZ^k$ that $h_l=f_l\circ g_l$ for $1\leq l \leq k$).

Theorem \ref{thmExsGenClass} will be a consequence of Theorem \ref{thmFiltVer} after checking that the maps $h$, $h_l$, $g_l$ and $f_l$ satisfy all necessary conditions.







The following is a natural and straight-forward generalization of \cite[Lemma 2.1]{Llo-16-II}.

\begin{lemma}
\label{lemNewConnFib}
 Let $X$ be a connected compact complex manifold, $E$ an elliptic curve and $f: X \rightarrow E$ a surjective holomorphic map. If there is a subgroup $A=\ZZ^2 \leq \pi_1 X$ such that $f_{\ast} A \leq \pi_1 E$ is a finite index subgroup and $f_{\ast} \pi_1 X = \pi_1 E$, then $f$ has connected fibres.
\end{lemma}

\begin{proof}
 Since $f$ is proper, by Stein factorisation, there is a compact Riemann surface $S$, such that $f$ factors as
 \[
 \xymatrix{ X \ar[r]^{h_1} \ar[rd]_f & S\ar[d] ^{h_2}\\ & E}
 \]
 where $h_1$ is holomorphic with connected fibres and $h_2$ is holomorphic and finite-to-one. In particular, $h_2$ is a (possibly ramified) covering map. 
 
 By assumption, the restriction $f_{\ast}|_A$ is injective. Thus, $h_{1,\ast} A\leq \pi_1 S$ defines a $\ZZ^2$-subgroup of $\pi_1S$. It follows that $S$ is an elliptic curve and $h_2: S\to E$ is an unramified cover. Surjectivity of $f_{\ast}: \pi_1 S\to \pi_1 E$ implies that $S=E$. Hence, $f$ has connected fibres.
\end{proof}


\begin{lemma}
\label{propConnGenExs}
  Under the assumptions of Theorem \ref{thmExsGenClass}, the maps $h$, $h_l$, $f_l$ and $g_l$, $1\leq l \leq k$ have connected fibres.
\end{lemma}
\begin{proof}
It suffices to consider the maps $f_l$, as the maps $g_l$ clearly have connected fibres and the connectedness of the fibres of the maps $h_l$ then follows from the identity $h_l=f_l\circ g_l$.

Choose a generic $x^0\in E^{\times k}=Y^k$ as in Section \ref{sec:Restrictions} and let $H_l=f_l^{-1}(x^{0,l})$, $\overline{H}_l=h_l^{-1}(x^{0,l})$ be the corresponding fibres. The proof is by induction on $l$.

First consider $l=1$. Then we have
\[
f_1= \pi_1 \circ \left(v_k\mid \dots \mid v_r\right) \circ \left(\alpha_k,\dots, \alpha_r\right): S_{\gamma_k}\times \dots \times S_{\gamma_r}\to Y^k/Y^{k-1}=\left\{0\right\} \times E.
\]
In particular, we have $f_1 =\sum_{j=k}^r v_{k,j}\cdot \alpha_j$. By assumption on $\mathcal{C}$, $r-k+1\geq 2$, and we have $v_{k,k}=1$ and $v_{k,j}\neq 0$ for $j\geq k$. Since $\alpha_k: S_{\gamma_k}\to E$ is surjective on fundamental groups, the same holds for $f_1$. Furthermore, the restriction $f_1|_{S_{\gamma_j}}$ defines a branched covering of $E$ for $k\leq j \leq r$. Hence, $f_{1,\ast}(\pi_1 S_{\gamma_j})\leq \pi_1 (\left\{0\right\} \times E)$ is a finite index subgroup for $k\leq j \leq r$. It follows that $f_1$ satisfies all conditions of Lemma \ref{lemNewConnFib} and therefore has connected fibres.

Let now $2\leq l \leq k$. By choice of $x^0$, the corestriction
\[
f_l|_{f_l^{-1}(x^{0,l}+ Y^{k-l+1}/Y^{k-l})}:f_l^{-1}(x^{0,l}+ Y^{k-l+1}/Y^{k-l}) \to x^{0,l}+ Y^{k-l+1}/Y^{k-l}
\]
has smooth generic fibre $H_l$. Since by definition $\pi_l(v_{k-l+1})=\left(\begin{array}{c}1\\0\\ \vdots\\ 0\end{array}\right)$, we obtain
\[
f_l^{-1}(x^{0,l}+ Y^{k-l+1}/Y^{k-l})=S_{\gamma_{k-l+1}}\times f_{l-1}^{-1}(x^{0,l-1})= S_{\gamma_{k-l+1}}\times H_{l-1},
\]
with $H_{l-1}$ smooth and connected by induction assumption.

Choose a basepoint $z_0\in S_{\gamma_{k-l+1}}$. By assumption on $\mathcal{C}$ the set $\pi_l(\left\{v_{k-l+2},\dots,v_r\right\})$ spans $\mathds{C}^l$. Thus, the map
\[
\sum_{j=k-l+2}^r \pi_l(v_j)\cdot \alpha_j: S_{\gamma_{k-l+2}}\times \dots \times S_{\gamma_r} \to Y^k/Y^{k-l} = \left\{0\right\}\times  E^{\times l}
\]
is a surjective holomorphic map. Thus, the same holds for the restriction $f_l|_{\left\{z_0\right\} \times S_{\gamma_{k-l+2}}\times \dots \times S_{\gamma_r}}$. It follows that 
\[
f_l|_{\left\{z_0\right\} \times H_{l-1}} : \left\{z_0\right\} \times H_{l-1}\to x^{0,l}+Y^{k-l+1}/Y^{k-l}
\]
is a surjective holomorphic map between compact complex manifolds and therefore $f_{l,\ast}(\pi_1 H_{l-1})\leq \pi_1 E$ is a finite index subgroup. On the other hand $\alpha_{k-l+1}$ is surjective on fundamental groups by assumption. Hence, $f_l|_{f_l^{-1}(x^{0,l}+ Y^{k-l+1}/Y^{k-l})}$ satisfies the assumptions of Lemma \ref{lemNewConnFib}. Therefore $f_l$ has connected smooth generic fibres.
\end{proof}




\begin{proposition}
Under the assumptions of Theorem \ref{thmExsGenClass} consider the filtration $Y^l=E^{\times l}\times \left\{0\right\}$ of $E^{\times k}$ where $\pi_l: E^{\times k}\rightarrow Y^k/Y^{k-l}= \left\{0\right\}  \times E^{\times l}$ is the projection onto the last $l$ coordinates. 
 
 Then the map $h$ satisfies the condition that $h_l=\pi_l\circ h$ has fibrelong isolated singularities for $0\leq l \leq k$. More precisely, the factorisation $h_l=f_l\circ g_l$ satisfies that $g_l$ is a regular fibration and $f_l$ has isolated singularities. Furthermore, the dimension of the smooth generic fibre $H_l$ of $f_l$ is r-k for $1\leq l \leq k$.
 
 \label{propIsolSingGen}
\end{proposition}

\begin{proof}
Recall that by definition of $f_l$ and $g_l$ we have $h_l=f_l\circ g_l$. The map $g_l$ is clearly a regular fibration. To see that the map $f_l$ has isolated singularities consider its differential
 \begin{equation}
  \D f_l = \left(\D \pi_l (v_{k-l+1})\cdot \mathrm{d} \a_{k-l+1},   \dots  ,\D\pi_l(v_r) \cdot \mathrm{d} \a_r\right).
 \label{eqnCritPt} 
 \end{equation}

Note that by definition of $\pi_l$ the vector $\D \pi_l(v_i)$ is the vector in $\ZZ^{l}$ consisting of the last $l$ entries of $v_i$. By property (P') any $k$ vectors in $\mathcal{C}$ form a linearly independent set. Furthermore we chose $\mathcal{C}$ such that $\mathcal{E}_1= \left\{v_1,\dots, v_k\right\}$ is the standard basis of $\ZZ^k$. This implies that the set
\[
\mathcal{C}_l = \left\{ \D \pi_l (v_{k-l+1}), \dots, \D\pi_l(v_r)\right\} \subset \ZZ^l
\]
also has property (P'). In particular, any choice of $l$ vectors in $\mathcal{C}_l$ forms a linearly independent set.

It follows from \eqref{eqnCritPt} that a point $(z_{k-l+1},\cdots, z_r)\in  S_{\g_{k-l+1}}\times \cdots \times S_{\g_r}$ is a critical point of $f_l$ if and only if $z_i$ is a critical point of $\alpha_i$ for at least $r-k+1$ of the $z_i$, where $k-l+1\leq i\leq r$.

Thus, the set of critical points $C_{f_l}$ of $f_l$ is the union $C_{f_l}=\bigcup \limits_{i\in I_l} B_{l,i}$ of a finite number of $(l-1)$-dimensional submanifolds $B_{l,i}\subset S_{\g_{k-l+1}}\times \cdots \times S_{\g_r}$ with the property that for every surface factor $S_{\g_j}$ of $S_{\g_{k-l+1}}\times \cdots \times S_{\g_r}$ the projection of $B_{l,i}$ onto $S_{\g_j}$ is either surjective or has finite image. Linear independence of any $l$ vectors in $\mathcal{C}_l$ implies that the restriction of $f_l$ to any of the $B_{l,i}$ is locally (and thus globally) finite-to-one. Hence, the intersection $C_{f_l}\cap H_{l,y}$ is finite for any fibre $H_{l,y}=f_l^{-1}(y)$, $y\in \left\{0\right\}\times E^{\times l}$. 

In particular, the map $f_l$ has isolated singularities. It follows immediately from the definition of $f_l$ that the smooth generic fibre $H_l$ of $f_l$ has dimension $r-(k-l)- l=r-k$.
\end{proof}


\begin{proof}[Proof of Theorem \ref{thmExsGenClass}]
 The proof follows from Theorem \ref{thmFiltVerGen}, Lemma \ref{propConnGenExs}  and Proposition \ref{propIsolSingGen}. Indeed, by Lemma \ref{propConnGenExs} and Proposition \ref{propIsolSingGen} combined with the fact that $F_l$ is a direct product of closed hyperbolic surfaces, all assumptions in Theorem \ref{thmFiltVerGen} are satisfied with $n=r-k$. Thus, the map $h$ induces a short exact sequence 
 \[
  1\rightarrow \pi_1 \overline{H}\rightarrow \pi_1 S_{\g_1}\times \cdots \times \pi_1 S_{\g_r} \stackrel{h_{\ast}}{\rightarrow} \pi_1 E^{\times k} \rightarrow 1 
 \]
 and isomorphisms $\pi_i \overline{H} \cong \pi_i (S_{\g_1}\times \cdots \times S_{\g_r})\cong 0$ for $2\leq i \leq r-k-1$.
 
 Since $\overline{H}$ is the smooth generic fibre of a holomorphic map, it is a compact complex submanifold and thus a compact projective submanifold of the projective manifold $S_{\g_1}\times \dots \times S_{\g_r}$.Thus, $\pi_1\overline{H}$ is a K\"ahler group (and in fact even a projective group). Furthermore, $\overline{H}$ can be endowed with the structure of a finite CW-complex.
 
 It follows that we can construct a classifying space $K(\pi_1\overline{H},1)$ from the finite CW-complex $\overline{H}$ by attaching cells of dimension at least $r-k+1$. Hence, there is a $K(\pi_1 \overline{H},1)$ with finitely many cells in dimension less than or equal to $r-k$. Thus, the group $\pi_1 \overline{H}$ is of type $\mathcal{F}_{r-k}$. Finally, by Lemma \ref{lemNotProdNotFr} in Section \ref{secNotProd}, the group $\pi_1 \overline{H}$ is irreducible and not of type $\mathcal{F}_{r-k+1}$.
\end{proof}

\begin{proof}[Proof of Theorem \ref{thmIntroA}]
Theorem \ref{thmIntroA} is now a direct consequence of Theorem \ref{thmExsGenClass}.
\end{proof}

 \begin{proof}[Proof of Theorem \ref{corthmIntroA}] 
 The only thing that does not follow immediately from Theorem \ref{thmIntroA} and its proof is that $\pi_1 \overline{H}$ is a full subdirect product. Considering the set $\mathcal{C}$ where $\left\{v_1,\dots,v_k\right\}$ is a standard basis of $\ZZ^k$ and $v_{k+1}=v_1+\dots +v_k$ we see that the complement $\mathcal{C}\setminus \left\{v_i\right\}$ of any vector contains a basis of $\ZZ^k$. Choosing $\alpha_{k+1}$ such that it induces an epimorphism on fundamental groups then assures that the restriction of $h_{\ast}$ to $\pi_1 S_{g_1}\times \dots \times \pi_1 S_{g_{i-1}}\times 1 \times \pi_1 S_{g_{i+1}}\times \dots \times \pi_1 S_{g_r}$ is surjective on fundamental groups for $1\leq i \leq r$ (see proof of Lemma \ref{lemNotProdNotFr} for more details on $h_{\ast}$). It follows that $\pi_1\overline{H}$ is subdirect. It is full, because the image of the restriction of $h_{\ast}$ to any factor is abelian.
 \end{proof}
