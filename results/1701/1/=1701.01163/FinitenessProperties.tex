\section{Finiteness properties and irreducibility}
\label{secNotProd}

In this section we want to determine the precise finiteness properties of our examples and prove that they are irreducible.

 Let $H\leq G = G_1\times\dots \times G_r$ be a subgroup of a direct product of groups $G_1,\cdots, G_r$. For every $1\leq i_1 < \dots < i_k \leq r$ denote by $p_{i_1,\cdots,i_k}: G \rightarrow G_{i_1}\times \dots \times G_{i_k}$ the canonical projection. We say that the group $H$ \textit{virtually surjects onto $k$-tuples} if for every $1\leq i_1 < \cdots < i_k\leq r$ the group $p_{i_1,\cdots, i_k}(H)$ has finite index in $G_{i_1}\times \dots \times G_{i_k}$. We say that $H$ is \textit{surjective on $k$-tuples} if for every $1\leq i_1 < \cdots < i_k\leq r$ we have equality $p_{i_1,\cdots,i_k}(H)=G_{i_1}\times \dots \times G_{i_k}$. We say that $H$ is \textit{virtually surjective on pairs} (VSP) if $H$ virtually surjects onto $2$-tuples.
 
 We further say that $H$ is \textit{coabelian of rank $k$} if there is an epimorphism $\phi: G_1\times \dots \times G_r\to \ZZ^k$ such that $H=\ker \phi$, and that $H$ is virtually coabelian of rank $k$ if there are finite index subgroups $H_0\leq H$ and $G_{i,0}\leq G_i$ such that $H_0\leq G_{1,0}\times \dots \times G_{r,0}$ is coabelian of rank $k$. It is not hard to see that the coabelian rank of $H$ is invariant under passing to finite index subgroups (see also Remark \ref{remNotIsom}).

For subgroups of direct products of limit groups, a close relation between their finiteness properties and virtual surjection to $k$-tuples has been observed (see \cite{BriHowMilSho-13},\cite{Koc-10}, also \cite{Kuc-14}). In fact if a subgroup $H\leq G_1\times \cdots \times G_r$ of a direct product of finitely presented groups is subdirect (i.e. surjects onto 1-tuples) then $H$ is finitely generated; and if it is VSP then $H$ is itself finitely presented \cite[Theorem A]{BriHowMilSho-13}. The converse is not true in general; it is true though if $G_1,\cdots, G_r$ are (non-abelian) limit groups and $H$ is full subdirect \cite[Theorem D]{BriHowMilSho-13}.

More generally it is conjectured \cite{Koc-10} that, for $G_1,\cdots, G_r$ non-abelian limit groups and $H\leq G_1\times \cdots \times \times G_r$ a full subdirect product, the following are equivalent:
\begin{enumerate}
 \item $H$ is of type $\mathcal{F}_k$;
 \item $H$ virtually surjects onto $k$-tuples.
\end{enumerate}  

Kochloukova proved that (1) implies (2) and gave conditions under which (2) implies (1).

\begin{theorem}[{Kochloukova \cite[Theorem C]{Koc-10}}]
 For $r\geq 1$ let $G_1, \dots , G_r$ be non-abelian limit groups, let $H\leq G_1\times \dots \times G_r$ be a full subdirect product, and let $2\leq k \leq r$. If $H$ is of type $\mathcal{F}_k$ then $H$ virtually surjects onto $k$-tuples. The converse is true if $H$ is virtually coabelian.
 \label{thmKochlVSPk}
\end{theorem}

Note that in Kochloukova's statement of Theorem \ref{thmKochlVSPk} the condition is that $H$ has the homological finiteness type $\rm{FP}_k(\mathds{Q})$. By \cite[Corollary E]{BriHowMilSho-13} this is however equivalent to type $\mathcal{F}_k$ for subgroups of direct products of limit groups. In general we only have that $\mathcal{F}_k$ implies $\rm{FP}_k(\mathds{Q})$ (see for instance \cite[Section 8.2]{Geo-08}).

\begin{remark}
It should be noted that the examples constructed in Theorem \ref{thmExtendedRange} are in some sense degenerate examples of type $\mathcal{F}_{r-m-1}$, but not of type $\mathcal{F}_{r-m}$, as the only $(r-m)$-tuple in $\pi_1 S_{\gamma_1}\times \dots \times S_{\gamma_r}$ onto which $\pi_1 \overline{H}$ does not (virtually) surject is $\pi_1 S_{\gamma_{m+1}}\times \dots \times \pi_1 S_{\gamma_r}$.
\end{remark}

We shall need the following auxiliary result which is a consequence of Theorem \ref{thmKochlVSPk}.

\begin{lemma}
 Let $G_1,\cdots, G_r$ be groups and $Q$ be a finitely generated abelian group. Let $\phi: G_1 \times \dots \times G_r \rightarrow Q$ be an epimorphism. Assume that the subgroup $H=\ker \phi \leq G_1\times \dots \times G_r$ virtually surjects onto $m$-tuples. Then the group $\phi( G_{i_1}\times \cdots \times G_{i_{r-m}})\leq Q$ is a finite index subgroup of $Q$ for every $1\leq i_1 < \dots < i_{r-m}\leq r$.
 
 Under the stronger assumption that $H$ surjects onto $m$-tuples, the restriction  of $\phi$ to $G_{i_1}\times \dots \times G_{i_{r-m}}$ is surjective for all $1\leq i_1 < \dots < i_{r-m}\leq r$.
\label{lemSurjComp}
\end{lemma}

\begin{proof}
 Assume that $H$ virtually surjects onto $m$-tuples. Consider a product $G_{i_1}\times \dots \times G_{i_{r-m}}$ of $r-m$ factors. We may assume that $i_j=j$.
 
Let $g\in Q$ be an arbitrary element. By surjectivity of $\phi$ there exist elements $h_1\in G_1\times \cdots \times G_{r-m}$ and $h_2 \in G_{r-m+1}\times \cdots \times G_r$ such that $g=\phi(h_1)\cdot \phi(h_2)$. Since $H$ virtually surjects onto $m$-tuples there is $k\geq 1$ such that $h_2^k\in p_{r-m+1,\cdots,r}(H)$. Hence, there is $\overline{h}_1\in G_{1}\times \cdots \times G_{r-m}$ such that $\overline{h}_1 \cdot h_2^k\in H=\ker \phi$. In particular it follows that $\phi(h_2^k)=\phi((\overline{h}_1)^{-1})$. As a consequence we obtain that $g^k = \phi(h_1)^k \cdot \phi(h_2)^k = \phi(h_1)^k\cdot \phi((\overline{h}_1)^{-1})\in \phi(G_{1}\times \cdots \times G_{r-m})$.

We proved that the abelian group $Q/\phi(G_{1}\times \cdots \times G_{r-m})$ has the property that each of its elements is torsion. This implies that $Q/\phi(G_{1}\times \cdots \times G_{r-m})$ is finite and thus $\phi(G_{1}\times \cdots \times G_{r-m})$ is a finite index subgroup of $Q$.

The second part follows immediately, since we can choose $k=1$ in the proof if H surjects onto $G_{r-m+1}\times \dots \times G_r$.
\end{proof}


\begin{corollary}
 Let $\phi: \L_1 \times \cdots \times \L_r\rightarrow Q$ be an epimorphism, where $\L _1,\cdots,\L_r$ are non-abelian limit groups and $Q$ is a finitely generated abelian group. If $\ker \phi$ is a full subdirect product of type $\mathcal{F}_m$ then the image $\phi(\L_{i_1} \times \cdots \times \L_{i_{r-m}})\leq Q$ is a finite index subgroup of $Q$ for all $1\leq i_1< \cdots < i_{r-m}\leq r$.
 \label{propCoNilpFm}
\end{corollary}
\begin{proof}
This is a direct consequence of Lemma \ref{lemSurjComp} and Theorem \ref{thmKochlVSPk}.
\end{proof}



As another consequence of Theorem \ref{thmKochlVSPk} we obtain. 

\begin{lemma}
 Let $G\leq \L_1\times \dots \times \L_r$ be a full subdirect product of $r$ non-abelian limit groups $\L_i$ of type $\mathcal{F}_m$ which is virtually a product $H_1\times H_2$. Then, after possibly reordering factors, there is $1\leq s \leq r$ such that $H_1\leq \L_1 \times \dots \times \L_s$ and $H_2\leq \L_{s+1}\times \dots \times \L_r$. Furthermore one of the following holds:
 \begin{enumerate}
  \item at least one of $H_1$ and $H_2$ is of type $\mathcal{F}_{\infty}$;
  \item $G$ virtually surjects onto at least one $2m$-tuple.
 \end{enumerate}
 \label{lemmNotProdKochl}
\end{lemma}
\begin{proof}
 Centralizers of elements in limit groups are infinite cyclic. Hence, subdirectness of $G$ implies that there is $1\leq s \leq r$ such that $H_1\leq \L_1 \times \dots \times \L_s$ and $H_2\leq \L_{s+1}\times \dots \times \L_r$. Theorem \ref{thmKochlVSPk} implies that $G$ and thus $H_1\times H_2$ virtually surjects onto $m$-tuples in $\L_1 \times \dots \times \L_r$. If either $s\leq m$ or $r-s\leq m$, say $s\leq m$, then $H_1 \times H_2$ surjects onto a finite index subgroup of $\L_1 \times \dots \times \L_s$. However, the projection is isomorphic to $H_1$. Thus, $H_1$ is a finite index subgroup of $\L_1 \times \dots \times L_s$ and therefore of type $\mathcal{F}_{\infty}$.
 
 Now assume that $s, ~ r-m > m$. In this case $H_1 \times H_2 $ virtually surjects onto $m$-tuples in $\L_1 \times \dots \times \L_s$ and $\L_{s+1}\times \dots \times \L_{r}$. Factoring the projections onto such $m$-tuples through the projections onto $\L_1 \times \dots \times \L_s$ and $\L_{s+1}\times \dots \times \L_r$, we see that $H_1\leq \L_1 \times \dots \times \L_s$ and $H_2\leq \L_{s+1}\times \dots \times \L_r$ also virtually surject onto $m$ tuples. Thus, $H_1\times H_2$ virtually surjects onto at least one $2m$-tuple and the same holds for $G$. 
\end{proof}

\begin{remark}
Note that Lemma \ref{lemmNotProdKochl} can also be applied to full subgroups $G\leq \L_1\times \dots \times \L_r$, after replacing the $\L_i$ by the projections of $G$ to $\L_i$ and $G$ by the quotient $G/Z(G)$ by the center $Z(G)$, since $G/Z(G)$ and $G$ have the same finiteness type by \cite[Proposition 2.7]{Bie-81}.
\end{remark}



We shall also need the following result by Kuckuck.

\begin{proposition}[{\cite[Corollary 3.6]{Kuc-14}}]
 Let $G\leq \L_1 \times \cdots \times \L_r$ be a full subdirect product of a direct product of $r$ non-abelian limit groups $\L_i$, $1\leq i \leq r$. If $G$ virtually surjects onto $m$ tuples for $m> \frac{r}{2}$ then $G$ is virtually coabelian. In particular, $G$ is virtually coabelian if $G$ is of type $\mathcal{F}_m$. 
 
 More precisely, we have that in either case there exist finite index subgroups $\L'_i\leq \L_i$, a free abelian group $A$ and a homomorphism
 \[
 \phi: \L'_1\times \cdots \times \L'_r \rightarrow A
 \]
such that $\ker \phi\leq G$ is a finite index subgroup.
\label{thmCoNilpFm}
\end{proposition}

We will require the following consequence of Theorem \ref{thmKochlVSPk} and Proposition \ref{thmCoNilpFm}:
\begin{corollary}
\label{corFinPropsProjFactors}
 Let $r\geq 1$ and let $G\leq \L_1 \times \cdots \times \L_r$ be a full subdirect product of non-abelian limit groups $\L_i$, $1\leq i \leq r$. Assume that $G$ is of type $\mathcal{F}_m$ with $m\geq 0$. For $k\geq 0$ with $m>\frac{k}{2}$ and $1\leq i_1 < \dots < i_k\leq r$ the projection $p_{i_1,\cdots,i_k}(G)\leq \L_{i_1}\times \dots \times \L_{i_k}$ is of type $\mathcal{F}_m$.
\end{corollary}
\begin{proof}
 By Theorem \ref{thmKochlVSPk} the group $G\leq \L_1 \times \cdots \times \L_r$ virtually projects onto $m$-tuples. Hence, the projection $\overline(G):=p_{i_1,\cdots,i_k}(G)\leq \L_{i_1}\times \cdots \times \L_{i_k}$ is full subdirect and virtually surjects onto $m$-tuples with $m> \frac{k}{2}$. By Proposition \ref{thmCoNilpFm} $\overline{G}$ is virtually coabelian in $\L_{i_1}\times \cdots \times \L_{i_k}$. Hence, the subgroup $\overline{G}\leq \L_{i_1}\times \cdots \times \L_{i_k}$ is full subdirect, virtually coabelian, and virtually projects onto $m$-tuples. The converse direction of Theorem \ref{thmKochlVSPk} then implies that $\overline{G}$ is of type $\mathcal{F}_m$.
\end{proof}

As a consequence of the results in this section we can determine the precise finiteness properties and irreducibility of the groups arising from our construction in Theorem \ref{thmExsGenClass}.
\begin{lemma}
 Under the assumptions of Theorem \ref{thmExsGenClass} and with the same notation, let  $\phi=h_{\ast}: \pi_1 S_{\g_1}\times \cdots \times \pi_1 S_{\g_r}\rightarrow \pi_1 E^{\times k}$ be the induced epimorphism on fundamental groups. Then $\ker \phi \cong \pi_1 \overline{H}$ is irreducible of type $\mathcal{F}_{r-k}$ and not of type $\mathcal{F}_{r-k+1}$.
 \label{lemNotProdNotFr}
\end{lemma}
\begin{proof}
 By the proof of Theorem \ref{thmExsGenClass} we know that $\ker \phi$ is of type $\mathcal{F}_{r-k}$. Hence, we only need to proof that $\ker \phi$ is not of type $\mathcal{F}_{r-k}$ and that $\ker \phi$ has no finite index subgroup which is a direct product of two non-trivial groups.
 
 As in the proof of Theorem \ref{thmExtendedRange} we have
 \[
  \phi(g_1,\cdots,g_r)=\sum_{i=1}^r v_i\cdot \alpha_i(g_i) \in (\pi_1 E)^{\times k}\cong (\ZZ^2)^{k}\cong \ZZ^{2k}
 \]
 for $(g_1,\cdots,g_r)\in \pi_1 S_{\g_1}\times \cdots \times \pi_1 S_{\g_r}$.
 
 Since the maps $\alpha_i$ are finite sheeted branched coverings, the image $\alpha_{i,\ast}(\pi_1S_{\g_i})\leq \pi_1 E$ is a finite index subgroup for $1\leq i \leq r$. The assumption that the $v_i$ satisfy property (P') implies that the image $\phi(\pi_1 S_{\g_{i_1}}\times \cdots \times \pi_1 S_{\g_{i_k}})\leq \pi_1 E^{\times k}$ of any $k$ factors is a finite index subgroup of $\pi_1 E^{\times k}\cong \ZZ^{2k}$, $1\leq i_1 < \cdots < i_k\leq r$.
 
 
 Since we have $r\geq k+2$ factors and any $k$ factors map to a finite index subgroup of $\pi_1 E^{\times k}$ the kernel of 
\[ 
 \phi_0=\phi|_{\L_1\times \cdots \times \L_r}: \L_1 \times \cdots \times \L_r\rightarrow \pi_1 E^{\times k}.
\]
is subdirect, after passing to finite index subgroups $\L_i\leq \pi_1 S_{\g_i}$. Note that the image $\rm{im} \phi_0\leq \pi_1 E^{\times k}$ is a finite index subgroup, thus isomorphic to $\ZZ^{2k}$, and that $\ker \phi_0 \leq \ker \phi$ is a finite index subgroup. The intersection $\L_i \cap \ker \phi_0\unlhd \L_i$ is a non-trivial normal subgroup of infinite index in $\L_i$, since $\phi(\L_i)\cong \ZZ^2$. Thus, $\ker \phi_0$ is a full subdirect product of $\L_1 \times \cdots \times \L_r$.
 
 Since the image of the restriction of $\phi_0$ to any factor $\L_i$ is isomorphic to $\ZZ^2$, the image of the restriction of $\phi$ to any $k-1$ factors $\L_{i_1}\times \cdots \times \L_{i_{k-1}}$ ($1\leq i_1 < \cdots < i_{k-1}$) is isomorphic to $\ZZ^{2(k-1)}$ (by the same argument as for $k$ factors). In particular, $\phi(\L_{i_1}\times \cdots \times \L_{i_{k-1}})$ is not a finite index subgroup of the image ${\rm{im}} \phi_0\cong \ZZ^{2k}$. By Corollary \ref{propCoNilpFm}, $\ker \phi_0$ and therefore its finite extension $\ker \phi \geq \ker \phi_0$ cannot be of type $\mathcal{F}_{r-k+1}$. 
 
 Assume that there is a finite index subgroup $H_1 \times H_2 \leq \ker \phi$ which is a product of two non-trivial groups $H_1$ and $H_2$. By Lemma \ref{lemmNotProdKochl} we may assume that (after reordering factors) $H_1 \leq \pi_1 S_{\g_1} \times \cdots \times \pi_1 S_{\g_s}$, $H_2 \leq \pi_1 S_{\g_{s+1}}\times \cdots \pi_1 S_{\g_r}$, for some $1\leq s \leq r-1$. If $H_1$ (or $H_2$) is of type $\mathcal{F}_{\infty}$ then it follows from \cite[Theorem A]{BriHowMilSho-09} that $H_1$ is virtually a product of finitely generated subgroups $\G_i \leq \pi_1 S_{\g_i}$, $1\leq i \leq s$. Since $\ker \phi_0$ is subdirect in $\L_1 \times \cdots \times \L_r$ and $\ker \phi_0 \cap (H_1\times H_2)\leq \ker \phi_0$ has finite index, the $\G_i$ must be finite index subgroups of the $\pi_1 S_{\g_i}$. This contradicts that the restriction of $\phi$ to any finite index subgroup of $\pi_1 S_{\g_i}$ has infinite image. Hence, by Lemma \ref{lemmNotProdKochl}, $\ker \phi$ virtually surjects onto at least one $2(r-k)$-tuple. However, the genericity condition (P') satisfied by $\mathcal{C}$ implies that $\ker \phi$ does not surject onto any $(r-k+1)$-tuple (the argument is the same as in the proof of Lemma \ref{lemSurjComp}). In particular $\ker \phi$ is irreducible.
\end{proof}

\begin{addendum}
 Note that the proof of Lemma \ref{lemNotProdNotFr} also shows that if we consider $\phi$ where the set $\mathcal{C}$, as defined in Theorem \ref{thmExsGenClass}, does not have the generic property described in Lemma \ref{lemLinIndep} then $\ker \phi$ must have finiteness type less than $\mathcal{F}_{r-k}$. In fact it shows that the finiteness type of $\ker \phi$ is at most $\mathcal{F}_{r-l}$ where $l-1$ is the size of a maximal subset of $\mathcal{C}$ which does not form a basis of $\CC^k$. Indeed we made use of this observation to construct the examples in Section \ref{secExCombined}.
 \label{addFinPropVary}
\end{addendum}





