\section{Restrictions on coabelian K\"ahler groups}
\label{secResCoabKGs}

All non-trivial examples of K\"ahler subgroups of direct products of surface groups constructed so far, are obtained as kernels of maps from a direct product of surface groups to a free abelian group. Hence, a natural special case of Delzant and Gromov's question is the following question.

\begin{question}
 Let $S_{g_1},\dots, S_{g_r}$ be closed hyperbolic Riemann surfaces, let $k\in \ZZ$ and let $\phi: \pi_1 S_{g_1}\times\dots \times \pi_1 S_{g_r}\to \ZZ^k$ be an epimorphism. When is $\ker \phi$ a K\"ahler group?
\end{question}

We will show that for $\ker \phi$ to be K\"ahler, $k$ must be even. This will follow from a study of the more general question of finding constraints on homomorphisms
\[
 \psi: G \rightarrow \pi_1 S_{g_1} \times \dots \times \pi_1 S_{g_k}
\]
from a K\"ahler group $G$ to a direct product of surface groups.

We will require the following easy lemma (see also \cite[Lemma 2.1]{Ara-11}).

\begin{lemma}
Let $X,Y$ be compact K\"ahler manifolds and let $f:X\rightarrow Y$ be a holomorphic map. Then the images, kernels and cokernels of the induced maps
\[
 f_{\ast} : H_1(\pi_1 X,\ZZ)=(\pi_1 X)_{ab} \rightarrow H_1(\pi_1 Y, \ZZ)=(\pi_1 Y)_{ab},
\]
\[
f^{\ast}: H^1(\pi_1 X,\ZZ) \rightarrow H^1(\pi_1Y,\ZZ)
\]
have even rank.
\label{lemHodgeHol}
\end{lemma}

\begin{proof}
Holomorphic maps between compact K\"ahler manifolds induce homomorphisms of Hodge structures
\[
 f^{\ast}: H^{\ast}(Y,\CC)\rightarrow H^{\ast}(X,\CC),
\]
\[
f_{\ast} : H_{\ast}(X,\CC)\rightarrow H_{\ast}(Y,\CC).
\]

We obtain commutative diagrams
\[
 \xymatrix{ H^{1,0}(Y,\CC)\ar[r]^{f^{\ast}}\ar[d]^{\cong} & H^{1,0}(X,\CC)\ar[d]^{\cong}\\
 			H^{0,1}(Y,\CC)\ar[r]^{f^{\ast}}\ar[r]^{f^{\ast}} & H^{0,1}(X,\CC)},
\]
where the vertical isomorphisms are induced by complex conjugation with respect to the complex structure on $X$, and
\[
 \xymatrix{ H^1(Y,\ZZ)\ar[r]^{f^{\ast}}\ar[d]^{\cong}& H^1(X,\ZZ)\ar[d]^{\cong}\\
 			H^1(\pi_1Y,\ZZ)\ar[r]^{f^{\ast}} &H^1(\pi_1 X,\ZZ).}
\]
The result in cohomology follows from $H^1(X,\CC)\cong H^{1,0}(X,\CC)\oplus H^{0,1}(X,\CC)$ and $H^1(Y,\CC)\cong H^{1,0}(Y,\CC)\oplus H^{0,1}(X,\CC)$. The proof in homology is analogous.
\end{proof}

As a consequence we obtain:
\begin{lemma}
 Let $G=\pi_1 X$ be the fundamental group of a compact K\"ahler manifold $X$ and assume that there is a homomorphism $\phi: G \rightarrow \G_{g_1}\times \dots \G_{g_r}$ which is induced by a holomorphic map and has coabelian image $\mathrm{Im}\phi = \overline{G}$. Then $\overline{G}$ is coabelian of even rank, i.e. there is $l\geq 0$ and $\psi: \G_{g_1}\times \dots \times \G_{g_r}\rightarrow \ZZ^{2l}$ such that $\overline{G}=\ker \psi$.
 \label{corNewCoab}
\end{lemma}
\begin{proof}
 Indeed, let $f: X \rightarrow S_{g_1}\times \dots \times S_{g_r}$ be a holomorphic map with $f_{\ast}=\phi$. Since $\overline{G}\leq \G_{g_1}\times \dots \times \G_{g_r}$ is coabelian, there is a short exact sequence
 \[
    1 \rightarrow \phi(\pi_1 X)=\overline{G} \rightarrow \G_{g_1}\times \dots \times \G_{g_r}\stackrel{\psi}{\rightarrow} \ZZ^k\rightarrow 1
 \]
 for some $k\geq 0$ and homomorphism $\psi$. By right exactness of abelianization we obtain an exact sequence
 \[
  \overline{G}_{ab}\stackrel{\phi_{ab}}{\rightarrow} (\G_{g_1}\times \dots \times \G_{g_r})_{ab}\rightarrow \ZZ^k \rightarrow 1.
 \] 
 It follows that $\ZZ^k$ is the cokernel of $\phi_{ab} =f_{\ast}$ and thus, by Lemma \ref{lemHodgeHol}, $k=2l$ for some integer $l\geq 0$. 
\end{proof}

We saw in Section \ref{secNotProd} that the image of $\phi$ is coabelian if it has sufficiently strong finiteness properties.
\begin{proposition}
Let $G=\pi_1 X$ and assume that there is a homomorphism $\phi: G\rightarrow \G_{g_1}\times \dots \times \G_{g_r}$ that is induced by a holomorphic map $f: X \rightarrow S_{g_1}\times \dots \times S_{g_r}$. Assume further that the image $\overline{G}=\phi(G)\leq \G_{g_1}\times \dots \times \G_{g_r}$ is full and of type $\mathcal{F}_k$ for $k>\frac{r}{2}$. Then $\overline{G} \leq \G_{g_1}\times \dots \times \G_{g_r}$ is virtually coabelian of even rank.
\label{propNewCoab}
\end{proposition}
\begin{proof}
 Since the composition $X\rightarrow S_{g_i}$ of $f$ with the projection to $S_{g_i}$ is holomorphic, it is either surjective or constant. The latter can not happen, since we assumed that $\overline{G}\leq \G_{g_1}\times \dots \times \G_{g_r}$ is full. Hence, the projection $p_i(\overline{G})\leq \G_{g_i}$ is a finite index subgroup. It follows that $\overline{G}\leq p_1(\overline{G})\times \dots \times p_r(\overline{G})$ is a full subdirect product of surface groups.
 
By Proposition \ref{thmCoNilpFm}, there exist $l\geq 0$, finite index subgroups $\G_{h_i}\leq p_i(\overline{G})$ and $\overline{G}_0\leq \overline{G}$, and an epimorphism $\psi: \G_{h_1}\times \dots \times \G_{h_r}\rightarrow \ZZ^l$ such that $\overline{G}_0=\ker \psi$. 

Let $q: X_0\rightarrow X$ be the holomorphic cover corresponding to $\overline{G}_0$ and let $q_i: S_{h_i}\rightarrow S_{g_i}$ be the holomorphic covers corresponding to $\G_{h_i}$. Since $(f\circ q)_{\ast} (\pi_1 X_0)\leq \G_{h_1}\times \dots \times \G_{h_r}$, there is a lift $g: X_0 \rightarrow S_{h_1}\times \dots \times S_{h_r}$ making the diagram
\[
\xymatrix{ & S_{h_1}\times \dots \times S_{h_r}\ar[d]^{(q_1,\dots,q_r)}\\
		   X_0 \ar[ur]^g \ar[r]^{f\circ q} & S_{g_1}\times \dots \times S_{g_r}}
\]
commutative. Considering local charts we see that the map $g$ is holomorphic. It follows from Lemma \ref{corNewCoab} that $\overline{G}_0\leq \G_{h_1}\times \dots \times \G_{h_r}$ is coabelian of even rank.
\end{proof}


Before proceeding to state and prove the main result of this section, we want to recall some results from the literature that we shall need.

For $g\geq 0$ and $\mm=(m_1,\dots,m_s)$, we denote by $S_{g,\mm}$ the closed orbisurface of genus $g$ with $s\geq 0$ cone points $D=\left\{p_1,\dots,p_r\right\}$ of multiplicities $m_i\geq 2$. Its \textit{orbifold fundamental group} is the group
\[
 \pi_1 S_{g,\mm} = \pi_1 (S_g \setminus D)/ \left\langle\left\langle \gamma_i ^{m_i} \mid i=1,\dots,s \right\rangle\right\rangle
\]
for $\gamma_i$, $1\leq i \leq r$, a loop bounding a small disc around $p_i$. We say that $S_{g,\mm}$ is equipped with a holomorphic structure if the underlying surface $S_g$ is equipped with a holomorphic structure. For a complex manifold $X$, a map $f: X\rightarrow S_{g,\mm}$ is called holomorphic if for each $p_i\in D$ there is a disc neighbourhood $U_i$ in which $f$ factors through a holomorphic map to the $m_i$-fold branched cover of $U_i$ in $p_i$.

\begin{theorem}
 Let $X$ be a compact K\"ahler manifold, let $G=\pi_1 X$, and let $\phi: G \rightarrow \pi_1 S_{g,\mm}$ be an epimorphism. Then $\phi$ factors through an epimorphism $\psi: G\rightarrow \pi_1 S_{h,\nn}$ with finitely generated kernel, which is induced by a holomorphic map $X \to S_{h,\nn}$.
 
 Moreover, $\phi$ has finitely generated kernel if and only if $\phi$ is induced by a holomorphic map $f: X \to S_{g,\mm}$ with connected fibres (for a suitable complex structure on $S_{g,\mm}$), such that the critical values of $f$ are the cone points $p_i$ and the multiplicity of the singular fibre over $p_i$ is $m_i$.
 \label{lemSiuBeauCat}
\end{theorem}

The surface group case of the first part of Theorem \ref{lemSiuBeauCat} is due to Siu \cite{Siu-87} and Beauville \cite{Bea-91}. This version of Theorem \ref{lemSiuBeauCat} is proved in \cite{Del-16}. A first explicit version of this result can be found in \cite{Cat-03}; however it it was probably known much earlier (see discussion in \cite{Kot-12}).

\begin{theorem}[{Napier, Ramachandran \cite{NapRam-01}}]
 Let $X$ be a K\"ahler manifold, let $G=\pi_1 X$ be its fundamental group and let $\phi: G\rightarrow \ZZ$ be an epimorphism with non-finitely generated kernel. Then $\phi$ factors through an epimorphism $\psi: G\rightarrow \pi_1^{orb} S_{g,\mm}$, where $S_{g,\mm}$ is a closed orientable hyperbolic Riemann orbisurface of genus $g\geq 2$ and $\psi$ has finitely generated kernel. The homomorphism $\phi$ is induced by a holomorphic fibration $X\rightarrow S_{g,\mm}$ for some complex structure on $S_{g,\mm}$.
\label{thmNapRam}
\end{theorem}

\begin{theorem}[{Bridson, Miller \cite[Theorem 4.6]{BriMil-09}}]
Let $\Lambda$ be a non-abelian  surface group, let $A$ be any group and let $G\leq \Lambda \times A$. Assume that $G$ is finitely presented and that the intersection $\Lambda \cap G$ is non-trivial. Then $G\cap A$ is finitely generated.
\label{thmBriMil}
\end{theorem}




\begin{lemma}
 Let $G$ be a K\"ahler group and let $\phi: G\rightarrow \overline{G}\leq \G_{g_1}\times \dots \times \G_{g_r}$ be an epimorphism such that the projections $p_i\circ \phi: G \rightarrow \G_{g_i}$ to factors have finitely generated kernel. Then the projection $p_i(\overline{G})\leq \G_{g_i}$ is either cyclic or of finite index.
 \label{lemSurCyc}
\end{lemma}

\begin{proof}
 Assume that $p_i(\overline{G})\leq \G_{g_i}$ is a non-cyclic subgroup. Since $p_i(\overline{G})$ is finitely generated, it is either finitely generated free or a surface group. Surface subgroups of $\G_{g_i}$ are precisely the finite index subgroups. Thus, assume that $p_i(\overline{G})$ is a finitely generated free group.
 
 There is an epimorphism $\theta: p_i(\overline{G})\rightarrow \ZZ$ with non-finitely generated kernel (since infinite index normal subgroups of free groups are not finitely generated). It follows that the kernel of the composition $\theta \circ p_i\circ \phi: G \rightarrow \ZZ$ is not finitely generated. By Theorem \ref{thmNapRam}, we obtain a commutative diagram
 \[
 \xymatrix{ G\ar[r]^{p_i\circ\phi}\ar[d]^{\psi} & p_i(\overline{G})\ar[d]^{\theta}\\
 			 \pi_1^{orb}S_{h,\mm}\ar[r]& \ZZ},
 \]
where $\psi$ is an epimorphism with finitely generated kernel for some closed hyperbolic Riemann orbisurface $S_{h,\mm}$.

It follows that the image $p_i(\phi(\ker(\psi)))\unlhd p_i(\overline{G})$ is a finitely generated normal subgroup. It has infinite index, since $\left|\ZZ\right|=\infty$. Thus, it is the trivial group. Hence, we obtain $\ker(\psi)\leq \ker (p_i\circ \phi)$. 

By assumption, the kernel of the projection $p_i\circ \phi : G \rightarrow \G_{g_i}$ is finitely generated. Since the only infinite index finitely generated normal subgroup of a closed hyperbolic Riemann orbisurface fundamental group is the trivial group, we also have $\ker (p_i\circ \phi) \leq \ker (\psi)$. It follows that $p_i(\overline{G})\cong \pi_1^{orb} S_{h,\mm}$, a contradiction.
\end{proof}

As a direct consequence of Lemma \ref{lemSurCyc}, we obtain:

\begin{corollary}
 Every K\"ahler subgroup $G$ of a direct product of finitely many surface groups is of the form $G\cong \ZZ^N \times G_0$ with $N\geq 0$ and $G_0$ a full subdirect product of finitely many surface groups.
\end{corollary}

We are now ready to prove Theorem \ref{thmNewCoab}. For the reader's convenience we recall it here.

\begin{thmIntro1}
 Let $G=\pi_1 X$ with $X$ compact K\"ahler and let $\phi: G \rightarrow \overline{G}$ be a surjective homomorphisms onto a subgroup $\overline{G}\leq \G_{g_1}\times \dots \times \G_{g_r}$. Assume that $\phi$ has finitely generated kernel and that $\overline{G}$ is full and of type $\mathcal{F}_m$ for $m\geq 2$.
 
 Then, after reordering factors, there is $s\geq 0$ such that, for any $k< 2m$ and any $1\leq i_1 < \dots < i_k\leq s$, the projection $p_{i_1,\dots,i_k}(\overline{G})\leq \G_{g_{i_1}}\times \dots \times \G_{g_{i_k}}$ is virtually coabelian of even rank. Furthermore, $\mathrm{Z}(\overline{G})= \overline{G} \cap \left(\G_{g_{s+1}}\times \dots \times \G_{g_r}\right)\leq p_{s+1,\dots,r}(\overline{G})\cong \ZZ^{r-s}$ is a finite index subgroup. 
\end{thmIntro1}

\begin{proof}
 By Lemma \ref{lemSurCyc}, the projections $p_i(\overline{G})\leq \G_{g_i}$ are either cyclic or of finite index. Since $\overline{G}\leq \G_{g_1}\times \dots \times \G_{g_r}$ is full they can not be trivial. Thus, after reordering factors, we may assume that there is $s\geq 0$ such that $p_i(\overline{G})\leq \G_{g_i}$ is a finite index subgroup for $1\leq i \leq s$ and $\ZZ$ for $s+1\leq i \leq r$. We may further assume that $p_i(\overline{G})=\G_{g_i}$ for $1\leq i \leq s$, since finite index subgroups of surface groups are surface groups. Hence, we obtain a short exact sequence 
 \[
  1 \rightarrow \ZZ^{r-s}\rightarrow \overline{G} \rightarrow p_{1,\dots,s}(\overline{G})\rightarrow 1
 \] 
where $p_{1,\dots,s}(\overline{G})\leq \G_{g_1}\times \dots \times \G_{g_s}$ is a full subdirect product. Since $\ZZ^{r-s}$ is of type $\mathcal{F}_{\infty}$, the group $p_{1,\dots,s}(\overline{G})$ is of type $\mathcal{F}_m$ \cite[Proposition 2.7]{Bie-81}.

For $1\leq i \leq s$, the kernel of the projection $p_i\circ\phi: G\rightarrow \G_{g_i}$ is the extension
\[
 1 \rightarrow \ker \phi \rightarrow \ker(p_i\circ \phi)\rightarrow \overline{G}\cap \left(\G_{g_1}\times \dots \times \G_{g_{i-1}}\times 1 \times\G_{g_{i+1}}\times \dots \times \G_{g_r}\right)\rightarrow 1.
\]

By Theorem \ref{thmBriMil}, the group $\overline{G}\cap \left(\G_{g_1}\times \dots \times \G_{g_{i-1}}\times 1 \times\G_{g_{i+1}}\times \dots \times \G_{g_r}\right)$ is finitely generated, since $\overline{G}$ is finitely presented. Extensions of finitely generated groups by finitely generated groups are finitely generated. Thus, the group $\ker(p_i\circ \phi)$ is finitely generated.

Theorem \ref{lemSiuBeauCat} implies that the epimorphism $p_i\circ \phi: G\rightarrow \G_{g_i}$ is induced by a holomorphic map $f_i: G\rightarrow S_{g_i}$ with respect to a suitable complex structure on $S_{g_i}$. It follows that the map
\[
f=(f_1,\dots,f_s): X\rightarrow S_{g_1}\times \dots \times S_{g_s}
\] 
is a holomorphic map inducing the compostion $p_{1,\dots,s}\circ \phi$ on fundamental groups.

For any $k\geq 0$ and $1\leq i_1< \dots < i_k\leq s$, the projection $S_{g_1}\times \dots \times S_{g_s}\rightarrow S_{g_{i_1}}\times \dots \times S_{g_{i_k}}$ is holomorphic and hence so is its composition $f_{i_1,\dots,i_k}:X\rightarrow S_{g_{i_1}}\times \dots\times S_{g_{i_k}}$ with $f$. Thus, the homomorphism $p_{i_1,\dots,i_k}\circ \phi: G\rightarrow p_{i_1,\dots,i_k}(\overline{G})$ is induced by the holomorphic map $f_{i_1,\dots,i_k}$.
 
 
If $k<2 m$, Corollary \ref{corFinPropsProjFactors} implies that $p_{i_1,\dots,i_k}(\overline{G})\leq \G_{g_{i_1}}\times \dots \times \G_{g_{i_k}}$ is of type $\mathcal{F}_m$. Hence, Proposition \ref{propNewCoab} implies that $p_{i_1,\dots,i_k}(\overline{G})$ is virtually coabelian of even rank.

Finally, it follows from the fact that the centralizer of every non-trivial element in a surface group is cyclic that $\mathrm{Z}(\overline{G})=\overline{G}\cap\left( \G_{g_{s+1}}\times \dots \times \G_{g_r}\right)$.
\end{proof}

\begin{remark}
Observe that the proof of Theorem \ref{thmNewCoab} shows more generally that if $\phi :G \rightarrow \overline{G}\leq \G_{g_1}\times \dots \times \G_{g_r}$ is a homomorphism from $G=\pi_1 X$, with $X$ compact K\"ahler, onto a finitely presented full subdirect product $\overline{G}\leq \G_{g_1}\times \dots \times \G_{g_r}$ of surface groups, then $\phi = f_{\ast}$ for a holomorphic map $f: X \to S_{g_1}\times \dots \times S_{g_r}$.
\label{rmkNewCoab1}
\end{remark}


Note further that the proof shows that the same conclusion holds if we replace the assumption that $\overline{G}$ is of type $\mathcal{F}_m$ by the assumption that $\overline{G}\leq \G_{g_1}\times \dots \times \G_{g_r}$ is virtually coabelian and finitely presented.
\begin{corollary}
\label{rmkNewCoab}
 Let $G=\pi_1 X$ with $X$ compact K\"ahler and let $\phi: G \rightarrow \overline{G}$ be a surjective homomorphisms onto a subgroup $\overline{G}\leq \G_{g_1}\times \dots \times \G_{g_r}$. Assume that $\phi$ has finitely generated kernel and that $\overline{G}$ is virtually coabelian and finitely presented.
 
 Then, for any $0\leq k \leq r$ and any $1\leq i_1 < \dots < i_k\leq r$, the projection $p_{i_1,\dots,i_k}(\overline{G})\leq \G_{g_{i_1}}\times \dots \times \G_{g_{i_k}}$ is virtually coabelian of even rank. 

\end{corollary}
\begin{proof}
 After passing to a finite index subgroup, we may assume that $\overline{G}$ is coabelian. Since $\overline{G}$ is finitely presented, it is also a full subdirect product. Now the same arguments as in the proof of Theorem \ref{thmNewCoab} show that $\phi$ is induced by a holomorphic map. The fact that projections of coabelian subgroups of direct products of groups are themselves virtually coabelian completes the proof.
\end{proof}

Corollary \ref{corNewCoabExs} is a direct consequence of Corollary \ref{rmkNewCoab}. More generally, we have 

\begin{corollary}
\label{corNewCoabExs2}
 With the same notation as in Corollary \ref{corNewCoabExs}, for $G=\ker \psi$ coabelian of odd rank and $G_1$ any finitely presented group, $H=G \times G_1$ is not K\"ahler. 
\end{corollary}
\begin{proof}
This is an immediate consequence of Corollary \ref{rmkNewCoab} applied to the projection $H\to G$.
\end{proof}

\begin{remark}
 Note that in particular Corollary \ref{corNewCoabExs2} applies to the direct product of any two full subdirect products of surface groups which are coabelian of odd rank. Thus, we can use Corollary \ref{corNewCoabExs2} to construct full subdirect products of surface groups, which are not K\"ahler and coabelian of even rank.
 \label{remNewCoabExs}
\end{remark}

This provides us with large classes of examples of non-K\"ahler subgroups of direct products of surface groups. As we will see in Section \ref{secSESCoab}, many of the examples in Corollary \ref{corNewCoabExs} share the property that they are non-K\"ahler for the elementary reason that their first Betti number is odd.

We want to emphasize that a particularly strong version of Corollary \ref{corNewCoabExs} holds in the case of a direct product of three surface groups.


\begin{theorem}
Let $X$ be a compact K\"ahler manifold and let $G=\pi_1 X$. Let $\psi: G \to  \G_{g_1}\times \G_{g_2}\times \G_{g_3}$ be a homomorphism such that the projection $p_i \circ \psi$  has finitely generated kernel for $1\leq i \leq r$ and the image is finitely presented. Then one of the following holds:
 \begin{enumerate}
  \item $G=\pi_1 R$ for $R$ a closed Riemann surface of genus $\geq 0$;
  \item $G= \ZZ^k$ for $k\in \left\{1,2,3\right\}$
  \item $G$ is virtually $\ZZ^k \times \Gamma_h$ for $h\geq 2$ and $k\in \left\{1,2\right\}$;
  \item $G$ is virtually a direct product $\ZZ^k\times \Gamma_{h_1}\times \Gamma_{h_2}$ for $h_1, h_2\geq 2$ and $k\in \left\{0,1\right\}$;
  \item $G$ is virtually coabelian of even rank.
 \end{enumerate}
 \label{thm3Factor}
\end{theorem}
\begin{proof}
 Since centralizers in surface groups are cyclic, every free abelian subgroup of $G$ has rank $\leq 3$. We will distinguish cases, making repeated use of Lemma \ref{lemSurCyc}. If $G$ has cyclic projection to all factors, then $G$ is abelian and thus (2) holds. Hence, we may assume that $G$ is not abelian. Assume first that $G$ is not a full subgroup. After projecting away from factors which have trivial intersection with $G$, we either obtain that $G$ is a subgroup of a surface group in which case (1) holds, or a full subgroup of a direct product of two surface groups. If the latter holds then either $G$ is subdirect in a direct product $\Gamma_{\g_1}\times \Gamma_{\g_2}$ and the VSP property implies that (4) holds, or (3) holds with $k=1$. If $G$ is full and not subdirect in $\G_{\g_1}\times \G_{g_2}\times \G_{\g_3}$ then we must also have (3) with $k=2$ or (4). Else $G$ must be full subdirect (after passing to finite index subgroups of the $\G_{g_i}$) and it follows from finite presentability, Proposition \ref{propNewCoab}, and Remark \ref{rmkNewCoab1} that (5) holds.
\end{proof}

\begin{remark}
Note that the assumption that $p_i\circ \psi$ has finitely generated kernel in Theorem \ref{thm3Factor} can be replaced by the assumption that $p_i \circ \psi$ has either cyclic image or is induced by a holomorphic map for $1\leq i \leq r$ would suffice, as we could then apply Proposition \ref{propNewCoab} directly to obtain the conclusion. 
\end{remark}

We obtain a constraint on K\"ahler subgroups of direct products of three surface groups.

\begin{corollary}
 Let $G\leq \G_{g_1}\times \G_{g_2}\times \G_{g_3}$ be K\"ahler. Then one of the following holds:
 \begin{enumerate}
  \item $G=\pi_1 R$ for $R$ a closed Riemann surface of genus $\geq 0$;
  \item $G$ is virtually a direct product $\Gamma_{h_1}\times \Gamma_{h_2}$ for $h_1, h_2\geq 2$;
  \item $G$ is virtually $\ZZ^2 \times \Gamma_h$ for $h\geq 2$;
  \item $G$ is virtually coabelian of even rank.
 \end{enumerate}
 \label{cor3Factor}
\end{corollary}

\begin{proof}
This is a direct consequence of Theorem \ref{thm3Factor} and the fact that the first Betti number of a K\"ahler group is even.
\end{proof}




