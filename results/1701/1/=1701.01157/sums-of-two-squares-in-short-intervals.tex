%; whizzy document
%%%%%%%%%%%%%%%%%%%%%%%%%%%%%%%%%%%%%%%%%%%%%%%%%%%%%%%%%%%%%%%%%%
%%%%%%%%%%%%%%%%%%%%%%%%%%%  PREAMBLE  %%%%%%%%%%%%%%%%%%%%%%%%%%%
%%%%%%%%%%%%%%%%%%%%%%%%%%%%%%%%%%%%%%%%%%%%%%%%%%%%%%%%%%%%%%%%%%

\documentclass[12pt, reqno, twoside, letterpaper]{amsart}
% \usepackage[fixmeSHOW, 
%             nixSHOW,
%             nixnixSHOW,
%             commentsSHOW,
%             jetsamSHOW,
%             bibnixSHOW,
%             arXivOFF]{sums-of-two-squares-in-short-intervals}

\usepackage[fixmeHIDE, 
            nixHIDE,
            nixnixHIDE,
            nixaltHIDE,
            commentsHIDE,
            jetsamHIDE,
            bibnixHIDE,
            arXivON]{sums-of-two-squares-in-short-intervals}

%%%%%%%%%%%%%%%%%%%%%%%%%%%%%%%%%%%%%%%%%%%%%%%%%%%%%%%%%%%%%%%%%%
%%%%%%%%%%%%%%%%%%%%%%%%%%% TITLE  ETC %%%%%%%%%%%%%%%%%%%%%%%%%%%
%%%%%%%%%%%%%%%%%%%%%%%%%%%%%%%%%%%%%%%%%%%%%%%%%%%%%%%%%%%%%%%%%%

\title[Poisson spacings between sums of two squares]
      {Poisson distribution for gaps between sums of two
       squares and level spacings for toral point scatterers}

\author[T.\ Freiberg]{Tristan Freiberg}

\address{Department of Pure Mathematics, 
         University of Waterloo, 
         Waterloo ON, CANADA.}

\email{tfreiberg@uwaterloo.ca}

\author[P.\ Kurlberg]{P\"ar Kurlberg}

\address{Department of Mathematics, 
         KTH Royal Institute of Technology, 
         Stockholm, SWEDEN.}

\email{kurlberg@kth.se}

\author[L.\ Rosenzweig]{Lior Rosenzweig}

\address{Department of Mathematics, 
         ORT Braude College, 
         Karmiel, ISRAEL.}

\email{liorr@braude.ac.il}

\date{\today}

%%%%%%%%%%%%%%%%%%%%%%%%%%%%%%%%%%%%%%%%%%%%%%%%%%%%%%%%%%%%%%%%%%
%%%%%%%%%%%%%%%%%%%%%%%%%%%  DOCUMENT  %%%%%%%%%%%%%%%%%%%%%%%%%%%
%%%%%%%%%%%%%%%%%%%%%%%%%%%%%%%%%%%%%%%%%%%%%%%%%%%%%%%%%%%%%%%%%%

\begin{document}

%%%%%%%%%%%%%%%%%%%%%%%%%%%%%%%%%%%%%%%%%%%%%%%%%%%%%%%%%%%%%%%%%%
%%%%%%%%%%%%%%%%%%%%%%%%%%%% ABSTRACT %%%%%%%%%%%%%%%%%%%%%%%%%%%%
%%%%%%%%%%%%%%%%%%%%%%%%%%%%%%%%%%%%%%%%%%%%%%%%%%%%%%%%%%%%%%%%%%

\begin{abstract} 
%
We investigate the level spacing distribution for the quantum
spectrum of the square billiard.  
%
Extending work of Connors--Keating, and Smilansky, we formulate 
an analog of the Hardy--Littlewood prime $k$-tuple conjecture 
for sums of two squares, and show that it implies that the 
spectral gaps, after removing degeneracies and rescaling, are 
Poisson distributed.  
%
Consequently, by work of Rudnick and Uebersch\"ar, the level 
spacings of arithmetic toral point scatterers, in the weak 
coupling limit, are also Poisson distributed.
%
We also give numerical evidence for the conjecture and its 
implications.
%
\end{abstract}

\maketitle

%%%%%%%%%%%%%%%%%%%%%%%%%%%%%%%%%%%%%%%%%%%%%%%%%%%%%%%%%%%%%%%%%%
%%%%%%%%%%%%%%%%%%%%%%%%%%%% SECTION 01 %%%%%%%%%%%%%%%%%%%%%%%%%%
%%%%%%%%%%%%%%%%%%%%%%%%%%%%%%%%%%%%%%%%%%%%%%%%%%%%%%%%%%%%%%%%%%

\section{Introduction}
 \label{sec:intro}

According to the Berry--Tabor conjecture \cite{BT:77}, the energy 
levels for generic integrable systems should be Poisson 
distributed in the semiclassical limit.  
%
As noted by Connors and Keating \cite{CK:97}, the square billiard, 
though integrable, is not generic: due to spectral degeneracies, 
the level spacing distribution tends to a $\delta$-function at 
zero.
%
However, if we remove the degeneracies and rescale so that the 
mean spacing is unity, numerics indicate Poisson spacings.

\begin{figure}[ht]
  \centering
\includegraphics[width=7.5cm]{plot-4}

\caption{%
  Rescaled gaps between consecutive energy levels in
  $[10^{99}, 10^{99} + 110000]$, after removing degeneracies. 
%Bins: $\protect\sage{bins}$.
  %
  The rescaled gaps have mean one; without rescaling
  the mean gap is $19.42\cdots$.
  %
  Number of gaps: $5663$.  
  %
  We also plot the density function (red in color
  printout) $P(x) = \e^{-x}$, consistent with Poisson spacings.
  }
\end{figure}

The energy levels of the square billiard, say with side length 
$2\pi$, are number theoretical in nature, and given by 
$a^2 + b^2$ for $a,b \in \ZZ$.
%
After removing degeneracies and rescaling, we are led to study 
the nearest neighbor spacing distribution 

\begin{equation}
 \label{eq:nnsd}
 \frac{1}{\speccount(x)}
  \#
   \bigg\{
    \sts_{n} \le x : \frac{\sts_{n + 1} - \sts_{n}}{x/\speccount(x)} < \lambda 
   \bigg\} 
\end{equation}
(as $x \to \infty$), where $\sts_n$ denotes the $n$th smallest 
element of the set  
\begin{equation}
 \label{eq:defSSN}
 \SS \defeq \{a^2 + b^2 : a,b \in \ZZ\}, 
  \quad 
   \text{and}
    \quad 
     \speccount(x)
      \defeq 
       \#\{\sts_{n} \le x : \sts_n \in \SS\}.
\end{equation}
%
(In our setting the leading order of the density of states is
asymptotically equal to $C/\sqrt{\log x}$ as $x \to \infty$ [cf.\   
\eqref{eq:sotsnt}], and hence the spacing distribution of the 
unfolded levels $\big(C\sts_n/\sqrt{\log \sts_n}\big)_{n \ge 1}$ 
has the same asymptotic distribution as the gaps in 
\eqref{eq:nnsd}.)

%
\begin{nixnix}
% 
This is the probability that, given $\sts_n \le x$, there is some 
integer $b > \lambda x/\speccount(x)$ such that 
\[
\ind{\SS}(\sts_n + b)
 \prod_{j = 1}^{b - 1} 
  \big(1 - \ind{\SS}(\sts_n + j)\big)
   =
    1,
\]
where $\ind{\SS}$ is the indicator function of $\SS$.

If the positions of the levels $\sts_n$, $n \le x$, were 
uncorrelated, this probability would be 
\[
\sum_{b > \lambda x/\speccount(x)}
 \frac{\speccount(x)}{x}
  \bigg(1 - \frac{\speccount(x)}{x}\bigg)^{b - 1},
\]
which, viewed as a Riemann sum approximation to the integral 
$\int_{\lambda}^{\infty} \e^{-t} \dd{t}$, would suggest that 
the nearest neighbor spacing distribution does indeed follow 
a Poisson law, i.e.\ that 
\[
\frac{1}{\speccount(x)}
 \#\bigg\{\sts_{n} \le x : \frac{\sts_{n + 1} - \sts_{n}}{x/\speccount(x)} > \lambda \bigg\}
  \sim 
   \e^{-\lambda}
    \quad 
     (x \to \infty).
\]

If the positions of the levels $\sts_n$, $\sts_n \le x$, were 
uncorrelated, this would indeed follow a Poisson law.

However, being sums of two squares, the levels are not 
uncorrelated.

For instance, four consecutive integers cannot all be in $\SS$, 
since one of them is congruent to $3$ modulo $4$.

Nevertheless, taking into account correlations modulo all prime 
powers, we still (conjecturally) expect \eqref{eq:nnsd} to follow 
a Poisson law, consistent with numerics.
%
\end{nixnix}
%

Rather than studying the spacing distribution directly, we shall
proceed by investigating {\em unordered} $k$-tuples of elements in 
$\SS$.
%
Thus, given $k \ge 1$ and $\bh = \{h_1,\ldots,h_k\} \subseteq \ZZ$ 
with $\card \bh = k$, consider the correlation function
\begin{equation}
 \label{eq:defklevcor}
 R_k(\bh;x)
  \defeq 
   \frac{1}{x}
    \sum_{n \le x}
     \ind{\SS}(n + h_1)
      \cdots 
       \ind{\SS}(n + h_k),
\end{equation}
where $\ind{\SS}$ denotes the indicator function of $\SS$.
%
If $\bh = \{0\}$, this is the level density 
\begin{equation}
 \label{eq:deflevden}
 R_1(x) 
  \defeq 
   \frac{\speccount(x)}{x}.
\end{equation}
%
By a classical result of Landau \cite{LAN:08},
\begin{equation}
 \label{eq:sotsnt}
   R_1(x) 
    \sim 
     \frac{C}{\sqrt{\log x}}
      \quad 
    (x \to \infty),
\end{equation}
where $C > 0$ is an explicitly given constant (see 
\eqref{eq:defLanRamconst}).
%
%
To formulate an analog of (\ref{eq:sotsnt}) for $k > 1$ we need 
some further notation.  
%
Given a prime $p \not \equiv 1 \bmod 4$, define%
\begin{equation}
 \label{eq:delthp}
  \delta_{\bh}(p)
   \defeq 
    \lim_{\alpha \to \infty}
     \frac{
      \#\{0 \le a < p^{\alpha} : 
             \forall h \in \bh, 
              a + h \equiv \sots \bmod p^{\alpha}\}
          }
          {p^{\alpha}}.
\end{equation}
%
(That the limit exists is shown in Section \ref{sec:prelims}, cf.\  
Propositions \ref{prop:Sp3h} and \ref{prop:S2h}.)
%
Further, for $k \ge 1$ and a set 
$\bh = \{h_1,\ldots,h_k\} \subseteq \ZZ$ with $\card \bh = k$, we 
define the {\em singular series} for $\bh$ by  
\begin{equation}
 \label{eq:defsssP}
  \mathfrak{S}_{\bh}
   \defeq 
    \prod_{p \not\equiv 1 \bmod 4}
     \frac{\delta_{\bh}(p)}{\big(\delta_{\bz}(p)\big)^k},
\end{equation}
with $\delta_{\{0\}}(p)$ and $\delta_{\bh}(p)$ as in
\eqref{eq:delthp}.
%
We note that $\delta_{\{0\}}(p) > 0$ for all 
$p \not\equiv 1 \bmod 4$, and that the  product converges to a 
nonzero limit if $\delta_{\bh}(p) > 0$ for all 
$p \not \equiv 1 \bmod 4$ (cf.\ Proposition \ref{prop:sssc}).
%
If $\delta_{\bh}(p) = 0$ for some $p \not\equiv 1 \bmod 4$, we 
define $\mathfrak{S}_{\bh}$ to be zero;
%we  remark that
it is easy to see that
$R_k(\bh;x) = 0$ for all $x$ if $\mathfrak{S}_{\bh} = 0$.

We can now formulate an analog of the Hardy--Littlewood prime 
$k$-tuple conjecture.
%
\begin{conjecture}
 \label{con:sotsktups}
%
Fix $k \ge 1$, and a set $\bh = \{h_1,\ldots,h_k\} \subseteq \ZZ$ 
with $\card \bh = k$. 
%
If $\mathfrak{S}_{\bh} > 0$, then
\begin{equation}
 \label{eq:sotsktups}
  R_k(\bh;x)
    \sim 
     \mathfrak{S}_{\bh}
      \big(R_1(x)\big)^k 
       \quad      
     (x \to \infty).
\end{equation}
\end{conjecture}
%
\noindent 
Our main result, Theorem \ref{thm:main} below, is conditional on 
the hypothesis that \eqref{eq:sotsktups} holds on average.
%
To be precise, let $\cE_{\bh}(x)$ be defined by the relation 
\begin{equation}
 \label{eq:defEterm}
   R_k(\bh;x)
    \eqdef
     \big(\mathfrak{S}_{\bh} + \cE_{\bh}(x)\big)
      \big(R_1(x)\big)^k.
\end{equation}
%
Further, let $\Delta^k$ be the region in $\RR^k$ defined by  
\begin{equation}
 \label{eq:defsimplex}
  \Delta^k
   \defeq 
    \{(x_1,\ldots,x_k) \in \RR^k : 0 < x_1 < \cdots < x_k\},
\end{equation}
%
and, given $\sC \subseteq \Delta^k$ and $y \in \RR$, let $y\sC$ 
be the dilation of $\sC$ defined by  
\[
 y\sC \defeq \{(yx_1,\ldots,yx_k) : (x_1,\ldots,x_k) \in \sC\}.
\]
%
% Conjecture \ref{con:sotsktups} asserts that, for a {\em fixed} 
% set $\bh$ such that $\mathfrak{S}_{\bh} > 0$, we have 
% $|\cE_{\bh}(x)| \to 0$ as $x \to \infty$.
%
% Further, if $\mathfrak{S}_{\bh} = 0$, then $\cE_{\bh}(x) = 0$ for 
% all $x \ge 1$.
%
Our hypothesis is that the error term $\cE_{\bh}(x)$ is small when 
averaged over dilates of certain bounded convex subsets.

\begin{hypothesis}[$k,\sC,\bo$]
%
Fix an integer $k \ge 1$ and a bounded convex set 
$\sC \subseteq \Delta^k$.
%
Set $\bo \defeq \emptyset$ or set $\bo \defeq \{0\}$.
%
Let $x$ and $y$ be real parameters tending to infinity in such a 
way that $yR_1(x) \sim 1$.
%
There exists a function $\varepsilon(x)$, with 
$\varepsilon(x) \to 0$ as $x \to \infty$, such that for $x$ 
sufficiently large in terms of $k$ and $\sC$,
\begin{equation}
 \label{eq:hyp}
   \bigg|
    \sum_{(h_1,\ldots,h_k) \in y\sC \cap \, \ZZ^k}
     \cE_{\bo \cup \bh}(x)
   \bigg|
 \le 
  \varepsilon(x)
   \sum_{(h_1,\ldots,h_k) \in y\sC \cap \, \ZZ^k}
    \mathfrak{S}_{\bo \cup \bh},
\end{equation}
where $\bh = \{h_1,\ldots,h_k\}$ in both summands. 
\end{hypothesis}

Under the above hypothesis we find that the spacing distribution 
\eqref{eq:nnsd} is indeed Poissonian.
%
Moreover, the distribution of the number of points in intervals of 
size comparable to the mean spacing is consistent with that of a 
Poisson process.
%
(We remark that our hypothesis can be weakened slightly --- see 
Section \ref{sec:main}.)

\begin{theorem}
 \label{thm:main}
%
Let $x$ and $y$ be real parameters tending to infinity in such a 
way that $yR_1(x) \sim 1$.
%
Fix integers $m \ge 0$ and $r \ge 1$, and fix 
$\lambda,\lambda_1,\ldots,\lambda_r \in \RR^{+}$.
%
Assume that Hypothesis
\textup{(}$k,\sC,\{0\}$\textup{)}\textup{)}
\textup{(}respectively, 
Hypothesis
\textup{(}$k,\sC,\emptyset$\textup{)}
holds 
for all $k \ge 1$, and all bounded, convex sets 
$\sC \subseteq \Delta^k$.
%
Then \textup{(}a\textup{)} \textup{(}respectively, 
\textup{(}b\textup{)}\textup{)} holds.


\textup{(}a\textup{)} We have  
%
\begin{equation}
 \label{eq:thm:mainc}
  \frac{1}{\speccount(x)}
   \#\{\sts_n \le x : \forall j \le r, \sts_{n + j} - \sts_{n + j - 1} \le \lambda_j y\}
    \sim 
     \prod_{j = 1}^r 
      \int_0^{\lambda_j} \e^{-t} \dd{t} %(1 - \e^{-\lambda_j})
       \quad 
     (x \to \infty).
\end{equation}

\textup{(}b\textup{)} We have 
%
\begin{equation}
 \label{eq:thm:main}
  \frac{1}{x}
   \#\{n \le x : \speccount(n + \lambda y) - \speccount(n) = m\}
%    \sums[n \le x][{\# \SS \cap (n,n + \lambda y] = m}] 1
    \sim 
     \e^{-\lambda}\frac{\lambda^m}{m!}
       \quad     
     (x \to \infty).
\end{equation}


\end{theorem}
In \cite{RU:14}, Rudnick and Uebersch\"ar considered the spectrum 
of ``toral point scatterers'', namely the Laplace operator, 
perturbed by a delta potential, on two dimensional tori.  
%
They showed that the level spacings of the perturbed eigenvalues, 
in the weak coupling limit, have the same distribution as the 
level spacings of the unperturbed eigenvalues (after removing 
multiplicities).
%
An interesting consequence of Conjecture \ref{con:sotsktups} (or 
\eqref{eq:hyp}) is thus that the Berry--Tabor conjecture holds for 
toral point scatterers, in the weak coupling limit, for arithmetic 
tori of the form $\RR^2/\ZZ^2$.

We remark that Gallagher \cite{GAL:76} proved the analog of 
Theorem \ref{thm:main} (b) for primes.
%
Just as in his proof, a key technical result is that the singular 
series is of average order one, over certain geometric regions.

\begin{proposition}
 \label{prop:sssa}
%
Fix an integer $k \ge 1$, and a bounded convex set 
$\sC \subseteq \Delta^k$.
%
Set $\bo \defeq \emptyset$ or set $\bo \defeq \{0\}$.
%
As $y \to \infty$, we have  
\begin{align}
 \label{eq:sssa}
  \sum_{(h_1,\ldots,h_k) \in y\sC \cap \, \ZZ^k} 
   \mathfrak{S}_{\bo \cup \bh} 
   & =
    y^k \Big( \vol(\sC) + O\big(y^{-2/3 + o(1)}\big)\Big), 
\end{align}
where $\bh = \{h_1,\ldots,h_k\}$ in the summand, and $\vol$ 
stands for volume in $\RR^k$.
\end{proposition}

{\bfseries Acknowledgements.} 
%
We thank Z.\ Rudnick for stimulating discussions on the subject 
matter, and D.\ Koukoulopoulos for his comments on an early 
version of the paper.
%
T.\ F.\ was partially supported by a grant from the G\"oran 
Gustafsson Foundation for Research in Natural Sciences and 
Medicine.
%
P.\ K.\ and L.\ R.\  were partially supported by grants from the 
G\"oran Gustafsson Foundation for Research in Natural Sciences and 
Medicine, and the Swedish Research Council (621-2011-5498).
%
L.\ R. wishes to thank and acknowledge the Mathematics department at KTH, being his home institute during the period where most of the work on this paper was done.

%%%%%%%%%%%%%%%%%%%%%%%%%%%%%%%%%%%%%%%%%%%%%%%%%%%%%%%%%%%%%%%%%%
%%%%%%%%%%%%%%%%%%%%%%%%%%%% SECTION 02 %%%%%%%%%%%%%%%%%%%%%%%%%%
%%%%%%%%%%%%%%%%%%%%%%%%%%%%%%%%%%%%%%%%%%%%%%%%%%%%%%%%%%%%%%%%%%

\section{Discussion}
\label{sec:discussion}

Connors and Keating \cite{CK:97} determined the singular series 
for shifted pairs of sums of two squares and gave a probabilistic
derivation of Conjecture \ref{con:sotsktups} for $k = 2$, and 
found that it matched numerics quite well (to within $2\%$).
%
Smilansky \cite{SMI:13} then expressed the singular series for 
pairs as products of $p$-adic densities, and showed that its mean 
value (over short intervals of shifts) is consistent with a 
Poisson distribution, and that the same is true for sums of two 
squares, on assuming a uniform version of 
Conjecture \ref{con:sotsktups} for $k = 2$.  
%
He also determined the singular series for triples corresponding 
to the shifts $\bh = \{0,1,2\}$.
%

As already mentioned, the analog of Theorem \ref{thm:main} (b) for 
primes is due to Gallagher; in \cite{GAL:76} he showed that an 
appropriate form of the Hardy--Littlewood prime $k$-tuples 
conjecture implies the prime analog of \eqref{eq:thm:mainc}.
%
(That it implies the prime analog of \eqref{eq:thm:main} is 
mentioned in Hooley's survey article \cite[p.\ 137]{HOO:72}.)
%
To show that the singular series is one on average (i.e., the 
prime analog of Proposition \ref{prop:sssa}), Gallagher uses 
combinatorial identities for Stirling numbers of the second kind.
%
In \cite{KOW:11}, Kowalski developed an elegant probabilistic
framework for evaluating averages of singular series.  
%
Rather than using combinatorial identities, he showed that a 
certain duality between $k$-th moments of $m$-tuples and 
$m$-th moments of $k$-tuples holds 
(cf.\ \cite[Theorem 1]{KOW:11}).  
%
That the $k$-th moment of $1$-tuples equals one is essentially 
trivial; by duality he obtains the non-trivial consequence that 
first moments of $k$-tuples also equals one.
%
(Note that \eqref{eq:sssa} can be viewed as a first moment of 
$k$-tuples when $\bo = \emptyset$.)  

Our approach originates with techniques developed in
\cite{KR:99, KUR:00}, and further refined in \cite{GK:08, KUR:09}.
%
Loosely speaking, the singular series $\mathfrak{S}_{\bh}$ is 
expanded into local factors of the form $1 + \epsilon_{\bh}(p)$,
and thus
\[
 \mathfrak{S}_{\bh} = \prod_{p} (1 + \epsilon_{\bh}(p))
 =
  \sums[d \ge 1][\text{squarefree}] \epsilon_{\bh}(d),
\]
where $\epsilon_{\bh}(1) = 1$ and 
$\epsilon_{\bh}(d) \defeq \prod_{p|d} \epsilon_{\bh}(p)$.  
%
Hence
\[
 \sum_{\bh}
  \mathfrak{S}_{\bh} 
   =
    \sums[d \ge 1][\text{squarefree}]
     \sum_{\bh}
      \epsilon_{\bh}(d),
\]
and the main term is given by $d = 1$.  
%
For $d$ large, $|\epsilon_{\bh}(d)|$ can be shown to be small on 
average.
%
For $d$ small, we use that $\epsilon_{\bh}(d)$ (approximately) 
only depends on $\bh \bmod d$, together with complete cancellation 
when summing over the {\em full} set of residues modulo $d$, i.e.,
$\sum_{\bh \bmod d} \epsilon_{\bh}(d) = 0$.  
%
This follows, via the Chinese remainder theorem, from local
cancellations
$ 
\sum_{\bh \bmod p} \epsilon_{\bh}(p) = 0
$, 
which in turn can be deduced from the following easily verifiable
identity: given {\em any} subset $X_p \subseteq \ZZ/p\ZZ$, we have
(cf.\ Lemma \ref{lem:cancel} (b) and its proof for more details):
\[
 \sum_{(h_1, h_2, \ldots, h_k) \in (\ZZ/p\ZZ)^k} 
  \#\{ m \in \ZZ/p\ZZ : m + h_1, m + h_2, \ldots, m + h_k \in X_p\}
   =
    \big(\#X_p\big)^{k}.
\]

However, unlike the setup in \cite{KR:99,GK:08,KUR:09}, where the
local error terms $\epsilon_{\bh}(p)$ are determined by
$\bh \bmod p$, in the current setting the image of 
$\bh \bmod p^{\alpha}$, for any fixed $\alpha$, is not sufficient 
to determine $\epsilon_{\bh}(p)$.  
%
On the other hand, the function $\bh \to \epsilon_{\bh}(p)$ has 
nice $p$-adic regularity properties, allowing us to approximate
$\epsilon_{\bh}(p)$ by truncations 
$\epsilon_{\bh}(p^{\alpha})$ such that 
$\epsilon_{\bh}(p^{\alpha})$ only depends on 
$\bh \bmod p^{\alpha}$, and 
$
 \epsilon_{\bh}(p) - \epsilon_{\bh}(p^{\alpha}) 
  \ll 
   1/p^{\alpha - 1}
$ 
for all $\bh$.  
%
Apart from making the arguments more complicated, we also get
a weaker error term: if $\epsilon_{\bh}(p)$ only depended on
$\bh \bmod p$, we would get a relative error of size 
$y^{-1 + o(1)}$, rather than $y^{-2/3 + o(1)}$.
%
We also note that David, Koukoulopoulos and Smith \cite{DKS:15}, 
in studying statistics of elliptic curves, have developed quite 
general methods for finding asymptotics of weighted sums
$\sum_{\bh} w_{\bh} \mathfrak{S}_{\bh}$, provided that the local
factors have $p$-adic regularity properties similar to the ones
above. 
%
In fact, Proposition \ref{prop:sssa}, though with a weaker error 
term, can be deduced from \cite[Theorem 4.2]{DKS:15}.

We finally remark that the corresponding question in the function 
field setting is better understood --- Bary--Soroker and Fehm 
\cite{BS-F} recently showed that the sums of squares analog of the 
$k$-tuple conjecture holds in the large $q$-limit for the 
function field setting (e.g., replacing $\mathbb{Z}$ by 
$\mathbb{F}_{q}[T]$ and $\mathbb{Z}[i]$ by 
$\mathbb{F}_q[\sqrt{-T}]$).


\subsection{Evidence towards Conjecture \ref{con:sotsktups}.} 
 \label{sec:sotsHLktups}

We begin by formulating a qualitative version of 
Conjecture \ref{con:sotsktups}.
%
\begin{conjecture} 
 \label{con:qualktups}
Fix $k \ge 1$, and a set 
$\bh = \{h_1,\ldots,h_k\} \subseteq \ZZ$ with $\card \bh = k$. 
%
If $\mathfrak{S}_{\bh} > 0$, then there exist infinitely many 
integers $n$ such that $n + \bh \subseteq \SS$.
\end{conjecture}
%
\noindent
% 
We  remark that whether or not $\mathfrak{S}_{\bh} > 0$ can be 
determined by a finite computation: this follows from 
Propositions \ref{prop:S2h} and \ref{prop:Sp3h}. 
%
%****************************************************************%
%************************* START DETAIL *************************%
%****************************************************************%
%
\begin{nixnix}
%
For recall that $\bh$ is $\SS$-admissible if and only if 
$\delta_{\bh}(p) > 0$ for all $p$.
%
(See Propositions \ref{prop:S2h} and \ref{prop:Sp3h} et seq.)
%
For $p \nmid \det(\bh)$ we have $\delta_{\bh}(p) > 0$ by 
Propositions \ref{prop:S2h} (c) and \ref{prop:Sp3h} (c).
%
For the finitely many $p$ dividing $\det(\bh)$, we have 
$\delta_{\bh}(p) > 0$ if and only if either 
$\card \bh_p \ne \emptyset$ or, in the case $p \equiv 3 \bmod 4$, 
$\V_{\bh}(p^{\alpha}) \ne \emptyset$ for 
$\alpha = 1 + \max_{i \ne j} \nu_p(h_i - h_j)$, in the case 
$p = 2$, $\T_{\bh}(2^{\alpha + 1}) \ne \emptyset$ for 
$\alpha = 2 + \max_{i \ne j} \nu_2(h_i - h_j)$.
%
(See Propositions \ref{prop:S2h} (b) and \ref{prop:Sp3h} (b).)
%
\end{nixnix}
%
%****************************************************************%
%************************** END DETAIL **************************%
%****************************************************************%
%
%
Examples of sets $\bh$ for which $\mathfrak{S}_{\bh} = 0$ are
$\{0,1,2,3\}$ and $\{0,1,2,4,5,8,16,21\}$: any translate of 
$\{0,1,2,3\}$ contains an integer congruent to $3$ modulo $4$, and 
hence $\delta_{\bh}(2) = 0$; any translate of 
$\{0,1,2,4,5,8,16,21\}$ contains an integer congruent to $3$ or 
$6$ modulo $9$, and hence $\delta_{\bh}(3) = 0$.

It is possible to show that $\mathfrak{S}_{\bh} > 0$ for {\em any} 
set $\bh$ containing at most three integers.
%
The question of whether, for any $h_1,h_2,h_3 \in \ZZ$, we have 
$n + \{h_1,h_2,h_3\} \subseteq \SS$ for infinitely many $n$, was 
apparently raised by Littlewood: Hooley \cite{HOO:73} showed, 
using the theory of ternary quadratic forms, that 
Conjecture \ref{con:qualktups} indeed holds for $k \le 3$.
%
The conjecture remains open for $k \ge 4$.

For fixed $k \ge 1$ and $\bh = \{h_1,\ldots,h_k\}$ with 
$\card \bh = k$, the upper bound 
\[
 \sum_{n \le x}
  \ind{\SS}(n + h_1)\cdots \ind{\SS}(n + h_k)
   \ll_k
    \frac{x}{(\log x)^{k/2}}
     \prods[p \equiv 3 \bmod 4][p \mid h_j - h_j][\text{some $i < j$}]
      \bigg(1 + \frac{k}{p}\bigg),
\]
can be deduced from Selberg's sieve (see \cite{SEL:77}), which is 
of the correct order of magnitude, according to 
Conjecture \ref{con:sotsktups}.
%
The special case $\bh = \{0,1\}$ is due to Rieger \cite{RIE:65}; 
the special case $\bh = \{0,1,2\}$ is due to Cochrane and Dressler 
\cite{CD:87}; the general case is due to Nowak \cite{NOW:05}.

Lower bounds are more subtle.  
%
For $k = 2$, Hooley \cite{HOO:74} and Indlekofer \cite{IND:74} 
showed that, for any nonzero integer $h$, 
\[
  \sum_{n \le x} \ind{\SS}(n)\ind{\SS}(n + h)
   \gg
    \frac{x}{\log x}
     \prods[p \mid h][p \equiv 3 \bmod 4]
      \bigg(1 + \frac{1}{p}\bigg),
\]
but we are not aware of any such bounds for $k \ge 3$.

We remark that Iwaniec deduced the asymptotic
$
 \sum_{n \le x} \ind{\SS}(n)\ind{\SS}(n + 1) \sim 3x/(8\log x)
$, 
as $x \to \infty$, from an analog of the Elliott--Halberstam 
conjecture for sums of two squares 
(cf.\ \cite[Corollary 2, (2.3)]{IWA:76}).
%
However, note that the leading term constant $3/8$ disagrees with 
the one due to Connors and Keating \cite{CK:97}, namely $1/2$.  
%
(We also obtain the constant $1/2$; see Figure \ref{fig:table1} 
below for a numerical comparison.)

\subsection{Numerical evidence}
\label{sec:numerical-evidence}
%
Using Propositions \ref{prop:S2h} (b), (c) and 
\ref{prop:Sp3h} (b), (c), we can give $\mathfrak{S}_{\bh}$ 
explicitly, as in the following examples.
%
Let us first record that the constant $C$ in \eqref{eq:sotsnt} is 
the Landau--Ramanujan constant, given by 
\begin{equation}
 \label{eq:defLanRamconst}
  C
   \defeq 
    \frac{1}{\sqrt{2}}
     \prod_{p \equiv 3 \bmod 4}
      \bigg(
       1 - \frac{1}{p^2}
      \bigg)^{-1/2}
       =
        0.764223\ldots.
\end{equation}
%
It is straightforward to verify that   
\begin{equation}
 \label{eq:sss01}
  \mathfrak{S}_{\{0,1\}}
   =
    \frac{1}{2C^2} 
     =
      0.856108\ldots.
\end{equation}
%
If \eqref{eq:sotsktups} holds with $\bh = \{0,1\}$ then, by 
\eqref{eq:sotsnt} and \eqref{eq:sss01}, 
\[
 \speccount(\{0,1\}; x)
  \defeq 
   \sum_{n \le x} \ind{\SS}(n)\ind{\SS}(n + 1)
    \sim 
     \frac{x}{2C^2}
      \big(R_1(x)\big)^2
       \sim 
        \frac{x}{2\log x}
         \quad 
       (x \to \infty).
\]
The agreement with numerics is quite good (to within $1 \%$).
%
\begin{figure}[ht]
%
\begin{tabular}{|r|r|r|r|}
\hline 
$x$ & {\small $\speccount(\{0,1\}; x)$ } & {\small $x\mathfrak{S}_{\{0,1\}}(R_1(x))^2$ } & Ratio \\
\hline
1000000000  &  25927011 &   25690391.1 & 1.00921  \\
2000000000  &  50042411 &   49603435.5 & 1.00885  \\
3000000000  &  73560246 &   72930222.0 & 1.00864  \\
4000000000  &  96705170 &   95891759.7 & 1.00848  \\
5000000000  & 119584162 &  118589346.3 & 1.00839  \\
6000000000  & 142253331 &  141080935.2 & 1.00831  \\
7000000000  & 164749254 &  163403937.1 & 1.00823  \\
8000000000  & 187100631 &  185584673.5 & 1.00817  \\
9000000000  & 209327440 &  207642640.3 & 1.00811  \\
\hline
\end{tabular}   
%
\caption{Observed data vs prediction for $\bh = \{0,1\}$. }
%
\label{fig:table1}
%
\end{figure}


As the simplest example with $k = 3$, we verify that 
\[
  \mathfrak{S}_{\{0,1,2\}}
   =
    \frac{A}{4C^2}, 
     \quad 
      A 
      \defeq \prod_{p \equiv 3 \bmod 4}
       \bigg(1 - \frac{2}{p(p - 1)}\bigg),
\]
so Conjecture \ref{con:sotsktups} implies that 
\[
  \speccount(\{0,1,2\}; x)
   \defeq 
    \sum_{n \le x} \ind{\SS}(n)\ind{\SS}(n + 1)\ind{\SS}(n + 2)
     \sim 
      \frac{Ax}{4C^2}
       \big(R_1(x)\big)^3
        \sim 
         \frac{ACx}{4(\log x)^{3/2}}
\]
as $x \to \infty$.
%
Here, the agreement between numerics and model is only 
to within $10 \%$.

\begin{figure}[ht]
\begin{tabular}{|r|r|r|r|}
\hline
$x$ & {\small $\speccount(\{0,1,2\}; x)$ } & {\small $x\mathfrak{S}_{\{0,1,2\}}(R_1(x))^3$ } & Ratio \\
\hline
1000000000  &  1490691 &   1362419.3 & 1.09415  \\
2000000000  &  2818128 &   2584683.5 & 1.09032  \\
3000000000  &  4093602 &   3762317.2 & 1.08805  \\
4000000000  &  5338091 &   4912433.3 & 1.08665  \\
5000000000  &  6560430 &   6042800.3 & 1.08566  \\
6000000000  &  7764604 &   7157833.6 & 1.08477  \\
7000000000  &  8954282 &   8260369.7 & 1.08400  \\
8000000000  & 10132295 &   9352396.2 & 1.08339  \\
9000000000  & 11299877 &  10435380.5 & 1.08284  \\
\hline
\end{tabular}  
%
\caption{Observed data vs prediction for $\bh=\{0,1,2\}$. }
%
\label{fig:table2}
%
\end{figure}

%
\begin{nixnix}
%
\section{Motivating the probabilistic model}
%
It is elementary to show (see Section \ref{sec:prelims}) that 
%
\begin{equation}
 \label{eq:defSp}
 \SS = \bigcap_{p \not\equiv 1 \bmod 4} S_p,
  \quad 
   \text{where}
    \quad  
 S_p 
  \defeq 
   \bigcap_{\alpha = 1}^{\infty}
    \{n \in \ZZ : n \equiv \sots \bmod p^{\alpha}\},
\end{equation}
the first intersection being over all primes  
$p \not\equiv 1 \bmod 4$, and $n \equiv \sots \bmod p^{\alpha}$ 
denoting that $n \equiv \sts \bmod p^{\alpha}$ for some 
$\sts \in \SS$. 
%
It is also elementary to show that, for all 
$p \not\equiv 1 \bmod 4$, the asymptotic density 
\[
 \lim_{x \to \infty}
  \frac{1}{x}
   \#\{n \le x : n + \bh \subseteq S_p\}, 
    \quad 
     n + \bh \defeq \{n + h : h \in \bh\},
\] 
exists, and is equal to 
\begin{equation}
% \label{eq:delthp}
  \delta_{\bh}(p)
   \defeq 
    \lim_{\alpha \to \infty}
     \frac{
      \#\{0 \le a < p^{\alpha} : 
             \forall h \in \bh, 
              a + h \equiv \sots \bmod p^{\alpha}\}
          }
          {p^{\alpha}},
\end{equation}
%
which exists, and is nonzero if $\bh = \{0\}$, by 
Propositions \ref{prop:S2h} (a) and \ref{prop:Sp3h} (a).
%
Thus, $R_k(\bh;x)$ is the probability that 
$n + \bh \subseteq S_p$ for all $p \not\equiv 1 \bmod 4$, and, 
regarding the probabilities of $n + \bh$ being contained in $S_p$ 
and $S_q$ as more or less independent for $p \ne q$, we therefore 
expect that 
\[
 \frac{R_k(\bh;x)}{\big(R_1(x)\big)^k}
  =
  \frac{R_k(\bh;x)}{\big(R_1(\bz;x)\big)^k}
   \sim 
    \prod_{p \not\equiv 1 \bmod 4}
     \frac{\delta_{\bh}(p)}{\big(\delta_{\bz}(p)\big)^k}
      \quad 
       (x \to \infty),
\]
provided that the product converges to a nonzero number.

\begin{definition}
 \label{def:Sss}
 %
Fix $k \ge 1$, and a set 
$\bh = \{h_1,\ldots,h_k\} \subseteq \ZZ$ with $\card \bh = k$.
%
The {\em singular series} for $\bh$ is given by  
\begin{equation}
 \label{eq:defsssP}
  \mathfrak{S}_{\bh}
   \defeq 
    \prod_{p \not\equiv 1 \bmod 4}
     \frac{\delta_{\bh}(p)}{\big(\delta_{\bz}(p)\big)^k},
\end{equation}
with $\delta_{\{0\}}(p)$ and $\delta_{\bh}(p)$ as in 
\eqref{eq:delthp}.
\end{definition}

Proposition \ref{prop:Sp3h} (c) shows that, for sufficiently 
large primes $p \equiv 3 \bmod 4$, 
$\delta_{\bz}(p)^{-k}\delta_{\bh}(p) = 1 + O_k(1/p^2)$, and so 
the product in \eqref{eq:defsssP} converges to a nonzero number 
unless $\delta_{\bh}(p) = 0$ for some $p \not\equiv 1 \bmod 4$.
%
Proposition \ref{prop:sssc} shows that $\mathfrak{S}_{\bh} \ne 0$ 
if and only if $\bh$ is $\SS$-admissible.


\begin{definition}
 \label{def:Sadm}
%
Fix $k \ge 1$, and a set $\bh = \{h_1,\ldots,h_k\} \subseteq \ZZ$ 
with $\card \bh = k$.
%
We say that $\bh$ is {\em $\SS$-admissible} if, and only if, for 
all primes $p \not\equiv 1 \bmod 4$, there exists $n \in \ZZ$ such 
that $n + \bh \subseteq S_p$, where $S_p$ is as in 
\eqref{eq:defSp}.
%   
\end{definition}

\noindent
%
In view of \eqref{eq:sotsnt}, for $\SS$-admissible $\bh$, 
\eqref{eq:sotsktups} is equivalent to 
\begin{equation}
 \label{eq:equivsotsktups}
 \sum_{n \le x}
  \ind{\SS}(n + h_1)\cdots \ind{\SS}(n + h_k)
%   \sim 
%    \mathfrak{S}_{\bh}
%     x
%      \big(R_1(x)\big)^k
       \sim 
        C^k\mathfrak{S}_{\bh}
        x(\log x)^{-k/2}
         \quad 
       (x \to \infty),
\end{equation}
and so we have the following qualitative version of 
Conjecture \ref{con:sotsktups}.

\begin{conjecture} 
 \label{con:qualktups-hide}
Fix $k \ge 1$, and a set 
$\bh = \{h_1,\ldots,h_k\} \subseteq \ZZ$ with $\card \bh = k$. 
%
If $\bh$ is $\SS$-admissible, then there exist infinitely many 
integers $n$ for which $n + \bh \subseteq \SS$.
\end{conjecture}

\noindent
%
It is plain that, if $\bh$ is not $\SS$-admissible, then there 
are no integers $n$ for which $n + \bh \subseteq \SS$.
%
\end{nixnix}

%%%%%%%%%%%%%%%%%%%%%%%%%%%%%%%%%%%%%%%%%%%%%%%%%%%%%%%%%%%%%%%%%%
%%%%%%%%%%%%%%%%%%%%%%%%%%%% SECTION 03 %%%%%%%%%%%%%%%%%%%%%%%%%%
%%%%%%%%%%%%%%%%%%%%%%%%%%%%%%%%%%%%%%%%%%%%%%%%%%%%%%%%%%%%%%%%%%

\section{Notation}
 \label{sec:notation}

We define the set of natural numbers as 
$\NN \defeq \{1,2,\ldots\}$.
%
The letter $p$ stands for a prime, $n$ for an integer.
%
We let $\sots\ $ stand for a generic element of $\SS$, possibly a 
different element each time.
%
Thus, for instance, $a + h\equiv \sots \, \bmod p^{\alpha}$ denotes 
that $a + h \equiv \sts \bmod p^{\alpha}$ for some $\sts \in \SS$.
%
We view $k$ as a fixed natural number, and $\bh$ as a nonempty, 
finite set of integers, with $\card \bh = k$ unless otherwise 
indicated.
%
We let $n + \bh \defeq \{n + h : h \in \bh\}$.
%
For $n \in \NN$, $\omega(n)$ denotes the number of distinct prime 
divisors of $n$, $\nu_p(n)$ the $p$-adic valuation of $n$.
%
(We also define $\nu_p(0) \defeq \infty$.) 
%
That $\nu_p(n) = \alpha$ may also be denoted by 
$p^{\alpha} \emid n$.
%
The radical of $n$ is $\rad(n) \defeq \prod_{p \mid n} p$, not to 
be confused with the squarefree part of $n$, viz.\  
$\sqfr(n) \defeq \prod_{p \emid n} p$.
%
By the least residue of an integer $a$ modulo $n$ we mean the 
integer $r$ such that $a \equiv r \bmod n$ and $0 \le r < n$.
%
When written in an exponent, $\alpha \bmod 2$ is to be interpreted 
as the least residue of $\alpha$ modulo $2$: for instance, 
$p^{\alpha \bmod 2} = 1$ if $\alpha$ is even.

We view $x$ as a real parameter tending to infinity.
%
Expressions of the form $A \sim B$ denote that $A/B \to 1$ as 
$x \to \infty$.
%
We also view $y$ as real parameter tending to infinity, 
typically in such a way that $y \sim x/\speccount(x)$.
%
We may assume that $x$ and $y$ are sufficiently large in terms of 
any fixed quantity.
%
%
Expressions of the form $A = O(B)$, $A \ll B$ and $B \gg A$ all 
denote that $|A| \le c|B|$, where $c$ is some positive constant, 
throughout the domain of the quantity $A$.
%
The constant $c$ is to be regarded as independent of any parameter 
unless indicated otherwise by subscripts, as in 
$A = O_{k}(B)$ ($c$ depends on $k$ only), $A \ll_{k,\lambda} B$ 
($c$ depends on $k$ and $\lambda$ only), etc.
%
By $o(1)$ we mean a quantity that tends to zero as $y \to \infty$.

%%%%%%%%%%%%%%%%%%%%%%%%%%%%%%%%%%%%%%%%%%%%%%%%%%%%%%%%%%%%%%%%%%
%%%%%%%%%%%%%%%%%%%%%%%%%% SECTION 04 %%%%%%%%%%%%%%%%%%%%%%%%%%%%
%%%%%%%%%%%%%%%%%%%%%%%%%%%%%%%%%%%%%%%%%%%%%%%%%%%%%%%%%%%%%%%%%%

\section{Deducing Theorem \ref{thm:main} from Proposition \ref{prop:sssa}}
 \label{sec:main}

Given $\vbi = (i_1,\ldots,i_r) \in \NN^r$ such that 
$i_1 + \cdots + i_r = k$, and 
$\vbl = (\lambda_1,\ldots,\lambda_r) \in \RR^r$, let 
\begin{equation}
 \label{eq:defThet}
 \Theta_{\vbi,\vbl}
  \defeq 
   \{(x_1,\ldots,x_k) \in \Delta^k : x_{i_1 + \cdots + i_j} - x_{i_1 + \cdots + i_{j - 1}} \le \lambda_j, j = 1,\ldots,r\},
\end{equation}
where for $j = 1$ we let $x_{i_1 + i_{j - 1}} = x_0 \defeq 0$. 
%
In the case where $r = 1$ and $\vbl = (\lambda)$, 
\begin{equation}
 \label{eq:defThetkl}
 \Theta_{\vbi,\vbl}
  = 
   \Theta_{k,\lambda}
    \defeq 
     \{(x_1,\ldots,x_k) \in \RR^k : 0 < x_1 < \cdots < x_k \le \lambda\}. 
\end{equation}
%
The following proof shows that Theorem \ref{thm:main} (a) and (b) 
hold under slightly weaker hypotheses than the ones stated:
for (a), it is enough to assume that 
Hypothesis \textup{(}$k,\Theta_{\vbi,\vbl},\emptyset$\textup{)}, 
where $\vbi = (i_1,\ldots,i_r)$ and 
$\vbl = (\lambda_1,\ldots,\lambda_r)$, holds for all $k \ge r$, 
and all $\vbi \in \NN^r$ satisfying $i_1 + \cdots + i_r = k$;
for (b), it is enough to assume that 
Hypothesis \textup{(}$k,\Theta_{k,\lambda},\emptyset$\textup{)} 
holds for all $k \ge 1$.

\begin{proof}[Deduction of Theorem \ref{thm:main}]
%
As this argument has appeared many times in the literature, we 
merely give an outline of it and provide references.
%
(a) 
% 
To ease notation, we let $\vbi = (i_1,\ldots,i_r)$, 
$\vbh = (h_1,\ldots,h_k)$, $\bh = \{h_1,\ldots,h_k\}$, and 
\[
 \speccount(\{0\} \cup \bh; x)
  \defeq 
   \sum_{n \le x} 
    \ind{\SS}(n)\ind{\SS}(n + h_1)\cdots \ind{\SS}(n + h_k).
\]
% and, finally, 
% $
%  \gap{n} \defeq \sts_{n + 1} - \sts_n
% $.
%
Let $\ell \ge 0$ be an integer, arbitrarily large but fixed.
%
An inclusion-exclusion argument (see \cite{HOO:65iii}, 
\cite[Appendix A]{KR:99} or \cite[Key Lemma 2.4.12]{KS:99}) shows 
that 
\begin{align}
 \begin{split}
  \label{eq:bded1}
 & 
   \sum_{k = r}^{r + 2\ell + 1}
    (-1)^{k - r}
     \sum_{i_1 + \cdots + i_r = k}
      \hspace{5pt}
       \sum_{\vbh \, \in \, y\Theta_{\vbi,\vbl} \cap \, \ZZ^k}
%       \sum_{n \le x}
%        \ind{\SS}(n)\ind{\SS}(n + h_1)\cdots \ind{\SS}(n + h_k)
          \speccount(\{0\} \cup \bh; x)
 \\
 & \hspace{30pt} 
  \le 
%     \#\{\sts_n \le x : \sts_{n + j} - \sts_{n + j - 1} \le \lambda_j y, j = 1,\ldots,r\}
     \sums[\sts_n \le x]
          [\sts_{n + j} - \sts_{n + j - 1} \le \lambda_j y] %[\gap{n + j - 1} \le \lambda_j y]
          [j = 1,\ldots,r] 1 
%  \\
%  & \hspace{5pt}
   \le 
    \sum_{k = r}^{r + 2\ell}
     (-1)^{k - r}
      \sum_{i_1 + \cdots + i_r = k}
       \hspace{5pt}
        \sum_{\vbh \, \in \, y\Theta_{\vbi,\vbl} \cap \, \ZZ^k}
%        \sum_{n \le x}
%         \ind{\SS}(n)\ind{\SS}(n + h_1)\cdots \ind{\SS}(n + h_k).
           \speccount(\{0\} \cup \bh; x),
  \end{split}            
\end{align}
the sums over $i_1 + \cdots + i_r = k$, here and below, being over 
all $\vbi \in \NN^r$ for which $i_1 + \cdots + i_r = k$.
%
We make the substitution \eqref{eq:defEterm}, with 
$\{0\} \cup \bh$ and $k + 1$ in place of $\bh$ and $k$; 
we apply Hypothesis ($k,\Theta_{\vbi,\vbl},\{0\}$) for all 
$k$ and $\vbi$ satisfying $r \le k \le r + 2\ell + 1$ and  
$i_1 + \cdots + i_r = k$;
we use Proposition \ref{prop:sssa}, and our 
assumption that $yR_1(x) \sim 1$, i.e.\ $y \sim x/\speccount(x)$, 
as $x \to \infty$.
%
Thus, we deduce from \eqref{eq:bded1} that 
\begin{equation}
  \label{eq:bded2}
   \sum_{k = r}^{r + 2\ell + 1}
     (-1)^{k - r}
      \sum_{i_1 + \cdots + i_r = k}
       \vol(\Theta_{\vbi,\vbl})
        \le 
         \liminf_{x \to \infty}
          \frac{1}{\speccount(x)}
%           \#\{\sts_n \le x : \sts_{n + j} - \sts_{n + j - 1} \le \lambda_j y, j = 1,\ldots,r\}
           \sums[\sts_n \le x]
                [\sts_{n + j} - \sts_{n + j - 1} \le \lambda_j y] %[\gap{n + j - 1} \le \lambda_j y]
                [j = 1,\ldots,r] 1
\end{equation}
and 
\begin{equation}
 \label{eq:bded3}
          \limsup_{x \to \infty}
           \frac{1}{\speccount(x)}
%            \#\{\sts_n \le x : \sts_{n + j} - \sts_{n + j - 1} \le \lambda_j y, j = 1,\ldots,r\}
            \sums[\sts_n \le x]
                 [\sts_{n + j} - \sts_{n + j - 1} \le \lambda_j y] %[\gap{n + j - 1} \le \lambda_j y]
                 [j = 1,\ldots,r] 1
           \le 
            \sum_{k = r}^{r + 2\ell}
            (-1)^{k - r}
             \sum_{i_1 + \cdots + i_r = k}
              \vol(\Theta_{\vbi,\vbl}).            
\end{equation}
Since 
$
 \vol(\Theta_{\vbi,\vbl}) 
  = \lambda_1^{i_1}\cdots \lambda_r^{i_r}/(i_1!\cdots i_r!)
$, 
the sums on the left and right of \eqref{eq:bded2} and 
\eqref{eq:bded3} are truncations of the Taylor 
series for $(1 - \e^{-\lambda_1})\cdots (1 - \e^{-\lambda_r})$.
%
We have chosen $\ell$ arbitrarily large, so we may conclude that 
\eqref{eq:thm:mainc} holds, provided 
Hypothesis ($k,\Theta_{\vbi,\vbl},\{0\}$) does whenever 
$k \ge r$ and $i_1 + \cdots + i_r = k$.
%

(b)
%
We use an argument of Gallagher \cite{GAL:76}, who proved an 
analogous result for primes.
%
Let $\ell \ge 1$ be an integer, arbitrarily large but fixed.
%
We have   
\begin{align*}
  \sum_{n \le x}
   \big(\speccount(n + \lambda y) - \speccount(n) \big)^{\ell}
  & =
     \sum_{n \le x}
      \bigg(\sum_{0 < h \le \lambda y} \ind{\SS}(n + h)\bigg)^{\ell}
 \\
  & = 
     \sum_{n \le x}
      \sum_{0 < h_1,\ldots,h_{\ell} \le \lambda y} \ind{\SS}(n + h_1)\cdots \ind{\SS}(n + h_{\ell})
 \\
  & =
       \sum_{k = 1}^{\ell}
        \varrho(\ell,k)
         \hspace{-5.8pt}
         \sum_{0 < h_1 < \cdots < h_k \le \lambda y}
          \hspace{2pt}
           \sum_{n \le x}
            \ind{\SS}(n + h_1)\cdots \ind{\SS}(n + h_k),
\end{align*}
where $\varrho(\ell,k)$ denotes the number of maps from 
$\{1,\ldots,\ell\}$ onto $\{1,\ldots,k\}$.
%
Thus,  
\[
 \frac{1}{x}
  \sum_{n \le x}
   \big(\speccount(n + \lambda y) - \speccount(n) \big)^{\ell}
  =
    \sum_{k = 1}^{\ell}
     \bigg(\frac{\speccount(x)}{x}\bigg)^k
      \varrho(\ell,k)
       \sum_{0 < h_1 < \cdots < h_k \le \lambda y} 
        \big(\mathfrak{S}_{\bh} + \cE_{\bh}(x)\big),
\]
with $\bh = \{h_1,\ldots,h_k\}$ in the last summand.
%
To sum over $0 < h_1 < \cdots < h_k \le \lambda y$ is to sum over 
$(h_1,\ldots,h_k) \in y\Theta_{k,\lambda} \cap \ZZ^k$ (see 
\eqref{eq:defThetkl}).
%
If Hypothesis ($k,\Theta_{k,\lambda},\emptyset$) holds then for 
some function $\varepsilon(x)$ with $\varepsilon(x) \to 0$ 
($x \to \infty$), we have  
\[
 \sum_{0 < h_1 < \cdots < h_k \le \lambda y} 
  \big(\mathfrak{S}_{\bh} + \cE_{\bh}(x)\big)
   =
    \big(1 + O_{\lambda,k}(\varepsilon(x))\big)
     \sum_{0 < h_1 < \cdots < h_k \le \lambda y} \mathfrak{S}_{\bh}.
\]
%
Applying Proposition \ref{prop:sssa} (noting that 
$\vol(\Theta_{k,\lambda}) = \lambda^k/k!$), and our assumption 
that $yR_1(x) \sim 1$, i.e.\ $y \sim x/\speccount(x)$, as 
$x \to \infty$, we see that if 
Hypothesis ($k,\Theta_{k,\lambda},\emptyset$) holds for 
$1 \le k \le \ell$, then 
\begin{equation}
 \label{eq:ellthmoment}
 \frac{1}{x}
  \sum_{n \le x}
   \big(\speccount(n + \lambda y) - \speccount(n) \big)^{\ell}
    \sim 
     \sum_{k = 1}^{\ell}
      \varrho(\ell,k)
       \frac{\lambda^k}{k!} 
        \quad 
      (x \to \infty).
\end{equation}
%
Gallagher's calculation in \cite[Section 3]{GAL:76} shows that 
$
 \sum_{k = 1}^{\ell} \varrho(\ell,k)\lambda^k/k!
$
is the $\ell$th moment of the Poisson distribution with parameter 
$\lambda$, and that the corresponding moment generating function 
is entire.
%
Since a Poisson distribution is determined by its moments, it 
follows (see \cite[Section 30]{BIL:95}) that for any given 
$m \ge 0$, \eqref{eq:thm:main} holds as $x \to \infty$, provided 
Hypothesis ($k,\Theta_{k,\lambda},\emptyset$) holds for all 
$k \ge 1$.
%
\end{proof}

%%%%%%%%%%%%%%%%%%%%%%%%%%%%%%%%%%%%%%%%%%%%%%%%%%%%%%%%%%%%%%%%%%
%%%%%%%%%%%%%%%%%%%%%%%%%% SECTION 05 %%%%%%%%%%%%%%%%%%%%%%%%%%%%
%%%%%%%%%%%%%%%%%%%%%%%%%%%%%%%%%%%%%%%%%%%%%%%%%%%%%%%%%%%%%%%%%%
 
\section{Preliminaries}
 \label{sec:prelims}
 
A positive integer $n$ is a sum of two squares if and only if
\[
 n =
   2^{\beta_2}
    \prod_{p \equiv 1 \bmod 4}p^{\beta_p}
     \prod_{p \equiv 3 \bmod 4}p^{2\beta_p},
\]
where $\beta_2,\beta_p$ denote nonnegative integers.
%
(See \cite[Theorem 366]{HW:38}.)
%
In view of this and the next proposition, whose proof, being  
routine and elementary, is omitted, we have 
$\SS = \bcap_p S_p$, where 
$
 S_p 
  = 
   \bcap_{\alpha \ge 1} \{n \in \ZZ : n \equiv \sots \bmod p^{\alpha}\}.
$
%(see \eqref{eq:defSp}), as claimed in Section \ref{sec:ktups}.
%
Further, as $S_p = \ZZ$ for primes $p \equiv 1 \bmod 4$, we may 
write $\SS = \bcap_{p \not\equiv 1 \bmod 4} S_p$.

\begin{proposition}
 \label{prop:S2S3S1}
%
Let $n \in \ZZ$.
%
We have $n \in S_2$ if and only if either $n = 0$ or 
$n = 2^{\beta}m$ for some $\beta \ge 0$ and 
$m \equiv 1 \bmod 4$.
%
For $p \equiv 3 \bmod 4$, we have $n \in S_p$ if and only if 
either $n = 0$ or $n = p^{2\beta}m$ for some $\beta \ge 0$ and 
$m \not\equiv 0 \bmod p$.
%
For $p \equiv 1 \bmod 4$, we have $S_p = \ZZ$.
\end{proposition}

Let us introduce some notation in order to state further results.
%
Given a nonempty, finite set $\bh \subseteq \ZZ$, let
\begin{equation}
 \label{eq:defdethDh}
 \det(\bh)
  \defeq 
   \prods[h,h' \in \bh][h > h'](h - h') > 0.
\end{equation}
%
%
Note that if $p \le k - 1$, where $k = \card \bh$, then two 
elements of $\bh$ occupy the same congruence class modulo $p$, so 
$p \mid \det(\bh)$.
%
In other words, if $p \nmid \det(\bh)$ then $k \le p$.

Let 
\begin{equation}
 \label{eq:defhp}
  \bh_p \defeq \{h' \in \bh : -h' + \bh \subseteq S_p\}.
\end{equation}
%
Note that $\bh_2$ contains at most one element, for if 
$h,h' \in \bh_2$ then $\pm(h - h') \in S_2$, which by 
Proposition \ref{prop:S2S3S1} holds only if $h - h' = 0$.
%
Similarly, if $k = 1$ or $k = 2$, then $\card \bh_2 = 1$.
%
By Proposition \ref{prop:S2S3S1}, $\bh_p$ for $p \equiv 3 \bmod 4$ 
consists precisely of those elements $h'$ of $\bh$ for which 
$2 \mid \nu_p(h - h')$ for every $h \in \bh$ with $h \ne h'$.
%
(Recall that $\nu_p(n)$ denotes the $p$-adic valuation of $n$.)
%
For instance, if $p \nmid \det(\bh)$ then $\bh_p = \bh$.

Given $\alpha \ge 1$, let 
\begin{equation}
 \label{eq:defTh}
  \T_{\bh}(2^{\alpha + 1})
   \defeq 
    \{0 \le a < 2^{\alpha + 1} : a + \bh \subseteq S_2 
      \,\, \hbox{and} \,\, 
       {\textstyle \max_{h \in \bh}} \nu_2(a + h) < \alpha\}.
\end{equation}
%
By Proposition \ref{prop:S2S3S1}, this is the (possibly empty) set 
of least residues $a$ modulo $2^{\alpha + 1}$ such that, for each  
$h \in \bh$, there is some $\beta \le \alpha - 1$ and 
$m \equiv 1 \bmod 4$ such that $a + h = 2^{\beta}m$.
%
Finally, for $p \equiv 3 \bmod 4$, let 
\begin{equation}
 \label{eq:defVh}
  \V_{\bh}(p^{\alpha})
   \defeq 
    \{0 \le a < p^{\alpha} : a + \bh \subseteq S_p 
      \,\, \hbox{and} \,\, 
       {\textstyle \max_{h \in \bh}} \nu_p(a + h) < \alpha\}.
\end{equation}
%
This is the (possibly empty) set of least residues $a$ modulo 
$p^{\alpha}$ such that, for each $h \in \bh$, there exists 
$\beta \le (\alpha - 1)/2$ for which $p^{2\beta} \emid a + h$. 
%
Note that, for $\alpha \ge 2$ and odd $p$, the difference between 
$\T_{\bh}(2^{\alpha})$ and $\V_{\bh}(p^{\alpha})$ is that 
$\T_{\bh}(2^{\alpha})$ contains only integers $a$ for which 
$\max_{h \in \bh} \nu_2(a + h) \le \alpha - 2$, whereas 
$\V_{\bh}(p^{\alpha})$ contains $a$ for which 
$\max_{h \in \bh} \nu_p(a + h) \le \alpha - 1$.
%
As may be expected in view of Proposition \ref{prop:S2S3S1}, we 
will need to treat $p = 2$ as a special case throughout.

Recall from \eqref{eq:delthp} that  
$
 \delta_{\bh}(p) 
  \defeq 
   \lim_{\alpha \to \infty} \card S_{\bh}(p^{\alpha})/p^{\alpha}
$, where 
\[
 S_{\bh}(p^{\alpha}) 
  \defeq 
   \{0 \le a < p^{\alpha} : \forall h \in \bh, a + h \equiv \sots \bmod p^{\alpha}\}.
\]
%
We have introduced $\V_{\bh}(p^{\alpha})$ because it is more 
convenient than $S_{\bh}(p^{\alpha})$ to work with.
%
It is not difficult to see that, for $p \not\equiv 1 \bmod 4$, 
$
 0 
  \le 
   \card S_{\bh}(p^{\alpha}) - \card T_{\bh}(p^{\alpha}) 
    \le 1
$ 
once $\alpha$ is sufficiently large.
%
(One may verify Proposition \ref{prop:S2S3S1} by showing 
that $n \equiv \sots \bmod 2^{\alpha}$ if and only if 
$n \equiv 2^{\beta}m \bmod 2^{\alpha}$ for some $\beta \ge 0$ and 
odd $m$, and, for $p \equiv 3 \bmod 4$, that 
$n \equiv \sots \bmod p^{\alpha}$ if and only if
$n \equiv p^{2\beta}m \bmod p^{\alpha}$ for some $\beta \ge 0$ 
and $m \not\equiv 0 \bmod p$.) 
%
Thus, the limit $\delta_{\bh}(p)$ exists if and only if 
$\lim_{\alpha \to \infty} \card \V_{\bh}(p^{\alpha})/p^{\alpha}$ 
exists, in which case the two are equal.

In the next two propositions, and throughout, we allow for the 
possibility that $k = 1$.
%
In case $\bh = \{h_1\}$, we define 
$\max_{i \ne j} \nu_p(h_i - h_j)$ to be zero (and 
$\det(\bh) \defeq 1$). 

\begin{proposition}
 \label{prop:S2h}
Let $\bh = \{h_1,\ldots,h_k\}$ be a set of $k \ge 1$ distinct 
integers.
%

\textup{(}a\textup{)}
%
The limits $\delta_{\bh}(2)$ 
\textup{(}see \eqref{eq:delthp}\textup{)} and 
$
 \lim_{\alpha \to \infty} 
  \card \T_{\bh}(2^{\alpha + 1})/2^{\alpha + 1}
$ exist, and are equal:
\begin{equation}
 \label{eq:defdelth2}
  \delta_{\bh}(2)
   =
    \lim_{\alpha \to \infty}
     \frac{\card \T_{\bh}(2^{\alpha + 1})}{2^{\alpha + 1}}.
\end{equation}
%
Moreover, for all $\alpha \ge 1$, we have  
\begin{equation}
 \label{eq:Thdelth2bnd}
  \bigg|
   \frac{\card \T_{\bh}(2^{\alpha + 1})}{2^{\alpha + 1}}
     -
      \delta_{\bh}(2)
  \bigg|
   \le 
    \frac{k}{2^{\alpha}}. 
\end{equation}
%

\textup{(}b\textup{)}
%
For any $\alpha \ge 2 + \max_{i \ne j} \nu_2(h_i - h_j)$, we have 
\begin{equation}
 \label{eq:delth2}
  \delta_{\bh}(2)
   =
    \frac{\card \T_{\bh}(2^{\alpha + 1}) + \card \bh_2}{2^{\alpha + 1}}, 
\end{equation}
the right-hand side being constant for $\alpha$ in this range.
%

\textup{(}c\textup{)}
%
If $2 \nmid \det(\bh)$ 
\textup{(}in which case $k \le 2$\textup{)}, then 
$\delta_{\bh}(2) = (1/2)^k$.
%
As a special case, we record here that $\delta_{\bz}(2) = 1/2$.
\end{proposition}

\begin{proof}
%
In essence, we use a Hensel-type argument: for $\alpha \ge 1$, the 
condition that $n \equiv \sots \bmod 2^{\alpha}$ can be lifted
to $n \equiv \sots \bmod 2^{\alpha + 1}$, unless $n = 2^{\alpha}m$
for some $m \equiv 3 \bmod 4$.

(a)
%
As already noted, to show that $\delta_{\bh}(2)$ and the 
right-hand side of \eqref{eq:defdelth2} exist and are equal, it 
suffices to show that the right-hand side exists.
%
Let $\alpha \ge 1$ and let $0 \le b < 2^{\alpha + 2}$, so 
$b = a + 2^{\alpha + 1}q$, where 
$0 \le a < 2^{\alpha + 1}$ and either $q = 0$ or $q = 1$.
%
Suppose that, for each $i$, there exists $\beta_i \le \alpha - 1$ 
and $m_i \equiv \pm 1 \bmod 4$ such that 
$b + h_i = 2^{\beta_i}m_i$.
%
Then, for each $i$, $a + h_i = 2^{\beta_i}m'_i$ 
and $a + 2^{\alpha + 1} + h_i = 2^{\beta_i}m''_i$, where 
$m'_i \equiv m''_i \equiv m_i \bmod 4$.
%
Recalling Proposition \ref{prop:S2S3S1} and definition 
\eqref{eq:defTh}, we see that the following statements are 
equivalent: 
(i) $b \in \T_{\bh}(2^{\alpha + 2})$; 
(ii) both $a$ and $a + 2^{\alpha + 1}$ are in 
$\T_{\bh}(2^{\alpha + 2})$;
(iii) $a \in \T_{\bh}(2^{\alpha + 1})$.

We have shown that we have a partition 
\[
 \T_{\bh}(2^{\alpha + 2})
  =
   \{a,a + 2^{\alpha + 1} : a \in \T_{\bh}(2^{\alpha + 1})\}
    \cup
     \U_{\bh}(2^{\alpha + 2}),
\]
where 
\[
 \U_{\bh}(2^{\alpha + 2}) 
  \defeq 
   \{0 \le b < 2^{\alpha + 2} : b + \bh \subseteq S_2 
        \,\, \hbox{\textup{and}} \,\, 
         {\textstyle \max_{h \in\bh} } \nu_2(b + h) = \alpha\}  
\]
is the set of elements $b$ of $\T_{\bh}(2^{\alpha + 2})$ for which 
$\nu_2(b + h_j) = \alpha$ for some $h_j \in \bh$.
%
Any element of $\U_{\bh}(2^{\alpha + 2})$ is a least
residue of 
$\pm 2^{\alpha} - h_j$ for some $h_j \in \bh$, of which there are 
at most $2k$.
%
We see that 
\[
   \frac{\card \T_{\bh}(2^{\alpha + 2})}{2^{\alpha + 2}}
  -
    \frac{\card \T_{\bh}(2^{\alpha + 1})}{2^{\alpha + 1}}
   =
      \frac{\card \U_{\bh}(2^{\alpha + 2})}{2^{\alpha + 2}}
       \le 
        \frac{k}{2^{\alpha + 1}}.
\]
%
Consequently, for any $\beta$ with $\beta \ge \alpha$, we have 
\[
 0
  \le 
   \frac{\card \T_{\bh}(2^{\beta + 1})}{2^{\beta + 1}}
  -
    \frac{\card \T_{\bh}(2^{\alpha + 1})}{2^{\alpha + 1}}
     =
      \sum_{r = 1}^{\beta - \alpha}
       \frac{\card \U_{\bh}(2^{\alpha + r + 1})}{2^{\alpha + r + 1}}
        <
         \frac{k}{2^{\alpha}}.
\]
%
It follows that the limit on the right-hand side of 
\eqref{eq:defdelth2} exists, and that \eqref{eq:Thdelth2bnd} holds 
for all $\alpha \ge 1$.

(b)
%
Assume that $\alpha \ge 2 + \max_{i \ne j} \nu_2(h_i - h_j)$.
%
Suppose that, for some $j$, there exists $q$ such that 
$b + h_j = 2^{\alpha}(1 + 2q)$.
%
We have $b + h_j \in S_2$ if and only if 
$2 \mid q$, equivalently, 
$b + h_j \equiv 2^{\alpha} \bmod 2^{\alpha + 2}$.
%
For $i \ne j$ 
%
we may write $h_i - h_j = 2^{\beta_{ij}}m_{ij}$ with 
$\beta_{ij} \le \alpha - 2$ and $m_{ij} \equiv \pm 1 \bmod 4$.
%
Thus,  
\[
 b + h_i 
  = 2^{\beta_{ij}}(m_{ij} + 2^{\alpha - \beta_{ij}}(1 + 2q))
\]
is in $S_2$ if and only if $m_{ij} \equiv 1 \bmod 4$, 
equivalently, $h_i - h_j \in S_2$.
%
By definition of $\bh_2$, this holds for each $i \ne j$ if and 
only if $h_j \in \bh_2$.
%
We have shown that $b \in \T_{\bh}(2^{\alpha + 2})$ and 
$\nu_2(b + h_j) = \alpha$ for some $h_j \in \bh$ if and only if  
$\bh_2$ is nonempty, $h_j$ is the (necessarily unique) element of 
$\bh_2$, and $b + h_j \equiv 2^{\alpha} \bmod 2^{\alpha + 2}$. 
%
Thus,
\[
  \U_{\bh}(2^{\alpha + 2})
   =
    \{0 \le b < 2^{\alpha + 2} : \exists h' \in \bh_2, b \equiv 2^{\alpha} - h' \bmod 2^{\alpha + 2}\},
\]
and $\card \U_{\bh}(2^{\alpha + 2}) = \card \bh_2$.
%
Also, 
$
 \card \T_{\bh}(2^{\alpha + 2}) 
 = 2\card \T_{\bh}(2^{\alpha + 1}) + \card \bh_2
$.
%
Hence 
\[
 \frac{\card \T_{\bh}(2^{\alpha + 2}) + \card \bh_2}{2^{\alpha + 2}}
  =
   \frac{\card \T_{\bh}(2^{\alpha + 1}) + \card \bh_2}{2^{\alpha + 1}}.
\]

(c) 
%
Suppose $2 \nmid \det(\bh)$.
%
If $k = 1$, i.e.\ if $\bh = \{h_1\}$, then the elements of 
$\T_{\bh}(8)$ are precisely the least residues of 
$1 - h_1, 2 - h_1$ and $5 - h_1$ modulo $8$.
%
Also, $\bh_2 = \bh$.
%
If $k = 2$, i.e.\ if $\bh = \{h_1,h_2\}$, then either 
$h_2 - h_1 \equiv 1 \bmod 4$ or $h_1 - h_2 \equiv 1 \bmod 4$.
%
Without loss of generality, suppose $h_2 - h_1 \equiv 1 \bmod 4$.
%
Then the sole element of $\T_{\bh}(8)$ is the least residue of 
$h_2 - 2h_1$ modulo $8$.
%
Also, $\bh_2 = \{h_1\}$.
%
Therefore, by (b), $\delta_{\bh}(2) = (1/2)^k$.
\end{proof}

For the next proposition, recall that $\alpha \bmod 2$, when 
written in an exponent, denotes the least residue of $\alpha$ 
modulo $2$.
%
For instance, $p^{\alpha \bmod 2} = 1$ if $\alpha$ is even.

\begin{proposition}
 \label{prop:Sp3h}
Let $\bh = \{h_1,\ldots,h_k\}$ be a set of $k \ge 1$ distinct 
integers, and let $p$ be a prime with $p \equiv 3 \bmod 4$.
%

\textup{(}a\textup{)}
%
The limits $\delta_{\bh}(p)$ 
\textup{(}see \eqref{eq:delthp}\textup{)} and 
$\lim_{\alpha \to \infty} \card \V_{\bh}(p^{\alpha})/p^{\alpha}$ 
exist, and are equal:
\begin{equation}
 \label{eq:defdelthp3}
  \delta_{\bh}(p)
   = 
    \lim_{\alpha \to \infty}
     \frac{\card \V_{\bh}(p^{\alpha})}{p^{\alpha}}.
\end{equation}
%
Moreover, for all $\alpha \ge 1$, we have 
\begin{equation}
 \label{eq:Vhdeltp3bnd}
  \bigg|
   \frac{\card \V_{\bh}(p^{\alpha})}{p^{\alpha}}
     -
      \delta_{\bh}(p)
  \bigg|
   \le 
    \frac{k}{p^{\alpha }}
     \bigg(1 + \frac{1}{p}\bigg)^{-1} 
      \frac{1}{p^{\alpha \bmod 2}}.
\end{equation}

\textup{(}b\textup{)}
%
For any $\alpha \ge 1 + \max_{i \ne j} \nu_p(h_i - h_j)$, we have 
\begin{equation}
 \label{eq:delthp3}
  \delta_{\bh}(p)
   =
    \frac{1}{p^{\alpha}}
     \bigg(
      \card \V_{\bh}(p^{\alpha}) + \card \bh_p \bigg(1 + \frac{1}{p}\bigg)^{-1} \frac{1}{p^{\alpha \bmod 2}}
     \bigg),
\end{equation}
the right-hand side being constant for $\alpha$ in this range.

\textup{(}c\textup{)}
%
We have  
\begin{equation}
 \label{eq:delthpropsp3}
  \delta_{\bh}(p)
   \ge 
    \bigg(1 + \frac{1}{p}\bigg)^{-1}
     \bigg(1 - \frac{\min\{k - 1,p\}}{p}\bigg),
\end{equation}
with {\bfseries equality} attained if $p \nmid \det(\bh)$ 
\textup{(}in which case $k \le p$\textup{)}.
%
As a special case, we record here that 
$\delta_{\bz}(p) = (1 + 1/p)^{-1}$.
\end{proposition}

\begin{proof}
%
(a)
%
As noted above the statement of Proposition \ref{prop:S2h}, to 
show that $\delta_{\bh}(p)$ and the right-hand side of 
\eqref{eq:defdelthp3} exist and are equal, it suffices to show 
that the right-hand side exists.
%
Let $\alpha \ge 1$ and let $0 \le b < p^{\alpha + 1}$.
%
Thus, $b = a + p^{\alpha}q$, where $0 \le a < p^{\alpha}$ and 
$0 \le q < p$.
%
Suppose that, for each $i$, there exists $\beta_i \le \alpha - 1$ 
and $m_i \not\equiv 0 \bmod p$ such that 
$b + h_i = p^{\beta_i}m_i$.
%
Then, for each $i$ and each $q'$, $0 \le q' < p$, we have 
$a + p^{\alpha}q' + h_i = p^{\beta_i}m_i'$, where 
$m_i' \equiv m_i \not\equiv 0 \bmod p$.
%
Recalling Proposition \ref{prop:S2S3S1} and definition 
\eqref{eq:defVh}, we see that the following are equivalent: 
(i) $b \in \V_{\bh}(p^{\alpha + 1})$; 
(ii) $a + p^{\alpha}q' + h_i \in \V_{\bh}(p^{\alpha + 1})$ for 
$0 \le q' < p$;
(iii) $a \in \V_{\bh}(p^{\alpha})$.

We have shown that we have a partition
\[
  \V_{\bh}(p^{\alpha + 1})
   = 
    \{a + p^{\alpha}q : a \in \V_{\bh}(p^{\alpha}), 0 \le q < p\}
     \cup 
      \W_{\bh}(p^{\alpha + 1}),
\]
where
\[
 \W_{\bh}(p^{\alpha + 1}) 
  \defeq 
   \{0 \le b < p^{\alpha + 1} : b + \bh \subseteq S_p 
        \,\, \hbox{\textup{and}} \,\, 
         {\textstyle \max_{h \in\bh} } \nu_p(b + h) = \alpha\} 
\]
is the set of elements $b$ of $\V_{\bh}(p^{\alpha + 1})$ for which 
$\nu_p(b + h_j) = \alpha$ for some $h_j \in \bh$.
%
Plainly, $\W_{\bh}(p^{\alpha + 1})$ is empty if $\alpha$ is odd.
%
(If $b + \bh \subseteq S_p$ then, by 
Proposition \ref{prop:S2S3S1}, $\nu_p(b + h_j)$ is even and hence 
not equal to any odd $\alpha$.)
%
Also, any element of $\W_{\bh}(p^{\alpha + 1})$ is a least residue 
of $p^{\alpha}q - h_j \bmod p^{\alpha + 1}$, for some 
$0 < q < p$ and $h_j \in \bh$, of which there are at most 
$(p - 1)k$.
%
We see that 
\begin{equation}
 \label{eq:VW}
 \frac{\card \V_{\bh}(p^{\alpha + 1})}{p^{\alpha + 1}}
  -
   \frac{\card \V_{\bh}(p^{\alpha})}{p^{\alpha}}
    =
     \frac{\card \W_{\bh}(p^{\alpha + 1})}{p^{\alpha + 1}},
\end{equation}
and that 
\begin{equation}
 \label{eq:VWb}
 0 
  \le 
   \frac{\card \W_{\bh}(p^{\alpha + 1})}{p^{\alpha + 1}}
    \le 
     \bigg(1 - \frac{1}{p}\bigg)\frac{k}{p^{\alpha}},
\end{equation}
with {\em equality} on the left if $\alpha$ is {\em odd}.
%
Consequently, for any $\beta$ with $\beta \ge \alpha$, we have 
\begin{align*}
 0
 \le
  \frac{\card \V_{\bh}(p^{\beta})}{p^{\beta}}
   -
    \frac{\card \V_{\bh}(p^{\alpha})}{p^{\alpha}}
 =
  \sum_{r = 1}^{\beta - \alpha}
   \frac{\card \W_{\bh}(p^{\alpha + r})}{p^{\alpha + r}}
    <
     \bigg(1 - \frac{1}{p}\bigg)\frac{k}{p^{\alpha}} 
      \sums[r - 1 \ge 0][r - 1 \equiv \alpha \bmod 2]
       \frac{1}{p^{r - 1}}.
\end{align*}
%
Since this last sum is equal to $1/(1 - 1/p^2)$ if $\alpha$ is 
even, and to $1/(p(1 - 1/p^2))$ if $\alpha$ is odd, we have 
\[
  0
   \le
    \frac{\card \V_{\bh}(p^{\beta})}{p^{\beta}} 
   - \frac{\card \V_{\bh}(p^{\alpha})}{p^{\alpha}}
      <
       \frac{k}{p^{\alpha}}
        \bigg(1 + \frac{1}{p}\bigg)^{-1}
         \frac{1}{p^{\alpha \bmod 2}}.
\]
%
It follows that the limit on the right-hand side of 
\eqref{eq:defdelthp3} exists, and that \eqref{eq:Vhdeltp3bnd} 
holds for all $\alpha \ge 1$.

(b)
%
Let $0 \le b < p^{\alpha + 1}$, and assume now that 
$\alpha \ge 1 + \max_{i \ne j} \nu_p(h_i - h_j)$.
%
Suppose that, for some $j$, we have $b + h_j = p^{\alpha}m_j$ 
for some $m_j \not\equiv 0 \bmod p$.
%
We have $b + h_j \in S_p$ if and only if $\alpha$ is even.
%
Let $i \ne j$.
%
We may write $h_i - h_j = p^{\beta_{ij}}m_{ij}$ with 
$\beta_{ij} \le \alpha - 1$ and $m_{ij} \not\equiv 0 \bmod p$.
%
Thus, 
$b + h_i = p^{\beta_{ij}}(m_{ij} + p^{\alpha - \beta_{ij}}m_j)$ 
is in $S_p$ if and only if $\beta_{ij}$ is even, equivalently, 
$h_i - h_j \in S_p$.
%
By definition of $\bh_p$, this holds for each $i \ne j$ if and 
only if $h_j \in \bh_{p}$.
%
In that case, for $0 \le q' < p$ with 
$q' \not\equiv -m_j \bmod p$, we have 
$b + p^{\alpha}q' + h_i \in S_p$ and 
$\nu_p(b + p^{\alpha}q' + h_i) = \beta_{ij} < \alpha$ 
for $i \ne j$;    
$b + p^{\alpha}q' + h_j \in S_p$ if and only if $b + h_j \in S_p$, 
and $\nu_p(b + p^{\alpha}q' + h_j) = \alpha$.
%
For $q' \equiv -m_j \bmod p$, 
$\nu_p(b + p^{\alpha}q' + h_j) > \alpha$.

Thus, if $\W_{\bh}(p^{\alpha + 1}) \ne \emptyset$, then 
$\alpha$ is even and $\bh_p \ne \emptyset$; and if 
$b \in \W_{\bh}(p^{\alpha + 1})$, then the $h_j$ for which 
$\nu_p(b + h_j) = \alpha$ is uniquely determined by $b$ and must 
lie in $\bh_p$.
%
If $\alpha$ is even, then, writing $h_j = p^{\alpha}q_j + r_j$, 
with $0 \le r_j < p^{\alpha}$, we see that 
\[
 \W_{\bh}(p^{\alpha + 1}) 
  = 
   \bcup_{h_j \in \bh_p}
    \{p^{\alpha}(q' + 1) - r_j : 0 \le q' < p, q' \not\equiv -q_j \bmod p\}.
\]
%
Thus, 
$
 \card \V_{\bh}(p^{\alpha + 1})
  = 
   p\card \V_{\bh}(p^{\alpha}) 
$
if $\alpha$ is odd, and 
$
 \card \V_{\bh}(p^{\alpha + 1})
  = 
   p\card \V_{\bh}(p^{\alpha}) + (p - 1)\card \bh_p 
$
if $\alpha$ is even.
%
%
Consequently, if $\alpha$ is odd then 
\[
 \frac{1}{p^{\alpha + 1}}
  \bigg(
     \card \V_{\bh}(p^{\alpha + 1}) + \card\bh_p\frac{p}{p + 1}
  \bigg)
   =
 \frac{1}{p^{\alpha}}
  \bigg(
     \card \V_{\bh}(p^{\alpha}) + \card\bh_p\frac{1}{p + 1}
  \bigg),
\]
while if $\alpha$ is even then 
\[
 \frac{1}{p^{\alpha + 1}}
  \bigg(
     \card \V_{\bh}(p^{\alpha + 1}) + \card\bh_p\frac{1}{p + 1}
  \bigg)
   =
 \frac{1}{p^{\alpha}}
  \bigg(
     \card \V_{\bh}(p^{\alpha}) + \card\bh_p\frac{p}{p + 1}
  \bigg).
\]

(c) Note that 
$
 \V_{\bh}(p)
  =
   \{0 \le a < p : \forall i, a \not\equiv -h_i \bmod p\} 
$,
so $\card \V_{\bh}(p) = p - \kappa$ where $\kappa$ is the 
number of distinct congruence classes in  
$\{h_i \bmod p : h_i \in \bh\}$.
%
Thus, $\kappa = k$ if and only if $p \nmid \det(\bh)$.
%
First, consider the case $p \mid \det(\bh)$, 
i.e.\ $\kappa \le k - 1$.
%
As $\delta_{\bh}(p) \ge 0$, \eqref{eq:delthpropsp3} is trivial for 
$p \le k - 1$, so let us assume that $k \le p$.
%
The relation \eqref{eq:VW} shows that 
$
 \card \V_{\bh}(p^{\alpha + 1})/p^{\alpha + 1} 
  \ge 
   \card \V_{\bh}(p^{\alpha})/p^{\alpha}
$ 
for $\alpha \ge 1$, and hence 
\[
 \delta_{\bh}(p)
  \ge 
   \frac{\card \V_{\bh}(p)}{p}
    \ge 
     \frac{p - (k - 1)}{p}
       > 
        1 - \frac{k}{p + 1}.
\]
%
The right-hand side of \eqref{eq:delthpropsp3} is equal to 
$1 - k/(p + 1)$ when $\min\{k - 1,p\} = k - 1$, as we are 
currently assuming. 
%
%
Next, consider the case $p \nmid \det(\bh)$, i.e.\ $\kappa = k$.
%
In this case, we have $\card \bh = \card \bh_p = k$ and, by 
\eqref{eq:delthp3},
\[
 \delta_{\bh}(p)
  = 
   \frac{1}{p}
    \bigg(\card \V_{\bh}(p) + \card\bh_p\bigg(1 + \frac{1}{p}\bigg)^{-1}\frac{1}{p} \bigg)
     =
      \bigg(1 + \frac{1}{p}\bigg)^{-1}
       \bigg(1 - \frac{k - 1}{p}\bigg),
\]
which is equal to the right-hand side of \eqref{eq:delthpropsp3} 
(since $p \ge \kappa = k$).
%
\end{proof}

Notice that, for all $p \not\equiv 1 \bmod 4$, we have 
$0 \le \delta_{\bh}(p) \le 1$, by definition.
%
%From 
% definitions \eqref{eq:defdelth2}, \eqref{eq:defdelthp3} and 
% Propositions \ref{prop:S2h} (b) and \ref{prop:Sp3h} (b), it 
% follows
% that $\delta_{\bh}(p) > 0$ if and only if  
% $n + \bh \subseteq S_p$ for some $n$, and so $\bh$ is 
%that $\bh$ is 
% $\SS$-admissible %(see Definition \ref{def:Sadm})
% if and only if 
% $\delta_{\bh}(p) > 0$ for all $p$. 
%
By the following proposition,
%$\SS$-admissibility of $\bh$
%is 
%thus equivalent to the nonvanishing of its singular series
%i.e.,
the nonvanishing of its singular series
$
  \mathfrak{S}_{\bh}
   \defeq 
    \prod_{p \not\equiv 1 \bmod 4}
%      \big(
      \delta_{\bz}(p)^{-k}
       \delta_{\bh}(p),
%      \big)
$
%(see Definition \ref{def:Sss}), as claimed in 
%Section \ref{sec:ktups}.
is equivalent to $\delta_{\bh}(p) > 0$ for all $p$.

%****************************************************************%
%************************* START DETAIL *************************%
%****************************************************************%
%
\begin{nixnix}
%
Consider $p \equiv 3 \bmod 4$.
%
Suppose $n + \bh \subseteq S_p$ for some $n$.
%
By Proposition \ref{prop:S2S3S1}, for each $h_i \in \bh$ we have 
$n + h_i = p^{2\beta_i}m_i$ for some $\beta_i \ge 0$ and 
$m_i \not\equiv 0 \bmod p$.
%
Let $\alpha > \max_{h_i \in \bh} 2\beta_i$, and let $a$ be the 
least residue of $n$ modulo $p^{\alpha}$. 
%
Say $n = a + p^{\alpha}q$.
%
Then, for each $h_i \in \bh$, 
$
 a + h_i 
  = p^{2\beta_i}(m_i + p^{\alpha - 2\beta_i}q)
$, 
and 
$
 m_i + p^{\alpha - 2\beta_i}q \equiv m_i \not\equiv 0 \bmod p
$.
%
Thus, $a + \bh \subseteq S_p$ by Proposition \ref{prop:S2S3S1}, 
and $\max_{h \in \bh} \nu_p(a + h) < \alpha$.
%
Hence $a \in \V_{\bh}(p^{\alpha})$. 
%
Since $\V_{\bh}(p^{\alpha}) \ne \emptyset$, 
$\delta_{\bh}(p^{\alpha}) \ge 1/p^{\alpha} > 0$ by 
Proposition \ref{prop:Sp3h} (b).

Conversely, suppose $\delta_{\bh}(p) > 0$.
%
Then by Proposition \ref{prop:Sp3h} (b), either 
$\V_{\bh}(p^{\alpha}) \ne \emptyset$ or $\bh_p \ne \emptyset$, 
where $\alpha = 1 + \max_{i \ne j} \nu_p(h_i - h_j)$.
%
If $\V_{\bh}(p^{\alpha}) \ne \emptyset$ then there is some $a$, 
$0 \le a < p^{\alpha}$, such that $a + \bh \subseteq S_p$ and 
$\max_{h \in \bh} \nu_p(a + h) < \alpha$.
%
If $\bh_p \ne \emptyset$, we have $h_j \in \bh$ such that 
$-h_j + h_i \in S_p$ for each $h_i \in \bh$.
%
In the first case, take $n = a$; in the second case take 
$n = -h_j$.
%
In either case, we have some $n$ such that 
$n + \bh \subseteq S_p$.

Similarly, $n + \bh \subseteq S_2$ for some $n$ if and only if 
$\delta_{\bh}(2) > 0$.
\end{nixnix}
%
%****************************************************************%
%************************** END DETAIL **************************%
%****************************************************************%

\begin{proposition}
 \label{prop:sssc}
%
Let $\bh = \{h_1,\ldots,h_k\}$ be a set of $k \ge 1$ distinct 
integers.
%
We have 
\begin{equation}
 \label{eq:sssc1}
 \e^{-(k - 1)}
  \le 
   \prods[p \not\equiv 1 \bmod 4][p \nmid \det(\bh)] 
    \delta_{\bz}(p)^{-k}\delta_{\bh}(p)
     \le 
      1,
\end{equation}
and the product converges.
%
Consequently, 
{\small 
\begin{equation}
 \label{eq:sssc2}
  \frac{2^{k} \delta_{\bh}(2) }{\e^{k - 1} }
   \prods[p \equiv 3 \bmod 4][p \mid \det(\bh)]
    \bigg(   
     \bigg( 1 + \frac{1}{p} \bigg)^k 
      \delta_{\bh}(p) 
    \bigg)
  \le 
   \mathfrak{S}_{\bh}
    \le 
     2^k\delta_{\bh}(2)
      \prods[p \equiv 3 \bmod 4][p \mid \det(\bh)]
       \bigg(    
        \bigg( 1 + \frac{1}{p} \bigg)^k
         \delta_{\bh}(p) 
       \bigg).
\end{equation}
}
\end{proposition}

\begin{proof}
%
If $2 \nmid \det(\bh)$ then $k \le 2$ and 
$\delta_{\bz}(2)^{-k}\delta_{\bh}(2) = 1$ by 
Proposition \ref{prop:S2h} (c), so only the primes 
$p \equiv 3 \bmod 4$ have any bearing on the product in 
\eqref{eq:sssc1}.
%
Let $p \equiv 3 \bmod 4$, and suppose $p \nmid \det(\bh)$.
%
By Proposition \ref{prop:Sp3h} (c), $k \le p$ and 
\begin{equation}
 \label{eq:pnmiddeth}
  \delta_{\bz}(p)^{-k}\delta_{\bh}(p)
   =
    \bigg(1 + \frac{1}{p}\bigg)^{k - 1}
     \bigg(1 - \frac{k - 1}{p}\bigg).
\end{equation}
%
Thus, $\delta_{\bz}(p)^{-k}\delta_{\bh}(p) = 1 + O_k(1/p^2)$, and 
consequently the product in \eqref{eq:sssc1} converges.

More precisely, from \eqref{eq:pnmiddeth} we have, on the one 
hand,  
\[
  \delta_{\bz}(p)^{-k}\delta_{\bh}(p)
   =
    1 
    - 
     \sum_{j = 2}^k 
      \bigg\{
       (k - 1)\binom{k - 1}{j - 1} - \binom{k - 1}{j}
      \bigg\}
      p^{-j}
       \le 
        1,
\]
with equality attained if $k = 1$, which gives the upper bound in 
\eqref{eq:sssc1}, and also the lower bound for $k = 1$.
%
On the other hand we have
\[
 \delta_{\bz}(p)^{-k}\delta_{\bh}(p)
  \ge 
   1 - \frac{(k - 1)^2}{p^2}.
\]
%
For $k = 2$ we see that the product in \eqref{eq:sssc1} is at 
least $\prod_{p \equiv 3 \bmod 4}(1 - 1/p^2)$, which is equal to 
$1/(2C^2) = 0.856108\ldots$ (with $C$ being the Landau--Ramanujan 
constant; see \eqref{eq:sotsnt}), and is 
greater than $\e^{-1}$.
%
For $k \ge 3$ we apply the basic inequality 
$\log(1 - x) \ge -x/(1 - x)$ ($0 \le x < 1$) to the above, 
obtaining 
\[
 \textstyle 
 \log \delta_{\bz}(p)^{-k}\delta_{\bh}(p)
  \ge 
  -\frac{(k - 1)^2}{p^2}\Big(1 - \frac{(k - 1)^2}{p^2}\Big)^{-1}
    \ge
   - \frac{(k - 1)^2}{p^2}\Big(1 - \frac{(k - 1)^2}{k^2}\Big)^{-1}
\]
(since $k \le p$).
%
Noting that  
$
 -\sum_{p \nmid \det(\bh)} 1/p^2
  \ge 
   -\sum_{n \ge k} 1/n^2
    \ge 
     -1/(k - 1)^2
$,
and that  
$-(1 - (k - 1)^2/k^2)^{-1} = -k^2/(2k - 1) > -(k - 1)$, then 
exponentiating, we see that product in \eqref{eq:sssc1} is greater 
than $\e^{-(k - 1)}$.
%
The inequalities in \eqref{eq:sssc2} follow upon recalling that 
$\delta_{\bz}(p) = (1 + 1/p)^k$ for $p \equiv 3 \bmod 4$ (see 
Proposition \ref{prop:Sp3h} (c)), and again that 
$\delta_{\bz}(2)^{-k}\delta_{\bh}(2) = 1$ if $2 \nmid \det(\bh)$ 
(see Proposition \ref{prop:S2h} (c)).
\end{proof}

%%%%%%%%%%%%%%%%%%%%%%%%%%%%%%%%%%%%%%%%%%%%%%%%%%%%%%%%%%%%%%%%%%
%%%%%%%%%%%%%%%%%%%%%%%%%% SECTION 06 %%%%%%%%%%%%%%%%%%%%%%%%%%%%
%%%%%%%%%%%%%%%%%%%%%%%%%%%%%%%%%%%%%%%%%%%%%%%%%%%%%%%%%%%%%%%%%%

\section{Proof of Proposition \ref{prop:sssa}}
 \label{sec:keyprop}

We will make use of the following elementary bounds.
%
Recall that, for $n \in \NN$, 
$\omega(n) \defeq \#\{p : p \mid n\}$, 
$\rad(n) \defeq \prod_{p \mid n} p$, and 
$\sqfr(n) \defeq \prod_{p \emid n} p$.
 
\begin{lemma} 
 \label{lem:omegabnd}
%
Let 
\begin{equation}
 \label{eq:defcN}
  \cN 
   \defeq 
    \{ab^2\rad(b) : a,b \in \NN, (a,b) = 1, a \text{ squarefree}\}.
\end{equation}
%
Fix any number $A \ge 1$.
%
For $y \ge 1$ and integers $D \ge 1$, we have 
\begin{equation}
 \label{eq:realbnd}
  \sums[n \in \cN][n > y]
   A^{\omega(n)}
    \frac{(D,\rad(n))}{n\sqfr(n)}
     \ll_A
      (1 + A)^{2\omega(D)}
       \frac{y^{O(1/\log\log 3y)}}{y^{2/3}},  
\end{equation}
and 
\begin{equation}
 \label{eq:realbnd2}
  \sums[n \in \cN][n \le y]
   \frac{A^{\omega(n)}}{\sqfr(n)}
    \ll_A
     y^{1/3 + O(1/\log\log 3y)}.
\end{equation}
\end{lemma}

\begin{proof}
%
Let $y \ge 1$ and let $D \ge 1$.
%
We claim that the following four bounds hold:
\begin{equation}
  \label{eq:omegasfbnd}
   \sums[n > y][\text{squarefree}] %\sumss[\flat][n > y]
    A^{\omega(n)}
     \frac{(D,n)}{n^2}
      \ll_A
       (1 + A)^{\omega(D)} \frac{y^{O(1/\log\log 3y)}}{y};
\end{equation}
\begin{equation}
 \label{eq:omegasfbnda}
  \sums[n \le y][\text{squarefree}] %\sumss[\flat][n \le y]
   A^{\omega(n)}
    \frac{(D,n)}{n}
     \ll_A
      (1 + A)^{\omega(D)} y^{O(1/\log\log 3y)};
\end{equation}
\begin{equation}
   \label{eq:omegaradbnd}
    \sum_{n^2\rad(n) > y}
     \frac{A^{\omega(n)}(D,\rad(n))}{n^2\rad(n)}
      \ll_A
       (1 + A)^{\omega(D)}\frac{y^{O(1/\log\log 3y)}}{y^{2/3}};
\end{equation}
and 
\begin{equation}
   \label{eq:smth13}
    \sum_{n^2\rad(n) \le y}
     A^{\omega(n)} 
      \ll_A
       y^{1/3 + O(1/\log\log 3y)}.
\end{equation}
%
Let us deduce \eqref{eq:realbnd} and \eqref{eq:realbnd2}.
%
The left-hand side of \eqref{eq:realbnd} is at most 
\[
   \sums[a \le y^{2/3}][\text{squarefree}]  %\sumss[\flat][a \le y^{2/3}] 
    A^{\omega(a)}\frac{(D,a)}{a^2}
    \sum_{b^2\rad(b) > y/a}
     \frac{A^{\omega(b)}(D,\rad(b))}{b^2\rad(b)}
      +
       \sums[a > y^{2/3}][\text{squarefree}] %\sumss[\flat][a > y^{2/3}] 
        A^{\omega(a)}\frac{(D,a)}{a^2}
        \sum_{b \ge 1}
         \frac{A^{\omega(b)}}{b^2}.
\]
%
By \eqref{eq:omegasfbnda} and \eqref{eq:omegaradbnd}, the first 
double sum is 
\[
    \ll_A
     (1 + A)^{\omega(D)} y^{-2/3 + o(1)}  
      \sums[a \le y^{2/3}] [\text{squarefree}] %\sumss[\flat][a \le y^{2/3}] 
       A^{\omega(a)}\frac{(D,a)}{a^{4/3}}
       \ll_A
        (1 + A)^{2\omega(D)}\frac{y^{O(1/\log\log 3y)}}{y^{2/3}}.
\]
%
By \eqref{eq:omegasfbnd}, and since 
$\sum_{b \ge 1} (A^{\omega(b)}/b^2) \ll_A 1$, 
\[
 \sums[a > y^{2/3}][\text{squarefree}] %\sumss[\flat][a > y^{2/3}] 
  A^{\omega(a)}\frac{(D,a)}{a^2}
  \sum_{b \ge 1}
   \frac{A^{\omega(b)}}{b^2}
    \ll_A 
     (1 + A)^{\omega(D)}\frac{y^{O(1/\log\log 3y)}}{y^{2/3}}.
\]
%
Combining gives \eqref{eq:realbnd}.
%
The left-hand side of \eqref{eq:realbnd2} is at most
\[
 \sums[a \le y][\text{squarefree}] % \sumss[\flat][a \le y] 
  \frac{A^{\omega(a)}}{a}
   \sum_{b^2\rad(b) \le y} A^{\omega(b)};
\]
applying \eqref{eq:omegasfbnda} and \eqref{eq:smth13} gives 
\eqref{eq:realbnd2}.

We now prove our claim.
%
For \eqref{eq:omegasfbnd}, we first consider the case $D = 1$.
%
Note that   
\begin{equation}
 \label{eq:mert}
 \sums[n_1 \le y][\text{squarefree}] %\sumss[\flat][n_1 \le y] 
  \frac{(A - 1)^{\omega(n_1)}}{n_1}
   \le 
    \prod_{p \le y}
     \bigg(1 + \frac{A - 1}{p}\bigg)
      \le 
       \prod_{p \le y} 
        \bigg(1 + \frac{1}{p}\bigg)^{A - 1}
         \ll_A
          (\log 3y)^{A - 1},
\end{equation}
because $1 + 1/p < \e^{1/p}$ and 
$\sum_{p \le y} 1/p = \log\log 3y + O(1)$ Mertens' theorem.
%
Now, 
\[
 \sums[n > y][\text{squarefree}] %\sumss[\flat][n > y] 
  \frac{A^{\omega(n)}}{n^2}
  =
   \sums[n > y][\text{squarefree}] %\sumss[\flat][n > y]
    \frac{1}{n^2}
     \sum_{n_1 \mid n} (A - 1)^{\omega(n_1)}
      \le 
       \sums[n_1 \ge 1][\text{squarefree}] %\sumss[\flat][n_1 \ge 1]
        \frac{(A - 1)^{\omega(n_1)}}{n_1^2}
         \sums[m > y/n_1][\text{squarefree}] %\sumss[\flat][m > y/n_1]
          \frac{1}{m^2},
\]
the inner sum being $O(n_1/y)$ for $n_1 \le y$ and $O(1)$ for 
$n_1 > y$. 
%
Thus, 
\[
 \sums[n > y][\text{squarefree}]%\sumss[\flat][n > y] 
  \frac{A^{\omega(n)}}{n^2}
  \ll_A
   \frac{(\log 3y)^{A - 1}}{y}
    +
     \sums[n_1 > y][\text{squarefree}] %\sumss[\flat][n_1 > y]
      \frac{(A - 1)^{\omega(n_1)}}{n_1^2}.
\]
%
If $A \le 2$ then this last sum is $O(1/y)$; otherwise, 
repeating the argument as many times as necessary gives
\[
 \sums[n > y] [\text{squarefree}] % \sumss[\flat][n > y] 
  \frac{A^{\omega(n)}}{n^2}
  \ll_A
   \frac{(\log 3y)^{A - 1}}{y}.
\]
%
It follows that, for any integer $d \ge 1$, 
\[
  \sums[n > y, \, d \mid n][\text{squarefree}] %\sumss[\flat]
   \frac{A^{\omega(n)}}{n^2}
    \ll_A
     \frac{A^{\omega(d)}}{d}
      \cdot 
       \frac{(\log 3y)^{A - 1}}{y}.
\]
%
For any integer $D \ge 1$, we trivially have 
$(D,n) \le \sum_{d \mid D, \, d \mid n} d$, and hence
%
\[
 \sums[n > y][\text{squarefree}] %\sumss[\flat][n > y]
  A^{\omega(n)} \frac{(D,n)}{n^2}
   \le 
    \sums[d \mid D][\text{squarefree}] %\sumss[\flat][d \mid D] d 
     \sums[n > y, \, d \mid n][\text{squarefree}]%\sumss[\flat][n > y][d \mid n] 
      \frac{A^{\omega(n)}}{n^2} 
       \ll_A 
        \frac{(\log 3y)^{A - 1}}{y}
         \sums[d \mid D][\text{squarefree}] \frac{A^{\omega(d)}}{d}. %\sumss[\flat][d \mid D] A^{\omega(d)}.
\]
%
Since 
$
 \sums[d \mid D, \, \text{squarefree}] A^{\omega(d)} 
  = 
   (1 + A)^{\omega(D)}
$
and 
$(\log 3y)^{A - 1} \ll_A y^{O(1/\log\log 3y)}$, 
this gives \eqref{eq:omegasfbnd}.
%
The bound \eqref{eq:omegasfbnda} follows from \eqref{eq:mert} and 
$(D,n) \le \sum_{d \mid D, \, d \mid n} d$.

For \eqref{eq:omegaradbnd}, we use the following ancillary bound.
%
We have 
\begin{equation}
 \label{eq:omegabndanc2}
  \sums[n > y][\rad(n) = m] \frac{1}{n} 
   \ll
    \frac{y^{O(1/\log\log 3y)}}{y}, 
\end{equation}
uniformly for integers squarefree integers $m \ge 1$.
%
To establish \eqref{eq:omegabndanc2}, we use an estimate involving 
smooth numbers: for $y \ge z \ge 2$, let 
\[
 \Psi(y,z) \defeq \#\{n \le y : p \mid n \implies p \le z\} 
\]
denote the number of $z$-smooth positive integers $n \le y$.
%
The following can be found in \cite[(1.19)]{GRA:08a}: for 
$y \ge z \ge 2$,    
\begin{equation}
 \label{eq:smth}
  \log \Psi(y,z)
   =
    \bigg(\frac{\log y}{\log z}\bigg)
    g\bigg(\frac{z}{\log y}\bigg)
      \bigg(1 + O\bigg(\frac{1}{\log z} + \frac{1}{\log\log x}\bigg)\bigg),
\end{equation}
where $g(w) = \log(1 + w) + w\log(1 + 1/w) \le w + 1$ ($w > 0$).
%
Noting that 
\[
 \sums[n \ge 1][\rad(n) = m] 
  \frac{1}{n^{1/2}}
   =
    \frac{1}{m^{1/2}}
     \sums[n \ge 1][\rad(n) \mid m] \frac{1}{n^{1/2}}
      =
       \frac{1}{m^{1/2}}
        \prod_{p \mid m}
         \bigg(\sum_{a \ge 0} \frac{1}{p^{a/2}}\bigg)
         =
          \prod_{p \mid m} \bigg(\frac{1}{p^{1/2} - 1}\bigg),
\]
we see that 
\begin{equation}
 \label{eq:ngeqy2}
  \sums[n > y^2][\rad(n) = m] \frac{1}{n}
   \le 
    \sums[n > y^2][\rad(n) = m] 
     \frac{1}{n}\bigg(\frac{n}{y^{2}}\bigg)^{1/2}
      \le 
       \frac{1}{y}
        \sums[n \ge 1][\rad(n) = m] 
         \frac{1}{n^{1/2}}
          \ll
           \frac{1}{y}.
\end{equation}
%
If $m > y^2$ then 
$
 \sum_{n > y, \, \rad(n) = m} 1/n 
  = 
   \sum_{n > y^2, \, \rad(n) = m} 1/n
$, and we are done.
%
Let us assume, then, that $y^2 \ge m$.
%
Let $\ell_1,\ldots,\ell_r$ denote the prime divisors 
of $m$, and let $p_1 = 2 < p_2 = 3 < \cdots < p_r$ 
denote the $r$ smallest primes.
%
Note that 
$
 \#\{(\alpha_1,\ldots,\alpha_r) \in \NN^r :
       \ell_1^{\alpha_1}\cdots \ell_r^{\alpha_r} \le y^2\}
   \le 
    \#\{(\alpha_1,\ldots,\alpha_r) \in \NN^r :
       p_1^{\alpha_1}\cdots p_r^{\alpha_r} \le y^2\}
$, i.e.\ note that 
$
 \#\{n \le y^2 : \rad(n) = m\}
  \le 
   \#\{n \le y^2 : \rad(n) = p_1\cdots p_r\}
$.
%
Since $y^2 \ge m \ge p_1\cdots p_r$,  
we have $4\log y^2 \ge 4\log m \ge 4\log (p_1\cdots p_r) > p_r$ by 
one of Chebyshev's bounds for primes, so if 
$\rad(n) = p_1\cdots p_r$, then $n$ is $y$-smooth, where 
$y = 4\log y^2$.
%
Therefore, 
\begin{equation}
 \label{eq:ylenley2}
 \sums[y < n \le y^2][\rad(n) = m] \frac{1}{n}
  <
   \frac{1}{y}
    \sums[n \le y^2][\rad(n) = m] 1
     \le 
      \frac{1}{y}
       \sums[n \le y^2][\rad(n) = p_1\cdots p_r] 1
        \le 
         \frac{\Psi(y^2,4\log y^2)}{y}
          \ll 
           \frac{y^{O(1/\log\log 3y)}}{y},
\end{equation}
where the last bound follows, upon exponentiating, from 
\eqref{eq:smth}.
%
Combining \eqref{eq:ngeqy2} and \eqref{eq:ylenley2} gives 
\eqref{eq:omegabndanc2}.

The left-hand side of \eqref{eq:omegaradbnd} is at most
\[
     \sums[m \le y^{1/3}][\text{squarefree}] %\sumss[\flat][m \le y^{1/3}] 
      \frac{A^{\omega(m)}(D,m)}{m}
       \sums[n^2 > y^{2/3}][\rad(n) = m]
        \frac{1}{n^2}
       +
         \sums[m > y^{1/3}][\text{squarefree}]%\sumss[\flat][m > y^{1/3}] 
          \frac{A^{\omega(m)}(D,m)}{m}
           \sums[n \ge 1][\rad(n) = m] \frac{1}{n^2}.
\]
%
By \eqref{eq:omegasfbnda} and \eqref{eq:omegabndanc2} (note that 
$1/n^2 < 1/(y^{1/3}n)$ when $n^2 > y^{2/3}$), we have 
\[
  \sums[m \le y^{1/3}][\text{squarefree}] %\sumss[\flat][m \le y^{1/3}] 
   \frac{A^{\omega(m)}(D,m)}{m}
    \sums[n^2 > y^{2/3}][\rad(n) = m]
     \frac{1}{n^2}
      \ll_A
       (1 + A)^{\omega(m)}
        \frac{y^{O(1/\log\log 3y)}}{y^{2/3}};
\]
by \eqref{eq:omegasfbnd} (note that $1/m^3 < 1/(y^{1/3}m^2)$ when 
$m > y^{1/3}$), and since 
\[
  \sums[n \ge 1][\rad(n) = m] \frac{1}{n^2} 
   =
    \frac{1}{m^2}
     \sums[n \ge 1][\rad(n) \mid m] \frac{1}{n^2}
      =
       \frac{1}{m^2}
        \prod_{p \mid m}
         \bigg(\sum_{a \ge 0} \frac{1}{p^{2a}}\bigg)
          \ll
           \frac{1}{m^2},
\]
we have           
\[
  \sums[m > y^{1/3}][\text{squarefree}] %\sumss[\flat][m > y^{1/3}] 
   \frac{A^{\omega(m)}(D,m)}{m}
    \sums[n \ge 1][\rad(n) = m] \frac{1}{n^2}
     \ll
      \sums[m > y^{1/3}][\text{squarefree}] %\sumss[\flat][m > y^{1/3}]
       \frac{A^{\omega(m)}(D,m)}{m^3}
        \ll_A
         \frac{(1 + A)^{\omega(D)}}{y^{2/3}}.
\]
%
Combining gives \eqref{eq:omegaradbnd}.

For \eqref{eq:smth13}, we note that since 
$\rad(n)^3 \le n^2\rad(n)$ and 
$A^{\omega(n)} = A^{\omega(\rad(n))}$, 
\[
 \sum_{n^2\rad(n) \le y} 
  A^{\omega(n)} 
   \le 
    \sums[a \le y^{1/3}][\text{squarefree}] %\sumss[\flat][a \le y^{1/3}] 
     A^{\omega(a)}
     \sums[b^2 \le y][\rad(b) = a] 1.
\]
%
An argument similar to the one leading up to \eqref{eq:ylenley2} 
shows that, uniformly for $a \le y^{1/3}$, we have 
$\sum_{b^2 \le y, \, \rad(b) = a} 1 \ll y^{O(1/\log\log 3y)}$, and 
\[
 \sums[a \le y^{1/3}][\text{squarefree}] %\sumss[\flat][a \le y^{1/3}] 
  A^{\omega(a)} 
  \le 
   y^{1/3}
    \sums[a \le y^{1/3}][\text{squarefree}] %\sumss[\flat][a \le y^{1/3}] 
     \frac{A^{\omega(a)}}{a}
     \ll_A
      y^{1/3 + O(1/\log\log 3y)}
\]
by \eqref{eq:omegasfbnda}.
%
Combining gives \eqref{eq:smth13}.
\end{proof}

To prove Proposition \ref{prop:sssa}, we express 
$\mathfrak{S}_{\bh}$ as a series.
%
To this end, let us introduce some notation and establish some 
basic inequalities.
%
Let a nonempty, finite set $\bh \subseteq \ZZ$ be given, and let 
$k \defeq \card \bh$.
%
Recall that $\T_{\bh}(2^{\alpha})$ is defined (and nonempty when 
$\bh = \{0\}$) for $\alpha \ge 2$, and for $p \equiv 3 \bmod 4$, 
$\V_{\bh}(p^{\alpha})$ is defined (and nonempty when 
$\bh = \{0\}$) for $\alpha \ge 1$.
%
Let us set $\T_{\bh}(1) \defeq \{1\}$ and 
$\T_{\bh}(2) \defeq \{1,2\}$ for completeness.
%
For $p \not\equiv 1 \bmod 4$ and $\alpha \ge 1$, we may then 
define 
\begin{equation}
 \label{eq:defiota}
 \epsilon_{\bh}(p^{\alpha})
  \defeq 
   \bigg(\frac{\card \T_{\bz}(p^{\alpha})}{p^{\alpha}}\bigg)^{-k}
    \bigg(\frac{\card \T_{\bh}(p^{\alpha})}{p^{\alpha}}\bigg)
  -
     \bigg(\frac{\card \T_{\bz}(p^{\alpha - 1})}{p^{\alpha - 1}}\bigg)^{-k}
      \bigg(\frac{\card \T_{\bh}(p^{\alpha - 1})}{p^{\alpha - 1}}\bigg). 
\end{equation}
%
Note that $\epsilon_{\bh}(2^2) = 0$ by definition.

\begin{lemma}
 \label{lem:iotah}
%
Let $\bh$ be a nonempty, finite set of integers, and let 
$k \defeq \card \bh$.

\textup{(}a\textup{)}
%
For $p \equiv 3 \bmod 4$ and {\bfseries even} $\alpha \ge 2$, we 
have $\epsilon_{\bh}(p^{\alpha}) = 0$.

\textup{(}b\textup{)}
%
For $p \not\equiv 1 \bmod 4$, we have
\begin{equation}
 \label{eq:iotlem1}
  \epsilon_{\bh}(p)
   \ll_k
    \frac{(\det(\bh),p)}{p^2}.
\end{equation}

\textup{(}c\textup{)}
%
For $p \not\equiv 1 \bmod 4$ and $\alpha \ge 1$, we have 
\begin{equation}
 \label{eq:iotlem2}
  \epsilon_{\bh}(p^{\alpha})
   \ll_k 
    \frac{(\det(\bh),p)}{p^\alpha}.
\end{equation}

\textup{(}d\textup{)}
%
For $\beta \ge 1$, we have  
\begin{equation}
 \label{eq:iotlem3}
   \delta_{\bz}(2)^{-k}\delta_{\bh}(2)  
   =
    1 + \sum_{\alpha = 2}^{\beta} \epsilon_{\bh}(2^{\alpha})
      + O_k\bigg(\frac{1}{2^{\beta}}\bigg).
\end{equation}
%
For $p \equiv 3 \bmod 4$ and $\beta \ge 1$, we have 
\begin{equation}
 \label{eq:iotlem4}
   \delta_{\bz}(p)^{-k}\delta_{\bh}(p)  
   =
    1 + \sum_{\alpha = 1}^{\beta} \epsilon_{\bh}(p^{2\alpha - 1})
      + O_k\bigg(\frac{1}{p^{2\beta}}\bigg).
\end{equation}
\end{lemma}

\begin{proof}
%
(a)
%
Let $p \equiv 3 \bmod 4$ and let $\alpha \ge 1$.
%
As can be seen from Proposition \ref{prop:Sp3h}, 
\eqref{eq:delthp3} and part (c), we have 
\begin{equation}
 \label{eq:iotalempf1}
 \frac{\card \V_{\bz}(p^{\alpha})}{p^{\alpha}}
  =
   \bigg(1 + \frac{1}{p}\bigg)^{-1}
    \bigg(1 - \frac{1}{p^{\alpha + \alpha \bmod 2}}\bigg).
\end{equation}
%
For even $\alpha$ we therefore have  
\[
 \epsilon_{\bh}(p^{\alpha})
  =
   \bigg(1 + \frac{1}{p}\bigg)^k
    \bigg(1 - \frac{1}{p^{\alpha}}\bigg)^k
     \bigg(
      \frac{\card \V_{\bh}(p^{\alpha})}{p^{\alpha}}
     -
        \frac{\card \V_{\bh}(p^{\alpha - 1})}{p^{\alpha - 1}}
     \bigg),
\]
and as we noted following \eqref{eq:VW} and \eqref{eq:VWb}, 
$
 \card \V_{\bh}(p^{\alpha})/p^{\alpha} 
  - 
   \card \V_{\bh}(p^{\alpha - 1})/p^{\alpha - 1}
    = 
     0
$. 

(b)
%
Consider $p \equiv 3 \bmod 4$ (the case $p = 2$ is similar).
%
Let $\alpha \ge 1$.
%
Define $\eta_{\bh}(p^{\alpha})$ and $\kappa_{\bh}(p)$ as the 
numbers given by the relations
\begin{equation}
 \label{eq:defetakappa}
 \frac{\card \V_{\bh}(p^{\alpha})}{p^{\alpha}}
  \eqdef 
   \delta_{\bh}(p) + \eta_{\bh}(p^{\alpha}) 
    \quad 
     \text{and}
      \quad 
 \delta_{\bh}(p) 
  \eqdef
   \bigg(1 + \frac{1}{p}\bigg)^{-1}
    \bigg(1 - \frac{\kappa_{\bh}(p)}{p}\bigg).
\end{equation}
%
Note that by Proposition \ref{prop:Sp3h}, \eqref{eq:Vhdeltp3bnd} 
and part (c),  
$
 |\eta_{\bh}(p^{\alpha})| 
  < 
   k/p^{\alpha + (\alpha \bmod 2)}
$ and 
$\kappa_{\bh}(p) \le \min\{k - 1,p\}$, with 
$\kappa_{\bh}(p) = k - 1$ if $p \nmid \det(\bh)$.
%
Also, $\kappa_{\bh}(p) \ge -1$ (because $\delta_{\bh}(p) \le 1$).
%
Since $\alpha + (\alpha \bmod 2) \ge 2$, we have 
\[
 \frac{\card \V_{\bh}(p^{\alpha})}{p^{\alpha}}
  =
   \bigg(1 + \frac{1}{p}\bigg)^{-1}
    \bigg(1 - \frac{\kappa_{\bh}(p)}{p} + O\bigg(\frac{k}{p^2}\bigg)\bigg).
\]
%
In the special case $\bh = \{0\}$ we can take 
$\kappa_{\bh}(p) = 0$.
%
We therefore have  
\begin{align*}
 \bigg(\frac{\card \V_{\bz}(p^{\alpha})}{p^{\alpha}}\bigg)^{-k}
  \frac{\card \V_{\bh}(p^{\alpha})}{p^{\alpha}}
  &
   =
    \bigg(1 + \frac{1}{p}\bigg)^{k - 1}
       \bigg(1 - \frac{\kappa_{\bh}(p)}{p} + O_k\bigg(\frac{1}{p^2}\bigg)\bigg)
   \\
  &
   =
     \bigg(1 + \frac{k - 1}{p} +O_k\bigg(\frac{1}{p^2}\bigg)\bigg)
      \bigg(1 - \frac{\kappa_{\bh}(p)}{p} + O_k\bigg(\frac{1}{p^2}\bigg)\bigg)
 \\
   & 
    = 
      1 + \frac{k - 1 - \kappa_{\bh}(p)}{p} + O_k\bigg(\frac{1}{p^2}\bigg).
\end{align*}

Writing $\xi_{\bh}(p) \defeq k - 1 - \kappa_{\bh}(p)$, we have
\[
%  \bigg|
  \bigg(\frac{\card \V_{\bz}(p^{\alpha})}{p^{\alpha}}\bigg)^{-k}
   \frac{\card \V_{\bh}(p^{\alpha})}{p^{\alpha}}
  -
    1
%  \bigg|
  \ll_k 
%    \frac{k - 1 - \kappa_{\bh}(p)}{p} + \frac{A_k}{p^2}
%     =
     \frac{\xi_{\bh}(p)}{p} + \frac{1}{p^2}.
\]
%
If $p \mid \det(\bh)$ then 
$\xi_{\bh}(p)/p = \xi_{\bh}(p)(\det(\bh),p)/p^2$, and if 
$p \nmid \det(\bh)$ then, as already noted, 
$\kappa_{\bh}(p) = k - 1$, i.e.\ $\xi_{\bh}(p) = 0$, so 
$\xi_{\bh}(p)/p = \xi_{\bh}(p)(\det(\bh),p)/p^2$ in any case.
%
Since, as already noted, 
$-1 \le \kappa_{\bh}(p) \le k - 1$, 
we have $0 \le \xi_{\bh}(p) \le k$.
%
Thus, 
\[
%  \bigg|
  \bigg(\frac{\card \V_{\bz}(p^{\alpha})}{p^{\alpha}}\bigg)^{-k}
   \frac{\card \V_{\bh}(p^{\alpha})}{p^{\alpha}}
  -
    1
%  \bigg|
  \ll_k 
   \frac{(\det(\bh),p)}{p^2} + \frac{1}{p^2} 
    \ll
     \frac{(\det(\bh),p)}{p^2}.
\]
%
For $\alpha = 1$, the left-hand side is equal to $\epsilon_{\bh}(p)$ 
(see \eqref{eq:defiota}), so this gives \eqref{eq:iotlem1}.

(c)
%
Consider $p \equiv 3 \bmod 4$ (the case $p = 2$ is similar).
%
Let $\alpha \ge 1$.
%
By (a) and (b), the result holds for $\alpha = 1$ and 
$\alpha \ge 2$ even, so we may assume that $\alpha \ge 3$ is odd.
%
In that case, using \eqref{eq:iotalempf1} in the definition 
\eqref{eq:defiota} of $\epsilon_{\bh}(p^{\alpha})$, we see that 
\begin{align*}
 \epsilon_{\bh}(p^{\alpha})
 & 
  =
   \bigg(1 + \frac{1}{p}\bigg)^k
    \bigg\{
     \bigg(1 - \frac{1}{p^{\alpha + 1}}\bigg)^{-k}
      \frac{\card \V_{\bh}(p^{\alpha})}{p^{\alpha}}
       -
        \bigg(1 - \frac{1}{p^{\alpha - 1}}\bigg)^{-k}
         \frac{\card \V_{\bh}(p^{\alpha - 1})}{p^{\alpha - 1}}
    \bigg\}
 \\
 & 
  =
   \bigg(1 + \frac{1}{p}\bigg)^k
    \bigg\{
     \frac{\card \V_{\bh}(p^{\alpha})}{p^{\alpha}}
       -
       \frac{\card \V_{\bh}(p^{\alpha - 1})}{p^{\alpha - 1}}
        +
         O_k\bigg(\frac{1}{p^{\alpha - 1}}\bigg) 
    \bigg\},   
\end{align*}
since, for any $\alpha \ge 1$, 
$(1 - 1/p^{\alpha})^{-k} = 1 + O_k(1/p^{\alpha})$ and 
$\card \V_{\bh}(p^{\alpha})/p^{\alpha} = O(1)$.
%
We deduce, from \eqref{eq:VW} and \eqref{eq:VWb}, that  
$\epsilon_{\bh}(p^{\alpha}) \ll_k 1/p^{\alpha - 1}$, which is   
\eqref{eq:iotlem2} in the case $p \mid \det(\bh)$.

Now consider the case $p \nmid \det(\bh)$.
%
Note that, by Proposition \ref{prop:Sp3h}, \eqref{eq:Vhdeltp3bnd} 
and part (c), we have, for any $\alpha \ge 1$,   
\[
 \frac{\card \V_{\bh}(p^{\alpha})}{p^{\alpha}}
  =
   \bigg(1 + \frac{1}{p}\bigg)^{-1}
    \bigg(1 - \frac{k - 1}{p} - \frac{k}{p^{\alpha + \alpha \bmod 2}}\bigg). 
\]
%
In view of this and (the special case) \eqref{eq:iotalempf1}, we 
have, for odd $\alpha \ge 3$,
\begin{align*}
 &  
 \epsilon_{\bh}(p^{\alpha})
  =
   \bigg(1 + \frac{1}{p}\bigg)^{k - 1}
    \bigg\{
     \bigg(1 - \frac{1}{p^{\alpha + 1}}\bigg)^{-k}
      \bigg(1 - \frac{k - 1}{p} - \frac{k}{p^{\alpha + 1}}\bigg)
 \\
 & \hspace{180pt}
       -
        \bigg(1 - \frac{1}{p^{\alpha - 1}}\bigg)^{-k}
         \bigg(1 - \frac{k - 1}{p} - \frac{k}{p^{\alpha - 1}}\bigg)
     \bigg\}.
\end{align*}
%
Since 
$
 (1 - 1/p^{\alpha + 1})^{-k} 
  = 
   1 + k/p^{\alpha + 1} + O_k(1/p^{\alpha + 2})
$,
we have 
\[
 \bigg(1 - \frac{1}{p^{\alpha + 1}}\bigg)^{-k}
  \bigg(1 - \frac{k - 1}{p} - \frac{k}{p^{\alpha + 1}}\bigg)
   =
    1 - \frac{k - 1}{p} + O_k\bigg(\frac{1}{p^{\alpha + 2}}\bigg);
\]
similarly, 
\[
 \bigg(1 - \frac{1}{p^{\alpha - 1}}\bigg)^{-k}
  \bigg(1 - \frac{k - 1}{p} - \frac{k}{p^{\alpha - 1}}\bigg)
   =
    1 - \frac{k - 1}{p} + O_k\bigg(\frac{1}{p^{\alpha}}\bigg).
\]
%
Combining gives $\epsilon_{\bh}(p^{\alpha}) \ll_k 1/p^{\alpha}$, 
i.e.\ \eqref{eq:iotlem2}, for odd $\alpha \ge 3$.

(d)
%
Consider $p \equiv 3 \bmod 4$ (the case $p = 2$ is similar).
%
Let $\beta \ge 1$.
%
We have
\[
 1 + \sum_{\alpha = 1}^{\beta} \epsilon_{\bh}(p^{2\alpha - 1})
  =
   1 + \sum_{\alpha = 1}^{2\beta} \epsilon_{\bh}(p^{\alpha})
  =
    \bigg(\frac{\card \V_{\bz}(p^{2\beta})}{p^{2\beta}}\bigg)^{-k}
     \bigg(\frac{\card \V_{\bh}(p^{2\beta})}{p^{2\beta}}\bigg),
\]
because $\epsilon_{\bh}(p^{\alpha}) = 0$ for $\alpha$ even (by 
(a)), and the middle sum telescopes.
%
Now, Proposition \ref{prop:Sp3h} (c) gives
$\delta_{\bz}(p)^{-k} = (1 + 1/p)^k$, and by definition of 
$\eta_{\bh}(p^{2\beta})$ (see \eqref{eq:defetakappa}), 
$
 \delta_{\bh}(p) 
  =
   \big(\card \V_{\bh}(p^{2\beta})/p^{2\beta}\big) - \eta_{\bh}(p^{2\beta})
$.
%
With these substitutions, and \eqref{eq:iotalempf1}, we verify 
that  
\begin{align*}
 &   
  \delta_{\bz}(p)^{-k}\delta_{\bh}(p) 
 - 
    \bigg(\frac{\card \V_{\bz}(p^{2\beta})}{p^{2\beta}}\bigg)^{-k}
     \bigg(\frac{\card \V_{\bh}(p^{2\beta})}{p^{2\beta}}\bigg)
%   \\ 
%  & \hspace{15pt}
%   =   
%     \bigg(1 + \frac{1}{p}\bigg)^{k}
%      \bigg(\frac{\card \V_{\bh}(p^{2\beta})}{p^{2\beta}} - \eta_{\bh}(p^{2\beta})\bigg)
%       -
%        \bigg(1 + \frac{1}{p}\bigg)^{k}
%         \bigg(1 - \frac{1}{p^{2\beta}}\bigg)^{-k}
%          \frac{\card \V_{\bh}(p^{2\beta})}{p^{2\beta}}
  \\
 & \hspace{30pt}
  =   
    \frac{\card \V_{\bh}(p^{2\beta})}{p^{2\beta}}
     \bigg(1 + \frac{1}{p}\bigg)^{k}
      \bigg(1 - \bigg(1 - \frac{1}{p^{2\beta}}\bigg)^{-k} - \eta_{\bh}(p^{2\beta})\bigg).
\end{align*}
% 
Now, $\card \V_{\bh}(p^{2\beta})/p^{2\beta} \le 1$, 
$(1 + 1/p)^{k} \ll_k 1$, 
$(1 - 1/p^{2\beta})^{-k} = 1 + O_k(1/p^{2\beta})$, and as noted in 
(b), Proposition \ref{prop:Sp3h}, \eqref{eq:Vhdeltp3bnd} and 
part (c) show that $|\eta_{\bh}(p^{2\beta})| < k/p^{2\beta}$.
%
Combining gives \eqref{eq:iotlem4}.
\end{proof}

For $n \in \NN$ such that $p \mid n$ implies 
$p \not\equiv 1 \bmod 4$, we extend \eqref{eq:defiota} by defining 
\[
 \epsilon_{\bh}(n)
  \defeq 
   \prod_{p^{\alpha} \emid \, n \,\,} \epsilon_{\bh}(p^{\alpha}).
\]
%
% (Note that $\epsilon_{\bh}(1) \defeq 1$ by convention.)
%
For such $n$, Lemma \ref{lem:iotah} (b) and (c) give
\begin{equation}
 \label{eq:iotad}
  |\epsilon_{\bh}(n)| 
   \le 
    A_k^{\omega(n)}\frac{(\det(\bh),\rad(n))}{n\sqfr(n)},   
\end{equation}
provided $A_k$ is sufficiently large in terms of $k$.
%
Since $\epsilon_{\bh}(2) = 0$ by definition, and by 
Lemma \ref{lem:iotah} (a), $\epsilon_{\bh}(n) = 0$ if either 
$\nu_2(n) = 1$ or $\nu_p(n)$ is even (and nonzero) for some 
$p \equiv 3 \bmod 4$.
%
Letting 
$
 \cN_1
  \defeq 
   \{n \in \cN : p \mid n \implies p \not\equiv 1 \bmod 4\}
$, 
where $\cN$ is as in \eqref{eq:defcN}, we define 
\begin{equation}
 \label{eq:defcD}
  \cD \defeq \cN_1 \cup \{2n : n \in \cN_1, 2 \mid n\}. 
\end{equation}
%
Thus, 
\[
  \cD 
   =
    \big\{2^{\alpha}p_1^{2\alpha_1 - 1}\cdots p_r^{2\alpha_r - 1} : \alpha \ge 0, \alpha \ne 1, r,\alpha_i \ge 1, p_i \equiv 3 \bmod 4 \text{ ($i \le r$)}\big\},
\]
and $\epsilon_{\bh}(n) = 0$ unless $n \in \cD$.
%
By definition \eqref{eq:defsssP} and Lemma \ref{lem:iotah} (d), 
\begin{equation}
 \label{eq:ssassum}
 \mathfrak{S}_{\bh}
  =
%  \prod_{p \not\equiv 1 \bmod 4}
%   \delta_{\bz}(p)^{-k}\delta_{\bh}(p)
%    =
    \bigg(1 + \sum_{\alpha \ge 2} \epsilon_{\bh}(2^{\alpha})\bigg)
     \prod_{p \not\equiv 1 \bmod 4}
      \bigg(1 + \sum_{\alpha \ge 1} \epsilon_{\bh}(p^{2\alpha - 1})\bigg)
   =
        1 + \sum_{d \in \cD} \epsilon_{\bh}(d),
\end{equation}
the last sum being absolutely convergent in view of 
Lemma \ref{lem:omegabnd} and \eqref{eq:iotad}.

For the purposes of stating and proving the next lemma, we define 
{\small 
% \begin{equation}
%  \label{eq:defiotaj}
\[
 \epsilon_{\bh}(p^{\alpha};j)
  \defeq 
   \bigg(\frac{\card \T_{\bz}(p^{\alpha})}{p^{\alpha}}\bigg)^{-j}
    \bigg(\frac{\card \T_{\bh}(p^{\alpha})}{p^{\alpha}}\bigg)
  -
     \bigg(\frac{\card \T_{\bz}(p^{\alpha - 1})}{p^{\alpha - 1}}\bigg)^{-j}
      \bigg(\frac{\card \T_{\bh}(p^{\alpha - 1})}{p^{\alpha - 1}}\bigg), 
% \end{equation}
\]
}

\noindent 
for $p \not\equiv 1 \bmod 4$, $\alpha \ge 1$, and $j \ge 1$; we 
then set 
$
 \epsilon_{\bh}(n;j) 
  \defeq \prod_{p^{\alpha} \emid n} 
   \epsilon_{\bh}(p^{\alpha};j)
$
for $n$ composed of primes $p \not\equiv 1 \bmod 4$.
%
Thus, $\epsilon_{\bh}(n) = \epsilon_{\bh}(n;j)$ when $j = \card \bh$.

\begin{lemma}
 \label{lem:cancel}
%
Set $\bo \defeq \emptyset$, or set $\bo \defeq \{0\}$.
%
Let $n \ge 2$ be such that $p \mid n$ implies 
$p \not\equiv 1 \bmod 4$, and let $R_1,\ldots,R_k$ be complete 
residue systems modulo $n$.
%
We have 
\[
 \sum_{h_1 \in R_1} 
  \cdots 
   \sum_{h_k \in R_k}
    \epsilon_{\bo \cup \bh}(n; \ocard \bo + k)
      =
       0,
\]
where $\bh = \{h_1,\ldots,h_k\}$ in the summand. 
%
\textup{(}Note that we may have $\card \bh < k$ here.\textup{)}
\end{lemma}

\begin{proof}
%
Let $p \not\equiv 1 \bmod 4$, $\alpha \ge 1$.
%
Suppose $\bh = \{h_1,\ldots,h_k\}$ and 
$\bh' = \{h_1',\ldots,h_k'\}$  
satisfy $h_i \equiv h_i' \bmod p^{\alpha}$, and hence 
$h_i \equiv h_i' \bmod p^{\alpha - 1}$ as well, for 
$i = 1,\ldots,k$.
%
For $p \equiv 3 \bmod 4$, it is clear from \eqref{eq:defVh} that
$
 \card \V_{\bo \cup \bh}(p^{\beta}) 
  = 
   \card \V_{\bo \cup \bh'}(p^{\beta})
$ 
for $\beta = \alpha$, and for $\beta = \alpha - 1$ as well.
%
Thus, 
$
 \epsilon_{\bo \cup \bh}(p^{\alpha}; \ocard \bo + k)
 = 
  \epsilon_{\bo \cup \bh'}(p^{\alpha}; \ocard \bo + k)
$.
%
Similarly, we have   
$
 \epsilon_{\bo \cup \bh}(2^{\alpha}; \ocard \bo + k) 
  = 
   \epsilon_{\bo \cup \bh'}(2^{\alpha}; \ocard \bo + k)
$
(see \eqref{eq:defTh}).
%
Therefore, by the Chinese remainder theorem, 
\[
 \sum_{h_1 \in R_1} 
  \cdots 
   \sum_{h_k \in R_k}
    \epsilon_{\bo \cup \bh}(n;\ocard \bo + k)
      =
       \prod_{p^{\alpha} \emid \, n \,} 
        \bigg(
         \sum_{h_1 \in \ZZ_{p^{\alpha}}}
          \cdots 
           \sum_{h_k \in \ZZ_{p^{\alpha}}}
            \epsilon_{\bo \cup \bh}(p^{\alpha};\ocard \bo + k) 
         \bigg),
\]
where $\bh = \{h_1,\ldots,h_k\}$ in both summands, and 
$\ZZ_{p^{\alpha}} \defeq \{0,\ldots,p^{\alpha} - 1\}$.
%
It therefore suffices to show that  
\begin{equation}
 \label{eq:cancelp}
  \sum_{h_1 \in \ZZ_{p^{\alpha}}}
   \cdots 
    \sum_{h_k \in \ZZ_{p^{\alpha}}}
     \epsilon_{\bo \cup \bh}(p^{\alpha};\ocard \bo + k) 
      =
       0
\end{equation}
for all $p \not\equiv 1 \bmod 4$ and $\alpha \ge 1$.

Consider the case $\bo = \emptyset$.
%
For $p \equiv 3 \bmod 4$ and $\alpha \ge 1$, we have 
\[
 \sum_{h_1 \in \ZZ_{p^{\alpha}}}
  \cdots 
   \sum_{h_k \in \ZZ_{p^{\alpha}}}
    \card \V_{\bh}(p^{\alpha})
   =
   \sum_{a \in \ZZ_{p^{\alpha}}}
    \sums[h_1 \in \ZZ_{p^{\alpha}}][a + h_1 \in S_p][\nu_p(a + h_1) < \alpha]
     \cdots 
      \sums[h_k \in \ZZ_{p^{\alpha}}][a + h_k \in S_p][\nu_p(a + h_k) < \alpha] 1,
\]
as can be seen by applying the definition \eqref{eq:defVh} of 
$\V_{\bh}(p^{\alpha})$ and changing the order of summation.
%
For $i = 1,\ldots,k$, each sum over $h_i$ on the right-hand side 
enumerates a translation of $\V_{\bz}(p^{\alpha})$, so the 
entire sum (i.e.\ the left-hand side) is equal to 
$p^{\alpha}(\card \V_{\bz}(p^{\alpha}))^k$.
%
Whence 
\[
 \sum_{h_1 \in \ZZ_{p^{\alpha}}}
  \cdots 
   \sum_{h_k \in \ZZ_{p^{\alpha}}}
    \bigg(\frac{\card \V_{\bz}(p^{\alpha})}{p^{\alpha}}\bigg)^{-k}
     \bigg(\frac{\card \V_{\bh}(p^{\alpha})}{p^{\alpha}}\bigg)
      =
       p^{k\alpha}.
\]
%
%****************************************************************%
%************************* START DETAIL *************************%
%****************************************************************%
%
\begin{nixnix}
%
\begin{align*}
  \sum_{h_1 \in \ZZ_{p^{\alpha}}}
   \cdots 
    \sum_{h_k \in \ZZ_{p^{\alpha}}}
     \card \V_{\bh}(p^{\alpha})
  & = 
      \sum_{h_1 \in \ZZ_{p^{\alpha}}}
       \cdots 
        \sum_{h_k \in \ZZ_{p^{\alpha}}}
         \sums[a \in \ZZ_{p^{\alpha}}][\forall i, a + h_i \in S_p][\forall i, \nu_p(a + h_i) < \alpha] 1
 \\ 
  & = 
   \sum_{a \in \ZZ_{p^{\alpha}}}
    \sums[h_1 \in \ZZ_{p^{\alpha}}][a + h_1 \in S_p][\nu_p(a + h_1) < \alpha]
     \cdots 
      \sums[h_k \in \ZZ_{p^{\alpha}}][a + h_k \in S_p][\nu_p(a + h_k) < \alpha] 1
 \\
 & = 
   \sum_{a \in \ZZ_{p^{\alpha}}}
    \sums[a + h_1 \in \ZZ_{p^{\alpha}}][a + h_1 \in S_p][\nu_p(a + h_1) < \alpha]
     \cdots 
      \sums[a + h_k \in \ZZ_{p^{\alpha}}][a + h_k \in S_p][\nu_p(a + h_k) < \alpha] 1   
 \\
 & = 
    \sum_{a \in \ZZ_{p^{\alpha}}}
     (\card \V_{\bz}(p^{\alpha}))
      \cdots 
       (\card \V_{\bz}(p^{\alpha}))
 \\
 & = 
    p^{\alpha}(\card \V_{\bz}(p^{\alpha}))^k
\end{align*}
%
\end{nixnix}
%
%****************************************************************%
%************************** END DETAIL **************************%
%****************************************************************%
%
Since 
\[
 \sum_{h_1 \in \ZZ_{p^{\alpha}}}
  \cdots 
   \sum_{h_k \in \ZZ_{p^{\alpha}}}
    \card \V_{\bh}(p^{\alpha - 1})
      =
      p^k
       \sum_{h_1 \in \ZZ_{p^{\alpha - 1}}}
        \cdots 
         \sum_{h_k \in \ZZ_{p^{\alpha - 1}}}
          \card \V_{\bh}(p^{\alpha - 1}),
\]
we similarly have 
\[
 \sum_{h_1 \in \ZZ_{p^{\alpha}}}
  \cdots 
   \sum_{h_k \in \ZZ_{p^{\alpha}}}
    \bigg(\frac{\card \V_{\bz}(p^{\alpha - 1})}{p^{\alpha - 1}}\bigg)^{-k}
     \bigg(\frac{\card \V_{\bh}(p^{\alpha - 1})}{p^{\alpha - 1}}\bigg)
      =
       p^kp^{k(\alpha - 1)} 
        =
         p^{k\alpha}.
\]
%
Subtracting gives \eqref{eq:cancelp} for $\alpha \ge 1$.
%
In a similar fashion, we obtain \eqref{eq:cancelp} in the case 
$\bo = \{0\}$.
%
An analogous argument gives the same results for $p = 2$.
%
%****************************************************************%
%************************* START DETAIL *************************%
%****************************************************************%
%
\begin{nixnix}
%
\[
 \sum_{h_1 \in \ZZ_{p^{\alpha}}}
  \cdots 
   \sum_{h_k \in \ZZ_{p^{\alpha}}}
    \card \V_{\{0\} \cup \bh}(p^{\alpha})
   =
   \sums[a \in \ZZ_{p^{\alpha}}][a \in S_p][\nu_p(a) < \alpha]
    \sums[h_1 \in \ZZ_{p^{\alpha}}][a + h_1 \in S_p][\nu_p(a + h_1) < \alpha]
     \cdots 
      \sums[h_k \in \ZZ_{p^{\alpha}}][a + h_k \in S_p][\nu_p(a + h_k) < \alpha] 1
       =
        \big(\card \V_{\bz}(p^{\alpha})\big)^{1 + k}.
\]
%
\end{nixnix}
%
%****************************************************************%
%************************** END DETAIL **************************%
%****************************************************************%
%
%****************************************************************%
%************************* START DETAIL *************************%
%****************************************************************%
%
\begin{nixnix}
%
Applying the definition \eqref{eq:defTh} of 
$\T_{\bh}(2^{\alpha + 1})$ and changing the order of summation 
yields 
\begin{align*}
  \sum_{h_1 \in \ZZ_{2^{\alpha + 1}}}
   \cdots 
    \sum_{h_k \in \ZZ_{2^{\alpha + 1}}}
     \card \T_{\bh}(2^{\alpha + 1})
  & = 
      \sum_{h_1 \in \ZZ_{2^{\alpha + 1}}}
       \cdots 
        \sum_{h_k \in \ZZ_{2^{\alpha + 1}}}
         \sums[a \in \ZZ_{2^{\alpha + 1}}][\forall i, a + h_i \in S_2][\forall i, \nu_2(a + h_i) < \alpha] 1
 \\ 
  & = 
   \sum_{a \in \ZZ_{2^{\alpha + 1}}}
    \sums[h_1 \in \ZZ_{2^{\alpha + 1}}][a + h_1 \in S_2][\nu_2(a + h_1) < \alpha]
     \cdots 
      \sums[h_k \in \ZZ_{2^{\alpha + 1}}][a + h_k \in S_2][\nu_2(a + h_k) < \alpha] 1
 \\
 & = 
   \sum_{a \in \ZZ_{2^{\alpha + 1}}}
    \sums[a + h_1 \in \ZZ_{2^{\alpha + 1}}][a + h_1 \in S_2][\nu_2(a + h_1) < \alpha]
     \cdots 
      \sums[a + h_k \in \ZZ_{2^{\alpha + 1}}][a + h_k \in S_2][\nu_2(a + h_k) < \alpha] 1   
 \\
 & = 
    \sum_{a \in \ZZ_{2^{\alpha + 1}}}
     (\card \T_{\bz}(2^{\alpha + 1}))
      \cdots 
       (\card \T_{\bz}(2^{\alpha + 1}))
 \\
 & = 
    2^{\alpha + 1}(\card \T_{\bz}(2^{\alpha + 1}))^k.
\end{align*}
%
\end{nixnix}
%
%****************************************************************%
%************************** END DETAIL **************************%
%****************************************************************%
\end{proof}

In the proof of Proposition \ref{prop:sssa}, we also make use of 
basic lattice point counting arguments, as in the final two 
lemmas below.

\begin{lemma}
 \label{lem:dethap}
%
Let $\cD$ be as in \eqref{eq:defcD}.
%
Set $\bo \defeq \emptyset$, or set $\bo \defeq \{0\}$.
%
Fix an integer $k \ge 1$, and a number $M_k \ge 1$ that depends on 
$k$ only.
%
Also, fix $B \ge 1$.
%
For $y \ge 1$, we have
\begin{equation}
 \label{eq:lemdetbnd}
  \sums[d \in \cD][d > y] 
   \frac{M_k^{\omega(d)}}{d\sqfr(d)}
    \sum_{0 < h_1 < \cdots < h_k \le By}
    (\det(\bo \cup \bh),\rad(d))
      \ll_{k,B}
       y^{k - 2/3 + O(1/\log\log 3y)},
\end{equation}
where $\bh = \{h_1,\ldots,h_k\}$ in the summand.
\end{lemma}

\begin{proof}
%
Let $y \ge 1$.
%
Let us first show that, for any squarefree integer $c \ge 1$, 
\begin{equation}
 \label{eq:lemdethappf1}
 \underset{c \mid \det(\{0,h_1,\ldots,h_k\})}
  { 
   \sum_{0 < h_1 < \cdots < h_k \le By}
  } 1
  \le  
   k^{2\omega(c)}
    \bigg(\frac{(By)^k}{c} + O_k\big((By)^{k - 1}\big)\bigg).
\end{equation}
%
Let $h_0 = 0,h_1,\ldots,h_k$ be pairwise distinct integers and  
suppose that $c$ divides $\prod_{0 \le i < j \le k}(h_i - h_j)$.
%
Then, since $c$ is squarefree, there exist pairwise coprime 
positive integers $c_{ij}$ such that 
$c = \prod_{0 \le i < j \le k} c_{ij}$ and 
$c_{ij} \mid h_i - h_j$, $0 \le i < j \le k$.
%
Therefore, 
\[
 \underset{c \mid \det(\{h_0,h_1,\ldots,h_k\})}
  { 
   \sum_{0 < h_1 < \cdots < h_k \le By}
  } 1
   \le 
    \sums[c = c_{01}\cdots c_{(k-1)k}] 
     \hspace{5pt}
     \underset{0 \le i < j \le k - 1 \implies c_{ij} \mid h_i - h_j}
    { 
       \sum_{h_1 \in I_{By}}
        \sum_{h_2 \in I_{By}}
         \cdots 
          \sum_{h_{k - 1} \in I_{By}}
    } 
     \hspace{5pt}
      \sums[h_k \in I_{By}][0 \le i \le k - 1 \implies c_{ik} \mid h_i - h_k] 1,
\]
where on the right-hand side, the outermost sum is over all 
decompositions of $c$ as a product of $\binom{k + 1}{2}$ positive 
integers, and $I_{By} \defeq (0,By]$.

Consider the decomposition $c = c_{01}\cdots c_{(k - 1)k}$.
%
Let us define $c_{j} \defeq \prod_{i = 0}^{j - 1} c_{ij}$ for 
$j = 1,\ldots,k$.
%
Notice that $c = \prod_{j = 1}^k c_j$.
%
By the Chinese remainder theorem, the condition on $h_k$ in the 
innermost sum above is equivalent to $h_k$ being in some 
congruence class modulo $c_k$, uniquely determined by 
$h_0,h_1,\ldots,h_{k - 1}$.
%
The sum is therefore equal to $By/c_k + O(1)$.
%
Iterating this argument $k$ times we see that the inner sum over 
$h_1,\ldots,h_k$ is equal to  
\[
  \prod_{j = 1}^k
   \bigg(\frac{By}{c_j} + O(1)\bigg)
    =
     \frac{(By)^k}{c} + O_k((By)^{k - 1}).
\]
%
The bound \eqref{eq:lemdethappf1} follows by combining and noting 
that, since $c$ is squarefree, the number of ways of writing $c$ 
as a product of $\binom{k + 1}{2}$ positive integers is 
$\binom{k + 1}{2}^{\omega(c)}$, and that 
$\binom{k + 1}{2} \le k^2$.

For $\bh = \{h_1,\ldots,h_k\}$, with $h_1,\ldots,h_k$ pairwise 
distinct, nonzero integers, and any $d \in \NN$, we trivially have 
$
 (\det(\bo \cup \bh),\rad(d)) 
   \le 
    \sum_{c \mid \det(\{0,h_1,\ldots,h_k\}), \, \rad(d)} c
$.
%
If $h_1,\ldots,h_k \le By$ as well, then $p \mid c$ implies 
$p \le By$.
%
From this and \eqref{eq:lemdethappf1}, it follows that 
\[
 \sum_{0 < h_1 < \cdots < h_k \le By} 
  (\det(\bo \cup \bh,\rad(d))
   \ll_{k,B}
    y^k
     \sum_{c \mid \rad(d)} k^{2\omega(c)}
      +
        y^{k - 1}
         \sums[c \mid \rad(d)][p \mid c \implies p \le By]
           ck^{2\omega(c)},
\]
where $\bh = \{h_1,\ldots,h_k\}$ in the summand on the left.
%
Now, for $c \mid \rad(d)$ we have 
$k^{2\omega(c)} \le k^{2\omega(d)}$, and 
$\sum_{c \mid \rad(d)} 1 = 2^{\omega(d)}$.
%
Applying these bounds to the left-hand side of 
\eqref{eq:lemdetbnd}, we see that it is 
\begin{equation}
 \label{eq:lemdetpf1}
%   \sums[d \in \cD][d > y] 
%    \frac{A_k^{\omega(d)}}{d\sqfr(d)}
%     \sum_{0 < h_1 < \cdots < h_k \le By}
%     (\det(\bo \cup \bh),\rad(d))
      \ll_{k,B}
       y^k
        \sums[d \in \cD][d > y] 
         \frac{A_k^{\omega(d)}}{d\sqfr(d)}
        +
           y^{k - 1} 
            \sum_{d \in \cD}
             \frac{A_k^{\omega(d)}}{d\sqfr(d)}
              \sums[c \mid \rad(d)][p \mid c \implies p \le By] c, 
\end{equation}
where $A_k$, here and below, denotes a sufficiently large number 
depending on $k$, which may be a different number at each 
occurrence.

By definition \eqref{eq:defcD} of $\cD$, for every $d \in \cD$, we 
have $d = n$ or $d = 2n$ for some $n \in \cN$, where $\cN$ is 
as in \eqref{eq:defcN}. 
%
Therefore, as a direct consequence of Lemma \ref{lem:omegabnd}, we 
have 
\begin{equation}
 \label{eq:lemdetpf2}
 \sums[d \in \cD][d > y] 
  \frac{A_k^{\omega(d)}}{d\sqfr(d)}
   \ll_k 
    \frac{y^{O(1/\log\log 3y)}}{y^{2/3}}.
\end{equation}
%
More specifically, for every $d \in \cD$, we have 
$d = ab^2\rad(b)$ or $d = 2ab^2\rad(b)$ for some uniquely 
determined $a,b \in \NN$, where $a$ is squarefree and $(a,b) = 1$.
%
Furthermore, $d$ is not exactly divisible by $2$, and so we have 
$2 \nmid a$ in the case $d = ab^2\rad(b)$, while $2 \mid ab$ in 
the case $d = 2ab^2\rad(b)$.
%
In either case, we have the following:
$A_k^{\omega(d)} = A_k^{\omega(a)}A_k^{\omega(b)}$;  
$d\sqfr(d) = a^2b^2\rad(b)$ or 
$d\sqfr(d) = 2a^2b^2\rad(b)$; and $\rad(d) = a\rad(b)$.
%
Thus, if $c \mid \rad(d)$, then $c = c_1c_2$, where $c_1 \mid a$
and $c_2 \mid \rad(b)$.
%
Consequently,  
\[
 \sum_{d \in \cD}
  \frac{A_k^{\omega(d)}}{d\sqfr(d)}
   \sums[c \mid \rad(d)][p \mid c \implies p \le By] c
    \ll
     \sums[a \ge 1][\text{squarefree}] %\sumss[\flat][a \ge 1]
      \frac{A_k^{\omega(a)}}{a^2}
       \sum_{b \ge 1} \frac{A_k^{\omega(b)}}{b^2\rad(b)}
        \sums[c_1 \mid a][p \mid c_1 \implies p \le By] c_1
         \sums[c_2 \mid \rad(b)][p \mid c_2 \implies p \le By] c_2.
\]

Now,  
\[
 \sums[a \ge 1][\text{squarefree}] %\sumss[\flat][a \ge 1]
  \frac{A_k^{\omega(a)}}{a^2}
   \sums[c_1 \mid a][p \mid c_1 \implies p \le By] c_1
    \le 
     \sums[c_1 \ge 1][\text{squarefree}][p \mid c_1 \implies p \le By] %\sumss[\flat][c_1 \ge 1][p \mid c_1 \implies p \le By] 
      \frac{A_k^{\omega(c_1)}}{c_1}
       \sums[a_1 \ge 1][\text{squarefree}] %\sumss[\flat][a_1 \ge 1] 
        \frac{A_k^{\omega(a_1)}}{a_1^2}
         \ll_k
          \sums[c_1 \ge 1][\text{squarefree}][p \mid c_1 \implies p \le By] %\sumss[\flat][c_1 \ge 1][p \mid c_1 \implies p \le By] 
           \frac{A_k^{\omega(c_1)}}{c_1}; 
\]
as can be seen by writing $a = a_1c_1$ and changing order of 
summation; also 
\[
 \sums[c_1 \ge 1][\text{squarefree}][p \mid c_1 \implies p \le By] %\sumss[\flat][c_1 \ge 1][p \mid c_1 \implies p \le By] 
  \frac{A_k^{\omega(c_1)}}{c_1}
  \le 
   \prod_{p \le By}
    \bigg(1 + \frac{A_k}{p}\bigg)
%      \le 
%       \prod_{p \le By}
%        \bigg(1 + \frac{1}{p}\bigg)^{A_k}
        \ll_{k,B} (\log 3y)^{A_k}. 
\]
%
(See \eqref{eq:mert}.)
%
Next, note that since 
$
 \sum_{c_2 \mid \rad(b)} c_2 
  \le 
   \rad(b) \sum_{c_2 \mid \rad(b)} 1
    \le 
     2^{\omega(b)}\rad(b)
$, 
\[
 \sum_{b \ge 1} \frac{A_k^{\omega(b)}}{b^2\rad(b)}
  \sums[c_2 \mid \rad(b)][p \mid c_2 \implies p \le By] c_2 
   \le 
    \sum_{b \ge 1} \frac{A_k^{\omega(b)}}{b^2}
     \sum_{c_2 \mid \rad(b)} 1
      \le 
       \sum_{b \ge 1} \frac{A_k^{\omega(b)}}{b^2}
        \ll_k 1.
\]
%
%(again replacing $A_{k}$ by a bigger constant.) 
Combining all of this gives 
\begin{equation}
 \label{eq:lemdetpf3}
  \sum_{d \in \cD}
   \frac{A_k^{\omega(d)}}{d\sqfr(d)}
    \sums[c \mid \rad(d)][p \mid c \implies p \le By] c
     \ll_{k,B}
      (\log 3y)^{A_k}.
\end{equation}
%
Finally, we obtain \eqref{eq:lemdetbnd} by combining 
\eqref{eq:lemdetpf1} with \eqref{eq:lemdetpf2} and 
\eqref{eq:lemdetpf3}.
\end{proof}

\begin{lemma}
 \label{lem:lip}
%
Fix an integer $k \ge 1$ and a bounded convex set 
$\sC \subseteq \RR^k$.
%
For $y \ge 1$ we have 
$
 \#(y\sC \cap \ZZ^k)
  =
   y^k\vol(\sC) + O_{k,\sC}(y^{k - 1}).
$
\end{lemma}

\begin{proof}
This is a special case of \cite[pp.\ 128--129]{LAN:94}.
\end{proof}

\begin{proof}[Proof of Proposition \ref{prop:sssa}]
%
Fix an integer $k \ge 1$ and a bounded convex set 
$\sC \subseteq \Delta^k$, where 
$
 \Delta^k \defeq \{(x_1,\ldots,x_k) \in \RR^k : 0 < x_1 < \cdots < x_k\}
$ 
(see \eqref{eq:defsimplex}). 
%
Set $\bo \defeq \emptyset$ or set $\bo \defeq \{0\}$.
%
Let $y \ge 1$.
%
To ease notation throughout, let $\cH \defeq y\sC \cap \ZZ^k$, 
$\vbh = (h_1,\ldots,h_k)$, and $\bh = \{h_1,\ldots,h_k\}$.
%
Note that $0 < h_1 < \cdots < h_k \ll_{\sC} y$ for $\vbh \in \cH$.
%
Also, let $A_k$ stand for a sufficiently large number depending on 
$k$, which may be a different number at each occurrence.

In view of \eqref{eq:ssassum} we see, upon partitioning the sum 
over $d$ and changing order of summation, that  
\begin{equation}
 \label{eq:lemsssapf1}
 \sum_{\vbh \in \cH} \mathfrak{S}_{\bo \cup \bh}
  =
   \sum_{\vbh \in \cH} 1
    +
      \sums[d \in \cD][d \le y]
       \sum_{\vbh \in \cH} \epsilon_{\bo \cup \bh}(d)
      +
       \sums[d \in \cD][d > y]
        \sum_{\vbh \in \cH} \epsilon_{\bo \cup \bh}(d),
\end{equation}
with $\cD$ as defined in \eqref{eq:defcD}.
%
By Lemma \ref{lem:lip}, we have 
\begin{equation}
 \label{eq:volH}
  \sum_{\vbh \in \cH} 1
   =
    y^k \vol(\sC) + O_{k,\sC}(y^{k - 1}).
\end{equation}
%
By \eqref{eq:iotad} and Lemma \ref{lem:dethap}, we have 
\begin{equation}
 \label{eq:sum3}
 \sums[d \in \cD][d > y]
  \sum_{\vbh \in \cH} |\epsilon_{\bo \cup \bh}(d)|
   \le 
    \sums[d \in \cD][d > y]
     \sum_{\vbh \in \cH} A_k^{\omega(d)}\frac{(\det(\bo \cup \bh),\rad(d))}{d\sqfr(d)}
      \ll_{k,\sC} 
       y^{k - 1}\frac{y^{O(1/\log\log 3y)}}{y^{2/3}}.
\end{equation}

Consider the middle sum on the right-hand side of 
\eqref{eq:lemsssapf1}.
%
Let $d$ be any element of $\cD$ with $d \le y$, and partition 
$\RR^k$ into cubes 
\[
 C_{d,\vbt} 
  \defeq 
   \{(x_1,\ldots,x_k) \in \RR^k : t_id \le x_i < (t_i + 1)d, i = 1,\ldots,k\},
\]
with $\vbt \defeq (t_1,\ldots,t_k)$ running over $\ZZ^k$.
%
Each $\vbh \in \cH$ is a point in a unique cube of this form: we 
call $\vbh$ a {\em $d$-interior} point if this cube is entirely 
contained in $y\sC$, and $\vbh$ a {\em $d$-boundary} point if this 
cube has a nonempty intersection with the boundary of $y\sC$.
%
We partition $\cH$ into $d$-interior points and $d$-boundary 
points.
%
As $\vbh$ runs over all $d$-interior points of $\cH$, $h_i$ 
($i = 1,\ldots,k$) runs over a pairwise disjoint union of complete 
residue systems modulo $d$, none of which contain $0$.
%
By Lemma \ref{lem:cancel} (we have 
$\card(\bo \cup \bh) = \ocard \bo + k$ for each $\vbh \in \cH$), 
it follows that 
\begin{equation}
 \label{eq:2ndlast}
  \sums[d \in \cD][d \le y] 
   \sum_{\vbh \in \cH}
    \epsilon_{\bo \cup \bh}(d)
  =
   \sums[d \in \cD][d \le y]
    \sums[\vbh \in \cH][\text{$d$-boundary}]
     \epsilon_{\bo \cup \bh}(d).
\end{equation}

By \eqref{eq:iotad}, and the aforementioned trivial bound for 
$(\det(\bo \cup \bh),\rad(d))$,  
\begin{align*} 
 \sums[d \in \cD][d \le y]
  \sums[\vbh \in \cH][\text{$d$-boundary}]
  |\epsilon_{\bo \cup \bh}(d)|
 & 
    \le 
     \sums[d \in \cD][d \le y] \frac{A_k^{\omega(d)}}{d\sqfr(d)}
      \sums[\vbh \in \cH][\text{$d$-boundary}] 
      (\det(\bo \cup \bh),\rad(d))
 \\
 & 
         \le 
          \sums[d \in \cD][d \le y] \frac{A_k^{\omega(d)}}{d\sqfr(d)}
           \sum_{c \mid \rad(d)} c
            \sums[\vbh \in \cH][\text{$d$-boundary}][c \mid \det(\{0,h_1,\ldots,h_k\})] 1.
\end{align*}
%
For each $d \in \cD$ with $y/d \ge 1$, the proof of 
Lemma \ref{lem:lip} (see \cite[pp.\ 128--129]{LAN:94}) shows that 
there are $\ll_{k,\sC} (y/d)^{k - 1}$ cubes $C_{d,\vbt}$ that have 
a nonempty intersection with the boundary of $y\sC$.
%
For each such boundary cube $C_{d,\vbt}$, the corresponding 
$d$-boundary points are all in $C_{d,\vbt} \cap \ZZ^k$, which is a 
product of complete residue systems modulo $d$, and, given that 
$c \mid \rad(d)$ (and hence $c \mid d$), the condition 
$c \mid \det(\{0,h_1,\ldots,h_k\})$ is equivalent to 
$c \mid \det(\{0,h'_1,\ldots,h'_k\})$ when 
$h_i \equiv h'_i \bmod d$, $i = 1,\ldots,k$.

If follows that, for $d \in \cD$ with $d \le y$, and for 
$c \mid \rad(d)$, we have 
\[
 \sums[\vbh \in \cH][\text{$d$-boundary}][c \mid \det(\{0,h_1,\ldots,h_k\})] 1
  \ll_{k,\sC}
   \frac{y^{k - 1}}{d^{k - 1}}
    \sums[0 < h_1 < \cdots < h_k \le d][c \mid \det(\{0,h_1,\ldots,h_k\})] 1
     \ll_k 
      y^{k - 1} d \bigg(\frac{A_k^{\omega(c)}}{c}\bigg)
\]
by \eqref{eq:lemdethappf1}. 
%
Whence 
\[
 \sums[d \in \cD][d \le y]
  \sums[\vbh \in \cH][\text{$d$-boundary}]
  |\epsilon_{\bo \cup \bh}(d)|
   \ll_{k,\sC}
    y^{k - 1} 
      \sums[d \in \cD][d \le y] \frac{A_k^{\omega(d)}}{\sqfr(d)}
       \sum_{c \mid \rad(d)} A_k^{\omega(c)}
        \le 
         y^{k - 1} 
          \sums[d \in \cD][d \le y] \frac{A_k^{\omega(d)}}{\sqfr(d)},
\]
since $\sum_{c \mid \rad(d)} A_k^{\omega(c)}$ is at most 
$
  A_k^{\omega(d)} \sum_{c \mid \rad(d)} 1
   = 
    (2A_k)^{\omega(d)}
$.
%
By \eqref{eq:realbnd2}, this last sum is 
$\ll_k y^{1/3 + O(1/\log\log 3y)}$.
%
Combining, we obtain 
\begin{equation}
 \label{eq:last}
  \sums[d \in \cD][d \le y] 
   \sum_{\vbh \in \cH}
    \epsilon_{\bo \cup \bh}(d)
     \ll_{k,\sC}
      y^{k - 1}y^{1/3 + O(1/\log\log 3y)}.
\end{equation}
%
Combining \eqref{eq:lemsssapf1} with \eqref{eq:volH}, 
\eqref{eq:sum3}, and \eqref{eq:last} gives \eqref{eq:sssa}.
\end{proof}

 


%%%%%%%%%%%%%%%%%%%%%%%%%%%%%%%%%%%%%%%%%%%%%%%%%%%%%%%%%%%%%%%%%%
%%%%%%%%%%%%%%%%%%%%%%%%%%%% APPENDIX %%%%%%%%%%%%%%%%%%%%%%%%%%%%
%%%%%%%%%%%%%%%%%%%%%%%%%%%%%%%%%%%%%%%%%%%%%%%%%%%%%%%%%%%%%%%%%%

\begin{nix}
\appendix 

\section{Some elementary verifications}
 \label{sec:A1}
\end{nix}
 
%****************************************************************%
%************************* START DETAIL *************************%
%****************************************************************%
%
\begin{nixnix}
%
\begin{proof}[Deduction of Theorem \ref{thm:main} (b) in detail]
(b) 
% 
To ease notation, we let $\vbi = (i_1,\ldots,i_r)$, 
$\vba = (a_1,\ldots,a_r)$, $\vbh = (h_1,\ldots,h_k)$,  
$\bh = \{h_1,\ldots,h_k\}$, and 
\[
 \speccount(\{0\} \cup \bh; x)
  \defeq 
   \sum_{n \le x} 
    \ind{\SS}(n)\ind{\SS}(n + h_1)\cdots \ind{\SS}(n + h_k).
\]
%
Given $\vbi,\vba \in \NN^r$, let 
\[
 N_{\vbi,\vba}(x)
  \defeq 
   \sums[0 < h_1 < \cdots < h_{i_1 + \cdots + i_r}]
        [h_{i_1 + \cdots + i_j} = a_j, \, j = 1,\ldots,r]
     \hspace{5pt}
      \sum_{n \le x}
       \ind{\SS}(n)\ind{\SS}(n + h_1)\cdots \ind{\SS}(n + h_{i_1 + \cdots + i_r}).
\]
%
Let $\ell \ge 0$ be an integer, arbitrarily large but fixed.
%
We claim that 
\begin{equation}
 \label{eq:inc-exc}
 \sum_{k = r}^{r + 2\ell + 1}
  (-1)^{k - r}
   \sum_{i_1 + \cdots + i_r = k}
    N_{\vbi,\vba}(x)
     \le 
      \sums[\sts_n \le x]
           [\sts_{n + j} - \sts_n = a_j]
           [j = 1,\ldots,r] 1
       \le
        \sum_{k = r}^{r + 2\ell}
         (-1)^{k - r}
          \sum_{i_1 + \cdots + i_r = k}
           N_{\vbi,\vba}(x) 
\end{equation}
for any $\vba \in \NN^r$ with $a_1 < \cdots < a_r$, the inner 
sums (here and below) are over all $\vbi \in \NN^r$ such that 
$i_1 + \cdots + i_r = k$.
%
Now, 
\begin{equation}
 \label{eq:inc-exc2}
  \sums[\sts_n \le x]
       [\sts_{n + j} - \sts_{n + j - 1} \le \lambda_j y]
       [j = 1,\ldots,r] 1
   =
     \sums[\vba \, \in \, \NN^r]
          [0 < a_j - a_{j - 1} \le \lambda_j y]
          [j = 1,\ldots,r]  
       \hspace{5pt} 
        \sums[\sts_n \le x]
             [\sts_{n + j} - \sts_n = a_j]
             [j = 1,\ldots,r] 1,
\end{equation}
where $a_0 \defeq 0$.
%
Given $\vbi \in \NN^r$ with $i_1 + \cdots + i_r = k$ we 
have, with $\Theta_{\vbi,\vbl}$ as in \eqref{eq:defThet}, 
\begin{equation}
 \label{eq:inc-exc3}
 \sums[\vba \, \in \, \NN^r]
      [0 < a_j - a_{j - 1} \le \lambda_j y]
      [j = 1,\ldots,r]
  N_{\vbi,\vba}(x)
   =
    \sum_{\vbh \, \in \, y\Theta_{\vbi,\vbl} \cap \, \ZZ^k}
%        \sum_{n \le x}
%         \ind{\SS}(n)\ind{\SS}(n + h_1)\cdots \ind{\SS}(n + h_k).
        \speccount(\{0\} \cup \bh; x).
\end{equation}
%
Combining \eqref{eq:inc-exc}, \eqref{eq:inc-exc2} and 
\eqref{eq:inc-exc3}, then changing order of summation, we obtain 
\begin{align}
 \begin{split}
  \label{eq:inc-exc4}
 & 
   \sum_{k = r}^{r + 2\ell + 1}
    (-1)^{k - r}
     \sum_{i_1 + \cdots + i_r = k}
      \sum_{\vbh \, \in \, y\Theta_{\vbi,\vbl} \cap \, \ZZ^k}
%        \sum_{n \le x}
%         \ind{\SS}(n)\ind{\SS}(n + h_1)\cdots \ind{\SS}(n + h_k)
        \speccount(\{0\} \cup \bh; x)
 \\
 & \hspace{30pt} 
  \le 
%     \#\{\sts_n \le x : \sts_{n + j} - \sts_{n + j - 1} \le \lambda_j y, j = 1,\ldots,r\}
     \sums[\sts_n \le x]
          [\sts_{n + j} - \sts_{n + j - 1} \le \lambda_j y]
          [j = 1,\ldots,r] 1 
%  \\
%  & \hspace{5pt}
   \le 
    \sum_{k = r}^{r + 2\ell}
     (-1)^{k - r}
      \sum_{i_1 + \cdots + i_r = k}
       \sum_{\vbh \, \in \, y\Theta_{\vbi,\vbl} \cap \, \ZZ^k}
%        \sum_{n \le x}
%         \ind{\SS}(n)\ind{\SS}(n + h_1)\cdots \ind{\SS}(n + h_k).
        \speccount(\{0\} \cup \bh; x).
  \end{split}            
\end{align}

The substitution \eqref{eq:defEterm}, with $\{0\} \cup \bh$ and 
$k + 1$ in place of $\bh$ and $k$, yields 
\begin{align}
 \begin{split}
  \label{eq:inc-exc5}
 & 
   \sum_{k = r}^{r + 2\ell + 1}
    (-1)^{k - r}
     \bigg(\frac{\speccount(x)}{x}\bigg)^k
     \sum_{i_1 + \cdots + i_r = k}
      \sum_{\vbh \, \in \, y\Theta_{\vbi,\vbl} \cap \, \ZZ^k}
       \bigg(\mathfrak{S}_{\{0\} \cup \bh} + \cE_{\{0\} \cup \bh}(x)\bigg)
 \\
 & \hspace{15pt} 
  \le 
   \frac{1}{\speccount(x)}
%     \#\{\sts_n \le x : \sts_{n + j} - \sts_{n + j - 1} \le \lambda_j y, j = 1,\ldots,r\}
     \sums[\sts_n \le x]
          [\sts_{n + j} - \sts_{n + j - 1} \le \lambda_j y]
          [j = 1,\ldots,r] 1    
 \\
 & \hspace{30pt}
   \le 
    \sum_{k = r}^{r + 2\ell}
     (-1)^{k - r}
      \bigg(\frac{\speccount(x)}{x}\bigg)^k
       \sum_{i_1 + \cdots + i_r = k}
        \sum_{\vbh \, \in \, y\Theta_{\vbi,\vbl} \cap \, \ZZ^k}
         \bigg(\mathfrak{S}_{\{0\} \cup \bh} + \cE_{\{0\} \cup \bh}(x)\bigg).
  \end{split}            
\end{align}
% 
By applying Hypothesis ($k,\Theta_{\vbi,\vbl},\{0\}$) for all 
$k$ and $\vbi$ satisfying $r \le k \le r + 2\ell + 1$ and  
$i_1 + \cdots + i_r = k$, Proposition \ref{prop:sssa}, and our 
assumption that $y \sim x/\speccount(x)$ as $x \to \infty$, it is 
straightforward to deduce from \eqref{eq:inc-exc5} that 
\begin{align*}
%  \begin{split}
%   \label{eq:inc-exc6}
 & 
   \Big(1 + O_{r,\ell,\vbl}\big(\varepsilon(x)\big)\Big)
    \sum_{k = r}^{r + 2\ell + 1}
     (-1)^{k - r}
      \sum_{i_1 + \cdots + i_r = k}
       \vol(\Theta_{\vbi,\vbl})
 \\
 &  \hspace{15pt} 
      \le 
       \frac{1}{\speccount(x)}
%           \#\{\sts_n \le x : \sts_{n + j} - \sts_{n + j - 1} \le \lambda_j y, j = 1,\ldots,r\}
          \sums[\sts_n \le x]
               [\sts_{n + j} - \sts_{n + j - 1} \le \lambda_j y]
               [j = 1,\ldots,r] 1
%  \\
%  &  \hspace{60pt}
         \le 
          \Big(1 + O_{r,\ell,\vbl}\big(\varepsilon(x)\big)\Big) 
           \sum_{k = r}^{r + 2\ell}
           (-1)^{k - r}
            \sum_{i_1 + \cdots + i_r = k}
             \vol(\Theta_{\vbi,\vbl}), 
%   \end{split}             
\end{align*}
where $\varepsilon(x)$ is some function, not necessarily the same 
as in \eqref{eq:hyp}, such that $\varepsilon(x) \to 0$ as 
$x \to \infty$. 
%
Consequently, 
\begin{equation}
  \label{eq:inc-exc7}
   \sum_{k = r}^{r + 2\ell + 1}
     (-1)^{k - r}
      \sum_{i_1 + \cdots + i_r = k}
       \vol(\Theta_{\vbi,\vbl})
        \le 
         \liminf_{x \to \infty}
          \frac{1}{\speccount(x)}
%           \#\{\sts_n \le x : \sts_{n + j} - \sts_{n + j - 1} \le \lambda_j y, j = 1,\ldots,r\}
           \sums[\sts_n \le x]
                [\sts_{n + j} - \sts_{n + j - 1} \le \lambda_j y]
                [j = 1,\ldots,r] 1
\end{equation}
and 
\begin{equation}
 \label{eq:inc-exc8}
          \limsup_{x \to \infty}
           \frac{1}{\speccount(x)}
%            \#\{\sts_n \le x : \sts_{n + j} - \sts_{n + j - 1} \le \lambda_j y, j = 1,\ldots,r\}
            \sums[\sts_n \le x]
                 [\sts_{n + j} - \sts_{n + j - 1} \le \lambda_j y]
                 [j = 1,\ldots,r] 1
           \le 
            \sum_{k = r}^{r + 2\ell}
            (-1)^{k - r}
             \sum_{i_1 + \cdots + i_r = k}
              \vol(\Theta_{\vbi,\vbl}).            
\end{equation}
Since 
$
 \vol(\Theta_{\vbi,\vbl}) 
  = \lambda_1^{i_1}\cdots \lambda_r^{i_r}/(i_1!\cdots i_r!)
$, 
the sums on the left and right of \eqref{eq:inc-exc7} and 
\eqref{eq:inc-exc8} are truncations of the Taylor 
series for $(1 - \e^{-\lambda_1})\cdots (1 - \e^{-\lambda_r})$.
%
We have chosen $\ell$ arbitrarily large, so we may conclude that 
\eqref{eq:thm:mainc} holds, provided 
Hypothesis ($k,\Theta_{\vbi,\vbl},\{0\}$) does whenever 
$k \ge r$ and $i_1 + \cdots + i_r = k$.

It remains only to prove our claim \eqref{eq:inc-exc}.
%
Let $\vba \in \NN^r$ with $a_1 < \cdots < a_r$ be given.
%
First of all note that, for any $\vbi \in \NN^r$, 
\begin{equation}
 \label{Aeq:inc-excb1}
 N_{\vbi,\vba}(x)
  =
   \sum_{n \le x} \ind{\SS}(n)
    \sums[0 < h_1 < \cdots < h_{i_1 + \cdots + i_r}]
         [h_{i_1 + \cdots + i_j} = a_j, \, j = 1,\ldots r]
          \ind{\SS}(n + h_1)\cdots \ind{\SS}(n + h_{i_1 + \cdots + i_r}).
\end{equation}
%
For $n \in \ZZ$, let $M_j(n,a_j)$ be the number of elements of 
$\SS$ in-between $n + a_{j - 1}$ and $n + a_j$, i.e.\ 
\[
 M_j(n)
  =
   M_j(n,a_j)
    \defeq 
      \card \SS \cap (n + a_{j - 1},n + a_j),    
\] 
for $j = 1,\ldots,r$, where $a_0 \defeq 0$.
%
From \eqref{Aeq:inc-excb1} it is not difficult to see that, for 
any $\vbi \in \NN^r$, 
\begin{equation}
 \label{Aeq:inc-excb2}
 N_{\vbi,\vba}(x)
  =
   \sum_{n \le x}
    \bigg\{
     \ind{\SS}(n)
      \ind{\SS}(n + a_1)\cdots \ind{\SS}(n + a_r)
       \binom{M_1(n)}{i_1 - 1}
        \cdots 
         \binom{M_r(n)}{i_r - 1} 
    \bigg\},
\end{equation}
where, as usual, $\binom{0}{j} = 1$ for $j \ge 0$ and 
$\binom{m}{j} = 0$ for $j > m$.
%
Note that, given $n,n + a_1,\ldots,n + a_r \in \SS$, these are 
{\em consecutive} elements of $\SS$, i.e.\ 
$n + a_j = \sts_{t + j}$, $j = 0,\ldots,r$, for some $t$, if and 
only if $M_1(n) = \cdots = M_r(n) = 0$.

Next, for any integer $k \ge r$ and any nonnegative integers 
$M_1,\ldots,M_r$, we have 
\begin{equation}
 \label{Aeq:inc-excb3}
 \sum_{i_1 + \cdots + i_r = k}
  \binom{M_1}{i_1 - 1}
   \cdots 
    \binom{M_r}{i_r - 1}
     =
      \binom{M_1 + \cdots + M_r}{k - r}.
\end{equation}
%
To see this, note that each summand on the left-hand side is the 
number of ways of choosing $k - r$ objects from a set $X$ of size 
$M_1 + \cdots + M_r$, in such a way that $i_j - 1$ objects are 
chosen from a subset $X_j \subseteq X$ of size $M_j$, and where 
$X = X_1 \cup \cdots \cup X_r$ is a partition.
%
Summing over all partitions of $k$ into $r$ positive integers, we 
end up with the total number of ways to choose $k - r$ objects 
from $X$, viz.\ the right-hand side.
%
Also, 
\begin{align}
 \label{Aeq:inc-excb4}
 \sum_{k - r \ge 0} 
  (-1)^{k - r}
   \binom{M_1 + \cdots + M_r}{k - r}
    =
     \begin{cases}
      1 & \text{$M_1 + \cdots + M_r = 0$,} \\
      0 & \text{otherwise,}
     \end{cases}
\end{align}
the left-hand side being the binomial expansion of 
$(1 - 1)^{M_1 + \cdots + M_r}$ in the second case.
%
Furthermore, for any nonnegative integers $M_1,\ldots,M_r$ we have  
\begin{align}
 \begin{split}
  \label{Aeq:inc-excb5}
 & 
 \sum_{k - r = 0}^{2\ell + 1}
  (-1)^{k - r}
   \binom{M_1 + \cdots + M_r}{k - r}
 \\
 & \hspace{30pt}
    \le 
     \sum_{k - r \ge 0} 
      (-1)^{k - r}
       \binom{M_1 + \cdots + M_r}{k - r}
        \le 
         \sum_{k - r = 0}^{2\ell}
          (-1)^{k - r}
           \binom{M_1 + \cdots + M_r}{k - r},
  \end{split}
\end{align}
as can be verified by using the recurrence relation
$
 \binom{m}{i} = \binom{m - 1}{i} + \binom{m - 1}{i - 1}
$.

Combining \eqref{Aeq:inc-excb2}, \eqref{Aeq:inc-excb3} and 
\eqref{Aeq:inc-excb4} we find, after changing order of summation, 
that 
\begin{equation}
 \label{Aeq:inc-excb6}
  \sums[n \le x][\text{consecutive}] 
   \ind{\SS}(n)\ind{\SS}(n + a_1)\cdots \ind{\SS}(n + a_r)
    =
  \sum_{k - r \ge 0} (-1)^{k - r}
   \sum_{i_1 + \cdots + i_r = k} 
    N_{\vbi,\vba}(x),
\end{equation}
where in the summand on the left-hand side, ``consecutive'' 
indicates summation restricted to those $n$ for which 
$n + a_j = \sts_{t + j}$, $j = 0,\ldots,r$, for some $t$.
%
Combining \eqref{Aeq:inc-excb2}, \eqref{Aeq:inc-excb3} and 
\eqref{Aeq:inc-excb5} we similarly find that 
\begin{align}
 \begin{split}
  \label{Aeq:inc-excb7}
 & 
  \sum_{k - r = 0}^{2\ell + 1} (-1)^{k - r}
   \sum_{i_1 + \cdots + i_r = k} 
    N_{\vbi,\vba}(x)
 \\
 & \hspace{15pt} 
  \le 
   \sums[n \le x][\text{consecutive}] 
    \ind{\SS}(n)\ind{\SS}(n + a_1)\cdots \ind{\SS}(n + a_r)
  \le 
   \sum_{k - r = 0}^{2\ell} (-1)^{k - r}
    \sum_{i_1 + \cdots + i_r = k} 
     N_{\vbi,\vba}(x).
 \end{split}
\end{align}
%
These are the claimed inequalities in \eqref{eq:inc-exc}, since 
\[
 \sums[n \le x][\text{consecutive}] 
  \ind{\SS}(n)\ind{\SS}(n + a_1)\cdots \ind{\SS}(n + a_r)
   =
    \sums[\sts_t \le x][\sts_{t + j} - \sts_t = a_j][j = 1,\ldots,r] 1.
\]

\end{proof}
%
\end{nixnix}
%
%****************************************************************%
%************************** END DETAIL **************************%
%****************************************************************%
%

\begin{nix}
%
\begin{proof}[Proof of Proposition \ref{prop:S2S3S1}] 
%
(a)
%
Trivially, $0 \equiv \sots \bmod 2^{\nu}$ for all $\nu \ge 1$.
%
Let $\beta \ge 0$ and $m \equiv 1 \bmod 4$.
%
We have $2^{\beta} = \sots$, and we claim that 
$m \equiv \sots \bmod 2^{\nu}$ for all $\nu \ge 1$.
%
By Brahmagupta's identity, it follows that 
$2^{\beta}m \equiv \sots \bmod 2^{\nu}$ for all $\nu \ge 1$.
%
We trivially have $m \equiv \sots \bmod 2^{\nu}$ for $\nu = 1,2$.
%
For $\nu \ge 2$, if $m \equiv a^2 + b^2 \bmod 2^{\nu}$ then 
$a + b \equiv 1 \bmod 2$ and either 
$m \equiv a^2 + b^2 \bmod 2^{\nu + 1}$ or 
\[
 m \equiv a^2 + b^2 + 2^{\nu}
    \equiv (a + 2^{\nu - 1})^2 + (b + 2^{\nu - 1})^2
     \bmod 2^{\nu + 1}.
\]

Next, suppose $n \ne 0$.
%
Then $n = 2^{\beta}m$ for some $\beta \ge 0$ and 
$m \equiv \pm 1 \bmod 4$.
%
We claim that if $2^{\beta}m \equiv \sots \bmod 2^{\beta + 2}$ 
then $m \equiv 1 \bmod 4$.
%
This holds trivially for $\beta = 0$.
%
For $\beta \ge 0$, if 
$2^{\beta + 1}m \equiv a^2 + b^2 \bmod 2^{\beta + 3}$ then 
$a \equiv b \bmod 2$ and, letting $c = (a + b)/2$ and 
$d = (a - b)/2$, we see that  
$2^{\beta}m \equiv c^2 + d^2 \bmod 2^{\beta + 2}$.

%
(b)
%
Trivially, $0 \equiv \sots \bmod p^{\nu}$ for all $\nu \ge 1$.
%
Let $\beta \ge 0$ and $m \not\equiv 0 \bmod p$.
%
We have $p^{2\beta} = \sots$, and we claim that 
$m \equiv \sots \bmod p^{\nu}$ for all $\nu \ge 1$.
%
In view of Brahmagupta's identity, it follows that 
$p^{2\beta}m \equiv \sots \bmod p^{\nu}$ for all $\nu \ge 1$.
%
For $\nu = 1$ note that since, by Euclid's lemma, the sets 
\[
 \{m - a^2 \bmod p : a = 0,\ldots,p - 1\}
  \quad 
   \text{and} 
    \quad 
     \{b^2 \bmod p : b = 0,\ldots,p - 1\}
\]
both contain $(p + 1)/2$ congruence classes, their intersection 
must be nonempty.
%
Hence $m \equiv a^2 + b^2 \bmod p$ for some $a$ and $b$.
%
For $\nu \ge 1$, if $m \equiv a^2 + b^2 \bmod p^{\nu}$ then 
$m \equiv a^2 + b^2 + p^{\nu}r \bmod p^{\nu + 1}$ for some integer 
$r$ and, without loss of generality, $a \not\equiv 0 \bmod p$ 
(because  $m \not\equiv 0 \bmod p$).
%
In that case we have $2aa' \equiv 1 \bmod p^{\nu}$ for some 
integer $a'$, and so
$
  m \equiv a^2 + b^2 + p^{\nu}r
     \equiv (a + p^{\nu})^2a'r + b^2 
      \bmod p^{\nu + 1}
$.

Next, suppose $n \ne 0$.
%
Then $n = p^{\alpha}m$ for some $\alpha \ge 0$ and 
$m \not\equiv 0 \bmod p$.
%
We claim that if $p^{\alpha}m \equiv \sots \bmod p^{\alpha + 1}$ 
then $\alpha$ is even.
%
Suppose for a contradiction that 
$p^{\alpha}m \equiv a^2 + b^2 \bmod p^{\alpha + 1}$ but 
$\alpha$ is odd.
%
Then $a^2 \equiv -b^2 \bmod p$ and so, since $(p - 1)/2$ is even 
(as $p \equiv 3 \bmod 4$), $a^{p - 1} \equiv -b^{p - 1} \bmod p$.
%
In view of Fermat's little theorem we must have 
$a \equiv b \equiv 0 \bmod p$.
%
Letting $c = a/p$ and $d = b/p$, we see that 
$p^{\alpha}m \equiv p^2(c^2 + d^2) \bmod p^{\alpha + 1}$.
%
This gives a contradiction for $\alpha = 1$, and for 
$\alpha \ge 3$ implies that 
$p^{\alpha - 2}m \equiv c^2 + d^2 \bmod p^{\alpha - 1}$.

(c) 
%
We have $p^{\beta} = \sots$ for all $\beta \ge 0$ by Fermat's 
theorem on sums of two squares.
%
If $m \not\equiv 0 \bmod p$ then $m \equiv \sots \bmod p$ by 
the argument in the first paragraph of (b).
\end{proof}
%
\end{nix}

\begin{nix}
%
\begin{proposition}
 \label{prop:tripadmiss}
Let $h_1,h_2,h_3 \in \ZZ$.
%
The set $\bh = \{h_1,h_2,h_3\}$ is $\SS$-admissible.
\end{proposition}

\begin{proof}
%
If $p > 3$ then $\{-h_1,-h_2,-h_3\}$ is not a complete set of 
residues modulo $p$, and by Proposition \ref{prop:S2S3S1}, we have 
$n + \bh \subseteq S_p$ whenever $n + h_i \not\equiv 0 \bmod p$ 
for $i = 1,2,3$.
%
For $p = 3$, note that the least residues $b$ modulo $3^3$ for 
which $3 \emid b$ are $3,6,12,15,21$ and $24$, and consider the 
set 
\[
 \{-h_i,3 - h_i,6 - h_i,12 - h_i,15 - h_i,21 - h_i,24 - h_i : i = 1,2,3\}.
\]
%
This consists of representatives of at most 21 distinct 
congruence classes modulo $3^3$.
%
Thus, there exist integers $n$ such that, for $i = 1,2,3$, we have 
\[
 \text{$n + h_i \not\equiv 0,3,6,12,15,21$ or $24 \bmod 3^3$},
\]
meaning that $\nu_3(n + h_i) = 0$ or $\nu_3(n + h_i) = 2$.
%
By Proposition \ref{prop:S2S3S1}, for such $n$ we have 
$n + \bh \subseteq S_p$. 

Finally, consider $p = 2$.
%
By Proposition \ref{prop:S2S3S1}, $n \in S_2$ if and only if 
$n = 0$ or $n = 2^{\beta}m$ with $\beta \ge 0$ and 
$m \equiv 1 \bmod 4$.
%
Equivalently, $n \in S_2$ if and only if either $n = 0$ or there 
is some $\alpha \ge 0$ such that 
$n \equiv 2^{\beta}m \bmod 2^{\alpha + 2}$, where  
$0 \le \beta \le \alpha$ and $m \equiv 1 \bmod 4$.
%
Note that for $n \ne 0$, we have $n \in S_2$ if and only if 
$-n \not\in S_2$.

If there is some $h_i \in \bh$ such that 
$-h_i + \bh \subseteq S_2$, then there is nothing more to prove, 
so assume this is not the case.
%
This means that for each $h_i \in \bh$ there is some $h_j \in \bh$ 
such that $h_j - h_i = 2^{\beta_{ji}}m_{ji}$ with 
$\beta_{ji} \ge 0$ and $m_{ji} \equiv 3 \bmod 4$.
%
Without loss of generality, suppose $h_3 - h_1 \not\in S_2$.
%
Then $h_1 - h_3 \in S_2$, so it must be that 
$h_2 - h_3 \not\in S_2$.
%
Then $h_3 - h_2 \in S_2$, so it must be that 
$h_1 - h_2 \not\in S_2$.
%
Then $h_2 - h_1 \in S_2$.
%
We have: 
$h_2 - h_1 = 2^{\beta_{21}}m_{21}$,
$h_3 - h_2 = 2^{\beta_{32}}m_{32}$, and 
\[
 h_3 - h_1 = 2^{\beta_{32}}m_{32} + 2^{\beta_{21}}m_{21} 
           = 2^{\beta_{31}}m_{31},
\]
where $m_{21} \equiv m_{32} \equiv 1 \bmod 4$ and 
$m_{31} \equiv 3 \bmod 4$.
%
Suppose further, without loss of generality, that 
$\beta_{21} \le \beta_{32}$.

Consider the case $\beta_{21} = \beta_{32} = \beta$ (say).
%
We have   
$
  2^{\beta}(m_{32} + m_{21}) = 2^{\beta_{31}}m_{31} 
$,
and as $m_{32} + m_{21} \equiv 2 \bmod 4$, we must have 
$\beta_{31} = \beta + 1$. 
%
Thus, for any $n$ satisfying 
$n + h_1 \equiv 2^{\beta + 2} \bmod 2^{\beta + 4}$, we have  
\[
 n + h_2 \equiv 2^{\beta + 2} + 2^{\beta}m_{21}
          \equiv 2^{\beta}(4 + m_{21})
           \bmod 2^{\beta + 4},
\]
and 
\[
 n + h_3 \equiv 2^{\beta + 2} + 2^{\beta + 1}m_{31}
          \equiv 2^{\beta + 1}(2 + m_{31})
           \bmod 2^{\beta + 4}.
\]
%
It follows that $n + \bh \subseteq S_2$.

Now consider the case $\beta_{21} < \beta_{32}$.
%
We have 
$
 2^{\beta_{21}}(2^{\beta_{32} - \beta_{21}}m_{32} + m_{21}) 
  = 2^{\beta_{31}}m_{31} 
$.
%
Since 
$
 m_{31} \equiv 2^{\beta_{32} - \beta_{21}}m_{32} + m_{21} 
         \equiv 1 \bmod 2
$, 
we must have $\beta_{21} = \beta_{31} = \beta$ (say).
%
Then 
$
 2^{\beta_{32} - \beta}m_{32} 
  \equiv m_{31} - m_{21} 
   \equiv 3 - 1 
    \equiv 2 \bmod 4
$, 
so in fact $\beta_{32} = \beta + 1$.
%
In summary:  
$
 h_2 - h_1 = 2^{\beta}m_{21}
$, 
$
 h_3 - h_2 = 2^{\beta + 1}m_{32}
$,
and 
$
 h_3 - h_1 = 2^{\beta}(2m_{32} + m_{21})
$, 
where $m_{21} \equiv m_{32} \equiv 1 \bmod 4$.
%
If $\beta \ge 2$ take $n$ satisfying 
$n + h_1 \equiv 1 \bmod 2^{\beta + 3}$;
if $\beta = 1$ take $n$ satisfying 
$n + h_1 \equiv 2m_{21} \bmod 32$; 
if $\beta = 0$ take $n$ satisfying 
$n + h_1 \equiv m_{21} \bmod 16$.
%
In each case we have $n + \bh \subseteq S_2$.
\end{proof}
%
\end{nix}

\begin{nix}
\begin{proposition}
 \label{prop:S2hest}
Let $\bh = \{h_1,\ldots,h_k\}$ be a set of $k \ge 1$ integers, and 
assume that $0 \le h_1 < \cdots < h_k$.
%
Let $\alpha = 2 + \max_{i \ne j} \nu_2(h_i - h_j)$.
%
For $x \ge 1$ we have  
\begin{equation}
 \label{eq:S2hest}
  \frac{1}{x}
   \sums[n \le x][\forall i, n + h_i \in S_2] 1
  =
    \delta_{\bh}(2)
     \bigg\{
      1 + O\bigg(\frac{2^{\alpha}(1 + \card \bh_2\log(x + h_k))}{x}\bigg)
     \bigg\}.
\end{equation}
\end{proposition}

\begin{proof} 
%
Let $x \ge 1$.
%
We partition the sum in \eqref{eq:S2hest} as follows:
\begin{equation}
 \label{eq:S2hestpf1}
 \sums[n \le x][\forall i, n + h_i \in S_2] 1
  =
  \sums[n \le x][\forall i, n + h_i \in S_2][\forall i, \nu_2(n + h_i) < \alpha] 1
  \hspace{15pt} + 
    \sums[n \le x][\forall i, n + h_i \in S_2][\exists j, \nu_2(n + h_j) = \alpha] 1
    \hspace{15pt} +
      \sums[n \le x][\forall i, n + h_i \in S_2][\exists j, \nu_2(n + h_j) > \alpha] 1.
\end{equation}
%
(For a given $n$, if $\nu_2(n + h_i) = \beta$ and 
$\nu_2(n + h_j) > \beta$, then $\nu_2(h_i - h_j) = \beta$, 
implying 
$\beta \le \max_{i \ne j} \nu_2(h_i - h_j) \le \alpha - 2$, so 
there is certainly no overlap between the last two sums.)
%
By Proposition \ref{prop:S2S3S1}, $n + h_i \in S_2$ 
(for $n + h_i \ne 0$) if and only if there exists $\beta \ge 0$ 
and $m \equiv 1 \bmod 4$ such that $n + h_i = 2^{\beta}m$.
%
Using this, we verify that on the right-hand side of 
\eqref{eq:S2hestpf1}, the condition on $n$ in the first sum holds 
if and only if $n \equiv a \bmod 2^{\alpha + 1}$ for some $a$ in 
the (possibly empty) set 
\[
 \T_{\bh}(2^{\alpha + 1})
  =
   \{0 \le a < 2^{\alpha + 1} : \forall i, a + h_i \in S_2
      \,\, \hbox{and} \, \, \nu_2(a + h_i) < \alpha\}.
\]
%
Now,
\[
 \sum_{a \in \T_{\bh}(2^{\alpha + 1})}
  \sums[n \le x][n \equiv a \bmod 2^{\alpha + 1}] 1
   =
    x\frac{\card \T_{\bh}(2^{\alpha + 1})}{2^{\alpha + 1}} 
    + 
     O(\card \T_{\bh}(2^{\alpha + 1})),
\]
but the left-hand side here is also equal to 
\[
  \sum_{n \le x} 1
   -
    \sums[0 \le b < 2^{\alpha + 1}][b \not\in \T_{\bh}(2^{\alpha + 1})] 
     \sums[n \le x][n \equiv b \bmod 2^{\alpha + 1}] 1
   =
    x\frac{\card \T_{\bh}(2^{\alpha + 1})}{2^{\alpha + 1}} 
    + 
     O(2^{\alpha + 1} - \card \T_{\bh}(2^{\alpha + 1})).
\]
%
Hence 
\begin{equation}
 \label{eq:S2hestpf2}
 \sums[n \le x][\forall i, n + h_i \in S_2][\forall i, \nu_2(n + h_i) < \alpha] 1
  =
   x 
    \frac{\card \T_{\bh}(2^{\alpha + 1})}{2^{\alpha + 1}}
    + 
     O(\min\{\card \T_{\bh}(2^{\alpha + 1}),2^{\alpha + 1} - \T_{\bh}(2^{\alpha + 1})\}).
\end{equation}

In the second sum on the right-hand side of \eqref{eq:S2hestpf1}, 
we claim that the condition on $n$ holds if and only if the set 
$
 \bh_2 = \{h_j \in \bh : \forall i, h_j - h_i \in S_2\} 
$
is nonempty and 
$n + h_j \equiv 2^{\alpha} \bmod 2^{\alpha + 2}$ for the 
(necessarily unique) $h_j$ in $\bh_2$.
%
Thus,  
\begin{equation}
 \label{eq:S2hestpf3}
 \sums[n \le x][\forall i, n + h_i \in S_2][\exists j, \nu_2(n + h_j) = \alpha] 1
  =
   \card\bh_2
    \bigg(\frac{x}{2^{\alpha + 2}} + O(1)\bigg).
\end{equation}
%
To verify the claim, suppose that for some $j$ we have 
$n + h_j = 2^{\alpha}(1 + 2q)$.
%
By Proposition \ref{prop:S2S3S1}, this is in $S_2$ if and only if 
$2 \mid q$.
%
Now let $i \ne j$, so that there exists 
$\beta_{ij} \le \max_{i \ne j} \nu_2(h_i - h_j) \le \alpha - 2$ 
and $m_{ij} \equiv \pm 1 \bmod 4$ such that 
$h_i - h_j = 2^{\beta_i}m_i$.
%
Thus, 
$
 n + h_i 
  = 2^{\beta_{ij}}
   (m_{ij} + 2^{\alpha - \beta_{ij}}(1 + 2q)) 
$ 
is in $S_2$ if and only if $m_{ij} \equiv 1 \bmod 4$, 
equivalently, $h_i - h_j \in S_2$.
%
By definition of $\bh_2$, this holds for each $i \ne j$ if and 
only if $h_j \in \bh_2$.

By the same argument, the condition on $n$ in the third sum on the 
right-hand side of \eqref{eq:S2hestpf1} holds if and only if 
$n + h_j = 2^{\beta}m$ for some $\beta > \alpha$, 
$m \equiv 1 \bmod 4$, and $h_j$ in the (possibly empty) set 
$\bh_2$.
%
Thus,  
\[
 \sums[n \le x][\forall i, n + h_i \in S_2][\exists j, \nu_2(n + h_j) > \alpha] 1 
  =
   \sum_{h_j \in \bh_2}
    \sum_{\beta > \alpha}
     \sums[n \le x][n + h_j = 2^{\beta}m][m \equiv 1 \bmod 4] 1.
\] 
%
Assume $x + h_k \ge 2^{\alpha + 1}$, and let $\gamma$ be the 
integer such that $2^{\gamma} \le x + h_k < 2^{\gamma + 1}$.
%
For any $h_j \in \bh$ we have 
\[
 \sum_{\beta > \alpha}
  \sums[n \le x][n + h_j = 2^{\beta}m][m \equiv 1 \bmod 4] 1
   =
    \sum_{\beta = \alpha + 1}^{\gamma}
     \sums[h_j < 2^{\beta}m \le x + h_j][m \equiv 1 \bmod 4] 1
      =
      \frac{x}{4}
       \sum_{\beta = \alpha + 1}^{\gamma}
        \frac{1}{2^{\beta}} 
         + 
          O(\gamma).
\]
%
Since
\[
 \sum_{\beta = \alpha + 1}^{\gamma}
  \frac{1}{2^{\beta}}
   =
    \frac{1}{2^{\alpha}}
     -
      \frac{1}{2^{\gamma}}
      =
       \frac{1}{2^{\alpha}} + O\bigg(\frac{1}{x}\bigg), 
\]
and since $\gamma \le \log(x + h_k)/\log 2$, we see that 
\[
 \sum_{\beta > \alpha}
  \sums[n \le x][n + h_j = 2^{\beta}m][m \equiv 1 \bmod 4] 1 
   =
    \frac{x}{2^{\alpha + 2}} + O(\log(x + h_k)).
\]
%
This also holds trivially if $x + h_k < 2^{\alpha + 1}$, so in 
any case we have 
\begin{equation}
 \label{eq:S2hestpf4}
 \sums[n \le x][\forall i, n + h_i \in S_2][\exists j, \nu_2(n + h_j) > \alpha] 1 
  =
   \card\bh_2\frac{x}{2^{\alpha + 2}} + O(\card\bh_2\log x).
\end{equation}

Combining \eqref{eq:S2hestpf1} with \eqref{eq:S2hestpf2}, 
\eqref{eq:S2hestpf3} and \eqref{eq:S2hestpf4} gives 
\[
  \sums[n \le x][\forall i, n + h_i \in S_2] 1
  =
    \frac{x}{2^{\alpha + 1}}
     \big(\card \T_{\bh}(2^{\alpha + 1}) + \card \bh_2\big)
     + 
      O\big(
        \card \T_{\bh}(2^{\alpha + 1}) 
       + \card\bh_2 \log(x + h_k) 
       \big).
\]
%
(This estimate also holds with 
$2^{\alpha + 1} - \card \T_{\bh}(2^{\alpha + 1})$ in place of 
$\card \T_{\bh}(2^{\alpha + 1})$ in the $O$-term.)
%
By Proposition \ref{prop:S2h} (b), 
$
  \delta_{\bh}(2)
   =
    (\card \T_{\bh}(2^{\alpha + 1}) + \card \bh_2)/2^{\alpha + 1} 
$, 
giving the main term in \eqref{eq:S2hest}.
%
For the $O$-term, note that 
\[
 \card \T_{\bh}(2^{\alpha + 1}) + \card\bh_2 \log(x + h_k)
  \le 
   2^{\alpha + 1}\delta_{\bh}(2)\big(1 + \card \bh_2\log(x + h_k)\big). 
\]
\end{proof}



\begin{proposition}
 \label{prop:Sp3hest}
Let $\bh = \{h_1,\ldots,h_k\}$ be a set of $k \ge 1$ integers, and 
assume that $0 \le h_1 < \cdots < h_k$.
%
Let $p \equiv 3 \bmod 4$, and let 
$\alpha = 1 + \max_{i \ne j} \nu_p(h_i - h_j)$.
%
For $x \ge 1$ we have  
\begin{equation}
 \label{eq:Sp3hest}
  \frac{1}{x}
   \sums[n \le x][\forall i, n + h_i \in S_p] 1
  =
   \delta_{\bh}(p)
    \bigg\{ 
     1 + O\bigg(\frac{p^{\alpha}}{x} + \frac{p^{\alpha + \alpha \bmod 2} \log(x + h_k)}{x\log p}\bigg)
    \bigg\}.
\end{equation}
%
In the $O$-term, $\log(x + h_k)$ may be replaced by $0$ if 
$\card \bh_p = \emptyset$.
%
Also, if $\alpha = 1$ then, in the $O$-term, $p^{\alpha}$ and 
$p^{\alpha + \alpha \bmod 2}$ may be replaced by $k$ and $k^2$ 
respectively.
\end{proposition}

\begin{proof} 
%
By Proposition \ref{prop:S2S3S1}, $n + h_i \in S_p$ (for 
$n + h_i \ne 0$) if and only if $2 \mid \nu_p(n + h_i)$.
%
Let $x \ge 1$.
%
We partition the sum in \eqref{eq:Sp3hest} as follows:
\begin{equation}
 \label{eq:Sp3hestpf1}
 \sums[n \le x][\forall i, n + h_i \in S_p] 1
 =
  \sums[n \le x][\forall i, 2 \mid \nu_p(n + h_i)][\forall i, \nu_p(n + h_i) < \alpha] 1
   \hspace{15pt}
  + 
   \sums[n \le x][\forall i, 2 \mid \nu_p(n + h_i)][\exists j, \nu_p(n + h_j) \ge \alpha] 1.
\end{equation}
%
%****************************************************************%
%************************* START DETAIL *************************%
%****************************************************************%
%
\begin{nixnix} 
%
On the right-hand side of \eqref{eq:Sp3hestpf1}, the condition on 
$n$ in the first sum holds if and only if 
$n \equiv a \bmod p^{\alpha}$ for some $a$ in the (possibly 
empty) set 
\[
 \V_{\bh}(p^{\alpha})
  =
   \{0 \le a < p^{\alpha} : \forall i, 2 \mid \nu_p(a + h_i) \,\, \hbox{and} \,\, \nu_p(a + h_i) < \alpha\}.
\]
%
Now,
\[
 \sum_{a \in \V_{\bh}(p^{\alpha})}
  \sums[n \le x][n \equiv a \bmod p^{\alpha}] 1
   =
    x\frac{\card \V_{\bh}(p^{\alpha})}{p^{\alpha}} 
    + 
     O(\card \V_{\bh}(p^{\alpha})),
\]
but the left-hand side here is also equal to 
\[
  \sum_{n \le x} 1
   -
    \sums[0 \le b < p^{\alpha}][b \not\in \V_{\bh}(p^{\alpha})] 
     \sums[n \le x][n \equiv b \bmod p^{\alpha}] 1
   =
    x\frac{\card \V_{\bh}(p^{\alpha})}{p^{\alpha}} 
    + 
     O(p^{\alpha} - \card \V_{\bh}(p^{\alpha})).
\]
%
Hence 
\begin{equation}
 \label{eq:Sp3hestpf2}
 \sums[n \le x][\forall i, 2 \mid \nu_p(n + h_i)][\forall i, \nu_p(n + h_i) < \alpha] 1
  =
   x\frac{\card \V_{\bh}(p^{\alpha})}{p^{\alpha}} 
    + 
     O(\min\{\card \V_{\bh}(p^{\alpha}),p^{\alpha} - \card \V_{\bh}(p^{\alpha})\}).
\end{equation}

Consider the second sum on the right-hand side of 
\eqref{eq:Sp3hestpf1}. 
%
For a given $n$, there is at most one $h_j$ in $\bh$ such that 
$\nu_p(n + h_j) \ge \alpha$. 
%
Therefore, the condition on $n$ in the second sum holds if and 
only if $2 \mid \nu_p(n + h_j)$ and $\nu_p(n + h_j) \ge \alpha$ 
for some $h_j$ in the (possibly empty) set 
$
 \bh_p 
  =
   \{h_j \in \bh : \forall i \ne j, 2 \mid \nu_p(h_i - h_j)\} 
$.
%
Thus,  
\[
 \sums[n \le x][\forall i, 2 \mid \nu_p(n + h_i)][\exists j, \nu_p(n + h_j) \ge \alpha] 1 
  =
   \sum_{h_j \in \bh_p}
    \sum_{\beta \ge \frac{\alpha}{2}}
     \sums[n \le x][p^{2\beta} \emid n + h_j] 1.
\]

Assume $x + h_k \ge p^{\alpha}$, and let $\gamma$ be the integer 
such that $p^{\gamma} \le x + h_k < p^{\gamma + 1}$.
%
For any $h_j \in \bh$ we have 
\begin{align*}
  \sum_{\beta \ge \frac{\alpha}{2}}
   \sums[n \le x][p^{2\beta} \emid n + h_j] 1
 & =
     \sum_{\frac{\alpha}{2} \le \beta \le \frac{\gamma}{2}} 
      \Big\{ 
            \sums[n \le x][p^{2\beta} \mid n + h_j] 1
           - 
             \sums[n \le x][p^{2\beta + 1} \nmid n + h_j] 1
      \Big\}
 \\
 & = 
   x\bigg(1 - \frac{1}{p}\bigg)
      \sum_{\frac{\alpha}{2} \le \beta \le \frac{\gamma}{2}} 
       \frac{1}{p^{2\beta}}
       + 
        O(\gamma).
\end{align*}
%
Since  
\begin{align*}
 \sum_{\frac{\alpha}{2} \le \beta \le \frac{\gamma}{2}} 
  \frac{1}{p^{2\beta}}
   & 
   =
    \bigg(1 - \frac{1}{p^2}\bigg)^{-1}
     \bigg(\frac{1}{p^{\alpha + (\alpha \bmod 2)}} - \frac{p^{\gamma \bmod 2}}{p^{\gamma + 2}}\bigg)
  \\
   & 
    =
       \bigg(1 - \frac{1}{p}\bigg)^{-1}
        \bigg(1 + \frac{1}{p}\bigg)^{-1}
         \frac{1}{p^{\alpha + (\alpha \bmod 2)}}
       + O\bigg(\frac{1}{x}\bigg),
\end{align*}
and since $\gamma \le \log(x + h_k)/\log p$, we see that 
%
\[
 \sum_{\beta \ge \frac{\alpha}{2}}
  \sums[n \le x][p^{2\beta} \emid n + h_j] 1
   =
    x\bigg(1 + \frac{1}{p}\bigg)^{-1}
      \frac{1}{p^{\alpha + (\alpha \bmod 2)}}
       +
       O\bigg(\frac{\log(x + h_k)}{\log p}\bigg).  
\]
%
This also holds trivially if $x + h_k < p^{\alpha}$, so in any 
case we have 
\begin{equation}
 \label{eq:Sp3hestpf3}
  \sums[n \le x][\forall i, 2 \mid \nu_p(n + h_i)][\exists j, \nu_p(n + h_j) \ge \alpha] 1
   =
    \card\bh_p \,
     x\bigg(1 + \frac{1}{p}\bigg)^{-1}
       \frac{1}{p^{\alpha + (\alpha \bmod 2)}}
        +
         O\bigg(\card \bh_p \frac{\log(x + h_k)}{\log p}\bigg).
\end{equation}

Combining \eqref{eq:Sp3hestpf1} with \eqref{eq:Sp3hestpf2} and 
\eqref{eq:Sp3hestpf3} gives
\begin{align*}
   \sums[n \le x][\forall i, n + h_i \in S_p] \hspace{-5pt} 1
  = 
  \textstyle 
   \frac{x}{p^{\alpha}}
    \Big(\card \V_{\bh}(p^{\alpha}) + \card \bh_p\big(1 + \frac{1}{p}\big)^{-1}\frac{1}{p^{\alpha \bmod 2}}\Big)
     + 
      O\Big(
        \card \V_{\bh}(p^{\alpha})+ \card\bh_p \frac{\log(x + h_k)}{\log p} 
       \Big).
\end{align*}
%
\end{nixnix}
%
%****************************************************************%
%************************** END DETAIL **************************%
%****************************************************************%
%
We argue along the lines of the proof of 
Proposition \ref{prop:S2hest}, ending up with  
\begin{align*}
   \sums[n \le x][\forall i, n + h_i \in S_p] \hspace{-5pt} 1
  = 
  \textstyle 
   \frac{x}{p^{\alpha}}
    \Big(\card \V_{\bh}(p^{\alpha}) + \card \bh_p\big(1 + \frac{1}{p}\big)^{-1}\frac{1}{p^{\alpha \bmod 2}}\Big)
     + 
      O\Big(
        \card \V_{\bh}(p^{\alpha})+ \card\bh_p \frac{\log(x + h_k)}{\log p} 
       \Big),
\end{align*}
which also holds with $p^{\alpha} - \card \V_{\bh}(p^{\alpha})$ in 
place of $\card \V_{\bh}(p^{\alpha})$ in the $O$-term. 
%
Proposition \ref{prop:Sp3h} (b) then gives the main term in 
\eqref{eq:Sp3hest}.
%
For the $O$-term, note that 
$
 p^{\alpha}\delta_{\bh}(p)
  \ge 
   \card \V_{\bh}(p^{\alpha})  
$
and 
$
 p^{\alpha + \alpha \bmod 2}\delta_{\bh}(p) 
  \ge 
   \card \bh_p
$.
%
If $\bh_p = \emptyset$ then the term with $\log(x + h_k)$ may be 
omitted.

To deal with the special case where $\alpha = 1$, i.e.\ 
$p \nmid \det(\bh)$, we note that $p - \card \V_{\bh}(p) = 1$ and 
$\bh_p = \bh$.
%
Since we may take $p^{\alpha} - \card \V_{\bh}(p^{\alpha})$ in place 
of $\card \V_{\bh}(p^{\alpha})$ in the $O$-term above, 
\[
 \sums[n \le x][\forall i, n + h_i \in S_p] 1
  =
   x\delta_{\bh}(p)
     + 
      O\bigg(
        1 + k\frac{\log(x + h_k)}{\log p} 
       \bigg),
\]
where $\delta_{\bh}(p) = 1 - k/(p + 1) \ge 1/(k + 1)$ (see 
Proposition \ref{prop:Sp3h} (c)).
\end{proof}

\end{nix}

\begin{nix}
%   
\begin{proposition}
 \label{Aprop:sssk=2}
%
Let $h$ be a nonzero integer.
%
We have 
\begin{equation}
 \label{Aeq:delthk=2}
  \delta_{\{0,h\}}(2) = \frac{1}{4}
   \quad 
    (\nu_2(h) = 0), 
     \quad 
      \delta_{\{0,h\}}(2)  
       = 
        \frac{2^{\nu_2(h) + 1} - 3}{2^{\nu_2(h) + 2}}
         \quad 
          (\nu_2(h) \ge 1), 
\end{equation}
and for $p \equiv 3 \bmod 4$ we have 
\begin{equation}
 \label{Aeq:delthp3k=2}
 \delta_{\{0,h\}}(p) 
  =
   \bigg(1 + \frac{1}{p}\bigg)^{-1}
    \bigg(1 - \frac{1}{p^{\nu_p(h) + 1}}\bigg).
\end{equation}
%
Consequently, if 
$h = 2^{\alpha}p_1^{\alpha_1}\cdots p_r^{\alpha_r}q$, 
where $\alpha \ge 0$, $p_i \equiv 3 \bmod 4$ and $\alpha_i \ge 1$ 
for $i = 1,\ldots,r$, and $q$ is composed only of primes congruent 
to $1$ modulo $4$, then  
\begin{equation}
 \label{Aeq:sssk=2}
 \mathfrak{S}_{\{0,h\}}
  =
   \frac{2\delta_{\{0,h\}}(2)}{C^2}
    \prod_{i = 1}^r
     \bigg(1 - \frac{1}{p_i}\bigg)^{-1}
      \bigg(1 - \frac{1}{p_i^{\alpha_i + 1}}\bigg),
\end{equation}
where $C$ is the Landau--Ramanujan constant 
\textup{(}see \eqref{eq:defLanRamconst}\textup{)}.
\end{proposition}

\begin{proof}
%
Let $\bh = \{0,h\}$.
%
Let $\nu_2(h) = \alpha$, so that $h = 2^{\alpha}h'$, say.
%
If $h' \equiv 1 \bmod 4$ then $\bh_2 = \{0\}$, and if 
$h' \equiv 3 \bmod 4$ then $\bh_2 = \{h\}$, so in any case 
$\card \bh_2 = 1$.
%
First consider the case where $\alpha = 0$, i.e.\ 
$\bh = \{0,h'\}$.
%
We have 
\[
 \T_{\bh}(8)
  =
   \{0 \le a < 8 : a,a + h' \equiv 1,2 \, \hbox{or} \, 5 \bmod 8\}.
\]
%
Since $h' \equiv 1,3,5$ or $7 \bmod 8$, we verify that  
$\T_{\bh}(8) = \{1\},\{2\},\{5\}$ or $\{2\}$ respectively.
%
Thus, $\card \T_{\bh}(8) = 1$ and 
\[
 \delta_{\bh}(2) 
  =
   \frac{\card \T_{\bh}(8) + \card \bh_2}{8}
    =
     \frac{1 + 1}{8}
      = 
       \frac{1}{4}.
\]
%
\begin{nixnix}
%
Consider the case where $\alpha = 1$, i.e.\ 
$\bh = \{0,2h'\}$.
%
We have 
\[
 \T_{\bh}(16) 
  =
   \{0 \le a < 16 : a, a + 2h' \equiv 1,2,4,5,9,10 \, \hbox{or} \, 13 \bmod 16\}.
\]
%
As $2h' \equiv 2,6,10$ or $14 \bmod 16$, we verify that 
$\T_{\bh}(16) = \{2\}$, $\T_{\bh}(16) = \{4\}$, 
$\T_{\bh}(16) = \{10\}$ or $\T_{\bh}(16) = \{4\}$ respectively.
%
Thus, $\card \T_{\bh}(16) = 1$ and 
\[
 \delta_{\bh}(2) 
  =
   \frac{\card \T_{\bh}(16) + \card \bh_2}{16}
    =
     \frac{1 + 1}{16}
      = 
       \frac{1}{8}.
\]
%
\end{nixnix}

Next, consider the case where $\alpha \ge 1$.
%
Recall from Proposition \ref{prop:S2h} that in general, 
\[
 \T_{\bh}(2^{\alpha + 3}) 
  =
   \{a,a + 2^{\alpha + 2} : a \in \T_{\bh}(2^{\alpha + 2})\}
    \cup 
     \U_{\bh}(2^{\alpha + 3}),
\]
where $\card \U_{\bh}(2^{\alpha + 3}) = \card \bh_2$.
%
By definition,  
\[
 \T_{\bh}(2^{\alpha + 2}) 
  =
   \{0 \le a < 2^{\alpha + 2} : a,a + h \in S_2, \nu_2(a),\nu_2(a + h) \le \alpha\},
\]
%
We claim that if $a \in S_2$ and $\nu_2(a) = \alpha$, then either 
$a + h \not\in S_2$ or $\nu_2(a + h) > \alpha$.
%
For if $a = 2^{\alpha}m$ with $m \equiv 1 \bmod 4$, then   
$a + h = 2^{\alpha}(m + h')$ and $m + h' \equiv 2$ or 
$0 \bmod 4$ ($h' \equiv 1$ or $3 \bmod 4$).
%
Likewise, if $a + h \in S_2$ and $\nu_2(a + h) \le \alpha$, then 
either $a \not\in S_2$ or $\nu_2(a) > \alpha$.
%
Similarly, if $a \in S_2$ and $\nu_2(a) = \alpha - 1$ then 
$a + h \not\in S_2$, and likewise with $a$ and $a + h$ 
interchanged.
%
If $\nu_2(a) \le \alpha - 2$ or $\nu_2(a + h) \le \alpha - 2$, 
then $a,a + h \in S_2$ if and only if $a \in S_2$.

In view of all of this, 
$
 \T_{\bh}(2^{\alpha + 2})
  =
   \{0 \le a < 2^{\alpha + 2} : a \in S_2 \,\, \hbox{and} \,\, \nu_2(a) \le \alpha - 2\} 
$, 
and so 
\[
 \card \T_{\bh}(2^{\alpha + 2})
  =
   \sum_{0 \le \beta \le \alpha - 2}
    \hspace{5pt}
     \sums[0 \le m < 2^{\alpha + 2 - \beta} ][m \equiv 1 \bmod 4] 1
      =
       \sum_{0 \le \beta \le \alpha - 2}
         2^{\alpha - \beta}
        =
         2^{\alpha + 1} - 4.
\]
%
Since $\card \U_{\bh}(2^{\alpha + 3}) = \card \bh_2 = 1$, it 
follows that 
\[
 \card \T_{\bh}(2^{\alpha + 3}) 
  = 2(2^{\alpha + 1} - 4) + 1 = 2^{\alpha + 2} - 7, 
\]
and in turn that 
\[
 \delta_{\bh}(2)
  = 
   \frac{\card \T_{\bh}(2^{\alpha + 3}) + \card \bh_2}{2^{\alpha + 3}}
    =
     \frac{2^{\alpha + 2} - 7 + 1}{2^{\alpha + 3}}
      =
       \frac{2^{\alpha + 1} - 3}{2^{\alpha + 2}}.
\]

Let $p \equiv 3 \bmod 4$ and let $\gamma = \nu_p(h)$.
%
If $\nu_p(a) \le \gamma - 1$ then $\nu_p(a + h) = \nu_p(a)$, so 
$a,a + h \in S_p$ and $\nu_p(a),\nu_p(a + h) \le \gamma - 1$ if 
and only if $2 \mid \nu_p(a)$ and $\nu_p(a) \le \gamma - 1$.
%
Therefore,
\begin{align*}
 & \#\{0 \le a < p^{\gamma + 1} : a,a + h \in S_p, \nu_p(a),\nu_p(a + h) \le \gamma - 1\}
  \\
 & \hspace{30pt}  = 
    \sum_{0 \le \beta \le \frac{\gamma - 1}{2}}
     \sums[0 \le m < p^{\gamma + 1}][m \not\equiv 0 \bmod p] 1
  \\
  & \hspace{30pt} = 
     p^{\gamma + 1}
      \bigg(1 - \frac{1}{p}\bigg)
       \sum_{0 \le \beta \le \frac{\gamma - 1}{2}}
        \frac{1}{p^{2\beta}}
  \\
  & \hspace{30pt} = 
   p^{\gamma + 1}
    \bigg(1 + \frac{1}{p}\bigg)^{-1}
     \bigg(1 - \frac{1}{p^{\gamma + (\gamma \bmod 2)}}\bigg).
\end{align*}
%
If $\gamma$ is odd, this accounts for all of 
$\V_{\bh}(p^{\gamma + 1})$.
%
Also, $\bh_p = \emptyset$.
%
In that case we have  
\[
 \delta_{\bh}(p)
  =
   \frac{1}{p^{\gamma + 1}}
    \bigg(
     \card \V_{\bh}(p^{\gamma + 1}) + \card \bh_p\bigg(1 + \frac{1}{p}\bigg)^{-1} 
    \bigg)
     =
      p^{\gamma + 1}
    \bigg(1 + \frac{1}{p}\bigg)^{-1}
     \bigg(1 - \frac{1}{p^{\gamma + 1}}\bigg).
\]
%
If $\gamma$ is even then there may exist 
$a \in \V_{\bh}(p^{\gamma + 1})$ such that  
$\max\{\nu_p(a),\nu_p(a + h)\} = \gamma$.
%
This holds if and only if 
$\nu_p(a) = \nu_p(a + h) = \gamma$.
%
Writing $h = p^{\gamma}h''$ (so that $h'' \not\equiv 0 \bmod p$), 
we have $\nu_p(a) = \nu_p(a + h) = \gamma$ if and only if 
$a = p^{\gamma}m$, where 
$m \not\equiv 0 \bmod p$ and $m \not\equiv - h'' \bmod p$.

Therefore, if $\gamma$ is even then  
\[
 \#\{0 \le a < p^{\gamma + 1} : a, a + h \in S_p, \max\{\nu_p(a),\nu_p(a + h)\} = \gamma\}
  =
   p - 2.
\]
%
Also, $\bh_p = \bh$.
%
In that case we have 
\begin{align*}
 \delta_{\bh}(p) 
  & =
   \frac{1}{p^{\gamma + 1}}
    \bigg(
     \card \V_{\bh}(p^{\gamma + 1}) + \card \bh_p\bigg(1 + \frac{1}{p}\bigg)^{-1}\frac{1}{p}
    \bigg)
  \\
  & =
      \frac{1}{p^{\gamma + 1}}
       \bigg(
        p^{\gamma + 1}
         \bigg(1 + \frac{1}{p}\bigg)^{-1}
          \bigg(1 - \frac{1}{p^{\gamma}}\bigg)
           +
            p - 2
             + 
              \bigg(1 + \frac{1}{p}\bigg)^{-1}
               \frac{2}{p}
       \bigg)
  \\
  & = 
   \bigg(1 + \frac{1}{p}\bigg)^{-1}
    \bigg(1 - \frac{1}{p^{\gamma + 1}}\bigg),
\end{align*}
as before.

Writing $h = 2^{\alpha}p_1^{\alpha_1}\cdots p_r^{\alpha_r}q$ as in 
the statement of the proposition, we see that 
\begin{align*}
 \prod_{p \equiv 3 \bmod 4}
  \bigg(1 + \frac{1}{p}\bigg)^2
   \delta_{\bh}(p)
 & = 
  \prod_{i = 1}^{r}
   \bigg(1 + \frac{1}{p_i}\bigg)
    \bigg(1 - \frac{1}{p_i^{\alpha_i + 1}}\bigg)
     \prods[p \equiv 3 \bmod 4][p \nmid h]
      \bigg(1 + \frac{1}{p}\bigg)
       \bigg(1 - \frac{1}{p}\bigg)
 \\
 & = 
  \prod_{i = 1}^{r}
   \bigg(1 - \frac{1}{p_i}\bigg)^{-1}
    \bigg(1 - \frac{1}{p_i^{\alpha_i + 1}}\bigg)
     \prod_{p \equiv 3 \bmod 4}
      \bigg(1 - \frac{1}{p^2}\bigg).
\end{align*}
%
This last product is equal to $1/(2C^2)$ (see 
\eqref{eq:defLanRamconst}).
%
The left-hand side is $\mathfrak{S}_{\bh}$ without the factor 
of $2^2\delta_{\bh}(2)$.
\end{proof}
%
\end{nix}

\begin{nixnix}
%
\begin{proposition}
 \label{Sprop:sssc}
Let $\bh$ be a set of $k \ge 1$ distinct integers.
%
For $z \ge \max\{2,k\}$ we have 
\begin{equation}
 \label{Seq:sssc1}
  {\textstyle 
  \exp 
   \Big(
    \frac{-c_1(k - 1)^2 z}{(z^2 - (k - 1)^2)\log z}
   \Big)
  }
    \le 
     \prods[p \not\equiv 1 \bmod 4][p \ge z, \, p \nmid \det(\bh)] 
      \delta_{\bz}(p)^{-k}\delta_{\bh}(p)
     \le 
      1,
\end{equation}
where $c_1$ is an absolute positive constant.
%
In fact, the upper bound in \eqref{Seq:sssc1} holds for 
$z \ge 2$.
%
For $z \ge \min\{3,k\}$ we have 
\begin{equation}
 \label{Seq:sssc2}
 {\textstyle 
  \exp 
   \Big(
    \frac{-c_2(k - 1)^2\log |\det(\bh)|}{(z^2 - (k - 1)^2)\log\log 3|\det(\bh)|}
   \Big)
 }
    \le 
     \prods[p \not\equiv 1 \bmod 4][p \ge z, \, p \mid \det(\bh)]
      \delta_{\bz}(p)^{-k}\delta_{\bh}(p)
     \le  
       {\textstyle 
         \exp 
          \Big(
           \frac{c_3k\log |\det(\bh)|}{z\log\log 3|\det(\bh)|}
          \Big) 
 },
\end{equation}
where $c_2$ and $c_3$ are absolute positive constants.
\end{proposition}

\begin{proof}
%
If $k = 1$ then $\mathfrak{S}_{\bh} = 1$, so the estimates 
\eqref{Seq:sssc1} and  \eqref{Seq:sssc2} are 
trivial in this case.
%
Let us assume for the rest of the proof that $k \ge 2$.

If $2 \nmid \det(\bh)$, then $k = 2$ and 
$\delta_{\bz}(2)^{-2}\delta_{\bh}(2) = (1/2)^{-2}(1/4) = 1$ (see 
Proposition \ref{prop:S2h} (c)), so only the primes 
$p \equiv 3 \bmod 4$ have any bearing on the product in 
\eqref{Seq:sssc1}.
%
Let $p \equiv 3 \bmod 4$.
%
If $p \nmid \det(\bh)$ then $k \ge p$ and, by 
Proposition \ref{prop:Sp3h} (c),  
\[
%  \textstyle 
  \delta_{\bz}(p)^{-k}\delta_{\bh}(p)
%   =
%    \bigg(
%     1 - \frac{1}{p + 1}
%    \bigg)^{-k}
%    \bigg(
%     1 - \frac{k}{p + 1}
%    \bigg)   
  =
   \bigg(
    1 + \frac{1}{p}
   \bigg)^{k - 1}
   \bigg(
    1 - \frac{k - 1}{p}
   \bigg).   
\]
%
Therefore, 
\[
%  \textstyle 
  \delta_{\bz}(p)^{-k}\delta_{\bh}(p)
   =
    1 
    - 
     \sum_{j = 2}^k 
      \bigg\{
       (k - 1)\binom{k - 1}{j - 1} - \binom{k - 1}{j}
      \bigg\}
      p^{-j}
       \le 
        1,
\]
so we see that the upper bound in \eqref{Seq:sssc1} 
holds for any $z$.
%
Whether or not $p$ divides $\det(\bh)$ we have, by 
\eqref{eq:delthpropsp3}, 
\begin{equation}
 \label{Seq:ssscpf1}
%  \textstyle 
 \delta_{\bz}(p)^{-k}\delta_{\bh}(p)
  \ge
   \bigg(1 + \frac{k - 1}{p}\bigg)
    \bigg(1 - \frac{\min\{k - 1,p\}}{p}\bigg).
\end{equation}
%
For $z \ge k$ we therefore have 
\begin{align}
 \begin{split}
  \label{Seq:ssscpf2} 
%   \textstyle 
  -\log \prods[p \not\equiv 1 \bmod 4][p \ge z \, p \nmid \det(\bh)]
    \delta_{\bz}(p)^{-k}\delta_{\bh}(p) 
%    \textstyle
 &
      \le 
      -\sum_{p \ge z} \log\bigg(1 - \frac{(k - 1)^2}{p^2}\bigg)
 \\
 & 
%     \textstyle 
     \le 
      (k - 1)^2\bigg(1 - \frac{(k - 1)^2}{z^2}\bigg)^{-1}
       \sum_{p \ge z}
        \frac{1}{p^2}
 \\
 & 
%   \textstyle 
    \ll 
     (k - 1)^2\bigg(1 - \frac{(k - 1)^2}{z^2}\bigg)^{-1}  
      \frac{1}{z\log z}.
 \end{split}
\end{align}
%
(The last bound follows from the bound 
$\sum_{p \le x} 1 \ll x/\log x$, $x \ge 2$, via partial 
summation.)
%
Upon exponentiating, we obtain the lower bound in 
\eqref{Seq:sssc1}.

Next, for $p \ge z \ge 3$ we have $\delta_{\bz}(p)^{-1} = 1 + 1/p$ 
and $\delta_{\bh}(p) \le 1$, so 
\[
%  \textstyle 
  \log \prods[p \not\equiv 1 \bmod 4][p \ge z, \, p \mid \det(\bh)] 
   \delta_{\bz}(p)^{-k}\delta_{\bh}(p) 
    \le 
     k
      \sums[p \ge z][p \mid \det(\bh)] \log\bigg(1 + \frac{1}{p}\bigg)
       \le
        \frac{k}{z}
         \sum_{p \mid \det(\bh)} 1. 
\]
%
The upper bound in \eqref{Seq:sssc2} follows upon exponentiating, 
after applying the bound 
$\sum_{p \mid n} 1 \ll \log n/\log\log n$, $n \ge 3$.
%
For $p \ge z \ge k$, we have, by \eqref{Seq:ssscpf1},  
\[
%  \textstyle 
  \delta_{\bz}(p)^{-k}\delta_{\bh}(p) 
   \ge 1 - \frac{(k - 1)^2}{p^2}
    \ge 1 - \frac{(k - 1)^2}{z^2} 
    > 0.
\]
%
For $z \ge \min\{3,k\}$ we therefore have, similarly to 
\eqref{Seq:ssscpf2}, 
\[
%  \textstyle 
 -\log \prods[p \not\equiv 1 \bmod 4][p \ge z, \, p \mid \det(\bh)]
   \delta_{\bz}(p)^{-k}\delta_{\bh}(p) 
    \le 
     \frac{(k - 1)^2}{z^2}
      \bigg(1 - \frac{(k - 1)^2}{z^2}\bigg)^{-1}
       \sum_{p \mid \det(\bh)} 1,
\]
giving the lower bound in \eqref{Seq:sssc2}.
\end{proof}
%
\end{nixnix}

%%%%%%%%%%%%%%%%%%%%%%%%%%%%%%%%%%%%%%%%%%%%%%%%%%%%%%%%%%%%%%%%%%
%%%%%%%%%%%%%%%%%%%%%%%%%%% REFERENCES %%%%%%%%%%%%%%%%%%%%%%%%%%%
%%%%%%%%%%%%%%%%%%%%%%%%%%%%%%%%%%%%%%%%%%%%%%%%%%%%%%%%%%%%%%%%%%

\begin{thebibliography}{33}

\bibnix{
\bibitem{ARY:15a}
\auth{Aryan, F.}
\bibhref{http://dx.doi.org/10.1112/S0025579314000151}
        {``The distribution of $k$-tuples of reduced residues.''}
{\em Mathematika.} 61(1):72--88, 2015. 
}

\bibnix{
\bibitem{ARY:15b}
\auth{Aryan, F.}
\bibhref{http://dx.doi.org/10.1093/imrn/rnv061}
        {``Distribution of squares modulo a composite number.''}
{\em Int.\ Math.\ Res.\ Not.} (23):12405--12431, 2015.  
}

\bibnix{
\bibitem{BW:00}
\auth{Balog, A.\ and T.\ Wooley.}
\bibhref{http://dx.doi.org/10.4153/CJM-2000-029-6}
        {``Sums of two squares in short intervals.''}
{\em Canad.\ J.\ Math.} 52(4):673--694, 2000.
}

\bibnix{
\bibitem{BAN:85}
\auth{Bantle, G.}
\bibhref{http://dx.doi.org/10.1007/BF01168160}
        {``Obere absch\"atzung f\"ur die anzahl der $B$-zwillinge auf kurzen intervallen.''}
{\em Math.\ Z.} 189(4):561--570, 1985.
}

\bibnix{
\bibitem{BAN:86}
\auth{Bantle, G.}
\bibhref{http://pldml.icm.edu.pl/pldml/element/bwmeta1.element.bwnjournal-publisher-institute_of_mathematics_polish_academy_of_sciences}
        {``An asymptotic formula for $B$-twins.''}
{\em Acta Arith.} 47(4):297--312, 1986.
}

\bibitem{BS-F} 
\auth{Bary-Soroker, L.\ and A.\ Fehm.}
{``Correlations of sums of two squares and other arithmetic functions in function fields.''}
{\em Preprint.}


\bibitem{BT:77}
\auth{Berry, M.\ V.\ and M.\ Tabor.}
\bibhref{http://dx.doi.org/10.1098/rspa.1977.0140}
        {``Level clustering in the regular spectrum.''}
{\em Proc.\ R.\ Soc.\ Lond.\ Ser.\ A Math.\ Phys.\ Eng.\ Sci.}, 356(1686):375--394, 1977.


\bibitem{BIL:95}
\auth{Billingsley, P.}
\bibhref{https://books.google.ca/books?id=a3gavZbxyJcC}
        {\em Probability and measure.}
3rd edn. 
John Wiley \& Sons, New York, 1995.

\bibnix{
\bibitem{CD:14}
\auth{Cilleruelo, J.\ and J.-M.\ Deshouillers.}
\bibhref{http://arxiv.org/abs/1404.5237v2}
        {``Gaps in sumsets of $s$ pseudo $s$-th power sequences.''}
Preprint. \url{arXiv:1404.5237}, 2014.
}

\bibitem{CD:87}
\auth{Cochrane, T.\ and R.\ E.\ Dressler.}
\bibhref{http://dx.doi.org/10.1007/BF01210713}
        {``Consecutive triples of sums of two squares.''}
{\em Arch.\ Math.\ \textup{(}Basel\textup{)}} 49(4):301--304, 1987.

\bibitem{CK:97}
\auth{Connors, R.\ D.\ and J.\ P.\ Keating.}
\bibhref{http://dx.doi.org/10.1088/0305-4470/30/6/009}
        {``Two-point spectral correlations for the square billiard.''}
{\em J.\ Phys.\ A} 30(6):1817--1830, 1997.

\bibitem{DKS:15}
\auth{David, C., D.\ Koukoulopoulos and E.\ Smith.}
\bibhref{http://dx.doi.org/10.1007/s00208-016-1482-2}
        {``Sums of Euler products and statistics of elliptic curves.''}
{\em Math.\ Ann.}, pp.\ 1--68, 2016.

\bibnix{
\bibitem{DHL:98}
\auth{Deshouillers, J.-M., F.\ Hennecart and B.\  Landreau.}
\bibhref{http://pldml.icm.edu.pl/pldml/element/bwmeta1.element.bwnjournal-article-aav85i1p13bwm?q=bwmeta1.element.bwnjournal-number-aa-1998-85-1}
        {``Sums of powers: an arithmetic refinement to the probabilistic model of Erd{\H o}s and R\'enyi.''}
{\em Acta Arith.} 85(1):13--33, 1998.
}

\bibnix{
\bibitem{DIC:19}
\auth{Dickson, L.\ E.}
\bibhref{http://archive.org/details/historyoftheoryo02dickuoft}
        {\em History of the theory of numbers. Vol.\ II: Diophantine analysis.}
Chelsea Publishing Co., New York, 1966.        
}

\bibnix{
\bibitem{EST:31}
\auth{Estermann, T.}
\bibhref{http://dx.doi.org/10.1515/crll.1931.164.173}
        {``\"Uber die darstellungen einer zahl als differenz von zwei produkten.''}
{\em J.\ Reine Angew.\ Math.} 164:173--182, 1931.
}

\bibnix{
\bibitem{EST:32}
\auth{Estermann, T.}
\bibhref{http://dx.doi.org/10.1112/plms/s2-34.1.280}
        {``An asymptotic formula in the theory of numbers.''}
{\em Proc.\ London Math.\ Soc.} s2-34(1):280--292, 1932.
}

\bibnix{
\bibitem{FI:10}
\auth{Friedlander, J.\ and H.\ Iwaniec.}
\bibhref{https://books.google.ca/books?id=GJSKAwAAQBAJ}
        {\em Opera de cribro.}
American Mathematical Society Colloquium Publications, Vol.\ 57.
%
American Mathematical Society, Providence, RI, 2010.        
}

\bibnix{
\bibitem{GAL:74}
\auth{Gallagher, P.\ X.}
\bibhref{http://pldml.icm.edu.pl/pldml/element/bwmeta1.element.bwnjournal-article-aav24i5p491bwm?q=bwmeta1.element.bwnjournal-number-aa-1973-1974-24-5;3&qt=CHILDREN-STATELESS}
        {``Sieving by prime powers.''}
{\em Acta Arith.} 24(5):491--497, 1973-1974.
}

\bibitem{GAL:76}
\auth{Gallagher, P.\ X.}
\bibhref{http://dx.doi.org/10.1112/S0025579300016442}
        {``On the distribution of primes in short intervals.''}
{\em Mathematika} 23(1):4--9, 1976.

\bibnix{
\bibitem{GOG:75}
\auth{Goguel, J.\ H.}
\bibhref{http://dx.doi.org/10.1515/crll.1975.278-279.63}
        {``\"{U}ber summen von zuf\"alligen folgen nat\"urlicher zahlen.''}
{\em J.\ Reine Angew.\ Math.} 278/279:63--77, 1975.
}

\bibnix{
\bibitem{GC:1987}
\auth{Goldston, D.\ A.\ and A.\ Y.\ Cheer.}
\bibhref{http://www.math.sjsu.edu/~goldston/article07.PDF}
        {``A moment method for primes in short intervals.''}
{\em C.\ R.\ Math.\ Rep.\ Acad.\ Sci.\ Canada} 9(2):101--106, 1987.
}

\bibnix{
\bibitem{GRA:95}
\auth{Granville, A.}
\bibhref{http://dx.doi.org/10.1080/03461238.1995.10413946}
        {``Harald Cram\'er and the distribution of prime numbers.''}
{\em Scand.\ Actuar.\ J.} 1:12--28, 1995.
}

\bibitem{GRA:08a}
\auth{Granville, A.}
\bibhref{http://library.msri.org/books/Book44/files/09andrew.pdf}
        {``Smooth numbers: computational number theory and beyond.''}
pp.\ 267--323 in
{\em Algorithmic number theory: lattices, number fields, curves and cryptography.}
Eds. J.\ P.\ Buhler and P.\ Stevenhagen. 
Math.\ Sci.\ Res.\ Inst.\ Publ.\ Vol.\ 44.
Cambridge University Press, Cambridge, 2008.

\bibitem{GK:08}
\auth{Granville, A.\ and P.\ Kurlberg.}
\bibhref{http://dx.doi.org/10.1016/j.aim.2008.04.001}
        {``Poisson statistics via the Chinese remainder theorem.''}
{\em Adv.\ Math.} 218(6):2013--2042, 2008.

\bibnix{
\bibitem{HAR:40}
\auth{Hardy, G.\ H.}
\bibhref{https://books.google.ca/books?id=ECnHLtiCiNsC}
        {\em Ramanujan. Twelve lectures on subjects suggested by his life and work.}
Cambridge University Press, Cambridge, England, 1940.
}

\bibitem{HW:38}
\auth{Hardy, G.\ H.\ and E.\ M.\ Wright.}
\bibhref{https://books.google.ca/books?id=Uc5AAAAAIAAJ}
{\em An introduction to the theory of numbers.}
Clarendon Press, Oxford, 1938.

\bibnix{
\bibitem{HOO:63}
\auth{Hooley, C.}
\bibhref{http://eudml.org/doc/207303}
{``On the difference of consecutive numbers prime to $n$.''}
{\em Acta Arith.} 8(3):343--347, 1963.
}

\bibnix{
\bibitem{HOO:65ii}
\auth{Hooley, C.}
\bibhref{http://www.ams.org/mathscinet-getitem?mr=0186641}
{``On the difference between consecutive numbers prime to $n$. II''}
{\em Publ.\ Math.\ Debrecen} 12:39--49, 1965.
}

\bibitem{HOO:65iii}
\auth{Hooley, C.}
\bibhref{http://dx.doi.org/10.1007/BF01112354}
{``On the difference between consecutive numbers prime to $n$. III''}
{\em Math.\ Z.} 90(5):355--364, 1965.

\bibnix{
\bibitem{HOO:71}
\auth{Hooley, C.}
\bibhref{http://dx.doi.org/10.1007/BF02392056}
        {``On the intervals between numbers that are sums of two squares.''}
{\em Acta Math.} 127(1):279--297, 1971.
}

\bibitem{HOO:72}
\auth{Hooley, C.}
\bibhref{http://www.ams.org/mathscinet-getitem?mr=0384742}
        {``On the intervals between consecutive terms of sequences.''}
pp.\ 129--140 in 
{\em Proceedings of the Symposium in Pure Mathematics of the 
American Mathematical Society, held at St.\ Louis University, 
St.\ Louis, MO, March 27--30, 1972.}
Ed.\ H.\ G.\ Diamond. 
Proceedings of Symposia in Pure Mathematics, Vol.\ XXIV. 
Amer.\ Math.\ Soc., Providence, RI, 1973.

\bibitem{HOO:73}
\auth{Hooley, C.}
\bibhref{http://dx.doi.org/10.1016/0022-314X(73)90046-2}
        {``On the intervals between numbers that are sums of two squares: II.''}
{\em J.\ Number Theory.} 5(3):215--217, 1973.

\bibitem{HOO:74}
\auth{Hooley, C.}
\bibhref{http://dx.doi.org/10.1515/crll.1974.267.207}
        {``On the intervals between numbers that are sums of two squares. III.''}
{\em J.\ Reine Angew.\ Math.} 267:207--218, 1974.

\bibnix{
\bibitem{HOO:94}
\auth{Hooley, C.}
\bibhref{http://dx.doi.org/10.1515/crll.1994.452.79}
        {``On the intervals between numbers that are sums of two squares: IV.''}
{\em J.\ Reine Angew.\ Math.} 452:79--109, 1994.
}

\bibitem{IND:74}
\auth{Indlekofer, K.-H.}
\bibhref{bwmeta1.element.bwnjournal-article-aav26i2p207bwm}
        {``Scharfe untere absch\"atzung f\"ur die anzahlfunktion der $B$-zwillinge''}
{\em Acta Arith.} 26(2):207--212, 1974.

\bibnix{
\bibitem{IK:88}
\auth{Indlekofer, K.-H.\ and I.\ K{\'a}tai.}
\bibhref{http://dx.doi.org/10.1007/BF01196507}
        {``$B$-numbers in short intervals.''}
{\em Arch.\ Math.\ \textup{(}Basel\textup{)}} 50(5):453--458, 1988.
}

\bibitem{IWA:76}
\auth{Iwaniec, H.}
\bibhref{http://pldml.icm.edu.pl/pldml/element/bwmeta1.element.bwnjournal-article-aav29i1p69bwm?q=bwmeta1.element.bwnjournal-number-aa-1976-29-1;3&qt=CHILDREN-STATELESS}
        {``The half-dimensional sieve.''}
{\em Acta Arith.} 29(1):69--95, 1976.

\bibitem{KS:99}
\auth{Katz, N.\ M.\ and P.\ Sarnak.}
\bibhref{https://books.google.ca/books?id=wXyOPbzvowsC}
        {\em Random matrices, Frobenius eigenvalues, and monodromy.}
American Mathematical Society Colloquium Publications, Vol.\ 45.
American Mathematical Society, Providence, RI, 1999.

\bibnix{
\bibitem{KHA:10}
\auth{Khan, R.}
\bibhref{http://dx.doi.org/10.4153/CMB-2010-022-8}
        {``Spacings between integers having typically many prime factors.''}
{\em Canad.\ Math.\ Bull.} 53(1):102--117, 2010.
}

\bibitem{KOW:11}
\auth{Kowalski, E.}
\bibhref{http://dx.doi.org/10.4064/aa148-2-4}
        {``Averages of Euler products, distribution of singular series and the ubiquity of Poisson distribution.''}
{\em Acta Arith.} 148(2):153--187, 2011.

\bibitem{KR:99}
\auth{Kurlberg, P.\ and Z.\ Rudnick.}
\bibhref{http://dx.doi.org/10.1215/S0012-7094-99-10008-1}
        {``The distribution of spacings between quadratic residues.''}
{\em Duke Math.\ J.} 100(2):211--242, 1999.

\bibitem{KUR:00}
\auth{Kurlberg, P.}
\bibhref{http://dx.doi.org/10.1007/s11856-000-1277-7}
{''The distribution of spacings between quadratic residues. II.''}
{\em Israel J.\ Math.}, 120(A):205--224, 2000.

\bibitem{KUR:09}
\auth{Kurlberg, P.}
\bibhref{http://dx.doi.org/10.1142/S1793042109002237}
        {``Poisson spacing statistics for value sets of polynomials.''}
{\em Int.\ J.\ Number Theory} 5(3):489--513, 2009.


\bibitem{LAN:08}
\auth{Landau, E.}
\bibhref{https://zbmath.org/?q=an:39.0264.03}
        {``\"Uber die Einteilung der positiven ganzen Zahlen in vier 
        Klassen nach der Mindestzahl der zu ihrer additiven 
        Zusammensetzung erforderlichen Quadrate.''}
{\em Arch.\ der Math.\ u.\ Phys.\ \textup{(}3\textup{)}.} 13:305--312, 1908.

\bibitem{LAN:94}
\auth{Lang, S.}
\bibhref{http://dx.doi.org/10.1007/978-1-4612-0853-2}
        {\em Algebraic number theory.}
2nd edn.\ 
%
Graduate Texts in Mathematics, Vol.\ 110.        
%
Springer--Verlag, New York, 1994.

\bibnix{
\bibitem{MAT:12}
\auth{Matthiesen, L.}
\bibhref{http://dx.doi.org/10.4064/aa154-3-2}
        {``Linear correlations amongst numbers represented by positive definite binary quadratic forms.''}
{\em Acta Arith.} 154(3):235--306, 2012.
}

\bibnix{
\bibitem{MAT:13}
\auth{Matthiesen, L.}
\bibhref{http://dx.doi.org/10.4064/aa158-3-4}
        {``Correlations of representation functions of binary quadratic forms.''}
{\em Acta Arith.} 158(3):245--252, 2013.
}

\bibnix{
\bibitem{MAT:16}
\auth{Matthiesen, L.}
\bibhref{https://arxiv.org/abs/1606.04482}
        {``Linear correlations of multiplicative functions.''}
Preprint. \url{arXiv:1606.04482}, 2016.
}

\bibnix{
\bibitem{MAY:15}
\auth{Maynard, J.}
\bibhref{http://dx.doi.org/10.1007/978-3-319-22240-0}
        {``Sums of two squares in short intervals.''}
pp.\ 253--273 in
{\em Analytic number theory. In Honor of Helmut Maier's 60th Birthday.}
%
Eds.\ C.\ Pomerance and M.\ Th.\ Rassias.
%
Springer, Cham, 2015.
}

\bibnix{
\bibitem{MOR:93}
\auth{Moree, P.}
\bibhref{http://dx.doi.org/10.1007/BF03026546}
        {``On the number of $y$-smooth natural numbers $\leq x$ representable as a sum of two integer squares.''}
{\em Manuscripta Math.} 80(2):199--211, 1993.
}

\bibnix{
\bibitem{MOT:73}
\auth{Motohashi, Y.}
\bibhref{http://pldml.icm.edu.pl/pldml/element/bwmeta1.element.bwnjournal-article-aav23i4p401bwm?q=bwmeta1.element.bwnjournal-number-aa-1973-23-4;8&qt=CHILDREN-STATELESS}
        {``On the number of integers which are sums of two squares.''}
{\em Acta Arith.} 23(4):401--412, 1973.
}

\bibnix{ 
\bibitem{MOT:83}
\auth{Motohashi, Y.}
\bibhref{http://www.math.tifr.res.in/~publ/ln/tifr72.pdf}
        {\em Lectures on sieve methods and prime number theory.}
Tata Institute of Fundamental Research Lectures on Mathematics and Physics.
%
Published for the Tata Institute of Fundamental Research, Bombay; by Springer-Verlag, Berlin, 1983.
}

\bibitem{NOW:05}
\auth{Nowak, W.\ G.}
\bibhref{http://dx.doi.org/10.2298/PIM0591071N}
        {``On the distribution of $M$-tuples of $B$-numbers.''}
{\em Publ.\ Inst.\ Math.\ \textup{(}Beograd\textup{)} \textup{(}N.\ S.\textup{)}} 77(91):71--78, 2005.

\bibnix{
\bibitem{RAM:00}
\auth{Ramanujan, S.}
\bibhref{https://books.google.ca/books?id=oSioAM4wORMC}
        {\em Collected papers of Srinivasa Ramanujan.}
Eds.\ G.\ H.\ Hardy, P.\ V.\ Seshu Aiyar and B.\ M.\ Wilson.
%
Third printing of the 1927 original, with a new preface and 
commentary by Bruce C.\ Berndt.
%
AMS Chelsea Publishing, Providence, RI, 2000.
}

\bibnix{
\bibitem{RIC:82}
\auth{Richards, I.}
\bibhref{http://dx.doi.org/10.1016/0001-8708(82)90051-2}
        {``On the gaps between numbers which are sums of two squares.''}
{\em Adv.\ in Math.} 46(1):1--2, 1982.
}

\bibitem{RIE:65}
\auth{Rieger, G.\ J.}
\bibhref{http://dx.doi.org/10.1016/S1385-7258(65)50026-3}
        {``Aufeinanderfolgende zahlen als summen von zwei quadraten.''}
{\em Indag.\ Math.\ \textup{(}Proceedings\textup{)}} 68:208--220, 1965. 

\bibitem{RU:14}
\auth{Rudnick, Z.\ and H.\ Uebersch{\"a}r.}
\bibhref{http://dx.doi.org/10.1007/s00023-013-0241-0}
        {``On the eigenvalue spacing distribution for a point scatterer on the flat torus.''}
{\em Ann.\ Henri Poincar\'e} 15(1):1--27, 2014.

\bibnix{
\bibitem{SAR:97}
\auth{Sarnak, P.}
\bibhref{https://books.google.ca/books?id=Rc7YBVeg_O4C&pg=PA181} 
        {``Values at integers of binary quadratic forms.''}
pp.\ 181--203 in 
{\em Harmonic Analysis in Number Theory. Papers in Honour of Carl S.\ Herz.}   
%
Eds.\ S.\ W.\ Drury and M.\ Ram Murty.
%
Canadian Mathematical Society Conference Proceedings, Vol.\ 21. 
%
Amer.\ Math.\ Soc., Providence, RI, 1997.
}

\bibitem{SEL:77}
\auth{Selberg, A.}
\bibhref{link.springer.com/content/pdf/10.1007/BFb0063067.pdf}
        {``Remarks on multiplicative functions''.}
pp.\ 232--241 in 
{\em Number theory day \textup{(}Proc.\ Conf., Rockefeller Univ., New York, 1976\textup{)}.}
Lecture Notes In Mathematics, Vol.\ 626. 
Springer, Berlin, 1977.

\bibnix{
\bibitem{SHA:64}
\auth{Shanks, D.}
\bibhref{http://www.jstor.org/stable/2003407}
        {``The second-order term in the asymptotic expansion of $B(x)$.''}
{\em Math.\ Comp.} 18(85):75--86, 1964.
}

\bibitem{SMI:13}
\auth{Smilansky, Y.}
\bibhref{http://dx.doi.org/10.1142/S1793042113500516}
        {``Sums of two squares --- pair correlation and distribution in short intervals.''}
{\em Int.\ J.\ Number Theory} 9(7):1687--1711, 2013.

\bibnix{
\bibitem{SOU:07}
\auth{Soundararajan, K.}
\bibhref{http://arxiv.org/abs/math/0606408}
    {``The distribution of prime numbers.''} 
pp.\ 59--83 in 
{\em Equidistribution in number theory, an introduction.}
%
Eds.\ A.\ Granville and Z.\ Rudnick.
%
NATO Sci.\ Ser.\ II Math.\ Phys.\ Chem.\ 237.
%
Springer, Dordrech, 2007.
}

\bibnix{
\bibitem{STA:28}
\auth{Stanley, G.\ K.}
\bibhref{http://dx.doi.org/10.1112/jlms/s1-3.3.232}
        {``Two assertions made by Ramanujan.''}
{\em J.\ London Math.\ Soc.} s1-3(3):232--237, 1928.
}

\bibnix{
\bibitem{TEN:95}
\auth{Tenenbaum, G.}
\bibhref{https://books.google.ca/books?id=UEk-CgAAQBAJ}
        {\em Introduction to analytic and probabilistic number theory.}
%
Translated from the second French edition (1995) by C.\ B.\ Thomas.
%
Cambridge University Press, Cambridge, 1995.
}

\end{thebibliography}

%****************************************************************%
%****************************************************************%
%**************************** JETSAM ****************************%
%****************************************************************%
%****************************************************************%

\begin{jetsam}

\section{Jetsam}
 \label{sec:jetsam}
 
%%%%%%%%%%%%%%%%%%%%%%%%%%%%%%%%%%%%%%%%%%%%%%%%%%%%%%%%%%%%%%%%%%
%%%%%%%%%%%%%%%%%%%%%%%%%%% HISTORY %%%%%%%%%%%%%%%%%%%%%%%%%%%%%%
%%%%%%%%%%%%%%%%%%%%%%%%%%%%%%%%%%%%%%%%%%%%%%%%%%%%%%%%%%%%%%%%%%
 
Sums of two squares are historically, perhaps, the most studied 
integers after the primes.
%
(See Volume II, Chapter VI of Dickson's {\em History of the Theory 
of Numbers} \cite{DIC:19}.)
%
The special case of Brahmagupta's identity, 
\[
 (a^2 + b^2)(c^2 + d^2) 
  =
   (ac + bd)^2 + (ad - bc)^2,
\]
is implicit in Diophantus' {\em Arithmetica}, and was also derived 
by Fibonacci (in his {\em Liber Quadratorum}), Bachet, and 
doubtless many others.
%
Fermat's theorem on sums of two squares, which was anticipated by 
Girard and first completely proved by Euler, states that all 
primes $p \equiv 1 \bmod 4$ are sums of two squares.
%
Because sums of two squares are not congruent to $3$ modulo $4$, 
we are led to the well-known characterization of nonzero sums of 
two squares as the integers whose canonical factorization is of 
the form  
\begin{equation}
 \label{eq:canfac}
   2^{\beta_2}
    \prod_{p \equiv 1 \bmod 4}p^{\beta_p}
     \prod_{p \equiv 3 \bmod 4}p^{2\beta_p}.
\end{equation}

This is a starting point for a proof of Landau's result 
\eqref{eq:sotsnt}.
%
A refinement of Landau's proof (see \cite[(4.6.4)]{HAR:40} or 
\cite[II.5, Theorem 3]{TEN:95})) shows that there exists a 
sequence of (explicitly given) numbers $D_1,D_2,\ldots$ such that, 
for all $N \ge 0$, we have
\begin{equation}
 \label{eq:sotsntii}
 \speccount(x)
  =
   \frac{Cx}{\sqrt{\log x}}
    \bigg\{
     1 + \frac{D_1}{\log x} + \cdots + \frac{D_N}{(\log x)^N} + 
       O_N\bigg(\frac{1}{(\log x)^{N + 1}}\bigg) 
    \bigg\}. 
\end{equation}
%
Unfortunately, $D_1,D_2,\ldots$ are difficult to evaluate 
numerically. 
%
As stated by Shanks \cite{SHA:64}, ``an unsolved problem of 
interest is to find [an approximation to $\speccount(x)$] that could 
be computed without undue difficulty by a {\em convergent} 
process, and which would be accurate to $O_N(x(\log x)^{-N})$ for 
[any given] $N$''.
%
Apropos of this, a rare ``mistake'' of Ramanujan is often 
recounted (see \cite[Chapter IV]{HAR:40}, \cite{STA:28, SHA:64}, 
or \cite[pp.\ xxiv--xxviii]{RAM:00} for instance). 

Unaware of Landau's work, Ramanujan claimed, in his very first   
correspondence with Hardy dated 1913 
(see \cite[pp.\ xxiv--xxviii]{RAM:00}), that 
\[
 \speccount(x)
  = 
   C\int_2^x \frac{\dd t}{\sqrt{\log t}} + \text{small error}, 
\]
where the ``small error'' is of order $\sqrt{x/\log x}$.
%
After partial integration, Ramanujan's claim would imply that 
$D_1 = 1/2$ in \eqref{eq:sotsntii}.
%
However, it turns out that $D_1 = 0.58194\ldots$, as was 
shown by Shanks \cite{SHA:64} (by correcting earlier work of  
Hardy's student Stanley \cite{STA:28}).
%
Even so, Ramanujan's incorrect estimate for $\speccount(x)$ is, as 
pointed out by Shanks \cite{SHA:64}, numerically a better 
approximation than $Cx/\sqrt{\log x}$ (because $D_1$ is close to 
$1/2$). 
%
We mention this lest the numerical evidence in support of 
Conjectures \ref{con:poisdist} and \ref{con:sotsktups} be not all 
that striking.
%
Also, we are circumspect in making more exact conjectures as to the 
nature of the error term implicit in \eqref{eq:sotsktups}.

In general, sequences like the sequence of sums of two squares, 
which are ``survivors'' of a sieving process of positive 
dimension, are expected to have many properties in common with the 
sequence of primes.
%
``Folklore'' conjectures entail analogs of the Hardy--Littlewood 
prime $k$-tuples conjecture for such sequences. 
%
Given the history outlined above, there is particular interest in 
the specific case of sums of two squares, yet it seems difficult 
to find an explicit formulation of a $k$-tuples conjecture for 
sums of two squares in the literature.

%%%%%%%%%%%%%%%%%%%%%%%%%%%%%%%%%%%%%%%%%%%%%%%%%%%%%%%%%%%%%%%%%%
%%%%%%%%%%%%%%%%%%%%%%% BASIC ESTIMATES %%%%%%%%%%%%%%%%%%%%%%%%%%
%%%%%%%%%%%%%%%%%%%%%%%%%%%%%%%%%%%%%%%%%%%%%%%%%%%%%%%%%%%%%%%%%%

\begin{proposition}
 \label{prop:S2est}
For $x \ge 2$ we have 
\[
  \sums[n \le x][n \in S_2] 1 = \frac{x}{2} + O(\log x).
\]
\end{proposition}

\begin{proof}
Let $x \ge 2$, and let $\alpha$ be the integer such that 
$2^{\alpha} \le x < 2^{\alpha + 1}$.
%
By Proposition \ref{prop:S2S3S1}, the positive integers $n$ in 
$S_2$ are precisely those of the form $2^{\beta}m$, where 
$\beta \ge 0$ and $m \equiv 1 \bmod 4$.
%
Thus, 
\begin{align*}
  \sums[n \le x][n \in S_2] 1
  & = 
   \sum_{\beta = 0}^{\alpha}
    \sums[m \le x/2^{\beta}][m \equiv 1 \bmod 4] 1
     =
      \frac{x}{4}\sum_{\beta = 0}^{\alpha} \frac{1}{2^{\beta}}
       + O(\alpha)
     =
      \frac{x}{2} - \frac{x}{2^{\alpha + 2}} + O(\alpha).
\end{align*}
%
Since $x/2^{\alpha + 2} < 1/2$ and $\alpha \le \log x/\log 2$, the 
result follows.
\end{proof}

\begin{proposition}
 \label{prop:Sp3est}
Let $p \equiv 3 \bmod 4$.
%
For $x \ge 1$ we have 
\[
   \sums[n \le x][n \in S_p] 1
    =
     x\bigg(1 + \frac{1}{p}\bigg)^{-1}
      + O\bigg(1 + \frac{\log x}{\log p}\bigg).
\]
\end{proposition}

\begin{proof}
%
First, suppose $1 \le x < p$.
%
By Proposition \ref{prop:S2S3S1}, 
$\{1,\ldots,p - 1\} \subseteq S_p$, so 
\[
  \sums[n \le x][n \in S_p] 1
  = 
   x + O(1)
    =
     x\bigg(1 + \frac{1}{p}\bigg)^{-1} + O(1).
\]
%
Now let $x \ge p$, and let $\alpha$ be the integer such that 
$p^{\alpha} \le x < p^{\alpha + 1}$.
%
By Proposition \ref{prop:S2S3S1}, the positive integers $n$ in 
$S_p$ are precisely those for which $p^{2\beta} \emid n$ for some 
$\beta \ge 0$.
%
Thus, 
\begin{align*}
 \sums[n \le x][n \in S_p] 1
  =
   \sum_{0 \le \beta \le \frac{\alpha}{2}}
    \Big\{
     \sums[n \le x][p^{2\beta} \mid n] 1
      -
       \sums[n \le x][p^{2\beta + 1} \nmid n] 1
    \Big\}
     =
      x\bigg(1 - \frac{1}{p}\bigg)
        \sum_{0 \le \beta \le \frac{\alpha}{2}}
         \frac{1}{p^{2\beta}}
         +
          O(\alpha).
\end{align*}
%
Since 
\[
 \sum_{0 \le \beta \le \frac{\alpha}{2}} \frac{1}{p^{2\beta}}
  =
   \bigg(1 - \frac{1}{p^2}\bigg)^{-1}
    \bigg(1 - \frac{p^{\alpha \bmod 2}}{p^{\alpha + 2}}\bigg)
   =
    \bigg(1 - \frac{1}{p}\bigg)^{-1}
     \bigg(1 + \frac{1}{p}\bigg)^{-1} 
   + O\bigg(\frac{1}{x}\bigg), 
\]
and since $\alpha \le \log x/\log p$, we see that  
\[
 \sums[n \le x][n \in S_p] 1
  = 
   x\bigg(1 + \frac{1}{p}\bigg)^{-1} 
   + O\bigg(\frac{\log x}{\log p}\bigg).
\]
%
The result follows by combining the estimates for $1 \le x < p$ 
and $x \ge p$.
\end{proof}

%%%%%%%%%%%%%%%%%%%%%%%%%%%%%%%%%%%%%%%%%%%%%%%%%%%%%%%%%%%%%%%%%%
%%%%%%%%%%%% AVERAGE OF SINGULAR SERIES FOR k = 2 %%%%%%%%%%%%%%%%
%%%%%%%%%%%%%%%%%%%%%%%%%%%%%%%%%%%%%%%%%%%%%%%%%%%%%%%%%%%%%%%%%%

\begin{lemma}
 \label{lem:sssak=2aux1}
Let $\bh = \{h_1,\ldots,h_k\}$ be a set of integers with 
$0 \le h_1 < \cdots < h_k \le y$.
%
If $y \ge \min\{\e^3,\e^k\}$ then 
\begin{equation}
 \label{eq:sotssingserconv3}
  \mathfrak{S}_{\bh}
   =
    \bigg(1 + O\bigg(\frac{k^3}{\log\log y}\bigg)\bigg)
     \prods[p \not\equiv 1 \bmod 4][p \le \log y] 
      \delta_{\bz}(p)^{-k}\delta_{\bh}(p).
\end{equation}
\end{lemma}

\begin{proof}
%
Let $z \ge \min\{3,k\}$ and suppose $y \le \e^z$.
%
Since $|\det(\bh)| \le y^{k^2} \le \e^{zk^2}$, we have 
$\log |\det(\bh)|/\log\log 3|\det(\bh)| \ll k^2z/\log z$.
%
Combining with \eqref{Seq:sssc1} and 
\eqref{Seq:sssc2} of 
Proposition \ref{Sprop:sssc}, we obtain 
\[
 \prods[p \equiv 3 \bmod 4][p > z]
  \delta_{\bz}(p)^{-k}\delta_{\bh}(p)
   =
    1 + O\bigg(\frac{k^3}{\log z}\bigg),
\]
from which \eqref{eq:sotssingserconv3} follows.
\end{proof}
 
Given a number $t \ge 1$ and a set of primes $\mathscr{P}$, let 
\begin{equation}
 \label{eq:defPsi}
  \Psi(t,\mathscr{P})
   \defeq 
%     \sums[n \le t][p \mid n \implies p \in \mathscr{P}] 1.
     \#\{n \le t : p \mid n \implies p \in \mathscr{P}\}.
\end{equation}
%
If $\mathscr{P}$ is the set of all primes less than or equal to 
some number $z$, then $\Psi(t,\mathscr{P})$ is the number of 
$z$-smooth numbers less than or equal to $t$, denoted $\Psi(t,z)$.

\begin{proposition}
 \label{prop:ssak=2pre}
Let $\mathscr{P}$ be any nonempty, finite subset of the primes 
that are not congruent to $1$ modulo $4$.
%
For $t \ge 1$, we have 
\[
  \sum_{g \le t}
   {\textstyle \prod_{p \in \mathscr{P}} } \delta_{\bz}(p)^{-2}\delta_{\{0,g\}}(p)
    =
     t 
      + O\Big(
              2^{\ecard\mathscr{P}}
               \Psi(t,\mathscr{P})  
                {\textstyle \prod_{p \in \mathscr{P}}\big(1 + \frac{1}{p}\big)}
         \Big).
\]
\end{proposition}

\begin{proof}
%
We begin with two observations, which we will use in the course of 
the proof.
%
First, let $g_1$ and $g_2$ be integers, and set $g = g_1g_2$.
%
Let $p$ be any prime and suppose $p \nmid g_1$, so that 
$\alpha \defeq \nu_p(g) = \nu_p(g_2)$.
%
Recall from Proposition \ref{Aprop:sssk=2} that in the case 
where $p \equiv 3 \bmod 4$, we have
\begin{equation*}
  \delta_{\bz}(p)^{-1}\delta_{\{0,g\}}(p)
   = 
    \delta_{\bz}(p)^{-1}\delta_{\{0,g_2\}}(p)
     =
       \bigg(1 - \frac{1}{p^{\alpha + 1}}\bigg),
\end{equation*}
while in the case where $p = 2$, we have 
\begin{equation*}
  \delta_{\bz}(2)^{-2}\delta_{\{0,g\}}(2)
   = 
    \delta_{\bz}(2)^{-2}\delta_{\{0,g_2\}}(2)
     =
        \begin{cases}
         1                                     & \alpha = 0    \\
         \frac{2^{\alpha + 1} - 3}{2^{\alpha}} & \alpha \ge 1.
        \end{cases}
\end{equation*}
%
Second, for $\beta \ge \alpha \ge 0$ (and any prime $p$) we have 
\begin{equation}
 \label{eq:ssak=2prepf1}
  \sum_{\alpha = 0}^{\beta}
   \frac{1}{p^{\alpha}}
    \bigg(1 - \frac{1}{p^{\alpha + 1}}\bigg)
     =
      \bigg(1 - \frac{1}{p^2}\bigg)^{-1}
      +
       O\bigg(\frac{1}{p^{\beta + 1}}\bigg);
\end{equation}
%
\begin{nixnix}
%
\begin{align*}
   \sum_{\alpha = 0}^{\beta}
    \frac{1}{p^{\alpha}}
     \bigg(1 - \frac{1}{p^{\alpha + 1}}\bigg)
 & = 
  \sum_{\alpha = 0}^{\infty} \frac{1}{p^{\alpha}}
   -
    \frac{1}{p}
     \sum_{\alpha = 0}^{\infty} \frac{1}{p^{2\alpha}}
      +
       O\bigg(\sum_{\alpha > \beta} \frac{1}{p^{\alpha}}\bigg)
 \\
 & = 
  \bigg(1 - \frac{1}{p}\bigg)^{-1}
   -
    \frac{1}{p}
     \bigg(1 - \frac{1}{p^2}\bigg)^{-1}
     +
      O\bigg(\frac{1}{p^{\beta + 1}}\bigg)
 \\
 & = 
  \bigg(1 - \frac{1}{p^2}\bigg)^{-1}
     +
      O\bigg(\frac{1}{p^{\beta + 1}}\bigg) 
\end{align*}
%
\end{nixnix}
%
we also have 
\begin{equation}
 \label{eq:ssak=2prepf2}
  \sum_{\alpha = 1}^{\beta}
   \frac{1}{2^{\alpha}}
    \cdot 
     \frac{2^{\alpha + 1} - 3}{2^{\alpha}}
    =
      1 + O\bigg(\frac{1}{2^{\beta}}\bigg).
\end{equation}
%
\begin{nixnix}
\begin{align*}
 \sum_{\alpha = 1}^{\beta}
  \frac{1}{2^{\alpha}}
   \cdot \frac{2^{\alpha + 1} - 3}{2^{\alpha}}
 & =
    \sum_{\alpha = 1}^{\beta}
     \bigg(\frac{2}{2^{\alpha}} - \frac{3}{2^{2\alpha}}\bigg)
 \\
 & = 
    \sum_{\alpha - 1 = 0}^{\infty} \frac{1}{2^{\alpha - 1}}
     -
     \frac{3}{4}
      \sum_{\alpha - 1 = 0}^{\infty} \frac{1}{4^{\alpha - 1}}
       +
        O\bigg(\sum_{\alpha > \beta} \frac{1}{2^{\alpha}}\bigg)
 \\
 & = 
    2 - \frac{3}{4}\bigg(\frac{3}{4}\bigg)^{-1} + O\bigg(\frac{1}{2^{\beta + 1}}\bigg) 
 \\
 & = 
     1 + O\bigg(\frac{1}{2^{\beta}}\bigg)
\end{align*}
\end{nixnix}

Suppose $\mathscr{P} = \{p_0,p_1,\ldots,p_r\}$, where $r \ge 0$ 
and $p_0,p_1,\ldots,p_r$ are distinct primes.
%
Suppose further that 
$p_1 \equiv \cdots \equiv p_r \equiv 3 \bmod 4$, while either 
$p_0 = 2$ or $p_0 \equiv 3 \bmod 4$ (to be specified).
%
If $g_1$ is any integer coprime with $p_0\cdots p_r$ then, by our 
first observation, 
\begin{equation*}
  \prod_{i = 0}^r
   \delta(p_i)^{-2}\delta_{\{0,g_1p_0^{\alpha_0}\cdots p_r^{\alpha_r}\}}(p_i)
    =
     \delta(p_0)^{-2}\delta_{\{0,p_0^{\alpha_0}\}}(p_0)
      \prod_{i = 1}^r\delta(p_i)^{-1}\bigg(1 - \frac{1}{p_i^{\alpha_i + 1}}\bigg).
\end{equation*}
%
\begin{nixnix}
\begin{equation*}
  \prod_{i = 0}^r
   \delta_{\{0,g_1p_0^{\alpha_0}\cdots p_r^{\alpha_r}\}}(p_i)
   =
     \prod_{i = 0}^r\delta_{\{0,p_i^{\alpha_i}\}}(p_i)
   =
        \delta_{\{0,p_0^{\alpha_0}\}}(p_0)
         \prod_{i = 1}^r\delta(p_i)\bigg(1 - \frac{1}{p_i^{\alpha_i + 1}}\bigg) 
\end{equation*}
\end{nixnix}
%
Now let $t \ge 1$, and for $0 \le j \le r$ let 
\[
 A_{r - j}
  \defeq 
   \{(\alpha_0,\ldots,\alpha_{r - j}) : p_0^{\alpha_0}\cdots p_{r - j}^{\alpha_{r - j}} \le t\}.
\]
%
We have 
\begin{align*}
  &
    \sum_{g \le t}
    {\textstyle \prod_{i = 0}^r } \delta(p_i)^{-2}\delta_{\{0,g\}}(p_i) 
  \\
  & = 
    \sum_{(\alpha_0,\ldots,\alpha_r) \in A_r}
     \sums[g_1 \le t/p_0^{\alpha_0}\cdots p_r^{\alpha_r}]
      {\textstyle \prod_{i = 0}^r } \delta(p_i)^{-2}\delta_{\{0,g_1p_0^{\alpha_0}\cdots p_r^{\alpha_r}\}}(p_i)
 \\
  & = 
   \bigg(\prod_{i = 1}^r \delta(p_i)^{-1}\bigg)
    \sum_{(\alpha_0,\ldots,\alpha_r) \in A_r}
     \delta(p_0)^{-2}\delta_{\{0,p_0^{\alpha_0}\}}(p_0)
      {\textstyle \prod_{i = 1}^r\big(1 - \frac{1}{p_i^{\alpha_i + 1}}\big)}
       \sums[g_1 \le t/p_0^{\alpha_0}\cdots p_r^{\alpha_r}] 1.
\end{align*}
%
To estimate this last sum we apply the sieve of 
Eratosthenes--Legendre: 
\begin{equation*}
  \sums[g_1 \le t/p_0^{\alpha_0}\cdots p_r^{\alpha_r}][(g_1,p_0\cdots p_r) = 1] 1
 =
   \frac{t}{p_0^{\alpha_0}\cdots p_r^{\alpha_r}}
   \prod_{i = 0}^r\bigg(1 - \frac{1}{p_i}\bigg)
    + 
     O(2^r).
\end{equation*}
%
\begin{nixnix}
\begin{align*}
 \sums[g_1 \le t/p_0^{\alpha_0}\cdots p_r^{\alpha_r}][(g_1,p_0\cdots p_r) = 1] 1
  & =
   \sum_{g_1 \le t/p_0^{\alpha_0}\cdots p_r^{\alpha_r}}
    \sums[d \mid g_1][d \mid p_0\cdots p_r] \mu(d)
 \\
  & = 
   \sum_{d \mid p_0\cdots p_r}
    \mu(d)
     \sums[g_1 \le t/p_0^{\alpha_0}\cdots p_r^{\alpha_r}][g_1 \equiv 0 \bmod d] 1
 \\
 & = 
   \sum_{d \mid p_0\cdots p_r}
    \mu(d)
     \bigg(\frac{t}{dp_0^{\alpha_0}\cdots p_r^{\alpha_r}} + O(1)\bigg) 
 \\
  & = 
   \frac{t}{p_0^{\alpha_0}\cdots p_r^{\alpha_r}}
    \sum_{d \mid p_0\cdots p_r} \frac{\mu(d)}{d}
    +
     O\bigg(\sum_{d \mid p_0\cdots p_r} |\mu(d)|\bigg)
 \\
 & = 
  \frac{t}{p_0^{\alpha_0}\cdots p_r^{\alpha_r}}
   \prod_{i = 0}^r\bigg(1 - \frac{1}{p_i}\bigg)
    + 
     O(2^r) 
\end{align*}
\end{nixnix}
%
Combining and recalling that 
$\delta(p_i)^{-1} = 1 + 1/p_i$ for $i = 1,\ldots,r$, we obtain 
\begin{align}
 \begin{split}
  \label{eq:ssak=2prepf3} 
 & 
   \sum_{g \le t}
   {\textstyle \prod_{i = 0}^r } \delta(p_i)^{-2}\delta_{\{0,g\}}(p_i) 
 \\
 & = 
  t\bigg(1 - \frac{1}{p_0}\bigg)
   \prod_{i = 1}^r \bigg(1 - \frac{1}{p_i^2}\bigg) 
    \sum_{(\alpha_0,\ldots,\alpha_r) \in A_r}
     \frac{\delta(p_0)^{-2}\delta_{\{0,p_0^{\alpha_0}\}}(p_0)}{p_0^{\alpha_0}}
      \prod_{i = 1}^r\frac{1}{p_i^{\alpha_i}}\bigg(1 - \frac{1}{p_i^{\alpha_i + 1}}\bigg)
 \\
 & \hspace{120pt} + 
     O\bigg(
       2^r(\card A_r)\prod_{i = 1}^r\bigg(1 + \frac{1}{p_i}\bigg) 
      \bigg).
  \end{split}
\end{align}
%
By repeating ($r$ times) an argument that uses 
\eqref{eq:ssak=2prepf1}, we verify that 
\begin{align}
 \begin{split}
  \label{eq:ssak=2prepf4}
 & 
  \prod_{i = 1}^r\bigg(1 - \frac{1}{p_i^2}\bigg)
   \sum_{(\alpha_0,\ldots,\alpha_r) \in A_r}
    \frac{\delta(p_0)^{-2}\delta_{\{0,p_0^{\alpha_0}\}}(p_0)}{p_0^{\alpha_0}}
     \prod_{i = 1}^r 
      \frac{1}{p_i^{\alpha_i}}
       \bigg(1 - \frac{1}{p_i^{\alpha_i + 1}}\bigg) 
  \\
 & \hspace{30pt} =
   \sum_{\alpha_0 \in A_0}
    \frac{\delta(p_0)^{-2}\delta_{\{0,p_0^{\alpha_0}\}}(p_0)}{p_0^{\alpha_0}}
     +
      O\bigg(\frac{1}{t}\sum_{j = 1}^{r - 1} \card A_{r - j}\bigg).
 \end{split}
\end{align}
%
\begin{nixnix}
Given $(\alpha_0,\ldots,\alpha_{r - 1}) \in A_{r - 1}$, let 
$\beta_r$ be the integer such that 
\[
 p^{\beta_r} 
  \le t/(p_0^{\alpha_0}\cdots p_{r - 1}^{\alpha_{r-1}}) 
   < p_r^{\beta_r + 1}.
\]
%
Below, the second equality follows from 
\eqref{eq:ssak=2prepf1}:  
\begin{align*}
 & 
 \sum_{(\alpha_0,\ldots,\alpha_r) \in A_r}
  \frac{\delta(p_0)^{-2}\delta_{\{0,p_0^{\alpha_0}\}}(p_0)}{p_0^{\alpha_0}}
   \prod_{i = 1}^r 
    \frac{1}{p_i^{\alpha_i}}
     \bigg(1 - \frac{1}{p_i^{\alpha_i + 1}}\bigg)
 \\
 & \hspace{10pt} = 
  \sum_{(\alpha_0,\ldots,\alpha_{r-1}) \in A_{r-1}}
   \frac{\delta(p_0)^{-2}\delta_{\{0,p_0^{\alpha_0}\}}(p_0)}{p_0^{\alpha_0}}
    \prod_{i = 1}^{r - 1} 
     \frac{1}{p_i^{\alpha_i}}
      \bigg(1 - \frac{1}{p_i^{\alpha_i + 1}}\bigg)
       \sum_{\alpha_r = 0}^{\beta_r}
        \frac{1}{p_r^{\alpha_r}}
         \bigg(1 - \frac{1}{p_r^{\alpha_r}}\bigg)
 \\
 & \hspace{10pt} = 
  \sum_{(\alpha_0,\ldots,\alpha_{r-1}) \in A_{r-1}}
   \frac{\delta(p_0)^{-2}\delta_{\{0,p_0^{\alpha_0}\}}(p_0)}{p_0^{\alpha_0}}
    \prod_{i = 1}^{r - 1} 
     \frac{1}{p_i^{\alpha_i}}
      \bigg(1 - \frac{1}{p_i^{\alpha_i + 1}}\bigg)
 \\
 & \hspace{180pt} \times 
       \bigg\{
        \bigg(1 - \frac{1}{p_r^2}\bigg)^{-1}
          + O\bigg(\frac{p_0^{\alpha_0}\cdots p_{r-1}^{\alpha_{r-1}}}{t} \bigg)
       \bigg\}
 \\
 & \hspace{10pt} =
    \bigg(1 - \frac{1}{p_r^2}\bigg)^{-1}
     \sum_{(\alpha_0,\ldots,\alpha_{r-1}) \in A_{r-1}}
      \frac{\delta(p_0)^{-2}\delta_{\{0,p_0^{\alpha_0}\}}(p_0)}{p_0^{\alpha_0}}
       \prod_{i = 1}^{r - 1} 
        \frac{1}{p_i^{\alpha_i}}
         \bigg(1 - \frac{1}{p_i^{\alpha_i + 1}}\bigg)
 \\
 & \hspace{210pt} + 
           O\bigg(\frac{\card A_{r-1}}{t}\bigg).
\end{align*}
%
Repeating this argument $r - 1$ more times, we obtain
\begin{align*}
 & 
  \prod_{i = 1}^r\bigg(1 - \frac{1}{p_i^2}\bigg)
   \sum_{(\alpha_0,\ldots,\alpha_r) \in A_r}
    \frac{\delta(p_0)^{-2}\delta_{\{0,p_0^{\alpha_0}\}}(p_0)}{p_0^{\alpha_0}}
     \prod_{i = 1}^r 
      \frac{1}{p_i^{\alpha_i}}
       \bigg(1 - \frac{1}{p_i^{\alpha_i + 1}}\bigg) 
  \\
 & \hspace{30pt} =
   \sum_{\alpha_0 \in A_0}
    \frac{\delta(p_0)^{-2}\delta_{\{0,p_0^{\alpha_0}\}}(p_0)}{p_0^{\alpha_0}}
     +
      O\bigg(\frac{1}{t}\sum_{j = 1}^{r - 1} \card A_{r - j}\bigg).
\end{align*}
\end{nixnix}
%
If $p_0 \equiv 3 \bmod 4$ then repeating the argument one more 
time leads to 
\begin{align}
 \begin{split}
  \label{eq:ssak=2prepf5}
 & 
  \bigg(1 - \frac{1}{p_0}\bigg)
   \prod_{i = 1}^r\bigg(1 - \frac{1}{p_i^2}\bigg)
    \sum_{(\alpha_0,\ldots,\alpha_r) \in A_r}
     \frac{\delta(p_0)^{-2}\delta_{\{0,p_0^{\alpha_0}\}}(p_0)}{p_0^{\alpha_0}}
      \prod_{i = 1}^r 
       \frac{1}{p_i^{\alpha_i}}
        \bigg(1 - \frac{1}{p_i^{\alpha_i + 1}}\bigg) 
  \\
 & \hspace{30pt} =
   1
     +
      O\bigg(\frac{1}{t}\sum_{j = 1}^{r} \card A_{r - j}\bigg).
 \end{split}
\end{align}
%
If $p_0 = 2$, then by a similar argument that uses 
\eqref{eq:ssak=2prepf2}, we obtain 
\begin{equation*}
   \bigg(1 - \frac{1}{p_0}\bigg)
    \sum_{\alpha_0 \in A_0}
     \frac{\delta(p_0)^{-2}\delta_{\{0,p_0^{\alpha_0}\}}(p_0)}{p_0^{\alpha_0}}
   = 
    1 + O\bigg(\frac{1}{2^{\beta_0}}\bigg).
\end{equation*}
%
\begin{nixnix}
Assume now that $p_0 = 2$.
%
Let $\beta_0$ be the integer such that 
$2^{\beta_0} \le t < 2^{\beta_0 + 1}$.
%
By \eqref{eq:ssak=2prepf2} we have 
\begin{equation*}
   \bigg(1 - \frac{1}{p_0}\bigg)
    \sum_{\alpha_0 \in A_0}
     \frac{\delta(p_0)^{-2}\delta_{\{0,p_0^{\alpha_0}\}}(p_0)}{p_0^{\alpha_0}}
   = 
   \frac{1}{2}
    \bigg\{ 
     1
      +
     \sum_{\alpha_0 = 1}^{\beta_0}
      \frac{1}{2^{\alpha_0}}
       \cdot 
        \frac{2^{\alpha_0 + 1} - 3}{2^{\alpha_0}}
    \bigg\}
   = 
    1 + O\bigg(\frac{1}{2^{\beta_0}}\bigg).
\end{equation*}
\end{nixnix}

Putting this into \eqref{eq:ssak=2prepf4}, we again obtain 
\eqref{eq:ssak=2prepf5}.
%
In any case, putting \eqref{eq:ssak=2prepf5} into 
\eqref{eq:ssak=2prepf3}, we finally obtain 
\[ 
  \sum_{1 \le g \le t}
  {\textstyle \prod_{i = 0}^r } \delta(p_i)^{-2}\delta_{\{0,g\}}(p_i) 
   =
    t 
     + 
      O\bigg(
        2^r(\card A_r)\prod_{i = 1}^r\bigg(1 + \frac{1}{p_i}\bigg) 
        + \sum_{j = 1}^r \card A_{r - j} 
       \bigg).
\]
%
As 
$
 \sum_{j = 1}^r \card A_{r - j} 
  \le r(\card A_r) 
   \le 2^r(\card A_r)
$, 
we can disregard this sum in the above $O$-term.
%
We complete the proof by noting that $A_r$ is in one-to-one 
correspondence with the integers $n \le t$ whose prime divisors 
all lie in $\mathscr{P} = \{p_0,\ldots,p_r\}$, and so  
$\card A_{r} = \Psi(t,\mathscr{P})$.
\end{proof}

\begin{proposition}
 \label{prop:sssak=2}
%
For $y \ge \e^3$ we have 
\[
  \sum_{0 \le h_1 < h_2 \le y}
   \mathfrak{S}_{\{h_1,h_2\}}
    =
     \frac{y^2}{2} 
      \bigg(1 + O\bigg(\frac{1}{\log\log y}\bigg)\bigg).
\]
\end{proposition}

\begin{proof}
%
It suffices to establish the estimate for integral $y$, so we 
assume for convenience that $y$ is an integer with $y > \e^3$.
%
By Lemma \ref{lem:sssak=2aux1}, we have 
\begin{equation}
 \label{eq:sssak=2pf1}
 \sum_{0 \le h_1 < h_2 \le y}
  \mathfrak{S}_{\{h_1,h_2\}}
   =
    \bigg(1 + O\bigg(\frac{1}{\log\log y}\bigg)\bigg)
     \sum_{0 \le h_1 < h_2 \le y}
      {\textstyle \prod_{p \in \mathscr{P}} }
       \delta_{\bz}(p)^{-2}\delta_{\{h_1,h_2\}}(p).
\end{equation}
%
For any prime $p$ and integers $h_1$ and $h_2$, 
$\delta_{\{h_1,h_2\}}(p)$ depends only on $\nu_p(h_2 - h_1)$, 
i.e.\ $\delta_{\{h_1,h_2\}}(p) = \delta_{\{0,h_2 - h_1\}}(p)$.
%
Using this, our assumption that $y$ is an integer, and partial 
summation, we verify that  
\[
  \sum_{0 \le h_1 < h_2 \le y}
   {\textstyle \prod_{p \in \mathscr{P}} }
     \delta_{\bz}(p)^{-2}\delta_{\{h_1,h_2\}}(p)
  =
   \int_1^y 
    \Big(\sum_{g \le t} 
    {\textstyle \prod_{p \in \mathscr{P}} }
      \delta_{\bz}(p)^{-2}\delta_{\{0,g\}}(p)\Big) \dd t.
\]
%
(In fact this holds for any integer $y \ge 1$ and any set of 
primes $\mathscr{P}$.)
%
\begin{nixnix}
%
Let $y$ be any number with $y \ge 1$, and let $\mathscr{P}$ be any 
set of primes.
%
In what follows, to ease notation we we set 
\[
 \Pi_{\mathscr{P},h_1,h_2}
  = 
   {\textstyle \prod_{p \in \mathscr{P}} } \delta_{\bz}(p)^{-2}\delta_{\{h_1,h_2\}}(p)
    \quad 
     \text{and}
      \quad 
       \Pi_{\mathscr{P},0,g}
      = 
        {\textstyle \prod_{p \in \mathscr{P}} } \delta_{\bz}(p)^{-2}\delta_{\{0,g\}}(p).
\]
%
We have 
\begin{align*}
 \sum_{0 \le h_1 < h_2 \le y}
  \Pi_{\mathscr{P},h_1,h_2}
 & =
    \sum_{g \le y}
     \sums[0 \le h_1 < h_2 \le y][h_2 - h_1 = g] 
      \Pi_{\mathscr{P},h_1,h_2}
 \\
 & = 
        \sum_{g \le y} 
         \Pi_{\mathscr{P},0,g}
          \sums[0 \le h_1 < h_2 \le y][h_2 - h_1 = g] 1
 \\
 & =  
            \sum_{g \le y} 
             \Pi_{\mathscr{P},0,g}
              \big(y - g + O(1)\big)
 \\
 & = 
                y\sum_{g \le y}
                  \Pi_{\mathscr{P},0,g}
                 -
                   \sum_{g \le y} 
                   g\Pi_{\mathscr{P},0,g}
                  +
                     O(y{\textstyle \prod_{p \in \mathscr{P}} } \delta_{\bz}(p)^{-2})
 \\
 & = 
  \int_1^y 
   \Big(\sum_{g \le t} \Pi_{\mathscr{P},0,g}\Big) \dd t
    + 
     O(y{\textstyle \prod_{p \in \mathscr{P}} } \delta_{\bz}(p)^{-2}),
\end{align*}
where the last equality is obtained by partial summation.
%
Note that if $y$ is an integer, then in the third equality we can 
replace $y - g + O(1)$ by $y - g$, eliminating the two subsequent 
$O$-terms.
\end{nixnix}

Applying Proposition \ref{prop:ssak=2pre} to the integrand, we 
see that 
\begin{equation}
 \label{eq:sssak=2pf2}
   \sum_{0 \le h_1 < h_2 \le y}
   {\textstyle \prod_{p \in \mathscr{P} }} \delta_{\bz}(p)^{-2}\delta_{\{h_1,h_2\}}(p)
 =
    \frac{y^2}{2} 
     +
      O\Big(
            2^{\ecard \mathscr{P}}
            {\textstyle \prod_{p \in \mathscr{P}}  \big(1 + \frac{1}{p}\big) }
              \int_1^y \Psi(t,\mathscr{P}) \dd t 
       \Big).
\end{equation}
%
\begin{nixnix}
%
{\small 
\begin{align*}
   \sum_{0 \le h_1 < h_2 \le y}
   {\textstyle \prod_{p \in \mathscr{P} }} \delta_{\bz}(p)^{-2}\delta_{\{h_1,h_2\}}(p)
 & =
     \int_1^y 
      \Big(\sum_{g \le t} {\textstyle \prod_{p \in \mathscr{P} }} \delta_{\bz}(p)^{-2}\delta_{\{0,g\}}(p)\Big) \dd t
 \\
 & =
    \int_1^y t \dd t  
     +
      O\Big(
          2^{\ecard \mathscr{P}}
         {\textstyle \prod_{p \in \mathscr{P}}  \big(1 + \frac{1}{p}\big) }
          \int_1^t \Psi(t,\mathscr{P}) \dd t 
       \Big)
 \\
 & =
    \frac{y^2}{2} 
     +
      O\Big(
          2^{\ecard \mathscr{P}}
         {\textstyle \prod_{p \in \mathscr{P}}  \big(1 + \frac{1}{p}\big) }
          \int_1^y \Psi(t,\mathscr{P}) \dd t 
       \Big)
\end{align*} 
%
}
\end{nixnix}
%
As $2^{\ecard \mathscr{P}}$ is far greater than 
$\prod_{p \in \mathscr{P}}\big(1 + \frac{1}{p}\big)$, the bound 
$
 \prod_{p \in \mathscr{P}}\big(1 + \frac{1}{p}\big)
  \le
   \log y
$
will suffice for this product.
%
By the prime number theorem for arithmetic progressions, we have 
\[
 \card \mathscr{P}
  =
   \frac{\log y}{2\log\log y}
    +
     O\bigg(\frac{\log y}{(\log\log y)^2}\bigg),
\]
so it follows that 
$
  \textstyle 
 2^{\ecard \mathscr{P}}
     \prod_{p \in \mathscr{P}}  \big(1 + \frac{1}{p}\big)  
  \ll
   y^{2/(5\log\log y)}  
$.
%
\begin{nixnix}
\[
  \textstyle 
 2^{\ecard \mathscr{P}}
     \prod_{p \in \mathscr{P}}  \big(1 + \frac{1}{p}\big)  
  \ll
   \exp\Big\{\frac{\log y}{\log\log y}\Big(\frac{\log 2}{2} + O\Big(\frac{1}{\log\log y}\Big)\Big)\Big\}.
\]
\end{nixnix}
%
Since $\Psi(t,\mathscr{P}) \le \Psi(t,\log y)$, and since the 
bound $\Psi(x,z) \ll x^{1 - 1/(2\log z)}$ holds uniformly for 
$x \ge z \ge 2$ (see \cite[III.5, Theorem 1]{TEN:95}), we have  
\[
 \int_{1}^y \Psi(t,\mathscr{P}) \dd t
  \ll
   \int_1^{y} t^{1 - 1/(2\log\log y)} \dd t
    \ll
     y^{2 - 1/(2\log\log y)}.
\]
%
\begin{nixnix}
For a constant $c > 0$ we have
\[
 \int t^{1 - \frac{1}{2c}} \dd t
  =
   \frac{2ct^{2 - \frac{1}{2c}}}{4c - 1} + \text{constant}.
\]
\end{nixnix}
%
Combining the last two bounds, we see that  
\begin{equation}
 \label{eq:sssak=2pf3}
 2^{\ecard \mathscr{P}}
    {\textstyle \prod_{p \in \mathscr{P}}  \big(1 + \frac{1}{p}\big) }
      \int_1^y \Psi(t,\mathscr{P}) \dd t
       \ll
        y^{2 - 1/(10\log\log y)}.
\end{equation}
%
\begin{nixnix}
Indeed, 
\[
 2^{\ecard \mathscr{P}}
    {\textstyle \prod_{p \in \mathscr{P}}  \big(1 + \frac{1}{p}\big) }
     \int_1^y \Psi(t,\mathscr{P}) \dd t
  \ll
   y^2
   \textstyle 
    \exp\Big\{-\frac{\log y}{\log\log y}\Big(\frac{1 - \log 2}{2} + O\Big(\frac{1}{\log\log y}\Big)\Big)\Big\},
\]
and $(1 - \log 2)/2 = 0.1534\ldots$.
\end{nixnix}

Combining \eqref{eq:sssak=2pf1}, \eqref{eq:sssak=2pf2} and 
\eqref{eq:sssak=2pf3} gives the result.
\end{proof}

%%%%%%%%%%%%%%%%%%%%%%%%%%%%%%%%%%%%%%%%%%%%%%%%%%%%%%%%%%%%%%%%%%
%%%%%%%%%%%%%%%%%%%%%%%%%% SECTION J5 %%%%%%%%%%%%%%%%%%%%%%%%%%%%
%%%%%%%%%%%%%%%%%%%%%%%%%%%%%%%%%%%%%%%%%%%%%%%%%%%%%%%%%%%%%%%%%%

\section{Proposition \ref{prop:sssa} with a weaker error term, and a proof}
 \label{Jsec:keyprop}

\begin{proposition}
 \label{Jprop:sssa}
%
Fix an integer $k \ge 1$ and a bounded convex set 
$\sC \subseteq \Delta^k$.
%
Set $\bo \defeq \emptyset$ or set $\bo \defeq \{0\}$.
%
For $y \ge 1$ we have 
\begin{align}
 \label{Jeq:sssa}
  \sum_{(h_1,\ldots,h_k) \in y\sC \cap \, \ZZ^k} 
   \mathfrak{S}_{\bo \cup \bh} 
   & =
    y^k \Big( \vol(\sC) + O_{k,\sC}\big(\e^{-\sqrt{\log y}}\big)\Big), 
\end{align}
where $\bh = \{h_1,\ldots,h_k\}$ in the summand and $\vol$ stands 
for volume in $\RR^k$.
\end{proposition}

The proof of Proposition \ref{Jprop:sssa} involves a 
basic lattice point counting argument.
%
Lemma \ref{Jlem:lip} is a special case of 
\cite[pp.\ 128--129]{LAN:94}.

\begin{lemma}
 \label{Jlem:lip}
%
Fix an integer $k \ge 1$ and a bounded convex set 
$\sC \subseteq \RR^k$.
%
For $y \ge 1$ we have 
$
 \#(y\sC \cap \ZZ^k)
  =
   y^k\vol(\sC) + O_{k,\sC}(y^{k - 1}).
$
\end{lemma}

To prove Proposition \ref{Jprop:sssa}, we 
express $\mathfrak{S}_{\bh}$ as a series.
%
To this end, for a nonempty, finite set $\bh \subseteq \ZZ$ with 
$\card \bh = k$, let 
\[
 \epsilon_{\bh}(p) 
  \defeq \delta_{\bz}(p)^{-k}\delta_{\bh}(p) - 1
  \quad 
   \text{and}
    \quad 
     \epsilon_{\bh}(d)
      \defeq 
       \textstyle
        \prod_{p \mid d} \epsilon_{\bh}(p)
\]
for integers $d$ in the set
\[ 
  \cD 
   \defeq 
    \{\text{$n \in \cD$ : $n$ squarefree and $p \mid n \implies p \not\equiv 1 \bmod 4$}\}.
\]
%
Note that $1 \in \cD$ and $\epsilon_{\bh}(1) \defeq 1$ by 
convention.
%
For $p \equiv 3 \bmod 4$ we have,
since
$\delta_{\bz}(p) = (1 + 1/p)^{-1}$ and $0 \le \delta_{\bh}(p) \le 1$,  
that $-1 \le \epsilon_{\bh}(p) \le (1 + 1/p)^k - 1 < 2^k/p$.
%
Since $p \le k - 1$ implies $p \mid \det(\bh)$, i.e.\ 
$1/p = (\det(\bh),p)/p^2$, we have 
\begin{equation}
 \label{Jeq:epshpbnd}
  |\epsilon_{\bh}(p)|
   \le 
    A_k\frac{(\det(\bh),p)}{p^2}
\end{equation}
for such $p$, where $A_k$ denotes, here and throughout Section 
\ref{Jsec:keyprop}, a sufficiently large (not necessarily optimal) 
number depending on $k$ --- possibly a different number each time.
%
Inequality \eqref{Jeq:epshpbnd} also holds for $p \ge k$, for in 
that case Proposition \ref{prop:Sp3h} (c) gives
$
 \delta_{\bh}(p) 
  \ge 
   \big(1 + \frac{1}{p}\big)^{-1}
    \big(1 - \frac{k - 1}{p}\big)
     > 
      0
$, 
and hence 
\[
 \epsilon_{\bh}(p)
  \ge 
   \bigg(1 + \frac{1}{p}\bigg)^{k-1}
    \bigg(1 - \frac{k - 1}{p}\bigg)
     -
      1
       \ge 
       -\frac{(k - 1)^2}{p^2}.
\]
%
Note that \eqref{Jeq:epshpbnd} trivially holds for $p = 2$.
%
Thus, 
\begin{equation}
 \label{Jeq:epsbndd}
  |\epsilon_{\bh}(d)|
   \le 
    A_k^{\omega(d)}\frac{(\det(\bh),d)}{d^2}
\end{equation}
for all $d \in \cD$.
%
Finally, notice (recall Definition \ref{def:Sss}) that 
\begin{equation}
 \label{Jeq:defsssS}
 \mathfrak{S}_{\bh}
  =
     \prod_{p \not\equiv 1 \bmod 4}
      \big(1 + \epsilon_{\bh}(p)\big)
  =
      \sum_{d \in \cD}
       \epsilon_{\bh}(d).
\end{equation}
%
The sum converges absolutely in view of \eqref{Jeq:epsbndd} and the 
following elementary bound, which we will also use in the proof 
of Proposition \ref{Jprop:sssa}.
%
Recall that for $n \in \NN$, 
$\omega(n) \defeq \#\{\text{$p$ prime : $p \mid n$}\}$ and  
$\rad(n) \defeq \prod_{p \mid n} p$. 

\begin{lemma} 
 \label{Jlem:omegabnd}
%
Fix any number $A$ satisfying $A \ge 1$.
%
For $x \ge 1$ we have, uniformly for nonzero integers $D$, the 
bound 
\[
 \sumss[\flat][n > x]
  A^{\omega(n)}
   \frac{(D,n)}{n^2}
    \ll_A
     \frac{(\log 3x)^{A - 1}}{x}
      \sumss[\flat][d \mid D] A^{\omega(D)},
\]
where $\sumsstxt[\flat]$ denotes summation restricted to 
squarefree integers.
\end{lemma}

\begin{proof}
%
Let $x \ge 1$.
%
We first consider the case $D = 1$.
%
Note that   
\begin{equation}
 \label{Jeq:mert}
 \sumss[\flat][n_1 \le x] 
  \frac{(A - 1)^{\omega(n_1)}}{n_1}
   \le 
    \prod_{p \le x}
     \bigg(1 + \frac{A - 1}{p}\bigg)
      \le 
       \prod_{p \le x} 
        \bigg(1 + \frac{1}{p}\bigg)^{A - 1}
         \ll_A
          (\log 3x)^{A - 1},
\end{equation}
because 
$1 + 1/p < \e^{1/p}$ and 
$\sum_{p \le x} 1/p = \log\log 3x + O(1)$ by one of Mertens' 
theorems.
%
Now, 
\[
 \sumss[\flat][n > x] \frac{A^{\omega(n)}}{n^2}
  =
   \sumss[\flat][n > x] 
    \frac{1}{n^2}
     \sum_{n_1 \mid n} (A - 1)^{\omega(n_1)}
      \le 
       \sumss[\flat][n_1 \ge 1]
        \frac{(A - 1)^{\omega(n_1)}}{n_1^2}
         \sumss[\flat][m > x/n_1]
          \frac{1}{m^2},
\]
the inner sum being $O(n_1/x)$ for $n_1 \le x$ and $O(1)$ for 
$n_1 > x$. 
%
Thus, 
\[
 \sumss[\flat][n > x] \frac{A^{\omega(n)}}{n^2}
  \ll_A
   \frac{(\log 3x)^{A - 1}}{x}
    +
     \sumss[\flat][n_1 > x]
      \frac{(A - 1)^{\omega(n_1)}}{n_1^2}.
\]
%
If $A - 1 \le 1$ then this last sum is $O(1/x)$; otherwise, 
repeating the argument as many times as necessary gives
\[
 \sumss[\flat][n > x] \frac{A^{\omega(n)}}{n^2}
  \ll_A
   \frac{(\log 3x)^{A - 1}}{x}.
\]
%
It is straightforward to deduce from this that for any nonzero 
integer $d$, 
\[
 \sumss[\flat][n > x][d \mid n] \frac{A^{\omega(n)}}{n^2}
  \ll_A
   \frac{A^{\omega(d)}}{d}\cdot 
    \frac{(\log 3x)^{A - 1}}{x}.
\]
%
Letting $D$ be any nonzero integer, we trivially have 
$(D,n) \le \sum_{d \mid D, \, d \mid n} d$, hence
%
\[
 \sumss[\flat][n > x]
  A^{\omega(n)} \frac{(D,n)}{n^2}
   \le 
    \sumss[\flat][d \mid D] d 
     \sumss[\flat][n > x][d \mid n] \frac{A^{\omega(n)}}{n^2}
      \ll_A 
       \frac{(\log 3x)^{A - 1}}{x}
        \sumss[\flat][d \mid D] A^{\omega(D)}.
\]
\end{proof}

\begin{lemma}
 \label{Jlem:dethap}
%
Fix an integer $k \ge 1$, and a squarefree integer $d \ge 1$.
%
For $y \ge 1$ we have
\[
 \underset{d \mid \det(\{0,h_1,\ldots,h_k\})}
  { 
   \sum_{0 < h_1 < \cdots < h_k \le y}
  } 1
  \le 
   k^{2\omega(d)}
    \bigg(\frac{y^k}{d} + O_k(y^{k - 1})\bigg).
\]
\end{lemma}

\begin{proof}
%
Let $h_0 = 0,h_1,\ldots,h_k$ be pairwise distinct integers and  
suppose $d$ divides $\prod_{0 \le i < j \le k}(h_i - h_j)$.
%
Then, since $d$ is squarefree, there exist pairwise coprime 
positive integers $d_{ij}$ such that 
$d = \prod_{0 \le i < j \le k} d_{ij}$ and 
$d_{ij} \mid h_i - h_j$, $0 \le i < j \le k$.
%
Therefore, 
\[
 \underset{d \mid \det(\{h_0,h_1,\ldots,h_k\})}
  { 
   \sum_{0 < h_1 < \cdots < h_k \le y}
  } 1
   \le 
    \sums[d = d_{01}\cdots d_{(k-1)k}] 
     \hspace{5pt}
     \underset{0 \le i < j \le k - 1 \implies d_{ij} \mid h_i - h_j}
    { 
       \sum_{h_1 \in I_y}
        \sum_{h_2 \in I_y}
         \cdots 
          \sum_{h_{k - 1} \in I_y}
    } 
     \hspace{5pt}
      \sums[h_k \in I_y][0 \le i \le k - 1 \implies d_{ik} \mid h_i - h_k] 1,
\]
where on the right-hand side, the outermost sum is over all 
decompositions of $d$ as a product of $\binom{k + 1}{2}$ positive 
integers, and $I_y \defeq (0,y]$.

Consider the decomposition $d = d_{01}\cdots d_{(k - 1)k}$.
%
Let us define $d_{j} \defeq \prod_{i = 0}^{j - 1} d_{ij}$ for 
$j = 1,\ldots,k$.
%
Notice that $d = \prod_{j = 1}^k d_j$.
%
By the Chinese remainder theorem, the condition on $h_k$ in the 
innermost sum above is equivalent to $h_k$ being in some 
congruence class modulo $d_k$, uniquely determined by 
$h_0,h_1,\ldots,h_{k - 1}$.
%
The sum is therefore equal to $y/d_k + O(1)$.
%
Iterating this argument $k$ times we see that the inner sum over 
$h_1,\ldots,h_k$ is equal to  
\[
  \prod_{j = 1}^k
   \bigg(\frac{y}{d_j} + O(1)\bigg)
    =
     \frac{y^k}{d} + O_k(y^{k - 1}).
\]
%
The result follows by combining and noting that, since $d$ is 
squarefree, the number of ways of writing $d$ as a product of 
$\binom{k + 1}{2}$ positive integers is 
$\binom{k + 1}{2}^{\omega(d)}$, and that 
$\binom{k + 1}{2} \le k^2$.
\end{proof}

Before the final lemma, let us introduce one more piece of 
notation.
%
For $\alpha \ge 1$, $\T_{\bz}(2^{\alpha + 1})$ and 
$\V_{\bz}(p^{\alpha})$ are nonempty.
%
For integers $j \ge 1$, let  
\[
 \upsilon_{\bh}(2^{\alpha + 1};j) 
  \defeq 
   \bigg(\frac{\card \T_{\bz}(2^{\alpha + 1})}{2^{\alpha + 1}}\bigg)^{-j}
    \bigg(\frac{\card \T_{\bh}(2^{\alpha + 1})}{2^{\alpha + 1}}\bigg) - 1, 
\]
and for $p \equiv 3 \bmod 4$, define
\[
 \upsilon_{\bh}(p^{\alpha};j) 
  \defeq 
   \bigg(\frac{\card \V_{\bz}(p^{\alpha})}{p^{\alpha}}\bigg)^{-j}
    \bigg(\frac{\card \V_{\bh}(p^{\alpha})}{p^{\alpha}}\bigg) - 1.  
\]
%
Note that in the exponent we have the parameter $j$, not $k$.
%
As one might expect, $\upsilon_{\bh}(p^{\alpha};k)$ is a good 
approximation to $\epsilon_{\bh}(p)$ when $\alpha$ is large, and 
we can take advantage of quite strong cancellation in summing 
$\upsilon_{\bh}(p^{\alpha};k)$ rather than $\epsilon_{\bh}(p)$.

\begin{lemma}
 \label{Jlem:cancel}
%
\textup{(}a\textup{)}
%
Set $\bo \defeq \emptyset$ or set $\bo \defeq \{0\}$.
%
Let $(\alpha_p)_{p \not\equiv 1 \bmod 4}$ be a sequence of 
positive integers with $\alpha_2 \ge 2$.
%
Let $d > 1$ be an integer, none of whose prime divisors 
are congruent to $1$ modulo $4$, and let $R_1,\ldots,R_k$ be 
complete residue systems modulo $\prod_{p \mid d} p^{\alpha_p}$.

\textup{(}a\textup{)}
We have 
\[
 \sum_{h_1 \in R_1} 
  \cdots 
   \sum_{h_k \in R_k}
    \prod_{p \mid d} \upsilon_{\bo \cup \bh}(p^{\alpha_p};\ocard \bo + k)
      =
       0,
\]
where $\bh = \{h_1,\ldots,h_k\}$ in the summand. 
%
\textup{(}Note that we may have $\card \bh < k$ here.\textup{)}

\textup{(}b\textup{)}
%
Let $\bh = \{h_1,\ldots,h_k\}$ be a set of integers that are not 
necessarily distinct.
%
For $p \equiv 3 \bmod 4$ and $\alpha \ge 1$, we have
\begin{equation}
 \label{Jeq:upsbnd1}
 |\upsilon_{\bh}(p^{\alpha};k)|
  \le 
   A_k
    \bigg(
     \frac{k - \card \bh}{p}
      + \frac{(\det(\bh),p)}{p^2}
    \bigg) 
\end{equation}
and 
\begin{equation}
 \label{Jeq:upsbnd2}
 |\epsilon_{\bh}(p) - \upsilon_{\bh}(p^{\alpha};k)|
  \le 
   A_k\bigg(\frac{k - \card \bh}{p} + \frac{1}{p^{\alpha + \alpha \bmod 2}}\bigg),
\end{equation}
where $A_k$ is a \textup{(}sufficiently large\textup{)} quantity 
depending on $k$.
%
The bounds \eqref{Jeq:upsbnd1} and \eqref{Jeq:upsbnd2} also hold for 
$p = 2$, provided $\alpha \ge 2$.
\end{lemma}

\begin{proof}
%
(a)
%
Let $\bh = \{h_1,\ldots,h_k\}$ and $\bh' = \{h_1',\ldots,h_k'\}$  
satisfy $h_i \equiv h_i' \bmod p^{\alpha}$ for $i = 1,\ldots,k$. 
%
If $p \equiv 3 \bmod 4$, it is clear from \eqref{eq:defVh} that 
$
 \card \V_{\bo \cup \bh}(p^{\alpha}) 
  = 
   \card \V_{\bo \cup \bh'}(p^{\alpha})
$, 
and hence 
$
 \upsilon_{\bo \cup \bh}(p^{\alpha};\ocard \bo + k)
 = 
  \upsilon_{\bo \cup \bh'}(p^{\alpha};\ocard \bo + k)
$.
%
Similarly, we have (from \eqref{eq:defTh}) that 
$
 \upsilon_{\bo \cup \bh}(2^{\alpha + 1};\ocard \bo + k) 
  = 
   \upsilon_{\bo \cup \bh'}(2^{\alpha + 1};\ocard \bo + k)
$.
%
Thus, by the Chinese remainder theorem, 
\[
 \sum_{h_1 \in R_1} 
  \cdots 
   \sum_{h_k \in R_k}
    \prod_{p \mid d} \upsilon_{\bo \cup \bh}(p^{\alpha_p};\ocard \bo + k)
      =
       \prod_{p \mid d} 
        \bigg(
         \sum_{h_1 \in \ZZ_{p^{\alpha_p}}}
          \cdots 
           \sum_{h_k \in \ZZ_{p^{\alpha_p}}}
            \upsilon_{\bo \cup \bh}(p^{\alpha_p};\ocard \bo + k) 
         \bigg),
\]
where $\bh = \{h_1,\ldots,h_k\}$ in both summands, and 
$\ZZ_{p^{\alpha_p}} \defeq \{0,\ldots,p^{\alpha_p} - 1\}$.
%
For $p \equiv 3 \bmod 4$ we have 
\[
 \sum_{h_1 \in \ZZ_{p^{\alpha_p}}}
  \cdots 
   \sum_{h_k \in \ZZ_{p^{\alpha_p}}}
    \card \V_{\bh}(p^{\alpha_p})
   =
   \sum_{a \in \ZZ_{p^{\alpha_p}}}
    \sums[h_1 \in \ZZ_{p^{\alpha_p}}][a + h_1 \in S_p][\nu_p(a + h_1) < \alpha_p]
     \cdots 
      \sums[h_k \in \ZZ_{p^{\alpha_p}}][a + h_k \in S_p][\nu_p(a + h_k) < \alpha_p] 1,
\]
as can be seen by applying the definition \eqref{eq:defVh} of 
$\V_{\bh}(p^{\alpha})$ and changing the order of summation.
%
For $i = 1,\ldots,k$, each sum over $h_i$ on the right-hand side 
enumerates a translation of $\V_{\bz}(p^{\alpha_p})$, so the 
entire sum (i.e.\ the left-hand side) is equal to 
$p^{\alpha_p}(\card \V_{\bz}(p^{\alpha_p}))^k$.
%
%****************************************************************%
%************************* START DETAIL *************************%
%****************************************************************%
%
\begin{nixnix}
%
\begin{align*}
  \sum_{h_1 \in \ZZ_{p^{\alpha}}}
   \cdots 
    \sum_{h_k \in \ZZ_{p^{\alpha}}}
     \card \V_{\bh}(p^{\alpha})
  & = 
      \sum_{h_1 \in \ZZ_{p^{\alpha}}}
       \cdots 
        \sum_{h_k \in \ZZ_{p^{\alpha}}}
         \sums[a \in \ZZ_{p^{\alpha}}][\forall i, a + h_i \in S_p][\forall i, \nu_p(a + h_i) < \alpha] 1
 \\ 
  & = 
   \sum_{a \in \ZZ_{p^{\alpha}}}
    \sums[h_1 \in \ZZ_{p^{\alpha}}][a + h_1 \in S_p][\nu_p(a + h_1) < \alpha]
     \cdots 
      \sums[h_k \in \ZZ_{p^{\alpha}}][a + h_k \in S_p][\nu_p(a + h_k) < \alpha] 1
 \\
 & = 
   \sum_{a \in \ZZ_{p^{\alpha}}}
    \sums[a + h_1 \in \ZZ_{p^{\alpha}}][a + h_1 \in S_p][\nu_p(a + h_1) < \alpha]
     \cdots 
      \sums[a + h_k \in \ZZ_{p^{\alpha}}][a + h_k \in S_p][\nu_p(a + h_k) < \alpha] 1   
 \\
 & = 
    \sum_{a \in \ZZ_{p^{\alpha}}}
     (\card \V_{\bz}(p^{\alpha}))
      \cdots 
       (\card \V_{\bz}(p^{\alpha}))
 \\
 & = 
    p^{\alpha}(\card \V_{\bz}(p^{\alpha}))^k
\end{align*}
%
\end{nixnix}
%
%****************************************************************%
%************************** END DETAIL **************************%
%****************************************************************%
%
Since 
\[
 \upsilon_{\bh}(p^{\alpha_p};k) 
  \defeq 
   \big[
    \big(\card \V_{\bz}(p^{\alpha_p})/p^{\alpha_p}\big)^{-k}
     \big(\card \V_{\bh}(p^{\alpha_p})/p^{\alpha_p}\big) 
   \big] - 1,
\]
it follows that  
\[
 \sum_{h_1 \in \ZZ_{p^{\alpha_p}}}
  \cdots 
   \sum_{h_k \in \ZZ_{p^{\alpha_p}}}
    \upsilon_{\bh}(p^{\alpha_p};k)
   =
   0.
\]
%
In a similar fashion we obtain
\[
 \sum_{h_1 \in \ZZ_{p^{\alpha_p}}}
  \cdots 
   \sum_{h_k \in \ZZ_{p^{\alpha_p}}}
    \upsilon_{\{0\} \cup \bh}(p^{\alpha_p};1 + k)
   =
   0. 
\]
%
%****************************************************************%
%************************* START DETAIL *************************%
%****************************************************************%
%
\begin{nixnix}
%
Similarly, 
\[
 \sum_{h_1 \in \ZZ_{p^{\alpha_p}}}
  \cdots 
   \sum_{h_k \in \ZZ_{p^{\alpha_p}}}
    \card \V_{\{0\} \cup \bh}(p^{\alpha_p})
   =
   \sums[a \in \ZZ_{p^{\alpha_p}}][a \in S_p][\nu_p(a) < \alpha_p]
    \sums[h_1 \in \ZZ_{p^{\alpha_p}}][a + h_1 \in S_p][\nu_p(a + h_1) < \alpha_p]
     \cdots 
      \sums[h_k \in \ZZ_{p^{\alpha_p}}][a + h_k \in S_p][\nu_p(a + h_k) < \alpha_p] 1,
\]
which is equal to $\big(\card \V_{\bz}(p^{\alpha_p})\big)^{1 + k}$, 
and since 
\[
  \upsilon_{\{0\} \cup \bh}(p^{\alpha_p};1 + k) 
  \defeq 
   \big[
    \big(\card \V_{\bz}(p^{\alpha_p})/p^{\alpha_p}\big)^{-(1 + k)}
     \big(\card \V_{\{0\} \cup \bh}(p^{\alpha_p})/p^{\alpha_p}\big) 
   \big] - 1,
\]
it follows that 
\[
 \sum_{h_1 \in \ZZ_{p^{\alpha_p}}}
  \cdots 
   \sum_{h_k \in \ZZ_{p^{\alpha_p}}}
    \upsilon_{\{0\} \cup \bh}(p^{\alpha_p};1 + k)
   =
   0 
\]
as well.
%
\end{nixnix}
%
%****************************************************************%
%************************** END DETAIL **************************%
%****************************************************************%
%
An analogous argument gives the same results for $p = 2$ 
($\alpha_2 \ge 2$).
%
%****************************************************************%
%************************* START DETAIL *************************%
%****************************************************************%
%
\begin{nixnix}
%
Applying the definition \eqref{eq:defTh} of 
$\T_{\bh}(2^{\alpha + 1})$ and changing the order of summation 
yields 
\begin{align*}
  \sum_{h_1 \in \ZZ_{2^{\alpha + 1}}}
   \cdots 
    \sum_{h_k \in \ZZ_{2^{\alpha + 1}}}
     \card \T_{\bh}(2^{\alpha + 1})
  & = 
      \sum_{h_1 \in \ZZ_{2^{\alpha + 1}}}
       \cdots 
        \sum_{h_k \in \ZZ_{2^{\alpha + 1}}}
         \sums[a \in \ZZ_{2^{\alpha + 1}}][\forall i, a + h_i \in S_2][\forall i, \nu_2(a + h_i) < \alpha] 1
 \\ 
  & = 
   \sum_{a \in \ZZ_{2^{\alpha + 1}}}
    \sums[h_1 \in \ZZ_{2^{\alpha + 1}}][a + h_1 \in S_2][\nu_2(a + h_1) < \alpha]
     \cdots 
      \sums[h_k \in \ZZ_{2^{\alpha + 1}}][a + h_k \in S_2][\nu_2(a + h_k) < \alpha] 1
 \\
 & = 
   \sum_{a \in \ZZ_{2^{\alpha + 1}}}
    \sums[a + h_1 \in \ZZ_{2^{\alpha + 1}}][a + h_1 \in S_2][\nu_2(a + h_1) < \alpha]
     \cdots 
      \sums[a + h_k \in \ZZ_{2^{\alpha + 1}}][a + h_k \in S_2][\nu_2(a + h_k) < \alpha] 1   
 \\
 & = 
    \sum_{a \in \ZZ_{2^{\alpha + 1}}}
     (\card \T_{\bz}(2^{\alpha + 1}))
      \cdots 
       (\card \T_{\bz}(2^{\alpha + 1}))
 \\
 & = 
    2^{\alpha + 1}(\card \T_{\bz}(2^{\alpha + 1}))^k.
\end{align*}
%
Since 
$
 \upsilon_{\bh}(2^{\alpha + 1}) 
  \defeq 
   \big[
    \big(\card \T_{\bz}(2^{\alpha + 1})/2^{\alpha + 1}\big)^{-k}
     \big(\card \T_{\bh}(2^{\alpha + 1})/2^{\alpha + 1}\big) 
   \big] - 1
$, 
it follows that 
\[
 \sum_{h_1 \in \ZZ_{2^{\alpha + 1}}}
  \cdots 
   \sum_{h_k \in \ZZ_{2^{\alpha + 1}}}
    \upsilon_{\bh}(2^{\alpha + 1})
   =
   0.
\]
%
\end{nixnix}
%
%****************************************************************%
%************************** END DETAIL **************************%
%****************************************************************%

(b)
%
Let $\alpha \ge 1$ and $p \equiv 3 \bmod 4$.
%
Define $\eta_{\bh}(p^{\alpha})$ and $\kappa_{\bh}(p)$ as the 
numbers given by the relations
\[
 \frac{\card \V_{\bh}(p^{\alpha})}{p^{\alpha}}
  \eqdef 
   \delta_{\bh}(p) + \eta_{\bh}(p^{\alpha}) 
    \quad 
     \text{and}
      \quad 
 \delta_{\bh}(p) 
  \eqdef
   \bigg(1 + \frac{1}{p}\bigg)^{-1}
    \bigg(1 - \frac{\kappa_{\bh}(p)}{p}\bigg).
\]
%
Note that by Proposition \ref{prop:Sp3h}, \eqref{eq:Vhdeltp3bnd} 
and part (c),  
$
 |\eta_{\bh}(p^{\alpha})| 
  < 
   2(\card \bh)/p^{\alpha + (\alpha \bmod 2)}
$ and 
$\kappa_{\bh}(p) \le \min\{\card \bh - 1,p\}$, with 
$\kappa_{\bh}(p) = \card \bh - 1$ if $p \nmid \det(\bh)$.
%
Also, $\kappa_{\bh}(p) \ge -1$ (because $\delta_{\bh}(p) \le 1$).
%
Since $\card \bh \le k$ and $\alpha + (\alpha \bmod 2) \ge 2$, we 
have 
\[
 \frac{\card \V_{\bh}(p^{\alpha})}{p^{\alpha}}
  =
   \bigg(1 + \frac{1}{p}\bigg)^{-1}
    \bigg(1 - \frac{\kappa_{\bh}(p)}{p} + O\bigg(\frac{k}{p^2}\bigg)\bigg).
\]
%
In the special case $\bh = \{0\}$ we can take 
$\kappa_{\bh}(p) = 0$.
%
We therefore have  
\begin{align*}
 \bigg(\frac{\card \V_{\bz}(p^{\alpha})}{p^{\alpha}}\bigg)^{-k}
  \frac{\card \V_{\bh}(p^{\alpha})}{p^{\alpha}}
  &
   =
    \bigg(1 + \frac{1}{p}\bigg)^{k - 1}
     \bigg[
       \bigg(1 - \frac{\kappa_{\bh}(p)}{p} + O_k\bigg(\frac{1}{p^2}\bigg)\bigg)
     \bigg]
   \\
  &
   =
     \bigg(1 + \frac{k - 1}{p} +O_k\bigg(\frac{1}{p^2}\bigg)\bigg)
      \bigg[1 - \frac{\kappa_{\bh}(p)}{p} + O_k\bigg(\frac{1}{p^2}\bigg)\bigg]
 \\
   & 
    = 
      1 + \frac{k - 1 - \kappa_{\bh}(p)}{p} + O_k\bigg(\frac{1}{p^2}\bigg).
\end{align*}

\noindent 
Thus, writing 
$\kappa_{\bh}(p) \eqdef \card \bh - 1 - \xi_{\bh}(p)$, say, we have 
\[
 |\upsilon_{\bh}(p^{\alpha};k)|
  \le 
   \frac{k - 1 - \kappa_{\bh}(p)}{p} + \frac{A_k}{p^2}
    =
     \frac{k - \card \bh}{p} + \frac{\xi_{\bh}(p)}{p} + \frac{A_k}{p^2}.
\]
%
(Recall that we use $A_k$ to denote a sufficiently large number 
depending on $k$.)
%
If $p \mid \det(\bh)$ then 
$\xi_{\bh}(p)/p = \xi_{\bh}(p)(\det(\bh),p)/p^2$, and if 
$p \nmid \det(\bh)$ then, as already noted, 
$\kappa_{\bh}(p) = \card \bh - 1$, i.e.\ $\xi_{\bh}(p) = 0$, so 
$\xi_{\bh}(p)/p = \xi_{\bh}(p)(\det(\bh),p)/p^2$ in any case.
%
Since, as already noted, 
$-1 \le \kappa_{\bh}(p) \le \card \bh - 1$, 
we have $0 \le \xi_{\bh}(p) \le \card \bh \le k$, so we may write 
$|\xi_{\bh}(p)| \le A_k$.
%
Hence
\[
 |\upsilon_{\bh}(p^{\alpha};k)|
  \le 
   \frac{k - \card \bh}{p} + A_k\frac{(\det(\bh),p)}{p^2} + \frac{A_k}{p^2}
    \le
     \frac{k - \card \bh}{p} + A_k\frac{(\det(\bh),p)}{p^2},
\]
giving \eqref{Jeq:upsbnd1}.

As can be seen from Proposition \ref{prop:Sp3h}, 
\eqref{eq:delthp3} and part (c), we in fact have
\[
 \frac{\card \V_{\bz}(p^{\alpha})}{p^{\alpha}}
  =
   \bigg(1 + \frac{1}{p}\bigg)^{-1}
    \bigg(1 - \frac{1}{p^{\alpha + \alpha \bmod 2}}\bigg).
\]
%
Letting $j = \card \bh$ so that 
$
 \epsilon_{\bh}(p) 
  = \delta_{\bz}(p)^{-j}\delta_{\bh}(p)
  = (1 + 1/p)^{j}\delta_{\bh}(p)
$, 
we see that
{\small 
\begin{align*}
 & 
 \epsilon_{\bh}(p) - \upsilon_{\bh}(p^{\alpha};k)
  \\ 
 & \hspace{15pt}
  =   
    \bigg(1 + \frac{1}{p}\bigg)^{j}
     \bigg(\frac{\card \V_{\bh}(p^{\alpha})}{p^{\alpha}} - \eta_{\bh}(p^{\alpha})\bigg)
      -
       \bigg(1 + \frac{1}{p}\bigg)^{k}
        \bigg(1 - \frac{1}{p^{\alpha + \alpha \bmod 2}}\bigg)^{-k}
         \frac{\card \V_{\bh}(p^{\alpha})}{p^{\alpha}}
  \\
 & \hspace{15pt}
  =   
   \bigg(1 + \frac{1}{p}\bigg)^{j}
    \frac{\card \V_{\bh}(p^{\alpha})}{p^{\alpha}}
     \bigg\{1 - \bigg(1 + \frac{1}{p}\bigg)^{k - j}\bigg(1 - \frac{1}{p^{\alpha + \alpha \bmod 2}}\bigg)^{-k}\bigg\}
      - \bigg(1 + \frac{1}{p}\bigg)^{j}\eta_{\bh}(p^{\alpha}). 
\end{align*}
}

\noindent 
Now, $(1 + 1/p)^{j} \le A_k$, 
$\card \V_{\bh}(p^{\alpha})/p^{\alpha} \le 1$, 
$\eta_{\bh}(p^{\alpha}) \le A_k/p^{\alpha + \alpha \bmod 2}$ (see 
above), and the term $\{\cdots\}$ in brackets is at most 
$(k - j)/p + A_k/p^2$, but of course if $k = j$ 
then it is at most $A_k/p^{\alpha + \alpha \bmod 2}$.
%
Hence \eqref{Jeq:upsbnd2}.

The case for \eqref{Jeq:upsbnd1} and \eqref{Jeq:upsbnd2} with 
$p = 2$ ($\alpha \ge 2)$ is similar.
%
%****************************************************************%
%************************* START DETAIL *************************%
%****************************************************************%
%
\begin{nixnix}
%
To do...
\end{nixnix}
%
%****************************************************************%
%************************** END DETAIL **************************%
%****************************************************************%
%
\end{proof}

\begin{proof}[Proof of Proposition \ref{Jprop:sssa}]
%
Fix an integer $k \ge 1$ and a bounded convex set 
$\sC \subseteq \Delta^k$, where 
$
 \Delta^k \defeq \{(x_1,\ldots,x_k) \in \RR^k : 0 < x_1 < \cdots < x_k\}
$ 
(see \eqref{eq:defsimplex}). 
%
Set $\bo \defeq \emptyset$ or set $\bo \defeq \{0\}$.
%
Let $y \ge 1$.
%
To ease notation throughout, let $\cH \defeq y\sC \cap \ZZ^k$, 
$\vbh = (h_1,\ldots,h_k)$, and $\bh = \{h_1,\ldots,h_k\}$.
%
Note that $0 < h_1 < \cdots < h_k \ll_{\sC} y$ for $\vbh \in \cH$.
%
Now set $z \defeq \e^{2\sqrt{\log y}}$.
%
The estimate \eqref{Jeq:sssa} is trivial if $y \ll_{k,\sC} 1$, so 
we may assume that $\cH \ne \emptyset$, and also that 
$\log\log z \ge 1$.

In view of \eqref{Jeq:defsssS} we see, upon partitioning the sum 
over $d$ and changing order of summation, that  
\begin{equation}
 \label{Jeq:lemsssapf1}
 \sum_{\vbh \in \cH} \mathfrak{S}_{\bo \cup \bh}
  =
   \sum_{\vbh \in \cH} \epsilon_{\bo \cup \bh}(1)
    +
      \sums[d \in \cD][1 < d \le z]
       \sum_{\vbh \in \cH} \epsilon_{\bo \cup \bh}(d)
      +
       \sums[d \in \cD][d > z]
        \sum_{\vbh \in \cH} \epsilon_{\bo \cup \bh}(d).
\end{equation}
%
Since $\epsilon_{\bo \cup \bh}(1) = 1$ we have, by 
Lemma \ref{Jlem:lip}, that 
\begin{equation}
 \label{Jeq:volH}
  \sum_{\vbh \in \cH} \epsilon_{\bo \cup \bh}(1)
   =
    y^k \vol(\sC) + O_{k,\sC}(y^{k - 1}).
\end{equation}
%
We show that the second and third sums make a negligible 
contribution to the right-hand side of \eqref{Jeq:lemsssapf1}.

Recall that $A_k$ conveniently stands for a sufficiently large 
(not necessarily optimal) number depending on $k$ (or more 
precisely, in this proof, on $\ocard \bo + k$), possibly a 
different number in any two occurrences.
%
By \eqref{Jeq:epsbndd} and the trivial bound 
$
 (\det(\bo \cup \bh),d) 
   \le 
    \sum_{c \mid d, \, c \mid \det(\bo \cup \bh)} c
$, 
we have 
\[
 \sums[d \in \cD][d > z]
  \sum_{\vbh \in \cH} |\epsilon_{\bo \cup \bh}(d)|
   \le 
    \sums[d \in \cD][d > z]
     \sum_{\vbh \in \cH} A_k^{\omega(d)}\frac{(\det(\bo \cup \bh),d)}{d^2}
      \le 
       \sums[d \in \cD][d > z]
        \frac{A_k^{\omega(d)}}{d^2}
         \sum_{c \mid d}  
          \sums[\vbh \in \cH][c \mid \det(\bo \cup \bh)] c.
\]
%
If $\vbh \in \cH$ then $0 < h_1 < \cdots < h_k \ll_{\sC} y$, and 
if $c \mid \det(\bo \cup \bh)$ then $c \mid \det(\{0\} \cup \bh)$, 
even in the case where $\bo$ is empty, so Lemma \ref{Jlem:dethap} 
yields, for squarefree $d$, 
\[
 \sum_{c \mid d}  
  \sums[\vbh \in \cH][c \mid \det(\bo \cup \bh)] c
   \le 
    \sum_{c \mid d} c 
     \sums[0 < h_1 < \cdots < h_k \ll_{\sC} y][c \mid \det(\{0,h_1,\ldots,h_k\})] 1
      \ll_{k,\sC}
    A_k^{\omega(d)}
     \bigg(
      y^k 
      +
       y^{k - 1}
        \sums[c \mid d, \, p \mid c \implies p \ll_{\sC} y] c
     \bigg).
\]
%
(If $0 < h_1 < \cdots < h_k \ll_{\sC} y$ and 
$c \mid \det(\{0,h_1,\ldots,h_k\})$ then all prime divisors of $c$ 
are $\ll_{\sC} y$;  
if $c \mid d$ then $\omega(c) \le \omega(d)$; 
if $d$ is squarefree then $\sum_{c \mid d} 1 = 2^{\omega(d)}$.)
%
Recalling that $\sumsstxt[\flat]$ denotes summation restricted to 
squarefree integers, we have 
\[
 \sumss[\flat][d \ge 1] 
  \frac{A_k^{\omega(d)}}{d^2}
   \sums[c \mid d, \, p \mid c \implies p \ll_{\sC} y] c
    \le 
     \sumss[\flat][c \ge 1, \, p \mid c \implies p \ll_{\sC} y] \frac{A_k^{\omega(c)}}{c}
      \sumss[\flat][b \ge 1] \frac{A_k^{\omega(b)}}{b^2}
       \ll_k
        \sumss[\flat][c \ge 1, \, p \mid c \implies p \ll_{\sC} y] \frac{A_k^{\omega(c)}}{c},
\]
as can be seen by writing $d = bc$ and changing order of 
summation, and    
\[
 \sumss[\flat][c \ge 1, \, p \mid c \implies p \ll_{\sC} y] \frac{A_k^{\omega(c)}}{c}
  \le 
   \prod_{p \ll_{\sC} y}
    \bigg(1 + \frac{A_k}{p}\bigg)
%      \le 
%       \prod_{p \ll_{\sC} y}
%        \bigg(1 + \frac{1}{p}\bigg)^{A_k}
        \ll_{k,\sC} (\log y)^{A_k}  
\]
(see \eqref{Jeq:mert}).
%
Combining and applying Lemma \ref{Jlem:omegabnd} (with $D = 1$), we 
see that 
{\small 
\begin{equation}
 \label{Jeq:lemsssapf2}
 \sums[d \in \cD][d > z]
  \sum_{\vbh \in \cH} |\epsilon_{\bo \cup \bh}(d)|
   \ll_{k,\sC}
    y^k\sums[d \in \cD][d > z] \frac{A_k^{\omega(d)}}{d^2}
     + y^{k - 1}(\log y)^{A_k} 
     \ll_{k,\sC}
       y^k\frac{(\log z)^{A_k}}{z} + y^{k - 1}(\log y)^{A_k}.
\end{equation}
}

\noindent 
(Recall that $\cD$ by definition contains only squarefree 
integers.)

For the sum in \eqref{Jeq:lemsssapf1} over $1 < d \le z$, let 
$(\alpha_p)_{p \not\equiv 1 \bmod 4}$ be the sequence of integers 
satisfying 
\[
 1 + \frac{\log z}{\log p}
  < 
   \alpha_p
    \le 
     2 + \frac{\log z}{\log p}
\]
for all $p \not\equiv 1 \bmod 4$.
%
We claim that for $d \in \cD$ with $1 < d \le z$, we have 
\begin{equation}
 \label{Jeq:dbnds}
  \textstyle 
   d^2z^{1/2}
   <
    \prod_{p \mid d} p^{\alpha_p}
     <
      \e^{c_1(\log z)^2/\log\log z} 
\end{equation}
for a suitable absolute constant $c_1 > 0$.
%
To see this, let $\theta_p = (\log z)/\log p$, so that 
$p^{\theta_p} = z$ and $1 + \theta_p < \alpha_p \le 2 + \theta_p$.
%
For $p \le z$, we have $\theta_p \ge 1$.
%
For $1 \le \theta_p < 2$ (i.e.\ $z^{1/2} < p \le z$), 
$\alpha_p = 3$ and $1/\theta_p > 1/2$, hence 
$
 p^{\alpha_p} 
  = 
   p^2p
    =
     p^2z^{1/\theta_p}
      >
       p^2z^{1/2}
$.
%
For $\theta_p \ge 2$ (i.e.\ $p \le z^{1/2}$), $\theta_p/2 \ge 1$ 
and 
$
 \alpha_p - \theta_p/2 
  = 
   \alpha_p - \theta_p + \theta_p/2
    > 1 + \theta_p/2
     \ge 
      2
$,
hence 
$
 p^{\alpha_p}
  =
   p^{\alpha_p - \theta_p/2}p^{\theta_p/2}
     =
      p^{\alpha_p - \theta_p/2}z^{1/2}
       > p^2z^{1/2}
$.
%
Thus, if $d$ is squarefree and $d \le z$, then 
$
 \prod_{p \mid d} p^{\alpha_p}
  > 
   \prod_{p \mid d} p^2z^{1/2}
    = 
     d^2z^{\omega(d)/2}
$.
%
If $d > 1$ then $\omega(d) \ge 1$.
%
For squarefree $d \le z$ we have  
$
 \prod_{p \mid d} p^{\alpha_p}
  \le 
   \prod_{p \mid d} (p^2z)
    = 
     d^2z^{\omega(d)}
      <
       z^{2 + \omega(d)}
$.
%
By the elementary bound $\omega(d) \ll (\log d)/\log\log d$ we 
have 
$
 2 + \omega(d) \le c_1(\log z)/\log\log z
$
for $d \le z$, where $c_1 > 0$ is a suitable absolute constant.
%
Hence 
$
 z^{2 + \omega(d)} 
  \le 
   \e^{c_1(\log z)^2/\log\log z}.
$

Let us set   
$
 \upsilon_{\bo \cup \bh}(p^{\alpha_p}) 
  \defeq \upsilon_{\bo \cup \bh}(p^{\alpha_p}; \ocard \bo + k)
$
to ease notation in what follows.
%
Writing
$
 \epsilon_{\bo \cup \bh}(d)
  =
   \prod_{p \mid d}\big(\upsilon_{\bo \cup \bh}(p^{\alpha_p}) + \epsilon_{\bo \cup \bh}(p) - \upsilon_{\bo \cup \bh}(p^{\alpha_p})\big)
$
and developing the product, the sum in \eqref{Jeq:lemsssapf1} over 
$1 < d \le z$ becomes 
\[
 \sums[d \in \cD][1 < d \le z] 
  \sum_{\vbh \in \cH}
   \bigg\{ 
    \prod_{p \mid d} \upsilon_{\bo \cup \bh}(p^{\alpha_p})
    + 
      \sums[bc = d][c > 1] 
       \prod_{p \mid b} \upsilon_{\bo \cup \bh}(p^{\alpha_p})
        \prod_{p \mid c} \big(\epsilon_{\bo \cup \bh}(p) - \upsilon_{\bo \cup \bh}(p^{\alpha_p})\big)
  \bigg\}.
\]
%
%****************************************************************%
%************************* START DETAIL *************************%
%****************************************************************%
%
\begin{nixnix}
%
We have 
\begin{align*}
 \epsilon_{\bo \cup \bh}(d)
 & =
   \prod_{p \mid d}\big(\upsilon_{\bo \cup \bh}(p^{\alpha_p}) + \epsilon_{\bo \cup \bh}(p) - \upsilon_{\bo \cup \bh}(p^{\alpha_p})\big)
 \\
 & =
     \prod_{p \mid d} \upsilon_{\bo \cup \bh}(p^{\alpha_p})
 \\
 & \hspace{30pt} + 
       \sums[bc = d][c > 1] 
        \prod_{p \mid b} \upsilon_{\bo \cup \bh}(p^{\alpha_p})
         \prod_{p \mid c} \big(\epsilon_{\bo \cup \bh}(p) - \upsilon_{\bo \cup \bh}(p^{\alpha_p})\big), 
\end{align*}
hence 
\begin{align*}
 & 
 \sums[d \in \cD][1 < d \le z] 
  \sum_{\vbh \in \cH}
   \epsilon_{\bo \cup \bh}(d)
 \\
 & \hspace{15pt} 
   =
     \sums[d \in \cD][1 < d \le z] 
      \sum_{\vbh \in \cH}
       \bigg\{ 
        \prod_{p \mid d} \upsilon_{\bo \cup \bh}(p^{\alpha_p})
 \\
 & \hspace{30pt} + 
          \sums[bc = d][c > 1] 
           \prod_{p \mid b} \upsilon_{\bo \cup \bh}(p^{\alpha_p})
            \prod_{p \mid c} \big(\epsilon_{\bo \cup \bh}(p) - \upsilon_{\bo \cup \bh}(p^{\alpha_p})\big)
      \bigg\}.
\end{align*}
%
\end{nixnix}
%
%****************************************************************%
%************************** END DETAIL **************************%
%****************************************************************%
%
By Lemma \ref{Jlem:cancel} (b), \eqref{Jeq:upsbnd1}, for squarefree 
$b$ and $\vbh \in \cH$, we see, upon noting that 
$\ocard \bo + k - \card(\bo \cup \bh) = 0$, that  
$
 \prod_{p \mid b} |\upsilon_{\bo \cup \bh}(p^{\alpha_p})|
%   \le 
%    \prod_{p \mid b} \frac{A_k\det(\bo \cup \bh),p)}{p^2}
    \le 
     A_k^{\omega(b)}\det(\bo \cup \bh,b)/b^2
$.
%
Similarly, applying Lemma \ref{Jlem:cancel} (b), 
\eqref{Jeq:upsbnd2}, we obtain, for 
squarefree $c$ and $\vbh \in \cH$, the bound 
$
 \prod_{p \mid c} |\epsilon_{\bo \cup \bh}(p) - \upsilon_{\bo \cup \bh}(p^{\alpha_p})| 
%   \le 
%    \prod_{p \mid c} A_k/p^{\alpha + \alpha \bmod 2}
    \le 
     A_k^{\omega(c)}\prod_{p \mid c} 1/p^{\alpha_p + \alpha_p \bmod 2}
$.
%
Furthermore, by the lower bound in \eqref{Jeq:dbnds}, we have 
$\prod_{p \mid c} 1/p^{\alpha_p} < 1/(c^2z^{1/2})$.
%
Combining, we see that 
{\small
\begin{align*} 
 & 
 \sums[d \in \cD][1 < d \le z] 
  \sum_{\vbh \in \cH}
   \sums[bc = d][c > 1] 
    \prod_{p \mid b} |\upsilon_{\bo \cup \bh}(p^{\alpha_p})|
     \prod_{p \mid c} |\epsilon_{\bo \cup \bh}(p) - \upsilon_{\bo \cup \bh}(p^{\alpha_p})|
 \\
 & \hspace{5pt}
   \le 
    z^{-1/2}
     \sums[d \in \cD][1 < d \le z] 
      \sum_{\vbh \in \cH}
       \sums[bc = d][c > 1] 
        \frac{A_k^{\omega(b)}(\det(\bo \cup \bh),b)}{b^2}
         \cdot 
          \frac{A_k^{\omega(c)}}{c^2}
   \le 
    z^{-1/2}
     \sums[d \in \cD][1 < d \le z] 
      \frac{A_k^{\omega(d)}}{d^2}
       \sum_{a \mid d}
        \sums[\vbh \in \cH][a \mid \det(\bo \cup \bh)] a.
\end{align*}
}

\noindent 
(For the last inequality note that in the innermost sum, 
$A_k^{\omega(b)}A_k^{\omega(c)} = A_k^{\omega(d)}$, 
$(\det(\bo \cup \bh),b) \le (\det(\bo \cup \bh),d)$, 
$\sum_{bc = d} 1 = 2^{\omega(d)}$, then use the trivial bound 
$
 (\det(\bo \cup \bh),d) 
   \le 
    \sum_{a \mid d, \, a \mid \det(\bo \cup \bh)} a
$.)
%
We invoke Lemma \ref{Jlem:dethap} again, this time noting that 
if $a \mid d$ and $d \le z \ll y$, then 
$y^k/a + O_k(y^{k - 1}) \ll_k y^k/a$; also,  
$\sum_{a \mid d} A_k^{\omega(a)} \le A_k^{\omega(d)}$ (recall that 
$d$ is squarefree and the convention $A_{k}$ might denote 
different constants.)
%
We find that 
{\small 
\begin{align*}
 \sums[d \in \cD][1 < d \le z] 
  \frac{A_k^{\omega(d)}}{d^2}
   \sum_{a \mid d}
    \sums[\vbh \in \cH][a \mid \det(\bo \cup \bh)] \hspace{-6pt} a
 & 
  \le 
   \sums[d \in \cD][1 < d \le z] 
    \frac{A_k^{\omega(d)}}{d^2}
     \sum_{a \mid d}
      \sums[0 < h_1 < \cdots < h_k \ll_{\sC} y][a \mid \det(\{0,h_1,\ldots,h_k\})] a
 & \hspace{-6pt} 
   \ll_{k,\sC}
    y^k 
    \sumss[\flat][d \ge 1]  
     \frac{A_k^{\omega(d)}}{d^2}
      \ll_{k,\sC}
       y^k,
\end{align*}
} 
so combining yields 
\begin{equation}
 \label{Jeq:penult}
  \sums[d \in \cD][1 < d \le z] 
   \sum_{\vbh \in \cH}
    \epsilon_{\bo \cup \bh}(d)
     =
      \bigg( \,
       \sums[d \in \cD][1 < d \le z] 
        \sum_{\vbh \in \cH}
         \prod_{p \mid d} \upsilon_{\bo \cup \bh}(p^{\alpha_p})
      \bigg)
       +
         O_{k,\sC}\big(y^kz^{-1/2}\big).
\end{equation}

Consider an arbitrary $d \in \cD$ with $1 < d \le z$.
%
We set $d_{\alpha} \defeq \prod_{p \mid d} p^{\alpha_p}$, and 
partition $\RR^k$ into cubes 
\[
 C_{d_{\alpha},\vbt} 
  \defeq 
   \{(x_1,\ldots,x_k) \in \RR^k : t_id_{\alpha} \le x_i < (t_i + 1)d_{\alpha}, i = 1,\ldots,k\},
\]
with $\vbt \defeq (t_1,\ldots,t_k)$ running over $\ZZ^k$.
%
Each $\vbh \in \cH$ is a point in a unique cube of this form: we 
call $\vbh$ a {\em $d_{\alpha}$-interior} point if this cube is 
entirely contained in $y\sC$, and $\vbh$ a 
{\em $d_{\alpha}$-boundary} point if this cube has a nonempty 
intersection with the boundary of $y\sC$.
%
We partition $\cH$ into $d_{\alpha}$-interior points and 
$d_{\alpha}$-boundary points.
%
As $\vbh$ runs over all $d_{\alpha}$-interior points of $\cH$, 
$h_i$ ($i = 1,\ldots,k$) runs over a pairwise disjoint union of 
complete residue systems modulo $d_{\alpha}$.
%
By Lemma \ref{Jlem:cancel} (a), it follows that 
\begin{equation}
 \label{Jeq:2ndlast}
  \sums[d \in \cD][1 < d \le z] 
   \sum_{\vbh \in \cH}
    \prod_{p \mid d} \upsilon_{\bo \cup \bh}(p^{\alpha_p})
  =
   \sums[d \in \cD][1 < d \le z]
    \sums[\vbh \in \cH][\text{$d_{\alpha}$-boundary}]
     \prod_{p \mid d} \upsilon_{\bo \cup \bh}(p^{\alpha_p}).
\end{equation}

For each $d \in \cD$, $1 < d \le z$ we have, by \eqref{Jeq:dbnds}, 
that $d_{\alpha} < \e^{c_1(\log z)^2/\log \log z}$.
%
Since $z = \e^{2\sqrt{\log y}}$, this means that 
$y/d_{\alpha} \ge y^{1 - O(1/\log\log y)}$.
%
From the proof of Lemma \ref{Jlem:lip} (see 
\cite[pp.\ 128--129]{LAN:94}), one can see that there are 
$\ll_{k,\sC} (y/d_{\alpha})^{k - 1}$ cubes $C_{d_{\alpha},\vbt}$ 
that have a nonempty intersection with the boundary of $y\sC$,  
and there are at most $d_{\alpha}^k$ points $\vbh$ of $\cH$ in 
each such cube.
%
Hence, in all, there are 
$
 \ll_{k,\sC} 
  y^{k - 1}d_{\alpha} < y^{k - 1}\e^{c_1(\log z)^2/\log \log z}
$ 
$d_{\alpha}$-boundary points $\vbh \in \cH$.
%
In view of this and Lemma \ref{Jlem:cancel} (b), 
\eqref{Jeq:upsbnd1}, we see that 
\begin{align}
 \label{Jeq:last}
  \begin{split}
 & 
  \sums[d \in \cD][1 < d \le z]
   \sums[\vbh \in \cH][\text{$d_{\alpha}$-boundary}]
    \prod_{p \mid d} |\upsilon_{\bo \cup \bh}(p^{\alpha_p})|
 \\
 & \hspace{30pt}
     \le 
      \sums[d \in \cD][1 < d \le z] 
       \sums[\vbh \in \cH][\text{$d_{\alpha}$-boundary}]
        \frac{A_k^{\omega(d)}}{d}
         \ll_{k,\sC}
          y^{k - 1}\e^{c_1(\log z)^2/\log\log z}(\log z)^{A_k}.
  \end{split}
\end{align}
%
(Here we have again used the elementary bound \eqref{Jeq:mert}.)

Combining \eqref{Jeq:penult}, \eqref{Jeq:2ndlast} and 
\eqref{Jeq:last}, we obtain 
\begin{equation}
 \label{Jeq:ult}
  \sums[d \in \cD][1 < d \le z] 
   \sum_{\vbh \in \cH}
    \epsilon_{\bo \cup \bh}(d) 
     \ll_{k,\sC}
      y^{k - 1}\e^{2c_1(\log z)^2/\log\log z} 
       + 
        y^kz^{-1/2}. 
\end{equation}
%
Putting \eqref{Jeq:volH}, \eqref{Jeq:lemsssapf2} and \eqref{Jeq:ult} 
into \eqref{Jeq:lemsssapf1}, and recalling that 
$z = \e^{2\sqrt{\log y}}$, we obtain \eqref{Jeq:sssa}.
\end{proof}

\end{jetsam}

\end{document}

