% !TeX root = main.tex
\section{Preliminaries}


\subsection{CAT(0) cube complexes and pocsets}

We begin by a short survey of definitions concerning CAT(0) cube complexes and pocsets. A reader who is acquainted with the basic terminology can skip this subsection. For further details see, for example, Sageev \cite{Sag12}.

A \emph{cube complex} is a collection of euclidean cubes of various dimensions in which subcubes have been identified isometrically. 

A simplicial complex is \emph{flag} if every $(n+1)$-clique in its 1-skeleton spans a $n$-simplex.
A cube complex is \emph{non-positively curved} (NPC) if the link of every vertex is a flag simplicial complex. It is a \emph{\CCC} if moreover it is simply connected.

A cube complex $\CC{X}$ can be equipped with two natural metrics, the euclidean and the $L^1$-metric. With respect to the former $\CC{X}$ is NPC if and only if it is NPC \`{a} la Gromov (see Gromov \cite{Gro87} or Bridson and Haefliger \cite{BrHa99}). While the latter is more natural to the combinatorial structure of CAT(0) cube complexes described below.

Given a cube $\CCc{C}$ and an edge $\CCc{e}$ of $\CCc{C}$. The midcube of $\CCc{c}$ associated to $\CCc{e}$ is the convex hull of the midpoints of $\CCc{e}$ and the edges parallel to $\CCc{e}$.
A \emph{hyperplane} associated to $\CCc{e}$ is the smallest subset containing the midpoint of $\CCc{e}$ and such that if it contains a midpoint of an edge it contains all the midcubes containing it.
Every hyperplane $\hyp{h}$ in a \CCC $\CC{X}$ separates $\CC{X}$ into exactly two components, see for example Niblo and Reeves \cite{NiRe98}, called the \emph{halfspaces} associated to $\hyp{h}$. A hyperplane can thus also be abstractly viewed as a pair of complementary halfspaces. 
%The \emph{carrier} $N(\hyp{h})$ of $\hyp{h}$ is the union of the cubes intersecting $\hyp{h}$.
For a \CCC $\CC{X}$ we denote by $\Hyp{H}=\Hyp{H}(X)$ the set of all hyperplanes in $\CC{X}$, and by $\Hs{H}=\Hs{H}(X)$ the set of all halfspaces. For each halfspace $\hs{h}\in \Hs{H}$ we denote by $\comp{\hs{h}}\in\Hs{H}$ its complementary halfspace, and by $\hyp{h}\in\Hyp{H}$ its bounding hyperplane, which we also identify with the pair $\{\hs{h},\comp{\hs{h}}\}$.

%A hyperplane in a \CCC \emph{separates} two points if each one belongs to a different halfspace. Conversely two hyperplanes are \emph{separated} by a point if there is no inclusion relation between the two halfspaces containing the point.
If two halfspaces $\hs{h}$ and $\hs{k}$ are such that none of $\hs{h}\cap \hs{k}$, $\comp{\hs{h}}\cap\hs{k}$, $\hs{h}\cap \comp{\hs{k}}$ and $\comp{\hs{h}}\cap \comp{\hs{k}}$ is empty, we write $\hs h \pitchfork \hs k$. %hyperplanes $\hyp h$ and $\hyp k$ intersect, we write $\hyp h \pitchfork \hyp k$.

We adopt Roller's viewpoint of Sageev's construction. Recall from Roller \cite{Rol98} that a \emph{pocset} is a triple $(\Hs{P},\le,\comp{})$ of a poset $(\Hs{P},\le)$ and an order reversing involution $\comp{}:\Hs{P}\to\Hs{P}$ satisfying $\hs{h}\neq\comp{\hs{h}}$ and $\hs{h}$ and $\comp{\hs{h}}$ are incomparable for all $\hs{h}\in\Hs{P}$.

The set of halfspaces $\Hs{H}$ of a \CCC has a natural pocset structure given by inclusion relation, and the complement operation $\comp{}$. Roller's construction starts with a locally finite pocset $(\Hs{P},\le,\comp{})$ of finite width  (see Sageev \cite{Sag12} for definitions) and constructs a \CCC $\CC{X}(\Hs{P})$ such that $(\Hs{H}(X),\subseteq, \comp{})=(\Hs{P},\le,\comp{})$. 
%We briefly recall the construction, for more details see Roller \cite{Rol98} or Sageev \cite{Sag12}.

%Two halfspaces are \emph{compatible} if their intersection is not empty in the cube complex. A subset of $\Hs{H}$ is an ultrafilter if and only if its halfspaces are pairwise compatible and it is maximal for this property.

%\todo{ Define pocset, interval (directed), and how pocsets of intervals are posets. define our kind of "crosses"}

\subsection{Tracks and patterns}\label{tracks and patterns}
The following definition of tracks and patterns is the same as in \cite{BeLa16}. It is a higher dimensional analogue of the definition of tracks and patterns (or ``$1$-patterns'') in Dunwoody \cite{Dun85}. 
As we describe in the next subsection, the $d$-patterns are used to construct $d$-dimensional CAT(0) cube complexes.

\begin{definition}
	A  \emph{drawing} on a $2$-dimensional simplicial complex $\simp{K}$ is a non empty union of simple paths in the faces of $\simp{K}$ such that:
	\begin{enumerate}
		\item on each face there is a finite number of paths,
		\item the two endpoints of each path are in the interior of distinct edges,
		\item the interior of a path is in the interior of a face,
		\item no two paths in a face have a common endpoint,
		\item if a point $x$ on an edge $\simpe{e}$ is an endpoint then in every face containing $\simpe{e}$ there exists a path having $x$ as an endpoint.
	\end{enumerate}
	
	A \emph{pre-track} is a minimal drawing. A pre-track is \emph{self-intersecting} if it contains two intersecting paths.
	
	Denote by $\usimp{K}$ the universal cover.
	
	\begin{itemize}
		\item A pre-track is a \emph{track} if none of its pre-track lifts in $\usimp{K}$ is self-intersecting.
		\item A \emph{pattern} is a set of tracks whose union is a drawing.
		\item A \emph{$d$-pattern} is a pattern such that the size of any collection of lifts of its tracks in $\usimp{K}$ that pairwise intersect is at most $d$.
	\end{itemize}
\end{definition}

We will sometimes view a pattern as the unions of its tracks in $\simp{K}$.

\subsection{The pocset structures associated to a pattern}\label{pocset} 

Let $\uptrn{P}$ be a pattern on a simply connected 2-simplex $\usimp{K}$.
For each track $\utrk{t}$ of $\uptrn{P}$, the set $\usimp{K}^0$ is naturally split by $\utrk{t}$ in two components  $\hs{h}_{\utrk{t}}$ and $\comp{\hs{h}_{\utrk{t}}}$ (see Dunwoody \cite{Dun85}). 
We call these components the \emph{halfspaces defined by $\utrk{t}$}, and the collection of all halfspaces is denoted by $\Hs{H}=\Hs{H}(\ptrn{P})$. 
This collection forms a locally finite pocset with respect to inclusion and complement operation $\comp{}$. 
If moreover $\uptrn{P}$ is a $d$-pattern, then $\Hs{H}$ has finite width. We denote by $\CC{X}=X(\Hs{H})$ the \CCC constructed from the pocset $\Hs{H}$. 
Note that the dimension of $\CC{X}$ is at most $d$. 

Note that the map $\Hypmap{\phi}$ sending $\utrk{t}\in\ptrn{P}$ to the hyperplane $\{\hs{h}_{\utrk{t}},\comp{\hs{h}_{\utrk{t}}}\} \in \Hyp{H}=\Hyp{H}(\CC{X})$ is not injective. 

\begin{definition}[parallelism]
	Two tracks of a pattern are \emph{parallel} if they define the same halfspaces. In other words if they have the same image under the map $\Hypmap{\phi}$.
\end{definition}

\subsection{Resolutions} \label{resolutions}
Let $\gp{G}$ be a finitely presented group and  $\simp{K}$ be a finite triangle complex such that $\gp{G} = \pi_1(\simp{K})$.
Given an action of $G$ on $\CC{X}$ a $d$-dimensional \CCC, we can associate a (non canonical) $d$-pattern $\ptrn{P}$ on $\simp K$ in the following way.

First build $\varphi$ a $\gp{G}$-equivariant map from $\usimp{K}$ the universal cover of $\simp{K}$ to $\CC{X}$ by arbitrarily assigning an image for a representative of each orbit of vertices of $\usimp{K}$, and then extending $\gp{G}$-equivariantly to all vertices, edges and triangles. 
The pullback of the hyperplanes of $\CC{X}$ is a $\gp{G}$-equivariant pattern on $\usimp{K}$ that induces a pattern $\ptrn{P}$ on $\simp{K}$.

As describe previously, the pattern $\ptrn{P}$ is associated to a pocset structure and a \CCC $\CC{X}'$ called a \emph{resolution} of $\CC{X}$. This resolution is naturally endowed with a $\gp{G}$-equivariant map to $\CC{X}$.

 Proofs and more properties of resolutions can be found in \cite{BeLa16,BeLa16b}.

\subsection{Intervals, crosses, meets and joins}


	Let $\CC{X}$ be a CAT(0) cube complex, and let $\CCv{x},\CCv{y}$ be two vertices in $\CC{X}$. The \emph{interval}  $\Int{I}=[\CCv{x},\CCv{y}]$ spanned by $\CCv{x}$ and $\CCv{y}$ is the poset of all halfspaces satisfying $\CCv{x}\in\hs{h}$ and $\CCv{y}\in\comp{\hs{h}}$.


\begin{remark}
	We remark that usually the interval is defined to be the $L^1$ convex hull of $\CCv{x}$ and $\CCv{y}$.
	For an interval $\Int{I}$ the set $\{\hs{h},\comp{\hs{h}} | \hs{h}\in\Int{I}\}$ is naturally a pocset.
	The associated cube complex is isomorphic to the $L^1$ convex hull of $\CCv{x}$ and $\CCv{y}$ in $\CC{X}$.
\end{remark}


	A \emph{cross} in a cube complex $\CC{X}$ is a collection of pairwise crossing hyperplanes.
	Similarly, a \emph{cross} in an interval $\Int{I}$ is a pairwise incomparable collection of halfspaces.
	The dimension of a cross is its size.

	Let $\Int{I}$ be an interval. On the set of crosses of $\Int{I}$ we define the \emph{meet} (denoted $\meet$) and  \emph{join} (denoted $\join$) operations by:
	\begin{itemize}
		\item $\CCc{C} \meet \CCc{C'} = \left\{ \hs{h} \in \CCc{C} \cup \CCc{C}' | \nexists \hs{k} \in \CCc{C} \cup \CCc{C}',~ \hs{k} < \hs{h} \right\}$.
		\item $\CCc{C} \join \CCc{C'} = \left\{ \hs{h} \in \CCc{C} \cup \CCc{C}' |\nexists \hs{k} \in \CCc{C} \cup \CCc{C}',~ \hs{k} > \hs{h} \right\}$.
	\end{itemize}


By definition, the meet and join are again crosses in the interval $\Int{I}$.
%The fact that the meet and the join are crosses is a direct application of the Diagonal rule.

\begin{observation}
	With respect to these operations the set of crosses of $\Int{I}$ form a (distributive) lattice.
	
	Moreover $\#\CCc{C}+ \#\CCc{C'} \leq \#\left(\CCc{C} \meet \CCc{C'}\right) + \#\left( \CCc{C} \join \CCc{C'}\right)$, and  $\CCc{C}\cup \CCc{C'} = \left(\CCc{C} \meet \CCc{C'}\right) \cup \left( \CCc{C} \join \CCc{C'}\right)$.
	
\end{observation}