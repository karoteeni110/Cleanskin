% !TeX root = main.tex
%%% Theorem environement
\newtheorem{theorem}{Theorem}[section]
\newtheorem{innertheorembis}{Theorem}
\newenvironment{theorembis}[1]
  {\renewcommand\theinnertheorembis{#1}\innertheorembis}
  {\endinnertheorembis}
\newtheorem{lemma}[theorem]{Lemma}
\newtheorem{definition}[theorem]{Definition}
\let\olddefinition\definition
\renewcommand{\definition}{\olddefinition\normalfont}
\newtheorem{corollary}[theorem]{Corollary}
\newtheorem{proposition}[theorem]{Proposition}
\newtheorem{fact}{Fact}
\newtheorem*{claim}{Claim}
\newtheorem{remark}[theorem]{Remark}
\let\oldremark\remark
\renewcommand{\remark}{\oldremark\normalfont}
\newtheorem{observation}[theorem]{Observation}
\let\oldobservation\observation
\renewcommand{\observation}{\oldobservation\normalfont}
\newtheorem{question}[theorem]{Question}
\let\oldquestion\question
\renewcommand{\question}{\oldquestion\normalfont}

%%% Command Shortcuts
\newcommand{\CCC}{CAT(0) cube complex\xspace}
\newcommand{\poc}{p.o.c\xspace}
\newcommand{\intc}{intercross\xspace} 
\newcommand{\countc}{countercross\xspace}
\newcommand{\intcs}{intercrosses\xspace}
\newcommand{\countcs}{countercrosses\xspace}
\newcommand{\countcer}{countercrosser\xspace}
\newcommand{\countcers}{countercrossers\xspace}
\newcommand{\R}{\mathbb R }
\newcommand{\Z}{\mathbb Z }
\newcommand{\Q}{\mathbb Q}
\newcommand{\N}{\mathbb N}
\newcommand{\Link}{\mathrm{Link}}
\newcommand{\inter}[2]{[#1 #2]}
\newcommand{\Int}[1]{\mathcal{#1}}
\newcommand{\adjP}{locally parallel\xspace}
%\newcommand{\note}[1]{\begin{Huge} #1 \end{Huge}}
%\newcommand{\stopper}{stopper\xspace}
%\newcommand{\stoppers}{stoppers\xspace}
\newcommand{\blockingpair}{p.o.p\xspace}
\newcommand{\blockingpairs}{p.o.p\xspace}
%shortcuts

\newcommand{\actson}{\curvearrowright} %group action
\newcommand{\Stab}{\mathrm{Stab}} %Stabilizer
\newcommand{\homeo}{\cong} %homeomorphic
\newcommand{\uc}[1]{\tilde{#1}} %homeomorphic

%variable types
\newcommand{\gp}[1]{#1} %group
\newcommand{\gpelt}[1]{#1} %group element

\newcommand{\mfld}[1]{#1} %manifold
\newcommand{\smfld}[1]{#1} %submanifold

%cube complexes
\newcommand{\CC}[1]{\mathbf{#1}} %CAT(0) cube complex - X,Y
\newcommand{\CCv}[1]{\mathbf{#1}} %CCC vertex  -  x,y,z, m
\newcommand{\CCc}[1]{\mathbf{#1}} %CCC cube - C
\newcommand{\CCmap}[1]{#1} %map of cube complexes - f,F
\newcommand{\seg}[1]{#1} %segment
\newcommand{\itvl}[1]{#1} %interval

\newcommand{\cCC}[1]{{\mathbf{#1}^c}} %CAT(0) cube complex - X,Y coarse
\newcommand{\cCCv}[1]{\mathbf{#1}^c} %CCC vertex  -  x,y,z, m coarse
\newcommand{\cCCc}[1]{\mathbf{#1}^c} %CCC cube - C coarse

\newcommand{\fCC}[1]{{\mathbf{#1}^f}} %CAT(0) cube complex - X,Y fine
\newcommand{\fCCv}[1]{\mathbf{#1}^f} %CCC vertex  -  x,y,z, m fine
\newcommand{\fCCc}[1]{\mathbf{#1}^f} %CCC cube - C fine

\newcommand{\oCC}[1]{{\mathbf{#1}^o}} %CAT(0) cube complex - X,Y original
\newcommand{\oCCv}[1]{\mathbf{#1}^o} %CCC vertex  -  x,y,z, m original
\newcommand{\oCCc}[1]{\mathbf{#1}^o} %CCC cube - C original


%halfspaces and hyperplanes.
\newcommand{\Hyp}[1]{\hat{\mathcal{#1}}} %collection of all hyps - H
\newcommand{\hyp}[1]{\hat{\mathfrak{#1}}} %hyperplane - h,k
\newcommand{\Hypmap}[1]{\hat{#1}_* } %map of Hyps - f


\newcommand{\cross}{\pitchfork}
\newcommand{\up}{upper\xspace}
\newcommand{\low}{lower\xspace}
\newcommand{\diverge}{diverge\xspace}

\newcommand{\lup}{ ^{\rm{up}}}
\newcommand{\llow}{ ^{\rm{low}}}
\newcommand{\ltame}{ ^{\rm{tame}}}
\newcommand{\lwild}{ ^{\rm{wild}}}


\newcommand{\Hs}[1]{\mathcal{#1}} %collection of all halfspaces - H
\newcommand{\hs}[1]{\mathfrak{#1}} %halfspace - h,k
\newcommand{\comp}[1]{{#1}^*} %complement of halfspace
\newcommand{\Hsmap}[1]{#1_*} %map of Halfspaces - f

\newcommand{\cHyp}[1]{\hat{\mathcal{#1}}^c} %collection of all hyps - H coarse
\newcommand{\chyp}[1]{\hat{\mathfrak{#1}}^c} %hyperplane - h,k coarse

\newcommand{\cHs}[1]{\mathcal{#1}^c} %collection of all halfspaces - H coarse
\newcommand{\chs}[1]{\mathfrak{#1}^c} %halfspace - h,k coarse

\newcommand{\fHyp}[1]{\hat{\mathcal{#1}}^f} %collection of all hyps - H fine
\newcommand{\fhyp}[1]{\hat{\mathfrak{#1}}^f} %hyperplane - h,k fine

\newcommand{\fHs}[1]{\mathcal{#1}^f} %collection of all halfspaces - H fine
\newcommand{\fhs}[1]{\mathfrak{#1}^f} %halfspace - h,k fine

\newcommand{\oHyp}[1]{\hat{\mathcal{#1}}^o} %collection of all hyps - H original
\newcommand{\ohyp}[1]{\hat{\mathfrak{#1}}^o} %hyperplane - h,k original

\newcommand{\oHs}[1]{\mathcal{#1}^o} %collection of all halfspaces - H original
\newcommand{\ohs}[1]{\mathfrak{#1}^o} %halfspace - h,k original

\newcommand{\simp}[1]{#1} %Simplicial complex - K
\newcommand{\simpv}[1]{#1} %Simplicial complex - vertex - v
\newcommand{\simpe}[1]{#1} %Simplicial complex - edge - e
\newcommand{\simpf}[1]{#1} %Simplicial complex - face - f
\newcommand{\simps}[1]{#1} %Simplicial complex - 3-simplex - \sigma

\newcommand{\usimp}[1]{\tilde{#1}} %universal covering (UC) Simplicial complex - K
\newcommand{\usimpv}[1]{\tilde{#1}} %UC Simplicial complex - vertex - v
\newcommand{\usimpe}[1]{\tilde{#1}} %UC Simplicial complex - edge - e
\newcommand{\usimpf}[1]{\tilde{#1}} %UC Simplicial complex - face - f

\newcommand{\trk}[1]{#1} %track - t
\newcommand{\ptrn}[1]{{\mathcal{#1}}} %pattern - P
\newcommand{\utrk}[1]{{\tilde{#1}}} %track in the UC - t
\newcommand{\uptrn}[1]{{\tilde{\mathcal{#1}}}} %pattern in the UC - P

\newcommand{\chainofhyps}[4]{\left((\hs{#1}_{#3},\hs{#2}_{#3}),\ldots,(\hs{#1}_{#4},\hs{#2}_{#4})\right)}
\newcommand{\chainoftuples}[4]{\left((\hs{#1}^{1}_{#3},\dots,\hs{#1}^{#2}_{#3}),\ldots,(\hs{#1}^{1}_{#4},\dots,\hyp{#1}^{#2}_{#4})\right)}
\newcommand{\chainofcrosses}[3]{\left(\CCc{#1}_{#2},\ldots,\CCc{#1}_{#3}\right)}
\newcommand{\setofhyps}[4]{\left\{(\hs{#1}_{#3},\hs{#2}_{#3}),\ldots,(\hs{#1}_{#4},\hs{#2}_{#4})\right\}}


\newcommand{\meet}{\wedge} %track - t
\newcommand{\bigmeet}{\bigwedge} %track - t
\newcommand{\join}{\vee} %track - t
\newcommand{\bigjoin}{\bigvee} %track - t



\DeclareFontFamily{U}{matha}{\hyphenchar\font45}
\DeclareFontShape{U}{matha}{m}{n}{
      <5> <6> <7> <8> <9> <10> gen * matha
      <10.95> matha10 <12> <14.4> <17.28> <20.74> <24.88> matha12
      }{}
\DeclareSymbolFont{matha}{U}{matha}{m}{n}
%\DeclareMathSymbol{\join}         {2}{matha}{"5F}
%\DeclareMathSymbol{\meet}           {2}{matha}{"5E}