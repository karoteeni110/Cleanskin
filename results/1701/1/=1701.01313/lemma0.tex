% !TeX root = main.tex
\section{Intercrosses and Countercrosses}


Let $\Int{I}$ be an interval. Let $\hs{h}<\hs{k}$ be two halfspaces of $\Int{I}$. 
We say that $\hs{h}$ and $\hs{k}$ are \emph{adjacent} if there is no halfspace $\hs{t}$ such that $\hs{h}<\hs{t}<\hs{k}$. An \emph{\intc} with respect to $\hs{h}<\hs{k}$ is a (non empty) cross $\CCc{C} \subset \Int{I}$ decomposed as two disjoints sets $\CCc{C} = \CCc{H} \cup \CCc{K}$ such that
\begin{itemize}
%\item the sets $\CCc{H}$ and $\CCc{K}$ are not empty,
\item  every element of $\CCc{H}$ is transverse to $\hs{h}$,
\item  every element of $\CCc{K}$ is transverse to $\hs{k}$ and disjoint from $\hs{h}$.
\end{itemize}


Let $\hs{h}<\hs{k}$ be two halfspaces in $\Int{I}$ and let $\CCc{C} = \CCc{H} \cup \CCc{K}$ be an \intc for $\hs{h}$ and $\hs{k}$. A \emph{\countc} is a cross $\CCc{C}' \subset \Int{I}$ such that:
\begin{itemize}
\item $\# \CCc{C} < \# \CCc{C}'$,
\item if $\CCc{K} \neq \emptyset$, there exist elements $\hs{k}'\in\CCc{C}'$ and $\hs{k}\in\CCc{K}$ such that $\hs{k'}\le\hs{k}$,
\item if $\CCc{K} = \emptyset$, then $\hs h\in\CCc{C}'$,
\item there exists an element $\hs{k}'\in\CCc{C}'$ such that $\hs{k}'\ge\hs h$.
\end{itemize}

%If $\hs{h}>\hs{k}$, a cross is a $\hs{h}$-\emph{\countc} for $\CCc{C}$, if it is a $\hs{h}$-\emph{\countc} for the reverse order on halfspaces.

%When it will be obvious from context, we will write \countc instead of $\hs{h}$-\countc.



Given two halfspaces $\hs{h}<\hs{k}$, we say that $\hs{k}$ is \emph{locally parallel} to $\hs{h}$ if they are adjacent and for any \intc $\CCc{C}$ between them and any other adjacent pair $\hs{h}<\hs{k'}$ admits an \intc of dimension greater or equal to the one of $\CCc{C}$.



We emphasize the fact that these definitions are oriented. In particular, if $\hs{k}$ is locally parallel to $\hs{h}$ in $\Int{I}$, it does not imply that $\comp{\hs{h}}$ is locally parallel to $\comp{\hs{k}}$ with respect to the inverse orientation of $\Int{I}$.

%Most of the time we will omit to precise if the local parallelism is from above or from below.

%A pair $(\hs{h}, \hs{k})$ is \adjP if $\hs{k}$ is \adjP to $\hs{h}$. \todo{why do we need this line?}\todo{I just wanted to fix the asymmetry in the notation}

\begin{lemma}\label{lemma0}
Let $\Int{I}$ be an interval, and let $\hs{h}$ be a non-maximal halfspace. Then there exists an adjacent halfspace $\hs{k}>\hs h$, for which any \intc admits a \countc.

%By symmetry, if $\hs{h}$ is a a non-minimal halfspace, there exist $\hs{k}$, adjacent to and below $\hs h$, for which any \intc admits a $\hs h$-\countc.
\end{lemma}

We call such a halfspace a \emph{\countcer}.

\begin{proof}

Let $\Hs{K} = \left\{\hs{k}_1,\dots, \hs{k}_n\right\}$ be the set of halfspaces adjacent to and above $\hs{h}$.
If one element of $\Hs{K}$ does not share an \intc with $\hs{h}$ then it verifies the Lemma.

Otherwise for each $i$, let $\CCc{C}_i = \CCc{H}_i \cup \CCc{K}_i$ be an \intc for the pair $(\hs{h},\hs{k}_i)$. To prove the lemma we need to show that one of these \intcs admits a \countc.

Notice that if some $\CCc{K}_i$ is empty, then $\CCc{C}_i\cup \left\{\hs h\right\}$ is a \countc for $\CCc{C}_i$. Similarly, if $\CCc{K}_i$ is not empty and there is no halfspace in $\CCc{H}_i$ which is strictly below $\hs{k}_i$, then $\CCc{C}_i\cup \left\{\hs k_i\right\}$ is a \countc for $\CCc{C}_i$. We thus can assume that for all $i$ the set $\CCc{K}_i$ is non-empty and there exists $\hs{s}\in\CCc{H}_i$ such that $\hs{s}<\hs{k}_i$ (and in particular, $\CCc{H}_i$ is non-empty).

%The construction use the following fact:
Notice that for any $\hs{t} \in \bigcup_{\hs{k_i} \in \Hs{K}}\CCc{K_i}$ there exists $j$ such that $\hs{k}_j \leq \hs{t}$, and therefore for some $\hs{s} \in \CCc{H}_j$, we have $\hs{s} < \hs{t}$. This implies that $\left(\bigmeet_{\hs{k_i} \in \Hs{K}}\CCc{C}_i\right)\cap\left(\bigcup_{\hs{k_i} \in \Hs{K}}\CCc{K_i}\right) = \emptyset$.

 Let $\Hs{K}'$ be a minimal subset of $\Hs{K}$ such that $\bigmeet_{\hs{k}_i \in \Hs{K}'} \CCc{C}_i \cap \bigcup_{\hs{k}_i \in \Hs{K}'} \CCc{K}_i = \emptyset$.
 % (where $\CCc{C}_i = \CCc{H}'_i \cup \CCc{K}'_i$ is the \intc associated to $\hs{k}'_i$). 
  \begin{claim}
  For any proper non-empty subset $\Hs{K}'' \subset \Hs{K}'$, there exists $\hs{k}_j \in \Hs{K}'\setminus\Hs{K}''$ such that \[\CCc{K}_j \cap \left( \bigmeet_{\hs{k}_i\in \Hs{K}''} \CCc{C}_i \join \CCc{C}_j\right) \neq \emptyset.\]
  \end{claim}
  \begin{proof}By contradiction, assume that for all $\hs{k}_j \in  \Hs{K}'\setminus \Hs{K}''$, we have $\CCc{K}_j \subset  \bigmeet_{\hs{k}_i\in \Hs{K}''} \CCc{C}_i \meet \CCc{C}_j$. 
  But this implies that
  \[ \bigmeet_{\hs{k}_j\in \Hs{K}'\setminus \Hs{K}''} \CCc{C}_j \cap \bigcup_{\hs{k}_j \in  \Hs{K}'\setminus \Hs{K}''}\CCc{K}_j \subseteq \bigmeet_{\hs{k}_i\in \Hs{K}''} \CCc{C}_i \meet \bigmeet_{\hs{k}_j\in \Hs{K}'\setminus \Hs{K}''} \CCc{C}_j=\bigmeet_{\hs{k}_i \in \Hs{K}'} \CCc{C}_i.\] 
  But since $\Hs{K}'$ verifies $\bigmeet_{\hs{k}_i \in \Hs{K}'} \CCc{C}_i \cap \bigcup_{\hs{k}_i \in \Hs{K}'} \CCc{K}_i = \emptyset$, it would imply that  $\Hs{K}'\setminus \Hs{K}''$ verifies 
  \[\bigmeet_{\hs{k}_j\in \Hs{K}'\setminus \Hs{K}''} \CCc{C}_j \cap \bigcup_{\hs{k}_j \in  \Hs{K}'\setminus \Hs{K}''}\CCc{K}_j = \emptyset,\] which contradicts the minimality of $\Hs{K}'$.
\end{proof}
Let us now construct a \countc for some element of $\Hs{K}'$.
Choose some $\hs{k}_{i_1}\in\Hs{K}'$, and set $\Hs{K}''_1 = \{\hs{k}_{i_1}\}$ and $\CCc{D}_1 = \CCc{C}_{i_1}$.
We will construct subsets $\Hs{K}''_i\subset\Hs{K}'$, of size $i$, and crosses $\CCc{D}_i$ inductively, so that they satisfy:
\begin{itemize}
\item $\CCc{D}_i= \bigmeet _{\hs{k}_j\in\Hs{K}''_i} \CCc{C}_j$,
\item $\#\CCc{D}_i\ge\#\CCc{D}_{i-1}$, and
\item $\Hs{K}''_{i-1}$ is strictly contained in $\Hs{K}''_i$.
\end{itemize}

We construct $\CCc{D}_i$ from $\CCc{D}_{i-1}$ in the following way. By the claim there exists $\hs{k}_j\in\Hs{K}'\setminus\Hs{K}''_{i-1}$ such that an element of $\CCc{K}_j$ belongs to $\CCc{D}_{i-1} \join \CCc{C}_j$. If $\#(\CCc{D}_{i-1} \join \CCc{C}_j)>\#\CCc{C}_j$, then $\CCc{D}_{i-1} \join \CCc{C}_j$ is a \countc for $\CCc{C}_j$, and we are done. 
Otherwise $\#(\CCc{D}_{i-1} \meet \CCc{C}_j)\geq \# \CCc{D}_{i-1}$, and we can define $\Hs{K}''_i = \Hs{K}''_{i-1}\cup\{\hs{k}_j\}$ and accordingly $\CCc{D}_i = \CCc{D}_{i-1} \meet \CCc{C}_j$.
 
If we did not find a \countc in the process, we end up (after $p=\#\Hs{K}'$ steps) with $\Hs{K}''_p=\Hs{K}'$ and $\CCc{D}_{p} = \bigmeet_{\hs{k_i} \in \Hs{K}'}\CCc{C}_i$. But since $\CCc{D}_p \cap \CCc{K}_{i_1}=\emptyset$,  $\hs{h}$ is transverse to every element of  $\CCc{D}_{p}$ and therefore $\CCc{D}_{p}\cup \{\hs{h}\}$ (which satisfies $\#(\CCc{D}_{p}\cup \{\hs{h}\})> \#\CCc{D}_p\ge\ldots\ge\#\CCc{D}_1=\#\CCc{C}_{i_1}$) is a \countc for  $\hs{k}_{i_1}$.
\end{proof}
