% !TeX root = main.tex
\section{Bounds on locally parallel pairs of halfspaces}

%\begin{remark}\label{stronglytameimpliestamecountc}
%	If a chain of \intcs is $\CCc{K}$-tame, then the chain of \countcs is tame, and contains a halfspace which is equal or above $\hs h$.
%\end{remark}

\begin{lemma}\label{lemma1staircase}
Given an interval $\Int I$. There exists a constant $C$ depending only on the dimension such that at most $C$ pairs of \adjP halfspaces can form a staircase.
\end{lemma}

\begin{corollary}\label{lemma1}
Given an interval $\Int{I}$ and a point $\CCv{m}$ of $\Int{I}$. There exists a constant $C$ depending only on the dimension such that at most $C$ pairs of \adjP halfspaces are separated by $\CCv{m}$.
\end{corollary}


\begin{proof}[Proof of Corollary \ref{lemma1}]
If there is no bound, from Lemma \ref{reductionforpairs}, we can assume, that the pairs of \adjP halfspaces form a staircase or a ladder. But since each of the pairs is separated by $\CCc{m}$ it has to be a staircase. Lemma \ref{lemma1staircase} concludes.
\end{proof}

\begin{proof}[Proof of Lemma \ref{lemma1staircase}]
By contradiction, assume that for any $C$, there exists an interval $\Int I$ and a staircase of \adjP pairs $(\hs{h}_1, \hs{k}_1)\dots (\hs{h}_C, \hs{k}_C)$. %Recall that local parallelism is not symetrical. 
%However, up to taking more pairs and reversing the orientation of $\Int I$, we can assume that $\hs{h}_i < \hs{k}_i$ for all $i$.

For each $1\le i<C$, let $\hs{l}_i > \hs{h}_i$ and $\hs{o}_i > \hs{h}_i$ be halfspaces adjacent to $\hs{h}_i$ such that $\hs{l}_i \leq \hs{h}_{i+1}$ and $\hs{o}_i$ is a \countcer (see Lemma \ref{lemma0}). 
Moreover if $\hs{o}_i \leq \hs{h}_{i+1}$, we assume $\hs{o}_i=\hs{l}_i$. Let $\CCc{C}_{\hs{l},i}$ be an \intc for the pair $(\hs{h}_i, \hs{l}_i)$ of maximal dimension.

Since $\hs{l}_i\le\hs{h}_{i+1}$ for $1\le i<C$, the intercrosses $\left\{\CCc{C}_{\hs{l},1},\ldots,\CCc{C}_{\hs{l},C-1}\right\}$ are weakly tame with respect to $\chainofhyps{h}{k}{1}{C}$. Hence by Lemma \ref{reductionforcrosses} we may assume the following.
\begin{enumerate}
\item \label{l is o} The sequence of halfspaces $\left\{\hs{o}_i\right\}$ is either tame or wild with respect to the chain of pairs $\chainofhyps{h}{k}{1}{C}$. %Either $\hs{o}_i$ below $\hs{h}_{i+1}$ for all $i$, or $\hs{o}_i$ is transverse to $\hs{h}_j$ for $j>i$ \todo{use the terminology tame and wild?}. 
In particular either $\hs{o}_i = \hs{l}_i$ for all $i$ or $\hs{o}_i \neq  \hs{l}_i$ for all $i$.
\item The dimension of the $\CCc{C}_{\hs{l},i}$ is a constant that we denote $p$.
\item \label{reginccubes} The $\CCc{C}_{\hs{l},i}$ are regularly increasing.
\item Either $\CCc{C}_{\hs{l},i}$ contains a halfspace $\geq \hs{h}_i$, or $\CCc{C}_{\hs{l},i}\cup \hs{h}_i$ is a cross (which trivially contains a halfspace $\geq \hs{h}_i$).
\item \label{tameorwildcrosses} Every subchain of halfspaces of the chain of crosses is either tame or wild.%Either $\CCc{C}_{\hs{l},i}$ is tame for all $i$, or every chain of halfspace of $\CCc{C}_{\hs{l},i}$ is transverse to $\hs{h}_j$ for $j>i$ \todo{use wild?}.
\end{enumerate}
%The points \ref{l is o} and \ref{reductionforcrosses} follow from Lemma \ref{tameorunbounded}. The point \ref{reginccubes} follows from Lemmas \ref{Ramseyforcubes} and \ref{orderedimpliesincreasing}.

Note that since  $\CCc{C}_{\hs{l},i}$ is a \intc of maximal dimension for $\hs{h}_i$ and $\hs{l}$, by definition of \adjP the pair $(\hs{h}_i,\hs{k}_i)$ share no \intc of dimension $>p$.

If the $\CCc{C}_{\hs{l},i}$ are wild, then $\CCc{C}_{\hs{l},i} \cup \left\{\hs{h}_{i+2}\right\}$ is a \intc for the pair $(\hs{h}_{i+1}, \hs{k}_{i+1})$ since $\CCc{C}_{\hs{l},i}$ is transverse to $\hs{h}_{i+1}$ and $\hs{h}_{i+2}$ is transverse to $\hs{k}_{i+1}$. But this is a contradiction as the dimension of this \intc is $p+1$.

So the crosses $\CCc{C}_{\hs{l},i}$ are tame.
We first build a tame \intc for the pair $(\hs{h}_i, \hs{o}_i)$.

If $\hs{o}_i = \hs{l}_i$ for all $i$ then $\CCc{C}_{\hs{l},i}$ is the cross that we want. Otherwise the halfspaces $\hs{o}_i$ are wild, i.e, the chain of pairs $\chainofhyps{h}{o}{1}{C}$ form a staircase. 

 %Applying Lemma \ref{tameorunbounded}, we can assume that every subchain of $\left\{\CCc{C}_{\hs{l},i}\right\}$ is tame or wild. 
We can apply Lemma \ref{vertical horizontal trick}, to obtain a regularly increasing sequence of tame and $\CCc{K}$-tame \intc $\CCc{C}_{\hs{o},i}$ for the pairs $(\hs{h}_i,\hs{o}_i)$ for the even indices $1<i<C$.

%Applying Lemma \ref{tametostronglytame} we can assume that $\CCc{C}_{\hs{o},i}$ is $\CCc{K}$-tame. 
As $\hs{o}_i$ are \countcers, we can produce \countcs $\CCc{C}'_{\hs{o},i}$ of dimension $p+1$ for the crosses $\CCc{C}_{\hs{o},i}$. Tameness and $\CCc{K}$-tameness of $\CCc{C}_{\hs{o},i}$ and the definition of the \countc imply that $\CCc{C}'_{\hs{o},i}$ is tame and has an element above $\hs{h}_i$.

Using Lemma \ref{reductionforcrosses}, we can assume that the crosses $\CCc{C}'_{\hs{o},i}$ are regularly increasing and that every subchain of $\left\{\CCc{C}'_{\hs{o},i}\right\}$ is tame or wild.
We can then apply Lemma \ref{vertical horizontal trick}, to get \intcs of dimension $p+1$ for the pairs $(\hs{h}_i,\hs{k}_i)$ when $4|i$, which is a contradiction.
\end{proof}

\begin{lemma}\label{lemmaforlemma2}
	For all $n$ there exists $N=N(n,d)$ such that for every ladder of adjacent halfspaces $\chainofhyps{h}{k}{1}{N}$ and a regularly increasing sequence of $d$-dimensional \intcs $\chainofcrosses{C}{1}{N}$ there is a sequence  $\chainofcrosses{D}{i_1}{i_n}$ of $n$ regularly ordered crosses such that for every subchain $\hs{t}_{i_1},\ldots,\hs{t}_{i_n}$, either for all $1\le r\le n$ the halfspace $\hs{t}_{i_r}$ crosses $\hs{h}_{i_j}$ for all $j$ (in which case we call it \emph{unbounded}), or for all $1\le r\le n$ $\hs{t}_{i_r}$ is between $\hs{h}_{i_{r-1}}$ and $\hs{h}_{i_{r+1}}$ (in which case we call it \emph{bounded}).
\end{lemma}	

\begin{proof}
	Since $\chainofhyps{h}{k}{1}{N}$ is a ladder and $\chainofcrosses{C}{1}{N}$ are \intcs, it follows that $\chainofcrosses{C}{1}{N}$ are weakly tame with respect to $\chainofhyps{h}{k}{1}{N}$. Lemma \ref{reductionforcrosses} applied twice for the two orientations of the interval, gives a subsequence of crosses, which by abuse of notation we denote by $\chainofcrosses{C}{0}{n+1}$, in which every halfspaces is one of the 4 possible options of being tame/wild in the two directions. 
	Let us denote the partition of each cross into the 4 categories by $\CCc{C}_i^{ut,dt},\CCc{C}_i^{uw,dt},\CCc{C}_i^{ut,dw},\CCc{C}_i^{uw,dw}$, where the letters stand for \underline{u}p, \underline{d}own, \underline{t}ame and \underline{w}ild.
	For $i=1,\ldots,n$ form the crosses $\CCc{D}_i$ by \[\CCc{D}_i = \CCc{C}_i^{ut,dt}\cup\CCc{C}_0^{uw,dt}\cup\CCc{C}_{n+2}^{ut,dw}\cup\CCc{C}_i^{uw,dw}.\]
	It is easy to verify that the sets $\CCc{D}_i$  are \intcs and that they have the desired property with respect to $\chainofhyps{h}{k}{1}{n}$.
\end{proof}


\begin{lemma}\label{lemma2}
	Let $\CCv{x},\CCv{y}_1,\CCv{y}_2$ be three vertices, let $\CCv{m}$ be their median, and let $\Int{I}_i$, $i=1,2$, be the interval spanned between $\CCv{x}$ and $\CCv{y}_i$. There exists a constant $C$ depending only on the dimension such that at most $C$ pairs of adjacent halfspaces which separate $\CCv{x}$ and $\CCv{m}$, are \adjP in $\Int{I}_1$ but not in $\Int{I}_2$.
	
	The same statement is also true for the intervals $\Int{I}'_i =[\CCv{y}_i, \CCv{x}]$.
\end{lemma}

\begin{proof}
	By contradiction, let $(\hs{h}_1, \hs{k}_1)\dots (\hs{h}_C, \hs{k}_C)$ be such pairs. By Lemma \ref{reductionforpairs} we can assume that it forms a staircase or a ladder, and by 
	Lemma \ref{lemma1staircase} we can assume that it is a ladder. %$\hs{k}_i < \hs{h}_i$. \todo{call it strongly nicely ordered, or something}
	
	Let $\CCc{C}_i$ be the \intc of maximal dimension in $(\hs{h}_i,\hs{k}_i)$ in the interval $\Int{I}_2$. 
	By assumption, for every $i$ there exists a locally parallel halfspace $\hs{t}_i$ for $\hs{h}_i$ such that all the \intcs in $(\hs{h}_i,\hs{t}_i)$ in $\Int{I}_2$ have strictly smaller dimension than that of $\CCc{C}_i$.
	By Lemmas \ref{reductionforpairs} and \ref{lemma1staircase} we may assume that $(\hs{h}_1,\hs{t}_1)\ldots,(\hs{h}_n,\hs{t}_n)$ is a ladder
	and in particular separate $\CCv{x}$ and $\CCv{m}$. 
	This implies that they can be considered as halfspaces in $\Int{I}_1$ as well.
	Let $\CCc{D}_i$ be an \intc of maximal dimension for $(\hs{h}_i,\hs{t}_i)$ in the interval $\Int{I}_1$.
	
	Apply Lemma \ref{lemmaforlemma2}, for both $\CCc{C}_i$ and $\CCc{D}_i$. Denote by $\CCc{C}^b_i,\CCc{C}^{ub}_i$ (resp. $\CCc{D}^b_i,\CCc{D}^{ub}_i$) the \underline{b}ounded and \underline{u}n\underline{b}ounded halfspaces of $\CCc{C}_i$ (resp. $\CCc{D}_i$). Note that $\CCc{C}^b_i\cup \CCc{D}^{ub}_i$ (resp. $\CCc{C}^{ub}_i\cup\CCc{D}^{b}_i$) is an \intc for $(\hs{h}_i,\hs{t}_i)$ in $\Int{I}_1$ (resp. \intc for for $(\hs{h}_i,\hs{k}_i)$ in $\Int{I}_2$).
	Thus by assumption \[\#\CCc{C}^b_i+\#\CCc{C}^{ub}_i = \#\CCc{C}_i > \#\CCc{C}^{ub}_i + \#\CCc{D}^{b}_i.\] On the other hand, since $(\hs{h}_i,\hs{k}_i)$ is \adjP, \[\#\CCc{D}^b_i+\#\CCc{D}^{ub}_i=\#\CCc{D}_i \ge \#\CCc{C}^b_i + \#\CCc{D}^{ub}_i.\] Adding these two inequalities gives a contradiction.
	
	For the intervals $\Int{I}'_i$, the claim follows easily since if $(\hs{h},\hs{k})$ are locally parallel in $\Int{I}'_1$ then they must be locally parallel in $\Int{I}'_2$. This is because any halfspace which is greater than $\hs{h}$ in one of the intervals then it also belongs to the other interval.
\end{proof}