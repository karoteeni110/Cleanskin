\documentclass[aps,twocolumn,showpacs]{revtex4}
\usepackage{graphicx}
\usepackage[all]{xy}
\usepackage{amsmath}
\usepackage{amssymb}
\usepackage{epstopdf}
%%%%%%%%%%%%%%%%%%%%%%%%%%%%%%%%%%%%%%%%
\newcommand{\be}{\begin{equation}}
\newcommand{\ee}{\end{equation}}
\newcommand{\bn}{\begin{eqnarray}}
\newcommand{\en}{\end{eqnarray}}
\newcommand{\bes}{\begin{subequations}}
\newcommand{\ees}{\end{subequations}}
\newcommand{\wt}{\widetilde}
\newcommand{\bb}{\bibitem}
\newcommand{\p}{\partial}
%%%%%%%%%%%%%%%%%%%%%%%%%%%%%%%%%%%%%%%%
\begin{document}

\title{Cosmology from a variant speed of light scenario in $f(R,T)$ gravity}
\author{P.H.R.S. Moraes}
\affiliation{{\small {
ITA - Instituto Tecnol\'ogico de Aeron\'autica - Departamento de F\'isica, 12228-900, S\~ao Jos\'e dos Campos, S\~ao Paulo, Brasil \\
%Av. dos Astronautas 1758, S\~ao Jos\'e dos Campos, 12227-010 SP, Brazil\\
}
}}

\begin{abstract}

\end{abstract}

\pacs{}

\maketitle

\section{Introduction}\label{sec:int}

The observation of Type Ia Supernovae \cite{riess/1998,perlmutter/1999} and cosmic microwave background temperature \cite{hinshaw/2013} seems to support a universe undergoing a phase of accelerated expansion. In a universe composed mostly by matter, this feature is highly counterintuitive. Therefore to account for the acceleration, it is common to assume that the universe is made mostly ($\sim70\%$) by an exotic component, named dark energy (DE). The DE would have an equation of state (EoS) $\omega\sim-1$, which justifies the present universe dynamics. In standard cosmology (or $\Lambda$CDM model), the DE is mathematically described as a cosmological constant inserted in the Einstein's field equations (FEs). However, the cosmological constant, coincidence and dark matter problems, missing satellites, hierarchy problem and other shortcomings (see \cite{clifton/2012} and references therein) arising from $\Lambda$CDM model, yield the consideration of alternative cosmological models.

Therefore we are led to search for some kind of matter fields which generates negative pressure enough to account for the acceleration. A scalar field which goes slowly down to its potential, making the potential term to dominate over the kinetic one, can produce sufficient negative pressure, and such a cosmological scenario is named ``quintessence" (check, for instance, \cite{cds}-\cite{ms/2014}). As one can see in these references, it is also common to assume the universe dynamics is ruled by two, instead of one, scalar fields. In fact, in \cite{ms/2014}, such a two scalar field cosmological scenario has presented some advantages when compared to the one scalar field models, since it describes not only the accelerated phase of the universe, but also the phases in which the expansion decelerates, i.e., those for which the universe is dominated by radiation and matter.

% Another alternative for the standard cosmology issues are the family of $f(R)$ and $f(R,T)$ theories.

\section{Discussion}\label{sec:dis}

%\acknowledgments

\pagebreak

%%%%%%%%%%%%%%%%%%%%%%%%%%%%%%%%%%%%%%%%%%%%%%%%%%%%%%%%%%%%%%%%%%

\begin{thebibliography}{99}

\bb{riess/1998} A.G. Riess et al., The Astronomical Journal {\bf 116}, 1009 (1998).

\bb{perlmutter/1999} G. Perlmutter  et al., ApJ. {\bf 517}, 565 (1999). 

\bb{hinshaw/2013} G. Hinshaw {\it et al.}, ApJs {\bf 208}, 19 (2013).

\bibitem{clifton/2012} T. Clifton et al., Phys. Rep. {\bf 513}, (2012) 1. 

\bb{cds} R.R. Caldwell, R. Dave, and P.J. Steinhardt, Phys. Rev. Lett. {\bf 80}, 1582 (1998).

\bb{tsujikawa/2013} S. Tsujikawa, Class. Quant. Grav. {\bf 30}, 214003 (2013).

\bb{sahni/2000} V. Sahni and L. Wang, Phys. Rev. D {\bf 62}, 103517 (2000).

\bb{khurshudyan/2014} M. Khurshudyan et al., Int. J. Theor. Phys {\bf 53}, 2370 (2014).

\bb{bento/2002} M.C. Bento et al., Phys. Rev. D {\bf 65}, 067301 (2002).

\bb{ms/2014} P.H.R.S. Moraes and J.R.L. Santos, Phys. Rev. D {\bf 89}, 083516 (2014).

%\bb{linde} A. D. Linde, Phys. Lett {\bf 108B}, 389 (1982).

%\bb{albrecht} A. Albrecht and P. J. Steinhardt, Phys. Rev. Lett. {\bf 48}, 1220 (1982).

%\bb{liddle} A. R. Liddle and D. H. Lyth, Phys. Lett. B {\bf 291}, 391 (1992).

%\bb{vernizzi/2006} F. Vernizzi and D. Wands, J. Cosm. Astrop. Phys. {\bf 5}, 19 (2006).

%\bb{choi/2007} K.-Y. Choi, L. M. H. Hall and C. van de Bruck, J. Cosm. Astrop. Phys. {\bf 2}, 29 (2007).

%\bb{wands/2002} D. Wands {\it et al.}, Phys. Rev. D {\bf 66}, 043520 (2002).

%\bb{fujii/2000} Y. Fujii, Phys. Rev. D {\bf 62}, 064004 (2000).

%\bb{linde/1990} A. A. Linde, Physics Letters B {\bf 249}, 18 (1990).

%\bb{kofman/1997} L. Kofman, A. A. Linde and A. A. Starobinsky, Phys. Rev. D {\bf 56}, 3258 (1997).

%\bb{bertolami/1986} O. Bertolami and G.G. Ross, Physics Letters B {\bf 171}, 163 (1986).

%\bb{bls} D. Bazeia, L. Losano e J.R.L. Santos, Physics Letters A {\bf 377}, 1615 (2013).

%\bb{def_1} D. Bazeia, L. Losano, J.M.C. Malbouisson, Phys. Review D {\bf 66},  101701 (2002).

%\bb{motohashi/2014} H. Motohashi, A.A. Starobinsky and J. Yokohama, arXiv:astro-ph/1411.5021.

%\bb{kinney/2005} W.H. Kinney, Phys. Rev. D {\bf 72}, 023515 (2005).

%\bb{martin/2013} J. Martin, H. Motohashi and T. Suyama, Phys. Rev. D {\bf 87}, 023514 (2013).

%\bb{bglm} D. Bazeia, C. B. Gomes, L. Losano, and R. Menezes, Phys. Lett. B {\bf 633}, 415 (2006).

%\bb{guth/1981} A. H. Guth, Phys. Rev. D {\bf 23}, 347 (1981).

%\bb{ilic/2010} S. Ilic {\it et al.}, Phys. Rev. D {\bf 81}, 103502 (2010).

%\bb{allen/2004} S. W. Allen, R. W. Schmidt, K. Ebeling,  A. C. Fabian and L. van Speybroeck, MNRAS {\bf 353}, 457 (2004).

%\bb{eisenstein/2005} D.J. Eisenstein, I. Zehavi and D. W. Hogg, ApJ {\bf 633}, 560 (2005).  

%\bb{percival/2010} W. J. Percival et al., MNRAS {\bf 401}, 2148 (2010).

%\bb{jimenez/2003} R. Jimenez, L. Verde, T. Treu and D. Stern, ApJ {\bf 593}, 622 (2003).

%\bb{weinberg/1989} S. Weinberg, Reviews of Modern Physics {\bf 61}, 1 (1989).

%\bb{salopek} D. S. Salopek and J. R. Bond, Phys. Rev. D {\bf 42}, 3936 (1990).

%\bb{brans/1961} C. Brans and R.H. Dicke, Physical Review {\bf 124}, 925 (1961).

%\bb{mathiazhagan/1984} C. Mathiazhagan and V.B. Johri, Classical and Quantum Gravity {\bf 1}, L29 (1984).

%\bb{ratra/1988} B. Ratra and P. J. E. Peebles, Phys. Rev. D {\bf 37}, 3406 (1988).

%\bb{peebles/1988} P. J. E. Peebles and B. Ratra, ApJ {\bf 325}, L17 (1988).

%\bb{wetterich/1988} C. Wetterich, Nuclear Physics B {\bf 302}, 668 (1988).

%\bb{caldwell/1993} D.O. Caldwell and R.N. Mohapatra, Phys. Rev. D {\bf 48}, 3259 (1993).

%\bb{albrecht/2000} A. Albrecht and C. Skordis, Phys. Rev. Lett. {\bf 84}, 2076 (2000).

%\bb{padmanabhan/2002} T. Padmanabhan, Phys. Rev. D {\bf 66}, 021301 (2002).

%\bb{fhsw} J. A. Frieman, C. T. Hill, A. Stebbins, and I. Waga, Phys. Rev. Lett. {\bf 75}, 2077 (1995).

%\bb{sc} Sean M. Carroll, Phys. Rev. Lett. {\bf 81}, 3067 (1998). 

%\bb{zws} I. Zlatev, L. Wang, and P. J. Steinhardt, Phys. Rev. Lett. {\bf 82}, 896 (1999). 

%\bb{alr} P.P. Avelino, L. Losano and J.J. Rodrigues, Physics Letters B {\bf 699}, 10 (2011).

%\bb{blrr} D. Bazeia, L. Losano, J.J. Rodrigues and R. Rosenfeld,  European Physical Journal C {\bf 55}, 113 (2008).

%\bb{blp} D. Bazeia, L. Losano and A. B. Pavan, arXiv:astro-ph/0611021. 

%\bb{ablop} P.P. Avelino, D. Bazeia, L. Losano, J.C.R.E. Oliveira and A.B. Pavan, Phys. Rev. D {\bf 82}, 063534 (2010).

%\bb{blr} D. Bazeia, L. Losano and J.J. Rodrigues, arXiv:hep-th/0610028.

%\bb{bd} D. Bazeia and J. D. Dantas, Phys. Rev. D {\bf 85}, 067303 (2012).

%\bb{def_1} Bazeia, D.; Losano, L.; Malbouisson, J. M. C.  Phys. Review D {\bf 66},  101701 (2002).

%\bb{kinney} William H. Kinney, Phys. Rev. D {\bf 56}, 2002 (1997).

%\bb{markov} L. P. Grishchuk and Yu. V. Sidorav, in Fourth Seminar on Quantum Gravity, edited by M. A. Markov, V. A. Berezin and V. P. Frolov (Word Scientific, Singapore, 1988).

%\bb{muslimov} A. G. Muslimov, Class. Quantum Grav. {\bf 7}, 231 (1990).

%\bb{ryden/2003} B. Ryden, {\it Introduction to Cosmology} Addison Wesley, San Francisco, USA, 2003.

%\bb{dodelson/2003} S. Dodelson, {\it Modern Cosmology} Academic Press, Amsterdam, Netherlands, 2003.

\end{thebibliography}







\end{document}



