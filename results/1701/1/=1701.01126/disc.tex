%!TEX root = main.tex

Although many attention-based
models, including our model, 
achieve superior 
results in the textual entailment task,
we can still see the limitations
for this approach.

Despite those sentence pairs that require
more common knowledge to find the entailment relations,
we are more interested in sentences that are difficult
because they involve interesting linguistic properties.

Consider the following two pairs of sentences
that are difficult for current attention-based models:
\begin{enumerate}
\item \begin{itemize}
\item Premise: The boy loves the girl.
\item Hypothesis: The girl loves the boy.
\end{itemize}
Here the only difference between the two sentences
is the order/structure of the words. To handle this problem
the attention-based models should take the reordering
into consideration.
\item 
\begin{itemize}
\item Premise: A stuffed animal on the couch.
\item Hypothesis: An animal on the couch.
\end{itemize}
In this example, every hypothesis word occurs in the premise sentence,
but it is difficult to learn that ``a stuffed animal'' 
is not ``an animal''.
\end{enumerate}

