\documentclass[12pt]{amsart}
\usepackage{amsmath,amssymb,amsthm}
\usepackage{setspace}
\usepackage{moreverb}
\usepackage[dvips]{graphicx}
\setcounter{MaxMatrixCols}{20}

\oddsidemargin=0cm \evensidemargin=0cm
%\topmargin=.6truecm
%\topskip=1truecm
%\parskip=6.5pt
\textwidth 160mm \textheight 210mm



\newcommand{\ignore}[1]{}
\newenvironment{Macaulay2}{
\begin{spacing}{0.8}
%\small
\begin{quote}
\smallskip
%\hrule
\smallskip } { \smallskip
%\hrule
\end{quote}
\end{spacing}
\medskip
}

\newtheorem{theorem}{Theorem}[section]
\newtheorem{lemma}[theorem]{Lemma}
\newtheorem{corollary}[theorem]{Corollary}
\newtheorem{proposition}[theorem]{Proposition}
\newtheorem{algorithm}[theorem]{Algorithm}
\theoremstyle{definition}
\newtheorem{definition}[theorem]{Definition}
\newtheorem{example}[theorem]{Example}
\newtheorem{xca}[theorem]{Exercise}

\theoremstyle{remark}
\newtheorem{remark}[theorem]{Remark}

\numberwithin{equation}{section}

%    Absolute value notation
\newcommand{\abs}[1]{\lvert#1\rvert}

%    Blank box placeholder for figures (to avoid requiring any
%    particular graphics capabilities for printing this document).
\newcommand{\blankbox}[2]{%
  \parbox{\columnwidth}{\centering
%    Set fboxsep to 0 so that the actual size of the box will match the
%    given measurements more closely.
    \setlength{\fboxsep}{0pt}%
    \fbox{\raisebox{0pt}[#2]{\hspace{#1}}}%
  }%
}

\input xy
\xyoption{all}

\newcommand{\p}{\partial}
\newcommand{\grad}{\nabla}
\newcommand{\gr}{\mbox{\rm{gr}}}

\newcommand{\bD}{\mathbb{D}}
\newcommand{\bC}{\mathbb{C}}
\newcommand{\Cp}{{\mathbb{C}}^N}
\newcommand{\bN}{{\mathbb{N}}}
\newcommand{\bZ}{{\mathbb{Z}}}
\newcommand{\bQ}{{\mathbb{Q}}}
\newcommand{\bR}{\mathbb{R}}
\newcommand{\ea}{{\mathbb{A}}^{n}_{k}}
\newcommand{\adp}{k[x_1,\dots ,x_n]}
\newcommand{\ads}{k[[x_1,\dots ,x_n]]}
\newcommand{\ii}{i_1,\dots,i_r}
\newcommand{\xx}{x_1,\dots ,x_n}
\newcommand{\yy}{y_1,\dots ,y_r}
\newcommand{\X}{\rm X}
\newcommand{\Y}{\rm Y}

\newcommand{\B}{\Box}



\newcommand{\rad}{\mbox{\rm{rad}} \,}
\newcommand{\Ann}{\mbox{\rm{Ann}} \,}
\newcommand{\Ass}{\mbox{\rm{Ass}} }
\newcommand{\Min}{\mbox{\rm{Min}} }
\newcommand{\dg}{\mbox{\rm{deg}} \,}
\newcommand{\rk}{\mbox{\rm rk} }
\newcommand{\hlt}{\mathrm {ht\, }}
\newcommand{\bhlt}{\mathrm {bight\, }}
\newcommand{\dm}{\mathrm {dim }}
\newcommand{\su}{\mathrm {sup\, }}
\newcommand{\lL}{\rm length}
\newcommand{\depth}{\mathrm {depth\, }}
\newcommand{\grade}{\mathrm {grade\, }}
\newcommand{\reg}{\mbox{\rm{reg}} \,}
\newcommand{\cd}{\mathrm {cd\, }}
\newcommand{\pd}{\mathrm {pd }}
\newcommand{\id}{\mathrm {id }}
\newcommand{\fd}{\mathrm {fd\, }}

\newcommand{\Supp}{\mbox{\rm{Supp}} }
\newcommand{\supp}{\mbox{\rm{supp}} }
\newcommand{\spec}{\mathrm {Spec\, }}
\newcommand{\Proj}{\mbox{\rm Proj } }
\newcommand{\kr}{\mathrm {Ker\, }}
\newcommand{\ckr}{\mathrm {Coker\, }}
\newcommand{\im}{\mathrm {Im\, }}
\newcommand{\lcm}{\mathrm {lcm\, }}
%\newcommand{\gcd}{\mathrm {gcd\, }}



\newcommand{\Mod}{\mathrm{Mod}}
\newcommand{\End}{\mathrm{End}}
\newcommand{\Aut}{\mathrm{Aut}}
\newcommand{\Hom}{\mbox{\rm{Hom}} }
\newcommand{\Ext}{\mbox{\rm{Ext}} }
\newcommand{\Tor}{\mbox{\rm{Tor}} }
\newcommand{\Sol}{\mbox{\rm{Sol}} \,}

\newcommand{\sol}{\mbox{\mathbb{S}{\rm ol}} \,}
\newcommand{\Log}{\mbox{\rm{log}} \,}
\newcommand{\Perv}{\mbox{\rm{Perv}} \,}
\newcommand{\uHom}{  {\rm {\underline{Hom}}}   }

\newcommand{\uExt}{{\rm {\underline{Ext}}}}

\newcommand{\dli}{{\varinjlim}_{\,P}\,}
\newcommand{\dlii}{{\varinjlim}^{(i)}_{\,P}\,}
\newcommand{\ili}{\varprojlim}
\newcommand{\spos}{\mbox{supp}_{+}\,}



\newcommand{\fM}{{\mathfrak m}}
\newcommand{\fP}{\mathfrak{p}_{\gamma}}
\newcommand{\fp}{\mathfrak{p}}
\newcommand{\fQ}{\mathfrak{q}}
\newcommand{\fR}{\mathfrak{r}}
\newcommand{\cA}{{\mathcal A}}
\newcommand{\cB}{{\mathcal B}}
\newcommand{\cC}{{\mathcal C}}
\newcommand{\cD}{{\mathcal D}}
\newcommand{\cE}{{\mathcal E}}
\newcommand{\cF}{{\mathcal F}}
\newcommand{\cH}{{\mathcal H}}
\newcommand{\cI}{{\mathcal I}}
\newcommand{\cJ}{{\mathcal J}}
\newcommand{\cM}{{\mathcal M}}
\newcommand{\cN}{{\mathcal N}}
\newcommand{\cO}{{\mathcal O}}
\newcommand{\cP}{{\mathcal P}}
\newcommand{\cQ}{{\mathcal Q}}
\newcommand{\cR}{{\mathcal R}}
\newcommand{\cS}{{\mathcal S}}
\newcommand{\de}{{\delta}}
\newcommand{\la}{{\lambda}}
\newcommand{\Ga}{{\Gamma}}
\newcommand{\ga}{{\gamma}}
\newcommand{\Si}{{\Sigma}}

\newcommand{\I}{{I_{\alpha_1}\cap \dots \cap I_{\alpha_m}}}
\newcommand{\lra}{{\longrightarrow}}
\newcommand{\Lra}{{\Longrightarrow}}
\newcommand{\op}{{\oplus}}
\newcommand{\bop}{{\bigoplus}}
\newcommand{\pa}{{\partial}}


\newcommand{\ba}{\mathbf{a}}
\newcommand{\bb}{\mathbf{b}}
\newcommand{\bc}{\mathbf{c}}
\newcommand{\be}{\mathbf{e}}
\newcommand{\gmod}{\operatorname{* mod}}
\newcommand{\Sq}{\operatorname{Sq}}
\newcommand{\relint}{\operatorname{rel-int}}

\newcommand\kk{\Bbbk}
\newcommand\const{\underline{\kk}}
\newcommand\Dcom{\mathcal{D}^\bullet}
\newcommand\cExt{{\mathcal Ext}}
\newcommand\Db{{\mathsf D}^b}
\newcommand\GG{\mathbb G}
\newcommand{\bA}{\mathbf{A}}
\newcommand\LL{\mathbb L}




\newcommand{\Roos}{\mbox{Roos}}
\newcommand{\mc}{\mathcal}
\newcommand{\E}[1]{\mbox{E}_{R}(R/#1)}
\newcommand{\lc}[1]{H^{#1}_{I}(R)}
\newcommand{\chc}[2]{H^{#1}_c(#2)}
\newcommand{\vs}{\vspace{6mm}}
\renewcommand{\labelenumi}{\roman{enumi})}




\usepackage{color}
\definecolor{red}{rgb}{1.00,0.00,0.00}
\newcommand{\phil}[1]{{\color{red} \sf $\star\star$ Philippe: [#1]}}
\newcommand{\josep}[1]{{\color{blue} \sf $\star\star$ Josep: [#1]}}


\begin{document}



\title[Pruned cellular free resolutions of monomial ideals ]{Pruned cellular free resolutions of monomial ideals}

\author[J. \`Alvarez Montaner]{Josep \`Alvarez Montaner}

\address{Departament de Matem\`atiques\\
Universitat Polit\`ecnica de Catalunya, SPAIN} \email{Josep.Alvarez@upc.es}


\author[O. Fern\'andez-Ramos]{Oscar Fern\'andez-Ramos}
%\address{Dipartimento di Matematica,  Universit\`a degli study di Genova, ITALY}
 %\email{fernandez@dima.unige.it}
%\email{caribefresno@gmail.com}

\author[P. Gimenez]{Philippe Gimenez}
\address{Departamento  de \'Algebra,An\'alisis Matem\'atico,  Geometr\'ia y Topolog\'ia \&
Instituto de Investigaci\'on en Matem\'aticas de Valladolid (IMUVA),
Universidad de Valladolid, SPAIN} \email{caribefresno@gmail.com} \email{pgimenez@agt.uva.es}

\thanks{The first author is partially supported by the {\it Generalitat de Catalunya} grant 2014SGR-634 and 
the Spanish {\it Ministerio de Econom\'ia y Competitividad} grant MTM2015-69135-P. He 
is also with the Barcelona Graduate School of Mathematics
(BGSMath).
The third author is partially supported by the Spanish
{\it Ministerio de Econom\'ia y Competitividad} grant
%MTM2013-40775-P and
MTM2016-78881-P}

%\keywords {Betti numbers, Monomial ideals}

%\subjclass[2000]{Primary 13D45, 13N10}



\begin{abstract}
%We use discrete Morse theory to construct new  cellular free resolutions of a monomial ideal
%that are, in general, smaller than the Lyubeznik resolution. We also use our
%methods to give a different approach to the theory of splitting of monomial ideals.
Using discrete Morse theory, we give an algorithm that prunes the excess of information
in the Taylor resolution and constructs a new cellular free resolution for an arbitrary monomial ideal.
The pruned resolution is not simplicial in general, but we can slightly modify our algorithm in order to obtain
a simplicial resolution. We also show that the Lyubeznik resolution fits into our
pruning strategy.
The pruned resolution is not always minimal but it is a lot closer to the minimal resolution 
than the Taylor and the Lyubeznik resolutions as we will
see in some examples.
We finally use our methods to give a different approach to the theory of splitting of monomial ideals.
We deduce from this splitting strategy that the pruned resolution is always minimal in the case
of cycles and paths.

%\medskip
%This submission to MEGA-2017 is a full paper with complete proofs and it contains original results that have not
%been submitted elsewhere. It is an ongoing work, the final version will be submitted for publication soon.
%
\end{abstract}

\maketitle

\section{Introduction}

%\phil{He utilizado este comando para hacer comentarios. Josep, utiliza el siguiente para contestarme porfa}
%\josep{Este}
Let $R=\kk[x_1,\dots,x_n]$ be the polynomial ring over a field $\kk$ and $I\subseteq R$ a monomial ideal.
The study of minimal free resolutions of these ideals has been a very active area of research
during the last decades. There are topological and combinatorial formulae, as those of Hochster \cite{Hoc}
or Gasharov, Peeva and Welker \cite{GPW}, to describe their multigraded Betti numbers but, except for some
specific classes  of monomial ideals (see, e.g., \cite{EK}, \cite{Jac04} or \cite{FG}), the problem of describing a minimal
multigraded free resolution explicitly was shown to be difficult.

\vskip 2mm

Another strategy is to study non-minimal free resolutions.
These reveal less information than minimal free resolutions do but are
often much easier to describe. The most significant ones are  the Taylor resolution \cite{Tay66} and
the Lyubeznik resolution \cite{Lyu88}. An interesting feature of these two
resolutions is that they fit in the theory of simplicial resolutions
introduced by Bayer, Peeva and Sturmfels in \cite{BPS} and further extended
to regular cellular resolutions and CW-resolutions in \cite{BS} and \cite{JW} respectively. The idea behind
these three concepts is to associate to a free resolution of a monomial ideal a
simplicial complex (respectively a regular cell complex, a CW-complex) that carries in
its structure the algebraic structure of the free resolution. It is worth pointing
out that Velasco proved in \cite{Vel} that there exist monomial
ideals whose minimal free resolutions cannot be described by a CW-complex.

\vskip 2mm

By adapting the discrete Morse theory developed by Forman \cite{For} and
Chari \cite{CH},  Batzies and Welker provided in \cite{BW} a method to reduce
a given regular cellular resolution. In particular, they proved that the Lyubeznik
resolution can be obtained in this way from the Taylor resolution.
Let's point out that discrete Morse theory has the inconvenient that it can't be used
iteratively. To overcome this issue, one can use the algebraic discrete Morse theory developed independently by
 Sk\"oldberg \cite{Sko} and J\"ollenbeck and Welker \cite{JW}.
In this work, we use a similar strategy to reduce the Taylor
resolution and obtain cellular and simplicial free resolutions that
are closer to the minimal one than the Lyubeznik
resolution. Essentially, the information given by the Taylor
resolution can be encoded in a directed graph and the obstruction to
its minimality can be observed in some of the edges of this graph.
What we will do is to remove, in a convenient order, some of these
edges to provide a smaller resolution. In some sense, we are pruning
the excess of information given by the Taylor resolution in a simple and efficient way.

\vskip 2mm

The organization of this paper is as follows. In Section \ref{cellular},
we review the notion of  cellular resolution and introduce the basics
on discrete Morse theory that will be needed throughout this work.
In Section \ref{prune}, we present our main results. We first provide an algorithm
(Algorithm \ref{alg1}) that, starting from the Taylor resolution of a monomial ideal,
allows to construct a smaller cellular free resolution
(Theorem \ref{T1} and  Corollary \ref{C1}).
%Essentially we prune the excess of information given by the Taylor resolution in very simple and clean way.
%Unfortunately, the resolution that we obtain is in general non-minimal
The resolution that we obtain is not simplicial in general, but
we can adapt our pruning algorithm to produce a simplicial free resolution (Algorithm \ref{alg3}).
Indeed, the Lyubeznik resolution fits into this pruning strategy as shown in Algorithm \ref{alg2}.
Other variants of our method are also mentioned.

\vskip 2mm

In Section \ref{examples}, we illustrate our results with several examples.
We implemented our algorithms using CoCoALib \cite{cocoa}
for constructing pruned resolutions in the non-trivial examples contained in this section.
%Of course, our results could also be implemented in \cite{Sing} or \cite{GS}.
Finally, in Section \ref{split} we present a connection between our method and the
theory of Betti splittings introduced by Eliahou and Kervaire \cite{EK} and
later developed by Francisco, H\`a and Van Tuyl \cite{FHV}. We provide a sufficient condition
for having a Betti splitting by checking some prunings in our algorithm. We use this
approach to prove that the pruned resolution is minimal for edge ideals associated to paths and cycles.



\section{Cellular resolutions using discrete Morse theory} \label{cellular}

Let $R=\kk[x_1, \ldots, x_n]$ be the polynomial ring in $n$ variables with coefficients
in a field $\kk$. An ideal $I\subseteq R$ is monomial if it may be generated
by monomials ${\bf x}^\alpha:=x_1^{\alpha_1}\cdots x_n^{\alpha_n}$, where $\alpha \in \bZ_{\geq 0}^n$.
As usual, we denote $|\alpha|= {\alpha_1}+\cdots +{\alpha_n}$ and $\varepsilon_1,\dots,\varepsilon_n$
will be the natural basis of $\bZ^n$. Moreover, given a set of generators of a monomial ideal $I$, $\{m_1,\dots,m_r\}$,
we will consider  the monomials $m_{\sigma}:={\rm lcm}(m_i \hskip 2mm | \hskip 2mm \sigma_i=1)$ for any $\sigma \in \{0,1\}^r$.

\vskip 2mm

A $\bZ^n$-graded free resolution of $R/I$ is an exact sequence of free $\bZ^n$-graded
modules:
\begin{equation}\label{resolution of I}
\mathbb{F}_{\bullet}: \hskip 3mm \xymatrix{ 0 \ar[r]& F_{p}
\ar[r]^{d_{n}}& \cdots \ar[r]& F_1 \ar[r]^{d_1}& F_{0} \ar[r]& R/I
\ar[r]& 0},
\end{equation}
where the $i$-th term is of the form
$$F_i =\bigoplus_{\alpha \in \bZ^n} R(-\alpha)^{\beta_{i,\alpha}}\,.$$
We say that $\mathbb{F}_{\bullet}$ is minimal if the matrices of the
homogeneous morphisms $d_i: F_i\longrightarrow F_{i-1}$ do not contain
invertible elements. In this case, the exponents $\beta_{i,\alpha}$ form a
set of invariants of $R/I$ known as its {\it multigraded Betti numbers}.
Throughout this work, we will mainly consider the coarser $\bZ$-graded free resolution. In this case, we
will encode the $\bZ$-graded Betti numbers in the so-called  {\it Betti diagram} of $R/I$ where the entry
on the $i$th row and $j$th column of the table is $ \beta_{i,i+j}$:

\begin{center}
 $\begin{matrix}
&0&1&2&\cdots \\ \text{total:}& \text{.} &\text{.}&\text{.}& \\
\text{0:} & \beta_{0,0} & \beta_{1,1} & \beta_{2,2} & \cdots \\\text{1:}& \beta_{0,1} & \beta_{1,2} & \beta_{2,3} & \cdots \\
\vdots& \vdots & \vdots & \vdots \\\end{matrix}
$
\end{center}

\vskip 2mm

\subsection{Cellular resolutions}

A CW-complex $X$ is a topological space obtained by attaching cells of increasing dimensions
to a discrete %\phil{Es suficiente definir los 'finite CW-complex' no? En nuestro contexto es suficiente}
set of points $X^{(0)}$.  Let $X^{(i)}$ denote the set of $i$-cells of  $X$ and
consider the set of all cells $X^{(\ast)}:=\bigcup_{i\geq 0} X^{(i)}$. Then, we can view $X^{(\ast)}$
as a poset with the partial order  given by $\sigma' \leq \sigma$ if and only if $\sigma'$
is contained in the closure of $\sigma$. We can also give a $\bZ^n$-graded structure to $X$ by means of an
order preserving map $gr: X^{(\ast)} \lra \bZ_{\geq 0}^n$.


%In the sequel we will denote $X_{\leq \alpha}$ the sub-CW-complex of $X$ consisting of the cells $\sigma \in X^{(\ast)}$
%such that $gr(\sigma)\leq \alpha$, for any $\alpha \in \bZ_{\geq 0}^n$.


\vskip 2mm

We say that the free resolution (\ref{resolution of I}) is {\it cellular} (or is a {\it CW-resolution})
if there exists a $\bZ^n$-graded CW-complex $(X, gr)$ such that, for all $i\geq 1$:

\begin{itemize}
 \item[$\cdot$]
 there exists a basis $\{e_\sigma\}$ of $F_i$ indexed by the $(i-1)$-cells of $X$, such that if
 $e_\sigma \in R(-\alpha)^{\beta_{i,\alpha}}$ then  $gr(\sigma)=\alpha$, and


 \item[$\cdot$]
the differential $d_i: F_i\longrightarrow F_{i-1}$ is given
%\phil{for any $e_\sigma \in X^{(i)}$. OJO: $\sigma \in X^{(i)}$ no? Quitado y a\~nadido en la formula siguiente}
by
 $$e_\sigma \hskip 2mm \mapsto \sum_{\sigma \geq \sigma' \in X^{(i-1)}} [\sigma: \sigma'] \hskip 1mm {\bf x}^{gr(\sigma) - gr(\sigma') } \hskip 1mm e_{\sigma'}\ ,\quad \forall \sigma \in X^{(i)}$$
where $[\sigma: \sigma']$ denotes the coefficient of $\sigma'$ in the image of $\sigma$ by the differential map in the cellular homology
of $X$.
\end{itemize}
In the sequel, whenever we want to emphasize such a cellular structure, we will denote the free resolution as
$\mathbb{F}_{\bullet}=\mathbb{F}_{\bullet}^{(X,gr)}$. If $X$ is a simplicial complex, we say
that the free resolution is {\it simplicial}. This is the case for the following two well-known examples.

\vskip 2mm

$\bullet$ {\bf The Taylor resolution}:  The most recurrent example of simplicial free resolution is the Taylor resolution
discovered in \cite{Tay66}. Using the above terminology, we can describe it as follows.
%
%\vskip 2mm
%
Let $I=\langle m_1,\dots, m_r \rangle \subseteq R$ be a monomial ideal.
Consider the full simplicial complex on $r$ vertices, $X_{\tt Taylor}$, whose faces are labelled by
$\sigma \in \{0,1\}^r$ or, equivalently, by the corresponding monomials $m_\sigma$. We have a natural
$\bZ^n$-grading on $X_{\tt Taylor}$ by assigning $gr(\sigma)=\alpha \in \bZ^n$ where ${\bf x}^\alpha=m_\sigma$.
The {\it Taylor resolution} is the simplicial resolution $\mathbb{F}_{\bullet}^{(X_{\tt Taylor}, gr)}$.

\vskip 2mm
%
%\vskip 2mm
%
$\bullet$ {\bf The Lyubeznik resolution}:  Another important example of simplicial resolution is the one considered by Lyubeznik
in \cite{Lyu88}. Let's s start fixing an order $m_1 \leq \cdots \leq m_r$ on a generating set of a monomial ideal $I\subseteq R$.
Consider the simplicial subcomplex $X_{\tt Lyub} \subseteq X_{\tt Taylor}$ whose faces of dimension $s$ are labelled by
those $\sigma=\varepsilon_{i_0}+ \cdots + \varepsilon_{i_s} \in \{0,1\}^r$
such that, for all $t<s$ %\phil{mejor que $i_t<i_s$ no?}
and all $j <i_t$ %\phil{mejor que $m_j < m_{i_t}$ no?},
$$ m_j \not | \hskip 2mm {\rm lcm}(m_{i_t},\dots , m_{i_s}).$$
The {\it Lyubeznik resolution} is the simplicial resolution $\mathbb{F}_{\bullet}^{(X_{\tt Lyub}, gr)}$.

\vskip 2mm

%In general, neither the Taylor nor the Lyubeznik resolution are minimal.


\subsection{Discrete Morse theory}
Forman introduced in  \cite{For} the discrete Morse theory as a method to
reduce the number of cells in a CW-complex without changing its
homotopy type. Batzies and Welker adapted this technique in \cite{BW}
to the study of cellular resolutions; see also \cite{Wel07}.  Indeed, they used the
reformulation of discrete Morse theory in terms of acyclic matchings
given by Chari in \cite{CH} in order to obtain, given a  regular  cellular
resolution (most notably the Taylor resolution), a reduced
 cellular resolution.


\vskip 2mm

This is also our approach in this work. Let's start recalling from \cite{BW} the preliminaries on discrete Morse theory.
Consider the directed graph $G_X$ on the set of cells of a regular $\bZ^n$-graded CW-complex $(X,gr)$ which edges are given by
$$E_X=\{ \sigma \lra \sigma' \hskip 2mm | \hskip 2mm  \sigma' \leq \sigma, \hskip 2mm   \dim \sigma' = \dim \sigma -1 \}.$$
For a given set of edges $\cA \subseteq E_X$, denote by $G_X^{\cA}$ the graph obtained by reversing the direction
of the edges in $\cA$, i.e., the directed graph with edges\footnote{For the sake of clarity, the arrows that we reverse will be denoted
by $\Longrightarrow$.}
$$E_X^{\cA}= (E_X \setminus \cA) \cup \{ \sigma' \Longrightarrow \sigma \hskip 2mm | \hskip 2mm  \sigma \lra \sigma' \in \cA \}.$$
%
%\vskip 2mm
%
When each cell of $X$ occurs in at most one edge of $\cA$, we say that $\cA$ is a {\it matching} on $X$.
A matching $\cA$ is {\it acyclic} if the associated graph $G_X^{\cA}$ is acyclic, i.e., does not
contain any directed cycle.
Given an acyclic matching  $\cA$ on $X$, the $\cA$-{\it critical cells} of $X$ are the cells of $X$ that are
not contained in any edge of $\cA$.
Finally, an acyclic matching $\cA$ is {\it homogeneous} whenever $gr(\sigma)=gr(\sigma')$
for any edge $\sigma \lra \sigma' \in \cA$.
%
%\vskip 2mm
%
%The $\cA$-{\it critical cells} of $X$ are those cells of $X$ not contained in any edge of the acyclic matching $\cA$.

\begin{proposition}[{\cite[Proposition 1.2]{BW}}]
Let $(X,gr)$ be a regular $\bZ^n$-graded CW-complex and $\cA$ a homogeneous acyclic matching. Then, there is a
{\rm(}not necessarily regular{\rm)} CW-complex $X_{\cA}$
whose $i$-cells are in one-to-one correspondence with the $\cA$-critical $i$-cells of $X$, such that $X_{\cA}$ is
homotopically equivalent to $X$, and that inherits the $\bZ^n$-graded structure.
\end{proposition}

In the theory of cellular resolutions, we have the following consequence.

\begin{theorem}[{\cite[Theorem 1.3]{BW}}]
Let $I\subseteq R=\kk[x_1,\dots, x_n]$ be a monomial ideal. Assume that $(X,gr)$ is a regular $\bZ^n$-graded
 CW-complex that defines a cellular resolution $\mathbb{F}_{\bullet}^{(X,gr)}$ of $R/I$. Then,  for
a homogeneous acyclic  matching $\cA$ on $G_X$, the $\bZ^n$-graded CW-complex $(X_{\cA},gr)$ supports a cellular
resolution $\mathbb{F}_{\bullet}^{(X_{\cA},gr)}$ of $R/I$.
\end{theorem}

\vskip 2mm

The differentials of the cellular resolution
$\mathbb{F}_{\bullet}^{(X_{\cA},gr)}$ can be explicitely described
in terms of the  differentials of $\mathbb{F}_{\bullet}^{(X,gr)}$
(see \cite[Lemma 7.7]{BW}).

\subsection{Algebraic discrete Morse theory}

One of the main inconvenients of discrete Morse theory %as described by Batzies and Welker is
is that the CW-complex $(X_{\cA},gr)$ that we obtain for a given homogeneous acyclic matching $\cA$
is not necessarily regular. Therefore, we cannot always iterate the procedure.
To overcome such an obstacle, one may use {\it algebraic discrete Morse theory} developed independently by
 Sk\"oldberg \cite{Sko} and J\"ollenbeck and Welker \cite{JW}.


 \vskip 2mm

This approach works directly with an initial free resolution $\mathbb{F}_{\bullet}$
without paying attention whether it has or not a cellular structure.
Given a basis $X=\bigcup_{i\geq 0} X^{(i)}$ of the corresponding free modules $F_i$,
we may consider the directed graph $G_X$ on the set of basis elements with the corresponding
set of edges $E_X$. Then, we may follow the same constructions as in the previous subsection.
Namely, we may define an acyclic matching $\cA \subseteq E_X$ (see \cite[Definition 2.1]{JW})
but, in this case, we have to make sure that the coefficient $[\sigma : \sigma']$ in the differential
corresponding to an edge $\sigma \lra \sigma' \in \cA$ is a unit. We consider the $\cA$-critical
basis elements $X_\cA$ and construct a free resolution $\mathbb{F}_{\bullet}^{X_\cA}$ that is
homotopically equivalent to $\mathbb{F}_{\bullet}$ (see \cite[Theorem 2.2]{JW}).


\section{Pruning the Taylor resolution} \label{prune}

In \cite{BW}, the Lyubeznik resolution is obtained  from the Taylor resolution through discrete Morse theory
by detecting a suitable homogeneous acyclic matching $\cA$ on the simplicial complex $X_{\tt Taylor}$ such that $X_{\tt Lyub}=X_\cA$.
In this section, we use a similar approach to provide some new cellular free resolutions for
monomial ideals. We should point out that the framework considered in
\cite{BW} is slightly more general. To keep notations as simple as possible,
we decided to stick to the case of monomial ideals in a polynomial ring.
%The interested reader can easily extend the results below to their framework.

\vskip 2mm

\subsection{A cellular free resolution} \label{pruned1}

Let $I=\langle m_1,\dots, m_r \rangle \subseteq R$ be a monomial ideal.
Our starting point is the Taylor resolution $\mathbb{F}_{\bullet}^{(X_{\tt Taylor}, gr)}$.
This resolution is, in general, far from being minimal. In other words, the directed graph $G_{X_{\tt Taylor}}$ associated to ${X_{\tt Taylor}}$
contains a lot of unnecessary information.  Our goal is to prune  this
excess of information in a very simple way. More precisely, we give an algorithm that produces a homogeneous
acyclic matching $\cA_P$ on  ${X_{\tt Taylor}}$. Using
discrete Morse theory, this will provide a cellular free resolution of $R/I$.
It will not be minimal in general, but it will be smaller than the Lyubeznik resolution.
%as we will observe in some examples in the next section.

\vskip 2mm

\begin{algorithm}\label{alg1} { (Pruned resolution)}

\vskip 2mm

{\rm \noindent {\sc Input:} The set of edges $E_{X_{\tt Taylor}}$.

%\noindent {\sc Output:} The set of pruned edges $\cA_P \subseteq E_{X_{\tt Taylor}}$ \phil{por que poner 'OUTPUT' y luego 'RETURN'
%con la misma informacion en todos los algoritmos?}

\vskip 2mm

%{\bf Prune the extra components}

%\vskip 2mm

For $j$ from $1$ to $r$, incrementing by $1$

\vskip 2mm

\begin{itemize}

\item[\textbf{(j)}] ${\it Prune}$ the edge ${\sigma} \lra{\sigma + \varepsilon_j}$ for
all $\sigma \in \{0,1\}^r$ such that $\sigma_j=0$, where `prune'
means remove the edge\footnote{When we remove an edge, we also remove its two vertices and all the edges
passing through these two vertices.}
if it survived after step $(j-1)$ and
$gr(\sigma)=gr(\sigma + \varepsilon_j) $.
% and this edge survived to all the previous steps.
% $(j-1)$ \phil{no quedaria mas claro decir "if it survived at steps $1$ to $(j-1)$"?} and $gr(\sigma)=gr(\sigma + \varepsilon_j) $.

\end{itemize}

\vskip 2mm


\noindent {\sc Return:} The set $\cA_P$ of edges that have been pruned.

}

\end{algorithm}

\begin{example}\label{ex3gens}
 For the case $r=3$ we can visualize
 the steps of the algorithm  over the directed graph as follows:

 \hskip -2cm{\tiny
$${\xymatrix { &(1,1,1) \ar@2{->}[dr] &
\\ (1,1,0) \ar@2{->}[dr] \ar[ur]  & (1,0,1) \ar@2{->}[dr]|\hole \ar[u] & (0,1,1)
\\ (1,0,0) \ar[ur]|\hole \ar[u]&(0,1,0)  \ar[ur]& (0,0,1)  \ar[u]}}
%\\ & M \ar[ul] \ar@{.>}[u] \ar@{.>}[ur]& }}
\hskip .41cm
{\xymatrix { &(1,1,1) \ar@2{->}[d] &
\\ (1,1,0)\ar@2{->}[d] \ar[ur]  & (1,0,1)  & (0,1,1) \ar[ul] \ar@2{->}[d]
\\ (1,0,0) \ar[ur]|\hole &(0,1,0)  \ar[ul] \ar[ur]& (0,0,1) \ar[ul]|\hole }}
%\\& M \ar@{.>}[ul] \ar[u] \ar@{.>}[ur]& }}
\hskip .41cm
{\xymatrix { &(1,1,1)\ar@2{->}[dl] &
\\ (1,1,0)   & (1,0,1) \ar[u] \ar@2{->}[dl]|\hole & (0,1,1) \ar[ul] \ar@2{->}[dl]
\\ (1,0,0)  \ar[u]&(0,1,0)  \ar[ul] &(0,0,1) \ar[ul]|\hole \ar[u]}}
% \\& M \ar@{.>}[ul] \ar@{.>}[u] \ar[ur]& }}
$$}


The double arrows indicate the direction of the pruning step,
 that is, the arrows that will be pruned at each step  if the degree of their two vertices coincide
(and if they have not been pruned at a previous step).
\end{example}

The main result in this section is the following:

\begin{theorem} \label{T1} Let $\cA_P \subseteq E_{X_{\tt Taylor}}$ be the set of pruned edges obtained
using Algorithm \ref{alg1}.
Then $\cA_P$ is a homogeneous acyclic matching on $X_{\tt Taylor}$.
\end{theorem}

As a consequence, we get our desired cellular free resolution.

\begin{corollary}\label{C1}
Let $I \subseteq R=\kk[x_1, \ldots, x_n]$ be a monomial ideal and
$\cA_P \subseteq E_{X_{\tt Taylor}}$ be the set of pruned edges obtained using Algorithm \ref{alg1}.
Then, the $\bZ^n$-graded CW-complex $(X_{\cA_P},gr)$
supports a cellular free resolution $\mathbb{F}_{\bullet}^{(X_{\cA_P},gr)}$ of $R/I$.
\end{corollary}

%\phil{PREGUNTAS IMPORTANTES:
%\begin{itemize}
%\item
%La resolucion pruned que obtenemos es siempre regular? (is $X_{\cA_P}$ always a regular cellular complex?)
%Esto es importante, creo, en la parte de splitting para poder aplicar el pruning algorithm (cuya entrada
%tiene que ser un complejo regular) a $X'$ por ejemplo (que no tiene porque ser simplicial
%tal como se observa alli, pero >es siempre regular?).
%\item
%> Se puede hacer un pruning parcial? Es decir, >podemos dejar en algun paso del algoritmo, una arista que se podria podar? Es lo que
%hacemos en la version simplicial del pruning. Tambien creo que es importante poder hacerlo en la parte del splitting
%(ver comentarios alli).
%\end{itemize}}

\begin{remark}
The free resolution $\mathbb{F}_{\bullet}^{(X_{\cA_P},gr)}$, like the Lyubeznik resolution, strongly depends on the order of the
generators of the monomial ideal $I$. % as we will see with some examples in Section~\ref{examples}.
In general, it is neither simplicial (while the Lyubeznik resolution is always simplicial) nor minimal.
\end{remark}

The proof of Theorem \ref{T1} follows closely the one given in
\cite[\S3]{BW} to show that the Lyubeznik resolution can be obtained
using discrete Morse theory. However, we will need some reformulation of their preliminary results.

\vskip 2mm

%\phil{PARA MI: Todavia no me ha dado tiempo a revisar la prueba que dejo en ROJO para no olvidarme.
%Revisar desde aqui hasta el final de la prueba del Teorema 3.3

Let's start with some notations. Given  $\sigma \in \{0,1\}^r$, we define:

\vskip 2mm

\begin{itemize}
 \item[$\cdot$] $t(\sigma):= {\rm min}\left\{ \sigma' \leq \sigma \hskip 2mm | \hskip 2mm \exists \hskip 2mm 1\leq i \leq r
 \hskip 2mm {\rm s.t.} \hskip 2mm  \sigma'_i=0 \hskip 2mm {\rm and} \hskip 2mm gr(\sigma')=gr(\sigma' + \varepsilon_i) \right\};$

 \vskip 2mm

 \item[$\cdot$] $j(\sigma):= {\rm min}\left\{ i ,\ 1\leq i\leq r\hskip 2mm | \hskip 2mm  t(\sigma)_i=0 \hskip 2mm {\rm and} \hskip 2mm
 gr(t(\sigma))=gr(t(\sigma) + \varepsilon_i) \right\}.$
\end{itemize}

\vskip 3mm

Of course, $gr(t(\sigma))=gr(t(\sigma) + \varepsilon_i)$  implies $gr(\sigma)=gr(\sigma + \varepsilon_i)$. It follows that $j(\sigma)$
is the step in the algorithm where the edge containing $\sigma$, namely $\sigma \lra \sigma + \varepsilon_{j(\sigma)}$, is pruned.
 Note that we may  have $t(\sigma)=\sigma$.
 For simplicity, we will set $t(\sigma)=0$ when no edge containing $\sigma$ has been pruned
 in the algorithm.


 \vskip 4mm

The following properties will be useful in the proof of Theorem \ref{T1}:

\vskip 2mm

 \begin{itemize}
 \item[i)] If $gr(\sigma)=gr(\sigma + \varepsilon_k)$ for some $k$ s.t. $\sigma_k=0$, then $t(\sigma  + \varepsilon_k) \geq t(\sigma)$.

 \vskip 2mm


 \begin{proof} The result obviously holds if $t(\sigma)=0$. When $t(\sigma)\neq 0$, 
% It  is clear  by definition that the inequality $t(\sigma  + \varepsilon_k) < t(\sigma)$ is not possible. 
 the result follows
 from the fact that $gr(t(\sigma) )=gr(t(\sigma) + \varepsilon_{j(\sigma)})$ implies
 $gr(t(\sigma)+ \varepsilon_k )=gr(t(\sigma) + \varepsilon_k + \varepsilon_{j(\sigma)})$.
 \end{proof}


 \vskip 2mm

 \item[ii)] If $gr(\sigma)=gr(\sigma + \varepsilon_k)$  and  $t(\sigma  + \varepsilon_k) = t(\sigma) \neq 0$
 for some $k$ s.t. $\sigma_k=0$,
 then $j(\sigma  + \varepsilon_k) = j(\sigma)$.

  \vskip 2mm


 \begin{proof}
 We have $gr(t(\sigma) )=gr(t(\sigma) + \varepsilon_{j(\sigma)})$ so it follows that
 $$gr(t(\sigma + \varepsilon_k) )=gr(t(\sigma + \varepsilon_k) + \varepsilon_{j(\sigma)})$$ and therefore
 $j(\sigma  + \varepsilon_k) \geq j(\sigma)$.  On the other hand, we also have the equality
 $gr(t(\sigma + \varepsilon_k) )=gr(t(\sigma) + \varepsilon_{j(\sigma + \varepsilon_k)})$
 so  $$gr(t(\sigma) )=gr(t(\sigma ) + \varepsilon_{j(\sigma + \varepsilon_k)})$$ and the reverse inequality
 $j(\sigma  + \varepsilon_k) \leq j(\sigma)$ follows.
 \end{proof}

  \vskip 2mm

 \item[iii)] If $t(\sigma) \neq 0$, then
 $t(\sigma  - \varepsilon_{j(\sigma)}) = t(\sigma)= t(\sigma  + \varepsilon_{j(\sigma)})$.

   \vskip 2mm


 \begin{proof}
Assume that $ \varepsilon_{j(\sigma)} \leq \sigma$. Then we have $t(\sigma  - \varepsilon_{j(\sigma)}) \leq t(\sigma)$
by definition. To prove the reverse inequality we only have to notice that  $ \varepsilon_{j(\sigma)} \not\leq t(\sigma)$ and
equivalently $ t(\sigma) \leq \sigma - \varepsilon_{j(\sigma)} < \sigma$. Using similar arguments we also obtain
$t(\sigma)= t(\sigma  + \varepsilon_{j(\sigma)})$ when $ \varepsilon_{j(\sigma)} \not\leq \sigma$.
 \end{proof}


\end{itemize}

\begin{proof}{\it of Theorem \ref{T1}.} \hskip 2mm It is clear from its construction that the set $\cA_P$ is a matching
since we are removing edges (or pairs of cells) at each step of the algorithm. To check that it is acyclic, assume that we
have a cycle%\footnote{Por ser un matching las flechas deben ir alternando.}
$$\sigma_0 \Longrightarrow \tau_0 \lra \sigma_1 \Longrightarrow  \tau_1 \lra \cdots \Longrightarrow \tau_{k-1} \lra \sigma_k = \sigma_0\,.$$
Then $gr(\sigma_0)= gr(\tau_0) \leq gr(\sigma_1)=  \cdots = gr(\tau_{k-1})\leq gr(\sigma_0)$, and hence equality must hold
everywhere. Let's see that we also have
$j(\sigma_0)= j(\tau_0) =  \cdots = j(\tau_{k-1})=j(\sigma_0)$.

\vskip 2mm

We start considering the case of a reversed arrow
$\sigma \Longrightarrow \tau$, where $\sigma= \tau + \varepsilon_{j(\tau)}$. We have $0\neq t(\tau)= t(\tau + \varepsilon_{j(\tau)})$
by iii) so, by ii), we also have $j(\tau)=j(\tau + \varepsilon_{j(\tau)})=j(\sigma)$.

\vskip 2mm

For a direct arrow $\tau \lra \sigma$ we have $\tau=\sigma - \varepsilon_k$ for some $k$.
We have  $$0\neq t(\sigma - \varepsilon_k)\leq t(\sigma)\leq \sigma$$ were the first inequality is due to the fact that the
vertex $\tau$ must be pruned in the next link of the chain. The second inequality comes from i) and it can not be
strict because of the definition of $t(\sigma)$. Therefore $t(\sigma - \varepsilon_k)= t(\sigma)$ and we have
$j(\sigma - \varepsilon_k)= j(\sigma)$ by ii), that is, $j(\tau)= j(\sigma)$.

\vskip 2mm

Since all the elements in the cycle have the same $j=j(\sigma_0)= \cdots =j(\tau_{k-1})$, we deduce that such cycle
cannot exist due to the fact that we can only go in the direction of $\varepsilon_j$ when constructing the edges of the cycle.
At the end, we would have an edge with both directions and we get a contradiction.

\vskip 2mm

This shows that  $\cA_P$ is an acyclic matching on $X_{\tt Taylor}$, and it is homogeneous by construction of
 $\cA_P$ in our pruning algorithm.
\end{proof}

%}




A nice feature about the pruning algorithm is that
%\phil{Josep, revisa por favor el siguiente cambio: he cambiado 'we may always start with a minimal system of generators' por lo siguiente}
we do not need to care if the original system of generators of the monomial ideal
is minimal or not.
Roughly speaking, the pruning algorithm will always remove the
excess of information given by the extra generators.

\vskip 2mm

\begin{lemma}\label{minimal}
Let $I =\langle m_1,\dots, m_s \rangle \subseteq R$ be a monomial
ideal, and let $\{m_{i_1},\dots, m_{i_r} \}$ be its minimal set of generators
with $1\leq {i_1} < \cdots < {i_r} \leq s$.
Let $X_s$ and $X_r$
be the Taylor simplicial complexes associated to these two sets of
generators\footnote{We have that $X_r$ is a
subcomplex of $X_s$.} and  $\cA_P^s \subseteq E_{X_s}$, $\cA_P^r \subseteq
E_{X_r}$ be  the sets of pruned edges obtained by applying Algorithm
\ref{alg1} to each case. Then, there is an isomorphism of
$\bZ^n$-graded CW-complexes $(X_{\cA_P^r},gr) \cong
(X_{\cA_P^s},gr)$. In particular, both sets of generators lead to
the same cellular free resolution of $R/I$ in Corollary \ref{C1}.
\end{lemma}

\begin{proof}

Assume that $m_j$ % $j\in \{1,\dots,s \}$
is a generator of $I$ that does not belong to the minimal generating set $\{m_{i_1},\dots,
m_{i_r} \}$. We want to check that all the vertices
$\sigma=(\sigma_1,\dots,\sigma_s)$ with $\sigma_j=1$ in the graph
associated to the Taylor complex $X_s$ are pruned using Algorithm
\ref{alg1}.

\vskip 2mm

There exists a minimal generator $m_{i_k}$ such that $m_{i_k} |
m_j$. Therefore, at the step $(i_k)$ of Algorithm \ref{alg1}, we
prune all the edges $\sigma \lra \sigma + \varepsilon_{i_k}$
with $\sigma_j=1$ and $\sigma_{i_k}=0$
%$$\sigma=(\sigma_1,\dots, \overbrace{0}^{i_k} , \dots , \overbrace{1}^{j} , \dots, \sigma_s)$$
%$$\sigma + \varepsilon_{i_k}=(\sigma_1,\dots, \underbrace{1}_{i_k} , \dots , \underbrace{1}_{j} , \dots, \sigma_s),$$
that have survived in the previous steps. But if we have pruned,
at a previous step $(\ell)$ of the algorithm, the edge $\sigma
+\varepsilon_{i_k} \lra \sigma + \varepsilon_\ell
+\varepsilon_{i_k}$, then we have also pruned the edge $\sigma  \lra
\sigma + \varepsilon_\ell $ at the same step $(\ell)$ and hence neither the vertex $\sigma$ nor the vertex $\sigma + \varepsilon_{i_k}$
survived at the step $(\ell)$ of the algorithm.
%We conclude that all the extra information given by the generator
%$m_j$ is pruned using Algorithm \ref{alg1}.
%\phil{Revisar que la reformulacion de la prueba es correcta}
\end{proof}




\subsection{A simplicial free resolution} \label{simplicial1}
The cellular complex $X_{\cA_P}$ in Theorem \ref{T1} may not be simplicial, that is, it may not satisfy the property that
 given $\tau \in X_{\cA_P}$ then  $\sigma\in X_{\cA_P}$ for any $\sigma \leq \tau$.
In other words,
the pruned free resolution $\mathbb{F}_{\bullet}^{(X_{\cA_P},gr)}$
obtained in Corollary \ref{C1} is cellular but it may not be simplicial.
If we choose carefully the edges that we prune in  Algorithm
\ref{alg1} in order to preserve this property,  we will obtain a
simplicial free resolution of $R/I$ that, in general, will be bigger than the one in Corollary \ref{C1}.
%\phil{Revisar el parrafo anterior}

\vskip 2mm

\begin{algorithm}\label{alg3} { (Simplicial pruned resolution)}
%\phil{Si se han reformulado cosas antes en Algo \ref{alg1}, hay que cambiarlas aqui tambien}
\vskip 2mm

{\rm \noindent {\sc Input:} The set of edges $E_{X_{\tt Taylor}}$.

%\noindent {\sc Output:} The set of pruned edges $\cA_S \subseteq E_{X_{\tt Taylor}}$.

\vskip 2mm

%{\bf Prune the extra components}

%\vskip 2mm

For $j$ from $1$ to $r$, incrementing by $1$

\vskip 2mm

\begin{itemize}

\item[\textbf{(j)}] ${\it Prune}$ the edge ${\sigma} \lra{\sigma + \varepsilon_j}$ for
all $\sigma \in \{0,1\}^r$ such that $\sigma_i=0$ , where `prune'
means remove the edge if it survived after step $(j-1)$,
$gr(\sigma)=gr(\sigma + \varepsilon_j) $ and no
%\phil{sustitui '$\tau \geq \sigma$ survives after this step $(j)$' por}
face $\tau > \sigma$ survives at this step $(j)$.

\end{itemize}}

\vskip 2mm


\noindent {\sc Return:} The set $\cA_S$ of edges that have been pruned.

\end{algorithm}

We have that $\cA_S$ is an acyclic matching on  $X_{\tt Taylor}$
and the corresponding free resolution
$\mathbb{F}_{\bullet}^{(X_{\cA_S},gr)}$ is a simplicial free
resolution of the monomial ideal $I$.

\begin{remark}\label{rkIterate}
The following situation could happen: an edge ${\sigma} \lra{\sigma + \varepsilon_j}$,
that would be pruned at step $(j)$ of Algorithm \ref{alg1}, may not be pruned at step $(j)$ of Algorithm \ref{alg3}
because one face $\tau > \sigma$ survives at this step. But $\tau$ may be
pruned in a posterior step. In this case,
due to the fact that $X_{\cA_S}$ is simplicial,
we may apply Algorithm \ref{alg3} once again, with the set of edges $E_{X_{\cA_S}}$ as input intead of $E_{X_{\tt Taylor}}$.
The edge ${\sigma} \lra{\sigma + \varepsilon_j}$ will be pruned during this second iteration
and we will get a smaller simplicial resolution.
We will observe this phenomenon later in Example \ref{ex5path5cycle}
contructing the simplicial pruned resolution of the 5-cycle.

\end{remark}


\subsection{The Lyubeznik resolution revisited}

The Lyubeznik resolution can be also obtained from the Taylor resolution using our pruning algorithm.
In this case, the edges of the Taylor complex  $X_{\tt Taylor}$ that we prune are obtained
using the following:

\begin{algorithm}\label{alg2} { (The Lyubeznik resolution via pruning)}

\vskip 2mm

{\rm \noindent {\sc Input:} The set of edges $E_{X_{\tt Taylor}}$.

%\noindent {\sc Output:} The set of pruned edges $\cA_L \subseteq E_{X_{\tt Taylor}}$.

\vskip 2mm

%{\bf Prune the extra components}

%\vskip 2mm

For $j$ from $1$ to $r$, incrementing by $1$

\vskip 2mm

\begin{itemize}

\item[\textbf{(j)}] ${\it Prune}$ the edge ${\sigma} \lra{\sigma + \varepsilon_j}$ for
all $\sigma \in \{0,1\}^r$ such that $\sigma_i=0$ for all $i\leq j$,
where `prune' means remove the edge if it survived after step $(j-1)$
and $gr(\sigma)=gr(\sigma + \varepsilon_j) $.

\end{itemize}}

\vskip 2mm


\noindent {\sc Return:} The set $\cA_L$ of edges that have been pruned.
\end{algorithm}

We have that $\cA_L$ is a homogeneous acyclic matching on $X_{\tt Taylor}$
and that $X_{\cA_L}=X_{\tt Lyub}$. Moreover, the Lyubeznik resolution is
simplicial because the pruning that we consider in this case
preserves the property of being simplicial. In this sense, we may
understand the pruned resolution given in Corollary \ref{C1} and the
simplicial pruned resolution given in Subsection \ref{simplicial1} as a refinement of the Lyubeznik resolution.
Finally, we mention that generalizations of Lyubeznik resolutions have been studied by Novik in \cite{Nov}
using a completely different approach.


\subsection{Some variants of the pruning algorithm} \label{versions}
The methods developed in this work can be extended in several different directions.
The aim of this subsection is to present some of them.


\vskip 2mm


$\cdot$ {\it General setup:} Batzies and Welker \cite{BW} use a slightly more general framework
for discrete Morse theory than the one considered in this paper. The interested reader
should be able to adapt our pruning algorithm to their framework.

\vskip 2mm


$\cdot$ {\it Partial pruning:} Notice that the acyclic matchings considered in Algorithms \ref{alg1}, \ref{alg3}
and \ref{alg2} satisfy $\cA_P \supseteq \cA_S \supseteq \cA_L.$ One may also consider any convenient subset
$\cA_P \supseteq \cA'$ of pruning edges. Indeed, we may consider
many different variants using algebraic discrete Morse theory iteratively.
We only have to  pick a convenient edge at each iteration.


\vskip 2mm


$\cdot$ {\it Pruning other resolutions:} The method that we present here always starts with the Taylor resolution but
 we may start with other non-minimal free resolutions, the Lyubeznik resolution for example.
The advantage of the Taylor resolution is that it does not depend on the order of the generators, and
the simplicity of its construction makes our algorithm very easy to present and implement.


\vskip 2mm




$\cdot${\it $\nu$-invariants:}  A new set of invariants that measure the acyclicity of the linear strands of a minimal free
resolution of a graded ideal was introduced in \cite{AY}.  We may obtain an approximation to these invariants by applying
the following iteration of the pruning algorithm. We first apply Algorithm \ref{alg1} to obtain the
CW-complex $X_{\cA_P}$ and its corresponding free resolution. Then, we apply  the pruning algorithm to
$X_{\cA_P}$ with the following variant:


\vskip 2mm

\begin{itemize}

\item[\textbf{(j)}] ${\it Prune}$ the edge ${\sigma} \lra{\sigma + \varepsilon_j}$ for
all $\sigma \in \{0,1\}^r$ such that $\sigma_j=0$, where `prune'
means remove the edge if it survived after step $(j-1)$ and
$gr(\sigma)=gr(\sigma + \varepsilon_j) -1 $.
% and this edge survived to all the previous steps.
Here $gr(\sigma)=|\alpha| \in \bZ$ where ${\bf x}^\alpha=m_\sigma$.
% $(j-1)$ \phil{no quedaria mas claro decir "if it survived at steps $1$ to $(j-1)$"?} and $gr(\sigma)=gr(\sigma + \varepsilon_j) $.

\end{itemize}



\section{Examples} \label{examples}

In this section, we will illustrate that the
pruned free resolution described in Section \ref{pruned1} is fairly
close to the minimal free resolution in some examples.
Indeed, we will compare the pruned resolution to the
simplicial free resolution obtained in Section \ref{simplicial1} and to the
Lyubeznik resolution by means of their corresponding Betti diagram.

\vskip 2mm

\subsection{First examples}

Already for some simple examples we can appreciate the better behavior of the pruned
resolutions with respect to the Lyubeznik resolution.

\begin{example}\label{ex5path5cycle}
We are going to describe the steps of the pruning Algorithms described in Section \ref{prune}
for the edge ideals associated to a $5$-path and a $5$-cycle:
%

%

\vskip 2mm

$\bullet$ Let  $I=({x}_{1} {x}_{2}, {x}_{2} {x}_{3}, {x}_{3} {x}_{4}, {x}_{4} {x}_{5})$ be the edge ideal
of a $5$-path. The steps performed with Algorithm \ref{alg1} are the following:

\vskip 2mm

\noindent $\cdot$ {\bf Step (1):} No edge is pruned.


\noindent $\cdot$ {\bf Step (2):} We prune the edges $(1,0,1,0) \Leftarrow (1,1,1,0)$ and $(1,0,1,1) \Leftarrow (1,1,1,1)$.

%These two edges are also pruned using Algorithm \ref{alg3} but are not using Algorithm \ref{alg2}.

\noindent $\cdot$ {\bf Step (3):} We prune the edge $(0,1,0,1) \Leftarrow (0,1,1,1)$.

%This edge is not be pruned neither using Algorithm \ref{alg3} nor Algorithm \ref{alg2}.

\noindent $\cdot$ {\bf Step (4):} No edge is pruned.


\vskip 2mm

The two edges pruned in Step (2) are also pruned using Algorithm \ref{alg3} but are
not pruned using Algorithm \ref{alg2}. The edge pruned in Step (3) is not be pruned neither
using Algorithm \ref{alg3} nor Algorithm \ref{alg2}. Therefore, the Betti diagrams of the pruned, the simplicial pruned,
and the Lyubeznik resolutions
are  respectively:


\vskip 2mm

\begin{center}
%\hskip 1cm
{\small $\begin{matrix}
&0&1&2&3\\\text{total:}&1&4&4&1\\\text{0:}&1&\text{.}&\text{.}&\text{.}\\\text{1:}&\text{.}&4&3&\text{.}\\\text{2:}&\text{.}&\text{.}&1&1\\\end{matrix}
$
%\hskip 3cm
\hfill
$\begin{matrix}
&0&1&2&3\\
\text{total:}&1&4&5&2\\
\text{0:}&1&\text{.}&\text{.}&\text{.}\\
\text{1:}&\text{.}&4&3&1\\
\text{2:}&\text{.}&\text{.}&2&1\\
\end{matrix}$
\hfill
$\begin{matrix}
&0&1&2&3&4\\
\text{total:}&1&4&6&4&1\\
\text{0:}&1&\text{.}&\text{.}&\text{.}&\text{.}\\
\text{1:}&\text{.}&4&3&2&1\\
\text{2:}&\text{.}&\text{.}&3&2&\text{.}\\
\end{matrix}$
}
\hskip 1cm
\end{center}
%
%\noindent
The pruned resolution is minimal. The simplicial pruned resolution is not minimal but it is as small
as possible since the ideal $I$ has no minimal simplicial resolution as one can check using an argument
similar to the one used in \cite[Section 3.2]{Jac04} for the 4-cycle. The Lyubeznik resolution coincides with the Taylor
resolution since no edge has been pruned through Algorithm \ref{alg2}.

\vskip 2mm

$\bullet$  Let $I=({x}_{1} {x}_{2}, {x}_{2} {x}_{3}, {x}_{3} {x}_{4}, {x}_{4} {x}_{5}, {x}_{5} {x}_{1})$
be the edge ideal of a $5$-cycle. The steps performed with Algorithm \ref{alg1} are the following:


\vskip 2mm

\noindent $\cdot$ {\bf Step (1):} We prune four edges $(0,1,0,0,1) \Leftarrow (1,1,0,0,1)$,
$(0,1,1,0,1) \Leftarrow (1,1,1,0,1)$, $(0,1,0,1,1) \Leftarrow (1,1,0,1,1)$ and $(0,1,1,1,1) \Leftarrow (1,1,1,1,1)$.


\noindent $\cdot$ {\bf Step (2):} We prune  $(1,0,1,0,0) \Leftarrow (1,1,1,0,0)$ and $(1,0,1,1,0) \Leftarrow (1,1,1,1,0)$.



\noindent $\cdot$ {\bf Step (3):} We prune the edge $(0,1,0,1,0) \Leftarrow (0,1,1,1,0)$.



\noindent $\cdot$ {\bf Step (4):}  We prune  $(0,0,1,0,1) \Leftarrow (0,0,1,1,1)$ and $(1,0,1,0,1) \Leftarrow (1,0,1,1,1)$.

\noindent $\cdot$ {\bf Step (5):} We prune the edge $(1,0,0,1,0) \Leftarrow (1,0,0,1,1)$.

\vskip 2mm



The edges pruned at Step (1) are also pruned using Algorithm \ref{alg3} and Algorithm \ref{alg2}.
The two edges pruned at Step (2) can not be pruned  using Algorithm \ref{alg3} because the vertices
$(1,0,1,0,1)$ and $(1,0,1,1,1)$ remain at this step of the algorithm. The edges pruned at Step (3)
and (4) are also pruned in Algorithm \ref{alg3}. The edge pruned at Step (5) is not pruned in Algorithm \ref{alg3}.
Notice that the two edges that prohibited the pruning at Step (3) of Algorithm \ref{alg3} were pruned at Step (4).
This means that we can perform another round of Algorithm \ref{alg3} that will prune the two edges that could not be pruned
before at step (2) as observed in Remark \ref{rkIterate}.

\vskip 2mm

On the other hand, note that after the first step, no more edge can be pruned through Algorithm \ref{alg2}.
We thus obtain that the Betti diagrams of the pruned, the simplicial pruned, and the Lyubeznik resolutions are
the following respectively:

\vskip 2mm

\begin{center}
%\hskip 1cm
{\small
$\begin{matrix}
&0&1&2&3\\\text{total:}&1&5&5& 1\\\text{0:}&1&\text{.}&\text{.}&\text{.}\\\text{1:}&\text{.}&5&5&\text{.}\\\text{2:}&\text{.}&\text{.}&\text{.}&1\\\end{matrix}
$
%\hskip 3cm
\hfill
$\begin{matrix}
&0&1&2&3\\
\text{total:}&1&5&6&2\\
\text{0:}&1&\text{.}&\text{.}&\text{.}\\
\text{1:}&\text{.}&5&5&1\\
\text{2:}&\text{.}&\text{.}&1&1\\
\end{matrix}$
\hfill
$\begin{matrix}
&0&1&2&3&4\\
\text{total:}&1&5&9&7&2\\
\text{0:}&1&\text{.}&\text{.}&\text{.}&\text{.}\\
\text{1:}&\text{.}&5&5&4&2\\
\text{2:}&\text{.}&\text{.}&4&3&\text{.}\\
\end{matrix}$
}
\hskip 1cm
\end{center}
As in the previous example, the pruned resolution is minimal and the simplicial pruned resolution is not minimal but it is as small as possible.
%
%\vskip 2mm
%
%$\bullet$  $I=($${x}_{1} {x}_{2}$, ${x}_{2} {x}_{3}$, ${x}_{3} {x}_{4}$, ${x}_{4} {x}_{1}$, ${x}_{5} {x}_{1}$,
%${x}_{5} {x}_{2}$, ${x}_{5} {x}_{3}$,${x}_{5} {x}_{4}$$)$.
%
\end{example}

%\phil{begin add}
\begin{remark}
It is not surprising that in the previous examples we could find a cellular minimal resolution since both
 the 5-path and the 5-cycle fall into \cite[Example 1.7]{BS}.
\end{remark}
%\phil{end add}

\vskip 2mm

For larger examples the difference  between the pruned and the Luybeznik resolutions
becomes more apparent.

\vskip 2mm

\begin{example}
Consider the ideal %\phil{que quiere decir lo siguiente?} with non rigid strands in the Betti diagram
$
I=({x}_{1}^{4}, {x}_{2}^{4}, {x}_{2}^{2} {x}_{3}^{2}, {x}_{3}^{4}, {x}_{4}^{4}, {x}_{1} {x}_{4}^{2} {x}_{5},
{x}_{5}^{4}, {x}_{2}^{2} {x}_{6}^{2}, {x}_{6}^{4}, {x}_{4}^{2} {x}_{7}^{2}, {x}_{7}^{4}).
$
%
%\vskip 2mm
%
%\noindent
%\phil{A la vista del ejemplo anterior, no hablaria aqui de la simplicial pruned (salvo si tenemos ganas de calcularla a mano)}
%In this example (and in the following one), we use an implementation of our results in CoCoALib \cite{cocoa} for constructing pruned resolutions.
%We are currently working on the implementation of our results in \cite{Sing}. Of course, they could also be implemented in \cite{GS}.
The pruned resolution is minimal while the Lyubeznik resolution is not. The Betti diagrams are:

\vskip 2mm

\begin{center}
{\tiny $\begin{matrix}
&0&1&2&3&4&5&6&7\\\text{total:}&1&11&49&114&148&107&40&6\\\text{0:}&1&\text{.}&\text{.}&\text{.}&\text{.}&\text{.}&\text{.}&\text{.}\\\text{1:}&\text{.}&\text{.}&\text{.}&\text{.}&\text{.}&\text{.}&\text{.}&\text{.}\\\text{2:}&\text{.}&\text{.}&\text{.}&\text{.}&\text{.}&\text{.}&\text{.}&\text{.}\\\text{3:}&\text{.}&11&\text{.}&\text{.}&\text{.}&\text{.}&\text{.}&\text{.}\\\text{4:}&\text{.}&\text{.}&9&\text{.}&\text{.}&\text{.}&\text{.}&\text{.}\\\text{5:}&\text{.}&\text{.}&2&2&\text{.}&\text{.}&\text{.}&\text{.}\\\text{6:}&\text{.}&\text{.}&38&4&\text{.}&\text{.}&\text{.}&\text{.}\\\text{7:}&\text{.}&\text{.}&\text{.}&56&2&\text{.}&\text{.}&\text{.}\\\text{8:}&\text{.}&\text{.}&\text{.}&10&31&\text{.}&\text{.}&\text{.}\\\text{9:}&\text{.}&\text{.}&\text{.}&42&30&9&\text{.}&\text{.}\\\text{10:}&\text{.}&\text{.}&\text{.}&\text{.}&71&32&1&\text{.}\\\text{11:}&\text{.}&\text{.}&\text{.}&\text{.}&2&37&14&\text{.}\\\text{12:}&\text{.}&\text{.}&\text{.}&\text{.}&12&6&6&2\\\text{13:}&\text{.}&\text{.}&\text{.}&
\text{.}&\text{.}&22&6&\text{.}\\\text{14:}&\text{.}&\text{.}&\text{.}&\text{.}&\text{.}&\text{.}&11&2\\\text{15:}&\text{.}&\text{.}&\text{.}&\text{.}&\text{.}&1&\text{.}&1\\\text{16:}&\text{.}&\text{.}&\text{.}&\text{.}&\text{.}&\text{.}&2&\text{.}\\\text{17:}&\text{.}&\text{.}&\text{.}&\text{.}&\text{.}&\text{.}&\text{.}&1\\\end{matrix}
$ \hskip 1cm   $\begin{matrix}
&0&1&2&3&4&5&6&7&8&9&10\\\text{total:}&1&11&54&156&294&378&336&204&81&19&2\\\text{0:}&1&\text{.}&\text{.}&\text{.}&\text{.}&\text{.}&\text{.}&\text{.}&\text{.}&\text{.}&\text{.}\\\text{1:}&\text{.}&\text{.}&\text{.}&\text{.}&\text{.}&\text{.}&\text{.}&\text{.}&\text{.}&\text{.}&\text{.}\\\text{2:}&\text{.}&\text{.}&\text{.}&\text{.}&\text{.}&\text{.}&\text{.}&\text{.}&\text{.}&\text{.}&\text{.}\\\text{3:}&\text{.}&11&\text{.}&\text{.}&\text{.}&\text{.}&\text{.}&\text{.}&\text{.}&\text{.}&\text{.}\\\text{4:}&\text{.}&\text{.}&9&\text{.}&\text{.}&\text{.}&\text{.}&\text{.}&\text{.}&\text{.}&\text{.}\\\text{5:}&\text{.}&\text{.}&2&7&\text{.}&\text{.}&\text{.}&\text{.}&\text{.}&\text{.}&\text{.}\\\text{6:}&\text{.}&\text{.}&43&4&2&\text{.}&\text{.}&\text{.}&\text{.}&\text{.}&\text{.}\\\text{7:}&\text{.}&\text{.}&\text{.}&58&4&\text{.}&\text{.}&\text{.}&\text{.}&\text{.}&\text{.}\\\text{8:}&\text{.}&\text{.}&\text{.}&12&64&2&\text{.}&\text{.}&\text{.}&\text{.}&\text{.}\\\text{9:}&\text{.}&\text{.}&\text{.}&75&32&
44&\text{.}&\text{.}&\text{.}&\text{.}&\text{.}\\\text{10:}&\text{.}&\text{.}&\text{.}&\text{.}&106&48&22&\text{.}&\text{.}&\text{.}&\text{.}\\\text{11:}&\text{.}&\text{.}&\text{.}&\text{.}&18&114&48&7&\text{.}&\text{.}&\text{.}\\\text{12:}&\text{.}&\text{.}&\text{.}&\text{.}&68&40&77&30&1&\text{.}&\text{.}\\\text{13:}&\text{.}&\text{.}&\text{.}&\text{.}&\text{.}&86&44&42&12&\text{.}&\text{.}\\\text{14:}&\text{.}&\text{.}&\text{.}&\text{.}&\text{.}&10&85&30&18&2&\text{.}\\\text{15:}&\text{.}&\text{.}&\text{.}&\text{.}&\text{.}&34&16&50&10&6&\text{.}\\\text{16:}&\text{.}&\text{.}&\text{.}&\text{.}&\text{.}&\text{.}&33&10&23&2&1\\\text{17:}&\text{.}&\text{.}&\text{.}&\text{.}&\text{.}&\text{.}&2&27&4&6&\text{.}\\\text{18:}&\text{.}&\text{.}&\text{.}&\text{.}&\text{.}&\text{.}&9&2&10&\text{.}&1\\\text{19:}&\text{.}&\text{.}&\text{.}&\text{.}&\text{.}&\text{.}&\text{.}&5&\text{.}&3&\text{.}\\\text{20:}&\text{.}&\text{.}&\text{.}&\text{.}&\text{.}&\text{.}&\text{.}&\text{.}&3&\text{.}&\text{.}\\\text{21:}&\text{.}
&\text{.}&\text{.}&\text{.}&\text{.}&\text{.}&\text{.}&1&\text{.}&\text{.}&\text{.}\\\end{matrix}
$}
\end{center}
\end{example}

\vskip 2mm

%\phil{Me da un poco de miedo todo lo que afirmamos en la seccion siguiente. No hay algun ejemplo mas peque\~no donde podriamos hacer los
%calculos a mano donde se produzca este fenomeno?}


\subsection{Independence on the characteristic of the base field}
It is well-known that the Betti numbers of a monomial ideal depend on the characteristic of the base field.
This phenomenon can be understood through Hochster's formula.
Namely, given  a monomial ideal $I\subseteq R=\kk[x_1,\dots,x_n]$,
we may assume that it is squarefree since its Betti numbers can always be obtained from a squarefree monomial ideal using polarization.
%\phil{begin add}
Now, a squarefree monomial ideal $I$ is always the Stanley-Reisner ideal of a simplicial complex $\Delta$.
%\phil{end add}
For any squarefree degree $\alpha \in \{0,1\}^n$, we consider
$F_\alpha:=\{x_i \hskip 2mm | \hskip 2mm \alpha_i \neq 0 \} \subseteq \{x_1,\dots,x_n\}$.
Let $\Delta_{|F_\alpha}$ be the simplicial complex obtained by taking all the faces of $\Delta$
whose vertices belong to $F_\alpha$. Hochster's formula states that the Betti numbers of $R/I$ can be computed
as the dimensions of reduced simplicial homology groups:
$$\beta_{i,\alpha}(R/I)=\dm_{\kk} \widetilde{H}_{|\alpha|-i-1}(\Delta_{|F_\alpha}; \kk).$$
By the universal coefficient theorem, we may just compute the homology modules
over the integer numbers $\widetilde{H}_{|\alpha|-i-1}(\Delta_{|F_\alpha}; \bZ)$. It follows that Betti
numbers depend on the characteristic of the base field whenever these homology groups have torsion.

\vskip 2mm

\begin{example} \label{ex_p}
The most recurrent example that illustrates the behavior of Betti numbers with respect to the characteristic of the base field
is the Stanley-Reisner ideal associated to a minimal triangulation of $\mathbb{P}_{\bR}^2$.
Namely, consider the following ideal in $R=\kk[x_1,\dots, x_6]$:
$$
I=(x_1x_2x_3,x_1x_2x_4,x_1x_3x_5,x_2x_4x_5,x_3x_4x_5,x_2x_3x_6,x_1x_4x_6,x_3x_4x_6,x_1x_5x_6,x_2x_5x_6).
$$
%
%\vskip 2mm
%
%\noindent
The Betti diagrams in characteristic zero and two respectively are
%
%\vskip 2mm
%
\begin{center}
{\small $\begin{matrix}
&0&1&2&3\\\text{total:}&1&10&15&6\\\text{0:}&1&\text{.}&\text{.}&\text{.}\\\text{1:}&\text{.}&\text{.}&\text{.}&\text{.}\\\text{2:}&\text{.}&10&15&6\\\end{matrix}
$ \hskip 3cm $\begin{matrix}
&0&1&2&3&4\\\text{total:}&1&10&15&7&1\\\text{0:}&1&\text{.}&\text{.}&\text{.}&\text{.}\\\text{1:}&\text{.}&\text{.}&\text{.}&\text{.}&\text{.}\\\text{2:}&\text{.}&10&15&6&1\\\text{3:}&\text{.}&\text{.}&\text{.}&1&\text{.}\\\end{matrix}
$}
\end{center}
%
%\vskip 2mm
%
On the other hand, the results that we obtain with the pruned and the Lyubeznik resolution respectively are:

\vskip 2mm

\begin{center}
{\small $\begin{matrix}
&0&1&2&3&4\\\text{total:}&1&10&15&7&1\\\text{0:}&1&\text{.}&\text{.}&\text{.}&\text{.}\\\text{1:}&\text{.}&\text{.}&\text{.}&\text{.}&\text{.}\\\text{2:}&\text{.}&10&15&6&1\\\text{3:}&\text{.}&\text{.}&\text{.}&1&\text{.}\\\end{matrix}
$ \hskip 3cm
%$\begin{matrix}
%&0&1&2&3&4\\\text{total:}&1&10&15&8&2\\\text{0:}&1&\text{.}&\text{.}&\text{.}&\text{.}\\\text{1:}&\text{.}&\text{.}&\text{.}&\text{.}&\text{.}\\\text{2:}&\text{.}&10&15&6&2\\\text{3:}&\text{.}&\text{.}&\text{.}&2&\text{.}\\\end{matrix}
%$ \hskip 1cm
$\begin{matrix}
&0&1&2&3&4\\\text{total:}&1&10&27&27&9\\\text{0:}&1&\text{.}&\text{.}&\text{.}&\text{.}\\\text{1:}&\text{.}&\text{.}&\text{.}&\text{.}&\text{.}\\\text{2:}&\text{.}&10&15&18&9\\\text{3:}&\text{.}&\text{.}&12&9&\text{.}\\\end{matrix}
$}
\end{center}
%\phil{Es razonable comprobar la simplicial a mano? Sino, yo no la mencionaria aqui ya que ademas no aporta nada para lo que queremos decir}
\end{example}

%\vskip 4mm

\begin{remark}
%
%{\bf Para estar seguros, mejor preguntar a Welker}
%
%\vskip 2mm
%
All the free resolutions that we consider in this work are obtained
from the Taylor resolution using discrete Morse theory as in
\cite{BW} or \cite{JW}. If one follows closely the construction of the CW-complex
$X_{\cA}$ that is homotopically equivalent to $X_{\tt Taylor}$, one
may check that it is independent of the characteristic of the base
field.
%
%\vskip 2mm
%
In this sense, the result obtained with the pruned resolution in Example \ref{ex_p} is the best that we can achieve
using discrete Morse theory.
\end{remark}




\subsection{Minimality}

A necessary and sufficient condition for the minimality of a cellular free resolution
$\mathbb{F}_{\bullet}^{(X_{\cA},gr)}$  is described in \cite[Lemma 7.5]{BW}.
The pruned resolution obtained in Corollary \ref{C1} is minimal
if and only if the following condition is satisfied:

%In our situation, it boils down to the following condition:
%\phil{No seria mejor enunciar el resultado de la forma: The pruned resolution obtained in Corollary \ref{C1} is minimal
%if and only if ....}

\vskip 2mm

Recall that the  complex $X_{\tt Taylor}$ associated to  a monomial
ideal $I=\langle m_1,\dots, m_r \rangle$ has faces labelled by
$\sigma \in \{0,1\}^r$. Then, the cells $\sigma\in X_{\cA}$ that
survive to any of the pruning processes described in Section $\S 3$
must satisfy
 $gr(\sigma) \neq gr(\sigma + \varepsilon_j) $ whenever $\sigma_j=0$.

\vskip 2mm

\begin{remark}
Barile \cite{Bar} noticed that in the case of the Lyubeznik resolution, it is enough
checking out the aforementioned condition for maximal faces of $X_{\tt Lyub}$.
%\phil{No tengo acceso al articulo de Barile. Estamos seguro que dice esto? Me sorprende un poco pero podria ser}
\end{remark}

\vskip 2mm

The Taylor resolution is rarely minimal.
%{\bf (alg\'un ejemplo interesante?)}.
Some examples of minimal Lyubeznik resolutions
include shellable ideals (see \cite{BW}) and the matroid ideal of a
finite projective space (see \cite{Nov}). Finally, let's point out that Torrente and Varbaro \cite{TV15} provided a
fast algorithm for computing Betti diagrams of a monomial ideal taking as a starting point
the Lyubeznik resolution of the ideal. Using the pruned resolution as starting
point of their algorithm would speed up their
computation of the Betti diagram.






\section{Betti splittable monomial ideals} \label{split}
A technique that has been successfully applied to describe Betti
numbers of monomial ideals is based on the concept of splitting
introduced by Eliahou and Kervaire in \cite{EK}; see \cite{HVT2} for a nice survey.
A refinement of this concept was coined by Francisco, H\`a and Van Tuyl
in \cite{FHV}.

\begin{definition}
We say that $I=J+K$ is a {\it Betti splitting} for the monomial ideal $I$ if
the following formula for the $\bZ^n$-graded Betti numbers is satisfied:
 $$\beta_{i,\alpha}(I)= \beta_{i,\alpha}(J)+\beta_{i,\alpha}(K)+\beta_{i-1,\alpha}(J\cap
K)\,.$$
\end{definition}

We can mimic the same concept introducing the {\it pruned Betti numbers} as
the Betti numbers that we obtain in the pruned free resolution of
Section \ref{pruned1}, and  that we will  denote by
$\overline{\beta}_{i,\alpha}(I)$ to distinguish them from the usual Betti
numbers.

\begin{definition}
We say that $I=J+K$ is a {\it pruned Betti splitting} for the monomial ideal $I$ if
the following formula for the $\bZ^n$-graded Betti numbers is satisfied:
 $$\overline{\beta}_{i,\alpha}(I)= \overline{\beta}_{i,\alpha}(J)+\overline{\beta}_{i,\alpha}(K)+\overline{\beta}_{i-1,\alpha}(J\cap
K)\,.$$
\end{definition}

\vskip 2mm

\begin{remark}
Obviously, a pruned Betti splitting provides a Betti splitting
whenever the pruning algorithm gives a minimal free resolution
of $I$.
\end{remark}

\vskip 2mm

Necessary and sufficient conditions describing Betti splittings
have been given in \cite{FHV}. Using our main Algorithm \ref{alg1},
we will give some conditions that are easy to check and that provide a pruned Betti
splitting.

%\vskip 2mm

\subsection{A partial pruning of $J\cap K$}

Let $I=\langle m_1,\dots,m_r \rangle$ be a monomial ideal, and consider the Taylor
simplicial complex $X_{\tt Taylor}$ on $r$ vertices which
faces are labeled by $\sigma \in \{0,1\}^r$. Consider a
decomposition $I=J+K$ with $J=\langle m_1,\dots,m_s \rangle$ and $K=
\langle m_{s+1},\dots,m_r \rangle$. In order to compute the pruned
resolution of $J$ and $K$ we have to consider the subcomplexes:

\vskip 2mm

\begin{itemize}
 \item [$\cdot$] $X_J\subseteq X_{\tt Taylor}$  with faces labelled by
 $\sigma=(\sigma_1,\dots,\sigma_s,0,\dots,0) \in \{0,1\}^r$, and
 \item [$\cdot$] $X_K\subseteq X_{\tt Taylor}$  with faces labelled by
 $\sigma=(0,\dots,0,\sigma_{s+1},\dots,\sigma_r) \in \{0,1\}^r$.
\end{itemize}

\vskip 2mm

Denote by $X'$ the set obtained by removing the faces of
$X_J$ and $X_K$ from $X_{\tt Taylor}$. Notice that  $X'$
 is not a simplicial subcomplex of $X_{\tt Taylor}$ and, in particular,
 it is not the Taylor simplicial complex associated to the intersection.
 However, applying a partial pruning algorithm to $X_{J\cap K}$ we obtain  $X'$.
 The proof of this fact is quite tedious but straightforward.
 We will simply sketch it  and leave the details for the interested reader.


 \vskip 2mm

  Consider the Taylor complex  $X_{J\cap K}$
 associated to the (possibly non-minimal) set of generators of  $J\cap K$  given by
 the monomials
 $$\{ m_{1,s+1},\dots, m_{s,s+1},  m_{1,s+2}, \dots, m_{s,s+2},  \dots, m_{1,r}, \dots, m_{s,r}\},$$
%$\{ m_{i,j} \hskip 2mm | \hskip 2mm  i=1,\dots,s \hskip 2mm , \hskip 2mm j=s+1,\dots, r \},$
 where $m_{i,k}=\lcm(m_i,m_k)$.  In the sequel we will also denote $m_{i_1,\dots, i_\ell}=\lcm(m_{i_1},\dots, m_{i_\ell})$.
The Taylor complex $X_{J\cap K}$ is the full simplicial complex on $sr$ vertices with faces labelled by
$\sigma \in \{0,1\}^{sr}$. Notice that the  vertex $\varepsilon_{j}$  corresponds to the monomial
$m_{\varepsilon_j}:=m_{i,k}$ if we have
$j=ks+i$ for $k\in \{0,\dots, r-1\}$ and $i\in\{1,\dots, s\}$.  In general, a face $\sigma$ corresponds
to $m_\sigma:=\lcm (m_{\varepsilon_j} \hskip 2mm | \hskip 2mm \sigma_j = 1)$.


 \vskip 2mm

 Now we want to apply our Algorithm \ref{alg1} but we only want to prune the edges
 ${\sigma} \lra{\sigma + \varepsilon_j}$ whenever the corresponding $\lcm$'s involve the same
 monomials $m_i \in I$. For example, we have  that
 $\lcm(m_{i,b}, m_{a,k})=\lcm(m_{i,k},m_{i,b}, m_{a,k})=m_{i,k,a,b}$  so both $\lcm$'s
 involve the same monomials $m_i,m_k,m_a,m_b \in I$ for all $a,b$.
 This means that the corresponding edge would be pruned at step $j=ks+i$.


 %$\lcm(m_{i_1,k_1},\dots, m_{i_\ell,k_\ell})$  $\lcm(m_{i,k},m_{i_1,k_1},\dots, m_{i_\ell,k_\ell})$

 \vskip 2mm

 The partial pruning algorithm that we apply is the following:

 \vskip 2mm


 \begin{algorithm}\label{alg4} { (Partial pruning)}

\vskip 2mm

{\rm \noindent {\sc Input:} The set of edges $E_{X_{J\cap K}}$.

%\noindent {\sc Output:} The set of pruned edges $\cA_P \subseteq E_{X_{\tt Taylor}}$ \phil{por que poner 'OUTPUT' y luego 'RETURN'
%con la misma informacion en todos los algoritmos?}

\vskip 2mm

%{\bf Prune the extra components}

%\vskip 2mm

For $j$ from $1$ to $sr$, incrementing by $1$

\vskip 2mm

 \begin{itemize}



\item[\textbf{(j)}] ${\it Prune}$ the edge ${\sigma} \lra{\sigma + \varepsilon_j}$ for
all $\sigma \in \{0,1\}^{sr}$ such that $\sigma_j=0$, where `prune'
means remove the edge if  it survived after step $(j-1)$ and
$m_\sigma$ involves the monomials $m_i$ and $m_k$, where $j=ks+i$.
% and this edge survived to all the previous steps.
%Here $gr(\sigma)= m_\sigma:= \lcm (m_{\varepsilon_j} \hskip 2mm | \hskip 2mm \sigma_j = 1)$.
% $(j-1)$ \phil{no quedaria mas claro decir "if it survived at steps $1$ to $(j-1)$"?} and $gr(\sigma)=gr(\sigma + \varepsilon_j) $.

\end{itemize}

\vskip 2mm


\noindent {\sc Return:} The set $\cA'$ of edges that have been pruned.

}

\end{algorithm}


The tedious part is to check that using this algorithm we obtain $X'$, that is, $X_{\cA'}=X'$.
To illustrate this fact we present the following:



%\phil{Creo que dicho asi lo que dice la frase anterior no es cierto. Como decia antes, creo que lo que hacemos aqui es un pruning parcial
%sobre $X_{J\cap K}$ para obtener $X'$. Por eso de hecho luego los pruned Betti numbers de la interseccion
%se obtendran aplicando el pruning a $X'$ (si el pruning anterior hubiera sido completo, no habria nada mas que hacer).
%Por eso es importante antes ver si podemos hacer un pruning parcial (que creo que si, al menos bajo ciertas condiciones).}



%\phil{Tambien creo que habria que poner alguna prueba, dejame pensar si podemos decir algo corto pero convincente.}

\begin{example}
Let $I=\langle m_1,m_2,m_3,m_4 \rangle$ be a monomial ideal generated by four monomials.
We are going to consider the following decompositions:
%In this case, there essentially two ways to write $I$ as $I=J+K$ as before.


\vskip 2mm

$\cdot$ Consider  $I=J+K$ with $J=\langle m_1,m_2,m_3 \rangle$ and $K= \langle m_4 \rangle$.
In this case we have  $J\cap K= \langle m_{1,4}, m_{2,4},m_{3,4} \rangle $  so  $X'$
coincides with $X_{J\cap K}$.

\vskip 2mm

$\cdot$ If  $J=\langle m_1,m_2 \rangle$ and $K= \langle m_3, m_4 \rangle$ then
we have that $J\cap K= \langle m_{1,3}, m_{2,3}, m_{1,4}, m_{2,4} \rangle $.
The directed graph associated to the Taylor simplicial complex of $I$ is:

{\tiny $${\xymatrix { &&&& {\bf m_{1,2,3,4}} & \\
& {\bf m_{1,2,3}} \ar@{.>}[urrr] & & {\bf m_{1,2,4}}\ar[ur] & {\bf m_{1,3,4}} \ar[u]& {\bf m_{2,3,4}} \ar[ul] \\
m_{1,2} \ar[ur] \ar@{.>}[urrr]&{\bf m_{1,3}} \ar[u] \ar@{.>}[urrr]& {\bf m_{2,3}} \ar[ul] \ar@{.>}[urrr]& {\bf m_{1,4}} \ar[ur]|\hole \ar[u] & {\bf m_{2,4}} \ar[ul] \ar[ur]& m_{3,4} \ar[ul]|\hole \ar[u]\\
m_{1} \ar@{.>}[urrr] \ar[ur]|\hole \ar[u] &m_{2} \ar@{.>}[urrr] \ar[ul] \ar[ur]& m_{3} \ar@{.>}[urrr] \ar[ul]|\hole \ar[u]&  & m_{4} \ar[ul] \ar[u] \ar[ur] &  \\
%& \ar[ul] \ar[u] \ar[ur] &&&&
}}
$$}

\noindent
and
%In particular,
the monomials corresponding to the set $X'$ are indicated with bold letters.
On the other hand, applying the partial pruning algorithm to the Taylor complex $X_{J\cap K}$ we also get
the set $X'$ as shown in the following graph:

%\phil{pruning parcial, podria haber mas cancelaciones dependiendo de los generadores. Se puede hacer?
%Ademas, hace falta que $X'$ sea regular (no hace falta que sea simplicial) para poder aplicar efectivamente el pruning algorithm a $X'$.
%Lo es?}

{\tiny $${\xymatrix { &&&& m_{1,2,3,4} \ar@2{->}[dr]& \\
& m_{1,2,3,4} \ar@{.>}[urrr]\ar@2{->}[dr]  & & m_{1,2,3,4}\ar[ur] \ar@2{->}[d] & {\bf m_{1,2,3,4}} \ar[u]& m_{1,2,3,4}  \\
{\bf m_{1,2,3}} \ar@{.>}[urrr]\ar[ur] & {\bf m_{1,3,4}} \ar@{.>}[urrr]\ar[u] & m_{1,2,3,4}\ar@{.>}[urrr] & m_{1,2,3,4} \ar[ur]|\hole  & {\bf m_{2,3,4}} \ar[ul] \ar[ur]& {\bf m_{1,2,4}} \ar[ul]|\hole \ar[u]\\
{\bf m_{1,3}} \ar@{.>}[urrr] \ar[ur]|\hole \ar[u] & {\bf m_{2,3}} \ar@{.>}[urrr] \ar[ul] \ar[ur]& {\bf m_{1,4}} \ar@{.>}[urrr] \ar[ul]|\hole \ar[u]&  & {\bf m_{2,4}} \ar[ul] \ar[u] \ar[ur] &  \\
%& \ar[ul] \ar[u] \ar[ur] &&&&
}}
$$}

\end{example}


\subsection{A sufficient condition for Betti splittings}



Let $I=J+K$ be a decomposition of a monomial ideal.
In order to have a Betti splitting, it is enough to check that applying Algorithm \ref{alg1} to $I$,
the pruning steps are realized
independently within the edges associated to $X_J$, $X_K$ and $X'$.
This gives the following result that provides a sufficient condition to have a Betti splitting:

\begin{proposition}
Let $I=\langle m_1,\dots,m_r \rangle\subseteq \kk[x_1,\dots,x_n]$ be a monomial
ideal and consider the ideals $J=\langle m_1,\dots,m_s \rangle$ and $K=
\langle m_{s+1},\dots,m_r \rangle$.  If  we only perform pruning steps within the
edges of $X_J$, $X_K$ or $X'$ separatedly in Algorithm
\ref{alg1}, then $I=J+K$ is a pruned Betti splitting for $I$.
\end{proposition}

In the particular case that we remove just one generator from our initial ideal,
we obtain the following:

\begin{corollary}
Let $I=\langle m_1,\dots,m_r \rangle\subseteq \kk[x_1,\dots,x_n]$ be
a monomial ideal and consider the ideals $J=\langle
m_1,\dots,m_{r-1} \rangle$ and $K= \langle m_r \rangle$. If no
pruning has been made at the step $r$ of Algorithm \ref{alg1}, then
$I=J+K$ is a pruned Betti splitting for $I$.
\end{corollary}

\vskip 2mm

 Betti splitting techniques can be used to provide a recursive
 method to compute Betti numbers of monomial ideals. For example,
 the case of edge ideals of graphs has been successfully studied in \cite{HVT1} and \cite{FHV}
where they considered {\it splitting edges} and {\it
splitting vertices}. Splitting edges are easy to describe (see
\cite[Theorem 3.4]{FHV}) but, in general they are difficult to find.
On the other hand, every vertex is a splitting vertex except for some limit cases where the vertex is
isolated or its complement consists of isolated vertices.


\vskip 2mm
One can also check that any vertex also
provides a pruned Betti splitting.
Indeed, let $I=I(G)\subseteq \kk[x_1,\dots,x_n]$ be the edge ideal of a
graph $G$ on $n$ vertices. Consider a vertex, say $x_n$, and the
decomposition $I=J+K$ where $J=I(G\backslash \{x_n\})$ is the edge
ideal of the graph obtained from $G$ by removing
the vertex $x_n$ and all the edges passing through this vertex. Notice that  $K$ is the edge ideal of a bipartite graph
$\mathcal{K}_{1,d}$ where $d$ is the number of vertices adjacent to $x_n$ in $G$. We
have a pruned Betti splitting because the ideal $J$ does not involve
the variable $x_n$ and the pruning steps of the algorithm are
performed within  $X_{K}$ and $X'$ separately. Moreover, it
is not difficult to see that the pruning algorithm provides a minimal
free resolution for $K$, and hence
$$\overline{\beta}_{i,\alpha}(I)= \overline{\beta}_{i,\alpha}(J)+{\beta}_{i,\alpha}(K)+\overline{\beta}_{i-1,\alpha}(J\cap
K).$$
In particular, if Algorithm \ref{alg1} provides a minimal free
resolution for $J$ and $J\cap K$, then it is also provides a minimal free resolution for $I$.

\vskip 2mm

%\subsection{Examples}

For paths and cycles, we can use these splitting
techniques to prove that the pruning algorithm will always provide a minimal
free resolution as we already observed in Example \ref{ex5path5cycle} for the $5$-path and the $5$-cycle.
The Betti numbers of these ideals have already
been computed by Jacques in \cite{Jac04}. %The case of wheels can be found in \cite{EFMM}.

\vskip 2mm

\begin{example}[$n$-paths]\label{ex_npath}
Let $I=\langle
x_1x_2,x_2x_3,\dots,x_{n-1}x_n \rangle$ be the edge ideal of an
$n$-path. A decomposition $I=J+K$ with $J=\langle
x_1x_2,x_2x_3,\dots,x_{n-2}x_{n-1} \rangle$ and $K=\langle
x_{n-1}x_n \rangle$ is a pruned Betti splitting simply because the
ideal $J$ does not involve the vertex $x_n$. Therefore we have
$$\overline{\beta}_{i,\alpha}(I)= \overline{\beta}_{i,\alpha}(J)+\overline{\beta}_{i,\alpha}(K)+\overline{\beta}_{i-1,\alpha}(J\cap
K).$$ Indeed, this is a Betti splitting and the pruning algorithm
provides a minimal free resolution, thus
$${\beta}_{i,\alpha}(I)= {\beta}_{i,\alpha}(J)+{\beta}_{i,\alpha}(K)+{\beta}_{i-1,\alpha}(J\cap
K).$$ By induction, we get a minimal free resolution for the ideals
$J$ and $K$ so we only have to control the intersection in order to
get the desired result. Recall that, using Lemma \ref{minimal}, we
only have to consider a
 minimal set of generators. In our case we have
$$J\cap K=\langle \underbrace{x_1x_2x_{n-1}x_n, \dots,
x_{n-4}x_{n-3}x_{n-1}x_n}_{J'},\underbrace{x_{n-2}x_{n-1}x_n}_{K'}\rangle.
$$
If we consider the decomposition $J\cap K=J'+K'$, we have a Betti splitting.
Namely, the pruning algorithm applied on the ideals $J'$ and $J'\cap
K'$ is equivalent to the one given for the path $\langle x_1x_2,
\dots, x_{n-4}x_{n-3}\rangle$, and we are done by induction.
\end{example}


%Our starting point for Algorithm \ref{alg1} is the Taylor simplicial
%complex $X_{\tt Taylor}$ that is labelled by $\sigma \in
%\{0,1\}^{n-1}$.
%
%\vskip 2mm

\vskip 2mm

\begin{example}[$n$-cycles]
Let $I= \langle x_1x_2,x_2x_3,\dots,x_{n-1}x_n,x_nx_1 \rangle$ be the edge ideal of
an $n$-cycle. The decomposition $I=J+K$ with $J= \langle
x_1x_2,x_2x_3,\dots,x_{n-1}x_{n} \rangle$ and $K=\langle x_{n}x_1
\rangle$ is not a Betti splitting by \cite[Theorem 3.4]{FHV}.
Indeed, there is a pruning in the last step of Algorithm \ref{alg1}.

%In this case,  the Taylor simplicial complex $X_{\tt Taylor}$  is
%labelled by $\sigma \in \{0,1\}^{n}$.


\vskip 2mm

It is more convenient to consider a splitting vertex . Namely,
consider $I=J+K$ with $J= \langle x_1x_2,x_2x_3,\dots,x_{n-2}x_{n-1}
\rangle$ and $K=\langle x_{n-1}x_{n}, x_{n}x_1 \rangle$. Since $J$ and $K$
are the ideal of an $(n-1)$-path and a $3$-path respectively, the pruning algorithm provides a
minimal free resolution as shown in Example \ref{ex_npath}. Therefore we have:
$$\overline{\beta}_{i,\alpha}(I)= {\beta}_{i,\alpha}(J)+{\beta}_{i,\alpha}(K)+\overline{\beta}_{i-1,\alpha}(J\cap
K).$$


\vskip 2mm
A minimal set of generators of $J\cap K$ is
$$\langle \underbrace{x_2x_3x_{n-1}x_{n},\dots,x_{n-4}x_{n-3}x_{n-1}x_{n},
x_{n-2}x_{n-1}x_{n}}_{J'}, \underbrace{
x_1x_2x_{n},x_1x_3x_{4}x_{n},\dots,x_{1}x_{n-3}x_{n-2}x_{n}}_{K'}
\rangle\,.$$
%\vskip 2mm
We have a pruned Betti splitting given by $J\cap K= J' + K'$. The
pruning algorithm gives a minimal free resolution for the ideals
$J'$ and $K'$ as we have seen when dealing with the case of paths. A
minimal set of generators for the intersection $J'\cap K'$ is
$$\langle x_1x_2x_3x_{n-1}x_n,x_1x_3x_{4}x_{n-1}x_n, \dots ,
x_1x_{n-3}x_{n-2}x_{n-1}x_n,x_{1}x_2x_{n-2}x_{n-1}x_n \rangle$$ but
the pruning algorithm applied to this ideal is equivalent to the one
for the cycle $\langle x_2x_3,x_3x_{4}, \dots ,
x_{n-3}x_{n-2},x_2x_{n-2} \rangle$, and we are done by induction.
\end{example}

%$\bullet$ {\bf $n$-Wheels:} Let
%$I=(x_1x_2,x_2x_3,\dots,x_{n-2}x_{n-1},x_{n-1}x_{1},x_nx_1,\cdots,
%x_nx_{n-1})$ be the edge ideal of an $n$-wheel. We are also going to consider a splitting vertex
%so let $I=J+K$
%with $J=(x_1x_2,x_2x_3,\dots,x_{n-2}x_{n-1},x_{n-1}x_{1})$ and
%$K=(x_nx_1,\dots, x_nx_{n-1})$. We have that  $J$ is the ideal of a $n$-cycle.
%The Taylor simplicial complex
%$X_{\tt Taylor}$  is labelled by $\sigma \in \{0,1\}^{2n-2}$.



\vskip 3mm



\begin{thebibliography}{23}

\bibitem{cocoa}
J.~Abbot and A. M.~Bigatti,
\newblock{CoCoALib: a C++ library for doing Computations in Commutative Algebra,}
available at {\tt http://cocoa.dima.unige.it/cocoalib}.

\bibitem{AY}
J.\`Alvarez Montaner and K.~Yanagawa,
\newblock {\it Lyubeznik numbers of local rings and linear strands of graded ideals,}
\newblock Accepted in Nagoya Math. J.

\bibitem{Bar}
M.~Barile,
\newblock {\it On ideals whose radical is a monomial ideal,}
\newblock {Comm. Alg.} {\bf 33} (2005), 4479--4490.

\bibitem{BW}
E.~Batzies and V.~Welker,
\newblock {\it Discrete Morse theory for cellular resolutions,}
\newblock {J. Reine Angew. Math.} {\bf 543} (2002), 147--168.

\bibitem{BPS}
D.~Bayer, I.~Peeva  and B.~Sturmfels,
\newblock {\it Monomial resolutions,}
\newblock {Math. Res. Lett.} {\bf 5} (1998), 31--46.

\bibitem{BS}
D.~Bayer and B.~Sturmfels,
\newblock {\it Cellular resolutions of monomial modules,}
\newblock {J. Reine Angew. Math.} {\bf 502} (1998), 123--140.

\bibitem{CH}
M.~K.~Chari,
\newblock {\it On discrete Morse functions and combinatorial decompositions,}
\newblock{Discrete Math.} {\bf 217} (2000), 101--113.

%\bibitem{Sing}
%W. Decker, G.-M. Greuel, G. Pfister and H. Sch\"onemann,
%\newblock{{\sc Singular} {4-1-0} -- A computer algebra system for polynomial computations,}
%available at {\tt http://www.singular.uni-kl.de}.

\bibitem{EK}
S.~Eliahou and M.~Kervaire,
\newblock {\it Minimal resolutions of some monomial ideals,}
\newblock { J. Algebra} {\bf 129} (1990), 1--25.

%\bibitem{EFMM}
% E.~Emtander, R.~Fr\"oberg, F.~Mohammadi and S.~Moradi,
% \newblock {\it Poincar\'e series of some hypergraph algebras,}
 % \newblock {Math. Scand.} {\bf 112} (2013), 5--10.

\bibitem{FG}
 O.~Fern\'andez-Ramos and P.~Gimenez,
 \newblock {\it Regularity 3 in edge ideals associated to bipartite graphs,}
  \newblock {J. Algebraic Comb.} {\bf 39} (2014), 919--937.

\bibitem{For}
R.~Forman,
\newblock {\it Morse theory for cell complexes,}
\newblock  {Adv. Math.} {\bf 134} (1998), 90--145.

\bibitem{FHV}
C.~Francisco, H.~T.~H\`a and A.~Van Tuyl,
\newblock {\it Splittings of monomial ideals,}
\newblock  {Proc. Amer. Math. Soc.} {\bf 137} (2009), 3271--3282.

\bibitem{GPW}
V.~Gasharov, I.~Peeva, and V.~Welker,
\newblock {\it The lcm-lattice in monomial resolutions,}
\newblock { Math. Res. Lett.} {\bf 6} (1999), 521--532.

%\bibitem{GS}
%D.~Grayson and M.~Stillman,
%\newblock{Macaulay2, a software system for research in algebraic geometry,}
%available at {\tt http://www.math. uiuc.edu/Macaulay2}.

\bibitem{HVT1}
H.~T.~H\`a and A.~Van Tuyl,
\newblock {\it Splittable ideals and the resolution of monomial ideals,}
\newblock  {J. Algebra} {\bf 309} (2007), 405--425.

\bibitem{HVT2}
H.~T.~H\`a and A.~Van Tuyl,
\newblock {\it Resolutions of squarefree monomial ideals via facet ideals: a survey,}
\newblock  {Contemp. Math.} {\bf 441} (2007), 91--117.

\bibitem{Hoc}
M.~Hochster,
\newblock {\it Cohen-Macaulay rings, combinatorics, and simplicial complexes,}
Ring theory, II (Proc. Second Conf., Univ. Oklahoma, Norman, Okla., 1975), pp. 171-223.
Lecture Notes in Pure and Appl. Math., Vol. 26, Dekker, New York, 1977.

\bibitem{Jac04}
S.~Jacques,
\newblock {\it Betti Numbers of Graph Ideals}, Ph.D. Thesis, Univ. of Sheffield (2004), available at  {\tt arXiv:math/0410107}.

\bibitem{JW}
M.~J\"ollenbeck and V.~Welker,
\newblock {\it Minimal resolutions via algebraic discrete Morse theory},
\newblock  { Mem. Amer. Math. Soc.} {\bf 197}  (2009).

\bibitem{Lyu88}
G.~Lyubeznik,
\newblock {\it A new explicit finite free resolution of ideals generated by monomials in an $R$-sequence},
\newblock {\it J. Pure and Appl. Algebra} {\bf 51} (1988),  193--195.

\bibitem{Nov}
I.~Novik,
\newblock {\it Lyubeznik's resolution and rooted complexes,}
\newblock {J. Algebraic Comb.} {\bf 16} (2002), 97--101.

\bibitem{Sko}
E.~Sk\"oldberg,
\newblock {\it Morse theory from an algebraic viewpoint,}
\newblock { Trans. Amer. Math. Soc.} {\bf 358} (2006), 115--129.

\bibitem{Tay66}
D.~Taylor,
\newblock {\it Ideals generated by an $R$-sequence}, PhD-Thesis, University of Chicago, 1966.

\bibitem{TV15}
M-L~Torrente and M.~Varbaro,
\newblock {\it An alternative algorithm to compute the Betti table of a monomial ideal,}
\newblock  available at  {\tt arXiv:1507.01183}.

\bibitem{Vel}
M.~Velasco,
\newblock {\it Minimal free resolutions that are not supported by a CW-complex,}
\newblock  {J. Algebra} {\bf 319} (2008), 102--114.

\bibitem{Wel07}
V.~Welker,
\newblock {\it Discrete Morse theory and free resolutions,}
\newblock {in:  Algebraic Combinatorics}, Universitext, Springer, Berlin (2007), 81--172.

\end{thebibliography}

\end{document}







\section{Arithmetical rank of monomial ideals}


The {\it arithmetical rank} of $I$ is defined as
 $${\rm ara}\hskip 1mm I := {\rm min} \{ r \in \bN \hskip 1mm | \hskip 1mm {\rm there \hskip 2mm exists \hskip 2mm some} \hskip 2mm a_1,\dots,a_r \in I \hskip 2mm {\rm s.t.} \hskip 2mm \rad (a_1,\dots,a_r) = \rad I \}.$$
K.~Kimura \cite{Kim09}  gave an upper bound for this invariant in terms of the $L\textendash{\rm length}$  of $I$
defined  as the minimal length of all Lyubeznik resolutions of $I$.


\begin{theorem}{(Kimura \cite{Kim09})}
Let $\lambda$ be the $L\textendash{\rm length}$ of a monomial ideal $I$ of $R$, then
${\rm ara}\hskip 1mm I \leq \lambda$
\end{theorem}


 One may also define the  $P\textendash{\rm length}$ of $I$ as the minimal
length of all pruned resolutions of $I$. From its construction it is clear that
$P\textendash{\rm length} \leq L\textendash{\rm length}$ so it is natural to ask
whether the $P\textendash{\rm length}$ is also a bound for the arithmetical rank.


\vskip 2mm

The {\it cohomological dimension} of $I$ defined as
 $ {\rm cd}(I):={\rm max} \{ r \in \bN \hskip 2mm | \hskip 2mm H_I^r(R)\neq 0 \}$ is a lower bound for the arithmetical rank
so we have  $$ {\rm cd}(I) \leq {\rm ara}\hskip 1mm I \leq L\textendash{\rm length}$$
For squarefree monomial ideals Lyubeznik \cite{Lyu88} proved that ${\rm pd}_R(R/I)={\rm cd}(I)$ where ${\rm pd}_R(R/I)$
denotes the {\it projective dimension} of $R/I$, that is the length of a minimal resolution of $R/I$. In this case we have
$${\rm pd}_R(R/I) = {\rm cd}(I) \leq {\rm ara}\hskip 1mm I \leq L\textendash{\rm length}$$

\vskip 4mm

{\bf Aqu\'i salen muchas preguntas interesantes y Oscar ya ha trabajado algunos ejemplos.}

