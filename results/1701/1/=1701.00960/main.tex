\documentclass{llncs}

%\usepackage{moreverb}
%\usepackage{amsmath}
%\usepackage{comment}
\usepackage{alltt}
%\usepackage{hyperref}
\usepackage{framed}

\usepackage{amssymb}
\usepackage{graphics}
\usepackage{listings}
\usepackage[utf8x]{inputenc}
%\usepackage[T1]{fontenc}

%\usepackage{listingsutf8}
%\usepackage{wrapfig}

\usepackage{marginnote}
\usepackage{tikz}
\usetikzlibrary{shapes,arrows}
\usetikzlibrary{decorations.markings}

\newif\ifshowcomments
\showcommentstrue
%\showcommentsfalse

\ifshowcomments
\newcommand{\mynote}[2]{\marginnote{{\bfseries\sffamily\scriptsize#1}
 {\small$\blacktriangleright$\textsf{\emph{#2}}$\blacktriangleleft$}}}
\else
\newcommand{\mynote}[2]{}
\fi
\newcommand{\bs}[1]{\mynote{Badr}{#1}}
\newcommand{\mt}[1]{\mynote{Tahar}{#1}}
\newcommand{\jpb}[1]{\mynote{Jean-Paul}{#1}}
\newcommand{\mf}[1]{\mynote{Mamoun}{#1}}

\lstset{tabsize=2,inputencoding=utf8x,language=eventB}

%\begin{comment}
\newcommand{\textlambda}{\(\lambda\)}
\DeclareUnicodeCharacter{8712}{{\(\in\)}}
\DeclareUnicodeCharacter{8788}{{\(:\,=\)}}
\DeclareUnicodeCharacter{8838}{{\(\subseteq\)}}
\DeclareUnicodeCharacter{8229}{{\(.\,.\)}}
\DeclareUnicodeCharacter{8696}{{\(\mkern 6mu\mapstochar \mkern -6mu\rightarrow\)}}
\DeclareUnicodeCharacter{10496}{{\(\mkern 6mu\mapstochar \mkern -6mu\twoheadrightarrow\)}}
\DeclareUnicodeCharacter{10516}{{\(\mkern 9mu\mapstochar \mkern -9mu\rightarrowtail\)}}
\DeclareUnicodeCharacter{10518}{{\(\rightarrowtail \mkern  -18mu\twoheadrightarrow\)}} 
\DeclareUnicodeCharacter{57603}{{\(\lhd\mkern-9mu-\)}} % <+
\DeclareUnicodeCharacter{9665}{{\(\triangleleft\)}}
\DeclareUnicodeCharacter{9655}{{\(\triangleright\)}}
\DeclareUnicodeCharacter{8473}{{\(\mathbb{P}\)}}
%\end{comment}


%\tikzstyle{context} = [fill=red!20, minimum height=4em, rounded corners, text centered,]
%\tikzstyle{machine} = [fill=blue!20, minimum height=4em, text centered,]

\title{An Event-B framework \\ for the validation of Event-B refinement plugins \\
               (ongoing work) 
        }
\author{J.-P. Bodeveix, M. Filali, , M.-T. Bhiri, B. Siala}
\institute{IRIT CNRS UPS Université de Toulouse \\ 
                Université de Sfax
              }
\date{October, 2016}
\usepackage{fullpage}
\begin{document}

\maketitle


\begin{abstract}
%\begin{abstract}
%Although Deep Convolutional Neural Networks (CNNs) have liberated their power in various computer vision tasks, the most important components of CNN, convolutional layers and fully connected layers, are still limited to linear transformations. In this paper, we propose a novel Factorized Bilinear (FB) layer to model the pairwise feature interactions by considering the quadratic terms in the transformations. Compared with existing methods that tried to incorporate complex non-linearity structures into CNNs, the factorized parameterization makes our FB layer only require a linear increase of parameters and affordable computational cost. To further reduce the risk of overfitting of the FB layer, a specific remedy called DropFactor is devised during the training process. We also analyze the connection between FB layer and some existing models, and show FB layer is a generalization to them. Finally, we validate the effectiveness of FB layer on several widely adopted datasets including CIFAR-10, CIFAR-100 and ImageNet, and demonstrate superior results compared with various state-of-the-art deep models.

Neural Style Transfer~\cite{neuralart} has recently demonstrated very exciting results which catches eyes in both academia and industry. Despite the amazing results, the principle of neural style transfer, especially why the Gram matrices could represent style remains unclear. In this paper, we propose a novel interpretation of neural style transfer by treating it as a domain adaptation problem. Specifically, we theoretically show that matching the Gram matrices of feature maps is equivalent to minimize the Maximum Mean Discrepancy (MMD) with the second order polynomial kernel. Thus, we argue that the essence of neural style transfer is to match the feature distributions between the style images and the generated images. To further support our standpoint, we experiment with several other distribution alignment methods, and achieve appealing results. We believe this novel interpretation connects these two important research fields, and could enlighten future researches.

\end{abstract}
 We propose an Event-B framework for modeling the underlying theoretical foundations of Event-B.
The aim of this framework is to reuse, for Event-B itself, the  refinement development process.
This framework introduces first, a functional kernel through an Event-B context, then, it defines Event-B projects, their static and
dynamic semantics through Event-B machines. We intend to use this framework for the validation
of Event-B plugins related to distribution and for Event-B extensions related to composition and
decomposition.
\end{abstract}
%\setcounter{tocdepth}{3}   % à commenter
\pagestyle{plain} % à commenter


%\begin{section}{Introduction}

Transferring the style from one image to another image is an interesting yet difficult problem. There have been many efforts to develop efficient methods for automatic style transfer~\cite{hertzmann2001image,efros2001image,efros1999texture,shih2014style,kwatra2005texture}. Recently, Gatys \emph{et al.} proposed a seminal work~\cite{neuralart}: It captures the style of artistic images and transfer it to other images using Convolutional Neural Networks (CNN). This work formulated the problem as finding an image that matching both the content and style statistics based on the neural activations of each layer in CNN. It achieved impressive results and several follow-up works improved upon this innovative approaches~\cite{johnson2016perceptual,ulyanov2016texture,ruder2016artistic,ledig2016photo}. Despite the fact that this work has drawn lots of attention, the fundamental element of style representation: the Gram matrix in~\cite{neuralart} is not fully explained. The reason why Gram matrix can represent artistic style still remains a mystery.

In this paper, we propose a novel interpretation of neural style transfer by casting it as a special domain adaptation~\cite{beijbom2012domain,patel2015visual} problem. We theoretically prove that matching the Gram matrices of the neural activations can be seen as minimizing a specific Maximum Mean Discrepancy (MMD)~\cite{mmd}. This reveals that neural style transfer is intrinsically a process of distribution alignment of the neural activations between images. Based on this illuminating analysis, we also experiment with other distribution alignment methods, including MMD with different kernels and a simplified moment matching method. These methods achieve diverse but all reasonable style transfer results. Specifically, a transfer method by MMD with linear kernel achieves comparable visual results yet with a lower complexity. Thus, the second order interaction in Gram matrix is not a must for style transfer. Our interpretation provides a promising direction to design style transfer methods with different visual results. To summarize, our contributions are shown as follows:
\begin{enumerate}
\item First, we demonstrate that matching Gram matrices in neural style transfer~\cite{neuralart} can be reformulated as minimizing  MMD with the second order polynomial kernel.
\item Second, we extend the original neural style transfer with different distribution alignment methods based on our novel interpretation.
\end{enumerate}

\end{section}
\section{Introduction}

     Event-B~\cite{Abrial2010} is a method that has been proposed for building formal models together with their proofs.
As a matter of fact, it has been used for a large range of applications. Nevertheless, it
seems that, in general, it has not been applied to the field of software engineering by itself. 
In this paper, we report on an Event-B meta-framework and  two software engineering applications for which the use of the
Event-B methodology seemed to us worth to apply.  

The rest of the paper is organized as follows.  
%Section 2 gives an overview of the Event-B language. 
Section 2 outlines the main features of an Event-B framework. Section 3 discusses about two 
software applications. In conclusion, Section 4 considers some related work and sketch
future work directions.
 
%\section{A brief overview of Event-B}

%\input{framework}
\section{Towards an Event-B meta-level framework}

The proposed meta-level framework aims at validating Event-B model
transformations. We focus on transformations linked to a top-down,
refinement-based development process. Their goal is to assist the user
in producing refinements of his model through patterns parameterized
with the help of domain specific languages. Thus, a transformation pattern takes
as input an Event-B machine and some parameters. It produces either a
single machine or a set of machines. In the latter case, it is
necessary to model the project level -- not a single machine -- in order to consider the
interaction of the machines of the project. However, to make things simpler, we 
consider neither contexts, nor refinement links between
machines. 
Refinement will be taken into account at the meta level, 
each transformation producing a refinement of the project.

    \subsection{Methodology}

    We now propose a meta-level specification of an Event-B project in
    Event-B itself. The difficulty of such an exercise is to find the
    right level of abstraction and to identify which features should be
    modeled as constants and as variables. It is strongly linked with
    the objectives we have fixed. First, given the patterns we
    envision, predicates and expressions should be left as abstract as
    possible. Second, we target operations which should modify the
    project by adding new machines. Two orthogonal dynamics will thus
    be considered: project contents evolution and project operational
    semantics.  Furthermore, we try to use a refinement-based approach
    to specify the meta-level: its features will be introduced
    incrementally.

    \subsection{The global view}
Figure \ref{mch} describes the overall structure of a machine as a
class diagram. The conversion to Event-B is performed as follows:

\begin{itemize}
  \item \texttt{Machine} is introduced as a set, with
    \texttt{Machines} being the subset of existing machines. 
  \item Machine attributes and operations can be updated and are
    defined as variables.
  \item \texttt{Predicate}, \texttt{Ident} and \texttt{EventName}. \texttt{Ident}
    is partionned into \texttt{Var}, \texttt{Prime} denoting primed
    versions of machine variables and \texttt{Param}.
  \item \texttt{Event} is modeled as a triple with three projections
    (\texttt{Pars}, \texttt{Guard} and \texttt{Action}). 
\end{itemize}

\begin{figure}[hbt]
\centering
\resizebox{\linewidth}{!}{\includegraphics{eVB.pdf}}
\caption{Event-B machines}
\label{mch}
\end{figure}

%\input{functional}
    \subsection{The functional kernel}

           The functional kernel introduces abstraction of predicates and events as Event-B contexts.
A predicate is defined as a set of abstract states. It is mainly characterized by axioms stating the existence
of the \texttt{Free} function returning the set of the free variables of a predicate and the substitution
function. With respect to our specific needs concerning decomposition/composition and distribution
we also assume the existence of a \texttt{Conjuncts} function returning a set of  predicates of which conjunct
is equivalent to the initial predicate.  For instance, the conjuncts of ``p = TRUE'' is ``\{ p = TRUE \}'' and the
conjuncts of   ``p = TRUE $\wedge$ v = 2'' is ``\{ p = TRUE, v = 2 \}''.
An excerpt of of the Predicate context is the following:

\begin{framed}
\begin{small}
\begin{alltt} 
context cPredicate extends cIdent

sets State

constants Predicate Free Subst Proj Conjuncts ...

axioms
  @Predicate_def Predicate = ℙ(State)
  @Free_ty Free ∈ Predicate → ℙ(Ident)
  @Subst_ty Subst ∈ (Ident ⇸ Ident) → (Predicate → Predicate)
  @Proj_ty Proj ∈ ℙ(Ident) → (Predicate → Predicate)
  @Conjuncts_ty Conjuncts ∈ Predicate → ℙ1(Predicate)
  @Conjuncts_ax ∀ p· p ∈ Predicate ⇒  inter(Conjuncts(p)) = p
  @Free_Conjuncts ∀ p· p ∈ Predicate ⇒  union(Free[Conjuncts(p)]) = Free(p)
 ...
\end{alltt}
\end{small}
\end{framed}


    \subsection{The Event-B project structure}

Besides contexts, Event-B projects are modelled through the following
refinement steps:

\begin{itemize}
  \item \texttt{mProject} defines the overall structure of machines
    and a project as a set of machines and provides an event to add a
    machine to a project.

  \item \texttt{static\_semantics} adds wellformedness rules
    concerning the usage of identifiers within predicates. Machine
    addition is restricted to well formed machines.

%% couper en 2???  (invariant?? et preservation de l'invariant) puis
%% semantique op (step)

 \item \texttt{dynamics} adds the invariant preservation property and
    provides a dynamic semantics to a project through the introduction
    of a state and of the \texttt{step} event defining the operational
    semantics of the project.
\end{itemize}

    \subsection{Event-B project and machines}

An Event-B project is seen as a set of machines. Each machine has
variables, an invariant and a set of events indexed by event
names. In order to make easier the meta-level reasoning, we consider
that a machine has a unique invariant and that an event has a unique
guard and a unique action (seen as a before-after predicate). These predicates will be seen as conjunctive later.

\begin{framed}
\begin{small}
\begin{alltt} 
machine mProject sees cMachine cEvent 

variables Machines mVars mInv mEvents 

invariants
  @machines_ty Machines ⊆ Machine
  @mVars_ty mVars ∈ Machines → ℙ(Var)
  @mEvents_ty mEvents ∈ Machines → (EventName ⇸ Event)
  @mInvs_ty mInv ∈ Machines → Predicate
events
  ...
end
\end{alltt}
\end{small}
\end{framed}

The \texttt{mProject} machine also provides the \texttt{new\_machine} event for adding
machines to a project. Its takes seven parameters specifying the set of
machines to be added and for each of them a set of variables, an
invariant, event names, and parameters, guard and action of each event.


   \subsection{The static semantics}

The static semantics specifies visibility constraints for variables
and parameters: 
\begin{itemize}
  \item an invariant of a machine uses variables of this machine\footnote{For the moment, we do not take into account refinements and consequently the gluing invariant.}
  \item a guard of an event can use parameters of this event and
    variables of the  machine  the event belongs to.
  \item an action of an event can use parameters of this event,
    variables of the  machine and their primed versions.
\end{itemize}

\begin{framed}
\begin{small}
\begin{alltt}
machine static_semantics refines mProject
sees cMachine

variables Machines mVars mInv mEvents 

invariants
  @mInv_ctr ∀ m · m ∈ Machines ⇒ Free(mInv(m)) ⊆ mVars(m)
  @mGuards_ctr 
     ∀ m,e· m ∈ Machines ∧ e ∈ dom(mEvents(m))
       ⇒ Free((mEvents(m);Guard)(e)) ⊆ mVars(m) ∪ (mEvents(m);Pars)(e)
  @mActions_ctr 
      ∀ m,e· m ∈ Machines ∧ e ∈ dom(mEvents(m))
       ⇒ Free((mEvents(m);Action)(e)) ⊆ mVars(m) ∪ Next[mVars(m)] ∪ (mEvents(m);Pars)(e)
\end{alltt} 
\end{small}
\end{framed}

   \subsection{The dynamic semantics}

This refinement takes into account the dynamic of a project. First,
standard proof obligations are added to express that the machine
invariant is preserved by each event. The expression of proof obligations
takes advantage of the representation of a predicate as a set:
conjunction and implication are replaced by intersection and set inclusion.
Second the operational semantics
of a project is defined through the introduction of a state for the
subset of machines considered to be active, and a \texttt{step} event modelling the evolution of the
state. The state is declared as a decomposable predicate over machine
variables. It abstracts the usual view of a state as a valuation of each state
variable. Machine invariants should be satisfied by the state.

\begin{framed}
\begin{small}
\begin{alltt}
machine dynamics refines static_semantics
sees cMachine cEvent

variables Machines mVars mInv mEvents state

invariants
  @state_ty state ∈ Machines ⇸ Decomposable     // only defined on active machines
  @state_dync ∀m· m ∈ dom(state) ⇒ state(m) ⊆ mInv(m)
  @free_state ∀m· m ∈ dom(state) ⇒ Free(state(m)) ⊆ mVars(m)
  @mInv ∀m,e· m ∈ Machines ∧ e ∈ dom(mEvents(m))
           ⇒  mInv(m) ∩ (mEvents(m);Guard)(e) ∩ (mEvents(m);Action)(e) ⊆ Subst(Next)(mInv(m))
\end{alltt}
\end{small}
\end{framed}

The \texttt{step} event makes a machine of the project advance by
updating its state. It takes as parameters a machine \texttt{m}, an
event name \texttt{e}, a predicate \texttt{p} specifying the value of
the parameters. The event guards are supposed to be satisfied by the current
state of the machine. Then its state is updated by applying the machine
action. The new state is obtained by suppressing primed in the
projection on primed variables of the conjunction of the old state,
the parameters and action predicates.

\begin{framed}
\begin{small}
\begin{alltt}
  event step
    any m e p
    where
      @m_ty m ∈ dom(state)
      @e_ty e ∈ dom(mEvents(m))
      @p p ∈ Predicate
      @f Free(p) ⊆ Param
      @g state(m) ∩ p ⊆ (mEvents(m);Guard)(e)
    then
      @a state(m) ≔ Subst(Next∼)(Proj(Next[mVars(m)])(state(m) ∩ p ∩ (mEvents(m);Action)(e)))
  end
\end{alltt}
\end{small}
\end{framed}

We also introduce an event to change the active set of machines: some
\textit{old} machines can be replaced by \textit{new} machines taken
in the pool of currently inactive machines. This event can be seen as
a hot replacement of components. It should be transparent. For this
purpose, we suppose that the conjunction of old machine states is
equal to the conjunction of new machine states. A typical application
will be to replace a compound machine by its subcomponents once it has
been split.
%\input{case_studies}
\section{Case studies}

  We have experimented the above meta description on two Event-B model transformations.
The first transformation deals with a safe refinement development process
for distributed applications~\cite{[SBBF16]} . This development
process proposes successive steps for splitting and scheduling complex
events. These steps are
defined by refinement patterns. They are specified through domain
specific languages.  From these specifications, two refinements were
generated. In the first phase of this work, the generated refinements
had to be verified through the Event-B framework, i.e., the Rodin
verification platform. With respect to that work, our  motivation
was to assert that the application of the proposed patterns actually
produce refinements of the source machine, so that the generated
machines are \textit{correct by construction}. Thus, it should not be
necessary to validate these refinements for each application of the
corresponding pattern.
%% C'EST LA MEME APPLICATION: les 2 transfos viennent en amont des
%% plugins de decomposition existants. Elles travaillent sur la vue centralisée
The second transformation deals with Event-B by itself. Actually, the last developments of
Event-B propose to enhance Event-B by decomposition methods. This has lead to two 
proposals: the state-based~\cite{[HA10]}  and the event-based~\cite{[SB10]}. 
Both methods have strong 
theoretical foundations. Moreover, they have been validated by significant applications and have
been both implemented by plugins available through the Rodin platform~\cite{[RCTB11]}. With respect to these
studies, our second motivation was how to \textit{manage the theoretical background} that is required
for the justification of Event-B enhancements like decomposition methods.

%       \subsection{The distribution plugin}

%        \input{shared_event}


%\section{Related Work and Background}\label{sec:related}
\subsection{The Graphics Processing Unit (GPU)}
The GPU of today is a highly parallel, throughput-focused programmable processor. GPU programs (``kernels'') launch over a \emph{grid} of numerous \emph{blocks}; the GPU hardware maps blocks to available parallel cores. Each block typically consists of dozens to thousands of individual \emph{threads}, which are arranged into 32-wide \emph{warps}. Warps run under SIMD control on the GPU hardware. While blocks cannot directly communicate with each other within a kernel, threads within a block can, via a user-programmable 48~kB \emph{shared-memory}, and threads within a warp additionally have access to numerous warp-wide instructions. The GPU's global memory (DRAM), accessible to all blocks during a computation, achieves its maximum bandwidth only when neighboring threads access neighboring locations in the memory; such accesses are termed \emph{coalesced}.
In this work, when we use the term ``\emph{global}'', we mean an operation of device-wide scope. Our term ``\emph{local}'' refers to an operation limited to smaller scope (e.g., within a thread, a warp, a block, etc.), which we will specify accordingly. The major difference between the two is the cost of communication: global operations must communicate through global DRAM, whereas local operations can communicate through lower-latency, higher-bandwidth mechanisms like shared memory or warp-wide intrinsics.
Lindholm et al.~\shortcite{Lindholm:2008:NTA} and Nickolls et al.~\shortcite{Nickolls:2008:SPP} provide more details on GPU hardware and the GPU programming model, respectively.

We use NVIDIA's CUDA as our programming language in this work~\cite{NVIDIA:2016:CUDA}. CUDA provides several warp-wide voting and shuffling instructions for intra-warp communication of threads. All threads within a warp can see the result of a user-specified predicate in a bitmap variable returned by \texttt{\_\_ballot(predicate)}~\cite[Ch.~B13]{NVIDIA:2016:CUDA}. Any set bit in this bitmap denotes the predicate  being non-zero for the corresponding thread. Each thread can also access registers from other threads in the same warp with \texttt{\_\_shfl(register\_name, source\_thread)}~\cite[Ch.~B14]{NVIDIA:2016:CUDA}. Other shuffling functions such as \texttt{\_\_shfl\_up()} or \texttt{\_\_shfl\_xor()} use relative addresses to specify the source thread.
In CUDA, threads also have access to some efficient integer intrinsics, e.g., \texttt{\texttt{\_\_popc()}} for counting the number of set bits in a register.

\subsection{Parallel primitive background}
In this paper we leverage numerous standard parallel primitives, which we briefly describe here. A \emph{reduction} inputs a vector of elements and applies a binary associative operator (such as addition) to reduce them to a single element; for instance, sum-reduction simply adds up its input vector.
The \emph{scan} operator takes a vector of input elements and an associative binary operator, and returns an output vector of the same size as the input vector.
In exclusive (resp., inclusive) scan, output location $i$ contains the reduction of input elements 0 to $i-1$ (resp., 0 to $i$).
Scan operations with binary addition as their operator are also known as \emph{prefix-sum}~\cite{Harris:2007:PPS:nourl}.
Any reference to a multi- operator (multi-reduction, multi-scan) refers to running multiple instances of that operator in parallel on separate inputs. \emph{Compaction} is an operation that filters a subset of its input elements into a smaller output array while preserving the order.

\subsection{Multisplit and Histograms}
Many multisplit implementations, including ours, depend heavily on knowledge of the total number of elements within each bucket (bin), i.e., histogram computation.
Previous competitive GPU histogram implementations share a common philosophy: divide the problem into several smaller sized subproblems and assign each subproblem to a thread, where each thread sequentially processes its subproblem and keeps track of its own \emph{privatized} local histogram.
Later, the local histograms are aggregated to produce a globally correct histogram.
There are two common approaches to this aggregation: 1) using atomic operations to correctly add bin counts together (e.g., Shams and Kennedy~\shortcite{Shams:2007:EHA}), 2)~storing per-thread sequential histogram computations and combining them via a global reduction (e.g., Nugteren et al.~\shortcite{Nugteren:2011:HPP}).
The former is suitable when the number of buckets is large; otherwise atomic contention is the bottleneck.
The latter avoids such conflicts by using more memory (assigning exclusive memory units per-bucket and per-thread), then performing device-wide reductions to compute the global histogram.

The hierarchical memory structure of NVIDIA GPUs, as well as NVIDIA's more recent addition of faster but local shared memory atomics (among all threads within a thread block), provides more design options to the programmer.
With these features, the aggregation stage could be performed in multiple rounds from thread-level to block-level and then to device-level (global) results.
Brown et al.~\shortcite{Brown:2012:MFH} implemented both Shams's and Nugteren's 
aforementioned methods, as well as a variation of their own, focusing only on 8-bit data, considering careful optimizations that make the best use of the GPU, including loop unrolling, thread coarsening, and subword parallelism, as well as others.
Recently, NVIDIA's CUDA Unbound (CUB)~\cite{Merrill:2015:CUB} library has included an efficient and consistent histogram implementation that carefully uses a minimum number of shared-memory atomics to combine per-thread privatized histograms per thread-block, followed by aggregation via global atomics. CUB's histogram supports any data type (including multi-channel 8-bit inputs) with any number of bins.

Only a handful of papers have explored multisplit as a standalone primitive. He et al.~\cite{He:2008:RJG} implemented multisplit by reading multiple elements with each thread, sequentially computing their histogram and local offsets (their order among all elements within the same bucket and processed by the same thread), then storing all results (histograms and local offsets) into memory. Next, they performed a device-wide scan operation over these histogram results and scattered each item into its final position. Their main bottlenecks were the limited size of shared memory, an expensive global scan operation, and random non-coalesced memory accesses.%
\footnote{On an NVIDIA 8800 GTX GPU, for 64 buckets, He et al.\ reported 134~Mkeys/sec. As a very rough comparison, our GeForce GTX 1080 GPU has 3.7x the memory bandwidth, and our best 64-bucket implementation runs 126 times faster.}

Patidar~\cite{Patidar:2009:SPD} proposed two methods with a particular focus on a large number of buckets (more than 4k): one based on heavy usage of shared-memory atomic operations (to compute block level histogram and intra-bucket orders), and the other by iterative usage of basic binary split for each bucket (or groups of buckets). Patidar used a combination of these methods in a hierarchical way to get his best results.%
\footnote{On an NVIDIA GTX280 GPU, for 32 buckets, Patidar reported 762~Mkeys/sec. As a very rough comparison, our GeForce GTX 1080 GPU has 2.25x the memory bandwidth, and our best 32-bucket implementation runs 23.5 times faster.}
Both of these multisplit papers focus only on key-only scenarios, while data movements and privatization of local memory become more challenging with key-value pairs.


%\section{Conclusion}
\label{sec:conclusion}

This paper presents a matrix factorization based approach to text
outlier analysis. The approach is designed to adjust well to the
widely varying structures in different localities of the data, and
therefore provides more robust methods than competing models. The
approach has the potential to be applied to other domains with
similar structure, and as a specific example, we provide experiments
on market  basket data. We also presented extensive experimental
results, which illustrate the superiority of the approach.  
Our code can be downloaded from 
\url{https://github.com/ramkikannan/outliernmf} and 
tried with any text dataset. 

In this paper, we had a parallel implementation using the
Matlab's parallel computing toolbox to run in multicore environments.
In the future, we would like to explore a scalable implementation
of our algorithm. The solution is embarrassingly parallelizable,
and would like to experiment in web scale data. One of the potential
extension is incorporating temporal and spatial aspects into the model.
Such an extension, make the solution applicable to emerging 
applications such as topic detection and streaming data. 
%In the recent times,
%approximate matrix factorization techniques are explored by
%randomly sampling the input matrix. We can reduce the computation
%time for very large matrices using such sampling techniques. We would like
%to explore a sampling based solution for our model. 
We experimented
the solution primarily on text data and market basket data. In future
work, we will extend this broader approach to other domains such as
video data.

\section{Related Work and Conclusion}
  
    It is interesting to cite related works which have some connections
with ours.  First, Iliasov et al.~\cite{[ITLR09]} is a pioneering work
for dealing with the automation of development steps. For this
purpose, they propose the notion of refinement patterns.  Such
refinement patterns contain a syntactic description, applicability
conditions and proof obligations ensuring correctness preservation.
Unlike our approach where we stayed within an Event-B world,
\cite{[ITLR09]} adopt specific languages for representing Event-B
models and their so-called transformation rules. Last, the reuse of
the Event-B proof engine is not immediate.  Also, Cata{\~{n}}o et
al.~\cite{[CRW13]} adopt the so-called \textit{own medicine approach}
in the sense that they adopt Event-B for formalizing Event-B and JML
and the Rodin platform to discharge their proof obligations. With
respect to that our work is similar. However, their model is mainly
functional and their transformations are defined as functions. Their
correctness is stated through theorems. With respect to Event-B, we
have gone further since we have adopted a state-based approach. The
dynamic semantics as well as model transformations are defined as
events.  The correctness of the dynamic semantics and of the
transformations are obtained for free through the Event-B refinement.
Moreover, Cata{\~{n}}o et al.~\cite{[CRW13]} are concerned neither by
the validation of refinement patterns nor by the semantics of
composition.

   To conclude, Event-B proposes a refinement-based development method. In this
paper, we have studied how to support such a development method by
itself in order to formalize the underlying theoretical background:
the so-called meta level. The elaborated framework can also be used to
support Event-B enhancements as composition and decomposition methods.
As future work, we envision to broaden the coverage of our
framework. We are also interested in formalizing the links between
Event-B and temporal\cite{Hoang2016} or temporized~\cite{[GBF13]}
logics. More generally, the excplicit description of dynamic
behaviours through temporized patterns~\cite{[ADL12]} within an
Event-B framework looks challenging.
\bibliographystyle{abbrv}
\bibliography{biblio}

%\newpage

%\tableofcontents


\end{document}
