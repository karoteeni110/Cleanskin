\documentclass[12pt]{amsart}
\usepackage[top=1.5in, bottom=1.5in, left=1.4in, right=1.4in]{geometry}  
\usepackage{geometry}                % See geometry.pdf to learn the layout options. There are lots.
\geometry{letterpaper}                   % ... or a4paper or a5paper or ... 
%\geometry{landscape}                % Activate for for rotated page geometry
%\usepackage[parfill]{parskip}    % Activate to begin paragraphs with an empty line rather than an indent
\usepackage{graphicx}
\usepackage{amssymb}
\usepackage{epstopdf}
\usepackage{color}
\usepackage{url}
\usepackage{verbatim}
\usepackage[lined,algonl,boxed,norelsize]{algorithm2e}
\DeclareGraphicsRule{.tif}{png}{.png}{`convert #1 `dirname #1`/`basename #1 .tif`.png}

\title[Zonotopal bijections for regular matroids]{Geometric bijections for regular matroids, zonotopes, and Ehrhart theory}
\author{Spencer Backman, Matthew Baker, Chi Ho Yuen}
\date{\today}  % Activate to display a given date or no date


\def\M{\mathcal{M}}
\def\B{\mathcal{B}}
\def\J{\mathcal{J}}
\def\P{\mathcal{P}}
\def\I{\mathcal{I}}
\newcommand{\Ofrak}{{\mathfrak O}}



%%%%%%%%%%%%%%%%%%%%%%%%%%%%
% Blackboard Bold Alphabet %
%%%%%%%%%%%%%%%%%%%%%%%%%%%%
\renewcommand{\AA}{\mathbb{A}}
\newcommand{\CC}{\mathbb{C}}
\newcommand{\FF}{\mathbb{F}}
\newcommand{\PP}{\mathbb{P}}
\newcommand{\NN}{\mathbb{N}}
\newcommand{\QQ}{\mathbb{Q}}
\newcommand{\RR}{\mathbb{R}}
\newcommand{\ZZ}{\mathbb{Z}}

\numberwithin{equation}{section}
\theoremstyle{definition}

\newtheorem{theorem}{Theorem}[subsection]
\newtheorem{lemma}[theorem]{Lemma}
\newtheorem{definition}[theorem]{Definition}
\newtheorem{corollary}[theorem]{Corollary}
\newtheorem{conjecture}[theorem]{Conjecture}
\newtheorem{remark}[theorem]{Remark}
\newtheorem{proposition}[theorem]{Proposition}
\newtheorem{example}[theorem]{Example}


\newcommand{\Col}{\operatorname{Col}}
\newcommand{\Jac}{\operatorname{Jac}}
\newcommand{\Pic}{\operatorname{Pic}}
\newcommand{\la}{{\langle}}
\newcommand{\ra}{{\rangle}}
\newcommand{\indeg}{\operatorname{indeg}}
\newcommand{\outdeg}{\operatorname{outdeg}}
\newcommand{\sign}{\operatorname{sign}}
\newcommand{\coker}{\operatorname{coker}}
\newcommand{\row}{\operatorname{row}}

\DeclareRobustCommand{\rchi}{{\mathpalette\irchi\relax}}
\newcommand{\irchi}[2]{\raisebox{\depth}{$#1\chi$}}


\begin{document}

\begin{abstract}
Let $M$ be a {\em regular matroid}.  The {\em Jacobian group} ${\rm Jac}(M)$ of $M$ 
is a finite abelian group whose cardinality is equal to the number of {\em bases} of $M$. This group
generalizes the definition of the Jacobian group (also known as the critical group or sandpile group) $\Jac(G)$ of a graph $G$ (in which case bases of the corresponding regular matroid are spanning trees of $G$).  

There are many explicit combinatorial bijections in the literature between the Jacobian group of a graph ${\rm Jac}(G)$ and spanning trees.  However, most of the known bijections use {\em vertices} of $G$ in some essential way and are inherently ``non-matroidal''. In this paper, we construct a family of explicit and easy-to-describe bijections between the Jacobian group of a regular matroid $M$ and bases of $M$, many instances of which are new even in the case of graphs. We first describe our family of bijections in a purely combinatorial way in terms of orientations; more specifically, we prove that the Jacobian group of $M$ admits a canonical simply transitive action on the set ${\mathcal G}(M)$ of circuit-cocircuit reversal classes of $M$, and then define a family of combinatorial bijections $\beta_{\sigma,\sigma^*}$ between ${\mathcal G}(M)$ and bases of $M$.  (Here $\sigma$ (resp. $\sigma^*$) is an {\em acyclic signature} of the set of circuits (resp. cocircuits) of $M$.)
We then give a geometric interpretation of each such map $\beta=\beta_{\sigma,\sigma^*}$ in terms of zonotopal subdivisions which is used to verify that $\beta$ is indeed a bijection. 



Finally, we give a combinatorial interpretation of lattice points in the zonotope $Z$; by passing to dilations we obtain
a new derivation of Stanley's formula linking the Ehrhart polynomial of $Z$ to the Tutte polynomial of $M$. 
\end{abstract}

\maketitle

\section{Introduction}


\subsection{The main bijection in the case of graphs}

Let $G$ be a connected finite graph.
The {\em Jacobian group} ${\rm Jac}(G)$ of $G$ (also called the sandpile group, critical group, etc.) is a finite abelian group canonically associated to $G$ whose cardinality equals the number of spanning trees of $G$. Although there is no canonical bijection\footnote{Consider, for example, a 3-cycle: since ${\rm Aut}(G)$ acts transitively on the set of spanning trees, there can be no distinguished member of this 3-element set corresponding to the identity element of $\Jac(G)$.} between $\Jac(G)$ and the set ${\mathcal T}(G)$ of spanning trees of $G$, many constructions of combinatorial bijections starting with some fixed additional data are known.  We mention, for example: the Cori--Le Borgne bijections that use an ordering of the edges as well as a fixed vertex \cite{cori2001burning}, Perkinson, Yang and Yu's bijections that use an ordering of the vertices \cite{perkinson2015dfs}, and Bernardi's bijections that use a cyclic ordering of the edges incident to each vertex \cite{bernardi2008tutte}.

In this paper we describe a new family of combinatorial bijections between $\Jac(G)$ and ${\mathcal T}(G)$.  Our bijections are very simple to state, though proving that they are indeed bijections is not so simple.  Another feature is that our bijections are formulated in a ``purely matroidal'' way, and in particular they generalize from graphs to {\em regular matroids}.  We will first state the main result of this paper in the language of graphs, and then give the generalization to regular matroids.

What we will in fact do is establish a family of bijections between ${\mathcal T}(G)$ and the set ${\mathcal G}(G)$ of {\em cycle-cocycle equivalence classes} of orientations of $G$.  The latter was introduced by Gioan \cite{gioan2007enumerating,gioan2008circuit} and is known to be a torsor\footnote{This means that there is a canonical simply transitive group action of $\Jac(G)$ on ${\mathcal G}(G)$.} for $\Jac(G)$ in a canonical way. By fixing a class in ${\mathcal G}(G)$ to correspond to the identity element of $\Jac(G)$, we then obtain a bijection between $\Jac(G)$ and ${\mathcal T}(G)$. (By definition, ${\mathcal G}(G)$ is the set of equivalence classes of orientations of $G$ with respect to the equivalence relation generated by directed cycle reversals and directed cut reversals. We will write $[{\mathcal O}]$ to denote the equivalence class containing an orientation ${\mathcal O}$.)

To state our main bijection for graphs, let ${\mathcal C}(G)$ (resp. ${\mathcal C}^*(G)$) denote the set of simple cycles (resp. simple cuts, i.e., bonds) of $G$, and define a {\bf cycle signature} (resp. {\bf cut signature}) on $G$ to be a choice, for each $C \in {\mathcal C}(G)$ (resp. $C \in {\mathcal C}^*(G)$), of an orientation of $C$, identified with an element of the cycle lattice $\Lambda(G)$ (resp. the cut lattice $\Lambda^*(G)$). We call a cycle signature $\sigma$ (resp. cut signature $\sigma^*$) {\bf acyclic} if whenever $a_C$ are nonnegative reals with
\[
\sum_{C \in {\mathcal C}(G)} a_C \sigma(C) = 0
\]
in $\Lambda(G)$ 
(resp. $\sum_{C \in {\mathcal C}^*(G)} a_C \sigma^*(C) = 0$ in $\Lambda^*(G)$) we have $a_C = 0$ for all $C$.

\begin{example} \label{ex:smallestedge}
Fix a total order and reference orientation on the set $E(G)$ of edges of $G$, and orient each simple cycle (resp. simple cut) $C$ compatibly with the reference orientation of the smallest edge in $C$.  This gives an acyclic signature of ${\mathcal C}(G)$ (resp. ${\mathcal C}(G)$).

\end{example}


Recall that if $T$ is a spanning tree of $G$ and $e \not\in T$ (resp. $e \in T$), there is a unique simple cycle  $C(T,e)$ (resp. simple cut $C^*(T,e)$) contained in $T \cup \{ e \}$ (resp. containing $T \backslash \{ e \}$), called the {\em fundamental cycle} (resp. {\em fundamental cut}) associated to $T$ and $e$. With this notation in place, we can now state our main bijection in the case of graphs:

\begin{theorem} \label{thm:mainbijectionforgraphs}
Let $G$ be a connected finite graph, and fix acyclic signatures $\sigma$ and $\sigma^*$ of ${\mathcal C}(G)$ and ${\mathcal C}^*(G)$, respectively. Given a spanning tree $T \in {\mathcal T}(G)$, let ${\mathcal O}(T)$ be the orientation of $G$ in which we orient each $e \not\in T$ according to its orientation in $\sigma(C(T,e))$ and each $e \in T$ according to its orientation in $\sigma^*(C^*(T,e))$. Then the map $T \mapsto [{\mathcal O}(T)]$ is a bijection between ${\mathcal T}(G)$ and ${\mathcal G}(G)$.
\end{theorem}

The bijection in Theorem~\ref{thm:mainbijectionforgraphs} appears to be new even in the special case where $\sigma$ and $\sigma^*$ are defined as in Example~\ref{ex:smallestedge}.

For another application of Theorem~\ref{thm:mainbijectionforgraphs}, suppose that $G$ is a plane graph and define $\sigma$ by orienting each simple cycle of $G$ counterclockwise.  Similarly, define $\sigma^*$ by orienting each simple cycle of the dual graph $G^*$ clockwise and composing with the natural bijection between oriented cuts of $G$ and oriented cycles of $G^*$.
In this case, the simply transitive action of $\Jac(G)$ on ${\mathcal T}(G)$ afforded by Theorem~\ref{thm:mainbijectionforgraphs} coincides with the ``Bernardi torsor'' defined in \cite{bakeryao2016torsor} and {\em a posteriori} with the ``rotor-routing torsor'' defined in \cite{chan2015rotor,chan2015duality}.  In particular, we get a new ``geometric'' proof of the bijectivity of the Bernardi map.

\subsection{Generalization to regular matroids} \label{sec:introregular}

As mentioned previously, an interesting feature of the bijection given by Theorem~\ref{thm:mainbijectionforgraphs} is that it admits a direct generalization to {\em regular matroids}.

Regular matroids are a particularly well-behaved and widely studied class of matroids which contain graphic (and co-graphic) matroids as a special case. More precisely, a regular matroid can be thought of as an equivalence class of totally unimodular integer matrices\footnote{An $r \times m$ integer matrix $A$ with $r \leq m$ is called {\em totally unimodular} if every $k \times k$ submatrix has determinant in $\{ 0, \pm 1 \}$ for all $1 \leq k \leq r$. We say that totally unimodular $r \times m$ matrices $A,A'$ are {\em equivalent} if one can transform $A$ into $A'$ by multiplying on the left by an $r \times r$ unimodular matrix $U$, then permuting columns or multiplying columns by $-1$.}.  See \S\ref{sec:regularmatroids} for further details.

If $G$ is a graph, one can associate a regular matroid $M(G)$ to $G$ by letting $A$ be the modified adjacency matrix of $G$, where we choose a vertex $q \in V(G)$ and the rows of $A$ are indexed by $V(G) \backslash \{ q \}$.
By a theorem of Whitney, the equivalence class of $A$ determines the graph $G$ up to ``2-isomorphism'' (and in particular determines $G$ up to isomorphism if $G$ is assumed to be $3$-connected).

Let $M$ be a regular matroid.  In Section 4.3 of his Ph.D. thesis, Criel Merino defined the {\em critical group} (which we will call the {\em Jacobian}) ${\rm Jac}(M)$ of $M$, generalizing the critical group of a graph.  The group ${\rm Jac}(M)$ is a finite abelian group whose cardinality is equal to the number of {\em bases} of $M$.\footnote{The fact that these cardinalities are equal is essentially a translation of the natural extension of Kirchhoff's Matrix-Tree theorem to regular matroids \cite{maurer1976matrix},\cite[Theorem 4.3.2]{merino1999matroids}. A ``volume proof'' of the Matrix-Tree theorem for regular matroids based on zonotopal subdivisions is given in \cite{dall2014polyhedral}.  These authors do not consider the problem of giving explicit combinatorial bijections between bases of $M$ and the Jacobian group.}


One can also define the set ${\mathcal C}(M)$ of {\em signed circuits} of $M$ (resp. the set ${\mathcal C}^*(M)$ of {\em signed cocircuits} of $M$) is in a way which generalizes the corresponding objects when $M=M(G)$.  Similarly, one has a set $B(M)$ of {\bf bases} of $M$, generalizing the notion of spanning tree for graphs, and a set ${\mathcal G}(M)$ of cycle-cocycle equivalence classes generalizing the corresponding set for graphs. By results of Merino and Gioan, the cardinalities of $\Jac(M)$, $B(M)$, and ${\mathcal G}(M)$ all coincide. (Our results in this paper give independent proofs of these facts.)

Generalizing the known case of graphs \cite{backman2014riemann}, we prove:

\begin{theorem} \label{thm:torsortheorem}
${\mathcal G}(M)$ is canonically a torsor for $\Jac(M)$.
\end{theorem}


In view of this result, in order to construct a bijection between elements of $\Jac(M)$ and bases of $M$, it suffices to give a bijection between $B(M)$ and ${\mathcal G}(M)$.
One can generalize the notion of acyclic signature and fundamental cycles (resp. cuts) in a straightforward way from graphs to regular matroids. 
Theorem~\ref{thm:mainbijectionforgraphs} then admits the following generalization to regular matroids:


\begin{theorem} \label{thm:mainbijectionforregularmatroids}
Let $M$ be a regular matroid, and fix acyclic signatures $\sigma$ and $\sigma^*$ of ${\mathcal C}(M)$ and ${\mathcal C}^*(M)$, respectively. Given a basis $B \in B(M)$, let ${\mathcal O}(B)$ be the orientation of $M$ in which we orient each $e \not\in B$ according to its orientation in $\sigma(C(B,e))$ and each $e \in B$ according to its orientation in $\sigma^*(C^*(B,e))$. Then the map $B \mapsto [{\mathcal O}(B)]$ gives a bijection $\beta : B(M) \to {\mathcal G}(M)$. 
\end{theorem}



\medskip

Most known combinatorial bijections between elements of ${\rm Jac}(G)$ and spanning trees of a graph $G$ do not readily extend to the case of regular matroids, as they use vertices of the graph in an essential way.
The only other work we are aware of giving explicit bijections between elements of $\Jac(M)$ and bases of a regular matroid $M$ are the papers of Gioan and Gioan--Las Vergnas \cite{gioan2002correspond,gioan2005activity}\footnote{Technically speaking, Gioan and Las Vergnas do not produce a bijection between bases and elements of $\Jac(M)$; they produce a bijection between $B(M)$ and $\mathcal{X}(M;\sigma,\sigma^*)$, where $\sigma$ and $\sigma^*$ are determined by a total order on the edges and a reference orientation as in Example~\ref{edgeorder}; see \S\ref{sec:brief_overview} for the definition of $\mathcal{X}(M;\sigma,\sigma^*)$.} and the as-yet unpublished recent work of Shokrieh \cite{farbod2016draft}. Our family of combinatorial bijections appears to be quite different from those of Gioan--Las Vergnas. 

\subsection{Brief overview of the proof of the main combinatorial bijections} \label{sec:brief_overview}

Although the statement of Theorem~\ref{thm:mainbijectionforgraphs} and its generalization Theorem~\ref{thm:mainbijectionforregularmatroids} to regular matroids $M$ are completely combinatorial, we do not know any simple combinatorial proof.  Our proof involves the geometry of a zonotopal subdivision associated to a matrix $A$ representing $M$.

Concretely, fix a totally unimodular $r \times m$ matrix $A$ representing $M$, where $r$ is the rank of $A$. Denote by $V^* \subseteq \RR^E$ the row space of $A$ and by $\pi_{V^*}$ the orthogonal projection from $\RR^E$ to $V^*$. Let $u_e \in \RR^E$ be the standard coordinate vector corresponding to $e \in E$.  The {\em column zonotope} $Z_A \subset \RR^r$ (resp. {\em row zonotope} $\widetilde{Z_A} \subset \RR^E$) associated to $A$ is defined to be the Minkowski sum of the columns of $A$ (resp. the Minkowski sum of the orthogonal projections $\pi_{V^*}(u_e)$ for $e \in E$).
One checks easily that the linear transformation $L : v \mapsto Av$ gives an isomorphism from $V^*$ to $\RR^r$ taking $\widetilde{Z_A}$ to $Z_A$.
In particular, the $r$-dimensional zonotopes $\widetilde{Z_A}$ and $Z_A$ are isomorphic via a unimodular transformation.

An {\em orientation} $\mathcal{O}$ of $M$ is a function $E \to \{ -1, 1 \}$.
An orientation $\mathcal{O}$ is {\em compatible} with a signed circuit $C$ of $M$ if $\mathcal{O}(e) = C(e)$ for all $e$ in the support of $C$.

If $\mathcal{O}$ is an orientation and $C$ is a signed circuit compatible with $\mathcal{O}$, we can perform a {\em circuit reversal} taking $\mathcal{O}$ to the orientation $\mathcal{O}'$ 
defined by $\mathcal{O}'(e) = \mathcal{O}(e)$ if $e$ is not in the support of $C$ and $\mathcal{O}'(e) = -\mathcal{O}(e)$ if $e$ is in the support of $C$.

Let $\sigma$ be an acyclic signature of ${\mathcal C}(M)$.
We say that $\mathcal{O}$ is {\em $\sigma$-compatible} if every signed circuit $C$ of $M$ compatible with $\mathcal{O}$ is oriented according to $\sigma$.

\begin{theorem}
Every circuit-reversal equivalence class of orientations contains a unique $\sigma$-compatible orientation.
\end{theorem}

The connection between $\sigma$-compatible orientations and the zonotopes defined above is given by the following result.
For the statement, given an orientation $\mathcal{O}$ of $M$ and $e \in E$, define $w_e \in \RR^r$ to be $0$ if $\mathcal{O}(e) = -1$ and 
to be the $e^{\rm th}$ column of $A$ if $\mathcal{O}(e) = 1$.
Define $\psi(\mathcal{O}) \in Z_A$ by

\begin{equation}
\psi(\mathcal{O}) := \sum_{e \in E} w_e \in Z_A.
\end{equation}

\begin{theorem}
The map $\psi$ induces a bijection between circuit-reversal classes of orientations of $M$ and lattice points of the zonotope $Z_A$.
\end{theorem}

Fix a reference orientation $\mathcal{O}_0$ of $M$.
Each acyclic signature $\sigma$ of ${\mathcal C}(M)$ gives rise to a subdivision of $Z_A$ into smaller zonotopes $Z(B)$, one for each basis $B$ of $M$,
in the following way.

Let $B$ be a basis of $M$.  For each $e \not\in B$, define $v_e \in V^*$ to be $0$ if the reference orientation of $e$ coincides with the orientation of $e$
in $\sigma(C(B,e))$, and to be the $e^{\rm th}$ column of $A$ otherwise.
Define 
\[
Z(B) := \sum_{e \in B} [0,A_e] + \sum_{e \not\in B} v_e \subseteq Z_A \subset \RR^r.
\]

Note that $Z(B)$ is itself a zonotope, as it is congruent via translation to $\sum_{e \in B} [0,A_e]$.
Let $\widetilde{Z(B)}$ be the corresponding subset $L^{-1}(Z(B))$ of $\widetilde{Z_A}$.
The following result can be paraphrased as saying that the various $Z(B)$'s give a {\em zonotopal subdivision}\footnote{Also known in the literature on zonotopes as a {\em tiling} of $Z_A$.} $\Sigma$ of $Z_A$.

\begin{theorem} 
The union of $Z(B)$ over all bases $B$ of $M$ is equal to $Z_A$, and if $B,B'$ are distinct bases then the relative interiors of $Z(B)$ and $Z(B')$ are disjoint.
\end{theorem}

Similarly, via the map $L$, the various $\widetilde{Z(B)}$'s give a zonotopal subdivision $\widetilde{\Sigma}$ of $\widetilde{Z_A}$.

\medskip

We now explain briefly how these results are used to prove 
Theorem~\ref{thm:mainbijectionforregularmatroids}.  

Let $\sigma,\sigma^*$ be acyclic signatures of $C(M)$ and $C^*(M)$, respectively. An orientation is called {\em $(\sigma,\sigma^*)$-compatible} if it is both $\sigma$-compatible and $\sigma^*$-compatible, and we denote the set of such orientations by $\mathcal{X}(M;\sigma,\sigma^*)$.

\begin{theorem} \label{thm:SScompatibleorientation}
Let $\hat{\beta}$ be the map which sends a basis $B$ to the orientation $\mathcal{O}_B$ defined in Theorem~\ref{thm:mainbijectionforregularmatroids}. Let $\chi$ be the map which sends an orientation $\mathcal{O}$ to its circuit-cocircuit reversal class $[\mathcal{O}]$, so that  
 $\beta=\chi\circ\hat{\beta}$.  
\begin{enumerate}
\item The image of $\hat{\beta}$ is contained in $\mathcal{X}(M;\sigma,\sigma^*)$, and $\hat{\beta}$ gives a bijection between $B(M)$ and $\mathcal{X}(M;\sigma,\sigma^*)$.
\item The map $\chi$ restricted to $\mathcal{X}(M;\sigma,\sigma^*)$ induces a bijection between $\mathcal{X}(M;\sigma,\sigma^*)$ and $\mathcal{G}(M)$. 
\end{enumerate}
\end{theorem}

\begin{remark}
The proofs of Theorem~\ref{thm:torsortheorem} and Theorem~\ref{thm:mainbijectionforregularmatroids} do not assume {\em a priori} that  $|B(M)|=|\mathcal{X}(M;\sigma,\sigma^*)|=|\mathcal{G}(M)|=|\Jac(M)|$ for a regular matroid $M$, thus our work provides an independent proof of these equalities. Furthermore, we will show in Theorem~\ref{thm:realizablemainthm} below that the equality $|B(M)|=|\mathcal{X}(M;\sigma,\sigma^*)|$ continues to hold under the weaker assumption that $M$ is representable over $\mathbb{R}$.
\end{remark}

\medskip

Choose a vector $w \in {\mathbb R}^E$ which is compatible with $\sigma^*$, in the sense that $w \cdot \sigma^*(C) > 0$ for each cocircuit $C$ of $M$.
(The existence of such a vector is guaranteed by a simple application of the Farkas Lemma.) 
Since oriented circuits are orthogonal to oriented cocircuits, and the row space $V^*$ of $A$ is the span of the oriented cocircuits, 
modifying $w$ by an element of $(V^*)^\perp$ will give the same inner product $w \cdot \sigma^*(C)$ for each circuit $C$ of $M$.
Therefore it is natural to consider the orthogonal projection $w'$ of $w$ onto $V^*$.


Note that the zonotopal subdivision $\widetilde{\Sigma}$ of $\widetilde{Z_A}$ depends only on $\sigma$ (and the reference orientation $\mathcal{O}_0$) and the vector $w'$ depends only on $\sigma^*$.

The following theorem shows that the combinatorially defined map $\hat{\beta} : {\mathcal B}(M) \to {\mathcal X}(M)$ can be interpreted geometrically as first identifying a basis with a maximal cell in our zonotopal subdivision and then applying a ``shifting map''.


\medskip


\begin{theorem} \label{thm:betaphiagree}
\begin{enumerate}
\item Let $B$ be a basis of $M$. For all sufficiently small $\epsilon > 0$ the image of $\widetilde{Z(B)}$ under the map $v \mapsto v + \epsilon w'$ contains a unique lattice point $\widetilde{z_B}$ of $\widetilde{Z_A}$, which corresponds to a unique $(\sigma,\sigma^*)$-compatible discrete orientation $\mathcal{O}'_B$.
\item The map $\phi$ which takes each basis $B$ to the orientation $\mathcal{O}'_B$ coincides with the map $\hat{\beta}$ appearing in the statement of Theorem~\ref{thm:mainbijectionforregularmatroids}, and hence $\hat{\beta}$ gives a bijection between $B(M)$ and $\mathcal{X}(M;\sigma,\sigma^*)$.
\end{enumerate}
\end{theorem}

Theorem~\ref{thm:mainbijectionforregularmatroids} is a simple consequence of Theorem \ref{thm:SScompatibleorientation} and Theorem \ref{thm:betaphiagree}. 

\subsection{Continuous orientations}
It is useful to give a combinatorial interpretation of {\em all} points of the zonotope $Z_A$ (not just the lattice points) in terms of equivalence clases of {\em continuous orientations} of $M$.

Recall that an {\em orientation} (which we will refer to as a {\em discrete orientation} whenever there is a risk of confusion) is a function $E \to \{ -1, 1 \}$.
We define a {\em continuous orientation} of $M$ to be a function $E \to [-1,1]$.

In \S\ref{sec:equivclassorient}, we define the notions of continuous circuit and cocircuit reversals.

If we fix an acyclic signature $\sigma$ of $C(M)$, there is a natural way (generalizing the discrete case) to pick out a distinguished {\em $\sigma$-compatible} orientation from each continuous circuit-reversal class; see \S\ref{sec:signatures} for the definition.  We will show:

\begin{theorem} 
There is a natural bijection between circuit-reversal classes of continuous orientations of $M$ and points of the zonotope $Z_A$.  
\end{theorem}

We will use this result to give an alternate description of the zonotopal subdivisions $\Sigma$ and $\widetilde{\Sigma}$ of $Z_A$ and $\widetilde{Z_A}$, respectively, which were defined above.

\subsection{A Partial Extension to Matroids Representable over $\mathbb{R}$} \label{sec:realizablecase}

Although the equality $|B(M)|=|\mathcal{G}(M)|=|\Jac(M)|$ does not hold for general oriented matroids (indeed, $\Jac(M)$ is not even well-defined in the general case), the notions of acyclic circuit/cocircuit signatures and $(\sigma,\sigma^*)$-compatible orientations continue to make sense whenever $M$ is representable over ${\mathbb{R}}$. Furthermore, the geometric setup used to prove Theorem~\ref{thm:betaphiagree}, as well as the first half of Theorem~\ref{thm:SScompatibleorientation}, does not require $M$ to be regular but only representable over $\mathbb{R}$. Therefore we have the following result, which will be proved in Section~\ref{sec:proof}:


\begin{theorem} \label{thm:realizablemainthm}
Let $M$ be a matroid which is representable over $\mathbb{R}$, and let $\sigma,\sigma^*$ be acyclic signatures of $C(M),C^*(M)$, respectively. Then the map $\hat{\beta}:B(M)\rightarrow\mathcal{X}(M;\sigma,\sigma^*)$ is a bijection.
\end{theorem}

\subsection{Random sampling of bases} 

As in \cite{baker2013chip}, any computable bijection between bases and elements of $\Jac(M)$ gives rise an algorithm for randomly sampling bases of $M$. The idea is simple: by computing the Smith Normal Form of a matrix $A$ representing $M$, we can explicitly compute $\Jac(M)$ as a direct sum of finite abelian groups, and it is clear how to uniformly sample elements of such a group.

In order to make this into a practical method, one needs efficient algorithms for computing both the element of $\Jac(M)$ associated to a given basis and vice-versa.  In \S\ref{sec:bijectioncomputable} and Proposition~\ref{prop:groupactioncomputable}, we provide polynomial-time computable algorithms for these tasks with respect to the family of bijections given by Theorem~\ref{thm:mainbijectionforregularmatroids}.
Both our algorithm for computing the inverse of the map $\beta$ (resp. $\hat{\beta}$) from Theorem~\ref{thm:mainbijectionforregularmatroids} and our algorithm for computing the group action in Proposition~\ref{prop:groupactioncomputable} use ideas from linear programming.  

In the Appendix, we present a simple recursive procedure based on deletion and contraction for randomly sampling bases in a regular matroid which does not make use of Theorem~\ref{thm:mainbijectionforregularmatroids}.  This algorithm was inspired by a similar folklore result in the case of graphs \cite{kalaiblog}.

\subsection{Connections to Ehrhart theory and the Tutte polynomial}

Every matroid $M$ of rank $r$ has an associated {\em Tutte polynomial} $T_M(x,y)$, and every lattice polytope $P$ (e.g. the zonotope $Z_A$) has an associated {\em Ehrhart polynomial} $E_P(q)$ which counts the number of lattice points in positive integer dilates of $P$.  Using the relationship between $Z_A$ and $\sigma$-compatible (discrete or continuous) orientations of $M$, we obtain a new proof of the following identity originally due to Stanley:

\begin{equation} \label{eq:stanley}
E_Z(q) = q^r T_{M}(1 + 1/q,1).
\end{equation}

The proof involves defining a ``dilation'' $qM$ of $M$ for each positive integer $q$, with associated zonotope $qZ_A$. 

We also describe a direct bijective proof (without appealing to Ehrhart reciprocity) of the fact that the number of interior lattice points in $qZ_A$ is 
\[
q^r T_{M}(1-1/q,1).
\]



\subsection{Related literature}

The study of zonotopal tilings, i.e. tilings of a zonotope by smaller zonotopes, is a classical topic in the theory of oriented matroids first initiated by Shepard \cite{shephard1974zonotopes}.   The central theorem in this area is the Bohne-Dress Theorem \cite{bohne1992zonotopaler,dress1989oriented}, which states that the poset of zonotopal tilings ordered by refinement is isomorphic to the poset of 1-element lifts of the associated oriented matroid M.  We believe it should be possible to use the Bohne-Dress Theorem and the results in \cite{bayer1997discriminantal} to prove that the set of tilings of $Z_A$ arising from acyclic orientations of $C(M)$ is precisely the set of regular tilings of $Z_A$ (which correspond to the realizable lifts in the Bohne-Dress Theorem).  It should also be possible to use the results of \cite{birkett2000cayley} to give a more high-level explanation, in terms of Lawrence polytopes, Lawrence ideals, and Gr{\"o}bner bases, of the relationship between continuous and discrete $\sigma$-compatible orientations.  Farbod Shokrieh has indicated to us that his forthcoming paper \cite{farbod2016draft} may shed some light on these connections.

\section{Background}

\subsection{Regular matroids} \label{sec:regularmatroids}


In this section we recall the definition of regular matroids and collect some basic well-known facts about them.
We assume that the reader is familiar with the basic theory of matroids; some standard references include the book on matroids by Oxley \cite{oxley2006matroid} and the book on oriented matroids by Bj{\"o}rner et.~al.~\cite{bjorner1999oriented}.

\medskip

Let $A$ be a matrix with entries in a field $F$ whose columns are indexed by a set $E$.
We define $M_F(A)$ to be the matroid on $E$ for which 
a subset $X$ of $E$ is independent in $M_F(A)$ if and only if the corresponding columns of $A$ are linearly independent over $F$.

A matroid $M$ on $E$ is called {\em representable} over $F$ if $M=M_F(A)$ for some matrix $A$ as above.
In this case we say that $A$ {\em represents} $M$.
The rank of the matrix $A$ is equal to the rank of $r$ of $M$.

An $r \times m$ matrix $A$ of rank $r$ with integer entries is called {\em unimodular} if every $r \times r$ submatrix has determinant in $\{ 0, \pm 1 \}$, and {\em totally unimodular} if every $k \times k$ submatrix has determinant in $\{ 0, \pm 1 \}$ for all $1 \leq k \leq r$.

\begin{theorem}[{cf. \cite[Theorem~3.1.1]{white1987combinatorial}}] 
Let $M$ be a matroid.  The following are equivalent:
\begin{enumerate}
\item $M$ is representable over $\QQ$ by a totally unimodular matrix.
\item $M$ is representable over $\QQ$ by a unimodular matrix.
\item $M$ is representable over every field $F$.
\item $M$ is orientable\footnote{$M$ is called {\em orientable} if it is the underlying matroid of some oriented matroid.} and representable over the 2-element field $\FF_2$.

\end{enumerate}
A matroid satisfying any of these four equivalent conditions is called {\em regular}.
\end{theorem}



The following lemma (see \cite[Corollary 10.1.4]{oxley2006matroid}) is the key fact used to show that various definitions in the subject are independent of the choice of a totally unimodular matrix $A$ representing $M$:

\begin{lemma} \label{lem:transformation}
If $A,A'$ are totally unimodular $r \times m$ matrices representing $M$, one can transform $A$ into $A'$ by multiplying on the left by an $r \times r$ unimodular matrix $U$, then permuting columns or multiplying columns by $-1$.  
\end{lemma}

\medskip

If $M$ is a regular matroid of rank $r$ on $E$ and $A$ is any $r \times m$ totally unimodular matrix representing $M$ over $\QQ$, we define
$\Lambda_A(M) := {\rm ker}(A) \cap \ZZ^E$.  By Lemma~\ref{lem:transformation}, the isometry class of this lattice depends only on $M$, and not on the choice of the matrix $A$.  It is denoted by $\Lambda(M)$ and called the {\em circuit lattice} of $M$.

Similarly, we define $\Lambda^*_A(M)$ to be the intersection of the row space of $A$ with $\ZZ^E$, or equivalently the $\ZZ$-span of the rows of $A$.
The isometry class of this lattice also depends only on $M$.  It is denoted by $\Lambda^*(M)$ and called the {\em cocircuit lattice} of $M$.
If $M^*$ is the dual matroid of $M$ we have $\Lambda(M^*) \cong \Lambda^*(M)$.
(For proofs of all these statements, see \cite[\S{4.3}]{merino1999matroids} or \cite[\S{2.3}]{su2010lattice})

\medskip

The Jacobian group $\Jac(M)$ is defined to be the determinant group of $\Lambda(M)$, i.e., $\Jac(M) = \Lambda(M)^\# / \Lambda(M)$ where $\Lambda^\#$ is the {\em dual lattice} of $\Lambda$, i.e.,
\[
\Lambda^\# = \{ x \in \Lambda \otimes \QQ \; : \; \langle x,y \rangle \in \ZZ \; \forall \; y \in \Lambda \}.
\]

There are canonical isomorphisms 
\begin{equation} \label{eq:canonisom}
\Lambda(M)^\# / \Lambda(M) \cong \Lambda^*(M)^\# / \Lambda^*(M) \cong \frac{\ZZ^E}{\Lambda_A(M) \oplus \Lambda_A^*(M)}
\end{equation}
for every totally unimodular matrix $A$ representing $M$ (cf. \cite[Lemma~1 of \S{4}]{bacher1997lattice}).

In particular, there is a canonical isomorphism $\Jac(M^*) \cong \Jac(M)$.

The order of $\Jac(M)$ is equal to the number of {\em bases} of the matroid $M$ (cf. \cite[Theorem 4.3.2]{merino1999matroids}).
Moreover, we have $|\Jac(M)| = |{\rm det}(A^T A)|$ (cf. \cite[p.317]{godsil2013algebraic}), and in fact $\Jac(M)$ can naturally be identified with the cokernel of $A^T A$:

\begin{proposition}
The map $\frac{\ZZ^E}{\Lambda_A(M) \oplus \Lambda_A^*(M)}\rightarrow\coker(AA^T)$ given by $[\gamma]\mapsto [A\gamma]$ is well-defined and is an isomorphism.
\end{proposition}

\begin{proof}
The map is well-defined because $A(\Lambda_A(M) \oplus \Lambda_A^*(M))=A(\Lambda_A^*(M))=A(\Col_{\mathbb{Z}}A^T)=\Col_{\mathbb{Z}}AA^T$, the equality also shows the map is injective. It is surjective because $Ax=b$ has a solution in $\ZZ^E$ for every $b\in\mathbb{Z}^r$, using the unimodularity of $A$.
\end{proof} 

Let $C(\underline{M})$ (resp. $C^*(\underline{M})$) be the set of circuits (resp. cocircuits) of $M$.  
Let $A$ be any $r \times m$ totally unimodular matrix representing $M$ over $\QQ$.
An element $\alpha \in \Lambda_A(M)$ (resp. $\Lambda_A^*(M)$) is called a {\em signed circuit} (resp. {\em signed cocircuit}) of $M$ if $\alpha \neq 0$,
all coordinates of $\alpha$ are in $\{ 0, \pm 1 \}$, and the support of $\alpha$ is a circuit (resp. cocircuit) of $M$.
We let $C_A(M)$ (resp. $C^*_A(M)$) denote the set of signed circuits (resp. signed cocircuits) of $M$.
The notion of signed circuit (resp. signed cocircuit) is in fact intrinsic to $M$, independent of the choice of $A$, and thus it makes sense to speak of $C(M)$ and $C^*(M)$
as subsets of $\Lambda(M)$ and $\Lambda^*(M)$, respectively\footnote{This follows from \cite[Proposition 12]{su2010lattice}, which asserts that a nonzero element $\alpha \in \Lambda(M)$ is a circuit if and only if whenever $\alpha = \beta + \gamma$ with $\beta,\gamma \in \Lambda(M)$ nonzero, we have $\langle \beta, \gamma \rangle < 0$.} (cf. \cite[Lemma~10 and Proposition~12]{su2010lattice} and \cite[Theorem~4.3.4]{merino1999matroids}).

\medskip

There is a natural map $C(M) \to C(\underline{M})$ taking $\alpha \in C_A(M)$ to its support (with respect to any choice of $A$).
This map induces a bijection $C(M) / \langle \pm 1 \rangle \to C(\underline{M})$, i.e, for every circuit $\underline{C}$ of $M$ there are precisely two signed
circuits $\pm C$ with ${\rm supp}(C) = \underline{C}$.
(cf. \cite[Lemma 8]{su2010lattice} and \cite[Theorem~4.3.5]{merino1999matroids})
The same holds for cocircuits.


\subsection{Equivalence classes of orientations}
\label{sec:equivclassorient}

\medskip

An {\em (discrete) orientation} of a regular matroid $M$ is a map from the ground set $E$ of $M$ to $\{ -1, 1 \}$.  

An orientation ${\mathcal O}$ is {\em compatible} with a signed circuit $C$ of $M$ if ${\mathcal O}(e) = C(e)$ for all $e$ in the support of $C$.

The following is a basic fact about orientations of regular matroids (which in fact holds more generally for {\em oriented matroids} \cite[Corollary 3.4.6]{bjorner1999oriented}):

\begin{proposition} \label{prop:orientdecomp}
Given an orientation $\mathcal{O}$ of $M$ and $e \in E$, exactly one of the following holds:
\begin{enumerate}
\item There is a signed circuit $C$ of $M$ with $e \in {\rm supp}(C)$ such that ${\mathcal O}(f) = C(f)$ for every $f$ in the support of $C$.  In this case we say that $e$ belongs to the {\bf circuit part} of $\mathcal{O}$.
\item There is a signed cocircuit $C^*$ of $M$ with $e \in {\rm supp}(C^*)$ such that ${\mathcal O}(f) = C^*(f)$ for every $f$ in the support of $C^*$. In this case we say that $e$ belongs to the {\bf cocircuit part} of $\mathcal{O}$.
\end{enumerate}
\end{proposition}

The {\em circuit reversal system} is the equivalence relation on the set $O(M)$ of all orientations of $M$ generated by {\em circuit reversals},
in which we reverse the sign of ${\mathcal O}(e)$ for all $e$ in (the support of) some signed circuit $C$ compatible with ${\mathcal O}$.
We can make the same definitions for cocircuits by replacing $M$ with its dual matroid $M^*$.

\medskip

The {\em circuit-cocircuit reversal system} is the equivalence relation generated by both circuit and cocircuit reversals.  It is a theorem of Gioan (originally proved by a deletion-contraction argument) that the number of circuit-cocircuit equivalence classes of orientations is equal to the number of bases of $M$.

\medskip

We will give a {\em bijective} proof of Gioan's theorem, defining a natural action of ${\rm Jac}(M)$ on ${\mathcal G}(M)$ and proving:

\begin{theorem} \label{thm:torsor}
The action of ${\rm Jac}(M)$ on ${\mathcal G}(M)$ is simply transitive, i.e., ${\mathcal G}(M)$ is naturally a {\em torsor} for ${\rm Jac}(M)$.
\end{theorem}

If we fix an element of ${\mathcal G}(M)$ (e.g. by fixing a reference orientation of $M$), the simply transitive action in Theorem~\ref{thm:torsor} induces a bijection between ${\rm Jac}(M)$ and ${\mathcal G}(M)$.  Thus, in order to define a bijection between ${\rm Jac}(M)$ and ${\mathcal B}(M)$, it will suffice to construct a bijection between ${\mathcal G}(M)$ and ${\mathcal B}(M)$.  

\medskip

\subsection{Signatures and non-degeneracy}
\label{sec:signatures}

For this, we introduce the important notion of an {\em acyclic signature} on the set of circuits (resp. cocircuits) of $M$.  Recall that since every regular matroid is (canonically) orientable, the signed circuits of $M$ naturally come in pairs.  Let $C(\underline{M})$ (resp. $C^*(\underline{M})$) denote the set of circuits (resp. cocircuits) of the matroid $\underline{M}$ underlying $M$, and let $C(M)$ (resp. $C^*(M)$) denote the set of signed circuits (resp. signed cocircuits) of $M$.

\medskip

We define a {\em signature} of $C(\underline{M})$ to be a section 
$\sigma : C(\underline{M}) \to C(M)$.
We think of $\sigma$ as corresponding to a choice of orientation for each (unsigned) circuit $C_i$ of the matroid underlying $M$.
A signature $\sigma$ of $C(\underline{M})$ is called {\em acyclic} if the only solution to $\sum_{C_i \in C(\underline{M})} a_i \sigma(C_i) = 0$ with the $a_i$ non-negative numbers is the trivial solution where all $a_i$ are equal to zero.
(Signatures and non-degeneracy for $C^*(\underline{M})$  are defined analogously.) 

\medskip

As a simple consequence of {\em Gordan's alternative} in the theory of linear programming \cite[p.~478]{bjorner1999oriented} (which itself is a corollary of Farkas' lemma), we have the following criterion/alternative description of an acyclic signature. For the statement, fix a totally unimodular matrix $A$ representing $M$.

\begin{lemma} \label{lem:gordan}
Let $\sigma$ be a signature of $C(M)$. Then $\sigma$ is acyclic if and only if there exists $w \in {\mathbb R}^m$ such that $w\cdot\sigma(C)>0$ for each circuit $C$ of $M$. 
\end{lemma}

In the situation of Lemma~\ref{lem:gordan}, we say that $w$ {\em induces} $\sigma$. 
By the orthogonality of $C(M)$ and $C^*(M)$, given any pair of acyclic signatures $\sigma,\sigma^*$ of $C(M)$ and $C^*(M)$, respectively, there exists $w \in {\mathbb R}^m$ that induces both $\sigma$ and $\sigma^*$.

\medskip

Here are some interesting examples of acyclic signatures.

\medskip

\begin{example}\label{edgeorder}
Fix a total order $e_1 < \cdots < e_m$ of $E$ and a reference orientation ${\mathcal O}$ of $M$, and orient each circuit $C \in C(\underline{M})$ compatibly with the reference orientation of the smallest element in $C$.  This gives an acyclic signature of $C(\underline{M})$.

Indeed, suppose the signature is not acyclic and take some nontrivial expression $\sum_{C \in {\mathcal C}(G)} a_C \sigma(C) = 0$. Let $e$ be the minimum element appearing in some circuit in the support of this expression.  Then the element $e$ must be appear with different orientations in at least two different circuits, and thus one of these circuits is not oriented according to $\sigma$, a contradiction.
\end{example}



\begin{example}\label{planar}
Suppose $M$ is the cycle matroid of a plane graph $G$, and orient the circuits of $M$ counterclockwise.  This is an acyclic signature of $C(\underline{M})$.  

This example is in fact a special case of Example~\ref{edgeorder}.  Indeed, let $G^*$ the plane dual of $G$, let $q^*$ be the vertex of $G^*$ corresponding to the unbounded face of $G$, and fix a spanning tree $T^*$ of $G^*$.  Let $\mathcal{O}^*$ be any orientation of $G^*$ in which the edges of $T^*$ are oriented away from $q^*$, and fix any total order on $E(G^*)$ in which every edge of the rooted tree $T^*$ has a larger label than its ancestors. Using the natural bijection\footnote{If $e$ and $e^*$ are dual edges of $G$ and $G^*$, respectively, then given an orientation for $e^*$ we orient $e$ by rotating the orientation of $e^*$ clockwise locally near the crossing of $e$ and $e^*$.} between oriented edges of $G$ and of $G^*$, this gives an orientation $\mathcal{O}$ of $G$ and a total order $<$ on $E(G)$.  Then the cycle signature $\sigma$ associated to $(\mathcal{O},<)$ by the rule in Example~\ref{edgeorder} will orient every simple cycle of $G$ counterclockwise.
\end{example}






\subsection{The main combinatorial bijection}

If $B$ is a basis for $M$, $e \not\in B$, and $f \in B$, let $Z(e,B)$ denote the {\em fundamental circuit} corresponding to $e$ and $B$ (the unique circuit of $\underline{M}$ contained in $B \cup \{ e \}$), and (dually) let $Z^*(f,B)$ denote the {\em fundamental cocircuit} corresponding to $f$ and $B$.

\begin{theorem} \label{thm:combbijection}
Let $M$ be a regular matroid and let $\sigma,\sigma^*$ be acyclic signatures of $C(M)$ and $C^*(M)$, respectively.
Given $B \in {\mathcal B}(M)$, define a corresponding orientation $\mathcal{O}_B$ of $M$
by orienting each $e \not\in B$ according to its orientation in $\sigma(Z(e,B))$ and each $f \in B$ according to its orientation in $\sigma^*(Z^*(f,B))$.
Define $\beta : {\mathcal B}(M) \to {\mathcal G}(M)$ by letting $\beta(B)$ be the class of $\mathcal{O}_B$ in ${\mathcal G}(M)$.
Then $\beta$ is a bijection.
\end{theorem}


We now illustrate some interesting special cases of Theorem~\ref{thm:combbijection}.

\begin{example}\label{edgeorderB}
As in Example~\ref{edgeorder},
fix a total order $e_1 < \cdots < e_m$ of $E$ and a reference orientation ${\mathcal O}$ of $M$.  Orient each circuit (resp.~cocircuit) $C$ of $M$ compatibly with the reference orientation of the smallest element in $C$.  This gives an acyclic pair $(\sigma,\sigma^*)$.  Even when $M$ is the matroid associated to a graph $G$, 
the corresponding family of bijections provided by Theorem~\ref{thm:combbijection} appears to be new.
\end{example}

\begin{example} \label{ex:planarbijectionB}
In the special case where $G$ is a plane graph with planar dual $G^*$, it is natural to let $\sigma$ be the map which orient the circuits of $G$ counterclockwise and let $\sigma^*$ be the map which orients the cocircuits of $G$ compatibly with the clockwise orientation on the circuits of $G^*$.  
In this case, the bijection $\beta$ in in Theorem~\ref{thm:combbijection} coincides with the ``Bernardi bijection'' from
\cite{bakeryao2016torsor}.  This follows from \cite[Theorem 15]{yuen2015geometric}.  
\end{example}

\begin{example}
Let $G$ be a graph, and fix a vertex $q$ of $G$. In \cite{an2014canonical}, the authors prove that the break divisors of $G$ are the divisors associated to $q$-connected orientations offset by a chip at $q$. In other words (in the notation of \cite[Lemma 3.3]{an2014canonical}), a divisor $D$ is a break divisor if and only if $D = (q) + \nu_{\mathcal O}$ for some $q$-connected orientation $\mathcal O$.   They also show that break divisors of the corresponding metric graph $\Gamma$ induce a canonical subdivision of the $g$-dimensional torus $\Pic^g(\Gamma)$ into paralleletopes indexed by spanning trees of $G$, with the vertices of the subdivision corresponding to the break divisors of $G$. By applying a small generic shift to the vertices, this yields a family of ``geometric bijections'' between break divisors and spanning trees (cf.~\cite[Remark 4.26]{an2014canonical}).

We claim that the geometric bijections defined in \cite{an2014canonical} can be thought of as special cases of the bijections afforded by Theorem~\ref{thm:combbijection}.  To see this,
note first that by \cite[Theorem 10]{yuen2015geometric}, each geometric bijection gives rise in a natural way to an acyclic orientation $\sigma$ of the cycles of $G$.  To orient the cocycles of $G$, we fix a spanning tree $T_0$ of $G$.  Orient the edges of $T_0$ away from $q$ and label them $e_1$ through $e_{n-1}$ in a way such that every edge has a larger label than its ancestors.  Extend this data on $T_0$ arbitrarily to a total order and reference orientation of $E(G)$.  Let $\sigma^*$ be the corresponding acyclic orientation of the set of cocircuits of $G$ given by Example~\ref{edgeorder}.  
Given a spanning tree $T$, the orientation $\mathcal{O}_T$ associated to the pair $(\sigma,\sigma^*)$ by Theorem~\ref{thm:combbijection} will have the property that every edge $e$ in $T$ (considered as a tree rooted at $q$) is oriented away from $q$, and therefore $\mathcal{O}_T$ is $q$-connected \cite[Section 3]{backman2014partial}.  Let $D_T = \nu_{\mathcal{O}_T} + (q)$ be the corresponding break divisor.
Then $T \mapsto D_T$ will be the geometric bijection we started with.
\end{example}


\section{Matroids over $\mathbb{R}$ and the main combinatorial bijection}\label{sec:proof}

Throughout this section, $M$ will denote a matroid which is representable over $\mathbb{R}$.  
Note in particular that every regular matroid has this property.

\subsection{Continuous circuit reversals and the zonotope associated to a representation of $M$}
\label{sec:continuousreversals}

The main goal of this section is to prove Theorem~\ref{thm:realizablemainthm}, which (when specialized to regular matroids) is a major component in the proof of Theorem~\ref{thm:mainbijectionforregularmatroids}. 

Our proof is geometric. In order to explain the basic idea, we fix once and for all a real $r \times m$ matrix $A$ representing $M$, where $r$ is the rank of $M$ and the columns of $A$ are indexed by the elements of the ground set $E$ of $M$.

\medskip

We first briefly explain how certain important notions that we introduced for regular matroids extend more generally to matroids 
representable over $\mathbb{R}$. 

For every circuit $C$ of $M$, the elements in $\ker(A)$ whose support is $C$, together with the zero vector, form a one-dimensional subspace $U_C$ in $\ker(A)$. 
Conversely, the support of any support-minimal nonzero element of $\ker(A)$ corresponds to a circuit of $M$. 
The two rays of $U_C$ correspond to the two possible orientations of $C$. Hence we may identify a signed circuit of $M$ with an arbitrary vector in the ray, unless otherwise specified. 

The same holds for cocircuits if we replace $\ker(A)$ by the row space of $A$.

The definition of an acyclic circuit (resp.~cocircuit) signature follows verbatim from the discussion in Section~\ref{sec:signatures}, except that in the equation $\sum_{C_i \in C(\underline{M})} a_i \sigma(C_i) = 0$, each $\sigma(C_i)$ should be interpreted as a vector in $\ker(A)$ (resp.~$\row(A)$).

\medskip

We next state a general lemma relating elements of $\ker(A)$ to circuits of $M$.

\begin{lemma} \label{lem:circuitdecomposition}
Let $u\in\mathbb{R}^E$ be a vector in $\ker(A)$. Then $u$ can be written as a sum of signed circuits $\sum \lambda_C C$ with $\lambda_C>0$, such that the support of each $C$ is inside the support of $u$ and for each $e\in C$, the signs of $e$ in $C$ and $u$ agree.
\end{lemma}

\begin{proof} By restricting $A$ to the support of $u$ and negating columns, we may assume without loss of generality that all coordinates of $u$ are positive. The linear program $\min\{1^Tv: Av=0,v\geq 0, v_1=u_1\}$ is feasible as $u$ is a vector satisfying the constraints, so there exists an optimal solution which represents some signed circuit $C$. Pick the maximum $\lambda_C$ such that $\lambda_C C\leq u$ coordinate-wise. Since $u=\lambda_C C+(u-\lambda_C C)$ and the support of $u-\lambda_C C\in\ker(A)$ is smaller than the support of $u$, the result follows by induction.
\end{proof}

\medskip


A {\em continuous orientation} ${\mathcal O}$ of $M$ is a function $E \to [-1,1]$, and the {\em $e$-th coordinate} of $\mathcal{O}$ is the value $\mathcal{O}(e)$.  If ${\mathcal O}(e) \in \{-1,1\}$ for all $e \in M$, we say that ${\mathcal O}$ is a {\em discrete orientation}.

A continuous orientation ${\mathcal O}$ is {\em compatible} with a signed circuit $C$ of $M$ if ${\mathcal O}(e) \neq -\sign(C(e))$ for all $e$ in the support of $C$. 

Given a continuous orientation ${\mathcal O}$ compatible with a signed circuit $C$, a {\em continuous circuit reversal} with respect to $C$ replaces ${\mathcal O}$ by a new continuous orientation ${\mathcal O}-\epsilon C$ for some $\epsilon > 0$.  
(In particular, we require $\epsilon$ to be small enough so that $({\mathcal O}-\epsilon C)(e) \in [-1,1]$ for all $e \in E$; usually we choose the maximum of such $\epsilon$, so that at least one $e\in C$ satisfies $({\mathcal O}-\epsilon C)(e)=-\sign(C(e))$.)

The {\em continuous circuit reversal system} is the equivalence relation on the set $CO(M)$ of all continuous orientations of $M$ generated by all possible continuous circuit reversals. We can make the same definitions for cocircuits by replacing $M$ with its dual matroid $M^*$.

\medskip

Next we define the {\em (column) zonotope $Z_A$ associated to $A$} to be the Minkowski sum of the columns of $A$ (thought of as line segments in ${\mathbb R}^r$), i.e.,
\[
Z_A = \{ \sum_{i=1}^m c_i v_i \; : \; 0 \leq c_i \leq 1 \}
\]
where $v_1,\ldots,v_r$ are the columns of $A$.\footnote{Some authors consider variations on this zonotope, e.g. $\sum_{i=1}^m[-v_i,v_i]$, $\sum_{i=1}^m[-v_i/2,v_i/2]$, or $\sum_{i=1}^m[v_i^-,v_i^+]$, where $v^-$ and $v^+$ are the negative and positive parts of $v$, respectively.}


\begin{remark} \label{rmk:graphicmatrix}
When $M=M(G)$ is a graphic matroid, it is usually more convenient to take $A$ to be the full adjacency matrix of $G$, rather than a modified adjacency matrix with one row removed, when defining the corresponding zonotope.  This has the advantage of producing a canonically defined object, and since all of these different zonotopes are isomorphic, there is little harm in doing this.
\end{remark}

\medskip

There are several important connections between the zonotope $Z_A$ and equivalence classes of orientations of $M$.
For the statement, given $\alpha \in [-1,1]$ we denote by $\hat\alpha$ the real number $\frac{1}{2}(\alpha + 1) \in [0,1]$.
Define $\psi : O(M) \to Z_A$ be the map taking an orientation ${\mathcal O}$ (thought of as an element of $[-1,1]^m$) to 
\begin{equation} \label{eq:psidef}
\psi({\mathcal O}) := \sum_{i=1}^m \widehat{{\mathcal O}(e_i)} v_i \in Z_A.
\end{equation}

\medskip

\begin{proposition} \label{prop:ctslatticepointprop}
The map $\psi$ gives a bijection between continuous circuit-reversal classes of continuous orientations of $M$ and points of the zonotope $Z_A$.
\end{proposition}

\begin{proof} By definition, $\psi$ sends every continuous orientation to some point in $Z_A$, and $\psi$ is surjective. By the orthogonality of circuits and cocircuits, two continuous orientations in the same circuit-reversal class map to the same point of $Z_A$, so it remains to show the converse. Suppose $\psi(\mathcal{O})=\psi(\mathcal{O}')$. By Lemma~\ref{lem:circuitdecomposition}, $\mathcal{O}-\mathcal{O}'$ can be written as a weighted sum of signed circuits in which each signed circuit is compatible with $\mathcal{O}$, and $\mathcal{O}$ can be transformed to $\mathcal{O}'$ via the corresponding continuous circuit-reversals in any order (using the weights as the $\epsilon$'s).
\end{proof}

\medskip


\subsection{Distinguished orientations within each equivalence class} \label{sec:DistOrient}

If we fix an acyclic signature $\sigma$ of $C(M)$, there is a natural way to pick out a distinguished continuous orientation from each continuous circuit-reversal class.

Define a continuous orientation ${\mathcal O}$ to be {\em $\sigma$-compatible} if every signed circuit $C$ of $M$ compatible with ${\mathcal O}$ is oriented according to $\sigma$.  

\medskip

\begin{theorem} \label{thm:continuouscompatible}
Let $\sigma$ be an acyclic signature of $C(M)$. Then each continuous circuit-reversal class $M$ contains a unique $\sigma$-compatible continuous orientation.
\end{theorem}

\begin{proof} Recall that since $\sigma$ is acyclic, there exists $w\in\mathbb{R}^E$ such that $w\cdot\sigma(C)>0$ for every circuit $C$ of $M$. Consider the function $P(\mathcal{O}'):=w\cdot\mathcal{O}'$. If $-\sigma(C)$ is compatible with $\mathcal{O}$ for some circuit $C$, then performing a continuous circuit-reversal with respect to $-\sigma(C)$ strictly increases the value of $P$, so every maximizer of $P$ inside a class (if exists) must be $\sigma$-compatible. But the set of continuous orientations in a continuous circuit-reversal class, which can be identified with a closed subset of $CO(M)\cong[-1,1]^m$, is compact. 
As $P$ is continuous, a maximizer of $P$, and hence a $\sigma$-compatible continuous orientation, must exist in every continuous circuit-reversal class.

Now suppose there are two distinct $\sigma$-compatible continuous orientations $\mathcal{O},\mathcal{O}'$ in a continuous circuit-reversal class. By Lemma~\ref{lem:circuitdecomposition}, $\mathcal{O}$ can be transformed to $\mathcal{O}'$ via a series of continuous circuit-reversals in which each signed circuit involved is compatible with $\mathcal{O}$, hence agrees with $\sigma$. If the last signed circuit involved in the series of reversals is $C$, then $-C$ is a signed circuit compatible with $\mathcal{O}'$. Therefore $-C$ agrees with $\sigma$ as well, which is a contradiction.
This proves the uniqueness of the $\sigma$-compatible orientation in each class.
\end{proof}

\begin{remark}
By interpreting $\sigma$-compatible orientations as maximizers of the linear function $P$, it is easy to see that the map $\mu:Z_A\rightarrow CO(M)$, which takes a point $z$ of $Z_A$ to the unique $\sigma$-compatible continuous orientation in the continuous circuit-reversal class corresponding to $z$, is a continuous section to the map $\psi$.
\end{remark}

We will call orientations that are compatible with both $\sigma$ and $\sigma^*$ {\em $(\sigma,\sigma^*)$-compatible orientations}. 

\medskip

The set of {\em discrete} $(\sigma,\sigma^*)$-compatible orientations will be denoted by $\mathcal{X}(M;\sigma,\sigma^*)$.
In \S\ref{discretestuffforregularmatroids}, we will establish an analogue of Theorem~\ref{thm:continuouscompatible} for discrete orientations of {\em regular} matroids.


\subsection{Bi-orientations and bases}

Let ${\mathcal O}$ be a continuous orientation of $M$.  We call an element $e \in E$ {\em bi-oriented} with respect to ${\mathcal O}$ if 
${\mathcal O}(e) \in (-1,1)$.

Note that if we {\em orient} any bi-oriented element $e$ in a $\sigma$-compatible continuous orientation $\mathcal{O}$, i.e., we set $\mathcal{O}(e)$ equal to either 1 or $-1$, the new continuous orientation is still $\sigma$-compatible. 


\begin{proposition} \label{prop:uniqueext}
Let $\sigma$ be an acyclic signature of $C(M)$.
\begin{enumerate}
\item If ${\mathcal O}$ is a $\sigma$-compatible continuous orientation then the set of $e \in E$ which are bi-oriented with respect to ${\mathcal O}$ is independent in $M$.
\item If $B$ is a basis for $M$ and $b : B \to (-1,1)$ is any function, there is a unique $\sigma$-compatible continuous orientation 
${\mathcal O} = {\mathcal O}(B,b)$ such that ${\mathcal O}(e) = b(e)$ for all $e \in B$ and ${\mathcal O}(e) \in \{ \pm 1 \}$ for all $e \not\in B$.
\end{enumerate}
\end{proposition}

\begin{proof} For the first part, suppose the set $S$ of bi-oriented elements in a continuous orientation $\mathcal{O}$ is not independent. Then $S$ contains some circuit $C$, and $\mathcal{O}$ is compatible with both orientations of $C$, so $\mathcal{O}$ is not $\sigma$-compatible.

For the uniqueness assertion in (2), note that each element not in $B$ must be oriented in agreement with the orientation of its fundamental circuit given by $\sigma$. 
Combining this observation with the function $b : B \to (-1,1)$ gives a continuous orientation $\mathcal{O}$, which we claim is $\sigma$-compatible. If not, then $\mathcal{O}$ is compatible with $-\sigma(C)$ for some circuit $C$. Without loss of generality, we may assume that $|C\setminus B|\neq 0$ is minimum among all such circuits. Pick any $e\in C\setminus B$ and let $C'$ be the fundamental circuit of $e$ with respect to $B$. Then $\mathcal{O}$ is compatible with $\sigma(C')$. By choosing suitable vector representatives of the signed circuits $C,C'$, we may assume $\sigma(C)(e)=-\sigma(C')(e)$, and using Lemma~\ref{lem:circuitdecomposition} we write $-\sigma(C)-\sigma(C')=\sum_{D} \lambda_D D$ with $D$'s being signed circuits and $\lambda_D$'s positive. Since $\sigma(C)+\sigma(C')+\sum_{D} \lambda_D D=0$, at least one such $D$ is oriented opposite to $\sigma$ by acyclicity. Then we note that $D$ is compatible with $\mathcal{O}$: each element of $D$ is either in $B$ (which is bi-oriented in $\mathcal{O}$), or from $C\setminus C'$ and oriented as in $-\sigma(C)$ (which is compatible with $\mathcal{O}$). However, $D\setminus B\subset (C\setminus B)\setminus\{e\}$, contradicting the minimality of $C$.
\end{proof}


\subsection{Polyhedral subdivision of the zonotope}
\label{sec:polyhedralsubdivision}

Let $\sigma$ be an acyclic signature of $C(M)$.
For each basis $B$ of $M$, let $CO^\circ(B)$ be the set of $\sigma$-compatible continuous orientations of the form ${\mathcal O}(B,b)$ as $b$ ranges over all possible $b : B \to (-1,1)$.  Let $Z^\circ(B) = \psi(CO^\circ(B))$ be the projection of $CO^\circ(B)$ to $Z_A$, and let $Z(B)$ be the topological closure of $Z^\circ(B)$ in $Z_A$.  
Finally, let $CO(B) = \mu(Z(B))$ be the closure of $CO^\circ(B)$ in $CO(M)$.

\begin{theorem} \label{thm:zonotopedecomp}
\begin{enumerate}
\item The union of $Z(B)$ over all bases $B$ of $M$ is equal to $Z_A$, and if $B,B'$ are distinct bases then $Z^\circ(B)$ and $Z^\circ(B')$ are disjoint.
\item The collection of $Z(B)$ as $B$ varies over all bases $B$ for $M$ gives a polyhedral subdivision of $Z_A$ whose vertices (i.e., $0$-cells) correspond via $\psi$ to the $\sigma$-compatible discrete orientations of $M$. 
\end{enumerate}
\end{theorem}

\begin{proof} The only non-trivial part is the first half of (1). By Proposition \ref{prop:ctslatticepointprop} every point of $Z_A$ is of the form $\psi(\mathcal{O})$ for some continuous orientation $\mathcal{O}$, and by Theorem \ref{thm:continuouscompatible} we may assume $\mathcal{O}$ is $\sigma$-compatible. Hence by Proposition \ref{prop:uniqueext}, it suffices to show that if the set $\hat{B}$ of bi-oriented elements in $\mathcal{O}$ do not form a basis, then we can bi-orient one or more elements in $\mathcal{O}$ while maintaining $\sigma$-compatiblility; by induction, we will end up with the bi-oriented elements forming a basis $B$ of $M$, which implies that $\psi(\mathcal{O})$ is a limit point of $Z^\circ(B)$.

Suppose that for every $e\not\in\hat{B}$ such that $\hat{B}\cup\{e\}$ is independent in $M$, bi-orienting $e$ in $\mathcal{O}$ (in an arbitrary way) will cause the new continuous orientation $\mathcal{O}_e$ to no longer be $\sigma$-compatible. Then for every such $e$, $\mathcal{O}_e$ is compatible with $-\sigma(C_e)$ for some circuit $C_e$ of $M$ containing $e$. Pick, among all such elements $e$ and circuits $C_e$, the $C_e$ with the maximal value of $w\cdot\sigma(C_e)$, here we always choose the ``normalized'' $\sigma(C_e)$ with $|\sigma(C_e)(e)|=1$ for comparison.

The circuit $C_e$ must contain another element $f\not\in\hat{B}$ such that $\hat{B}\cup\{f\}$ is independent in $M$, hence by assumption there exists some circuit $C_f$ containing $f$ such that $\mathcal{O}_f$ is compatible with $-\sigma(C_f)$. The sign of $\sigma(C_e)$ and $\sigma(C_f)$ over $f$ are different, so we can normalize $\sigma(C_f)$ such that $\sigma(C_f)(f)=-\sigma(C_e)(f)$. By Lemma~\ref{lem:circuitdecomposition}, $-\sigma(C_e)-\sigma(C_f)$ can be written as a weighted sum $\sum_{i=1}^k \lambda_iC_i$ of signed circuits $C_i$'s with positive $\lambda_i$'s.  Each such signed circuit $C_i$ that does not contain $e$ must be compatible with $\mathcal{O}$ (hence $\sigma$), while those circuits that contain $e$ would at least be compatible with $\mathcal{O}_e$. Since $w\cdot(\sum_{i=1}^k \lambda_iC_i)=w\cdot(-\sigma(C_e)-\sigma(C_f))<0$, some circuit containing $e$ must appear in the sum and it is not compatible with $\mathcal{O}$; in particular, the sign of $e$ in $-\sigma(C_e)-\sigma(C_f)$ agrees with $-\sigma(C_e)$, and $0<|-\sigma(C_e)(e)-\sigma(C_f)(e)|\leq|-\sigma(C_e)(e)|=1$ as the sign of $\sigma(C_e)$ and $\sigma(C_f)$ over $e$ are different. Without loss of generality, the circuits containing $e$ are $C_1,C_2,\ldots,C_j$ and we further normalize them so that $C_i(e)=-\sigma(C_e)(e)$, and $\sum_{i=1}^j\lambda_i\leq 1$.

Now we have 
\[
w\cdot (\sum_{i=1}^j \lambda_iC_i)=w\cdot \left(-\sigma(C_e)-\sigma(C_f)-\sum_{i=j+1}^k \lambda_iC_i \right)<-w\cdot\sigma(C_e)<0,
\]
i.e., some there exists some $C_i$ with $1\leq i\leq j$ that disagrees with $\sigma$ and $w\cdot\sigma(C_i)>w\cdot\sigma(C_e)$, contradicting our choice of $C_e$.
\end{proof}

\begin{figure}[ht!]
\begin{center}
    \includegraphics[width=15cm, height = 15cm, keepaspectratio]{zonotopeK3_6a.pdf}
\end{center}
  \caption{The subdivision of the zonotope associated to $K_3$ as described in Theorem \ref{thm:zonotopedecomp} using $\sigma$ induced by the total order and reference orientation on the right as described in Example \ref{edgeorder}.
  The red edges are bi-oriented.}
  \label{zonotopeK3_6}
\end{figure}

\begin{remark} \label{rmk:facedescription}
From the description above, a cell $Z(B)$ can be identified with the paralleletope $[0,1]^B$, where the $e$-th coordinate of a point in $Z(B)$ (corresponding to a $\sigma$-compatible continuous orientation ${\mathcal O}$ via $\psi$) is the value $\widehat{{\mathcal O}(e)}$. In particular, restricting to a face of $Z(B)$ of codimension $i$ can be described as orienting $i$ elements in $B$.
\end{remark}

\begin{remark}
Although we will not make use of it, we mention without proof the following characterization of the incidence relations between cells in the polyhedral decomposition of $Z_A$.
Let $B$ be a basis, let $e\in B$, and let $K$ be the fundamental cocircuit of $e$ with respect to $B$. Choose an orientation for $e$, and denote by $F_e$ the facet of $Z(B)$ corresponding to orienting $e$ according to this choice (cf.~Remark~\ref{rmk:facedescription}).
Let $\mathcal{O}$ be any continuous orientation of the form $\mathcal{O}(B,b)$ and let $\mathcal{O'}$ be the continuous orientation obtained from $\mathcal{O}$ by orienting $e$ (according to the chosen orientation).
Then either (1) $K$ is a positive cocircuit in $\mathcal{O'}$, in which case $F_e$ lies on the boundary of $Z_A$, or (2) there exists a unique element $f\in K\setminus e$ such that the orientation obtained by reversing $f$ in $\mathcal{O'}$ is also $\sigma$-compatible, and $F_e$ is incident to $Z((B\setminus\{e\})\cup\{f\})$.
\end{remark}


\subsection{Geometric interpretation of the combinatorial map}

Let $\sigma,\sigma^*$ be acyclic signatures of $C(M)$ and $C^*(M)$, respectively.  By Lemma~\ref{lem:gordan}, there exists $w \in {\mathbb R}^m$ that induces both $\sigma$ and $\sigma^*$.
Our next goal is to show that the combinatorially defined basis-to-orientation map $\hat{\beta}$ (whose definition depends on $\sigma$ and $\sigma^*$) can be interpreted geometrically as a ``shifting map''.

\medskip

To present the calculation in our proof more clearly, for the rest of Section \ref{sec:proof}, we will work in the {\em cocircuit space} $V^*(M)$ of $M$, which is the ${\mathbb R}$-span of $C^*(M)$ (and is equal to the row space of $A$). Let $\pi_{V^*(M)}$ be the orthogonal projection from $\RR^E$ onto $V^*(M)$ and let $\{u_e:e\in E\}$ be the standard basis for $\RR^E$. Consider the (row) zonotope $\widetilde{Z_A} := \{ \sum_{e\in E} c_e \pi_{V^*}(u_e) \; : \; 0 \leq c_e \leq 1 \}\subset V^*(M)$.
The following lemma shows that $\widetilde{Z_A}$ and the previously defined zonotope $Z_A$ are naturally isomorphic:

\begin{lemma}
The map $L:v\mapsto Av$ is an isomorphism between $V^*(M)$ and $\RR^r$ taking $\widetilde{Z_A}$ to $Z_A$.
\end{lemma}

\begin{proof}
Since $AA^T$ has full rank, $V^*(M)=\{A^Tz:z\in\RR^r\}$ is isomorphic to $\RR^r$ via $L$. By the formula for orthogonal projection, we have $\pi_{V^*}(u_e)=A^T(AA^T)^{-1}Au_e$, so $L(\pi_{V^*}(u_e))=Au_e=A_e$.  
Thus $L(\sum_{e\in E} c_e \pi_{V^*}(u_e))=\sum_{e\in E} c_e A_e$ and $L(\widetilde{Z_A})=Z_A$.
\end{proof}

In particular, the subdivision of $Z_A$ constructed in Section \ref{sec:polyhedralsubdivision} 
induces a corresponding subdivision of $\widetilde{Z_A}$.  We denote by $\widetilde{Z(B)}$ the cell $L^{-1}(Z(B))$ in $\widetilde{Z_A}$.

\medskip

The key to defining the shifting map is the following lemma:

\begin{lemma} \label{lem:phiwelldefined}
If $w'$ is the orthogonal projection of $w$ onto the cocircuit space $V^*(M)$ of $M$, then for all sufficiently small $\epsilon > 0$ the image of $\widetilde{Z(B)}$ under the map $v \mapsto v + \epsilon w'$ contains a unique point corresponding (via $\psi$) to a $\sigma$-compatible discrete orientation ${\mathcal O}_B$.
\end{lemma}

\begin{proof} By Theorem \ref{thm:zonotopedecomp}, the vertices of each $\widetilde{Z(B)}$ correspond to $\sigma$-compatible discrete orientations. It therefore suffices to prove that $w'$ does not lie in the affine span of any facet of $\widetilde{Z(B)}$.
The affine span of a facet of $\widetilde{Z(B)}$ is spanned by directions of the form $\pi_{V^*}(e)$ for $e\in\hat{B}$ where $\hat{B}\subsetneq B$ is a proper subset of $B$ of size $r-1$.
Since $|\hat{B}|=r-1$, there is a cocircuit $K$ of $M$ avoiding $\hat{B}$. Any direction $v := \sum_{e\in\hat{B}}\lambda_e\pi_{V^*}(e)$ in the span satisfies $\langle v, \sigma^*(K)\rangle
=\sum_{e\in\hat{B}}\lambda_e\langle e, \sigma^*(K)\rangle=0$, since $\pi_{V^*}$ is self-adjoint and $\pi_{V^*}(\sigma^*(K))=\sigma^*(K)$. On the other hand, since $w$ induces $\sigma^*$, we have $\langle w', \sigma^*(K)\rangle=\langle w,\sigma^*(K)\rangle>0$.
\end{proof}

We define $\phi$ to be the map that takes a basis $B$ to the orientation ${\mathcal O}_B$ defined in Lemma \ref{lem:phiwelldefined}.

\begin{theorem} \label{thm:phibetaequal}
The map $\phi$ coincides with the previously defined map $\hat{\beta}$.
\end{theorem}

\begin{proof}

Let $B$ be a basis. Then $\phi(B)$ can be obtained by orienting each (bi-oriented) basis element from a continuous $\sigma$-compatible orientation in the interior of $\widetilde{Z(B)}$ (which is of the form $\mathcal{O}(B,b)$), so by the greedy procedure described in Proposition \ref{prop:uniqueext}, the elements outside $B$ are oriented according to their fundamental circuits, hence $\phi(B)$ agrees with $\hat{\beta}(B)$ outside $B$.

For elements inside $B$, we work with the basis $\{\pi_{V^*}(u_e):e\in B\}$ for $V^*(M)$ and write $w'=\sum_{e\in B}w_e\pi_{V^*}(u_e)$. 
Identifying $\widetilde{Z(B)}$ with $[0,1]^B$ and the vertices of $\widetilde{Z(B)}$ with $\{0,1\}^B$, if a vertex $v$ is identified with $(s_e:e\in B)$, then it corresponds to a $\sigma$-compatible discrete orientation where each element $e\in B$ is oriented in agreement with (resp. opposite to) its reference orientation when $s_e=1$ (resp. $s_e=0$). The cell $\widetilde{Z(B)}$ will contain $v$ after shifting if and only if the sign pattern of the $s_e$'s agrees with the sign pattern of the $w_e$'s, i.e., if and only if $s_e=1$ precisely when $w_e>0$.

Let $f \in B$, and let $K$ be the fundamental cocircuit of $f$ with respect to $B$.  Then $\sigma^*(K)$ is by definition the orientation of $K$ with the property that $\langle w',\sigma^*(K)\rangle>0$. By a calculation similar to the above, 
\[
\langle w',\sigma^*(K)\rangle=\sum_{e\in B}w_e\langle u_e, \sigma^*(K)\rangle=w_f\langle u_f, \sigma^*(K)\rangle,
\]
as $f$ is the unique element in $B\cap K$. If $w_f>0$, then $\langle u_f, \sigma^*(K)\rangle>0$ and the reference orientation of $f$ agrees with $\sigma^*(K)$, i.e., the orientation of $f$ in $\phi(B)$ is the same as the reference orientation of $f$. From the last paragraph, $f$ is oriented according to its reference orientation in $\hat{\beta}(B)$ as well, because $w_f>0$. A similar analysis in the case where $w_b<0$ yields the same conclusion that $\phi(B)(f)=\hat{\beta}(B)(f)$.
\end{proof}

\begin{proposition}\label{prop:betaCCM}
Let $B$ be a basis. Then $\hat{\beta}(B)$ is $(\sigma,\sigma^*)$-compatible.
\end{proposition}

\begin{proof} Since $\phi(B)$ is $\sigma$-compatible, $\hat{\beta}(B)$ is $\sigma$-compatible as well by Theorem \ref{thm:phibetaequal}.  And since the procedure described in Theorem \ref{thm:combbijection} is symmetric with respect to circuits and cocircuits, a dual argument shows that $\hat{\beta}(B)$ is $\sigma^*$-compatible.
\end{proof}

\begin{theorem}[Theorem~\ref{thm:realizablemainthm}] \label{thm:betabijective}
The map $\hat{\beta} : {\mathcal B}(M) \to {\mathcal X}(M)$ is a bijection.
\end{theorem}

\begin{proof} $\hat{\beta}=\phi$ is injective for the simple geometric reason that a vertex can only be contained in the interior of at most one cell $\widetilde{Z(B)}$ after shifting. To prove the surjectivity part, we need to show that for every $(\sigma,\sigma^*)$-compatible orientation $\mathcal{O}$, there exists a continuous orientation $\mathcal{O}'$ such that the displacement from $\mathcal{O}'$ to $\mathcal{O}$, interpreted as points of $\widetilde{Z_A}$, is $\pi_{V^*}(w)$ (here we assume $w$ is sufficiently short). For simplicity, we negate suitable columns of $A$ in order to assume without loss of generality that $\mathcal{O}\equiv {\bf 1}$, and we modify $w$ accordingly. For such to be determined $\mathcal{O}'$, denote by ${\bf f}_e\geq 0$ the difference between $\widehat{\mathcal{O}(e)}=1$ and $\widehat{\mathcal{O}'(e)}$. By an easy application of the formula for orthogonal projection, our condition on $\mathcal{O}'$ in terms of displacement becomes $A{\bf f}=Aw$, hence $\mathcal{O}'$ exists if and only if the linear program
\begin{equation}\label{inverse_LP}
\min\{{\bf 1}^T{\bf f}:A{\bf f}=Aw, {\bf f}\geq {\bf 0}\}
\end{equation}
is feasible. But the $\sigma^*$-compatible condition implies ``if $z^TA\geq 0$ (i.e. $z^TA$ is a non-negative sum of signed cocircuits), then $(z^TA)w\geq 0$'', which is the same as ``there exists no $z$ such that $z^TA\geq 0, z^T(Aw)<0$'', by the Farkas lemma, the latter condition is equivalent to the existence of some ${\bf f}\geq 0$ such that $A{\bf f}=Aw$.

\end{proof}

\begin{figure}[ht!]
\begin{center}
    \includegraphics[width=1\textwidth]{zonotopeK3_7.pdf}
\end{center}
  \caption{An example of the bijection for $K_3$ using the pair $(\sigma, \sigma^*)$ induced by the total order and reference orientation from Figure \ref{zonotopeK3_6}.}
\end{figure}

\subsection{Computability of the bijection} \label{sec:bijectioncomputable}

We now describe an inverse algorithm which furnishes an inverse to the map $\phi$, and hence to $\hat{\beta}$.
Again we assume the inputted $(\sigma,\sigma^*)$-compatible discrete orientation $\mathcal{O}$ is equal to ${\bf 1}$ for simplicity.
Suppose $\mathcal{O}$ was shifted into the cell $\widetilde{Z(B)}$ after moving by a displacement of $-\pi_{V^*}(w)$. By solving the linear program (\ref{inverse_LP}) we obtain a continuous orientation $\mathcal{O}'$ (resp. ${\bf f}$) in the cell $Z(B)$.
Therefore it remains to find the $\sigma$-compatible continuous orientation $\mathcal{O}''$ equivalent to $\mathcal{O}'$, and the desired basis $B$ will then be the set of bi-oriented elements in $\mathcal{O}''$.

To do so, we solve the ``max-flow'' linear program 
\begin{equation}\label{maxflow_LP}
\max\{w^T{\bf y}: A{\bf y}=0, {\bf f}_e-1\leq {\bf y}_e\leq {\bf f}_e,\forall e\}.
\end{equation}
Let $\tilde{\bf y}$ be an optimal solution.
Consider the continuous orientation $\mathcal{O}''$ given by $\mathcal{O}''(e)=\mathcal{O}'(e)+2\tilde{\bf y}_e$ for every $e$, we claim this is the $\sigma$-compatible continuous orientation we are looking for. 
The condition ${\bf f}_e-1\leq {\bf y}_e\leq {\bf f}_e$ in the linear program guarantees that $\mathcal{O}''$ is a valid continuous orientation, and the condition $A{\bf y}=0$ guarantees that $\mathcal{O}'$ is circuit-reversal equivalent to $\mathcal{O}''$. 
The orientation $\mathcal{O}''$ is $\sigma$-compatible: indeed, if $\mathcal{O}''$ is compatible with some $-\sigma(C)$, then one can easily check that $\tilde{\bf y}+\delta \sigma(C)$ is also a feasible solution to the linear program for sufficiently small $\delta>0$, contradicting the optimality of $\tilde{\bf y}$.

\begin{remark}
As a corollary to the fact that linear programming admits a polynomial-time algorithm \cite{schrijver1986LP}, the bit complexity of the solution ${\bf f}$ in (\ref{inverse_LP}) is polynomial in the bit complexity of the input ($A$, $w$). So the linear program (\ref{maxflow_LP}) runs in polynomial time as well. In particular, if $A$ is totally unimodular, then the bit complexity does not increase with respect to $A,w$ in both (\ref{inverse_LP}) and (\ref{maxflow_LP}).
\end{remark}

The linear program (\ref{maxflow_LP}), together with the dual version of it, imply the following:

\begin{proposition} \label{prop:CCMalgo}
There is a polynomial-time algorithm to compute the unique $(\sigma,\sigma^*)$-compatible continuous orientation circuit-cocircuit equivalent to a given continuous orientation.
\end{proposition}

Summarizing the discussion, we have the following theorem:

\begin{theorem} \label{prop:invalgo}
There is a polynomial-time algorithm to compute the inverse of $\hat{\beta}$.
\end{theorem}

\section{The discrete circuit-cocircuit reversal system for a regular matroid and its Jacobian}\label{discretestuffforregularmatroids}

We now return to the setting of regular matroids.  
Throughout this section, $M$ will denote a regular matroid and $A$ will be a totally unimodular integer matrix representing $M$.
We will investigate a discrete version the continuous circuit-reversal system explored in the previous section, and show that the $\sigma$-compatible discrete orientations also give distinguished representatives for this coarser equivalence relation.  Moreover, we will show that discrete circuit-reversal classes correspond to lattice points of the zonotope $Z_A$ (which by Theorem \ref{thm:zonotopedecomp} are precisely the vertices of the zonotopal subdivision $\Sigma$).  Finally, we show that the discrete circuit-cocircuit reversal system is canonically a torsor for $\Jac(M)$.

\subsection{The discrete circuit-cocircuit reversal system}\label{dccrs}

For a totally unimodular matrix $A$, we have the following integral version of Lemma~\ref{lem:circuitdecomposition}:

\begin{lemma} \label{lem:regularcircuitdecomposition}
Suppose $A$ is totally unimodular, and let $u$ be an integral vector in $\ker(A)\cap\mathbb{Z}^E$. Then $u$ can be written as an integral sum of signed circuits $\sum \lambda_C C$ with $\lambda_C>0$, such that the support of each $C$ is inside the support of $u$ and for each $e\in C$, the signs of $e$ in $C$ and $u$ agree. In particular, if $u$ is a $\{ 0, \pm 1 \}$-vector, then the $\lambda_C$'s are 1 and the $C$'s are disjoint.
\end{lemma}

\begin{proof} For the first assertion, we follow the same argument as in the proof of Lemma~\ref{lem:circuitdecomposition}, but we note that the first $\lambda_C$ we considered there must be an integer as the coordinates of $C$ are $0,\pm 1$.  In particular, $u-\lambda_C C\in\ker(A)$ is an integral vector with a smaller support, and we can proceed by induction. The second assertion follows easily from the first.
\end{proof}

A {\em discrete orientation} ${\mathcal O}$ of $M$ is a function $E \to \{-1,1\}$.

A discrete orientation ${\mathcal O}$ is {\em compatible} with a signed circuit $C$ of $M$ if ${\mathcal O}(e) \neq -C(e)$ for all $e$ in the support of $C$.

If $\mathcal{O}$ is a discrete orientation and $C$ is a signed circuit compatible with $\mathcal{O}$, we can perform a (discrete) {\em circuit reversal} taking $\mathcal{O}$ to the orientation $\mathcal{O}'$ defined by $\mathcal{O}'(e) = \mathcal{O}(e)$ if $e$ is not in the support of $C$ and $\mathcal{O}'(e) = -\mathcal{O}(e)$ if $e$ is in the support of $C$.
The {\em discrete circuit reversal system} is the equivalence relation on the set $CO(M)$ of all discrete orientations of $M$ generated by all possible discrete circuit reversals. 

We can make the same definitions for cocircuits by replacing $M$ with its dual matroid $M^*$.

\begin{proposition} \label{prop:latticepointprop}
The map $\psi$ from \S\ref{sec:continuousreversals} induces a bijection between discrete orientations of $M$ modulo discrete circuit reversals and lattice points of $Z_A$.
\end{proposition}

\begin{proof}
As the columns of $A$ are integral, $\psi$ takes an orientation of $M$ to a lattice point of $Z_A$; conversely, for any lattice point $y\in Z_A$ we know $A\hat{\alpha}=y, 0\leq \hat{\alpha}_i\leq 1\ \forall i$ has a solution $\hat{\alpha}$, but by the total unimodularity of $A$, $\hat{\alpha}$ can be chosen to be integral and hence corresponds to an orientation.  Thus the image of $\psi$ is precisely the set of lattice points of $Z_A$. By the orthogonality of circuits and cocircuits, two orientations in the same circuit-reversal class map to the same point of $Z_A$.  Conversely, suppose $\psi(\mathcal{O})=\psi(\mathcal{O}')$. 
By Lemma~\ref{lem:regularcircuitdecomposition},
$\mathcal{O}-\mathcal{O}'$ can be written as a sum of disjoint signed circuits in which each signed circuit is compatible with $\mathcal{O}$, and $\mathcal{O}$ can be transformed to $\mathcal{O}'$ via the corresponding circuit-reversals in any order.
\end{proof}

\begin{proposition} \label{prop:discretecompatible}
Each discrete circuit-reversal class of discrete orientations of $M$ contains a unique $\sigma$-compatible discrete orientation.
\end{proposition}

\begin{proof} The uniqueness assertion follows from Lemma~\ref{lem:regularcircuitdecomposition} and a similar argument as in Theorem~\ref{thm:continuouscompatible}. For existence, start with any orientation $\mathcal{O}$ in the class and reverse some signed circuit $C$ compatible with $\mathcal{O}$ but not compatible with $\sigma$. We claim that the process will eventually stop. Indeed, suppose not: since the number of discrete orientations of $M$ is finite, the orientation will without loss of generality return to $\mathcal{O}$ after reversing some signed circuits $C_1,\ldots,C_k$ in that order (the circuits might not be distinct). Then $-C_1-\cdots-C_k=0$, which means that $\sigma(C_1)+\cdots+\sigma(C_k)=0$, contradicting the acyclicity of $\sigma$. 
\end{proof}


\begin{corollary}
The lattice points of $Z_A$ are exactly the vertices of the subdivision $\Sigma$.
\end{corollary}

\begin{proof}
This follows from Propositions \ref{prop:latticepointprop} and \ref{prop:discretecompatible} and Theorem \ref{thm:zonotopedecomp}.  
\end{proof}

\begin{figure}[ht!]
\begin{center}
   \includegraphics[width=10cm, height = 10cm, keepaspectratio]{zonotopeK3_3b.pdf}
\end{center}
  \caption{The zonotope associated to $K_3$ and the circuit reversal classes associated to its lattice points by the map $\psi$ from Proposition~\ref{prop:ctslatticepointprop}.  Taking the acyclic signature $\sigma$ from Figure \ref{zonotopeK3_6}, the cycle in blue is $\sigma$-compatible, while the cycle in red is not.  (Note that we are using the full adjacency matrix of $K_3$ to define the zonotope, cf.~Remark~\ref{rmk:graphicmatrix}.) 
  }
  \label{zonotopeK3_3}
\end{figure}

Let $\chi:{\mathcal X}(M) \rightarrow {\mathcal G}(M)$ be the map which associates to each $(\sigma,\sigma^*)$-compatible orientation the discrete circuit-cocircuit reversal class which it represents.

\begin{theorem} [Part (2) of Theorem \ref{thm:SScompatibleorientation}] \label{thm:chibijective}
The map $\chi$ is a bijection.
\end{theorem}

\begin{proof}
This follows directly from Propositions \ref{prop:discretecompatible} and \ref{prop:orientdecomp}
\end{proof}

\begin{corollary} [Theorem~\ref{thm:mainbijectionforregularmatroids}]
The map $\beta:B(M)\rightarrow\mathcal{G}(M)$ given by $B\mapsto [\mathcal{O}_B]$ is a bijection.
\end{corollary}

\begin{proof}
The map $\hat{\beta}:B\mapsto \mathcal{O}_B$ is a bijection between $B(M)$ and $\mathcal{X}(M;\sigma,\sigma^*)$ by Theorem~\ref{thm:betabijective}. Now compose this map with $\chi$.
\end{proof}

\subsection{The circuit-cocircuit reversal system as a $\Jac(M)$-torsor}
\label{sec:Torsor}

In this section we will define a natural action of $\Jac(M)$ on the set ${\mathcal G}(M)$ of cycle-cocycle equivalence classes of orientations of $M$ and prove that the action is simply transitive. We will also discuss an efficient algorithm for computing this action, along with an application to randomly sampling bases of $M$.

\subsection{Definition of the action}

Recall from (\ref{eq:canonisom}) that $\Jac(M)$ can be identified with $\frac{\mathbb{Z}^E}{\Lambda_A(M)\oplus \Lambda_A^*(M)}$, where $A$ is any totally unimodular $r \times m$ matrix representing $M$. Note that the latter group is generated by $[\overrightarrow{e}],e\in M$ (where we use an overhead arrow to emphasize that we are keeping track of orientations). 

The group action $\Jac(M)\circlearrowright\mathcal{G}(M)$ is defined by linearly extending the following action of each generator $[\overrightarrow{e}]$ on circuit-cocircuit reversal classes: pick an orientation $\mathcal{O}$ from the class so that $e$ is oriented as $\overrightarrow{e}$ in $\mathcal{O}$, reverse the orientation of $e$ in $\mathcal{O}$ to obtain $\mathcal{O}'$, and set $[\overrightarrow{e}]\cdot[\mathcal{O}]=[\mathcal{O}']$.

This action generalizes the one defined in terms of {\em path reversals} by the first author in the graphical case \cite[Section 5]{backman2014riemann}.



\begin{figure}[ht!]
\begin{center}
    \includegraphics[width=10cm, height = 10cm, keepaspectratio]{Torsor_Example.pdf}
\end{center}
  \caption{Example of the torsor. Here the reference orientations of $e,f$ are the same as the orientation we begin with.}
  \label{Torsor_Example}
\end{figure}



Our main goal for the rest of this section will be to prove:

\begin{theorem} \label{Thm:Action}
(Theorem~\ref{thm:torsor})
The group action $\circlearrowright$ is well-defined and simply transitive. 
\end{theorem}

\subsection{Preliminary results}

Theorem \ref{Thm:Action} will be deduced from a series of intermediate results. We start with a few general lemmas regarding regular matroids and/or totally unimodular matrices.

\begin{remark}
For ease of exposition, in the rest of this section we will use the term {\em positive} (co)circuit (with respect to an orientation $\mathcal{O}$) to denote a signed (co)circuit that is compatible with $\mathcal{O}$. Furthermore, given an orientation $\mathcal{O}$ and a subset $X\subset E$, we denote by $_{-X}\mathcal{O}$ the orientation obtained by reversing elements of $X$ in $\mathcal{O}$.  For a (co)circuit $C$ of $\mathcal{O}$, we say that $_{-X}C$ is positive if $C$ is a positive (co)circuit of $_{-X}\mathcal{O}$. Finally, we denote by $\rchi_X$ the $\{0,1\}$-characteristic vector whose support is $X$.
\end{remark}

\begin{lemma} \label{Lem:ExtendCC}
Let $e\in E$, and suppose $X\subset E\setminus \{e\}$ is a positive cocircuit in $\mathcal{O}\setminus\{e\}$ but not in $\mathcal{O}$. Then $Y:=X\cup\{e\}$ is a cocircuit in $\mathcal{O}$, and either $Y$ or $_{-e}Y$ is positive.
\end{lemma}

\begin{proof}
By assumption there exists some $w\in\mathbb{Z}^r$ such that $w^TA|_{E\setminus\{e\}}=\rchi_X$. Hence $w^TA=\rchi_X+\lambda\rchi_{\{e\}}$ for some $\lambda$, which is not zero as $X$ is not a positive cocircuit in $\mathcal{O}$. By the dual of Proposition \ref{lem:circuitdecomposition}, $Y$ contains a cocircuit $D$. 
If $X\subset D$, then $D$ must also contain $e$, as $X$ itself is not a (positive) cocircuit.  On the other hand, if $D\cap X=\emptyset$, then $D=\{e\}$, which in turn shows that $X$ itself is a positive cocircuit, as $w'^TA=\rchi_{\{e\}}$ for some $w'$ implies $(w-\lambda w')^TA=\rchi_X$. Therefore $Y=D$ is a cocircuit, and $\lambda=\pm 1$, i.e. either $Y$ or $_{-e}Y$ is positive. 
\end{proof}

\begin{lemma} \label{Lem:AlmostPosCC}
Suppose $e \in M$ is contained in some positive circuit of $\mathcal{O}$, and that $Y$ is a subset of $E$ containing $e$ such that $_{-e}Y$ is a positive cocircuit. Then any positive circuit containing $e$ intersects $Y$ in exactly two elements.
\end{lemma}

\begin{proof}
Let $C$ be a positive circuit containing $e$, i.e., $A\rchi_C=0$. By assumption, there exists a vector $v$ such that $v^TA=\rchi_{Y-e}-\rchi_{\{e\}}$.  Then  $0=v^TA\rchi_C=|(Y-e)\cap C|-1$, i.e., $Y$ intersects $C$ in $e$ together with exactly one more element.
\end{proof}

\subsection{The action is well-defined}

In order to show that the action of $\Jac(M)$ on $\mathcal{G}(M)$ is well-defined, we first show that the corresponding action (which by abuse of notation we continue to write as $\circlearrowright$) of $\ZZ^E$ on $\mathcal{G}(M)$ is well-defined, then that the action descends to the quotient by $\Lambda_A(M)\oplus \Lambda_A^*(M)$.



\begin{proposition} \label{Prop:Torsor_WellDef}
For every $[\mathcal{O}]\in\mathcal{G}(M)$ and oriented element $\overrightarrow{e}$, there exists $\tilde{\mathcal{O}}\in [\mathcal{O}]$ so that $e$ is oriented as $\overrightarrow{e}$ in $\tilde{\mathcal{O}}$.
\end{proposition}

\begin{proof}
This follows from Proposition~\ref{prop:orientdecomp}, which guarantees that every oriented $\overrightarrow{e}$ is either contained in a positive circuit or a positive cocircuit.
If $e$ is not already oriented according to $\mathcal{O}$, replace $\mathcal{O}$ by the corresponding (co)circuit reversal.
 
\end{proof}

\begin{proposition} \label{Prop:IndepRep}
The action of $\overrightarrow{e}$ on $[\mathcal{O}]$ is independent of which orientation we choose.
\end{proposition}

\begin{proof}
It suffices to show that if $\mathcal{O}\sim\mathcal{O}'$ and the two orientations agree on $e$, then $_{-e}\mathcal{O}\sim\!_{-e}\mathcal{O}'$. By Lemma~ \ref{lem:circuitdecomposition} and its dual, $\mathcal{O}$ and $\mathcal{O}'$ differ by a disjoint union of positive circuits and cocircuits which do not contain $e$, and $_{-e}\mathcal{O}$ can be transformed to $_{-e}\mathcal{O}'$ by reversing these positive (co)circuits.
\end{proof}




\begin{proposition} \label{Prop:commute_action}
For any $\overrightarrow{e},\overrightarrow{f}\in \mathbb{Z}^E$ and $[\mathcal{O}]\in\mathcal{G}(M)$, $\overrightarrow{e}\cdot(\overrightarrow{f}\cdot [\mathcal{O}])=\overrightarrow{f}\cdot(\overrightarrow{e}\cdot[\mathcal{O}])$. Hence it is valid to extend $\cdot$ linearly, and $\circlearrowright$ is indeed a group action of $\ZZ^E$ on $\mathcal{G}(M)$ .
\end{proposition}

\begin{proof}
The statement is tautological if $\overrightarrow{e}=\overrightarrow{f}$. If $\overrightarrow{e}=-\overrightarrow{f}$, then without loss of generality the orientation of $e$ in $\mathcal{O}$ is $\overrightarrow{e}$.  Let $C$ be a positive (co)circuit containing $e$. On the one hand we have $\overrightarrow{f}\cdot(\overrightarrow{e}\cdot[\mathcal{O}])=\overrightarrow{f}\cdot[_{-e}\mathcal{O}]=[\mathcal{O}]$; on the other hand $\overrightarrow{e}\cdot(\overrightarrow{f}\cdot [\mathcal{O}])=\overrightarrow{e}\cdot(\overrightarrow{f}\cdot [_{-C}\mathcal{O}])=[_{-C}\mathcal{O}]=[\mathcal{O}]$, so the action order does not matter.

Otherwise $e\neq f$. We may again assume without loss of generality that $e$ is oriented as $\overrightarrow{e}$ in $\mathcal{O}$. The statement is easy if there exists some positive (co)circuit in $\mathcal{O}$ that contains $f$ but not $e$, as we can reverse it and obtain an orientation in which the orientations of $e,f$ are already $\overrightarrow{e},\overrightarrow{f}$.
So without loss of generality $e,f$ are in the circuit part of $\mathcal{O}$ and every positive circuit containing $f$ also contains $e$.  Fix any such positive circuit $C$. Then $f$ must be in some positive cocircuit $D'$ of $\mathcal{O}-e$, since otherwise $f$ is in some positive circuit of $\mathcal{O}-e$, which is a positive circuit in $\mathcal{O}$ avoiding $e$. By Lemma~\ref{Lem:ExtendCC},  $D:=D'\cup\{e\}$ is a cocircuit in $\mathcal{O}$ and $_{-e}D$ is positive, and by Lemma \ref{Lem:AlmostPosCC}, we know that $C\cap D=\{e,f\}$.

We have $\overrightarrow{e}\cdot[\mathcal{O}]=[_{-e}\mathcal{O}]=[_{-(D-e)}\mathcal{O}]$ as $_{-e}D$ is a positive cocircuit in $_{-e}\mathcal{O}$, and $\overrightarrow{f}\cdot[_{-(D-e)}\mathcal{O}]=[_{-(D-\{e,f\})}\mathcal{O}]$ as $f$ is oriented as $\overrightarrow{f}$ in $_{-(D-e)}\mathcal{O}$. On the other hand, $\overrightarrow{f}\cdot[\mathcal{O}]=\overrightarrow{f}\cdot[_{-C}\mathcal{O}]=[_{-(C-f)}\mathcal{O}]$, and $D$ is positive in $_{-(C-f)}\mathcal{O}$ since $C\cap D=\{e,f\}$, hence $\overrightarrow{e}\cdot [_{-(C-f)}\mathcal{O}]=\overrightarrow{e}\cdot[_{-(C\cup D-e)}\mathcal{O}]=[_{-(C\cup D)}\mathcal{O}]$. But $C$ is positive in $_{-(C\cup D)}\mathcal{O}$, so $[_{-(C\cup D)}\mathcal{O}]=[_{-(C\cup D)\triangle C}\mathcal{O}]=[_{-(D-\{e,f\})}\mathcal{O}]$.
\end{proof}


Now we know that $\mathbb{Z}^E\circlearrowright\mathcal{G}(M)$ is well-defined, so we show next that this  action descends to a group action $\Jac(M)\circlearrowright\mathcal{G}(M)$.

\begin{proposition} \label{Prop:JacActWD}
The stabilizer of the action on any $[\mathcal{O}]$ contains $\Lambda_A(M)\oplus \Lambda_A^*(M)$.
\end{proposition}

\begin{proof}
Let $\overrightarrow{C}\in \Lambda_A(M)$ be a signed circuit. Let $F$ be the set of elements in $C$ whose orientations in $\mathcal{O}$ are the same as in $\overrightarrow{C}$. Then $\overrightarrow{C}\cdot[\mathcal{O}]=(\sum_{\overrightarrow{e}\in\overrightarrow{C}\setminus F}\overrightarrow{e})\cdot[_{-F}\mathcal{O}]=(\sum_{\overrightarrow{C}\setminus F}\overrightarrow{e_i})\cdot[_{-(C-F)}\mathcal{O}]=[\mathcal{O}]$. The proof for $\Lambda_A^*(M)$ is similar.
\end{proof}

\subsection{The action is simply transitive}

\begin{proposition} \label{Prop:TransTorsor}
The group action $\Jac(M)\circlearrowright\mathcal{G}(M)$ is transitive.
\end{proposition}

\begin{proof}
Given any two orientations $\mathcal{O},\mathcal{O}'$, let $\gamma$ be the sum of the (oriented) elements in $\mathcal{O}$ whose orientation in $\mathcal{O}'$ is different; then $[\gamma]\cdot[\mathcal{O}]=[\mathcal{O}']$. 
\end{proof}

We know from Proposition \ref{Prop:JacActWD} that $\Jac(M)\circlearrowright\mathcal{G}(M)$ is a well-defined group action, which by Proposition \ref{Prop:TransTorsor} is transitive, and we know that $|\Jac(M)|=|\mathcal{G}(M)|$, so the action is automatically simple.  However, it seems worthwhile to give a direct proof of the simplicity of the action which does not make use of the equality $|\Jac(M)|=|\mathcal{G}(M)|$, since this yields an independent and ``bijective'' proof of the equality.  We begin with the following reduction.

\begin{proposition}
The simplicity of the group action $\Jac(M)\circlearrowright\mathcal{G}(M)$ is equivalent to the statement that every element of the quotient group $\frac{\mathbb{Z}^E}{\Lambda_A(M)\oplus \Lambda_A^*(M)}$ contains a coset representative whose coefficients are all $1,0,-1$.
\end{proposition}

\begin{proof}
Suppose the group action is simple. Let $[\gamma]\in\frac{\mathbb{Z}^E}{\Lambda_A(M)\oplus \Lambda_A^*(M)}$ be an element of $\Jac(M)$. Pick an arbitrary orientation $\mathcal{O}$ and an arbitrary orientation $\mathcal{O}'$ from $[\gamma]\cdot[\mathcal{O}]$. Let $\gamma_0\in \mathbb{Z}^E$ be the sum of (oriented) elements in $\mathcal{O}$ which have the opposite orientation in $\mathcal{O}'$. Then $[\gamma_0]\cdot[\mathcal{O}]=[\mathcal{O}']$, which by the simplicity of the action implies that $\gamma_0\in[\gamma]$.  The desired coset representative is $\gamma_0$.

Conversely, suppose such a set of coset representatives exists. We need to show that whenever $[\gamma]\in\frac{\mathbb{Z}^E}{\Lambda_A(M)\oplus \Lambda_A^*(M)}$ fixes some circuit-cocircuit reversal class, $[\gamma]=[0]$. By transitivity, $[\gamma]$ will fix every equivalence class in such a case, so suppose $[\gamma]$ fixes all circuit-cocircuit reversal classes. Without loss of generality, the coefficients of $\gamma$ are all $1,0,-1$ with support $F\subset E$. Pick an orientation $\mathcal{O}$ in which the orientation of every element of $F$ agrees with $\gamma$; then $[\mathcal{O}]=[\gamma]\cdot[\mathcal{O}]=[_{-F}\mathcal{O}]$. Therefore $\mathcal{O}\sim\!_{-F}\mathcal{O}$, meaning that $F$ is a disjoint union of positive circuits and cocircuits in $\mathcal{O}$, i.e., $\gamma\in \Lambda_A(M)\oplus \Lambda_A^*(M)$ and $[\gamma]=[0]$.
\end{proof}

\begin{proposition} \label{Prop:JacRep}
Every element of $\frac{\mathbb{Z}^E}{\Lambda_A(M)\oplus \Lambda_A^*(M)}$ contains a coset representative whose coefficients are all $1,0,-1$.
\end{proposition}

\begin{proof}
We will show that there is such a representative in $[\gamma]$ for every $\gamma=\sum_{e\in E} c_ee\in\frac{\mathbb{Z}^E}{\Lambda_A(M)\oplus \Lambda_A^*(M)}$ by lexicographic induction on $|\gamma|_\infty:=\max_{e\in E}|c_e|$ and the number of elements $e$ with $|c_e|=|\gamma|_\infty$. The assertion is clearly true if $|\gamma|_\infty\leq 1$. Otherwise suppose $|\gamma|_\infty>1$; by choosing a suitable reference orientation we may assume that all coefficients of $\gamma$ are non-negative. Pick an element $e$ whose coefficient $c_e$ equals $|\gamma|_\infty$ and pick a positive (co)circuit $C$ containing $e$. By subtracting $\gamma_C:=\sum_{f\in C} f$ from $\gamma$, all positive coefficients $c_f$ with $f\in C$ decrease by 1, while the zero coefficients become $-1$. Hence $|\gamma-\gamma_c|_\infty\leq|\gamma|_\infty$ and the number of elements $f$ with $|c_f|=|\gamma|_\infty$ strictly decreases. So by our induction hypothesis, there exists a representative with the desired form in $[\gamma-\gamma_C]=[\gamma]$.
\end{proof}

\begin{corollary} \label{Prop:SimpleTorsor}
The group action $\Jac(M)\circlearrowright\mathcal{G}(M)$ is simple.
\end{corollary}

%%%%%%%%%%%%%%%

%%%%%%%%%%%%%

\subsection{Computability of the group action}\label{computingstuff}

We now show that the simply transitive action of $\Jac(M)$ on ${\mathcal G}(M)$ is efficiently computable.

\begin{proposition} \label{prop:groupactioncomputable}
The action of $\Jac(M)$ on ${\mathcal G}(M)$ can be computed in polynomial time, given a totally unimodular matrix $A$ representing $M$.
\end{proposition}

\begin{proof}
First we note that computing the action of a generator $[\overrightarrow{e}]$ on a circuit-cocircuit reversal class can be done in polynomial time. To see this, it suffices by Proposition \ref{Prop:Torsor_WellDef} to find a positive circuit/cocircuit containing a given element $e$ in $M$. Note that $e$ is in some positive circuit of $M$ if and only if the integer program $\min({\bf 1}^Tv:Av=0, v_e=1, 0\leq v_i\leq 1, v_i\in\mathbb{Z})$ has a solution, and if a solution exists, the support of the minimizer $v$ is a positive circuit containing $e$. So it suffices to note that since $A$ is totally unimodular, we can solve this linear programming problem in polynomial time \cite{schrijver1986LP}. 
The cocircuit case is proved analogously.


It remains to show that it is possible to find, in polynomial time, a coset representative with small (polynomial-size) coefficients in each element of $\Jac(M)\cong\frac{\mathbb{Z}^E}{\Lambda_A(M)\oplus \Lambda_A^*(M)}$.
For the practical reason of generating random elements of $\Jac(M)$ (cf. Section \ref{sec:sam_algo}), we often start with a vector ${\bf y}\in\mathbb{Z}^r$ representing a coset of $\frac{\mathbb{Z}^r}{\Col_{\mathbb{Z}}(AA^T)}$, before ``lifting'' ${\bf y}$ to a vector $\gamma\in\mathbb{Z}^E$ that represents an element of $\Jac(M)$.
Thus we describe a two-step algorithm to in fact find a representative in $\mathbb{Z}^E$ where all coefficients belong to $\{ -1,0,1 \}$ (the existence of which is guaranteed by Proposition \ref{Prop:JacRep}), starting with an input vector in ${\bf y}\in\mathbb{Z}^r$
.

In step 1, replace ${\bf y}$ by ${\bf y}':={\bf y}-(AA^T)\lfloor (AA^T)^{-1}{\bf y}\rfloor$, where $\lfloor\ \rfloor$ is the coordinate-wise truncation. The new vector represents the same element in $\frac{\mathbb{Z}^r}{\Col_{\mathbb{Z}}(AA^T)}$, and it is equal to $(AA^T)((AA^T)^{-1}{\bf y}-\lfloor (AA^T)^{-1}{\bf y}\rfloor)$. Each coordinate of $(AA^T)^{-1}{\bf y}-\lfloor (AA^T)^{-1}{\bf y}\rfloor$ is between 0 and 1, and each coordinate of $AA^T$ is between $-m$ and $m$, so the absolute value of each coordinate of ${\bf y}'$ is at most $mr$. To work in $\frac{\mathbb{Z}^E}{\Lambda_A(M)\oplus \Lambda_A^*(M)}$, we have to solve the equation $A\gamma={\bf y}'$, which can be done by choosing an arbitrary basis $B$ and then solving $A|_B\gamma'={\bf y}'$. Since $A$ is totally unimodular, the absolute value of each coefficient of $\gamma'$ is at most $mr^2$.

In step 2, starting with the element $\gamma\in \mathbb{Z}^E$ we obtained in step 1, we now explain how to find in polynomial time a representative of $[\gamma]$ whose coefficients are all $0,1,-1$. 
The procedure described in Proposition \ref{Prop:JacRep} successively chooses a positive (co)circuit $C$ which contains an element $e$ whose coefficient $c_e$ is maximum in $\gamma$ (recall that we may assume all coefficients in $\gamma$ are non-negative by choosing a suitable reference orientation), then subtracts $\gamma_C$ from $\gamma$. An algorithmic optimization is to subtract $\lfloor\frac{c_e}{2}\rfloor\gamma_C$ from $\gamma$ at once instead. No new element with the absolute value of its coefficients being larger than $\lceil\frac{|\gamma|_\infty}{2}\rceil$ is created in each such step, so after every $O(m)$ steps the maximum absolute value of coefficients is halved, and in a total of $O(m\log m)$ steps the maximum absolute value of coefficients is reduced to at most 1. We remark that step 2 by itself can yield a polynomial time algorithm if we work in $\mathbb{Z}^E$ directly.
\end{proof}

\subsection{An algorithm for sampling bases of a regular matroid} \label{sec:sam_algo}

By mimicking the strategy from \cite{baker2013chip}, we can now produce a polynomial-time algorithm for randomly sampling bases of a regular matroid.  The high-level strategy is:
\begin{enumerate}
\item Compute ${\rm Jac}(M)$ as a product of cyclic groups.  Use this presentation to choose a random element $\gamma \in {\rm Jac}(M)$.
\item Given a reference orientation ${\mathcal O}$, compute $[{\mathcal O}'] := \gamma \cdot [{\mathcal O}] \in {\mathcal G}(M)$, where $\cdot$ is the group action from Theorem \ref{thm:torsor}.
\item Compute the basis $B$ corresponding to $[{\mathcal O}']$ (resp. the $(\sigma,\sigma^*)$-compatible orientation in $[\mathcal{O}']$), which can be found in polynomial time by Proposition \ref{prop:CCMalgo}.
\end{enumerate}


In the Appendix, we describe a much simpler polynomial-time algorithm to randomly sample bases of a regular matroid using deletion and contraction.





\section{Dilations, the Ehrhart polynomial, and the Tutte polynomial}

Metric graphs can either be viewed as limits of subdivisions of discrete graphs or as intrinsic objects. See, for example, Section 2 of  \cite{baker2006metrized}.
Similarly, one can view continuous orientations of regular matroids as a limit of discrete orientations or as intrinsic objects.  So far in this paper we have taken the latter viewpoint, but in this section we shift towards the former.  In doing so, we will see that the bridge between discrete and continuous orientations of regular matroids is intimately related to Ehrhart theory for unimodular zonotopes.  For example, we demonstrate how this perspective allows for a new derivation of a result of Stanley which states that the Ehrhart polynomial of a unimodular zonotope is a specialization of the Tutte polynomial.



\subsection{The Ehrhart polynomial and the Tutte polynomial}

The Tutte polynomial $T_M(x,y)$ is a bivariate polynomial associated to a matroid $M$ which encodes a wealth of information associated to $M$.  One of its key properties is that  $T_M(x,y)$ is ``universal'' with respect to deletion and contraction, in the following sense:

\begin{proposition}[see~{\cite[Theorem 1]{welsh2000potts}} and~{\cite[Theorem 2.16]{welsh1999tutte}}] \label{thm:gentutte}
Let $\mathbb{M}$ be the set of all matroids. Suppose $a,b,x_0,y_0 \in \RR$ and that $f\colon \mathbb{M} \to \RR$ is a function with $f(\emptyset)=1$ and such that for every matroid $M$ and every element $e$ of $M$,
\begin{align*}
f(M) &= af(M / e) + bf(G\setminus e) &\textrm{if $e$ is neither a loop nor a coloop}\\
f(M) &= x_0f(M \setminus e)  &\textrm{if $e$ is a coloop} \\
f(M) &= y_0f(M / e)  &\textrm{if $e$ is a loop.}\\
\end{align*}
Then  
\[f(M) = a^{rk(M)}b^{rk(M^*)}T_{M}(\frac{x_0}{a},\frac{y_0}{b}).\]
\end{proposition}

Given an integer polytope $P$, its {\it Ehrhart polynomial} $E_P(q)$ counts the number of lattice points in  $qP$, the $q$-th dilate of $P$.  The fact that such a polynomial exists for any integer polytope was proven by Ehrhart \cite{ehrhart1962polynomial}.  Let $M$ be a regular matroid represented by the totally unimodular $r \times m$ matrix $A$.
Given a positive integer $q$, define $qA$ to be the $r \times qm$ matrix obtained by repeating each column of $A$ $q$ times consecutively.
Let $qM$ be the corresponding regular matroid.  
Note that the zonotope $Z_{qA}$ associated to $qA$ is just the $q$-fold Minkowski sum $qZ_A$.

\medskip

Let $\sigma_q$ be an acyclic signature of $qM$. Using the interpretation of lattice points of $Z_{qA}$ as $\sigma_q$-compatible orientations of $qM$, we give a new proof of the following theorem.
 
\begin{theorem}[Stanley \cite{stanley1991ehrhart}]\label{Ehrhart}
Let $A$ be a totally unimodular matrix with associated zonotope $Z=Z_A$, and let $M$ be the corresponding regular matroid.  
Then
\[
E_Z(q) = q^{{\rm rk}(M)} T_{M}(1 + \frac{1}{q},1).
\]
\end{theorem}


 Stanley's result extends to general integer zonotopes, but {\it a priori} our proof does not.  For a modern treatment of Stanley's result in the context of the multiplicity Tutte polynomial, see \cite{moci2012arithmetic}.

Before giving the proof of Theorem~\ref{Ehrhart}, we need a few definitions.  

By a {\em partial orientation} of a regular matroid $M$,\footnote{In the case of graphs, Hopkins and the first author would call such objects ${\it type\, B \, partial\,  orientations}$, but we will suppress the term ``type B" here.} we mean a function $E \to \{-1, 0, 1\}$, where elements mapping to $0$ are called {\em bi-oriented}.    Given $t \in \mathbb{Z}_{>0}$, a {\em $t$-partial orientation of $M$} will be a partial orientation where each bi-oriented edge receives some integer weight $s$ with  $1 \leq s \leq t$. (By convention,  a $0$-partial orientation of $M$ will mean the same thing as an orientation.)  

Fix a reference orientation $\mathcal{O}_{\rm ref}$ on $M$.
Setting $t+1=q$, there is a map from orientations of $qM$ to $t$-partial orientations as follows.  
Given $e \in M$, if all $q$ copies of $e$ are oriented similarly, we map them to the corresponding orientation of $e$ in $M$.  On the other hand, if $s$ copies of $e$ of are oriented in agreement with $O_{\rm ref}$ and $q-s$ copies are oriented oppositely, with $1 \leq s \leq t$, we map this set of edges to a bi-oriented element of weight $s$ in $M$. 

A non-empty subset $F$ of $E$ is called a {\em potential circuit} in a $t$-partial orientation $\mathcal{O}$ if $F$ is a circuit of $M$ and there is a choice of orientation of each bi-oriented element so that $F$ becomes a positive circuit.  We will call a $t$-partial orientation of $M$ {\em circuit connected} if for each $e$ which is the minimum element in a potential circuit, either $e$ is not bi-oriented and is oriented in agreement with the reference orientation, or $e$ is bi-oriented and replacing it with the opposite orientation of $e$ in $\mathcal{O}_{\rm ref}$ does not produce any potential circuits containing $e$.

\begin{proof}
(of Theorem~\ref{Ehrhart})
For each positive integer $q$,
we will define an acyclic signature $\sigma_q$ on $C(qM)$. By Proposition \ref{prop:ctslatticepointprop}
 
and Proposition \ref{prop:discretecompatible}, it will then suffice to prove that the number of $\sigma_q$-compatible orientations of $qM$ is $q^{{\rm rk}(M)} T_{M}(1 + q^{-1},1)$.  As in Example \ref{edgeorder}, each $\sigma_q$ will come from a total order and reference orientation of $qM$. We now explain how given an arbitrary $\sigma_1$, we can naturally define $\sigma_q$.  Given $e \in M$, let $e_1, \dots e_q$ be the $q$ copies of $e$ in $q M$.  We orient the $e_i$ in $\sigma_q$ similarly to $e$ in $\sigma_1$, i.e., so that together they form a positive cocircuit in their induced matroid.  Let $e^i$ be the list of the elements of $M$ according to $\sigma_1$.  Given $e^i_k$ and $e^j_\ell$ in $qM$, we define $\sigma_q$ so that $e^i_k <_q e^j_\ell$ if $i<j$, or $i=j$ and $k<\ell$.

We are attempting to count objects associated to $qM$ using the Tutte polynomial of $M$, so we would first like to produce a bijective map from $\sigma_q$-compatible orientations of $qM$ to certain objects associated to $M$ alone.  To do this, note that given a $\sigma_q$-compatible orientation $\mathcal{O}$ of $qM$, a reference orientation $\mathcal{O}^q_{\rm ref}$, and a set of parallel elements $e_1, \dots e_q$, there are only $q+1$ possible orientations of these elements: $e_1 \dots e_k$ will be oriented in agreement with $\mathcal{O}^q_{\rm ref}$, for some $k=0,1,\ldots,q$, and $e_{k+1} \dots e_q$ will be oriented oppositely.  (If this were not the case, we would have a 2-element positive circuit whose minimum edge is oriented in disagreement with $\mathcal{O}^q_{\rm ref}$.)
Using this observation, it is not difficult to check that the map defined above from orientations of $qM$ to $t$-partial orientations of $M$ (where $t=q-1$) takes $\sigma_q$-compatible orientations of $qM$ bijectively to circuit connected $t$-partial orientations of $M$.

We first prove that the sets $X_{t,M\setminus e}$ and $X_{t,M/e}$ are the images of $X_{t,M}$ under deletion and contraction, respectively.  Given $\mathcal{O} \in X_{t,M\setminus e}$ (the case of $\mathcal{O} \in X_{t,M/ e}$ being similar), suppose that both orientations of $e$ produce $t$-partial orientations which are not elements of $X_{t,M}$.  This implies that both orientations of $e$ produce potential circuits $C_1$ and $C_2$ which are not $\sigma_q$-compatible.  For $1\leq i \leq 2$, we can choose orientations of the bioriented elements in $C_i$ to produce a circuit $C_i'$ which is not $\sigma_q$-compatible.  The sum $C_1'+C_2'$ is in the kernel of $A$ and does not contain $e$, therefore we can apply Lemma \ref{lem:circuitdecomposition} and decompose $C_1'+C_2'$ into a sum of directed circuits not containing $e$ such that the signs of the elements are inherited from $C_1'+C_2'$.  Let $e'$ be the minimum labeled element in $C_1' \cup C_2'$.  It is possible that $e'$ appears in only one of the circuits $C_1'$ or $C_2'$, otherwise it must be oriented similarly in both $C_1'$ and $C_2'$ as they are not $\sigma_q$-compatible. Thus $e'$ is in the support of $C_1'+C_2'$, and there exists a circuit $C_3$ containing $e'$ whose support is contained in the support of $C_1'+C_2'$.  Moreover, $C_3$ has size larger than 2 as $e'$ was oriented similarly in $C_1'$ and $C_2'$, thus it does not correspond to a bioriented element of $\mathcal{O}$.  By assumption, $e'$ is oriented in disagreement with its reference orientation, therefore $C_3$ is not $\sigma_q$-compatible.  After possibly rebiorienting some of the elements in $C_3$, we obtain a potential circuit in $\mathcal{O}$ which is not $\sigma_q$-compatible. This contradicts the assumption that $\mathcal{O} \in X_{t,M\setminus e}$. 

Let $X_{t,M}$ be the set of circuit connected $t$-partial orientations of $M$ (cf.~Figure~\ref{fig:Fig5}).
Let $e$ be the largest element of $M$.  
If $e$ is a loop, then $|X_{t,M}| = |X_{t,M\setminus e}|$, and if $e$ is a coloop, then $|X_{t,M}| = (t+2)|X_{t,M/e}|$.  
If $e$ is neither a bridge nor a loop, we claim that $|X_{t,M}| = |X_{t,M\setminus e}| + (t+1)|X_{t,M/e}|$.  
Given this claim, we conclude from Proposition \ref{thm:gentutte} that $$|X_{t,M}| =(t+1)^{\rm rk(M)}T_{M}(\frac{t+2}{t+1},1) =  q^{{\rm rk}(M)}T_{M}(\frac{q+1}{q},1)$$
as desired.

Take $\mathcal{O} \in X_{t,M}$ and let $\mathcal{O}_e$ be the set of $t$-partial orientations in $X_{t,M}$ which agree with $\mathcal{O}$ away from $e$.  We first observe that $\mathcal{O}_e$ includes a $t$-partial orientation with $e$ bioriented if and only if it includes $t$-partial orientations with $e$ oriented in each direction.  Furthermore, this is the case if and only if $\mathcal{O}/e \in X_{t,M/ e}$.  We always have that $\mathcal{O} \setminus e \in X_{t,M \setminus e}$ as deleting $e$ cannot cause a $t$-partial orientation to stop being circuit connected. Therefore, $|\mathcal{O}_e| =1 $ if and only if $\mathcal{O}/e \notin X_{t,M/e}$, and $|\mathcal{O}_e| = t+2 $ if and only if $\mathcal{O}/e \in X_{t,M/e}$.  The claim now follows by partitioning  $X_{t,M}$ into maximal sets of $t$-partial orientations which agree on every element except $e$.  
\end{proof}



\begin{remark}
The realizable part of the Bohne-Dress theorem states that the regular tilings of $Z_A$ by paralleletopes are dual to the generic perturbations of the central hyperplane arrangement defined by $A$. Hopkins and Perkinson \cite{hopkins2016bigraphical} investigated generic bigraphical arrangements, i.e. generic perturbations of twice the graphical arrangement,  and associated certain partial orientations, which they called {\it admissible}, to the regions in the complement of such an arrangement.  The aforementioned duality induces a geometric bijection between these regions and the lattice points in the twice-dilated graphical zonotope.  This in turn gives a bijection between the admissible partial orientations and the circuit connected partial orientations.  The enumeration of these two different classes of partial orientations both appear as specializations of the 
12-variable expansion of the Tutte polynomial from \cite{backman2015fourientation}, and the aforementioned duality interchanges a pair of symmetric variables.
\end{remark}

\subsection{Ehrhart reciprocity}\label{Ehrhartrec}



 Ehrhart reciprocity states that if $P$ is an integral polytope, and $E_P(q)$ is its Ehrhart polynomial, then the number of interior points of the $q$-th dilate of $P$ is $|E_P(-q)|$. Combining Ehrhart reciprocity and Stanley's result, one obtains the following corollary:
 
 
\begin{corollary} \label{cor:reciprocity}
The number of interior lattice points in $qZ_A$ is $$q^{{\rm rk}(M)}T_{M}(1-1/q,1).$$
\end{corollary}

\begin{remark} \label{rmk:reciprocity}
 Our proof of Stanley's formula also allows for a direct verification of Corollary~\ref{cor:reciprocity} in this setting without appealing to Ehrhart reciprocity.  Each facet of $qZ$ is determined by a positive cocircuit in $qM$.  Thus a point lies in the interior of $qZ$ if and only if the corresponding $\sigma_q$-compatible orientation of $qM$ contains no positive cocircuits, or equivalently, if every element in the corresponding circuit connected $(q-1)$-partial orientation of $M$ is contained in a potential circuit.  One can verify that these objects are enumerated by the corresponding Tutte polynomial specialization via deletion-contraction as illustrated above, although the argument is slightly more involved as one needs to take care to show that potential circuits and cocircuits can be treated separately.
 
 
For the case of graphs, various generalizations of the arguments used in the proof of Theorem~\ref{Ehrhart} are given in \cite{backman2015fourientation}. 
\end{remark}

\subsection{Other invariants of unimodular zonotopes}

The following theorem collects some known connections between evaluations of the Tutte polynomial and geometric quantities associated to unimodular zonotopes.

\begin{theorem}\label{thm:Tutteevals}
Let $Z$ be a unimodular zonotope.  Then:

\begin{itemize}
    \item $T_{M}(2,1)$ is the number of lattice points in $Z$.
    
    \item $T_{M}(0,1)$ is the number of interior lattice points in $Z$.
    
    \item $T_{M}(1,1)$ is the volume of $Z$.
    
    \item $T_{M}(2,0)$ is the number of vertices of $Z$.
    
\end{itemize}

\end{theorem}

\begin{proof}
The first two formulas follow from evaluating the Ehrhart polynomial at $q=1$ and $q=-1$.  The third follows from interpreting the volume of $Z$ as $${\rm Vol}(Z) = \lim_{q \rightarrow \infty} \frac{|\mathbb{Z}^n \cap qZ|}{q^{{\rm rk}(
M)}} = \lim_{q \rightarrow \infty} T_{M}(1+1/q,1) = T_{M}(1,1).$$

The fourth enumeration follows from the classical observation that the normal fan of the zonotope is the central hyperplane arrangment defined by $A$ and then applying Zaslavsky's theorem which says that the number of such regions is $T_{M}(2,0)$.
\end{proof}

\begin{remark}
Recall that $T_M(1,1)$ is equal to the number of bases of $M$, which is equal to $|{\rm Jac}(M)|$. One can show that each cell in our polyhedral decomposition of $Z_A$ has volume 1, which gives an alternate proof of the third evaluation in Theorem~\ref{thm:Tutteevals}.   
Taking the limit of $qZ_{A}$ as $q$ goes to infinity while scaling the lattice by $\frac{1}{q}$, the set $X_{q-1,M}$ approaches the set of $\sigma$-compatible continuous orientations of $M$ and we recover the subdivision from Theorem \ref{thm:zonotopedecomp} (see Figure~\ref{fig:Fig5}).
\end{remark}

\begin{figure}[ht!] \label{fig:Fig5}
\begin{center}
    \includegraphics[width=7cm, height = 7cm, keepaspectratio]{zonotopeK3_5.pdf}
\end{center}
  \caption{The set $X_{1,K_3}$ associated to the lattice points of $2Z_{K_3}$ using the acyclic signature $\sigma$ from Figure \ref{zonotopeK3_3}.  The bioriented edges are colored red. Taking the (suitably rescaled) limit of $qZ_{K_3}$ as $q$ goes to infinity,  the set $X_{q-1,K_3}$  induces the subdivision depicted in Figure \ref{zonotopeK3_6}.}
\end{figure}

%%%%%%%%%%%%%%%%%%%%%%%%%%%%%%%%%%%%%%%%%%%%%%%%%%%%%%%%%%%%%%%%%%%%%%

\appendix
\section{A Simple Basis Sampling Algorithm}

Fix an ordering $e_1<\ldots<e_m$ of the elements of a regular matroid $M$. We describe a recursive algorithm\footnote{In the following pseudocode, we assume that $M$ has at least one basis; it can be seen that under this assumption, the program will never run into instances in which the input matroid has no bases.} that finds the $N$-th basis of $M$ in lexicographic order, giving an efficiently computable bijection between $\{1,2,\ldots,b(M)\}$ and the bases of $M$. The algorithm is based on a few basic observations:
\begin{itemize}
\item The family of regular matroids is closed under deletion and contraction.
\item The number of bases $b(M)$ of a regular matroid $M$ can be computed efficiently using the matrix--tree theorem for regular matroids \cite{maurer1976matrix}. 
\item Bases containing $e_1$ are in bijection with the $b(M/e_1)$ bases of $M/e_1$ by $B\leftrightarrow B-e_1$; bases not containing $e_1$ are in bijection with the $b(M\setminus e_1)$ bases of $M\setminus e_1$ by $B\leftrightarrow B$.
\end{itemize}

\begin{algorithm}[h] \caption{Algorithm {\bf Basis} to find the $N$-th basis of a regular matroid.}
\KwIn{A regular matroid $M$, an integer $N$ between 1 and $b(M)$.}
\KwOut{The $N$-th basis $B$ of $M$ in lexicographic order.}
\BlankLine
\If{$|M|=1$}{
	Output $e$, the only element of $M$;\\
	Stop;
}
$e:=$ Smallest element of $M$;\\
$K:=b(M/e)$;\\
\eIf{$N\leq K$}{
	$B:={\rm Basis}(M/e,N)\cup\{e\}$;
}{
	$B:={\rm Basis}(M\setminus e,N-K)$;
}
\BlankLine
Output $B$.
\label{Algo:Simple}
\end{algorithm}

\section*{Acknowledgements}

The first author thanks Sam Hopkins for introducing him to Theorem \ref{thm:Tutteevals}, and Raman Sanyal for explaining that regular tilings of a zonotope can alternately be viewed as dual to generic perturbations of the associated hyperplane arrangement. The third author thanks Yin Tat Lee for the discussion on linear programming.
The second author's work was partially supported by the NSF research grant DMS-1529573.


\newpage

\bibliographystyle{plain}

\bibliography{RM_Geom_Bij}

\medskip

Hausdorff Center for Mathematics, Universit{\"a}t Bonn

53115 Bonn, Germany

Email address: \url{spencer.backman@hcm.uni-bonn.de}\\

School of Mathematics, Georgia Institute of Technology

Atlanta, Georgia 30332-0160, USA

Email address: \url{mbaker@math.gatech.edu}, \url{cyuen7@math.gatech.edu}

\end{document}
