\ifdefined\COMSOC%\ifCLASSOPTIONcompsoc
  \usepackage[nocompress]{cite}%1,2,3,4 will not be compressed as 1-4
\else
  \usepackage{cite} % normal IEEE
\fi
\ifdefined\ACM %ACM has Option clash for package graphicx
	\usepackage{amsmath}
\else
	\usepackage[cmex10]{amsmath} %ACM conflicts with cmex10 option
	\usepackage{amsthm} %ACM conflicts with this on \proof
%\ifCLASSINFOpdf
	\usepackage[pdftex]{graphicx} %\else \usepackage[dvips]{graphicx} \fi
\fi

\ifdefined\META     \usepackage{stmaryrd}    \fi %for \rrbracket

\newcommand{\begproof}{\ifdefined\dbcol\begin{IEEEproof}\else\begin{proof}\fi}
\newcommand{\Endproof}{\ifdefined\dbcol\end{IEEEproof}\else\end{proof}\fi}
\newcommand{\metacom}[1]{\ifdefined\META\bluepure{$\blacktriangleright$}#1\bluepure{$\rrbracket$}\fi}
\newcommand{\metafoot}[1]{\ifdefined\META\footnote{\bluepure{$\blacktriangleright$}#1}\fi}

%\usepackage{mdwmath}
%\usepackage{mdwtab}
% Also highly recommended is Mark Wooding's extremely powerful MDW tools,
% especially mdwmath.sty and mdwtab.sty which are used to format equations
% and tables, respectively. The MDWtools set is already installed on most
% LaTeX systems.

\usepackage[font=small]{subfig} %font=scriptsize/small/footnotesize
%--------- INFOCOM'14 camera-ready version used below -----------
%\usepackage[caption=false,textfont=sf,font=footnotesize]{subfig}
%\usepackage[font=small]{caption}%footnotesize
	%Package caption Warning: Unsupported document class (or package) detected, (caption)  usage of the caption package is not recommended.
%--------- INFOCOM'14 camera-ready version used above -----------

\usepackage{hyperref} %for special characters and smart line breaking in URL

%%%%%%%%%%%%% MY OWN ADDED PACKAGES %%%%%%%%%%%%%%%%%

\usepackage{amssymb}
\usepackage{bbm} %\mathbbm 1 %\newcommand{\ind}{1\hspace{-2.3mm}{1}}
\usepackage{verbatim}
\usepackage{color}
\usepackage[normalem]{ulem}%{soul}
%\usepackage{algorithmic}
%\usepackage{algorithm}
\usepackage[ruled,lined,linesnumbered]{algorithm2e}
%\usepackage{parskip}
%\usepackage{fullpage} %1-inch margin
%\usepackage[margin=1.5cm]{geometry}
%\usepackage[left=1.5cm,right=1.5cm,top=2cm,bottom=2cm,nohead]{geometry}%,nofoot
%\usepackage{anysize}
%\marginsize{2cm}{2cm}{1cm}{1cm} %\marginsize{left}{right}{top}{bottom}
%%%%%%%%%%%%%%%%%%%%%%%%%%%%%%%%%

% fontsize illustration: http://tex.stackexchange.com/questions/24599/what-point-pt-font-size-are-large-etc
%Command             10pt    11pt    12pt
%\tiny               5       6       6
%\scriptsize         7       8       8
%\footnotesize       8       9       10
%\small              9       10      10.95
%\normalsize         10      10.95   12
%\large              12      12      14.4
%\Large              14.4    14.4    17.28
%\LARGE              17.28   17.28   20.74
%\huge               20.74   20.74   24.88
%\Huge               24.88   24.88   24.88

\newtheorem{thm}{Theorem}
\newtheorem{lem}{Lemma}
\newtheorem{prop}{Proposition}
\newtheorem{coro}{Corollary}
\newtheorem{defn}{Definition}
\newtheorem{hypo}{Hypothesis}
\newtheorem{conj}{Conjecture}
\newtheorem{ex}{Example}
\newcommand{\eref}[1]{Eq.~\eqref{#1}} %IEEE style manual suggests using (1) only
\newcommand{\fref}[1]{Fig.~\ref{#1}}
\newcommand{\tref}[1]{Table~\ref{#1}}
\newcommand{\sref}[1]{Section~\ref{#1}}
\newcommand{\thmref}[1]{Theorem~\ref{#1}}
\newcommand{\lref}[1]{Lemma~\ref{#1}}
\newcommand{\pref}[1]{Proposition~\ref{#1}}
\newcommand{\cref}[1]{Corollary~\ref{#1}}
\newcommand{\dref}[1]{Definition~\ref{#1}}
%\newcommand{\href}[1]{Hypothesis~\ref{#1}}
\newcommand{\jref}[1]{Conjecture~\ref{#1}}
\newcommand{\aref}[1]{Algorithm~\ref{#1}}
\newcommand{\pcref}[1]{Procedure~\ref{#1}}
\newcommand{\vect}[1]{\boldsymbol{#1}}
\newcommand{\veclong}[1]{\overrightarrow{#1}} %but the arrow is a bit too big
\newcommand{\sumiN}{\sum_{i=1}^N}
\newcommand{\sumin}{\sum_{i=1}^n}
\newcommand{\suml}{\sum_{l=1}^N}
\newcommand{\sumk}{\sum_{k=1}^N}
\newcommand{\prodi}{\prod_{i=1}^N}
\newcommand{\blue}[1]{{\color{blue}\dotuline{#1}}}
\newcommand{\bluepure}[1]{{\color{blue}{#1}}}
%\newcommand{\blue}[1]{#1}
\newcommand{\red}[1]{{\color{red}\dashuline{#1}}}
%\newcommand{\red}[1]{#1}
\newcommand{\tts}[1]{\tt\small{#1}}
\newcommand{\ttscr}[1]{\tt\scriptsize{#1}}
\newcommand{\ttfoot}[1]{\tt\footnotesize{#1}}
\newcommand{\its}[1]{\it\small{#1}}
\newcommand{\itscr}[1]{\it\scriptsize{#1}}
\newcommand{\itfoot}[1]{\it\footnotesize{#1}}
\newcommand{\eqv}{\Leftrightarrow}%\iff is quite long
\newcommand{\means}{\Rightarrow}
\newcommand{\eps}{\epsilon}
\newcommand{\ovl}{\overline}
\newcommand{\nn}{\nonumber}
\newcommand{\opd}{\operatorname{d}\!}
\renewcommand{\d}[2]{\frac{\opd#1}{\opd#2}} % for derivatives
\newcommand{\dd}[2]{\frac{\opd^2\!#1}{\opd#2^2}} % for double derivatives
\newcommand{\pd}[2]{\frac{\partial #1}{\partial #2}} % for partial derivatives
\newcommand{\pdd}[2]{\frac{\partial^2 #1}{\partial {#2}^2}} % for double partial derivatives
\newcommand{\pddd}[3]{\frac{\partial^2 #1}{\partial {#2} \partial {#3}}}
\newcommand{\inv}[1]{\frac{1}{#1}} %inverse
\DeclareMathOperator*{\argmax}{arg\,max}
\DeclareMathOperator*{\argmin}{arg\,min}
\DeclareMathOperator\erf{erf}
\DeclareMathOperator*{\E}{\mathbb{E}}% Expectation symbol; Using the starred version of \DeclareMathOperator makes sure subscripts goes _beneath_ the symbol in display mode.

%\floatname{algorithm}{Procedure}
%\renewcommand{\algorithmicrequire}{\textbf{Input:}}
%\renewcommand{\algorithmicensure}{\textbf{Output:}}

\hyphenation{op-tical net-works semi-conduc-tor}
