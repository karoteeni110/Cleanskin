\section{Introduction}
Water-in-oil emulsions are commonly formed during petroleum production and pose serious threats to installations and quality of the final product. The electrical phase separation has been used in the petroleum industry for separating water-in-crude oil dispersion's by applying a high electric field onto the flowing emulsion to affect flocculate and coalescence of dispersed water droplets \cite{Cottrell1991,eow02}.
%Cottrell19912, 2001173,  
It has been realized that the emulsion is stabilized by a thin film formed between two drops when approaching each other. Thus demulsification requires rupturing of this thin liquid film. 
%\cite{Bhardwaj1994}. 
Generally, the main purpose of an applied electrical field is to promote contact between the drops and to help in drop\textendash drop coalescence. Pulsed DC (direct current) and AC (alternative current) electric fields are preferred over constant DC fields for efficient coalescence. Recent studies have helped to clarify important aspects of the process such as partial coalescence and drop\textendash drop non-coalescence but key phenomena such as thin film breakup and chain formation are still unclear \cite{Mhatre2015}.
Despite of the tremendous practical importance of enhanced coalescence, the mechanism of separation  is not fully understood \cite{Isaacs01} beyond the  perception that the electrical force facilitates the coalescence between small drops.\\
To help in understanding the inherent processes, computational models were designed  to simulate coalescense of droplets under realistic experimental conditions. Molecular dynamics (MD) method is an useful tool for the purpose.  Koplik and Banavar \cite{Koplik2} did a pioneer work in modeling the coalescence of two Lennard-Jones liquid droplets in a second immiscible fluid using MD simulations. The authors found that coalescence of liquid droplets  was completely driven by van der Waals and electrostatic interactions when the velocities of the droplets were small. The coalescence began when the molecules on the boundary of one droplet thermally drifted to the range of attraction of the other droplet and formed a string to attract both sides of the molecules. \\
Zhao et al.\cite{Zhao2004} reported a MD study of the coalescence of two nanometer-sized water droplets in n-heptane, a system that is commonly encountered in the oil sands industry. Similarly, the coalescence process was initiated by the molecules at the edge of the clusters, which interacted with each other and formed a bridge between two clusters. Eventually, these molecules attracted and pulled out other molecules from their own respective cluster to interact with those from the other cluster. Authors made an important conclusion that the coalescence in n-heptane would occurred only if the two droplets were very close to each other ($\sim 0.5\, nm$). If they were far apart (e.g., $1\, nm$), external driving forces should be applied. 

However, experimental results for electrical properties and electric field-induced rupture of single thin films are scarce, which limits the comparison with computations to several measurable quantities - pore formation, and the critical voltage for film rupture.\\ Anklam et al.  \cite{Anklam1999} experimentally demonstrated that the electric-filed induced pore formation was the reason for break-up of emulsion films. Panchev et al. \cite{Panchev200874} 
developed a method allowing simultaneous investigation of a single water-in-oil emulsion film by both microinterferometry and electrical measurements. This method allows in a single experiment to measure the critical voltage of film rupture, the film thickness, the drainage rate, and the disjoining pressure laying the groundwork for computational studies. \\
 
In this paper we present computational results for pore formation and film rupture obtained with a model, which we have designed to imitate the rupture of the film under a step-wise increase of the electric field as it has been applied  in the experiment \cite{Panchev200874}. The film is immersed in a sodium chloride solution. In the model and also in the experiment, the electric field is applied perpendicularly to the film, which separates two water droplets.\\

The model of the thin film developed for the present study can be considered as a useful starting basis  for a further study of the stability and the structure of thick emulsion films that are stabilized by indigenous crude oil surfactants, namely asphaltenes, resins and naphthenic acids. It is worth mentioning that so far there is almost complete lack of understanding of the intimate structural details of the crude petroleum-like films. Therefore, current industrial practice of utilization of chemical additives in combination with electric field applications has for long time been widely viewed as a ``work of art''. 
