\section{Conclusions}
The results of MD simulations and their analysis offer insight into the intimate structure of the film, namely the presence of a toluene core, neighboring a mixed boundary zone that contains altogether toluene and water molecules.  Application of external DC electric field leads to redistribution of electrical charges and to the accumulation of oppositely charged ions ($Na^+$ and $Cl^-$) on both sides of the film. Thus, the behavior of the system resembles a liquid capacitor, which charge increases with the rise of the external potential. 
In both $NVT$ and $NPT$ ensembles, {\it condenser plates}, where the charge density is maximal, are situated at the very  border between the bulk aqueous (water) phase and the mixed layer. No ion penetration is observed within the toluene core, thus leaving all the distribution of charges within the mixed zone and the bulk phase that could be attributed to the formation of hydration shells. When critical electric field is reached, within a certain time after the field application electric discharge occurs, indicating the beginning of the rupturing process. Visual snapshots of the evolution of the film area confirm the formation of a hole within the thinnest part of the initially non\textendash homogeneously thin film. 

Results clearly show that in $NPT$ simulations the critical instability is developed at much lower fields (75 $mV/nm$) than in NVT simulations (120 $mV/nm$). First experimental investigation on electro-induced rupture of real toluene-diluted bitumen emulsion films \cite{Panchev200874} shows that critical fields range between 4 and 11 $mV/nm$, depending on the bitumen concentration. Thus, $NPT$ simulation with a constant surface tension appears to be a better choice for further modeling of the systems that resemble more close the real films.
In the $NPT$ ensemble  we can expect that even lower values of the external electric field could rupture the toluene film if we prolong the simulation time. The compressive  action of the built\textendash up charges on both sides of the film is illustrated in the decrease of the thickness of the toluene core with the electric field. The behavior of the system resembles a capacitor with increasing charge with an increase of the external potential. However, in both type of simulations ($NVT$ and $NPT$), the width of the mixed zone and hence of the total film increases with the field increase. 
%The hypothesis that toluene molecules leave the core and enter the mixed zone together with the water molecules entering the zone from the bulk side, thus enlarging it, should be further elucidated in much greater detail by performing (…. some integral analysis?...). 
Moreover, the clarification of the detailed mechanism of the hole formation (wave\textendash like or pore\textendash like) and the role of thickness fluctuations on the rupturing process could be progressed through undertaking an extensive ``subbox'' thickness investigation, when the entire simulation box is divided into boxes along the xy\textendash plane, as each one of them being analyzed. 

In conclusion, we may argue that the model, we have developed for thin films, provides a ground for implementing a further complication of the investigated system, introducing surface active molecules, as well as a verification of our expectances for a decrease of the critical electric field when longer simulations are performed.     



