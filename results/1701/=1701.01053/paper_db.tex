%%%%%%%%%%%%%%%%%%%%%%%%%%%%%%%%%%%%%%%%%%%%%%%%%%%%%%%%%%%%%%%%%%%%%
%% This is a (brief) model paper using the achemso class
%% The document class accepts keyval options, which should include
%% the target journal and optionally the manuscript type.
%%%%%%%%%%%%%%%%%%%%%%%%%%%%%%%%%%%%%%%%%%%%%%%%%%%%%%%%%%%%%%%%%%%%%
\documentclass[journal=langmuir,manuscript=article]{achemso}
%\setkeys{acs}{articletitle=true}
%\documentclass[journal=jacsat,manuscript=article]{achemso}

%%%%%%%%%%%%%%%%%%%%%%%%%%%%%%%%%%%%%%%%
\usepackage{color}
%%%%%%%%%%%%%%%%%%%%%%%%%%%%%%%%%%%%%%%%

%%%%%%%%%%%%%%%%%%%%%%%%%%%%%%%%%%%%%%%%%%%%%%%%%%%%%%%%%%%%%%%%%%%%%
%% Place any additional packages needed here.  Only include packages
%% which are essential, to avoid problems later. Do NOT use any
%% packages which require e-TeX (for example etoolbox): the e-TeX
%% extensions are not currently available on the ACS conversion
%% servers.
%%%%%%%%%%%%%%%%%%%%%%%%%%%%%%%%%%%%%%%%%%%%%%%%%%%%%%%%%%%%%%%%%%%%%
\usepackage[version=3]{mhchem} % Formula subscripts using \ce{}

%%%%%%%%%%%%%%%%%%%%%%%%%%%%%%%%%%%%%%%%%%%%%%%%%%%%%%%%%%%%%%%%%%%%%
%% If issues arise when submitting your manuscript, you may want to
%% un-comment the next line.  This provides information on the
%% version of every file you have used.
%%%%%%%%%%%%%%%%%%%%%%%%%%%%%%%%%%%%%%%%%%%%%%%%%%%%%%%%%%%%%%%%%%%%%
%%\listfiles

%%%%%%%%%%%%%%%%%%%%%%%%%%%%%%%%%%%%%%%%%%%%%%%%%%%%%%%%%%%%%%%%%%%%%
%% Place any additional macros here.  Please use \newcommand* where
%% possible, and avoid layout-changing macros (which are not used
%% when typesetting).
%%%%%%%%%%%%%%%%%%%%%%%%%%%%%%%%%%%%%%%%%%%%%%%%%%%%%%%%%%%%%%%%%%%%%
\newcommand*\mycommand[1]{\texttt{\emph{#1}}}

%%
%% Custom packages
%%
%\usepackage{biblatex}
\usepackage{graphicx}
\usepackage{subcaption}
%\usepackage{mwe}
\usepackage{relsize} 
%% Added by Desi
%\usepackage{slashbox}

%%%%%%%%%%%%%%%%%%%%%%%%%%%%%%%%%%%%%%%%%%%%%%%%%%%%%%%%%%%%%%%%%%%%%
%% Meta-data block
%% ---------------
%% Each author should be given as a separate \author command.
%%
%% Corresponding authors should have an e-mail given after the author
%% name as an \email command. Phone and fax numbers can be given
%% using \phone and \fax, respectively; this information is optional.
%%
%% The affiliation of authors is given after the authors; each
%% \affiliation command applies to all preceding authors not already
%% assigned an affiliation.
%%
%% The affiliation takes an option argument for the short name.  This
%% will typically be something like "University of Somewhere".
%%
%% The \altaffiliation macro should be used for new address, etc.
%% On the other hand, \alsoaffiliation is used on a per author basis
%% when authors are associated with multiple institutions.
%%%%%%%%%%%%%%%%%%%%%%%%%%%%%%%%%%%%%%%%%%%%%%%%%%%%%%%%%%%%%%%%%%%%%
\author{Desislava Dimova}
\affiliation[University of Sofia]
{Department of Atomic Physics, University of Sofia}
\author{Stoyan Pisov}
\affiliation[University of Sofia]
{Department of Atomic Physics, University of Sofia}
\email{pisov@phys.uni-sofia.bg}
\phone{+359 (0)2 8161828}
\author{Nikolay Panchev}
\affiliation[Bulgarian Academy of Sciences]
{Institute of Physical Chemistry}
\email{patcho75@yahoo.com}
\author{Miroslava Nedyalkova}
\affiliation[University of Sofia]
{Department of General and Inorganic Chemistry, University of Sofia}
\author{Sergio Madurga}
\email{s.madurga@ub.edu}
\affiliation[University of Barcelona]
{Material Science and Physical Chemistry Department and IQTCUB}
\author{Ana Proykova}
\email{anap@phys.uni-sofia.bg}
\affiliation[University of Sofia]
{Department of Atomic Physics, University of Sofia}

%%%%%%%%%%%%%%%%%%%%%%%%%%%%%%%%%%%%%%%%%%%%%%%%%%%%%%%%%%%%%%%%%%%%%
%% The document title should be given as usual. Some journals require
%% a running title from the author: this should be supplied as an
%% optional argument to \title.
%%%%%%%%%%%%%%%%%%%%%%%%%%%%%%%%%%%%%%%%%%%%%%%%%%%%%%%%%%%%%%%%%%%%%
\title[Article title]
  {A model provides insight into electric field-induced rupture mechanism of water-in-toluene emulsion films
}

%%%%%%%%%%%%%%%%%%%%%%%%%%%%%%%%%%%%%%%%%%%%%%%%%%%%%%%%%%%%%%%%%%%%%
%% Some journals require a list of abbreviations or keywords to be
%% supplied. These should be set up here, and will be printed after
%% the title and author information, if needed.
%%%%%%%%%%%%%%%%%%%%%%%%%%%%%%%%%%%%%%%%%%%%%%%%%%%%%%%%%%%%%%%%%%%%%
\abbreviations{IR,NMR,UV}
\keywords{American Chemical Society, \LaTeX}

%%%%%%%%%%%%%%%%%%%%%%%%%%%%%%%%%%%%%%%%%%%%%%%%%%%%%%%%%%%%%%%%%%%%%
%% The manuscript does not need to include \maketitle, which is
%% executed automatically.
%%%%%%%%%%%%%%%%%%%%%%%%%%%%%%%%%%%%%%%%%%%%%%%%%%%%%%%%%%%%%%%%%%%%%
\begin{document}

%%%%%%%%%%%%%%%%%%%%%%%%%%%%%%%%%%%%%%%%%%%%%%%%%%%%%%%%%%%%%%%%%%%%%
%% The "tocentry" environment can be used to create an entry for the
%% graphical table of contents. It is given here as some journals
%% require that it is printed as part of the abstract page. It will
%% be automatically moved as appropriate.
%%%%%%%%%%%%%%%%%%%%%%%%%%%%%%%%%%%%%%%%%%%%%%%%%%%%%%%%%%%%%%%%%%%%%
%\begin{tocentry}
%\begin{center}
%\includegraphics[width=7.5cm]{toc_new.eps}
%\end{center}
%\end{tocentry}

%%%%%%%%%%%%%%%%%%%%%%%%%%%%%%%%%%%%%%%%%%%%%%%%%%%%%%%%%%%%%%%%%%%%%
%% The abstract environment will automatically gobble the contents
%% if an abstract is not used by the target journal.
%%%%%%%%%%%%%%%%%%%%%%%%%%%%%%%%%%%%%%%%%%%%%%%%%%%%%%%%%%%%%%%%%%%%%

\include{abstract}

%%%%%%%%%%%%%%%%%%%%%%%%%%%%%%%%%%%%%%%%%%%%%%%%%%%%%%%%%%%%%%%%%%%%%
%% Start the main part of the manuscript here.
%%%%%%%%%%%%%%%%%%%%%%%%%%%%%%%%%%%%%%%%%%%%%%%%%%%%%%%%%%%%%%%%%%%%%
%\section{Introduction}
%!TEX root = /Users/audrey/Dropbox/PhD/MOMAB/ArXiv/Latex/paper.tex

\section{Introduction}
\label{sec:intro}

Multi-objective optimization (MOO)~\cite{Coello2007} is a topic of great importance for real-world applications. Indeed, optimization problems are characterized by a number of conflicting, even contradictory, performance measures relevant to the task at hand. For example, when deciding on the healthcare treatment to follow for a given sick patient, a trade-off must be made between the efficiency of the treatment to heal the sickness, the side effects of the treatment, and the treatment cost. MOO is often tackled by combining the objective into a single measure (a.k.a.~scalarization). Such approaches are said to be \emph{a priori}, as the preferences over the objectives is defined before carrying out the optimization itself. The challenge lies in the determination of the appropriate scalarization function to use and its parameterization. Another way to conduct MOO consists in learning the optimal trade-offs (the so-called Pareto-optimal set). Once the optimization is completed, techniques from the field of multi-criteria decision-making are applied to help the user to select the final solution from the Pareto-optimal set. These \emph{a posteriori} techniques may require a huge number of evaluations to have a reliable estimation of the objective values over all potential solutions. Indeed, the Pareto-optimal set can be quite large, encompassing a majority, if not all, of the potential solutions. In this work, we tackle the MOO problem where the scalarization function \emph{exists} a priori, but might be unknown, in which case a user can act as a black box for articulating preferences. Integrating the user to the learning loop, she can provide feedback by selecting her preferred choice given a set of options -- the scalarization function lying in her head.

More specifically, we consider problems where outcomes are stochastic and costly to evaluate (e.g., involving a human in the loop). The challenge is therefore to identify the best solutions given random observations sampled from different (unknown) density distributions. We formulate this problem as multi-objective bandits, where we aim at finding the solution that maximizes the preference function while maximizing the performance of the solutions evaluated during the optimization. The Thompson sampling (TS)~\cite{Thompson1933} technique is a typical approach for bandits problems, where potential solutions are tried based on a Bayesian posterior over their expected outcome. Here we consider TS from multivariate normal (MVN) priors for multi-objective bandits.
% Let the \emph{right choice} denote the option that maximize the preference function -- the option that the user would select given that she had knowledge of the Pareto-optimal set. A learning algorithm for the multi-objective bandits setting aims at learning good-enough estimations of the available options to allow the user to make the right choices and its performance depends on the robustness of the preference function to the quality of estimations. We therefore need a measure for characterizing the quality of estimations required in order for the option maximizing the preference function to remain unchanged. For that purpose, we introduce the concept of preference radius providing the tolerance range over objective value estimations, such that the user preference would remain the same as if the Pareto-optimal set was known. We use this concept for providing a theoretical analysis of TS from MVN priors.
We introduce the concept of preference radius providing the tolerance range over objective value estimations, such that the \emph{best option} given the preference function remains unchanged. We use this concept for providing a theoretical analysis of TS from MVN priors.
%
Finally, we perform some empirical experiments to support the theoretical results and also highlight the importance of tackling multi-objective bandits problems as such instead of scalarizing those under the traditional bandit setting. 

% The original contributions of the paper consist in:
% \begin{itemize}
%     \item providing a general formulation of the MOO under the a priori multi-objective bandits setting;
%     \item proposing the preference radius to characterize the robustness of the preference function to the estimations quality;
%     \item proposing a theoretical analysis of the TS algorithm from MVN priors;
%     \item showing with empirical experiments that multi-objective bandits cannot simply be brought back to single-objective bandits.
% \end{itemize}


\section{The model and simulation procedure}

To accurately simulate the interfacial phenomena, we have applied the classical MD method for the case of two canonical ensembles - $NVT$  and  $NPT$. The choice of ensembles in MD simulations of finite-size systems has already been shown to play an important role in coexisting phases \cite{Pisov_2001}. MD Simulations provide detailed information on the molecular structure of the interface when the intermolecular potential is available \cite{Koplik,Koplik2014,Pisov_2012}.\\

The model system is a $5\, nm$ thick toluene film located perpendicularly to the $z$-axis  of the simulation box. The size of the box  $24.8 \times 24.8 \times 24.8\, nm$ ensures that no artifacts will appear when $3D$ periodic boundary conditions are implemented to diminish finite-size effects. The box contains also  water molecules and $Na^+$ and $Cl^-$ ions at a concentration of $1M$.  The force field parameters of the  ions $Na^+$ and $Cl^-$ included in the model are  taken from {\it Gromos96}  \cite{doi:10.1021/jp984217f}. Parameters for toluene molecules are derived from  benzyl side chain of phenylalanine molecule. Three-site $SPC$ (simple point charge) water model is used \cite{doi:10.1021/j100308a038}.

Large-scale molecular dynamics simulations of the model system are performed with the help of the  {\it GROMACS} package, designed to simulate the Newtonian equations of motion for systems with hundreds to millions of particles \cite{Berendsen199543}. The simulations are performed in  canonical $NVT$  and  $NPT$ ensembles which keep the total number of atoms constant; the temperature is $T = 298 K$.  In the $NVT$ ensemble the constant volume equals to the size of the simulation box, $24.8 \times 24.8 \times 24.8\, nm$. In the case of the $NPT$ ensemble the system is equilibrated at the constant pressure of $1\, bar$. After the equilibration, the simulation is performed at a constant surface tension $\gamma = 36.4\, mN/m$ between toluene and water \cite{Drelich_2002}. 
 In preliminary MD runs the  simulation time of  $5\, ns$ was determined to be sufficient for  thermodynamic equilibration of the total energy,  pressure, and temperature of the model system.\\
 An external electric field is applied  in the $z$ direction of the simulation box. In the $NVT$ ensemble the electric field strength is changed from $0$ to $120\, mV/nm$ in steps of $20\, mV/nm$,  while in the $NPT$ ensemble the strength is changed   from $0$ to $75\, mV/nm$ in steps of $25\, mV/nm$ .

\include{results}

%!TEX root = ../wbi.tex
\section{Conclusions}
\label{sec:conclusions}

In this paper we presented a software abstraction layer to simplify the development of whole-body controllers.
While there are already some whole-body control software libraries, they already define the controller structure and leave to the user only the possibility to specify objectives and constraints.

On the other hand the proposed library leaves complete freedom to the control designer by exposing all the information needed. It does not make any assumptions on the controller structure.
The whole-body abstraction library presents also the following advantages:
\begin{itemize}
    \item it decouples the writing of the controller from a particular robot implementation
    \item it decouples the writing of the controller from a specific dynamic library implementation
    \item it allows more concise and clear code as it represents uniquely the code needed to implement the mathematical formulation of the controller. All the implementation details are left to the library
    \item it allows to benchmark the controller on different platforms or with different implementations.
\end{itemize}
Furthermore, the possibility to expose the functionality at an higher level than C++ facilitates the writing of controllers as the results on the iCub robot clearly prove.

We voluntarily did not consider some aspects as they are out of the scope of the present contribution. 
Nevertheless they must be taken into account when a controller is implemented and used on the real system.
In particular the following details should be considered:
\begin{itemize}
    \item how are controllers run on the platform? Do they run as threads?
    \item how are controllers configured and initialized?
    \item how is communication with other software performed? For example, how are desired values provided to the controller, coming from a planner or higher-level control loop?
\end{itemize}
By not considering these details in the abstraction library, we render the library portable to different systems.
Indeed, the actual control law is not concerned by the previously listed implementation details.

While the more complex demos have been achieved by directly executing the Simulink model connected to the robot, we recognize the need to automatically generate self-contained C++ code.
The advantage is twofold.
On one side the autogenerated code is in general more optimized than the code directly executed in Simulink, even if less optimized than ad-hoc C++ code.
On the other side, this would remove the requirement of having a Simulink installation on the computers controlling the robot.


%\subsection{References} 

%\include{figures}


%\section{Extra information when writing JACS Communications}

%When producing communications for \emph{J.~Am.\ Chem.\ Soc.}, the
%class will automatically lay the text out in the style of the
%journal. This gives a guide to the length of text that can be
%accommodated in such a publication. There are some points to bear in
%mind when preparing a JACS Communication in this way.  The layout
%produced here is a \emph{model} for the published result, and the
%outcome should be taken as a \emph{guide} to the final length. The
%spacing and sizing of graphical content is an area where there is
%some flexibility in the process.  You should not worry about the
%space before and after graphics, which is set to give a guide to the
%published size. This is very dependent on the final published layout.

%You should be able to use the same source to produce a JACS
%Communication and a normal article.  For example, this demonstration
%file will work with both \texttt{type=article} and
%\texttt{type=communication}. Sections and any abstract are
%automatically ignored, although you will get warnings to this effect.

%%%%%%%%%%%%%%%%%%%%%%%%%%%%%%%%%%%%%%%%%%%%%%%%%%%%%%%%%%%%%%%%%%%%%
%% The "Acknowledgement" section can be given in all manuscript
%% classes.  This should be given within the "acknowledgement"
%% environment, which will make the correct section or running title.
%%%%%%%%%%%%%%%%%%%%%%%%%%%%%%%%%%%%%%%%%%%%%%%%%%%%%%%%%%%%%%%%%%%%%
\include{acknowledgement}

%Please use ``The authors thank \ldots'' rather than ``The
%authors would like to thank \ldots''.

%The author thanks Mats Dahlgren for version one of \textsf{achemso},
%and Donald Arseneau for the code taken from \textsf{cite} to move
%citations after punctuation. Many users have provided feedback on the
%class, which is reflected in all of the different demonstrations
%shown in this document.
%
%\end{acknowledgement}

%%%%%%%%%%%%%%%%%%%%%%%%%%%%%%%%%%%%%%%%%%%%%%%%%%%%%%%%%%%%%%%%%%%%%
%% The same is true for Supporting Information, which should use the
%% suppinfo environment.
%%%%%%%%%%%%%%%%%%%%%%%%%%%%%%%%%%%%%%%%%%%%%%%%%%%%%%%%%%%%%%%%%%%%%
%\begin{suppinfo}
%
%This will usually read something like: ``Experimental procedures and
%characterization data for all new compounds. The class will
%automatically add a sentence pointing to the information on-line:
%
%\end{suppinfo}

%%%%%%%%%%%%%%%%%%%%%%%%%%%%%%%%%%%%%%%%%%%%%%%%%%%%%%%%%%%%%%%%%%%%%
%% The appropriate \bibliography command should be placed here.
%% Notice that the class file automatically sets \bibliographystyle
%% and also names the section correctly.
%%%%%%%%%%%%%%%%%%%%%%%%%%%%%%%%%%%%%%%%%%%%%%%%%%%%%%%%%%%%%%%%%%%%%
\bibliographystyle{natbib}
\bibliography{references}

\end{document}
