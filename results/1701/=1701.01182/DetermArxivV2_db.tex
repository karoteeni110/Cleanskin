\documentclass[11pt]{article}
\usepackage[letter, center, frame]{crop}
\usepackage{amssymb, amsmath, amsthm, latexsym, fullpage}
\usepackage{eucal, fullpage, setspace}
\usepackage{ytableau}
\usepackage{undertilde}
\usepackage{cancel}
\usepackage{caption}
\usepackage{graphicx}
\usepackage[top=.95in,bottom=1in,left=1in,right=1in]{geometry}
\usepackage{changepage}
\usepackage[pdfborder={0 0 0}]{hyperref}













\newtheorem{thm}{Theorem}[section]
\newtheorem{lem}[thm]{Lemma}
\newtheorem{prop}[thm]{Proposition}
\newtheorem{cor}[thm]{Corollary}
\newtheorem{proc}[thm]{Procedure}
\newtheorem{fact}[thm]{Fact}
\newtheorem{prob}[thm]{Problem}
\theoremstyle{definition}
\newtheorem{defn}[thm]{Definition}
\theoremstyle{remark}
\newtheorem{rem}[thm]{Remark}
\newtheorem{ex}{Example}
\numberwithin{equation}{section}



\begin{document}




\title{Parabolic Catalan numbers count efficient inputs \\ for Gessel-Viennot flagged Schur function determinant}

\author{Robert A. Proctor \\ University of North Carolina \\ Chapel Hill, NC 27599 U.S.A. \\ rap@email.unc.edu \and Matthew J. Willis \\ Wesleyan University \\ Middletown, CT 06457 U.S.A. \\ mjwillis@wesleyan.edu}

\maketitle



\begingroup
    \fontsize{10pt}{12pt}\selectfont
    \begin{spacing}{1.1}

\centerline{\textbf{Abstract}}

\begin{adjustwidth*}{.5in}{.5in}


\vspace{1pc}Let $\lambda$ be a partition with no more than $n$ parts.  Let $\beta$ be a weakly increasing $n$-tuple with entries from $\{ 1, ... , n \}$.  The flagged Schur function in the variables $x_1, ... , x_n$ that is indexed by $\lambda$ and $\beta$ has been defined to be the sum of the content weight monomials for the semistandard Young tableaux of shape $\lambda$ whose values are row-wise bounded by the entries of $\beta$.  Gessel and Viennot gave a determinant expression for the flagged Schur function indexed by $\lambda$ and $\beta$; this could be done since the pair $(\lambda, \beta)$ satisfied their ``nonpermutable'' condition for the sequence of terminals of an $n$-tuple of certain lattice paths that they used to model the tableaux.  We generalize the notion of flagged Schur function by dropping the requirement that $\beta$ be weakly increasing.  Then we give a condition on the entries of $\lambda$ and $\beta$ for the pair $(\lambda, \beta)$ to be nonpermutable that is both necessary and sufficient.  When the parts of $\lambda$ are not distinct there will be multiple row bound $n$-tuples that will produce the same polynomial via the sum of tableau weights construction on $\lambda$.  We accordingly group the bounding $n$-tuples into equivalence classes and identify the most efficient $n$-tuple in each class for the determinant computation.  We have recently shown that many other sets of objects that are indexed by $n$ and $\lambda$ are enumerated by the number of these efficient $n$-tuples.  It is noted that the $GL(n)$ Demazure characters (key polynomials) indexed by 312-avoiding permutations can also be expressed with these determinants.

\end{adjustwidth*}



\end{spacing}

\endgroup

\vspace{.5pc}\noindent\textbf{Keywords.}  flagged Schur function, Gessel-Viennot method, sign reversing involution, nonintersecting lattice paths, Jacobi-Trudi identity

\vspace{1pc}\noindent\textbf{MSC Codes.}  05E05, 05A19


\begin{spacing}{1.35}



\section{Introduction}

No particular background is needed to read this largely self-contained paper. Several prior results that are needed in Sections 2 and 8 and for Corollaries \ref{cor777.2.5} and \ref{cor777.3} were obtained in the predecessor paper \cite{PW}.



Fix $n \geq 1$ throughout the paper.  Also fix a partition $\lambda$ that has $n$ nonnegative parts; this is a list of $n$ weakly decreasing nonnegative integers.  Flagged Schur functions are polynomials in $x_1, x_2, ... , x_n$ that were introduced by Lascoux and Sch{\"u}tzenberger in 1982 as they studied Schubert polynomials.  Given an $n$-tuple $\beta$ such that $1 \leq \beta_1 \leq \beta_2 \leq ... \leq \beta_n \leq n$, the flagged Schur function indexed by $\lambda$ and $\beta$ is defined to be the sum of the content weight monomial $x^{\Theta(T)}$ over the semistandard tableaux $T$ on the shape of $\lambda$ whose values are row-wise bounded by the respective entries of $\beta$.  Sometimes $\beta_i \geq i$ for $1 \leq i \leq n$ is also required to ensure nonvanishing.  However, in this paper the notation $s_\lambda(\beta;x)$ will more generally denote this sum when $\beta$ is only required to satisfy $\beta_i \geq i$ for $1 \leq i \leq n$.



Ira Gessel and X.G. Viennot were able \cite{GV} to express a flagged Schur function with a determinant by modelling its tableaux with nonintersecting $n$-tuples of lattice paths:  Their initial set-up fixed a sequence of $n$ lattice points to serve as sources for the respective paths, to which were assigned sinks from a set of $n$ fixed lattice points in any of the $n!$ possible ways.  These ``terminal'' lattice points were specified in terms of the entries of $\lambda$ and $\beta$.  Initially $\beta_i \leq \beta_{i+1}$ was not required.  Most of the resulting $n$-tuples of lattice paths contained intersections, and the desired tableaux corresponded only to the nonintersecting $n$-tuples of lattice paths for which the sinks were assigned from the set of terminals in their ``native'' order.  The terms in the signed sum expansion of the proposed determinant gave the weights that they assigned to the $n$-tuples of paths.  Then they introduced a sign reversing involution that paired up the intersecting $n$-tuples of paths so that the weights for these cancelled each other out from the expansion, leaving only the signed sum of the weights for the nonintersecting $n$-tuples of paths.  For this method to give the tableau weight sum $s_\lambda(\beta;x)$, they needed to require that the set of terminals specified by the pair $(\lambda, \beta)$ satisfied their ``nonpermutable'' property:  This required that any $n$-tuple of lattice paths that had a sequence of sinks coming from a nontrivially permuted assignment of the terminals had to contain an intersection.  As Stanley parenthetically noted in his presentation of their work in Theorem 2.7.1 of \cite{St1}, for any $\lambda$ it can be seen that requiring $\beta_1 \leq \beta_2 \leq ... \leq \beta_n$ will guarantee that $(\lambda, \beta)$ is nonpermutable.  Directly in terms of the entries of $\lambda$ and $\beta$, our main result gives a condition for a pair $(\lambda, \beta)$ to be nonpermutable that is necessary as well as being sufficient.  Although the references \cite{GV}, \cite{St1}, and \cite{PS} each provide determinants for skew flagged Schur functions, we limit our considerations to the general sums $s_\lambda(\beta;x)$ on nonskew shapes.



Demazure characters were introduced by Demazure in 1974 when he studied singularities of Schubert varieties.  Coincidences between the Demazure characters for $GL(n)$ (key polynomials) and flagged Schur functions were studied by Reiner and Shimozono \cite{RS} and then by Postnikov and Stanley \cite{PS}.  When the parts of $\lambda$ are not distinct, there are multiple row bound $n$-tuples $\beta, \beta', ...$ that will produce the same polynomial $s_\lambda(\beta;x)$ via the sum of tableau weights construction on the shape of $\lambda$.  The predecessor paper to this paper sharpened, deepened, and extended the results of \cite{RS} and \cite{PS}.  Much machinery was introduced and several special kinds of $n$-tuples were defined.  The foremost kinds were the ``$\lambda$-312-avoiding permutations'' and the ``gapless $\lambda$-tuples''.  The crucial information for an $n$-tuple $\beta$ was distilled into its ``critical list'', as its ``$\lambda$-core'' $\Delta_\lambda(\beta)$ was being computed.



It turns out that the machinery and notions that were introduced in \cite{PW} for the purposes of that paper are surprisingly well-suited to solving the problem of characterizing the nonpermutable pairs $(\lambda, \beta)$ that was implicitly raised by Stanley's Theorem 2.7.1 parenthetical remark.  In addition to re-using the notion of gapless $\lambda$-tuple and the closely related notion of ``gapless core $\lambda$-tuple'', here we also need to extend the $\lambda$-ceiling flag map $\Xi_\lambda$ of \cite{PW} so that we can introduce a new condition that requires $\beta \leq \Xi_\lambda(\beta)$.  Our main result, Theorem \ref{theorem777.1}, presents our characterization of the nonpermutable pairs $(\lambda, \beta)$.  Its two halves are proved with Proposition \ref{prop434.7} and \ref{prop581.6}.  Corollary \ref{cor777.2} gives the determinant expression for $s_\lambda(\beta;x)$ when $\beta$ satisfies the characterization with respect to $\lambda$.  Corollary \ref{cor777.2.5} indicates how results of \cite{PW} can be used to extend the realm of Corollary \ref{cor777.2}.  Corollary \ref{cor777.3} describes how Corollary \ref{cor777.2} can be used to give a determinant expression for certain $GL(n)$ Demazure characters;  this improves upon Corollary 14.6 of \cite{PS}.



In the last section, as in \cite{PW}, we define two row bound $n$-tuples $\beta, \beta'$ to be equivalent if the sets of tableaux on the shape $\lambda$ that satisfy these bounds are the same.  Proposition \ref{prop826.6} describes the equivalence classes of this relation within the set of row bound $n$-tuples that meet the criteria required to use the determinant expression.  Within an equivalence class, one can seek the $n$-tuple for which the total number of monomials appearing in the corresponding determinant is as small as possible.  Proposition \ref{prop826.9} identifies these ``maximum efficiency'' $n$-tuples as being the gapless $\lambda$-tuples that appeared in Corollary \ref{cor777.2.5}.  Corollary \ref{cor826.12} then notes that the number of gapless $\lambda$-tuples was shown in \cite{PW} to be the number of $\lambda$-312-avoiding permutations;  there this number was taken to be the definition of the ``parabolic Catalan number'' indexed by $n$ and $\lambda$.



When one sets all $\beta_i := n$, no special row bounds are imposed upon the tableaux and the resulting polynomial is the ordinary Schur function $s_\lambda(x)$.  The Gessel-Viennot method made the Jacobi-Trudi determinant expression of Theorem 7.16.1 of \cite{St2} for $s_\lambda(x)$ more efficient by reducing the number of variables that appeared in most of its entries.  When the parts of $\lambda$ are not distinct, Proposition \ref{prop826.9} says that Corollary \ref{cor777.2.5} provides a determinant for $s_\lambda(x)$ that is even more efficient in this regard.



One of the central themes of the predecessor paper \cite{PW} is continued into this paper.  Given a set $R \subseteq \{ 1, 2, ... , n-1 \}$, an ``$R$-tuple'' is an $n$-tuple with entries from $\{ 1, 2, ... , n \}$ that is equipped with ``dividers'' between some of its entries.  In these two papers the study of any one of the interrelated phenomena concerning sets of tableaux on the shape $\lambda$ begins with the determination of the set $R_\lambda \subseteq \{ 1, 2, ... , n-1 \}$ of the lengths of the columns in $\lambda$ that are less than $n$.  Much of the machinery needed to study these phenomena is formulated in terms of $R_\lambda$-tuples without reference to any other aspects of $\lambda$:  Five preliminary sections of \cite{PW} take place in the world of $R$-tuples, before shapes and tableaux are introduced.  Continuing a notation convention of \cite{PW}, after $\lambda$ has been introduced we replace `$R_\lambda$' in prefixes and subscripts with `$\lambda$'.  This reduces clutter while explicitly retaining the dependence upon $\lambda$, which setting $R := R_\lambda$ would lose.








\section{Definitions for $\mathbf{\emph{n}}$-tuples}

Let $i$ and $k$ be nonnegative integers.  Define $(i, k] := \{i+1, i+2, ... , k\}$ and $[k] := \{1, 2, ... , k \}$.  Except for $\zeta$, lower case Greek letters indicate $n$-tuples of non-negative integers; their entries are denoted with the same letter.  An $nn$-\textit{tuple} $\nu$ consists of $n$ \emph{entries} $\nu_i \in [n]$ indexed by \emph{indices} $i \in [1,n]$, which together form $n$ \emph{pairs} $(i, \nu_i)$.  Let $P(n)$ denote the poset of $nn$-tuples ordered by entrywise comparison.  Fix an $nn$-tuple $\nu$.  A \emph{subsequence} of $\nu$ is a sequence of the form $(\nu_i, \nu_{i+1}, ... , \nu_j)$ for some $i, j \in [n]$.  A \emph{staircase of $\nu$ within a subinterval $[i,j]$} for some $i, j \in [n]$ is a maximal subsequence of $(\nu_i, \nu_{i+1}, ... , \nu_j)$ whose entries increase by 1.  A \emph{plateau} in $\nu$ is a maximal constant nonempty subsequence of $\nu$.  An $nn$-tuple $\varphi$ is a \textit{flag} if $\varphi_1 \leq \ldots \leq \varphi_n$.  An \emph{upper tuple} is an $nn$-tuple $\beta$ such that $\beta_i \geq i$ for $i \in [n]$.




Fix $R \subseteq [n-1]$.  Denote the elements of $R$ by $q_1 < \ldots < q_r$ for some $r \geq 0$.  Set $q_0 := 0$ and $q_{r+1} := n$.  We use the $q_h$ for $h \in [r+1]$ to specify the locations of $r+1$ ``dividers'' within $nn$-tuples:  Let $\nu$ be an $nn$-tuple.  On the graph of $\nu$ in the first quadrant draw vertical lines at $x = q_h + \epsilon$ for $h \in [r+1]$ and some small $\epsilon > 0$.  These $r+1$ lines indicate the right ends of the $r+1$ \emph{carrels} $(q_{h-1}, q_h]$ \emph{of $\nu$} for $h \in [r+1]$.  An \emph{$R$-tuple} is an $nn$-tuple that has been equipped with these $r+1$ dividers.  Fix an $R$-tuple $\nu$;  we portray it by $(\nu_1, ... , \nu_{q_1} ; \nu_{q_1+1}, ... , \nu_{q_2}; ... ; \nu_{q_r+1}, ... , \nu_n)$.  Let $U_R(n)$ denote the subposet of $P(n)$ consisting of upper $R$-tuples.  Let $UF_R(n)$ denote the subposet of $U_R(n)$ consisting of upper flags.  Fix $h \in [r+1]$.  The $h^{th}$ carrel has $p_h := q_h - q_{h-1}$ indices.  An \emph{$R$-increasing tuple} is an $R$-tuple $\alpha$ such that  $\alpha_{q_{h-1}+1} < ... < \alpha_{q_h}$ for $h \in [r+1]$.  Let $UI_R(n)$ denote the subset of $U_R(n)$ consisting of $R$-increasing upper tuples.



We distill the crucial information from an upper $R$-tuple into a skeletal substructure called its ``critical list'', and at the same time define two functions from $U_R(n)$ to $U_R(n)$.  Fix $\beta \in U_R(n)$.  To launch a running example, take $n := 9, R := \{3, 8 \}$, and $\beta := (2,7,5;8,6,6,9,9;9)$.  We will be constructing the images $\delta$ and $\xi$ of $\beta$ under \emph{$R$-core} and \emph{$R$-platform} maps $\Delta_R$ and $\Xi_R$.  Fix $h \in [r+1]$.  Working within the $h^{th}$ carrel $(q_{h-1}, q_h]$ from the right we recursively find for $u = 1, 2, ...$ :  At $u = 1$ the \emph{rightmost critical pair of $\beta$ in the $h^{th}$ carrel} is $(q_h, \beta_{q_h})$.  Set $x_1 := q_h$.  Recursively attempt to increase $u$ by 1:  If it exists, the \emph{next critical pair to the left} is $(x_u, \beta_{x_u})$, where $q_{h-1} < x_u < x_{u-1}$ is maximal such that $\beta_{x_{u-1}} - \beta_{x_u} > x_{u-1} - x_u$.  For $x_u < i \leq x_{u-1}$, write $x_{u-1} =: x$ and set $\delta_i := \beta_x - (x-i)$ and $\xi_i := \beta_x$.  Otherwise, let $f_h \geq 1$ be the last value of $u$ attained.  For $q_{h-1} < i \leq x_{f_h}$, write $x_{f_h} =: x$ and again set $\delta_i := \beta_x - (x-i)$ and $\xi_i := \beta_x$.  The \emph{set of critical pairs of $\beta$ for the $h^{th}$ carrel} is $\{ (x_u, \beta_{x_u}) : u \in [f_h] \} =: \mathcal{C}_h$.  Equivalently, here $f_h$ is maximal such that there exists indices $x_1, x_2, ... , x_{f_h}$ such that $q_{h-1} < x_{f_h} < ... < x_1 = q_h$ and $\beta_{x_{u-1}} - \beta_{x_u} > x_{u-1} - x_u$ for $u \in (1, f_h]$.  The \emph{$R$-critical list for $\beta$} is the sequence $(\mathcal{C}_1, ... , \mathcal{C}_{r+1}) =: \mathcal{C}$ of its $r+1$ sets of critical pairs.  In our example $\mathcal{C} = ( \{ (1,2), (3,5) \}; \{(6,6),(8,9)\}; \{(9,9)\})$ and $\delta = (2,4,5;4,5,6,8,9;9)$ and $\xi = (2,5,5;6,6,6,9,9;9)$.  It can be seen that the $R$-core $\Delta_R(\beta) = \delta$ of $\beta$ and the $R$-platform $\Xi_R(\beta) = \xi$ of $\beta$ have the same critical list as $\beta$.  It can also be seen that $\Delta_R(\beta) \leq \beta$ and that $\Delta_R(\alpha) = \alpha$ for $\alpha \in UI_R(n)$.  If $(x, y_x)$ is a critical pair, we call $x$ a \emph{critical index} and $y_x$ a \emph{critical entry}.  We say that an $R$-critical list is a \emph{flag $R$-critical list} if whenever $h \in [r]$ we have $y_{q_h} \leq y_k$, where $k := x_{f_{h+1}}$.  The example critical list is a flag critical list.  If $\beta \in UF_R(n)$, then its $R$-critical list is a flag $R$-critical list.




A \emph{gapless core $R$-tuple} is an upper $R$-tuple $\eta$ whose critical list is a flag critical list.  Let $UGC_R(n)$ denote the set of gapless core $R$-tuples.  The example $\beta$ above is a gapless core $R$-tuple.  A \emph{gapless $R$-tuple} is an $R$-increasing upper tuple $\gamma$ whose critical list is a flag critical list.  Let $UG_R(n) \subseteq UI_R(n)$ denote the set of gapless $R$-tuples.  The example $\delta$ above is a gapless $R$-tuple.  Originally a gapless $R$-tuple was defined in Section 3 of \cite{PW} to be an $R$-increasing upper tuple $\gamma$ such that whenever there exists $h \in [r]$ with $\gamma_{q_h} > \gamma_{q_h+1}$, then $\gamma_{q_h} - \gamma_{q_h+1} + 1 =: s \leq p_{h+1}$ and the first $s$ entries of the $(h+1)^{st}$ carrel $(q_h, q_{h+1} ]$ are $\gamma_{q_h}-s+1, \gamma_{q_h}-s+2, ... , \gamma_{q_h}$.  Originally a gapless core $R$-tuple was defined in Section 3 of \cite{PW} to be an upper $R$-tuple $\eta$ whose $R$-core $\Delta_R(\eta)$ is a gapless $R$-tuple.  Those original definitions were shown there to be equivalent to these definitions in Proposition 4.2.  An upper $R$-tuple $\beta$ is \emph{bounded by its platform} if $\beta \leq \Xi_R(\beta)$.  Let $UBP_R(n)$ denote the set of such upper $R$-tuples.  The example $\beta$ above is not bounded by its platform.  Clearly $UG_R(n) \subseteq UGC_R(n)$ and $UF_R(n) \subseteq UGC_R(n)$.  From the definition of $\Xi_R$, it is clear that $UF_R(n) \subseteq UBP_R(n)$ and $UI_R(n) \subseteq UBP_R(n)$.  Since $UG_R(n) \subseteq UI_R(n)$ by definition, we have $UG_R(n) \subseteq UBP_R(n)$.



We illustrate some recent definitions.  First consider an $R$-increasing upper tuple $\alpha \in UI_R(n)$:  Each carrel subsequence of $\alpha$ is a concatenation of the staircases within the carrel in which the largest entries are the critical entries for the carrel.  Now consider the definition of a gapless $R$-tuple, which begins by considering a $\gamma \in UI_R(n)$:  This definition is equivalent to requiring for all $h \in [r]$ that if $\gamma_{q_h} > \gamma_{q_{h}+1}$, then the leftmost staircase within the $(h+1)^{st}$ carrel must contain an entry $\gamma_{q_h}$.





An \emph{$R$-ceiling flag} $\xi$ is an upper flag that is a concatenation of plateaus whose rightmost pairs are the $R$-critical pairs of $\xi$.  Let $UCeil_R(n)$ denote the set of $R$-ceiling flags.  It can be seen that the restriction of the $R$-platform map from $U_R(n)$ to $UG_R(n)$ is the $R$-ceiling map $\Xi_R: UG_R(n) \rightarrow UCeil_R(n)$ defined near the end of Section 5 of \cite{PW}.  So by that Proposition 5.4(ii) this restriction of $\Xi_R$ is a bijection from $UG_R(n)$ to $UCeil_R(n)$ with inverse $\Delta_R$, and for $\gamma \in UG_R(n)$ the upper flag $\xi := \Xi_R(\gamma)$ is the unique $R$-ceiling flag that has the same flag $R$-critical list as $\gamma$.









\section{Definitions of shapes, tableaux, polynomials}

A \emph{partition} is an $n$-tuple $\lambda \in \mathbb{Z}^n$ such that $\lambda_1 \geq \ldots \geq \lambda_n \geq 0$.  The \textit{shape} of $\lambda$, also denoted $\lambda$, consists of $n$ left justified rows with $\lambda_1, \ldots, \lambda_n$ boxes.  We denote its column lengths by $\zeta_1 \geq \ldots \geq \zeta_{\lambda_1}$.  Since the columns were more important than the rows in \cite{PW}, the boxes of $\lambda$ are transpose-indexed by pairs $(j,i)$ such that $1 \leq j \leq \lambda_1$ and $1 \leq i \leq \zeta_j$.  Define $R_\lambda \subseteq [n-1]$ to be the set of distinct column lengths of $\lambda$ that are less than $n$.  Using the language of Section 2 with $R := R_\lambda$, note that for $h \in [r+1]$ one has $\lambda_i = \lambda_{i^\prime}$ for $i, i^\prime \in (q_{h-1}, q_{h} ]$.  For $h \in [r+1]$ the coordinates of the $p_h$ boxes in the $h^{th}$ \emph{cliff} form the set $\{ (\lambda_i, i) : i \in (q_{h-1}, q_{h} ] \}$.  We will replace `$R_\lambda$' by `$\lambda$' in subscripts and in prefixes when using concepts from Section 2 via $R := R_\lambda$.



A \textit{(semistandard) tableau of shape $\lambda$} is a filling of $\lambda$ with values from $[n]$ that strictly increase from north to south and weakly increase from west to east.  Let $\mathcal{T}_\lambda$ denote the set of tableaux of shape $\lambda$.  Fix $T \in \mathcal{T}_\lambda$.  For $j \in [\lambda_1]$, we denote the one column ``subtableau'' on the boxes in the $j^{th}$ column by $T_j$.  Here for $i \in [\zeta_j]$ the tableau value in the $i^{th}$ row is denoted $T_j(i)$.  To define the \emph{content $\Theta(T) := \theta$ of $T$}, for $i \in [n]$ take $\theta_i$ to be the number of values in $T$ equal to $i$.  Let $x_1,\ldots, x_n$ be indeterminants.  The \textit{monomial} $x^{\Theta(T)}$ of $T$ is $x_1^{\theta_1}\ldots x_n^{\theta_n}$, where $\theta$ is the content $\Theta(T)$.



Let $\beta$ be a $\lambda$-tuple.  We define the \emph{row bound set of tableaux} to be $\mathcal{S}_\lambda(\beta) := \{ T \in \mathcal{T}_\lambda : T_j(i) \leq \beta_i \text{ for } j \in [0, \lambda_1] \text{ and } i \in [\zeta_j] \}$.  As in Section 12 of \cite{PW}, it can be seen that $\mathcal{S}_\lambda(\beta)$ is nonempty if and only if $\beta \in U_\lambda(n)$.  Fix $\beta \in U_\lambda(n)$.  As noted in Section 12 of \cite{PW}, it can be seen that $\mathcal{S}_\lambda(\beta)$ has a unique maximal element.  In \cite{PW} we introduced the \emph{row bound sum} $s_\lambda(\beta; x) := \sum x^{\Theta(T)}$, sum over $T \in \mathcal{S}_\lambda(\beta)$.  To connect to the literature, for $\varphi \in UF_\lambda(n)$ we also give the names \emph{flag bound set} and \emph{flag Schur polynomial} to $\mathcal{S}_\lambda(\varphi)$ and the flagged Schur function $s_\lambda(\varphi;x)$ respectively.  As in \cite{PW}, for $\eta \in UGC_\lambda(n)$ it is also useful to give the names \emph{gapless core bound set} and \emph{gapless core Schur polynomial} to $\mathcal{S}_\lambda(\eta)$ and $s_\lambda(\eta;x)$ respectively.



Proposition 12.1 of \cite{PW} stated that the collection of sets $\mathcal{S}_\lambda(\varphi)$ and of $\mathcal{S}_\lambda(\eta)$ are the same.  Thus the gapless core Schur polynomials are already available as flag Schur polynomials.  However, the additional indexing $\lambda$-tuples from $UGC_\lambda(n) \backslash UF_\lambda(n)$ are useful.  The following theme from \cite{PW} will be continued:  Here we will prove that the row bound sums  $s_\lambda(\beta;x)$ for $\beta \in U_\lambda(n) \backslash UGC_\lambda(n)$ are not ``good'' for the consideration at hand.








%$\mathbf{\emph{n}}$-tuples of


\section{Lattice paths and Gessel-Viennot determinant}

We introduce $n$-tuples of weighted lattice paths to model the tableaux in the row bound tableau set $\mathcal{S}_\lambda(\beta)$.  To obtain a close visual correspondence we first flip the $x$-$y$ plane containing the paths vertically so that its first quadrant is to the lower right (southeast) of the origin on the page.  Re-use our indexing of boxes in shapes with transposed matrix coordinates to coordinatize the points in this first quadrant of $\mathbb{Z} \times \mathbb{Z}$:  Let $l \geq j \geq 0$ and $k \geq i \geq 1$.  The lattice point $(j,i)$ is $j$ units to the east of $(0,0)$ and $i$ units to the south of $(0,0)$.  For $j \geq 1$, the directed line segment from $(j-1,i)$ to $(j,i)$ is an \emph{easterly step of depth} $i$.  A \emph{(lattice) path with source $(j,i)$ and sink $(l,k)$} is a connected set incident to $(j,i)$ and $(l,k)$ that is the union of $l-j$ easterly steps and $k-i$ \emph{southerly steps}.  The notation $... \rightarrow (j,i) \downarrow (j,k) \rightarrow (l,k) \downarrow ...$ indicates that an eastbound path arrives at $(j,i)$, turns right and proceeds south to $(j,k)$, turns left and proceeds east to $(l,k)$, and then turns right and proceeds south.  An \emph{$n$-path} is an $n$-tuple $(\Lambda_1, ... , \Lambda_n) =: \Lambda$ of paths such that the component path $\Lambda_m$ has source $(n-m,m)$ for $m \in [n]$.



Let $\beta \in P(n)$.  The $n$ points $(\lambda_1+n-1, \beta_1), (\lambda_2+n-2, \beta_2),... ,(\lambda_n,\beta_n)$ are \emph{terminals} and $(\lambda, \beta)$ is a \emph{terminal pair}.  This ``strictification'' of $\lambda$ ensures that the longitudes of the terminals are distinct.  Initially our $n$-paths $(\Lambda_1, ..., \Lambda_n)$ will use the terminals $(\lambda_1+n-1, \beta_1), (\lambda_2+n-2,\beta_2), ... , (\lambda_n, \beta_n)$ in this order as sinks for their respective components.  Given such an $n$-path $\Lambda$, as in the proof of Theorem 7.16.1 of \cite{St2} we attempt to create a corresponding tableau $T \in \mathcal{S}_\lambda(\beta)$.  For each $m \in [n]$ we record the weakly increasing depths of the successive easterly steps in the path $\Lambda_m$ from left to right in the boxes of the $m^{th}$ row of the shape $\lambda$:  Here the easterly step in $\Lambda_m$ from $(n-m+j-1,p)$ to $(n-m+j,p)$ is recorded as the value $p$ in the box $(j,m)$ for $T$.  The last value in the $m^{th}$ row cannot exceed $\beta_m$.  It can be seen that these values strictly increase down each column of $\lambda$ if and only if there are no intersections among the $\Lambda_m$ for $m \in [n]$.  Let $\mathcal{LD}_\lambda(\beta)$ denote the set of such \emph{disjoint} $n$-paths.  There is at least one such disjoint $n$-path if and only if $\beta$ is upper:  To confirm this, with the correspondence above re-use the observations made near the beginning of Section 12 of \cite{PW} that addressed the questions of when the set $\mathcal{S}_\lambda(\beta)$ is empty and nonempty.  When $\beta$ is upper, it can be seen that the recording process is bijective to the set $\mathcal{S}_\lambda(\beta)$.  Since it will be seen that the cliffs of $\lambda$ play a crucial role, we now determine $R_\lambda$ and regard $\beta$ as being a $\lambda$-tuple.  Summarizing:

\begin{fact}\label{fact315.5}We have $\mathcal{LD}_\lambda(\beta) \neq \emptyset$ if and only if $\beta \in U_\lambda(n)$.  For $\beta \in U_\lambda(n)$, the recording process is a bijection from the set of disjoint $n$-paths $\mathcal{LD}_\lambda(\beta)$ to the row bound tableau set $\mathcal{S}_\lambda(\beta)$.  \end{fact}




Fix $\beta \in U_\lambda(n)$.  To obtain the determinant expression for $s_\lambda(\beta;x)$ we will need to consider more general $n$-paths and introduce weights.  Let $\Lambda$ be an $n$-path with any sinks.  Assigning a weight monomial $x^{\Theta(\Lambda)}$ to $\Lambda$ in the following fashion emulates our assignment of the weight $x^{\Theta(T)}$ to a tableau $T \in \mathcal{T}_\lambda$ when $\Lambda \in \mathcal{LD}_\lambda(\beta)$, and it also extends the weight rule to all $n$-paths.  For $m \in [n]$ assign the weight $x_i$ to each easterly step of depth $i$ in the path $\Lambda_m$, and then multiply these weights over its easterly steps.  Multiply the weights of the $n$ component paths to produce a monomial we denote $x^{\Theta(\Lambda)}$.  When the sinks of $\Lambda$ are the terminals from $(\lambda, \beta)$ in their usual order, it can be seen that the multivariate generating function $\sum_{  \Lambda \in \mathcal{LD}_\lambda(\beta)  } x^{\Theta(\Lambda)}$ is our row bound sum $s_\lambda(\beta;x)$.  Let $j \geq 0, i \geq 1, l \geq 0, k \geq 1$, and set $u := l-j$.  If we sum the weights that are assigned to just one path as it varies over all paths from $(j,i)$ to $(l,k)$, we produce the \emph{complete homogeneous symmetric function} $h_u(i,k;x)$ in the variables $x_i, x_{i+1}, ... , x_k$:  Here $h_u(i,k;x) := 0$ for $u < 0$, and otherwise $h_u(i,k;x) := \sum x_{t_1}\cdots x_{t_u}$, sum over $i \leq t_1 \leq ... \leq t_u \leq k$.



We next consider $n$-paths that use the same terminals, but in a permuted order, for their list of sinks.  Let $\pi$ be a permutation of $[n]$.  Let $\pi.(\lambda, \beta)$ denote the list of terminals $(\lambda_{\pi_1}+n-\pi_1, \beta_{\pi_1}), \\ (\lambda_{\pi_2}+n-\pi_2, \beta_{\pi_2}), ... , (\lambda_{\pi_n}+n-\pi_n, \beta_{\pi_n})$.  Let $\mathcal{LD}_\lambda(\beta;\pi)$ denote the set of disjoint $n$-paths $(\Lambda_1, ... , \Lambda_n)$ with respective sinks $\pi.(\lambda,\beta)$.  The terminal pair $(\lambda, \beta)$ is \emph{nonpermutable} \cite{GV} if $\mathcal{LD}_\lambda(\beta;\pi) = \emptyset$ when $\pi \neq (1,2,...,n)$.



Here is our non-skew version of Theorem 2.7.1 of \cite{St1}; as in Theorem 7.16.1 of \cite{St2} we have replaced the disjoint $n$-paths with the corresponding tableaux:

\begin{prop}\label{prop315.6}Let $\beta \in U_\lambda(n)$.  If the terminal pair $(\lambda, \beta)$ is nonpermutable, then the row bound sum $s_\lambda(\beta;x)$ is given by the $n \times n$ determinant $| h_{\lambda_j-j+i}(i,\beta_j;x) |$.  \end{prop}

\noindent To produce this expression with Theorem 2.7.1 of \cite{St1}, use the remark above that expressed $s_\lambda(\beta;x)$ as the $\mathcal{LD}_\lambda(\beta)$ generating function and note that $(\lambda_j+n-j)-(n-i) = \lambda_j-j+i$.  Theorem 2.7.1 was proved with a signed involution pairing cancellation argument, as in \cite{GV}.





\section{Main results}

Our main result combines the forthcoming Propositions \ref{prop434.7} and \ref{prop581.6}:

\begin{thm}\label{theorem777.1}Let $\lambda$ be a partition and let $\beta$ be an upper $\lambda$-tuple.  The terminal pair $(\lambda, \beta)$ is nonpermutable if and only if $\beta$ is a gapless core $\lambda$-tuple that is bounded by its platform. \end{thm}

\noindent So under these circumstances we can employ the Gessel-Viennot method, as noted in Proposition \ref{prop315.6}:

\begin{cor}\label{cor777.2}Let $\beta \in U_\lambda(n)$.  If $\beta \in UGC_\lambda \cap UBP_\lambda(n)$ then $s_\lambda(\beta;x) = | h_{\lambda_j-j+i}(i,\beta_j;x) |$. \end{cor}

\noindent Although this determinant is not guaranteed to ``work'' when $\beta \in UGC_\lambda(n) \backslash UBP_\lambda(n)$, given our quotes in Section 8 of facts from \cite{PW} the polynomial $s_\lambda(\beta;x)$ for such a $\beta$ can be computed with the determinant using $\delta := \Delta_\lambda(\beta)$ instead of $\beta$ itself.  An example of the failure of the determinant for such a $\beta$ is given before Lemma \ref{lemma434.5}.

\begin{cor}\label{cor777.2.5}Let $\beta \in U_\lambda(n)$.  Set $\delta := \Delta_\lambda(\beta)$.  If $\beta \in UGC_\lambda(n)$ then $s_\lambda(\beta;x) = | h_{\lambda_j-j+i}(i,\delta_j;x) |$. \end{cor}


At the end of Section 14 of \cite{PW} we promised to give a determinant expression for certain $GL(n)$ Demazure characters (key polynomials) here.  General Demazure characters $d_\lambda(\pi;x)$ for $GL(n)$ can be recursively defined with divided differences as noted in Section 1 of \cite{PW} or defined as a sum of $x^{\Theta(T)}$ over a certain set of semistandard tableaux as in Section 14 of \cite{PW}.  Given that $UG_\lambda(n) \subseteq UBP_\lambda(n)$, the next statement is implied by Corollary \ref{cor777.2} and Theorem 14.2(ii) of \cite{PW}.  For this result that theorem gives $d_\lambda(\pi;x) = s_\lambda(\gamma;x)$.  Consult Section 3 of \cite{PW} for the definitions of the $\lambda$-permutations and the map $\Psi_\lambda$.

\begin{cor}\label{cor777.3}Let $\lambda$ be a partition and let $\pi$ be a $\lambda$-permutation.  If $\pi$ is $\lambda$-312-avoiding, then $\Psi_\lambda(\pi) =: \gamma$ is a gapless $\lambda$-tuple and $d_\lambda(\pi;x) = | h_{\lambda_j-j+i}(i,\gamma_j;x) |$. \end{cor}

\noindent A ``less efficient'' (in the sense of our Section 8) version of this expression appeared in the proof of Corollary 14.6 of \cite{PS} when Postnikov and Stanley applied their skew flagged Schur function determinant identity Equation 13.1 to their $\text{ch}_{\lambda,w}$.











\section{Necessary condition for nonpermutability}

Let $\beta \in U_\lambda(n)$.  We prepare for two proofs by constructing an $n$-path $\Lambda$ for each $d \in [q_r]$.  To see that each $\Lambda \in \mathcal{LD}_\lambda(\beta)$, we also describe its corresponding (clearly semistandard) tableau $T$.  Launching a running example, take $n=16$ and $\lambda = (7^3; 5^8; 3^2; 1^2; 0^1)$ and $\beta = (5,5,8;5,12,13,9,11,11,15,15;$ $16,16;14,16;16)$.  Set $\delta := \Delta_\lambda(\beta)$.  Here $\delta = (4,5,8;5,7,8,9,10,11,14,15;15,16;14,16;16)$.  Let $d \in [q_r]$.  For example, take $ d = 9$.  For $i \in (0,d-1]$ set $T_j(i) := i$ for $j \in (0, \lambda_i]$.  The corresponding paths $\Lambda_i$ are $(n-i,i) \rightarrow (\lambda_i + n-i,i) \downarrow (\lambda_i+n-i, \delta_i) \downarrow (\lambda_i+n-i, \beta_i)$.  Six of these eight paths are shown with dots in Figure 6.1.  Let $i \in (d-1, q_r]$.  Let $h \in [r]$ be such that $i \in (q_{h-1}, q_h]$.  For $j \in (0, \lambda_{q_{h+1}}]$ set $T_j(i):=i$.  For $j \in (\lambda_{q_{h+1}}, \lambda_{q_h}]$ set $T_j(i) := \delta_i$.  The corresponding paths $\Lambda_i$ are $(n-i,i) \rightarrow (\lambda_{q_{h+1}}+n-i,i) \downarrow (\lambda_{q_{h+1}}+n-i, \delta_i) \rightarrow (\lambda_i + n - i, \delta_i) \downarrow (\lambda_i+n-i, \beta_i)$.  For $i \in (q_r, n]$ set $T_j(i) := \delta_i$ $(=i)$ for $j \in (0, \lambda_i]$.  The corresponding paths $\Lambda_i$ are $(n-i,i) \rightarrow (\lambda_i+n-i, \delta_i) \downarrow (\lambda_i+n-i, \beta_i)$.  The dots indicate the depths $\delta_i$ on the ending longitudes of the paths.








For a determinant example pertinent to the following lemma, take $n := 3, \lambda := (1,1,0)$, and $\beta := (3,2,3)$.  Note that $\beta \in UGC_\lambda(n) \backslash UBP_\lambda(n)$, and so this lemma will imply that $(\lambda, \beta)$ is not nonpermutable.  Here $s_\lambda(\beta;x,y,z) = xy$, but the determinant of Proposition \ref{prop315.6} evaluates to $xy - z^2$.

\begin{lem}\label{lemma434.5}If $\beta \notin UBP_\lambda(n)$, then $(\lambda, \beta)$ fails to be nonpermutable.  \end{lem}

\begin{proof}Set $\Delta_\lambda(\beta) =: \delta \in UI_\lambda(n)$ and $\xi = \Xi_\lambda(\beta)$.  In the example we have $\xi = (5,5,8;5,11,$ $11,11,11,11,15,15;16,16;14,16;16)$. Since $\beta$ is a $\lambda$-tuple and $\xi_i = n$ for $i \in (q_r,n]$, the failure of 

\centerline{\includegraphics{Figure6point1Phase3}}

\vspace{0.5pc}\centerline{Figure 6.1.  Rewiring four component paths produces a nonpermutability violation.}



\vspace{1pc}\noindent boundedness for $\beta$ cannot occur in this last carrel.  Let $h \in [r]$ be such that there exists $t \in (q_{h-1}, q_h]$ such that $\beta_t > \xi_t$, and then let $c \in (q_{h-1}, q_h]$ be maximal such that $\beta_c > \xi_c$.  So $c$ is not a critical index, since $\beta_c \neq \xi_c$.  Let $d$ be the leftmost critical index in $(q_{h-1},q_h]$ such that $d > c$.  Here we have $h = 2, c = 6$, and $d = 9$.  Here $\beta_d = \delta_d = \xi_d = \xi_c < \beta_c$ implies $\delta_d+1 \leq \beta_c$.  Since $d \leq q_r$ we have $\lambda_d \geq 1$, which implies $\lambda_d + n - d - 1 \geq 0$.  Now refer to the $n$-path $\Lambda$ constructed above for this $d \in [q_r]$.  We rewire the last part of its $\Lambda_d$ to produce a new path $\Lambda_d^\prime$ as follows:  Rather than finishing with $... \rightarrow (\lambda_d + n-d, \delta_d) = (\lambda_d+n-d, \beta_d)$, the new path $\Lambda_d^\prime$ finishes with $...$ $(\lambda_d + n-d-1, \delta_d) \downarrow (\lambda_d+n-d-1, \delta_d+1) \rightarrow (\lambda_d + n-c, \delta_d+1) \downarrow (\lambda_c+n-c, \beta_c)$.  Four rewirings are shown with solid paths.  Here $\Lambda_d^\prime$ reaches $(\lambda_d+n-d-1, \delta_d)$, goes one unit to the south, then turns left onto the latitude $\delta_d + 1$ and goes $d-c+1$ units to the east, and then turns right to go straight south until it ends at $(\lambda_c+n-c, \beta_c)$.  This new southerly edge $(\lambda_d+n-d-1, \delta_d) \downarrow (\lambda_d+n-d-1, \delta_d+1)$ is not in use by $\Lambda_{d+1}$ (or a later path):  If $d = q_h$, then the longitude at $(\lambda_d+n-d)-1$ is not used by any component of $\Lambda$ since $\lambda_d > \lambda_{d+1}$ here implies that this longitude is strictly to the east of the longitude $\lambda_{d+1}+n-d-1$ on which $\Lambda_{d+1}$ finishes.  If $d < q_h$, note that $\delta_d+1 < \delta_{d+1}$ because $d$ is a critical index.  So here the southernmost point reached by $\Lambda_d^\prime$ on its new briefly used longitude at $\lambda_d+n-d-1$ is strictly to the north of the northernmost point on this longitude used by $\Lambda_{d+1}$, which descended to the depth $\delta_{d+1}$ on the longitude $\lambda_{q_{h+1}}+n-d-1$ to the west.  Either way, for $m = d-1, d-2, ... , c$, next successively rewire the finishes of $\Lambda_{d-1}, \Lambda_{d-2}, ... , \Lambda_c$ to respectively produce finishes for the paths $\Lambda_{d-1}^\prime, \Lambda_{d-2}^\prime, ... , \Lambda_c^\prime$ as follows:  Rather than travelling $(n-m, m) \rightarrow (\lambda_m+n-m,m) \downarrow (\lambda_m+n-m, \delta_m) \downarrow (\lambda_m+n-m, \beta_m)$, the new path $\Lambda_m^\prime$ travels $(n-m, m) \rightarrow (\lambda_m+n-m-1,m) \downarrow (\lambda_m+n-m-1, \delta_{m+1}) \downarrow (\lambda_m+n-m-1, \beta_{m+1})$.  Here $\Lambda_m^\prime$ is finishing by turning right one step early, using one (or more) new southerly step(s), and then using the final (possibly empty) ``southerly stilt'' that $\Lambda_{m+1}$ had been using to finish.  It can be seen that the ``further'' new southerly steps that could be used by $\Lambda_{d-1}'$ are not used by $\Lambda_d'$.  No intersections among these $d-c$ paths occur since the right turns that are each being executed one easterly step early are being coordinated along a staircase where $\lambda_m = \lambda_{q_h}$.  Given the choices of $c$ and $d$, for $i \in (c,d]$ we have $\beta_i \leq \xi_i = \xi_d = \delta_d$.  So $\beta_i < \delta_d +1$ for $i \in (c,d]$.  Hence $\Lambda_m^\prime$ does not intersect $\Lambda_d^\prime$ for $m \in [c, d-1]$.  When $m \notin [c,d]$ set $\Lambda_m^\prime := \Lambda_m$.  It can be seen that none of the rewired paths intersect any of these original paths.  We have constructed a disjoint $n$-path $\Lambda^\prime := (\Lambda_1^\prime, ... , \Lambda_n^\prime)$ whose respective sinks form a nontrivial permutation $\pi$ of the original ordered terminals.  Therefore $\mathcal{LD}_\lambda(\beta;\pi) \neq \emptyset$.  \end{proof}



For an example pertinent to the following lemma, take $n := 3, \lambda := (2,1,0)$, and $\beta := (3,2,3)$.  Note that $\beta \in UBP_\lambda(n) \backslash UGC_\lambda(n)$, and so this lemma will imply that $(\lambda, \beta)$ is not nonpermutable.  Here $s_\lambda(\beta;x,y,z) = x^2y + xy^2 + xyz$, but the determinant of Proposition \ref{prop315.6} evaluates to $x^2y + xy^2 + xyz - z^3$.

\begin{lem}\label{lemma434.6}If $\beta \notin UGC_\lambda(n)$, then $(\lambda, \beta)$ fails to be nonpermutable.  \end{lem}

\begin{proof}If $\beta \notin UBP_\lambda(n)$ apply Lemma \ref{lemma434.5};  otherwise $\beta \in UBP_\lambda(n)$.  Set $\Delta_\lambda(\beta) =: \delta \in UI_\lambda(n)$ and $\xi := \Xi_\lambda(\beta)$.  Having $\beta$ failing to be a gapless core $\lambda$-tuple is equivalent to having $\delta$ failing to be a gapless $\lambda$-tuple.  The only critical entry in the last carrel $(q_r, n]$ is $n$.  So there cannot be a failure of $\lambda$-gapless based upon having $\delta_{q_r} > n$.  Let $h \in (1, r]$ be such that $\delta$ fails to be $\lambda$-gapless based upon having $\delta_{q_{h-1}} > \delta_d$, where $d$ is the leftmost critical index in the $h^{th}$ carrel $(q_{h-1},q_h]$.  Set $c := q_{h-1}$.  In each of the two cases below we refer to the $n$-path $\Lambda$ for this $d$ constructed above.  Note that $\delta_d + 1 \leq \delta_c$.  Since $d \leq q_r$ in each case we have $\lambda_d \geq 1$, which implies $\lambda_d+n-d-1 \geq 0$.  These facts will allow us to rewire the path $\Lambda_d$ to produce the path $\Lambda_d^\prime$ in nearly the same fashion as in the previous proof.  The only difference is that the new path $\Lambda_d^\prime$ now has to make $\lambda_{q_{h-1}} - \lambda_{q_h}$ additional easterly steps just before reaching its finishing longitude of $\lambda_{q_{h-1}}+n- q_{h-1}$.  If $d = q_h$, then the reasoning used in the `$d = q_h$' case in the preceding proof to see that the southerly edge on the longitude $(\lambda_d+n-d)-1$ from depth $\delta_d$ to depth $\delta_d +1$ is not in use by $\Lambda_{d+1}$ can be re-used here.  Here $d$ is the only critical index for the carrel $(q_{h-1}, q_{h}]$.  If $d < q_h$, the reasoning used in the `$d < q_h$' case in the preceding proof to see that the early ``jog'' to the right is acceptable can be re-used here.  Here $d$ is the smallest critical index greater than $q_{h-1}$.  Either way, for $m = d-1, d-2, ... , c+1$, next successively rewire $\Lambda_{d-1}, \Lambda_{d-2}, ... , \Lambda_{c+1}$ to respectively produce paths $\Lambda_{d-1}^\prime, \Lambda_{d-2}^\prime, ... , \Lambda_{c+1}^\prime$ as in the previous proof.  Then rewire the path $\Lambda_c$ to produce the path $\Lambda_c^\prime$ in nearly the same fashion as in the previous proof.  The only difference is that the new path $\Lambda_c^\prime$ now makes $\lambda_{q_{h-1}}-\lambda_{q_h}$ fewer easterly steps just before reaching its finishing longitude of $\lambda_{q_h}+n-q_{h-1}-1$.  The observation in the previous proof concerning the coordination of the right turns among the shifted $d-c$ modified paths needs a tiny modification to account for this.  In each case the fact that $d$ is the smallest critical index larger than $c$ implies $\xi_i = \xi_d = \delta_d$ for $i \in (c,d]$.  Since $\beta \in UBP_\lambda(n)$, we have $\beta_i \leq \xi_i = \xi_d = \delta_d < \delta_d+1$ for $i \in (c,d]$.  The rest of this proof is the same as the end of the previous proof.  \end{proof}



Combine the contrapositives of these two lemmas:

\begin{prop}\label{prop434.7}Let $\beta \in U_\lambda(n)$.  If $(\lambda, \beta)$ is nonpermutable, then $\beta \in UGC_\lambda(n) \cap UBP_\lambda(n)$.  \end{prop}









\section{Sufficient condition for nonpermutability}


To prove the converse of Proposition \ref{prop434.7}, we will need:

\begin{lem}\label{lemma581.5}Let $\beta \in UBP_\lambda(n)$.  Set $\delta := \Delta_\lambda(\beta)$.  Let $\pi$ be a permutation of $[n]$.  Let $\Lambda \in \mathcal{LD}_\lambda(\beta;\pi)$.  For each $m \in [n]$, the component $\Lambda_m$ of $\Lambda$ must end with $(\lambda_{\pi_m}+n-\pi_m, \delta_{\pi_m}) \downarrow (\lambda_{\pi_m}+n-\pi_m, \beta_{\pi_m})$.  \end{lem}

\begin{proof}To avoid forming the inverse of $\pi$ and using double subscripts, we sidestep $\pi$ by refering to the original indices for the terminals.  Let $x$ be a critical index for $\beta$.  Let $x'$ be the largest critical index that is less than $x$;  if $x$ is the leftmost critical index then take $x' := 0$.  Here $\lambda_i = \lambda_x$ for $i \in (x', x]$.  For such $i$, let $M_i$ denote the component of $\Lambda$ that sinks at $(\lambda_x + n - i, \beta_i)$.  The claim is true for $M_x$ since $\delta_x = \beta_x$.  Let $i$ decrement from $x$ to $x' + 1$ and assume the claim is true for $i < i' \leq x$.  So each $M_{i'}$ ends with $(\lambda_x + n - i', \delta_{i'}) \downarrow (\lambda_x + n - i', \beta_{i'})$.  Set $\xi := \Xi_\lambda(\delta)$.  Note that $\xi_i = \xi_x = \delta_x = \beta_x$.  If $\beta_i = \delta_i$ there is nothing to show.  Otherwise $\delta_i = \delta_{i+1}-1$ and $\delta_i \leq \beta_i \leq \xi_i$ imply that $\delta_{i+1} \leq \beta_i \leq \xi_i$.  By the induction we see that $(\lambda_x + n - i', \delta_{i'})$ is unavailable to $M_{i}$ for $i < i' \leq x$.  So this path $M_i$ must pass through $(\lambda_x + n - i, \delta_i)$.  Then it must finish with $(\lambda_x + n - i, \delta_i) \downarrow (\lambda_x + n - i, \beta_i)$.  \end{proof}



Stanley remarked in Theorem 2.7.1 of \cite{St1} that $(\lambda, \beta)$ is nonpermutable when $\beta$ is a flag.  Since $UF_\lambda(n) \subseteq UGC_\lambda(n) \cap UBP_\lambda(n)$, the following proposition extends that remark.  His remark can be justified with either of the arguments that we describe within Case (i) of this proof, but referring to $\beta$ rather than to $\delta$.

\begin{prop}\label{prop581.6}Let $\beta \in U_\lambda(n)$.  If $\beta \in UGC_\lambda(n) \cap UBP_\lambda(n)$, then $(\lambda, \beta)$ is nonpermutable.  \end{prop}

\begin{proof}Let $\pi$ be a permutation of $[n]$ such that $\pi \neq (1,2,...,n)$.  For the sake of contradiction suppose $\mathcal{LD}_\lambda(\beta;\pi) \neq \emptyset$.  Find a descent in $\pi^{-1}$ and let $1 \leq i < k \leq n$ be such that $\pi_i = \pi_k+1$.  Set $m := \pi_k$.  Take $\Lambda \in \mathcal{LD}_\lambda(\beta;\pi)$.  Set $\delta := \Delta_\lambda(\beta) \in UG_\lambda(n)$.  By the lemma, without loss of generality we may revamp $\Lambda$ by replacing (with respect to their original indexing) the sequence $\beta$ of depths of its terminals with the sequence of shallower depths $\delta$.  This truncates its original paths by deleting their final stilts.  We consider the components $\Lambda_i$ and $\Lambda_k$ of $\Lambda$.  Here $\Lambda_i$ arises at $(n-i,i)$ and sinks at $(\lambda_{m+1}+n-m-1, \delta_{m+1})$.  Later $\Lambda_k$ arises at $(n-k,k)$ and sinks at $(\lambda_m+n-m, \delta_m)$.  Comparing the starting and finishing longitudes for $\Lambda_k$ to those for $\Lambda_i$, we have $n-k < n-i$ and $\lambda_m + n - m > \lambda_{m+1} +n-m-1$.  So every longitude that is visited by $\Lambda_i$ is later visited by the longer $\Lambda_k$.  Set $v := \lambda_{m+1}+n-m-1$;  the earlier path $\Lambda_i$ finishes on the longitude at $v$.  Let's say that the later path $\Lambda_k$ first reaches the longitude at $v$ on the latitude at $z$, for some $z \geq 1$. \\ (i)  First suppose that $z \leq \delta_{m+1}$, which is the finishing depth of $\Lambda_i$ on the longitude at $v$.  It is topologically evident that the path $\Lambda_k$ must intersect the path $\Lambda_i$;  this contradicts $\Lambda \in \mathcal{LD}_\lambda(\delta;\pi)$.  (For an explicit discrete proof, consider the minimum and maximum depths used on each of the $\lambda_{m+1}-m+i$ longitudes visited by both $\Lambda_i$ and $\Lambda_k$.  Inequalities and equalities among these $4(\lambda_{m+1}-m+i)$ depths can be used to find a longitude on which $\Lambda_i$ and $\Lambda_k$ intersect.)  \\ (ii)  Otherwise we have $z > \delta_{m+1}$.  See Figure 7.1.  Since $z$ cannot exceed the finishing depth $\delta_m$ for $\Lambda_k$, we have $z \leq \delta_m$.  Hence $\delta_m > \delta_{m+1}$.  But $\delta \in UG_\lambda(n)$ is $\lambda$-increasing.  This forces $m = q_h$ for some $h \in [r]$.  Set $s := \delta_m - \delta_{m+1}+1$.  Since $\delta$ is $\lambda$-gapless we have $s \leq p_{h+1}$ and $\delta_{m+1} = \delta_m - s +1$, $\delta_{m+2} = \delta_m - s + 2, ... , \delta_{m+s} = \delta_m$.  Starting at the sink $(v, \delta_{m+1})$ of $\Lambda_i$ and moving exactly to the 

\vspace{.5pc}\centerline{\includegraphics{Figure7point1Phase5}}

\begin{adjustwidth*}{1.3125in}{1.3125in}

\vspace{.5pc}\noindent Figure 7.1.  Paths $\Lambda_i$ and $\Lambda_k$ successively sink at terminals \\ $(\lambda_{m+1}+n-m-1, \delta_{m+1})$ and $(\lambda_m+n-m, \delta_m)$.

\end{adjustwidth*}



\noindent southwest, we note that the $s$ points $(v, \delta_{m+1}), (v-1, \delta_{m+1}+1), ... , (v-s+1, \delta_m)$ forming a staircase are terminals that are serving as sinks for some paths other than $\Lambda_k$.  Since $\Lambda_i$ and $\Lambda_k$ are paths, we have $i \leq \delta_{m+1}$ and $k \leq \delta_m$.  The source of $\Lambda_k$ is exactly to the southwest of the source of $\Lambda_i$ by $k-i$ diagonal steps.  Since the source of $\Lambda_i$ is weakly to the west of the longitude at $v$, if the source of $\Lambda_k$ is on one of the latitudes appearing in the staircase it must be weakly to the west of the point of the staircase on that latitude.  This implies that the source of $\Lambda_k$ is not on the same side of this staircase as $(v,z)$.  This is also clear if the source of $\Lambda_k$ is on a shallower latitude.  Since the path $\Lambda_k$ originates on the longitude at $n-k < v$ and reaches $(v,z)$ with $z \in (\delta_{m+1}, \delta_m]$, it must intersect this staircase.  This contradicts $\Lambda \in \mathcal{LD}_\lambda(\delta;\pi)$.  Hence $\mathcal{LD}_\lambda(\delta;\pi) \neq \emptyset$ is impossible when $\pi \neq (1,2,...,n)$. \end{proof}







\section{Equivalence and efficiency}



We group the valid $\lambda$-tuple inputs for computing row bound sums using the Gessel-Viennot method into equivalence classes, and identify the most efficient $\lambda$-tuple within each class.



When $\lambda$ has distinct parts, the row ending values for the unique maximal element of $\mathcal{S}_\lambda(\beta)$ are the entries of $\beta$.  Hence the sets $\mathcal{S}_\lambda(\beta)$ for $\beta \in U_\lambda(n)$ are distinct in this case.  For general $\lambda$, as in Section 12 of \cite{PW}, for $\beta, \beta' \in U_\lambda(n)$ define $\beta \approx_\lambda \beta'$ when $\mathcal{S}_\lambda(\beta) = \mathcal{S}_\lambda(\beta')$.  Proposition 12.3(i) of \cite{PW} stated that the sets $\mathcal{S}_\lambda(\beta)$ could be precisely labelled by requiring $\beta \in UI_\lambda(n)$, and that these $\lambda$-increasing upper tuples are the minimal elements of the equivalence classes in $U_\lambda(n)$ for $\approx_\lambda$.  Proposition 12.2 said that the results in Sections 4 and 5 of \cite{PW} for $\sim_R$ could be used for $\approx_\lambda$ by taking $R := R_\lambda$.  Lemma 5.1(i) there said that $\beta, \beta' \in U_\lambda(n)$ are equivalent exactly when $\Delta_\lambda(\beta) = \Delta_\lambda(\beta')$ or when they have the same critical list.  Since the $\beta \in U_\lambda(n) \backslash UGC_\lambda(n)$ are not valid $n$-tuple Gessel-Viennot inputs, the next statement considers only $UGC_\lambda(n)$ and $UF_\lambda(n)$.  Its two parts follow from Lemma 5.1(i), Proposition 4.2, and Proposition 5.2(ii)(iii) of \cite{PW}.

\begin{fact}\label{fact826.3}When $\approx_\lambda$ is restricted to $UGC_\lambda(n)$ and to $UF_\lambda(n)$, in each case the equivalence classes are the subsets consisting of $\lambda$-tuples that share a flag critical list.  More specifically:

\noindent (i)  In $UGC_\lambda(n)$ these subsets are the nonempty intervals in $U_\lambda(n)$ of the form $[\gamma, \kappa]$, where \\ $\gamma \in UG_\lambda(n)$ and $\kappa$ is a ``$\lambda$-canopy tuple''.

\noindent(ii)  In $UF_\lambda(n)$ these subsets are the nonempty intervals in $UF_\lambda(n)$ of the form $[\tau, \xi]$, where $\tau$ is a ``$\lambda$-floor flag'' and $\xi \in UCeil_\lambda(n)$.  \end{fact}



To describe the equivalence classes of valid $\lambda$-tuple inputs as intervals, we ``borrow'' the minimum element of Part (i) above and the maximum element of Part (ii) above:

\begin{prop}\label{prop826.6}The equivalence classes for the restriction of $\approx_\lambda$ to $UGC_\lambda(n) \cap UBP_\lambda(n)$ are the subsets of $UGC_\lambda(n) \cap UBP_\lambda(n)$ consisting of $\lambda$-tuples that share a flag critical list.  These subsets are the nonempty intervals in $U_\lambda(n)$ of the form $[\gamma, \xi]$, where $\gamma \in UG_\lambda(n)$ and $\xi \in UCeil_\lambda(n)$.  The equivalence class for a particular $\eta \in UGC_\lambda(n) \cap UBP_\lambda(n)$ has $\gamma = \Delta_\lambda(\eta)$ and $\xi = \Xi_\lambda(\gamma)$. \end{prop}

\noindent Since it was noted that $UF_\lambda(n) \subseteq UBP_\lambda(n)$ in Section 2, there is no need here to consider how the equivalence classes for $\approx_\lambda$ restrict to $UF_\lambda(n) \cap UBP_\lambda(n) = UF_\lambda(n)$.

\begin{proof}Two upper $\lambda$-tuples are equivalent exactly when they share a critical list.  And by Proposition 4.2(iii) of \cite{PW} every gapless core $\lambda$-tuple has a flag critical list.  Let $\eta \in UGC_\lambda(n) \cap UBP_\lambda(n)$, and denote its equivalence class in this set by $\langle \eta \rangle$.  By Proposition 5.2(ii)(i) of \cite{PW} and Fact \ref{fact826.3}(i), the minimum element of its equivalence class in $UGC_\lambda(n)$ is the gapless $\lambda$-tuple $\gamma := \Delta_\lambda(\eta)$.  In Section 2 it was noted that $UG_\lambda(n) \subseteq UBP_\lambda(n)$.  So $\gamma \in UGC_\lambda(n) \cap UBP_\lambda(n)$, and it must be the minimum element of $\langle \eta \rangle$.  Set $\xi := \Xi_\lambda(\gamma)$;  in Section 2 it was noted that $\xi$ has the same flag critical list as is shared by $\eta$ and $\gamma$.  Let $\eta' \in \langle \eta \rangle$.  Since it has the same critical list as $\gamma$, by the definition of $\Xi_\lambda$ we have $\Xi_\lambda(\eta') = \xi$.  By the definition of $UBP_\lambda(n)$ we have $\eta' \leq \xi$.  Hence $\xi$ is the maximum element of $\langle \eta \rangle$ and $\eta' \in [\gamma, \xi]$.  Suppose $\eta'' \in [\gamma, \xi]$.  By Lemma 5.1(i) and Proposition 5.2(i) of \cite{PW}, the critical list of $\eta''$ is the flag critical list shared by $\gamma$ and $\xi$.  So $\eta'' \in UGC_\lambda(n)$.  And $\eta'' \leq \xi = \Xi_\lambda(\eta'')$ implies $\eta'' \in UBP_\lambda(n)$.  Hence $\eta'' \in \langle \eta \rangle$. \end{proof}

\noindent So to compute $s_\lambda(\eta;x)$ for a given $\eta \in UGC_\lambda(n)$ we may apply the Gessel-Viennot method to any $\eta' \in [\gamma, \xi]$, where $\gamma$ and $\xi$ are respectively the unique gapless $\lambda$-tuple and the unique $\lambda$-ceiling flag that have the same flag critical list as $\eta$.  If one does not care about efficiency and wishes to use an upper flag, then at least the $\lambda$-ceiling flag $\xi$ will be available.  In his Theorem 2.7.1 \cite{St1}, Stanley noted that flags were valid inputs for the Gessel-Viennot method.  Via Proposition \ref{prop826.6}, our Theorem \ref{theorem777.1} implies that the Gessel-Viennot method cannot be used to compute a row bound sum $s_\lambda(\beta;x)$ for any upper $\lambda$-tuple $\beta$ that is not equivalent to an upper flag.  So Corollary \ref{cor777.2} does not provide determinant expressions for any new row bound sum polynomials.





We say $\eta \in UGC_\lambda(n) \cap UBP_\lambda(n)$ attains \emph{maximum efficiency} if $|h_{\lambda_j-j+i}(i,\eta_j;x)|$ has fewer total monomials among its entries than does the Gessel-Viennot determinant for any other $\eta' \in UGC_\lambda(n) \cap UBP_\lambda(n)$ that produces $s_\lambda(\eta;x)$.  Fix one $\eta \in UGC_\lambda(n) \cap UBP_\lambda(n)$ and set $\Delta_\lambda(\eta) =: \gamma \in UG_\lambda(n)$.  By Proposition 5.2(ii) of \cite{PW} this is the minimum element of $U_\lambda(n)$ that is equivalent to $\eta$.  Knowing $\gamma \leq \eta$ leads to:

\begin{prop}\label{prop826.9}  Let $\eta \in UGC_\lambda(n)$.  The gapless $\lambda$-tuple $\Delta_\lambda(\eta)$ attains maximum efficiency. \end{prop}

\begin{proof}To complete the proof, note that $\Delta_\lambda(\eta) \in UG_\lambda(n) \subseteq UBP_\lambda(n)$.  So Proposition \ref{prop315.6} can be applied.  Corollary 14.4(i) of \cite{PW} rules out an ``accidental'' polynomial equality between $s_\lambda(\eta;x)$ and any $s_\lambda(\beta;x)$ for which $\beta$ is not equivalent to $\eta$.  The $(i,j)$ entry of $|h_{\lambda_j-j+i}(i,\eta_j;x)|$ has ${\lambda_j - j + \eta_j \choose \lambda_j -j +i}$ monomials.  The sentences before the statement complete this proof. \end{proof}



So the $\gamma \in UG_\lambda(n)$ are the $\lambda$-tuples in $UGC_\lambda(n) \cap UBP_\lambda(n)$ that attain maximum efficiency.  If $\beta$ is replaced by $\gamma$, for each $j \in [n]$ the number of terms in the $(i,j)$ entry of the determinant will be reduced by a factor of  $[(\lambda_j-j+\gamma_j)_{(\lambda_j-j+i)}] \slash [(\lambda_j-j+\beta_j)_{(\lambda_j-j+i)}]$;  this is a ratio of falling factorials.  We have not been able to obtain this conversion with naive row and column operations.  In the $\beta \in UGC_\lambda(n) \backslash UBP_\lambda(n)$ example given before Lemma \ref{lemma434.5}, the ``attempted'' incorrect determinant expression for $s_\lambda(\beta;x)$ that uses $\beta$ cannot be converted with row and column operations to the correct determinant expression for $s_\lambda(\beta;x)$ that uses $\gamma := \Delta_\lambda(\beta)$.  So any row and column conversion that is proposed here must refer to the assumption $\beta \in UBP_\lambda(n)$.  If $\lambda_n > 0$, one can also factor out $(x_1x_2 \cdots x_n)^{\lambda_n}$ and work with $\lambda^\prime := (\lambda_1 - \lambda_n, \lambda_2 - \lambda_n, ... , 0)$.  Going further, when there are only $p := \zeta_1 < n$ nonempty rows in the shape $\lambda$, the determinant is equal to its upper left $p \times p$ minor because the last $n-p$ terminals coincide with the respective sources:  There are no paths from the first $p$ sources to these terminals, and the only path from one of the last $n-p$ sources to one of these last $n-p$ terminals is the null path at each source.



What does the equivalence class $[\gamma, \xi]$ for $\approx_\lambda$ look like in the path model?  Fix $\gamma \in UG_\lambda(n)$ and $h \in [r+1]$.  Since $UG_\lambda(n) \subseteq UI_\lambda(n)$, the graph of $\gamma$ above the portion $(q_{h-1}, q_h]$ of the $x$-axis can be decomposed into ``staircases'' whose rightmost indices are the critical indices.  When $\Xi_\lambda$ is applied to $\gamma$ to produce $\xi$, these staircases are converted to ``plateaus'' at the heights of the critical entries for $\gamma$ in this carrel.  Let $\eta \in [\gamma, \xi]$.  The graph of $\eta$ over this carrel lies between these graph portions for $\gamma$ and $\xi$.  To view the portions of these three gapless core $\lambda$-tuples as subsequences of the corresponding overall sequences of terminals, rotate this picture by $180^\circ$.  The partition $\lambda$ is constant on each of its carrels.  Lemma \ref{lemma581.5} said that the the $q_h - q_{h-1}$ lattice paths that arrive at these terminals for $\eta$ within a non-intersecting $n$-tuple of paths must pass through ``staircases'' of terminals specified by this portion of $\gamma$, and that the ending segments of these paths must then drop down in ``stilts'' to arrive at their terminals.  As the lengths of each of these stilts is varied from $\gamma_i$ to $\xi_i$ for $i \in (q_{h-1}, q_h]$, the weight of the $n$-tuple of paths is unaffected since no horizontal steps are present.



In \cite{PW} we defined the parabolic Catalan number $C_n^\lambda$ to be the number of ``$\lambda$-312-avoiding permutations''.  There in Theorem 18.1(ii) we noted that this is also the number of gapless $\lambda$-tuples.  Given this, the following result is a consequence of the two propositions in this section.  It was previewed as Part (xi) of Theorem 18.1 of that paper:

\begin{cor}\label{cor826.12}The number of valid upper $\lambda$-tuple inputs to the Gessel-Viennot determinant expression for flagged Schur polynomials on the shape $\lambda$ that attain maximum efficiency is $C_n^\lambda$.  \end{cor}

\noindent For a sequence of examples, let $m \geq 1$.  Suppose $\lambda$ is a partition whose shape's set of column lengths that are less than $2m$ is $R_\lambda = \{ 2, 4, ... , 2m-2 \}$.  Then the number of maximum efficiency inputs here is given by the member of the sequences A220097 of the OEIS \cite{Slo} that is indexed by $m$.











\end{spacing}

\end{document}








