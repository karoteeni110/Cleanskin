% !TeX root = main.tex

\section{Cubical acylindricity}

%For $k\in\N$ and a class of subgroups $\mathcal{C}$ which is closed under conjugation and subgroups, we say that the $G$ action on the CAT(0) cube complex $\CC{X}$ is \emph{$(k,\mathcal{C})$-acylindrical} if the common stabilizer of any $k$-chain of hyperplanes is in $\mathcal {C}$.\todo{remove}
% This notion should not be confused with acylindrical actions on metric spaces, even though there is a resemblance between the two when considering actions on trees.




%Now that we have proved Theorem \ref{main result} we turn to the proof of Theorem \ref{acylindrical accessibility}.\todo{write something better}

Using Theorem \ref{main result} and following the proof of Theorem 1 in \cite{Del99}, we prove Theorem \ref{acylindrical accessibility}.

\begin{proof}[Proof of Theorem \ref{acylindrical accessibility}]
	Let $\simp{K}$ be a presentation complex for $G$, so that $\pi_1 (\simp{K})=G$.
	Let $\CC{X}$ be a $d$-dimensional CAT(0) cube complex on which $G$ acts $(k,\mathcal{C})$-acylindrically on hyperplanes. 
	Pullback the hyperplanes of $\CC{X}$ to get a $d$-pattern $\ptrn{P}$ on $\simp{K}$ (see construction in Section \ref{resolutions}).
	Every hyperplane of $\CC{X}$ has at least one track in its pullback which is $G$-essential in the induced CAT(0) cube complex.
	Remove all non-$G$-essential tracks from the pattern.
	
	Let $R=R(k,d)$ be as in Observation \ref{Ramsey}, and let $C=C(K,d)$ be as in Theorem \ref{main result}.
	By the pigeon hole principle, if $\ptrn{P}$ has more than $R\cdot C$ tracks, then there are $R$ tracks which belong to the same parallelism class, and hence $k$ of them correspond to a chain in $\CC{X}$. 
	Let $\trk{t}$ be a track in this parallelism class.
	
	Since any element that stabilizes the hyperplane defined by $\trk{t}$ also stabilizes the set of tracks in the parallelism class of $\trk{t}$. 
	Thus, up to passing to a finite index subgroup it stabilizes each of the tracks in the parallelism class, and hence in the common stabilizer of the corresponding hyperplanes in $\CC{X}$.
	By the $(k,\mathcal{C})$-acylindricity on hyperplanes of the action, the stabilizer of $\trk{t}$ is in $\mathcal {C}$ since it stabilizes a chain of $k$ hyperplanes in $\CC{X}$. 
	The hyperplane defined by this track alone gives a $d$-pattern on $G$, which, by Proposition 3.2 of \cite{CaSa11} induces an essential $G$-action on a $d$-dimensional CAT(0) cube complex whose hyperplane stabilizers are in $\mathcal {C}$.
	Contradicting the assumption on $G$.
\end{proof}


\begin{proposition}\label{cube complexes to trees}
	Let $G$ be a finitely presented group. 
	\begin{enumerate}
	\item \label{one end implies no CCC over finite} If $G$ acts on finite dimensional \CCC with finite hyperplane stabilizers. Then either $G$ fixes a point or has more than one end.
	\item \label{Z CCC implies Z tree}
	If $G$ is moreover one-ended hyperbolic group and is not a triangle group. If $G$ acts on finite dimensional \CCC with virtually cyclic hyperplane stabilizers, then either $G$ fixes a point or $G$ splits over a cyclic group.
	\end{enumerate}
\end{proposition}

\begin{proof}
	Let $K$ be a the presentation complex of $G$.
	Let $\uptrn{P}$ be the pattern obtained by a pullback of the hyperplanes of the \CCC on which $G$ acts, and let $\CC{X}'$ be the induced cube complex.
	There are only finitely many orbits of hyperplanes in $\CC{X}'$.
	By Proposition 3.5 in \cite{CaSa11}, we may assume that the action is also essential by removing the non-essential tracks. 
	As always for finitely presented, $G$ acts cocompactly on the tracks of the pattern $\ptrn{P}$. 
	
	In the setting of \ref{one end implies no CCC over finite}, the tracks are essential and finite, proving that $G$ has more than one end.
	
	
	To prove \ref{Z CCC implies Z tree}, note that 
	by \ref{one end implies no CCC over finite}, either $G$ fixes a vertex of the resolution, and hence in the original action, or the track stabilizers are infinite virtually cyclic subgroups.
	In this case, since each track separates $\uc{K}$ to two essential components, and any virtually cyclic group is quasiconvex, we obtain a separating pair of points at the boundary.
	By Theorem 6.2 of \cite{Bow98}, this implies that $G$ splits over a virtually cyclic group.
\end{proof}

We finish this section by showing that acylindrical on hyperplanes actions on cube complexes and hyperbolic cubulations are the same for geometric actions.

\begin{proposition}\label{hyperbolicity acylindricity}
	Let $G$ be a group acting properly, cocompactly  on a CAT(0) cube complex $\CC{X}$. Then, $G$ is hyperbolic if and only if $G$ acts $(k,\mathcal F)$-acylindrically on hyperplanes, for some $k\in\N$.
\end{proposition}

\begin{proof}
	If $G$ is hyperbolic then the cube complex $\CC{X}$ is $\delta$-hyperbolic for some $\delta$, and hence if it is not $(k,\mathcal{F})$-acylindrically on hyperplanes for any $k\in\N$ then one can find an arbitrarily wide strip in $\CC{X}$, contradicting hyperbolicity.
	
	For the converse, by the Corollary of \cite{Bri95}, it suffices to show that there are no flats in $\CC{X}$. Assume $F$ is a 2-dimensional flat in $\CC{X}$. Let $\Hs{H}_F$ be the hyperplanes that are transverse to the flat $F$. There is a chain of hyperplanes in $\Hs{H}_F$ of length $k$ which intersect $F$ in parallel lines. This implies that there are two hyperplanes $\hyp{h},\hyp{k}$ whose common stabilizer is finite but their $R$ neighborhoods have unbounded intersection for some $R>0$. By a standard argument this implies that the common stabilizer is infinite, contradicting the acylindricity on hyperplanes.
	%	Let $\gamma$ be an infinite geodesic in the intersection of the $R$ neighborhoods of $\hyp{h}$ and $\hyp{k}$. The stabilizer of $\hyp{h}$ acts cocompactly on $\hyp{h}$. Let $g_n$ be elements of $\Stab_G(\hyp{h})$ that send $\gamma(0)$ within distance $K$ from $\gamma(n)$. Up to passing to a subsequence, and taking differences (i.e, $g_s' = g_n g_m^{-1)$), there are infinitely many elements in the common stabilizer of $\hyp{h}$ and $\hyp{k}$, contradicting the acylindricity assumption. 
\end{proof}

%The following corollary is an easy corollary from the works of Sela-Rips todo{ref Sela-Rips} and Delzant \todo{ ref Delzant}, that shows that there are only finitely many conjugacy classes of embeddings of a one-ended group into a hyperbolic group.
%However, it also follows easily from Corollary \ref{one ended acyl accessibility} and Proposition \ref{hyperbolicity acylindricity}.
%
%\begin{corollary}
%	Let $G$ be a hyperbolic cubulated group. Let $H$ be a one ended group, then there exists $C$, such that any embedding of $H$ into $G$ has at most $C$ orbits of $H$-essential hyperplanes.	
%\end{corollary}


%\begin{proposition} \label{Z CCC implies Z tree}
%	Let $G$ be a one-ended hyperbolic group. If $G$ acts on finite dimensional \CCC with virtually cyclic hyperplane stabilizers, then either $G$ fixes a point or $G$ splits over a cyclic group.
%\end{proposition}
%
%\begin{proof}
%	Let $K$ be a the presentation complex of $G$.
%	Let $\uptrn{P}$ be the pattern obtained by a pullback of the hyperplanes of the \CCC on which $G$ acts.
%	$G$ acts cocompactly on the tracks of the pattern. By the previous proposition, either $G$ fixes a vertex of the resolution, and hence in the original action, or the track stabilizers are infinite virtually cyclic subgroups.
%	In this case, since each track separates $\uc{K}$ to two components, and any virtually cyclic group is quasiconvex, we obtain a separating pair of points.
%	By Bowditch \todo{ref}, this implies that $G$ splilts over a virtually cyclic group.
%\end{proof}

