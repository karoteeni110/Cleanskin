% !TeX root = main.tex

\section{Reductions for sequences}

The goal of this section is to describe the various reductions we will use when considering sequences of pairs of halfspaces and crosses. We assume throughout that the intervals involved have dimension at most $d$.

%\begin{observation}
%	\label{silly obs}
%	In an interval $\Int{I}$ we will often use the following two observations, which we will call the Diagonal rule and the Sandwich rule.
	%\todo{do we really use them?} \todo{We use them when we do inversions of chains and stuff like this. We can however assume it is obvious, it is only 4 elements, let's not use $i$}
	%\begin{description}
		%\item[Diagonal rule] If $\hs{h}_1,\hs{h}_2,\hs{k}_1,\hs{k}_2$, satisfy, $\hs{h}_i\cross\hs{k}_i$ (for $i=1,2$), $\hs{h}_1\le\hs{h}_2$ and $\hs{k}_1\ge\hs{k}_2$, then $\hs{h}_i \cross \hs{k}_{3-i}$ ($i=1,2$).
		%\item[Sandwich rule] If $\hs{k}_1\le\hs{k}_2\le\hs{k}_3$ and $\hs{k}_1\cross\hs{h}\cross\hs{k}_3$ then $\hs{h}\cross \hs{k}_2$.
		%		\item[Basic] If $\hs{h},\hs{k}$ are \low and \up halfspaces respectively, then $\hs{h} \ngeq \hs{k}$.
		%		\item[Triangle] If $\hs{h}$ is \up and $\hs{k}$ is \low, and if $\hs{h}\cross\hs{k'}\ge\hs{k}$ then $\hs{h}\cross\hs{k}$.
		%		Similarly, $\hs{h}\ge\hs{h'}\cross\hs{k}$ then $\hs{h}\cross\hs{h}$.\todo{is it this way or the opposite?}
		%		\item If $\CCc{C}$ and $\CCc{C'}$ are crosses, then $(\CCc{C}\join\CCc{C}')\llow$ are transverse to $(\CCc{C}\meet\CCc{C}')\lup$.
	%\end{description}
%\end{observation}


 A \emph{chain} of halfspaces is a sequence of halfspaces $(\hs{h}_1,\ldots,\hs{h}_n)$ such that either $\hs{h}_1<\hs{h}_2\ldots<\hs{h}_n$, $\hs{h}_1>\hs{h}_2\ldots>\hs{h}_n$ or $\hs{h}_1=\hs{h}_2=\ldots=\hs{h}_n$. We say that the chain is \emph{increasing}, \emph{decreasing}, or \emph{constant} respectively.
 
 A \emph{chain of $p$-tuples} is a sequence of $p$-tuples of halfspaces $\chainoftuples{t}{p}{1}{n}$ such that for all $1\leq i\leq p$, the sequence $\hs{t}^i_1,\ldots,\hs{t}^i_n$ is a chain.
 
 A sequence of crosses $\chainofcrosses{C}{1}{N}$ is \emph{regularly ordered} if all crosses have same dimension $p$ and if there exists a chain of $p$-tuples $\chainoftuples{t}{p}{1}{n}$, such that $C_i = \left\{\hs{t}^1_i,\dots \hs{t}^p_i\right\}$. It is \emph{regularly increasing}, if non of the chains are decreasing.
 A \emph{subchain} of a regularly ordered sequence of crosses is one of the chains $\hs{t}^j_1,\dots, \hs{t}^j_n$.
% Given a regularly ordered sequence of crosses of dimension $p$, we define $p$ maps $\pi_p$ defined as $\pi_i(j) = \hs{h}^i_j$. 


\begin{observation}
	\label{Ramsey}
	For all $n$ there exists $R(n,d)$ such that any sequence of $R(n,d)$ (not necessarily distinct) halfspaces $\hs{h}_1,\ldots,\hs{h}_{R(n,d)}$ contains a subsequence which is a chain of length $n$.%, $\hs{h}_{i_1}<\ldots<\hs{h}_{i_n}$.
\qed
\end{observation}

By applying Observation \ref{Ramsey} several times one can deduce the following lemmas.

\begin{lemma}\label{Ramseypowerup}
	For every $n$ and every $p$ there exists $N=N(n,d,p)$ such that every sequence $\chainoftuples{t}{p}{1}{N}$ of $N$  $p$-tuples of halfspaces. Then there exist subsequence of $n$ $p$-tuples $\chainoftuples{t}{p}{i_1}{i_n}$ which is a chain of $p$-tuples. \qed
\end{lemma}

The following lemma, which follows from Ramsey's Theorem, shows that for an increasing chain of pairs one can reduce to one of two extreme scenarios:
\begin{itemize}
	\item a \emph{staircase} is an increasing chain of pairs $\chainofhyps{h}{k}{1}{n}$ such that $\hs k_i > \hs h_j$ for all $i\geq j$ and $\hs k_i \pitchfork \hs h_j$ for all $i<j$,
	
	\item a \emph{ladder} is an increasing chain of pairs $\chainofhyps{h}{k}{1}{n}$ such that $\hs h_i < \hs k_i$ for all $i$ and $\hs{k}_i < \hs h_{i+1}$ for all $i<n$.
\end{itemize}

\begin{lemma}\label{reductionforpairs}
	For every $n$ there exists $N=N(n,d)$ such that for every set of $N$ distinct adjacent pairs of halfspaces $\setofhyps{h}{k}{1}{N}$  there exists a increasing chain sequence of $n$ pairs $\chainofhyps{h}{k}{i_1}{i_n}$ which is either a staircase or a ladder.
\end{lemma}

\begin{proof}
	By applying Lemma \ref{Ramseypowerup} we may assume that  $\chainofhyps{h}{k}{1}{N}$ is a chain, and by reordering we may assume that that both subchain are increasing (notice that $d$ bounds the number of distinct adjacent halfspaces to a given halfspace $\hs{h}$).
	Let us consider the graph whose vertices are the pairs $\chainofhyps{h}{k}{1}{N}$, and whose edges are the pairs $(\hs{h}_i,\hs{k}_i)$ and $(\hs{h}_j,\hs{k}_j)$ ($i<j$) such that $k_i$ crosses $h_j$.
	By Ramsey's theorem there exists $N$ such that either there exists a $n$-clique or a $n$-independent set, these correspond to the staircase and ladder scenarios.
\end{proof}


	Given a chain of pairs $\chainofhyps{h}{k}{1}{n}$ and a chain of halfspaces $\hs{t}_1,\ldots,\hs{t}_{n-1}$. we say that the chain of halfspaces is \emph{tame} with respect to the chain of pairs if for all $1\leq i<n$, we have $\hs{t}_i<\hs{h}_{i+1}$. The chain of halfspaces is \emph{wild} if $\hs{t}_i \pitchfork \hs{h}_{j}$ for all $j>i$.

	Let $\chainofhyps{h}{k}{1}{n}$ be a chain of pairs  and $\chainofcrosses{C}{1}{n}$ be a regularly ordered chain of crosses. The chain of crosses is \emph{weakly tame} if for each halfspace $t_i \in C_i$ we have $\hs t_i\not > \hs h_{i+1}$. It is \emph{tame} if one of its subchains is tame. It is \emph{$\CCc{K}$-tame} if for all subchain of halfspaces  $\hs{t}_1,\ldots,\hs{t}_{n-1}$, either the chain is tame or for all $i$, $\hs{t}_i \pitchfork \hs{h}_i$.% \todo{is it what we want?}



Note that in the case of crosses, tame and $\CCc{K}$-tame imply weakly tame. However since $\CCc{K}$ may be empty, $\CCc{K}$-tame does not imply tame.

\begin{lemma}\label{reductionforcrosses}%\label{tameorunbounded}\label{Ramseyforcubes}\label{orderedimpliesincreasing}
	For all $n$ there exists $N=N(n,d)$ such that for every sequence  $\chainofcrosses{C}{1}{N}$ of crosses of dimension $p$ there exists a regularly increasing sequence $\chainofcrosses{D}{i_1}{i_n}$ of crosses of dimension $p$ such that if $\chainofcrosses{C}{1}{N}$ have any of the following properties
	\begin{itemize}
		\item tame;
		\item \intcs;
		\item having a subchain $(\hs{t}_1, \dots \hs{t}_n)$ such that $\hs{t}_i \geq \hs{h}_i$;
	\end{itemize}
	with respect to $\chainofhyps{h}{k}{1}{N}$, then $\chainofcrosses{D}{i_1}{i_n}$ have the same properties with respect to $\chainofhyps{h}{k}{i_1}{i_n}$.
	If moreover $\chainofcrosses{C}{1}{N}$ are weakly tame then one can choose  $\chainofcrosses{D}{i_1}{i_n}$ so that every subchain $(\hs{t}_{i_1}, \dots \hs{t}_{i_n})$ of $\chainofcrosses{D}{i_1}{i_n}$ is either tame of wild.
\end{lemma}

\begin{proof}
	By applying Lemma \ref{Ramseypowerup} we may pass to a subsequence of $N'$ crosses which is regularly ordered. By abuse of notation we will assume that $\chainofcrosses{C}{1}{N'}$ are regularly ordered.
	Let us consider the crosses
	$$\CCc{D}_i= \{\hs{t}_i | \{\hs{t}_j\}_j \text{ is non-decreasing}\} \cup \{\hs{t}_{n-i} | \{\hs{t}_j\}_j\text{ is decreasing}\}.$$
	The sequence $\chainofcrosses{D}{1}{N'}$ is a regularly increasing sequence of crosses. Moreover, each of the three properties in the lemma pass on to $\chainofcrosses{D}{1}{N'}$.
	
	As in the proof of Lemma \ref{reductionforpairs}, an application of Ramsey's theorem shows that for $N'$ big enough, we can pass to a subsequence of $n$  crosses, which by abuse of notation we will denote again by $\chainofcrosses{D}{1}{n}$ such that every subchain $(\hs{t}_1, \dots \hs{t}_n)$ of $\chainofcrosses{D}{1}{n}$ is either tame or wild. 
	Since the three properties pass to subsequences they remain true for $\chainofcrosses{D}{1}{n}$. 
	Note that weak tameness is needed to insure that each $\hs t_i$ is below or transverse to $\hs h_{i+1}$, and thus is not above all $\hs h_{j}$.
\end{proof}

\begin{lemma}\label{vertical horizontal trick}
	Let $\chainofhyps{h}{k}{1}{n}$ be a staircase, and let $\chainofcrosses{C}{1}{n}$ be a tame regularly increasing sequence of crosses of dimension $p$ such that any subchain is either tame or wild. Assume that each cross $\CC{C}_i$ contains a halfspace $\hs{s}_i$ such that $\hs{h}_i\leq\hs{s}_i$. Then there exists a regularly increasing  sequence of crosses $\chainofcrosses{D}{1}{n'}$ of dimension $p$ which are tame and $\CCc{K}$-tame \intc, with tame or wild subchains with respect to the chain of pairs $\chainofhyps{h}{k}{2}{2n'}$ of even indices where $n' =\lfloor\frac{ n-1}{2} \rfloor$.
\end{lemma}
 
\begin{proof}
	Let $\CCc{C}\lwild$ (resp. $\CCc{C}\ltame$) be the set of all wild (resp. tame) halfspaces in $\CCc{C}$. Then the sequence $\chainofcrosses{D}{1}{n-1}$ of crosses which are defined by $\CCc{D}_i = \CCc{C}\lwild_i \cup \CCc{C}\ltame_{i+2}$ are \intcs for $(\hs{h}_{i+1},\hs{k}_{i+1})$. %both $(\hs{h}_i,\hs{k}_i)$ and $(\hs{h}_{i+1},\hs{k}_{i+1})$. 
	The set $\CCc{D}_i$ is a cross because an element in $\CCc{C}\ltame_{i+2}$ cannot be strictly below an element of $\CCc{C}_{i}$ by the regular increasing order on $\CCc{C}_i$, and it cannot be strictly above an element of $\CCc{C}\lwild_i$ since it is below $\hs{h}_{i+3}$ and every element of $\CCc{C}\lwild_i$ crosses $\hs{h}_{i+3}$. 
	Moreover elements of $\CCc{C}\lwild_i$ intersect $\hs{h}_{i+1}$, elements  of $\CCc{C}\ltame_{i+2}$ are not smaller or equal to $\hs h_{i+1}$ since they intersect $\hs{s}_{i+2}$, and cannot be above $\hs{k}_{i+1}$ since  $\hs{k}_{i+1} \pitchfork \hs{h}_{i+3}$. Therefore $\CCc{D}_i$ is an \intc.
	
	Since $\chainofcrosses{C}{1}{n}$ are tame, the chain of crosses with odd indices $\{\CCc{D}_1, \CCc{D}_3, \dots\}$ is tame with respect to the subsequence of $\{(\hs{h}_{2},\hs{k}_{2}),(\hs{h}_{4},\hs{k}_{4}),\ldots\}$ of even indexed pairs. 
	It is also $\CCc{K}$-tame because the only halfspaces that do not intersect $\hs{h}_i$ are coming from $\CCc{C}\ltame_{i}$.
\end{proof}


%\begin{lemma}\label{tametostronglytame}
%	Let $\chainofhyps{h}{k}{1}{n}$ be a chain of adjacent pairs, and let $\chainofcrosses{C}{1}{n}$ be a sequence of regularly increasing sequence of \intcs such that every subchain of halfspaces is either tame or wild. Let $\CCc{C}\ltame$ (resp. $\CCc{C}\lwild$) be the set of all tame (resp. wild) halfspaces in $\CCc{C}$. Then the sequence $\chainofcrosses{D}{2}{n-1}$ of crosses which are defined by $\CCc{D}_i = \CCc{C}\lwild_{i-1} \cup \CCc{C}\ltame_{i}$ are $\CCc{K}$-tame \intcs for $(\hs{h}_{i},\hs{k}_{i})$.
%\end{lemma}

%\begin{proof}
%	The set $\CCc{D}_i$ is a cross as in the proof of Lemma \ref{vertical horizontal trick}. It is $\CCc{K}$-tame because the only halfspaces that do not intersect $\hs{h}_i$ are coming from $\CCc{C}\ltame_{i}$.
%\end{proof}


