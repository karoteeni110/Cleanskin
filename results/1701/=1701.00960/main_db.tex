\documentclass{llncs}

%\usepackage{moreverb}
%\usepackage{amsmath}
%\usepackage{comment}
\usepackage{alltt}
%\usepackage{hyperref}
\usepackage{framed}

\usepackage{amssymb}
\usepackage{graphics}
\usepackage{listings}
\usepackage[utf8x]{inputenc}
%\usepackage[T1]{fontenc}

%\usepackage{listingsutf8}
%\usepackage{wrapfig}

\usepackage{marginnote}
\usepackage{tikz}
\usetikzlibrary{shapes,arrows}
\usetikzlibrary{decorations.markings}

\newif\ifshowcomments
\showcommentstrue
%\showcommentsfalse

\ifshowcomments
\newcommand{\mynote}[2]{\marginnote{{\bfseries\sffamily\scriptsize#1}
 {\small$\blacktriangleright$\textsf{\emph{#2}}$\blacktriangleleft$}}}
\else
\newcommand{\mynote}[2]{}
\fi
\newcommand{\bs}[1]{\mynote{Badr}{#1}}
\newcommand{\mt}[1]{\mynote{Tahar}{#1}}
\newcommand{\jpb}[1]{\mynote{Jean-Paul}{#1}}
\newcommand{\mf}[1]{\mynote{Mamoun}{#1}}

\lstset{tabsize=2,inputencoding=utf8x,language=eventB}

%\begin{comment}
\newcommand{\textlambda}{\(\lambda\)}
\DeclareUnicodeCharacter{8712}{{\(\in\)}}
\DeclareUnicodeCharacter{8788}{{\(:\,=\)}}
\DeclareUnicodeCharacter{8838}{{\(\subseteq\)}}
\DeclareUnicodeCharacter{8229}{{\(.\,.\)}}
\DeclareUnicodeCharacter{8696}{{\(\mkern 6mu\mapstochar \mkern -6mu\rightarrow\)}}
\DeclareUnicodeCharacter{10496}{{\(\mkern 6mu\mapstochar \mkern -6mu\twoheadrightarrow\)}}
\DeclareUnicodeCharacter{10516}{{\(\mkern 9mu\mapstochar \mkern -9mu\rightarrowtail\)}}
\DeclareUnicodeCharacter{10518}{{\(\rightarrowtail \mkern  -18mu\twoheadrightarrow\)}} 
\DeclareUnicodeCharacter{57603}{{\(\lhd\mkern-9mu-\)}} % <+
\DeclareUnicodeCharacter{9665}{{\(\triangleleft\)}}
\DeclareUnicodeCharacter{9655}{{\(\triangleright\)}}
\DeclareUnicodeCharacter{8473}{{\(\mathbb{P}\)}}
%\end{comment}


%\tikzstyle{context} = [fill=red!20, minimum height=4em, rounded corners, text centered,]
%\tikzstyle{machine} = [fill=blue!20, minimum height=4em, text centered,]

\title{An Event-B framework \\ for the validation of Event-B refinement plugins \\
               (ongoing work) 
        }
\author{J.-P. Bodeveix, M. Filali, , M.-T. Bhiri, B. Siala}
\institute{IRIT CNRS UPS Université de Toulouse \\ 
                Université de Sfax
              }
\date{October, 2016}
\usepackage{fullpage}
\begin{document}

\maketitle


\begin{abstract}
%\input{abstract}
 We propose an Event-B framework for modeling the underlying theoretical foundations of Event-B.
The aim of this framework is to reuse, for Event-B itself, the  refinement development process.
This framework introduces first, a functional kernel through an Event-B context, then, it defines Event-B projects, their static and
dynamic semantics through Event-B machines. We intend to use this framework for the validation
of Event-B plugins related to distribution and for Event-B extensions related to composition and
decomposition.
\end{abstract}
%\setcounter{tocdepth}{3}   % à commenter
\pagestyle{plain} % à commenter


%%\begin{figure}\center
  %\missingfigure[figheight=.10\textheight, figwidth=\textwidth]{Graphical Abstract}
%  \includegraphics[height=.15\textheight]{graphical_abstract-crop}
%  \caption{Scheme of analyses involving the core structural connectivity matrix.\label{fig:process-illustration}}
%\end{figure}

Isolating the common brain connectivity network from a population is a main problem in current neuroscience~\cite{Bullmore2009,Gong2009,Wassermann2016}. Recent evidence suggests that there's a common and densely connected brain connectome across humans~\cite{Bassett2013}. In this work we present a new approach for selecting these common connections, combining recent topological hypotheses~\cite{Bassett2013}  and  current methods~\cite{Gong2009,Wassermann2016}.

Finding the common brain connectome across subjects has the potential to increase our understanding of the relationship between function and structure in the brain. This relationship is one of the main open questions in neuroscience~\cite{Bullmore2009,Donahue2016}. Moreover, knowledge about the most common connections in a population will facilitate clinical and cognitive Diffusion MRI analyses by reducing the number of surveyed connections, increasing the statistical power of those analyses. Finding the common connectome will also allow us to increase our knowledge about the brain structure by comparing core networks across different populations.

We formalize the problem of selecting the common connections combining graph theory and statistics. Then, we prove that the problem is \NP-Hard and propose a polynomial-time algorithm to find approximate solutions. To do this, we develop an exact polynomial-time algorithm for a relaxed version of the problem and prove the algorithm's correctness and complexity.

Currently, the most used algorithm to extract a population's core structural connectivity network (CSNC)~\cite{Gong2009} uses an statistical approach: first, compute a connectivity matrix for each subject; then, analize each connection separately with a hypothesis test, using as null hypothesis that that edge is not present in the population; finally, construct a binary graph with the edges for which the null hypothesis was rejected. The main problem of Gong et al.'s~\cite{Gong2009} algorithm is that the resulting graph can be a set of disconnected subgraphs. Moreover, recent studies have shown that the brain has a \emph{core} network tightly connected and a sparsely connected \emph{outer} one~\cite{Bassett2013}. In other words, this approach ignores the resulting network's topology. Performing statistical analyses in a feature set chosen by hypothesis testing incurs in the double dipping problem~\cite{Kriegeskorte2009}.

A newer approach to solve the CSNC problem, designed by Wassermann et al.~\cite{Wassermann2016}, uses graph theory to get a connected CSCN: first, compute a binary connectivity graph for each subject using a threshold;  for each possible connection compute the ``cost'' of including or excluding it from the common graph by evaluating in how many subjects that connection is present; finally, construct the binary graph with all the edges that is ``cheaper'' to include than to exclude and connect the resulting graph if it's disconnected, using the minimum possible cost. This algorithm guarantees that the resulting graph is connected, but the connection binarization discards significant information for the resulting common network. In other words, it discards information of the probability of each connection being in the brain. This is problematic because the resulting graph may include edges for which tractography assigned a very low existence probability across subjects. Also, the outer part of the brain, the connections which do not result in the core network, should also be sparsely connected~\cite{Bassett2013}, which this algorithm does not enforce.

In this work we propose, for the first time, a polynomial-time algorithm to obtain the CSCN of a population  addressing the issues listed above. Our algorithm combines the recent graph-theoretical approach~\cite{Wassermann2016} with the statistical awareness of the most popular one~\cite{Gong2009}. We start by formalizing the problem, which allow us to prove that it's \NP-Hard. Then, we propose a first algorithm that solves a relaxed version of the problem in an exact way, giving the best possible core graph for our formalization. Then, we adapt it to guarantee a connected result, agreeing with recent evidence on structural connectivity network topology \cite[e.g.]{Bassett2013}. Finally, we validate our approach using 300 subjects from the HCP database and comparing the performance of the networks obtained by our new approach, Wassermann et al.'s~\cite{Wassermann2016} and Gong et al.'s~\cite{Gong2009} predicting connectivity values from handedness in the core network.
\section{Introduction}

     Event-B~\cite{Abrial2010} is a method that has been proposed for building formal models together with their proofs.
As a matter of fact, it has been used for a large range of applications. Nevertheless, it
seems that, in general, it has not been applied to the field of software engineering by itself. 
In this paper, we report on an Event-B meta-framework and  two software engineering applications for which the use of the
Event-B methodology seemed to us worth to apply.  

The rest of the paper is organized as follows.  
%Section 2 gives an overview of the Event-B language. 
Section 2 outlines the main features of an Event-B framework. Section 3 discusses about two 
software applications. In conclusion, Section 4 considers some related work and sketch
future work directions.
 
%\section{A brief overview of Event-B}

%\input{framework}
\section{Towards an Event-B meta-level framework}

The proposed meta-level framework aims at validating Event-B model
transformations. We focus on transformations linked to a top-down,
refinement-based development process. Their goal is to assist the user
in producing refinements of his model through patterns parameterized
with the help of domain specific languages. Thus, a transformation pattern takes
as input an Event-B machine and some parameters. It produces either a
single machine or a set of machines. In the latter case, it is
necessary to model the project level -- not a single machine -- in order to consider the
interaction of the machines of the project. However, to make things simpler, we 
consider neither contexts, nor refinement links between
machines. 
Refinement will be taken into account at the meta level, 
each transformation producing a refinement of the project.

    \subsection{Methodology}

    We now propose a meta-level specification of an Event-B project in
    Event-B itself. The difficulty of such an exercise is to find the
    right level of abstraction and to identify which features should be
    modeled as constants and as variables. It is strongly linked with
    the objectives we have fixed. First, given the patterns we
    envision, predicates and expressions should be left as abstract as
    possible. Second, we target operations which should modify the
    project by adding new machines. Two orthogonal dynamics will thus
    be considered: project contents evolution and project operational
    semantics.  Furthermore, we try to use a refinement-based approach
    to specify the meta-level: its features will be introduced
    incrementally.

    \subsection{The global view}
Figure \ref{mch} describes the overall structure of a machine as a
class diagram. The conversion to Event-B is performed as follows:

\begin{itemize}
  \item \texttt{Machine} is introduced as a set, with
    \texttt{Machines} being the subset of existing machines. 
  \item Machine attributes and operations can be updated and are
    defined as variables.
  \item \texttt{Predicate}, \texttt{Ident} and \texttt{EventName}. \texttt{Ident}
    is partionned into \texttt{Var}, \texttt{Prime} denoting primed
    versions of machine variables and \texttt{Param}.
  \item \texttt{Event} is modeled as a triple with three projections
    (\texttt{Pars}, \texttt{Guard} and \texttt{Action}). 
\end{itemize}

\begin{figure}[hbt]
\centering
\resizebox{\linewidth}{!}{\includegraphics{eVB.pdf}}
\caption{Event-B machines}
\label{mch}
\end{figure}

%\input{functional}
    \subsection{The functional kernel}

           The functional kernel introduces abstraction of predicates and events as Event-B contexts.
A predicate is defined as a set of abstract states. It is mainly characterized by axioms stating the existence
of the \texttt{Free} function returning the set of the free variables of a predicate and the substitution
function. With respect to our specific needs concerning decomposition/composition and distribution
we also assume the existence of a \texttt{Conjuncts} function returning a set of  predicates of which conjunct
is equivalent to the initial predicate.  For instance, the conjuncts of ``p = TRUE'' is ``\{ p = TRUE \}'' and the
conjuncts of   ``p = TRUE $\wedge$ v = 2'' is ``\{ p = TRUE, v = 2 \}''.
An excerpt of of the Predicate context is the following:

\begin{framed}
\begin{small}
\begin{alltt} 
context cPredicate extends cIdent

sets State

constants Predicate Free Subst Proj Conjuncts ...

axioms
  @Predicate_def Predicate = ℙ(State)
  @Free_ty Free ∈ Predicate → ℙ(Ident)
  @Subst_ty Subst ∈ (Ident ⇸ Ident) → (Predicate → Predicate)
  @Proj_ty Proj ∈ ℙ(Ident) → (Predicate → Predicate)
  @Conjuncts_ty Conjuncts ∈ Predicate → ℙ1(Predicate)
  @Conjuncts_ax ∀ p· p ∈ Predicate ⇒  inter(Conjuncts(p)) = p
  @Free_Conjuncts ∀ p· p ∈ Predicate ⇒  union(Free[Conjuncts(p)]) = Free(p)
 ...
\end{alltt}
\end{small}
\end{framed}


    \subsection{The Event-B project structure}

Besides contexts, Event-B projects are modelled through the following
refinement steps:

\begin{itemize}
  \item \texttt{mProject} defines the overall structure of machines
    and a project as a set of machines and provides an event to add a
    machine to a project.

  \item \texttt{static\_semantics} adds wellformedness rules
    concerning the usage of identifiers within predicates. Machine
    addition is restricted to well formed machines.

%% couper en 2???  (invariant?? et preservation de l'invariant) puis
%% semantique op (step)

 \item \texttt{dynamics} adds the invariant preservation property and
    provides a dynamic semantics to a project through the introduction
    of a state and of the \texttt{step} event defining the operational
    semantics of the project.
\end{itemize}

    \subsection{Event-B project and machines}

An Event-B project is seen as a set of machines. Each machine has
variables, an invariant and a set of events indexed by event
names. In order to make easier the meta-level reasoning, we consider
that a machine has a unique invariant and that an event has a unique
guard and a unique action (seen as a before-after predicate). These predicates will be seen as conjunctive later.

\begin{framed}
\begin{small}
\begin{alltt} 
machine mProject sees cMachine cEvent 

variables Machines mVars mInv mEvents 

invariants
  @machines_ty Machines ⊆ Machine
  @mVars_ty mVars ∈ Machines → ℙ(Var)
  @mEvents_ty mEvents ∈ Machines → (EventName ⇸ Event)
  @mInvs_ty mInv ∈ Machines → Predicate
events
  ...
end
\end{alltt}
\end{small}
\end{framed}

The \texttt{mProject} machine also provides the \texttt{new\_machine} event for adding
machines to a project. Its takes seven parameters specifying the set of
machines to be added and for each of them a set of variables, an
invariant, event names, and parameters, guard and action of each event.


   \subsection{The static semantics}

The static semantics specifies visibility constraints for variables
and parameters: 
\begin{itemize}
  \item an invariant of a machine uses variables of this machine\footnote{For the moment, we do not take into account refinements and consequently the gluing invariant.}
  \item a guard of an event can use parameters of this event and
    variables of the  machine  the event belongs to.
  \item an action of an event can use parameters of this event,
    variables of the  machine and their primed versions.
\end{itemize}

\begin{framed}
\begin{small}
\begin{alltt}
machine static_semantics refines mProject
sees cMachine

variables Machines mVars mInv mEvents 

invariants
  @mInv_ctr ∀ m · m ∈ Machines ⇒ Free(mInv(m)) ⊆ mVars(m)
  @mGuards_ctr 
     ∀ m,e· m ∈ Machines ∧ e ∈ dom(mEvents(m))
       ⇒ Free((mEvents(m);Guard)(e)) ⊆ mVars(m) ∪ (mEvents(m);Pars)(e)
  @mActions_ctr 
      ∀ m,e· m ∈ Machines ∧ e ∈ dom(mEvents(m))
       ⇒ Free((mEvents(m);Action)(e)) ⊆ mVars(m) ∪ Next[mVars(m)] ∪ (mEvents(m);Pars)(e)
\end{alltt} 
\end{small}
\end{framed}

   \subsection{The dynamic semantics}

This refinement takes into account the dynamic of a project. First,
standard proof obligations are added to express that the machine
invariant is preserved by each event. The expression of proof obligations
takes advantage of the representation of a predicate as a set:
conjunction and implication are replaced by intersection and set inclusion.
Second the operational semantics
of a project is defined through the introduction of a state for the
subset of machines considered to be active, and a \texttt{step} event modelling the evolution of the
state. The state is declared as a decomposable predicate over machine
variables. It abstracts the usual view of a state as a valuation of each state
variable. Machine invariants should be satisfied by the state.

\begin{framed}
\begin{small}
\begin{alltt}
machine dynamics refines static_semantics
sees cMachine cEvent

variables Machines mVars mInv mEvents state

invariants
  @state_ty state ∈ Machines ⇸ Decomposable     // only defined on active machines
  @state_dync ∀m· m ∈ dom(state) ⇒ state(m) ⊆ mInv(m)
  @free_state ∀m· m ∈ dom(state) ⇒ Free(state(m)) ⊆ mVars(m)
  @mInv ∀m,e· m ∈ Machines ∧ e ∈ dom(mEvents(m))
           ⇒  mInv(m) ∩ (mEvents(m);Guard)(e) ∩ (mEvents(m);Action)(e) ⊆ Subst(Next)(mInv(m))
\end{alltt}
\end{small}
\end{framed}

The \texttt{step} event makes a machine of the project advance by
updating its state. It takes as parameters a machine \texttt{m}, an
event name \texttt{e}, a predicate \texttt{p} specifying the value of
the parameters. The event guards are supposed to be satisfied by the current
state of the machine. Then its state is updated by applying the machine
action. The new state is obtained by suppressing primed in the
projection on primed variables of the conjunction of the old state,
the parameters and action predicates.

\begin{framed}
\begin{small}
\begin{alltt}
  event step
    any m e p
    where
      @m_ty m ∈ dom(state)
      @e_ty e ∈ dom(mEvents(m))
      @p p ∈ Predicate
      @f Free(p) ⊆ Param
      @g state(m) ∩ p ⊆ (mEvents(m);Guard)(e)
    then
      @a state(m) ≔ Subst(Next∼)(Proj(Next[mVars(m)])(state(m) ∩ p ∩ (mEvents(m);Action)(e)))
  end
\end{alltt}
\end{small}
\end{framed}

We also introduce an event to change the active set of machines: some
\textit{old} machines can be replaced by \textit{new} machines taken
in the pool of currently inactive machines. This event can be seen as
a hot replacement of components. It should be transparent. For this
purpose, we suppose that the conjunction of old machine states is
equal to the conjunction of new machine states. A typical application
will be to replace a compound machine by its subcomponents once it has
been split.
%\input{case_studies}
\section{Case studies}

  We have experimented the above meta description on two Event-B model transformations.
The first transformation deals with a safe refinement development process
for distributed applications~\cite{[SBBF16]} . This development
process proposes successive steps for splitting and scheduling complex
events. These steps are
defined by refinement patterns. They are specified through domain
specific languages.  From these specifications, two refinements were
generated. In the first phase of this work, the generated refinements
had to be verified through the Event-B framework, i.e., the Rodin
verification platform. With respect to that work, our  motivation
was to assert that the application of the proposed patterns actually
produce refinements of the source machine, so that the generated
machines are \textit{correct by construction}. Thus, it should not be
necessary to validate these refinements for each application of the
corresponding pattern.
%% C'EST LA MEME APPLICATION: les 2 transfos viennent en amont des
%% plugins de decomposition existants. Elles travaillent sur la vue centralisée
The second transformation deals with Event-B by itself. Actually, the last developments of
Event-B propose to enhance Event-B by decomposition methods. This has lead to two 
proposals: the state-based~\cite{[HA10]}  and the event-based~\cite{[SB10]}. 
Both methods have strong 
theoretical foundations. Moreover, they have been validated by significant applications and have
been both implemented by plugins available through the Rodin platform~\cite{[RCTB11]}. With respect to these
studies, our second motivation was how to \textit{manage the theoretical background} that is required
for the justification of Event-B enhancements like decomposition methods.

%       \subsection{The distribution plugin}

%        \input{shared_event}


%\begin{section}{Related Work}
In this section, we briefly review some closely related works and the key concept MMD in our interpretation.
\begin{paragraph}{Style Transfer}
Style transfer is an active topic in both academia and industry. Traditional methods mainly focus on the non-parametric patch-based texture synthesis and transfer, which resamples pixels or patches from the original source texture images~\cite{hertzmann2001image,efros2001image,efros1999texture,liang2001real}. Different methods were proposed to improve the quality of the patch-based synthesis and constrain the structure of the target image. For example, the image quilting algorithm based on dynamic programming was proposed to find optimal texture boundaries in~\cite{efros2001image}. A Markov Random Field (MRF) was exploited to preserve global texture structures in~\cite{frigo2016split}. However, these non-parametric methods suffer from a fundamental limitation that they only use the low-level features of the images for transfer. 

Recently, neural style transfer~\cite{neuralart} has demonstrated remarkable results for image stylization. It fully takes the advantage of the powerful representation of Deep Convolutional Neural Networks (CNN). This method used Gram matrices of the neural activations from different layers of a CNN to represent the artistic style of a image. Then it used an iterative optimization method to generate a new image from white noise by matching the neural activations with the content image and the Gram matrices with the style image. This novel technique attracts many follow-up works for different aspects of improvements and applications. To speed up the iterative optimization process in~\cite{neuralart}, Johnson \emph{et al.}~\cite{johnson2016perceptual} and Ulyanov \emph{et al.}~\cite{ulyanov2016texture} trained a feed-forward generative network for fast neural style transfer. % \cite{gatys2016controlling} further extended the neural style transfer by introducing control over spatial location, color information and across spatial scale.\footnote{Neural Doodle and MRF work here} To stabilize the transfer results in one video, \cite{ruder2016artistic} further incorporated a temporal constraint that penalizes the optical flow between two frames~\footnote{Double check whether it is accurate.}. \cite{selim2016painting} proposed novel spatial constraints through gain map to extend the neural style transfer to head portrait painting transfer.
\textcolor{black}{To improve the transfer results in~\cite{neuralart}, different complementary schemes are proposed, including spatial constraints~\cite{selim2016painting}, semantic guidance~\cite{neuraldoodle} and Markov Random Field (MRF) prior~\cite{li2016combining}. There are also some extension works to apply neural style transfer to other applications. Ruder \emph{et al.}~\cite{ruder2016artistic} incorporated temporal consistence terms by penalizing deviations between frames for video style transfer. Selim \emph{et al.}~\cite{selim2016painting} proposed novel spatial constraints through gain map for portrait painting transfer. }
Although these methods further improve over the original neural style transfer, they all ignore the fundamental question in neural style transfer: \emph{Why could the Gram matrices represent the artistic style?} This vagueness of the understanding limits the further research on the neural style transfer. 
\end{paragraph}

\begin{paragraph}{Domain Adaptation}
Domain adaptation belongs to the area of transfer learning~\cite{pan2010survey}. It aims to transfer the model that is learned on the source domain to the unlabeled target domain. The key component of domain adaptation is to measure and minimize the difference between source and target distributions. The most common discrepancy metric is Maximum Mean Discrepancy (MMD)~\cite{mmd}, which measure the difference of sample mean in a Reproducing Kernel Hilbert Space. It is a popular choice in domain adaptation works~\cite{ddc,dan,long2016unsupervised}. Besides MMD, Sun \emph{et al.}~\cite{coral} aligned the second order statistics by whitening the data in source domain and then re-correlating to the target domain. In \cite{adabn}, Li \emph{et al.} proposed a parameter-free deep adaptation method by simply modulating the statistics in all Batch Normalization (BN) layers.
\end{paragraph}

\begin{paragraph}{Maximum Mean Discrepancy} Suppose there are two sets of samples $X=\{\mathbf{x}_i\}_{i=1}^{n}$ and $Y = \{\mathbf{y}_j\}_{j=1}^{m}$ where $\mathbf{x}_i$ and $\mathbf{y}_j$ are generated from distributions $p$ and $q$, respectively. Maximum Mean Discrepancy (MMD) is a popular test statistic for the two-sample testing problem, where acceptance or rejection decisions are made for a null hypothesis $p = q$~\cite{mmd}. Since the population MMD vanishes if and only $p = q$, the MMD statistic can be used to measure the difference between two distributions. Specifically, we calculates MMD defined by the difference between the mean embedding on the two sets of samples. Formally, the squared MMD is defined as:
\begin{small}
\begin{equation}\label{mmd}
\begin{aligned}
&  \text{MMD}^2[X, Y]\\
		 = ~ &\| \mathbf{E}_x[\phi(\mathbf{x})] - \mathbf{E}_y[\phi(\mathbf{y})] \|^2\\
		= ~&\| \frac{1}{n}\sum_{i=1}^{n}\phi(\mathbf{x}_i) - \frac{1}{m}\sum_{j=1}^{m}\phi(\mathbf{y}_j) \|^2\\
		= ~&\frac{1}{n^2}\sum_{i=1}^{n}\sum_{i'=1}^{n}\phi(\mathbf{x}_i)^T\phi(\mathbf{x}_{i'}) + 
		   \frac{1}{m^2}\sum_{j=1}^{m}\sum_{j'=1}^{m}\phi(\mathbf{y}_j)^T\phi(\mathbf{y}_{j'}) \\
		&   -\frac{2}{nm}\sum_{i=1}^{n}\sum_{j=1}^{m}\phi(\mathbf{x}_i)^T\phi(\mathbf{y}_{j}),
\end{aligned}
\end{equation}
\end{small}
where $\phi(\cdot)$ is the explicit feature mapping function of MMD. Applying the associated kernel function $k(\mathbf{x}, \mathbf{y}) = \langle\phi(\mathbf{x}), \phi(\mathbf{y})\rangle$, the Eq.~\ref{mmd} can be expressed in the form of kernel:
\begin{small}
\begin{equation}{\label{mmd_kernel}}
\begin{aligned}
&\text{MMD}^2[X, Y]\\
	= ~ & \frac{1}{n^2}\sum_{i=1}^{n}\sum_{i'=1}^{n}k(\mathbf{x}_i, \mathbf{x}_{i'}) + 
		   \frac{1}{m^2}\sum_{j=1}^{m}\sum_{j'=1}^{m}k(\mathbf{y}_j, \mathbf{y}_{j'}) \\
	&	   -\frac{2}{nm}\sum_{i=1}^{n}\sum_{j=1}^{m}k(\mathbf{x}_i, \mathbf{y}_j).
\end{aligned}
\end{equation}
\end{small}
The kernel function $k(\cdot, \cdot)$ implicitly defines a mapping to a higher dimensional feature space.
\end{paragraph}

\end{section}






















%We presented for the first time a polynomial algorithm to extract the core structural connectivity network of a population combining a graph-theoretical approach with statistic relevance of the connections, observing the recent evidence of the structural network topology.

Our results show that our algorithm outperforms, in the prediction experiment, the most used technique~\cite{Gong2009} as well as latest approaches~\cite{Wassermann2016}. In Table~\ref{table:number_of_features} we can see that our algorithm preserves, in average, more connections correlated with the handedness of the subjects. We can also see that despite being less stable than Wassermann et al.'s it is stabler than Gong et al.'s. Finally, Fig.~\ref{fig:prediction_performance} shows that, in the handedness prediction experiment, our method outperforms  Gong et al.'s and Wassermann et al's: the number of cases with lower AIC and MSE is larger in our case. Hence, our CSCN is better as linear model relating connectivity with handedness in terms of model fitting and prediction.


In terms of theoretical contributions, we formalized the problem, proved its difficulty and gave a novel algorithm for dealing with it. We then validated our approach by showing its power as feature selector for getting connections related to handedness with 300 real subjects' data. The experiment shows our method performs better than the currently available. Moreover, our method avoids the double dipping problem by not choosing the feature set with hypothesis testing.
\section{Related Work and Conclusion}
  
    It is interesting to cite related works which have some connections
with ours.  First, Iliasov et al.~\cite{[ITLR09]} is a pioneering work
for dealing with the automation of development steps. For this
purpose, they propose the notion of refinement patterns.  Such
refinement patterns contain a syntactic description, applicability
conditions and proof obligations ensuring correctness preservation.
Unlike our approach where we stayed within an Event-B world,
\cite{[ITLR09]} adopt specific languages for representing Event-B
models and their so-called transformation rules. Last, the reuse of
the Event-B proof engine is not immediate.  Also, Cata{\~{n}}o et
al.~\cite{[CRW13]} adopt the so-called \textit{own medicine approach}
in the sense that they adopt Event-B for formalizing Event-B and JML
and the Rodin platform to discharge their proof obligations. With
respect to that our work is similar. However, their model is mainly
functional and their transformations are defined as functions. Their
correctness is stated through theorems. With respect to Event-B, we
have gone further since we have adopted a state-based approach. The
dynamic semantics as well as model transformations are defined as
events.  The correctness of the dynamic semantics and of the
transformations are obtained for free through the Event-B refinement.
Moreover, Cata{\~{n}}o et al.~\cite{[CRW13]} are concerned neither by
the validation of refinement patterns nor by the semantics of
composition.

   To conclude, Event-B proposes a refinement-based development method. In this
paper, we have studied how to support such a development method by
itself in order to formalize the underlying theoretical background:
the so-called meta level. The elaborated framework can also be used to
support Event-B enhancements as composition and decomposition methods.
As future work, we envision to broaden the coverage of our
framework. We are also interested in formalizing the links between
Event-B and temporal\cite{Hoang2016} or temporized~\cite{[GBF13]}
logics. More generally, the excplicit description of dynamic
behaviours through temporized patterns~\cite{[ADL12]} within an
Event-B framework looks challenging.
\bibliographystyle{abbrv}
\bibliography{biblio}

%\newpage

%\tableofcontents


\end{document}
