\documentclass[a4page,10pt]{article}
\usepackage[top=0.8in, bottom=0.8in, left=0.5in, right=0.5in]{geometry}
\usepackage{fancyhdr}
\renewcommand{\thispagestyle}[1]{}
\pagestyle{fancy}
\fancyhead[RE,RO]{Technical Report}
\fancyfoot[RE,RO]{TCS Research}
\fancyfoot[LE,LO]{Copyright $\copyright$ 2015 Tata Consultancy Services Ltd.}
%\usepackage[dvips]{graphicx}
\usepackage[pdftex]{graphicx}
\usepackage[utf8]{inputenc}
\usepackage{hyperref}
\usepackage{tabularx}
\usepackage{floatrow}
\usepackage{multirow}
\newfloatcommand{capbtabbox}{table}[][\FBwidth]

\title{Minimally-Supervised Attribute Fusion for Data Lakes}


\begin{document}
\author{Karamjit Singh, Garima Gupta, Gautam Shroff, and Puneet Agarwal
%
% Optional short acknowledgment: remove next line if non-needed
%
% DO NOT MODIFY THE FOLLOWING '\vspace' ARGUMENT
\vspace{.3cm}\\
%
%Addresses and institutions (remove "1- " in case of a single institution)
TCS Research, New-Delhi, India\\
%
% Remove the next three lines in case of a single institution
}
\maketitle 


\begin{abstract}
Aggregate analysis, such as comparing country-wise sales versus global market share across product categories, is often complicated by the unavailability of common join attributes, e.g., category, across diverse datasets from different geographies
or retail chains, even after disparate data is technically ingested into a common \textit{data lake}.
Sometimes this is a missing data issue, while in other cases it may be inherent, e.g., the records in different 
geographical databases may actually describe different product `SKUs', or follow different norms
for categorization. Record linkage techniques, such as \cite{christen2008febrl} can be used to automatically map products in different data sources
to a common set of \textit{global} attributes, thereby enabling federated aggregation joins to be performed.
Traditional record-linkage techniques are typically unsupervised, relying textual similarity features across attributes
to estimate matches. In this paper, we present an ensemble model combining minimal supervision using Bayesian network
models together with unsupervised textual matching for automating such `attribute fusion'. We present results of our approach 
on a large volume of real-life data from a market-research scenario and compare with a standard record matching algorithm. 
Finally we illustrate how attribute fusion using machine 
learning could be included as a data-lake management feature, especially as our approach also
provides confidence values for matches, enabling human intervention, if required.
%\keywords{We would like to encourage you to list your keywords within
%the abstract section}
\end{abstract}


\section{Introduction}
% !TEX root = ../EDBT.tex
Traditional business intelligence is rapidly evolving to adopt modern big-data analytics architectures based on the concept of a `data lake', where, rather than  first integrating multiple historical data from diverse sources into a common star schema via extraction-transformation-load operations, the datasets are maintained in their raw form. This leads to a number of challenges; for example, dealing with incongruous join keys between different datasets. 

In this paper, we focus on a problem of fusion of information about consumer products, such as sales, market share, etc., which is spread across disparate databases belonging to different organizations, across which a product is \textit{not} identifiable via a common key. For example, a \textit{Global database (DB)}
might track overall market-share of global product categories. On the other hand, each \textit{Local DB} might track sales data within a geography 
using local-product-ids along with other characteristics, but \textit{not} the global category-id. As a result, an analytical task such as comparing the
sales of product categories within each geograpy against their global market share becomes difficult due to the lack of a natural join attribute between the databases.
(Note that the same product might be characterized using different attributes in different countries, including also textual \textit{description} of products entered manually by retailers, e.g., for carbonated drinks it usually contains information of brand, size, material used, packaging etc.) 

% \begin{figure}
% \centering
% \includegraphics[width=65mm]{Figures/local-char}
% \vspace{4pt}
% \caption{Local characteristics of the same product across four geographies and its corresponding global characteristics}
% \vspace{-18pt}
% \label{fig:local}
% \end{figure}

One way to perform analysis across disparate databases is by mapping records in each \textit{Local DB} to their corresponding global attributes (e.g., `category'
in the example above). However, preparing such mappings is a huge manual and complicated task because: a)~The cardinality (number of possible values) of local and global characteristics varies from tens to thousands, and b)~Uncertainty in the semantics of local characteristics of the same product from different geographies, leading to confusion in identifying the product category, even by human annotators.
 
Our aim is to help reduce cost of the operational process of creating and maintaining such global references by reducing manual workload via automation
via modern data-lake architecture that include automated fusion of federated databases. Our goal is to either make high confidence predictions, or abstain from making any prediction so that such records can be sent to human annotators. We want to minimize the number of such abstentions while maximizing the precision of the predictions.

\textbf{\textit{Attribute Fusion using Record Matching}}:
Consider two databases (see Figure~\ref{fig:Prob1}): a) \textit{Local DB($L$)} with each product $l$ having local characteristics $L_1, L_2,..., L_M$, e.g., flavor, brand, etc., and retailer descriptions ($D_i$), and b)~a \textit{Global DB($G$)} having $K$ global characteristics. The problem at hand is thus a \textit{record matching} problem where products in local database are to be mapped to global characteristic values (e.g. `category', or `global brand' etc.). 

Note that our objective is only to reconcile performance metrics (such as volume sales and market-share) across databases for each global characteristic \textit{independently}, e.g., sales vs market-share for each category, or alternatively each global brand, etc.
We can achieve this by solving $K$ different record matching problems,
as shown in Figure~\ref{fig:Prob1}: For each product, we shall predict each of the $K$ global characteristics
given local characteristics and retailer descriptions separately, as $\arg\max_j P(G_j | L_1, ..., L_M, D_i)$.

In this paper: a)~We address the problem of automating attribute fusion across diverse data sources that do not share a common join key.
b)~We augment traditional, fundamentally unsupervised text-similarly techniques with supervised, Bayesian network models in a confidence-based ensemble for automating the mapping process.
c)~Our approach additionally delivers confidence bounds on its predictions, so that human annotation can be employed when needed.
d)~We test our approach in a real-life market research scenario. We also compare it with available techniques \cite{christen2008febrl} and demonstrate that our approach outperforms FEBRL \cite{christen2008febrl}.
e) We illustrate how our approach has been integrated into a data-fusion platform\cite{singh2016visual} specifically designed to manage data-lakes containing disparate databases.

\textbf{Related Work:} Record linkage has been usually addressed via two categories of approaches, learning-based and non-learning based~\cite{kopcke2010evaluation}. Learning-based approaches such as FEBRL\cite{christen2008febrl} that uses SVM to learn a weighted combination
of similarity matching techniques followed by unsupervised matching, MARLIN~\cite{bilenko2003adaptive} uses similarity measures Edit Distance and Cosine and several learners. In non-learning based approaches, PPJoin+\cite{xiao2011efficient} is a single-attribute match approach using sophisticated filtering techniques for improved efficiency, and FellegiSunter\cite{fellegi1969theory} evaluates three of the similarity measures Winkler, Tokenset, Trigram. In\cite{poon2016ensemble}, an ensemble approach of two non-learning algorithms Fellegi-Sunter and Jaro-Wrinkler has been presented for record-linkage. In contrast, we use a confidence based ensemble approach that combines supervised learning using a Bayesian network model together with a non-learning based textual model.
Our approach also produces confidence bounds on the predictions that help to decide reliability of prediction.

\begin{figure}
\centering
\includegraphics[width=120mm]{Figures/Prob-combine}
%\vspace{-2pt}
\caption{Local and Global Database}
%\vspace{-16pt}
\label{fig:Prob1}
\end{figure}

% \begin{figure}
% \centering
% \includegraphics[width=65mm]{Figures/Prob-2}
% %\vspace{-8pt}
% \caption{$K$ record matching problems}
% %\vspace{-12pt}
% \label{fig:Prob2}
% \end{figure}




% \subsection{Paper Organization}
% The remainder of the paper is organized as follows: We begin with overview of our approach in Section~\ref{sec:over} followed by the detail description of BGM model in Section~\ref{sec:BGM}, where we present Bayesian approach to predict global characteristics, which includes tree-based structure learning, Bayesian parameter learning, and Bayesian inference using SQL databases. The TIR model to predict global characteristics using textual descriptions is presented in Section~\ref{sec:tir}. Next, we present an ensemble approach of these two models in Section~\ref{sec:ensemble}, where we calculate confidence of the prediction from each model and use it to combine predictions from both models. Results of our techniques and integrated approach using real world 
% data are presented in Section~\ref{sec:experiment}. Finally, after the brief description of related work in Section~\ref{sec:related}, we conclude in Section~\ref{sec:conc} highlighting the prevalence of the problem we have addressed.\label{sec:intro}
\section{Approach}
Each product $l$ in $L$ has two kind of information (1) $M$ Local characteristics and (2) Textual descriptions by retailers. In this section, we present our approach to predict the value of global characteristic $G_j$ for each product in $L$. We use two different models for two different datasets (1) Supervised Bayesian Model (SBM) using local characteristics, and (2) Unsupervised Textual Similarity (UTS) using descriptions to compute probability of every possible state $g_{j,t}, t = 1, 2, ..., m j$ of $G_j$. Finally, we use an weighted ensemble based approach to combine the probabilities of both models to predict the value of $G_j$.
\subsection{Supervised Bayesian Model}\label{sec:BGM}
Approach to build SBM comprises of:(1)~Network Structure Learning, (2)~Parameter Learning, \& (3)~Bayesian Inference. For structure learning, we propose a novel technique of learning Tree based Bayesian Networks(TBN), whereas for parameter learning and Bayesian inference, we use the idea of \cite{yadav2015business} that performs probabilistic queries using SQL queries on the database of conditional probability tables.

\textbf{TBN Structure Learning:}\label{sec:struct}
Bayesian networks are associated with parameters known as conditional probability tables (CPT), where a CPT of a node indicates the probability that each value of a node can take given all combinations of values of its parent nodes. In CPTs, the number of bins grows exponentially as the number of parents increases leaving fewer data instances in each bin for estimating the parameters. Thus, sparser structures often provide better estimation of the underlying distribution~\cite{koller2009probabilistic}. Also, if the number of states of each node becomes high and the learned model is complex, Bayesian inferencing becomes conceptually and computationally intractable~\cite{lam1994learning}.
Hence, tree-based structures can be useful for density estimation from limited data and in the presence of higher number of states for facilitating faster inferencing. We employ a greedy search, and score based approach for learning TBN structure.

Given the global characteristic $G_j$ and $M$ local characteristics, we find set of top $\eta$ most relevant local characteristics w.r.t. $G_j$ using mutual information. We denote these $\eta$ local characteristics by $Y^j(L)$. Further, we learn a \textit{Tree based Bayesian Network(TBN)} on random variables $X = \{X_r: r=1,2,...,\eta+1\}$, where each $X_r \in X$ is either local characteristic $L_i \in Y^j(L)$ or global characteristic $G_j$ 

Chow et al. in \cite{chow1968approximating} state that cross-entropy between the tree structures distributions and the actual underlying distribution is minimized when the structure is a maximum weight spanning tree(MST). So, in order to learn TBN structure, we first learn MST for the characteristics in the set $X$. We find the mutual information between each pair characteristics, denoted by $W(X_r,X_s)$. Further, we use the mutual information as the weight between each pair of characteristics and learn MST using Kruskal's algorithm.

\vspace{-15pt}
\begin{equation}
\scriptsize
\label{eq:cross_ent}
Total Weight(TW) = \sum_{r=1,Pa(X_{r}) \neq 0}^{\eta+1} W(X_{r},Pa(X_{r}))
\vspace{-5pt}
\end{equation}

By learning MST, order of search space of possible graphs is reduced to $2^{O(\eta)}$, from $2^{O((\eta)^2)}$. Using this MST we search for the directed graph with least cross-entropy, by flipping each edge directions sequentially to obtain $2^{\eta}$ directed graphs along with their corresponding TW calculated using Eq.~\ref{eq:cross_ent}. Graph with maximum TW (minimum cross-entropy)~\cite{lam1994learning} is chosen as the best graphical structure representative of underlying distribution.

\textbf{Parameter Learning and Inference:} To learn the parameters of Bayesian Network(CPTs), for every product $l$ in $L$ we compute the probabilities $p^l_{j,1}, p^l_{j,2},..., p^l_{j,m_j}$, for every state of $G_j$, given the observed values of local characteristics in the Bayesian network, using an approach described in\cite{yadav2015business}. Here, CPTs are learned from the data stored in RDBMS and all queries are also answered using SQL.
\subsection{Unsupervised Text Similarity}\label{sec:tir}
In this section, we present UTS approach to compute the probability $q^{l}_{j,t}$ of each possible state of the global characteristic $G_j$ using retailer descriptions.
Consider each product $l$ in $L$ has $r_l$ descriptions and %$d_{l,1}, d_{l,2}$, $..., d_{l,r_l}$. 
for each description $d_{l,r}$, where $r=1,2,..., r_l$, we find n-grams of adjacent words. Let $N_{l} = \{ n^l_{v}, v=1,2,...\}$ be the set of n-grams of all descriptions, where $f^l_{v}$ be the frequency of each $n^l_{v}$ defined as a ratio of the number of descriptions in which $n^l_{v}$ exists to the $r_l$.

For every state $g_{j,t}$ of $G_j$, we find the best matching n-gram from the set $N_l$ by calculating Jaro-Wrinkler distance between $g_{j,t}$ and every $n^l_v \in N_l$ and choose the n-gram, say $n^l_{v,t}$, with the maximum score $s^l_{j,t}$.  
Further, multiply the scores $s^l_{j,t}$ with the frequency of $n^l_{v,t}$ to get the new score i.e., $S^{l}_{j,t} = s^{l}_{j,t} \times f^s_{l,t}$.
Finally, we convert each score $S^{l}_{j,t}$ into the probability $q^{l}_{j,t}$ by using softmax scaling function.

\subsection{Ensemble of models}\label{sec:ensemble}
In ensemble approach, we first find confidence of each prediction in both the cases(SBM and UTS) and then use these confidence values as weights for weighted ensemble. Given the probability distribution \{$p^l_{j,t}: t=1, 2,..., m_j\}$ for the values of $G_j$ using SBM model, we find the confidence corresponds to each probability as
\vspace{-7pt}
\begin{equation}
\scriptsize
 C(p^l_{j,t}) = 1 -  \sqrt{\sum_{t^{'}=1}^{m_j}(p^l_{j,t^{'}} - h^l_{t^{'}}(t))^2}, t=1,2...,m_j
 \vspace{-8pt}
\label{eq:conf}
\end{equation}
where $h^l_{t^{'}}(t)$ is the ideal distribution, which is 1 when $t = t^{'}$ and 0 otherwise.
% \begin{equation}
% \scriptsize
%  q_j = \begin{cases}
%         1 & p^l_{j,t} \\
%         0 & Otherwise
%        \end{cases}
% \label{eq:ideal}       
% \end{equation}
Similarly, we can find the confidence $C(q^l_{j,t})$ of each probability $q^l_{j,t}$.

With the given probability dist. and the confidence values from both models, we take weighted linear sum of two probabilities to get the new probability distribution over the states of {\scriptsize $G_j$: $P^l_{j,t} = C(p^l_{j,t}) \times p^l_{j,t} + C(q^l_{j,t}) \times q^l_{j,t}$, $t = 1,2,..., m_j$}
and we choose the value of $G_j$ for maximum $P^l_{j,t}$.
%  \begin{equation}
%  \small
%   g_j = \{v^l_{j,t}:  max\{P^l_{j,t}: t =1 ,2,...,m_j\} \}
%  \end{equation}

\textbf{CoP}: For every prediction, we assign the confidence value called confidence of prediction (CoP). CoP is a measure that helps to decide whether the predicted value is trustworthy or not. Given the probability distribution \{$P^l_{j,t}: l=1,2,...,m_j$\} for the values of $g_j$, we calculate the CoP of the predicted value $g^l_{j,t}$ of $G_j$ by using Equation~\ref{eq:conf}.


\section{Experiments and results}\label{sec:experiment}
%!TEX root = ../wbi.tex
\section{Experiments}
\label{sec:experiments}

This section presents the implementation of the PD plus gravity compensation controllers briefly described in Section \ref{sub:control_examples}. 
We also discuss the results of a more complex controller, namely a momentum-based balancing controller which has been implemented with the Simulink interface described in Section~\ref{sub:simulink_interface}.

\subsection{PD plus Gravity Compensation} % (fold)
\label{sub:pd_plus_gravity_compensation}

This section reports the code for the  example presented in Section~\ref{sub:control_examples}, i.e. the code for the PD plus gravity compensation controller.

Because it is a simple example we show both the C++ code (see Code~\ref{code:wbi_init} and \ref{code:cpp_pd_plus_grav}) and the Simulink model diagram (see Figure~\ref{fig:figs_PD_plus_grav_simulink}).
Note that, while the Simulink diagram completely represents the controller, the C++ code snippet has been extracted from the main loop function, i.e. the function which runs at every iteration. 
How the control thread is created and managed depends on the particular system and it is outside the scope of the present paper.

The snippet of code in Code~\ref{code:wbi_init} shows how the specific YARP-based implementation is instantiated. 
In particular, the current implementation needs information about the URDF model representing the kinematic and dynamic information of the robot and the mapping between the model joints and the YARP control boards. This is provided by the object created at line $4$ and passed to the interface constructor at line $7$.
Additionally, the list of controlled joints are passed to the interface at line $19$, just before the interface initialization routine is called.

Reading the code in Code~\ref{code:cpp_pd_plus_grav} it is possible to observe how all the details regarding the specific robot platform are hidden by the library.
The object {\tt robot}, in fact, is accessed through its abstract type, as it can be also seen during its instantiation, i.e. in line $7$ of Code~\ref{code:wbi_init}.
In lines $4-7$ the state of the robot, i.e. $(q_j, \dot{q}_j)$, is read.
The feedforward term, corresponding to $G(q)$ is computed at lines $10 - 14$ where the last parameter is the resulting gravity compensation term.
Finally the error and the feedback term necessary to implement Eq.~\eqref{eq:pd_plus_grav_law} is computed in lines $17-22$. 
Because we did not use any specific mathematical library we explicitly computed the term $K_p \tilde{q}_j + K_d \dot{q}_j$ in the {\tt for} loop.
Finally, at line $25$ we send the torque command to the robot, which we previously setup to be controlled in torque mode.

\begin{algorithm}
    \centering
\begin{lstlisting}
//Properties.
// - Fill with model URDF path
// - Yarp controlboard mapping
yarp::os::Property wbiProperties = ...;

//create an instance of wbi
wbi::wholeBodyInterface* m_robot =
new yarpWbi::yarpWholeBodyInterface(
    "PD plus gravity", 
    wbiProperties);
    
if (!m_robot) {
    return false;
}

//Create list of controllable joints
wbi::IDList controlledJoints = ...;

m_robot->addJoints(controlledJoints);
if (!m_robot->init()) {
    return false;
}

\end{lstlisting}
\caption{C++ code snippet for library initialization}
\label{code:wbi_init}
\end{algorithm}

\begin{algorithm}
    \centering
\begin{lstlisting}
  wbi::Frame w_H_b; //identity + zero vector
  
  //read state
  (*@\textcolor{wbi_position}{robot->getEstimates(wbi::ESTIMATE\_JOINT\_POS,}@*)
                      (*@\textcolor{wbi_position}{positions);}@*)
  (*@\textcolor{wbi_velocity}{robot->getEstimates(wbi::ESTIMATE\_JOINT\_VEL,}@*)
                      (*@\textcolor{wbi_velocity}{velocities);}@*)
  
  //use model to compute feedforward
  robot->computeGravityBiasForces(
                      positions, 
                      w_H_b, 
                      grav,
                      gravityCompensation);
  
  //compute feedback.
  for (int i = 0; i < robot->getDoFs(); i++) {
      error(i)   = positions(i) - reference(i);
      torques(i) = gravityCompensation(i + 6) 
                 - kp(i) * error(i) 
                 - kd(i) * velocities(i);
  }
  
  //send desired torques to the robot
  (*@\textcolor{wbi_torque}{robot->setControlReference(torques);}@*)

\end{lstlisting}
\caption{C++ code for PD plus Gravity compensation}
\label{code:cpp_pd_plus_grav}
\end{algorithm}

Figure~\ref{fig:figs_PD_plus_grav_simulink} shows the same code implemented directly in Simulink.
It is evident how the block-based diagram is clearer with respect to its C++ counterpart.
Furthermore, the possibility to add scopes, or dump signal variables directly into Matlab workspace greatly increases its advantages with respect to directly coding in C++.

\begin{figure*}[t]
  \centering
  	\def\svgwidth{\textwidth}
    \import{figs/}{PD_plus_grav.pdf_tex}
  \caption{Simulink model diagram of the PD plus gravity compensation controller for a fixed-base robot}
  \label{fig:figs_PD_plus_grav_simulink}
\end{figure*}

% subsection pd_plus_gravity_compensation (end)

\subsection{Momentum-based Balance Control} % (fold)
\label{sub:subsection_name}

To show the power of the proposed architecture we present here a second example, i.e. we show the results of a momentum-based balancing controller which has been synthesized directly by using the Simulink interface. 
Given the complexity of the control problem we do not report here screenshots or code snippets of the Simulink model, but the model can be examined in \cite{WBTController}, while the mathematical formulation can be found in \cite{nava16}.

The YouTube\textsuperscript{\textcopyright} video \cite{iCubWithSim} shows the robot performing complex movements by using the controller implemented and running as a Simulink Model.
By using the \emph{yarpWholeBodyInterface} implementation we also leverage the capabilities of the YARP middleware to seamlessly connect to the real or simulated system.
In particular the test platform is the iCub humanoid robot \cite{Metta20101125}, endowed with $53$ degrees of freedom, 6-axis force/torque sensors and distributed tactile skin.
The robot is simulated on the Gazebo simulator \cite{Koenig04} by means of Gazebo-YARP plugins \cite{YarpGazebo2014}.
The same demo has also been implemented on a different configuration of the iCub platform \cite{UtubeHeiCub}.
Note that the two robots have a different set of degrees of freedom.
Thanks to the flexibility of the library, the controller code remains the same in both scenarios.

We encourage the interested reader to test the controller on the Gazebo Simulator. 
Instructions on how to run the controller can be found directly in the model repository {\tt readme} \cite{WBTController}.
% subsection subsection_name (end)
%\section{Related Work}\label{sec:related}
%Record linkage of entities across disparate datasets is a widely explored \cite{brizan2015survey}, which has been applied in wide variety of domains like environmental hazards\cite{acheson1979record}, drug safety\cite{noren2005hit}, and to different types of data, including text\cite{li2005semantic} and images~\cite{huang1998object}. While record linkage addresses the problem of extracting, matching and resolving entities in structured and unstructured data\cite{getoor2012entity} across disparate datasets, with or without join keys, our problem address different aspect of record linkage \textit{where disparate databases need to be fused in the absence of natural join key}. 

Record linkage problem has been addressed generally via two category of approaches, learning based and non-learning based~\cite{kopcke2010evaluation}. Learning based approaches include FEBRL\cite{christen2008febrl} which uses support vector machine (SVM) for learning suitable matcher combinations, and MARLIN(Multiply Adaptive Record Linkage with
Induction)\cite{bilenko2003adaptive} which uses two string similarity measures (Edit Distance and Cosine) and several learners, specifically SVM and decision trees. In non-learning based approaches, PPJoin+\cite{xiao2011efficient} is a single-attribute match approach (similarity
join) using sophisticated filtering techniques for improved efficiency, and FellegiSunter\cite{fellegi1969theory} evaluates three of the similarity measures provided by FEBRL (Winkler, Tokenset, Trigram) and has an lower and upper similarity threshold that can be adjusted. In\cite{poon2016ensemble}, an ensemble approach of two non-learning algorithms Fellegi-Sunter (FS) and Jaro-Wrinkler (JW) has been presented for record-linkage. In contrast, we use confidence based ensemble approach that combines learning based Bayesian model and a non-learning based textual model. Our approach also produces confidence bound on the predictions that help to decide reliability of prediction.

Bayesian Networks are used for modeling beliefs in various domains like bioinformatics\cite{friedman2000using}, medicine\cite{uebersax2004genetic}, manufacturing\cite{singh2015predictive}. While traditional approximate inference techniques for Bayesian graphical modeling are able to deal with larger networks, they are usually restricted to models with low cardinalities of attributes. In our approach of BGM, we handle high cardinality attributes by introducing a novel approach of learning restricted Tree based Bayesian network, which facilitates faster (exact) inferencing.  Our work in BGM is closest to\cite{yadav2015business}, which presents an approach to compute distributional queries by approximating the underlying joint distribution via a Bayesian network. In\cite{yadav2015business}, SQL database has been used for Bayesian  inferencing under the assumption of simple networks, which are learned entirely using domain knowledge. In our work of BGM, we present end to end approach of Bayesian graphical modeling which learns simple tree based structure followed by exact Bayesian inferencing accelerated by an SQL engine to predict global characteristics.

\section{Conclusion}\label{sec:conc}
%!TEX root = ../wbi.tex
\section{Conclusions}
\label{sec:conclusions}

In this paper we presented a software abstraction layer to simplify the development of whole-body controllers.
While there are already some whole-body control software libraries, they already define the controller structure and leave to the user only the possibility to specify objectives and constraints.

On the other hand the proposed library leaves complete freedom to the control designer by exposing all the information needed. It does not make any assumptions on the controller structure.
The whole-body abstraction library presents also the following advantages:
\begin{itemize}
    \item it decouples the writing of the controller from a particular robot implementation
    \item it decouples the writing of the controller from a specific dynamic library implementation
    \item it allows more concise and clear code as it represents uniquely the code needed to implement the mathematical formulation of the controller. All the implementation details are left to the library
    \item it allows to benchmark the controller on different platforms or with different implementations.
\end{itemize}
Furthermore, the possibility to expose the functionality at an higher level than C++ facilitates the writing of controllers as the results on the iCub robot clearly prove.

We voluntarily did not consider some aspects as they are out of the scope of the present contribution. 
Nevertheless they must be taken into account when a controller is implemented and used on the real system.
In particular the following details should be considered:
\begin{itemize}
    \item how are controllers run on the platform? Do they run as threads?
    \item how are controllers configured and initialized?
    \item how is communication with other software performed? For example, how are desired values provided to the controller, coming from a planner or higher-level control loop?
\end{itemize}
By not considering these details in the abstraction library, we render the library portable to different systems.
Indeed, the actual control law is not concerned by the previously listed implementation details.

While the more complex demos have been achieved by directly executing the Simulink model connected to the robot, we recognize the need to automatically generate self-contained C++ code.
The advantage is twofold.
On one side the autogenerated code is in general more optimized than the code directly executed in Simulink, even if less optimized than ad-hoc C++ code.
On the other side, this would remove the requirement of having a Simulink installation on the computers controlling the robot.


\bibliographystyle{splncs03}
%{\scriptsize
\bibliography{PAKDD_bib}
\end{document}
