\section{Conclusions}
\label{sect:conclusions}

We have investigated the relationship between the galaxy star formation rate
(SFR) and the black hole accretion rate (BHAR) of the central black hole (BH)
using the \eagle cosmological hydrodynamical simulation. Our main conclusions
are as follows: 

\begin{itemize}

\item We compared \eagle predictions to two recent observational studies in
\cref{fig:stanley2015,fig:delvecchio2015}. The simulation reproduces both the
flat trend of the mean SFR (\av{SFR}) as a function of BHAR found in the AGN
selected study of \citet{Stanley2015} and the approximately linear trend of the
mean BHAR (\av{BHAR}) as a function of SFR found in the SFR selected study of
\citet{Delvecchio2015}. 

\item There is a moderate difference in the \av{SFR}--BHAR relationship when
time-averaging each growth rate over a 100~Myr period for an AGN selected study
(\cref{fig:avandinst}).  However, this change was not found to be sufficient as
to revert the trend to an underlying linear relationship as has been proposed by
previous theoretical studies.

\item Examining the complete $z=1$ SFR--BHAR plane in
\cref{fig:sfr_vs_bhar_z1}, we found no evidence for a simple universal global
relationship between the two instantaneous growth rates. The difference between
the trends found for the \av{SFR}--BHAR and \av{BHAR}--SFR relations from AGN
and SFR selections respectively, is due to sampling different regions of this
complex plane. The complexity of this plane results from both the rate of
galactic star formation and the accretion rate of the central BH holding an
evolving connection to the host dark matter halo
(\cref{fig:sfr_vs_bhar_byhalo}).

\item For a discrete redshift, the characteristic SFR of a halo increases
smoothly with increasing halo mass (\cref{fig:sfr_vs_bhar_byhalo}). BHs in
haloes of mass \M{200} $\lesssim 10^{11.5}$\Msol accrete at a \squotes{low}
rate (\BHAR $< 10^{-4}$\Msolyr).  They then transition through haloes of mass
$10^{11.5} \sim 10^{12.5}$\Msol to a \squotes{high} rate (\BHAR $>
10^{-4}$\Msolyr) in haloes of mass \M{200} $\gtrsim 10^{12.5}$\Msol. However,
the scatter in the BHAR at fixed halo mass is very large (up to $\sim 6$~dex).
Galaxies with SFRs far below the characteristic SFR all contain massive BHs
(\M{BH} $\geq 10^{7}$\Msol).  

\item The median evolutionary trend for a galaxy's SFR and the accretion rate
of its central BH, averaged over 100~Myr, are insensitive to the final
properties of the system (\cref{fig:avHistory_vs_hm}). By equating these trends
together we found that the 100~Myr average SFR as a function of the 100~Myr
average BHAR can be split into three regimes, separated by the halo mass
(\cref{fig:sfr_vs_bhar_av}). BHs hosted by haloes below the characteristic
transition mass, \M{crit} \citep[][\M{200} $\sim 10^{12}$\Msol]{Bower2017},
fail to grow effectively, yet the galaxy continues to grow with the halo. Once
the halo reaches \M{crit} there is a non-linear \squotes{switch} of BH growth
that rapidly builds the mass of the BH. In the most massive haloes (\M{200} >
\M{crit}) both SFR and BHAR decline on average, with a roughly constant scaling
of SFR/BHAR $\sim 10^{3}$.

\end{itemize}

\section*{Acknowledgements}

We thank the referee Marta Volonteri for her useful comments, also to Flora
Stanley and Ivan Delvecchio for their data and useful conversations. 

This work was supported by the Science and Technology Facilities Council (grant
number ST/F001166/1 and ST/L00075X/1); European Research Council (grant numbers
GA 267291 \dquotes{Cosmiway} and GA 278594 \dquotes{GasAroundGalaxies}) and by
the Interuniversity Attraction Poles Programme initiated by the Belgian Science
Policy Office (AP P7/08 CHARM). RAC is a Royal Society University Research
Fellow.

This work used the DiRAC Data Centric system at Durham University, operated by
the Institute for Computational Cosmology on behalf of the STFC DiRAC HPC
Facility (\url{http://www.dirac.ac.uk}). This equipment was funded by BIS
National E-infrastructure capital grant ST/K00042X/1, STFC capital grant
ST/H008519/1, and STFC DiRAC Operations grant ST/K003267/1 and Durham
University. DiRAC is part of the National E-Infrastructure. We acknowledge
PRACE for awarding us access to the Curie machine based in France at TGCC, CEA,
Bruy\`{e}res-le-Ch\^{a}tel.

This work was supported by the Netherlands Organisation for Scientific Research
(NWO), through VICI grant 639.043.409, and the European Research Council under
the European Union's Seventh Framework Programme (FP7/2007- 2013) / ERC Grant
agreement 278594-GasAroundGalaxies.
