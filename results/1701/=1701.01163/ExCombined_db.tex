\section{Examples from maps onto a product of two elliptic curves}
\label{secExCombined}
Note that the one-dimensional approach pursued in \cite{Llo-16-II} can not be used to obtain irreducible examples with finiteness properties as in Theorem \ref{corthmIntroA}. Indeed, choose any holomorphic map $h: S_{g_1}\times \dots \times S_{g_r}\to E$ and a base point $(z_1,\dots, z_r)\in S_{g_1}\times \dots \times S_{g_r}$. Then, for $2\leq m\leq r$, the restriction $h|_{\left\{(z_1,\dots,z_{m-1})\right\} \times S_{g_m}\times \left\{(z_{m+1},\dots,z_r)\right\}}$ is either trivial or surjective and thus the image $h_{\ast}(1\times \dots \times 1\times  \pi_1 S_{g_m}\times 1 \times \dots \times 1)\leq \pi_1 E$ is either trivial or a finite index subgroup. Therefore, if $h$ induces a short exact sequence on fundamental groups, then $\ker h_{\ast}$ is a product of finitely many surface groups and a group of the form constructed in \cite[Theorem 1.1]{Llo-16-II}. In particular, it is not irreducible unless it is of type $\mathcal{F}_{r-1}$ and not of type $\mathcal{F}_r$.

However, we can obtain irreducible examples covering all possible finiteness types using holomorphic maps to a product of two elliptic curves. This is achieved by combining the construction in Section \ref{secExamples} with Theorem \ref{thmFiltVerGen} and Remark \ref{rmkFiltVerGenWeak}. 

As before, we choose elements $v_1,\dots, v_r \in \ZZ^2$, as well as ramified covers $\alpha_i: S_{\g_i}\to E$ of an elliptic curve $E$ and define a holomorphic map 
\begin{equation}
h=\sum_{i=1}^r v_i\cdot \alpha_i : S_{\g_1}\times \dots \times S_{\g_r}\to E^{\times 2}.
\label{eqdefh}
\end{equation}
However, we will vary our choices of the $v_i$ to extend the range of finiteness properties of our examples.

Let $r\geq 4$, $m\geq 1$ and $r-m\geq 3$. Choose $v_1,\dots, v_m=\left( \begin{array}{c} 1\\ 0 \end{array} \right)$, $v_{m+1}=\left(\begin{array}{c} 0\\ 1 \end{array}\right)$ and $\left\{ v_{m+2}, \dots, v_r \right\}\subset \ZZ^2$ such that $\left\{ v_i, v_j\right\}$ are linearly independent for $m\leq i < j \leq r$. 

For the remainder of this section we denote $\mathcal{C}=\left\{v_1,\dots,v_r\right\}$ and $h$ the holomorphic map in \eqref{eqdefh} defined by this choice of $\mathcal{C}$. The main result of this section is:

\begin{theorem}
 Let $\mathcal{C} \subset \ZZ^2$ and $h$ be as defined above with $r\geq 4$, $m\geq 1$ and $r-m\geq 3$. Assume that $\alpha_m$ and $\alpha_{m+1}$ are surjective on fundamental groups. Then the smooth generic fibre $\overline{H}$ of $h$ is connected and its fundamental group fits into a short exact sequence
 \[
  1\to \pi_1 \overline{H} \to \pi_1 S_{\g_1}\times \dots \times S_{\g_r}\stackrel{h_{\ast}}{\to} \pi_1 E^{\times 2} = \ZZ^4 \to 1.
 \]
 Furthermore, $\pi_1 \overline{H}\leq \pi_1 S_{\g_1}\times \dots \times S_{\g_r}$ is an irreducible K\"ahler subgroup of type $\mathcal{F}_{r-m-1}$ but not of type $\mathcal{F}_{r-m}$. Moreover, $\pi_i \overline{H} =\left\{0\right\}$ for $2\leq i \leq r-m-2$.
 \label{thmExtendedRange}
\end{theorem}


As a consequence we obtain an alternative proof of Theorem \ref{corthmIntroA}. More precisely, we have:
\begin{corollary}
 For every $r\geq 4$, $r-1\geq k\geq 2$ and $\g_1,\dots,\g_r\geq 2$ there is an irreducible full subdirect K\"ahler subgroup $G\leq \pi_1 S_{\g_1}\times \dots \times \pi_1 S_{\g_r}$ of type $\mathcal{F}_k$ but not of type $\mathcal{F}_{k+1}$, which is the kernel of a homomorphism $\pi_1 S_{\g_1}\times \dots \times \pi_1 S_{\g_r}\to \ZZ^4$.
 \label{corExtendedRange}
\end{corollary}
\begin{proof}
We may choose $v_{m+1}=\left(\begin{array}{c}1\\1 \end{array}\right)$. Note that then the subgroups constructed in Theorem \ref{thmExtendedRange} are full subdirect, by the conditions imposed on $\mathcal{C}$ combined with the fact that the kernel of the induced map $\alpha_{i,\ast}: \pi_1 S_{\g_i} \to \pi_1 E = \ZZ^2$ is non-trivial for $1\leq i \leq r$ and their image is a finite index subgroup of $\ZZ^2$. 

For any $g\geq 2$ there is a branched covering $S_g\to E$ inducing an epimorphism on fundamental groups. Thus, Theorem \ref{thmExtendedRange} covers all $k<r-1$. The case $k=r-1$ is immediate from \cite[Theorem 1.1]{Llo-16-II}. 
\end{proof}

The proof of Theorem \ref{thmExtendedRange} is very similar to the proof of Theorem \ref{thmExsGenClass}. As in Section \ref{secExamples} we denote by $\pi_l : E^{\times 2}\to \left\{ 0 \right\} \times E^{\times l}$ the projection onto the last $l$ factors and by $h_l=\pi_l\circ h$ the composition. The map $h_l$ factors as $h_l = f_l \circ g_l$ for $1\leq l \leq 2$ with
\[
 f_2=\left(v_1\mid \dots \mid v_r \right) \circ \left(\alpha_1, \dots, \alpha_r \right) : S_{\g_1}\times \dots \times S_{\g_r}\to  E^{\times 2},
\]
\[
 g_2= S_{\g_1}\times \dots \times S_{\g_r} \to S_{\g_{1}}\times \dots \times S_{\g_r},
\]
\[
 f_1=\pi_1 \circ \left(v_{m+1}\mid \dots \mid v_r \right) \circ \left(\alpha_{m+1}, \dots, \alpha_r \right) : S_{\g_{m+1}}\times \dots \times S_{\g_r}\to  \left\{0\right\} \times E,
\]
and
\[
 g_1= S_{\g_1}\times \dots \times S_{\g_r} \to S_{\g_{m+1}}\times \dots \times S_{\g_r},
\]
where $g_1$ is the canonical projection with fibre $F_1=S_{\g_1}\times \dots \times S_{\g_m}$ a product of closed hyperbolic surfaces.

\begin{lemma}
\label{lemConnGenExsExt}
 Under the assumptions of Theorem \ref{thmExtendedRange}, the maps $h_i$, $f_i$ and $g_i$, $i=1,2$, have connected fibres.
\end{lemma}
\begin{proof}
 Since the maps $\alpha_m$ and $\alpha_{m+1}$ are surjective on fundamental groups, we see that by choice of $\mathcal{C}$, $r$ and $m$, the same argument as in the proof of Lemma \ref{propConnGenExs} shows connectedness of fibres.
\end{proof}



\begin{lemma}
 The map $f_1$ and the corestriction $f_2|_{f_2^{-1}(E\times \left\{e\right\})} : f_2^{-1}(E\times \left\{e\right\}) \to E\times \left\{ e \right\}$ have isolated singularities for all $e\in E$. 
 \label{lemIsSingExt}
\end{lemma}

Note that the map $f_2$ does not have isolated singularities. However, by Remark \ref{rmkFiltVerGenWeak} it suffices to show that the corestriction $f_2|_{f_2^{-1}(E\times \left\{e\right\})} : f_2^{-1}(E\times \left\{e\right\}) \to E\times \left\{ e \right\}$ has isolated singularities for all $e\in E$ to obtain the consequences of Theorem \ref{thmFiltVerGen}.

\begin{proof}
 For the map $f_1$ this follows by the same argument as in the proof of Proposition \ref{propIsolSingGen}. 
 
 The differential of $f_2$ is 
\begin{align*}
 Df_2 &= \left(v_1\cdot D\alpha_1 \dots v_r\cdot D\alpha_r\right)\\
      &= \left( \left( \begin{array}{c} 1\\ 0 \end{array}\right)\cdot D\alpha_1 \cdots \left(\begin{array}{c} 1\\ 0 \end{array} \right)\cdot D\alpha_m ~ v_{m+1} \cdot D\alpha_{m+1}\cdots v_r \cdot D \alpha_r\right).
\end{align*}

Due to our assumptions on $\mathcal{C}$, a point $z=(z_1,\dots,z_r)\in S_{\g_1}\times \dots \times S_{\g_r}$ is a critical point of $f_2$ if and only if at least one of the following holds
\begin{enumerate}
 \item $D\alpha_{m+1} (z_{m+1})=\dots = D\alpha_r (z_r)=0$;
 \item $D\alpha_1 (z_1)=\dots = D\alpha_m(z_m)= 0$ and at least $r-m-1$ of the $D\alpha_i(z_i)$ vanish for $i\geq m+1$.
\end{enumerate}

The locus of points satisfying (2) is a union of finitely many one-dimensional subvarieties of $S_{\g_1} \times \dots \times S_{\g_r}$ and it intersects each fibre of $f_2$ in at most finitely many points. Hence, these singularities are isolated for $f_2$ and thus for $f_2|_{f_2^{-1}(E\times \left\{e\right\})}$.

Observe that for the points $z\in f_2^{-1}(E\times \left\{e\right\})$ satisfying (1) but not (2), there exists at least one $i$ with $1\leq i\leq m$ such that $D\alpha_i(z_i)\neq 0$. The composition of $f_2|_{\left\{(z_1,\dots,z_{i-1})\right\} \times S_{\g_i} \times \left\{ (z_{i+1},\dots, z_r\right\}}$ with the projection onto the second factor of $E^{\times 2}$ is constant. Thus, $\left\{(z_1,\dots,z_{i-1})\right\} \times S_{\g_i} \times \left\{ (z_{i+1},\dots, z_r\right\} \subset f_2^{-1}(E\times \left\{e\right\})$ . In particular, $D\alpha_i(z_i)\neq 0$ implies that $D(f_2|_{f_2^{-1}(E\times \left\{e\right\})})(z)\neq 0$. 

Since $E\times \left\{e\right\}$ is one-dimensional, $z$ is not a critical point of the restriction $f_2|_{f_2^{-1}(E\times \left\{e\right\})}$. Hence, all critical points of $f_2|_{f_2^{-1}(E\times \left\{e\right\})}$ satisfy (2). It follows that $f_2|_{f_2^{-1}(E\times \left\{e\right\})}$ has isolated singularities.
\end{proof}

\begin{proof}[Proof of Theorem \ref{thmExtendedRange}]
By Lemma \ref{lemConnGenExsExt}, Lemma \ref{lemIsSingExt}, Remark \ref{rmkFiltVerGenWeak}, and the fact that $F_1$ is a direct product of surface groups, we can apply Theorem \ref{thmFiltVerGen}. We obtain that $h$ induces a short exact sequence
\[
 1\rightarrow \pi_1 \overline{H} \rightarrow \pi_1 S_{\g_1}\times \dots \times \pi_1 S_{\g_r}\stackrel{h_{\ast}}{\rightarrow} \pi_1 E^{\times 2}=\ZZ^4\rightarrow 1
\]
and $\pi_i \overline{H}=0$ for $2\leq i \leq r-m-2$. Hence, as in the proof of Theorem \ref{thmExsGenClass}, $\pi_1 \overline{H}$ is K\"ahler (and in fact projective) of type $\mathcal{F}_{r-m-1}$.

By definition, we obtain that $\phi=h_{\ast}$ is given by the surjective map
 \[
  h_{\ast}(g_1,\cdots,g_r)=\phi(g_1,\cdots,g_r)=\sum_{i=1}^r v_i\cdot \alpha_i(g_i) 
\in \pi_1 E^{\times 2} = (\ZZ^2)^2\cong \ZZ^4
\]
for $(g_1,\cdots,g_r)\in \pi_1 S_{\g_1}\times \cdots \times \pi_1 S_{\g_r}$. Assume that there is a finite index subgroup $H_1 \times H_2 \leq \ker \phi$. 

Since $\ker \phi$ is a full subdirect product (after possibly passing to finite index subgroups of the $\pi_1 S_{\g_i}$), we obtain that there is a partition of $\left\{1,\dots, r\right\}$ into two sets $\left\{i_1,\dots,i_t\right\}$ and $\left\{ i_{t+1},\dots, i_r\right\}$ such that $H_1 \leq \pi_1 S_{\g_{i_1}}\times \dots \times \pi_1 S_{\g_{i_t}}$ and $H_2 \leq \pi_1 S_{\g_{i_{t+1}}}\times \dots \times S_{\g_r}$ (because non-trivial elements of surface groups have cyclic centralizers).

Thus, for any element $(x_{i_1},\dots,x_{i_r})\in H_1\times H_2\leq \Gamma_{\g_{i_1}}\times \dots \times \Gamma_{\g_{i_r}}$, we have $(x_{i_1},\dots,x_{i_t},1,\dots,1)\in H_1\times 1 \leq \ker \phi$ and $(1,\dots,1,x_{i_{t+1}},\dots,x_{i_r})\leq 1 \times H_2 \leq \ker \phi$. By our assumptions on $\mathcal{C}$ at least one of $\left\{v_{i_1},\dots,v_{i_t}\right\}$ and $\left\{v_{i_{t+1}},\dots,v_{i_r}\right\}$ contains a pair of linearly independent vectors, say $\left\{v_{i_{t+1}},v_{i_{t+2}}\right\}$ are linearly independent.

Let $A\leq \ZZ^4 = (\ZZ^2)^2$ be the finite index subgroup with $A= \mathrm{Im} \left( v_{i_{t+1}}\cdot\alpha_{i_{t+1},\ast} + v_{i_{t+2}}\cdot \alpha_{i_{t+2},\ast}\right)$. Then for any element $x_{i_1}\in \Lambda_{i_1}$ of the finite index subgroup $\Lambda_{i_1}:=\left( v_{i_1} \cdot \alpha_{i,\ast}\right)^{-1}(A)\unlhd \Gamma_{\g_{i_1}}$ there is $(x_{i_{t+1}},x_{i_{t+2}})\in \Gamma_{\g_{i_{t+1}}}\times \Gamma_{\g_{i_{t+2}}}$ with $(x_{i_1},1,\dots,1,x_{i_{t+1}},x_{i_{t+2}},1,\dots,1)\in \ker \phi$. Hence, the intersection $\ker \phi \cap \left(\Gamma_{\g_{i_1}}\times \Gamma_{\g_{i_{t+1}}}\times \Gamma_{\g_{i_{t+2}}}\right)$ projects to a finite index subgroup of $\Gamma_{\g_{i_1}}$. Since $H_1\times H_2\leq \ker \phi$ is a finite index subgroup, the same is true for the projection of the intersection $\left(H_1\times H_2\right) \cap \left(\Gamma_{\g_{i_1}}\times \Gamma_{\g_{i_{t+1}}}\times \Gamma_{\g_{i_{t+2}}}\right)$ to $\Gamma_{\g_{i_1}}$. Thus, $H_1\times 1 $ contains a finite index subgroup of $\Gamma_{\g_{i_1}}$. This is impossible, because $\ker \alpha_{i_1,\ast}\leq \Gamma_{i_1}$ has infinite index. It follows that $\ker \phi$ is irreducible.

Finally observe that $v_1=\dots=v_m=\left(\begin{array}{c} 1\\ 0 \end{array}\right)$, implies that the image $\phi(\Gamma_{\g_1}\times \dots \times \Gamma_{\g_m})\cong \ZZ^2 \leq \ZZ^4=\mathrm{Im} \phi$ is not a finite index subgroup. Thus, the group $\ker \phi$ is not of type $\mathcal{F}_{r-m}$, by Corollary \ref{propCoNilpFm} below.
\end{proof}

\begin{remark}
 Note that one could also apply the converse direction of Theorem \ref{thmKochlVSPk} to obtain that the groups constructed in Theorem \ref{thmExsGenClass} and Theorem \ref{thmExtendedRange} have finiteness type $\mathcal{F}_{r-k}$ (respectively $\mathcal{F}_{r-m-1}$). However, the proof we give actually provides the stronger result that in fact classifying spaces for our examples can be constructed from compact K\"ahler manifolds by attaching only cells of dimension larger than $r-k$ (respectively $r-m-1$).
\end{remark}

\begin{remark}
\label{remNotIsom}
  By \cite[Theorem C(3)]{BriHowMilSho-13} a group $G$ which is a full subdirect product $G\leq \L_1\times \dots \times \L_r$ of non-abelian limit groups $\L_i$ uniquely determines the $\L_i$ (up to order). Assume that $G_1,G_2\leq G$ are two finite index subgroups which are full subdirect products $G_i\unlhd \L_{1,i}\times \dots \times \L_{r,i}$ with $\L_{j,i}\leq \L_j$, finite index subgroups such that $A_i=(\L_{1,i}\times \dots \times \L_{r,i})/G_i$ is free abelian, where $1\leq j\leq r$, $i=1,2$. It is not hard to see that then $A_1\cong A_2$. As a consequence we see that the examples constructed in this section are not isomorphic to the examples in the previous section for $m\geq 2$ (not even virtually).
\end{remark}




\begin{remark}
 The construction described in this section generalises to epimorphisms to $\ZZ^{2k}$ for any $k\geq 2$. This produces irreducible coabelian K\"ahler subgroups of a direct product of $r\geq 3$ surface groups of type $\mathcal{F}_m$, not $\mathcal{F}_{m+1}$ for $2\leq m \leq r-k$. The same arguments as in Remark \ref{remNotIsom} allow us to show that two irreducible groups obtained in this way are not isomorphic if they have different coabelian ranks, even if they have the same finiteness properties. For simplicity of notation, we restricted ourselves to the case $k=2$ in our explicit computations.
\end{remark}


