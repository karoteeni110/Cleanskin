\section{A new construction method}

\label{secMainThm}

Let $X$ and $Y$ be complex manifolds and let $f:X\rightarrow Y$ be a surjective holomorphic map. Recall that a sufficient condition for the map $f$ to have isolated singularities is that the set of singular points of $f$ intersects every fibre of $f$ in a discrete set.

Before we proceed we fix some notation: For a set $M$ and subsets $A,B\subset M$ we will denote by $A\setminus B$ the set theoretic difference of $A$ and $B$. If $M=T^n$ is an $n$-dimensional torus then we will denote by $A-B=\left\{a-b\mid a\in A, b\in B\right\}$ the group theoretic difference of $A$ and $B$ with respect to the additive group structure on $T^n$. We will be careful to distinguish $-$ from set theoretic $\setminus$.

In this section we generalise the following result of Dimca, Papadima and Suciu to maps onto higher-dimensional tori.

\begin{theorem}[{\cite[Theorem C]{DimPapSuc-09-II}} ]
Let $X$ be a compact complex manifold and let $Y$ be a closed Riemann surface of genus at least one. Let $f: X\rightarrow Y$ be a surjective holomorphic map with isolated singularities and connected fibres. Let $\widehat{f}: \widehat{X}\rightarrow \widetilde{Y}$ be the pull-back of $f$ under the universal cover $p:\widetilde{Y} \rightarrow Y$ and let $H$ be the smooth generic fibre of $\widehat{f}$ (and therefore of $f$).

Then the following hold:
\begin{enumerate}
 \item $\pi_i(\widehat{X},H)=0$ for $ i \leq \mathrm{dim}H$;
 \item if $\mathrm{dim} H \geq 2$, then  $1\rightarrow \pi_1 H\rightarrow \pi_1 X\overset{f_{\ast}}\rightarrow \pi_1 Y\rightarrow 1$ is exact.
\end{enumerate}
\label{thmC'}
\end{theorem}

\subsection{Conjecture}
\label{secConjThmC}



Having isolated singularities yields strong restrictions on the topology of the fibres near the singularities. We will only make indirect use of these restrictions here, by applying Theorem \ref{thmC'}. For background on isolated singularities see \cite{Loo-84}.

\begin{conjecture}
Let $X$ be a compact connected complex manifold of dimension $n+k$ and let $Y$ be a $k$-dimensional complex torus or a Riemann surface of positive genus. Let $h: X\rightarrow Y$ be a surjective holomorphic map with connected generic fibre.

Let further $\widehat{h}:\widehat{X}\rightarrow \widetilde{Y}$ be the pull-back fibration of $h$ under the universal cover $p: \widetilde{Y}\rightarrow Y$ and let $H$ be the generic smooth fibre of $\widehat{h}$, or equivalently of $h$.

Suppose that $h$ has only isolated singularities. Then the following hold:
\begin{enumerate}
 \item $\pi_i(\widehat{X},H)=0$ for all $i\leq \mathrm{dim} H$;
 \item if, moreover, $\mathrm{dim} H\geq 2$ then the induced homomorphism $h_{\ast}: \pi_1 X\rightarrow \pi_1 Y$ is surjective with kernel isomorphic to $\pi_1 H$.
\end{enumerate}
\label{conjgenThmC}
\end{conjecture}

Conjecture \ref{conjgenThmC} is a generalisation of Theorem \ref{thmC'} to higher dimensions. It can be seen as a Lefschetz type result, since it says that in low dimensions the homotopy groups of the subvariety $H\subset \widehat{X}$ of codimension $n\geq 2$ coincide with the homotopy groups of $H$. The most classical Lefschetz type theorem is the Lefschetz Hyperplane Theorem (see \cite{GorMac-88} for a detailed introduction to Lefschetz type theorems). 

A potential proof strategy for Conjecture \ref{conjgenThmC} is presented in \cite{Llo-PhD}, but there are technical difficulties which we were not able to overcome yet. Here we will prove Theorem \ref{thmFiltVer}, which is a special case of Conjecture \ref{conjgenThmC}.


In fact we will prove the more general Theorem \ref{thmFiltVerGen} from which Theorem \ref{thmFiltVer} follows immediately. 

\subsection{Fibrelong isolated singularities}
\label{secFibrelongSingBriLlo}

To prove Theorem \ref{thmFiltVerGen} we make use of a generalisation of Theorem \ref{thmC'} which relaxes the conditions on the singularities of $h$.
 
\begin{definition}
 Let $X$, $Y$ be compact complex manifolds. We say that a surjective map $h: X\rightarrow Y$ has \textit{fibrelong isolated singularities} if it factors as
 \[
  \xymatrix{ X \ar[r]^g \ar[rd]^h & Z\ar[d]^f \\ & Y\\}
 \]
 where $Z$ is a compact complex manifold, $g$ is a regular holomorphic fibration, and $f$ is holomorphic with isolated singularities.
 \label{defAIS}
\end{definition}

For holomorphic maps with connected fibrelong isolated singularities, Bridson and the author proved

\begin{theorem}[{\cite[Theorem 2.2]{BriLlo-16}}]
\label{thm2}
Let $Y$ be a closed Riemann surface of positive genus and let $X$ be a compact K\"ahler manifold. Let $h:X\rightarrow Y$ be a surjective holomorphic map with connected generic (smooth) fibre $\overline{H}$.

If $h$ has fibrelong isolated singularities, $g$ and $f$ are as in Definition \ref{defAIS}, and $f$ has connected fibres of dimension $m\geq 2$, then the sequence 

\[  
1 \rightarrow \pi_1 \overline{H}\rightarrow \pi_1 X \overset{h_{\ast}}\rightarrow \pi_1 Y\rightarrow 1
\]
is exact.
\end{theorem}


We will also need the following proposition. Note that the hypothesis on $\pi_2 Z\to \pi_1F$  is automatically
satisfied if $\pi_1F$ does not contain a non-trivial normal abelian subgroup. This is the case, for example,
if $F$ is a direct product of hyperbolic surfaces.

\begin{proposition}[{\cite[Proposition 2.3]{BriLlo-16}}]
\label{prop1part2}
Under the assumptions of Theorem \ref{thm2}, if the map $\pi_2 Z\to \pi_1F$ associated to the fibration $g:X\to Z$ is trivial, then the long exact sequence induced by the fibration $F\hookrightarrow \overline{H}\rightarrow H$ reduces to a short exact sequence
\[
1\rightarrow \pi_1 F\rightarrow \pi_1 \overline{H}\rightarrow \pi_1 H\rightarrow 1.
\]
If, in addition, the fibre $F$ is aspherical, then $\pi_ i \overline{H} \cong \pi_i H \cong \pi_i X$ for $2\leq i \leq m - 1$.
\end{proposition}

\subsection{Restrictions on $h:X\rightarrow Y$ for higher-dimensional tori}
\label{sec:Restrictions}

Let $X$ be a compact complex manifold and let $Y$ be a complex torus of dimension $k$. Let $h: X\rightarrow Y$ be a surjective holomorphic map. Assume that there is a filtration
\[
 \left\{0\right\} \subset Y^0\subset Y^1 \subset \cdots \subset Y^{k-1}\subset Y^k=Y
\]
of $Y$ by complex subtori $Y^l$ of dimension $l$, $0\leq l \leq k$. Let $\pi_l: Y\rightarrow Y/Y^{k-l}$ be the canonical holomorphic projection.

Assume that the maps $h$ and $h_l=\pi_l\circ h: X\rightarrow Y/Y^{k-l}$ have connected fibres and fibrelong isolated singularities. In particular, there are compact complex manifolds $Z_l$ such that $h_l$ factors as
\[
 \xymatrix{ X \ar[r]^{g_l}\ar[rd]_{h_l} & Z_l\ar[d]^{f_l} \\ & Y/Y^{k-l}}.
\]
with $g_l$ a regular holomorphic fibration and $f_l$ surjective holomorphic with isolated singularities and connected fibres. Assume further that the smooth compact fibre $F_l$ of $g_l$ is connected and aspherical. We denote by $\overline{H}_l$ the connected smooth generic fibre of $h_l$ and by $H_l$ the connected smooth generic fibre of $f_l$. 

For a generic point $x^0=\left(x_1^0,\cdots, x_k^0\right)\in Y$ we claim that $x^{0,l}=x^{0,k}+Y^{k-l}\in Y/Y^{k-l}$ is a regular value of $h_l$ for $0\leq l \leq k$:  For $1\leq l \leq k$ there is a proper subvariety $V^l \subset Y/Y^{k-l}$ such that the set of critical values of $h_l$ is contained in $V^l$; any choice of $x^0$ in the open dense subset $Y \setminus \left(\cup_{l=1}^k \pi_l^{-1}(V^l)\right) \subset Y$ satisfies the assertion.
 
 The smooth generic fibres $\overline{H}_l=h_l^{-1}(x^{0,l})$ of $h_l$ form a nested sequence
 \[
  \overline{H}= \overline{H}_k\subset \overline{H}_{k-1}\subset \cdots \subset \overline{H}_0=X.
 \]

Consider the corestriction of $h_l$ to the elliptic curve $x^{0,l}+ Y^{k-l+1}/Y^{k-l}\subset Y/Y^{k-l}$. The map 
\[
 h_l|_{\overline{H}_{l-1}}: h_l^{-1}\left(x^{0,l}+ Y^{k-l+1}/Y^{k-l}\right)= h^{-1}\left(x^{0,k}+Y^{k-l+1}\right)=\overline{H}_{l-1}\rightarrow x^{0,l}+Y^{k-l+1}/Y^{k-l}
\]
is holomorphic surjective with fibrelong isolated singularities and connected smooth generic fibre $\overline{H}_l=h_l^{-1}(x^{0,l}+Y^{k-l})$.

Assume that the induced map $\pi_2 \overline{H}_{l-1}\rightarrow \pi_1 F_l$ is trivial for $1\leq l\leq k$. Then the following result holds:

\begin{theorem}
 Assume that $h:X\rightarrow Y$ has all the properties described in Paragraph \ref{sec:Restrictions} and that $n:= \mathrm{min}_{1\leq l \leq k} \mathrm{dim} H_l\geq 2$. Then the map $h$ induces a short exact sequence
 \[
  1\rightarrow \pi_1 \overline{H}\rightarrow \pi_1 X \stackrel{h_{\ast}}{\rightarrow} \pi_1 Y \cong \ZZ^{2k} \rightarrow 1
 \] 
and $\pi_i(\overline{H})\cong \pi_i (X)$ for $2\leq i \leq n-1$.
\label{thmFiltVerGen}
\end{theorem}

Note that Theorem \ref{thmFiltVer} is the special case of Theorem \ref{thmFiltVerGen} with $Z_l=X$ and $g_l=id_X$ for $1\leq l \leq k$.


\begin{proof}[Proof of Theorem \ref{thmFiltVerGen}]
 The proof uses an inductive argument reducing the statement to an iterated application of Theorem \ref{thm2} and Proposition \ref{prop1part2}.
 
Since $\mathrm{dim} H_l\geq n\geq 2$, Theorem \ref{thm2} and Proposition \ref{prop1part2} imply the restriction $h_l|_{\overline{H}_{l-1}}$ induces a short exact sequence
\begin{equation}
1\rightarrow \pi_1 \overline{H}_l \rightarrow \pi_1 \overline{H}_{l-1}\stackrel{h_{l\ast}}{\rightarrow} \pi_1 \left(x^{0,l}+ Y^{k-l+1}/Y^{k-l}\right) = \ZZ^2\rightarrow 1
\label{eqnSES1}
\end{equation}
and that $ \pi_i(\overline{H}_{l-1})\cong  \pi_ i(\overline{H}_l)$ for $2\leq i \leq \mathrm{dim} H_l-1$, where $1\leq l \leq k$. In particular, we obtain that $\pi_ i(\overline{H}_{l-1}) \cong \pi_i(\overline{H}_l)$ for $2\leq i \leq n-1$.
 
 Hence, we are left to prove that the short exact sequences in \eqref{eqnSES1} induce a short exact sequence
\[
1\rightarrow \pi_1 \overline{H} \rightarrow \pi_1 X\rightarrow \pi_1 Y=\ZZ^{2k}\rightarrow 1.
\]

For this consider the commutative diagram of topological spaces
\[
\xymatrix{
\overline{H} \ar@{^{(}->}[r] & X=\overline{H}_0=h_{k}^{-1}\left(x^{0,0}+ Y^k\right) \ar@{->>}[r] ^-{h} & x^{0,0}+ Y^k\\
\overline{H} \ar@{^{(}->}[r] \ar[u]^{=} & \overline{H}_1=h^{-1}\left(x^{0,1}+ Y^{k-1}\right) \ar@{->>}[r]^-{h} \ar@{^{(}->}[u] & x^{0,1}+ Y^{k-1} \ar@{^{(}->}[u] \\
  \ar[u]^{=} &  \ar@{^{(}->}[u] & \ar@{^{(}->}[u] \\
 \vdots & \vdots &\vdots \\
\overline{H} \ar@{^{(}->}[r] \ar[u]^{=} & \overline{H}_{k-1}=h_{k}^{-1}\left(x^{0,k-1}+ Y^1\right) \ar@{->>}[r]^-{h} \ar@{^{(}->}[u] & x^{0,k-1}+ Y^1 \ar@{^{(}->}[u] \\
\overline{H} \ar@{^{(}->}[r] \ar[u]^{=} & \overline{H}=\overline{H}_{k}=h_{k}^{-1}\left(x^{0,k}+ Y^0\right) \ar@{->>}[r]^-{h} \ar@{^{(}->}[u] & x^{0,k}+ Y^0 \ar@{^{(}->}[u] \\
} 
\]

This induces a commutative diagram of fundamental groups 
\begin{equation}
\xymatrix{
1\ar[r] & \pi_1 \overline{H} \ar@{^{(}->}[r] & \pi_1 X \ar@{->>}[r] ^-{h_{\ast}} & \pi_1(x^{0,0}+Y^{k})= \ZZ^{2k}\ar[r] &1\\
1\ar[r] & \pi_1 \overline{H} \ar@{^{(}->}[r] \ar[u]^{=} & \pi_1 \overline{H}_1 \ar@{->>}[r]^-{h_{\ast}} \ar@{^{(}->}[u] & \pi_1(x^{0,1}+Y^{k-1})=\ZZ ^{2k-2} \ar@{^{(}->}[u]\ar[r] &1 \\
 & \ar[u]^{=} &  \ar@{^{(}->}[u] & \ar@{^{(}->}[u] & \\
 & \vdots & \vdots &\vdots & \\
1\ar[r] & \pi_1 \overline{H} \ar@{^{(}->}[r] \ar[u]^{=} & \pi_1 \overline{H}_{k-1} \ar@{->>}[r]^-{h_{\ast}} \ar@{^{(}->}[u] & \pi_1(x^{0,k-1}+Y^1)=\ZZ ^2 \ar@{^{(}->}[u]\ar[r] &1 \\
1\ar[r] & \pi_1 \overline{H} \ar@{^{(}->}[r] \ar[u]^{=} & \pi_1 \overline{H} \ar@{->>}[r]^-{h_{\ast}} \ar@{^{(}->}[u] & \pi_1(x^{0,k}+Y^0)=1 \ar@{^{(}->}[u]\ar[r] &1 \\
}
\label{eqndiaggps}
\end{equation}
where injectivity of the vertical maps in the middle column follows from \eqref{eqnSES1}. The last two rows in this diagram are short exact sequences: The last row is obviously exact and the penultimate row is exact by \eqref{eqnSES1} for $l=k$.

We will now prove by induction (with $l$ decreasing) that the $l$-th row from the bottom
\[
1\rightarrow \pi_1 \overline{H} \rightarrow \pi_1 \overline{H}_l\rightarrow \pi_1(x^{0,l}+Y^{k-l})\rightarrow 1
\]
is a short exact sequence for $0\leq l\leq k$. 

Assume that the statement is true for $l$. We want to prove it for $l-1$. Exactness at $\pi_1\overline{H}$ follows from the sequence of injections $\pi_1 \overline{H}_l \hookrightarrow \pi_1 \overline{H}_{l-1}$.

For exactness at $\pi_1 (x^{0,l-1}+Y^{k-l+1})$ observe that, by the Ehresmann Fibration Theorem, the fibration $\overline{H}_{l-1}\rightarrow x^{0,l-1}+Y^{k-l+1}$ restricts to a locally trivial fibration $\overline{H}_{l-1}^*\rightarrow (x^{0,l-1}+Y^{k-l+1})^*$  with connected fibre $\overline{H}$ over the complement $(x^{0,l-1}+Y^{k-l+1})^*$ of the subvariety of critical values of $h$ in $x^{0,l-1}+Y^{k-l+1}$. Hence, the induced map $\pi_1 \overline{H}_{l-1}^*\rightarrow \pi_1 (x^{0,l-1}+Y^{k-l+1})^*$ on fundamental groups is surjective. Since the complements $\overline{H}_{l-1}\setminus \overline{H}_{l-1}^*$ and $(x^{0,l-1}+Y^{k-l+1})\setminus (x^{0,l-1}+Y^{k-l+1})^*$ are contained in complex analytic subvarieties of real codimension at least two, the induced map $\pi_1 \overline{H}_{l-1}\rightarrow \pi_1 (x^{0,l-1}+Y^{k-l+1})$ is surjective.

For exactness at $\pi_1 \overline{H}_{l-1}$ it is clear that $\pi_1 \overline{H}\leq \mathrm{ker}\left(\pi_1 \overline{H}_{l-1}\rightarrow \pi_1(x^{0,l-1}+Y^{k-l+1})\right)$. Hence, the only point that is left to prove is that $\pi_1 \overline{H}$ contains $$\mathrm{ker}\left(\pi_1 \overline{H}_{l-1}\rightarrow \pi_1(x^{0,l-1}+Y^{k-l+1})=\ZZ^{2k-2(l-1)}\right).$$
Let $g\in \mathrm{ker}\left(\pi_1 \overline{H}_{l-1}\stackrel{h_{\ast}}{\rightarrow} \pi_1(x^{0,l-1}+Y^{k-l+1}) \right)$. Then
\[
g\in \mathrm{ker}\left( \pi_1 \overline{H}_{l-1} \stackrel{h_{l\ast}}{\rightarrow} \pi_1 \left(x^{0,l}+ Y^{k-l+1}/Y^{k-l}\right)\right),
\]
since the map $h_{l\ast}$ factors through $h_{\ast}: \pi_1\overline{H}_{l-1}\rightarrow \pi_1(x^{0,l-1}+Y^{k-l+1})$.

By exactness of \eqref{eqnSES1} for $l$, this implies that there is $h\in \pi_1\overline{H}_l$ with $\iota_{l\ast}(h)=g$, where $\iota_l: \overline{H}_l\hookrightarrow \overline{H}_{l-1}$ is the inclusion map. It follows from commutativity of the diagram of groups \eqref{eqndiaggps} and injectivity of the vertical maps that $h\in \mathrm{ker}(\pi_1 \overline{H}_l\rightarrow \pi_1 (x^{0,l}+Y^{k-l}))$. The induction assumption now implies that $h\in \mathrm{Im} (\pi_1 \overline{H} \rightarrow \pi_1 \overline{H}_l)$.


Hence, by Induction hypothesis, $g\in \pi_1 \overline{H}$ and therefore the map $h|_{\overline{H}_{l-1}}$ does indeed induce a short exact sequence
\[
1\rightarrow \pi_1 \overline{H} \rightarrow \pi_1 \overline{H}_{l-1} \rightarrow \pi_1 Y^{k-l+1}\rightarrow 1.
\]

In particular, for $l=0$ we then obtain that $h$ induces a short exact sequence
\[
1\rightarrow \pi _1 \overline{H} \rightarrow \pi_1 X \rightarrow \pi_1 Y\rightarrow 1.
\]
\end{proof}

\begin{remark}
\label{rmkFiltVerGenWeak}
Note that in fact the proof of Theorem \ref{thmFiltVerGen} only requires that the corestrictions of the maps $h_l$ to $x^{0,l}+ Y^{k-l+1}/Y^{k-l}$ have isolated singularities and connected fibres. Thus, it suffices to check this weaker condition to apply Theorem \ref{thmFiltVerGen}. We will make use of this observation in Sections \ref{secExamples} and \ref{secExCombined}.
\end{remark}
