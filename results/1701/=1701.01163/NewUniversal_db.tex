\section{Universal homomorphism}
\label{secConsGens}

Delzant \cite[Theorem 2]{Del-08} and Corlette--Simpson \cite[Proposition 2.8]{CorSim-08} proved that for a K\"ahler group $G$ there is a finite number of Riemann orbisurfaces $S_{g_i,\mm_i}$ such that every epimorphism from G onto a surface group factors through one of the $\pi_1 ^{orb} S_{g_i,\mm_i}$. It is not difficult to see that their result can be stated as follows:

\begin{theorem}
 Let $X$ be a compact K\"ahler manifold and let $G=\pi_1 X$ be its fundamental group. Then there is $r\geq 0$ and closed hyperbolic Riemann orbisurfaces $S_{g_i,\mm_i}$ of genus $g_i \geq 2$ together with surjective holomorphic maps $f_i : X \rightarrow S_{g_i,\mm_i}$ with connected fibres, $1\leq i \leq r$, such that
 \begin{enumerate}
  \item the induced homomorphisms $f_{i,\ast} : G \rightarrow \pi_1 ^{orb} S_{g_i,\mm_i}$ are surjective with finitely generated kernel for $1\leq i \leq r$;
  \item the image of $\phi:= (f_{1,\ast}, \dots, f_{r,\ast}): G \rightarrow  \pi_1^{orb} S_{g_1, \mm_1} \times \dots \times \pi_1^{orb} S_{g_r,\mm_r}$ is full subdirect; and
  \item every epimorphism $\psi: G\rightarrow \pi_1^{orb} S_{h,\nn}$ onto a fundamental group of a closed hyperbolic Riemann orbisurface of genus $h\geq 2$ and multiplicities $\nn$ factors through $\phi$.
 \end{enumerate}
 \label{thmDelCorSim}
\end{theorem}

For a K\"ahler group $G$ we will call the homomorphism $\phi$ in Theorem \ref{thmDelCorSim} the \textit{universal homomorphism} (to a product of hyperbolic Riemann orbisurfaces), as it satisfies a universal property. 

\begin{lemma}
 Let $X$ be a compact K\"ahler manifold, let $G=\pi_1X$ be its fundamental group and let $\phi: G \rightarrow \pi_1^{orb} S_{g_1,\mm_1} \times \dots \times \pi_1 ^{orb} S_{g_r,\mm_r}$ be the universal homomorphism to a product of orbisurface groups defined in Theorem \ref{thmDelCorSim}. 
 
 Then $\phi$ is induced by a holomorphic map $f: X\rightarrow S_{g_1,\mm_1} \times \dots \times S_{g_r,\mm_r}$ and the image $G:=\phi(G)\leq \pi_1^{orb} S_{g_1,\mm_1} \times \dots \times \pi_1^{orb} S_{g_r,\mm_r}$ of $\phi$ is a finitely presented full subdirect product. 
 \label{lemUnivFinPres}
\end{lemma}
\begin{proof}
To simplify notation denote by $\G_i:= \pi_1^{orb} S_{g_i,\mm_i}$ the orbifold fundamental group of $S_{g_i,\mm_i}$ for $1\leq i \leq r$. The only part that is not immediate from Theorem \ref{thmDelCorSim} is that the image $\overline{G}$ of the restriction $\phi|_{G}$ is finitely presented. 

To see this, recall that by Theorem \ref{thmDelCorSim} the composition $p_i\circ \phi: G\rightarrow \pi_1^{orb} S_{g_i,\mm_i}$ of $\phi$ with the projection $p_i$ onto $\G_i$ has finitely generated kernel. Hence, the kernel 
 \[
 N_i:=\ker (p_i|_{\overline{G}}) = \overline{G} \cap \left( \G_1 \times \dots \times \G_{i-1}\times 1 \times \G_{i+1} \times \dots \times \G_r\right)\unlhd \overline{G}
 \]
of the surjective restriction $p_i|_{\overline{G}} : \overline{G} \rightarrow \G_i$ is a finitely generated normal subgroup of $\overline{G}$. 

Finite presentability is trivial for $r=1$, so assume that $r\geq 2$. Let $1\leq i < j \leq r$. The image of the projection $p_{i,j}(\overline{G}) \leq \G_i \times \G_j$ is a full subdirect product and $p_{i,j}(N_i)\unlhd p_{i,j}(\overline{G})$ is a normal finitely generated subgroup. Since by definition $p_{i,j}(N_i)\leq 1\times \G_j$ it follows from subdirectness of $p_{i,j}(\overline{G})$ that in fact $p_{i,j}(N_i)\unlhd 1 \times \G_j$ is a normal finitely generated subgroup. 

The group $p_{i,j}(N_i)$ is either trivial or has finite index in $\G_j$, since all finitely generated normal subgroups of $\G_j=\pi_1^{orb} S_{g_j,\mm_j}$ are either trivial or of finite index. The former is not possible, because $G$ is full. It follows that $p_{i,j}(N_i)\unlhd 1 \times \G_j$ is a finite index subgroup. Thus, $p_{i,j}(\overline{G}) \leq \G_i \times \G_j$ is a finite index subgroup. Since $i$ and $j$ were arbitrary we obtain that $G$ has the VSP property. Thus, $\overline{G}$ is finitely presented by \cite[Theorem A]{BriHowMilSho-13}.
\end{proof}

Lemma \ref{lemUnivFinPres} and its proof raise the natural question if there is a geometric analogue of the VSP property in this setting. More precisely, it is natural to ask if the composition $f_{ij}:X\rightarrow S_{g_i,\mm_i} \times S_{g_j,\mm_j}$ of the holomorphic map $f: X\rightarrow S_{g_1,\mm_1} \times \cdots \times S_{g_r,\mm_r}$ inducing the universal homomorphism and the holomorphic projection $S_{g_1,\mm_1} \times \cdots \times S_{g_r,\mm_r} \rightarrow S_{g_i,\mm_i} \times S_{g_j,\mm_j}$ is surjective. The answer to this question is positive. It follows immediately from Lemma \ref{lemUnivFinPres} and the following more general result.

\begin{proposition}
 Let $X$ be a compact K\"ahler manifold and let $G=\pi_1 X$. Let $\phi: G \to  \pi_1^{orb} S_{g_1,\mm_1} \times \cdots \times \pi_1^{orb} S_{g_r,\mm_r}$ be a homomorphism with finitely presented full subdirect image such that the composition $p_i\circ \phi : G \to \pi_1^{orb} S_{g_i,\mm_i}$ has finitely generated kernel for $1\leq i \leq r$.
 
 Then $\phi=f_{\ast}$ is realised by a holomorphic map $f=(f_1,\cdots,f_r): X \rightarrow S_{g_1,\mm_1} \times \cdots \times S_{g_r,\mm_r}$, for suitable complex structures on the $S_{g_i,\mm_i}$, and the holomorphic projection $f_{ij}=(f_i,f_j): X \rightarrow S_{g_i,\mm_i} \times S_{g_j,\mm_j}$ is surjective for $1\leq i < j \leq r$.
\label{propUnivFinPresHol}
\end{proposition}
\begin{proof}
 The part that $\phi$ is induced by a holomorphic map $f= (f_1,\cdots,f_r): X \rightarrow S_{g_1,\mm_1} \times \cdots \times S_{g_r,\mm_r}$ is immediate from the assumption that the $p_i\circ \phi$ have finitely generated kernel and Theorem \ref{lemSiuBeauCat}.
 
 Since $S_{g_i,\mm_i} \times S_{g_j,\mm_j}$ is connected of complex dimension $2$ and $f_{ij}$ is holomorphic it suffices to show that the image of $f_{ij}$ is 2-dimensional. After passing to regular finite covers $R_{\g_i}\rightarrow S_{g_i,\mm_i}$ by closed Riemann surfaces of genus $\g_i\geq 2$ and the induced finite-sheeted cover $X_0\rightarrow X$ with $\pi_1 X_0 = f_{\ast}^{-1}\left((\pi_1 R_{\g_1}\times \dots \times \pi_1 R_{\g_r} ) \cap f_{\ast}(\pi_1 X)\right)$, the $S_{g_i,\mm_i}$ can be replaced by closed Riemann surfaces $R_{\g_i}$ and the kernel of the epimorphisms $\pi_1 X_0\to \pi_1 R_{\g_i}$ is still finitely generated.
 
 Let $g=(g_1,\dots,g_r): X_0 \rightarrow R_{\g_1} \times \dots \times R_{\g_r}$ be the corresponding holomorphic map with finitely presented full subdirect image $g_{\ast}(\pi_1 X_0) = f_{\ast}(\pi_1 X_0)$ (the maps $g_i$ are not to be confused with the genus $g_i$ used above, which will not appear in the rest of the proof). For $1\leq i < j \leq r$ consider the image $g_{ij}(X_0)\subset R_{\g_i}\times R_{\g_j}$ of the holomorphic map $g_{ij}=(g_i,g_j): X_0 \rightarrow R_{\g_i}\times R_{\g_j}$. Assume for a contradiction that it is one-dimensional (equivalently, the image $f_{ij}(X) \subset S_{g_i,\mm_i} \times S_{g_j,\mm_j}$ of $f_{ij}=(f_i,f_j)$ is one-dimensional).
 
 Then Stein factorisation provides us with a factorisation
 \[
 \xymatrix{ X_0 \ar[r]^{h_2}\ar[dr]_{g_{ij}} & Y \ar[d]^{h_1}\\ & g_{ij}(X_0)}
 \]
 such that $h_1$ is holomorphic and finite-to-one, $h_2$ is holomorphic with connected fibres and $Y$ is a complex analytic space of dimension one. It is well-known that smoothness and compactness of $X$ and the fact that $Y$ is one-dimensional imply that we may assume that $Y$ is a closed Riemann surface.
 
 By projecting $g_{ij}(X_0)$ to factors we obtain factorisations
 \[
  \xymatrix{ X_0 \ar[r]^{h_2}\ar[rd]_{g_i} & Y\ar[d]^{q_i} \\ & R_{\g_i}}  \vspace{2cm}   \xymatrix{ X_0 \ar[r]^{h_2}\ar[rd]_{g_j}  & Y\ar[d]^{q_j} \\ & R_{\g_j}},
 \]
 in which all maps are surjective and holomorphic, and in particular $q_i$ and $q_j$ are finite-sheeted branched coverings. 
 
 Connectedness of the fibres of $h_2$ implies that the induced map $h_{2\ast} : \pi_1 X_0\rightarrow \pi_1 Y$ is surjective. Since the induced maps $g_{i,\ast}: \pi_1 X_0\rightarrow \pi_1 R_{\g_i}$ and $g_{j,\ast}: \pi_1 X_0\rightarrow \pi_1 R_{\g_j}$ are surjective it follows that $q_{i,\ast}$ and $q_{j,\ast}$ are surjective. Hence, the fact that the kernels of $g_{i,\ast}$ and $g_{j,\ast}$ are finitely generated implies that the kernels $\ker ~ q_{i,\ast}$ and $\ker ~ q_{j,\ast}$ are finitely generated. Since finitely generated normal subgroups of surface groups are trivial or of finite index, we obtain that $q_{i,\ast}$ and $q_{j,\ast}$ are isomorphisms.
 
 It follows that $g_{ij}$ factors as
 \[
  \xymatrix{ X_0 \ar[r]^{h_2}\ar[rd]_{g_{ij}}  & Y \ar[d] ^{(q_i,q_j)} \\ & R_{\g_i}\times R_{\g_j}},
 \]
 where the induced map $(q_{i,\ast},q_{j,\ast}):\pi_1 Y \rightarrow \pi_1 R_{\g_i}\times \pi_1 R_{\g_j}$ is injective. In particular the image of $(q_{i,\ast},q_{j,\ast})$ does not contain $\ZZ^2$ as a subgroup, because the fundamental group of the closed hyperbolic surface $\pi_1 Y$ does not. In contrast the VSP property and the fact that $\overline{G}$ is a finitely presented full subdirect product imply that the image $g_{ij,\ast}(\pi_1 X_0) \leq \pi_1 R_{\g_i}\times \pi_1 R_{\g_j}$ is a finite index subgroup and therefore contains $\ZZ^2$ as a subgroup. This contradicts the assumption. It follows that $g_{ij}(X_0)$ is 2-dimensional. 
\end{proof}

\begin{remark}
We have seen in the proof of Theorem \ref{thmNewCoab} in Section \ref{secResCoabKGs} that the kernel of $\phi$ in Proposition \ref{propUnivFinPresHol} being finitely generated is a sufficient condition for $\phi$ to be induced by a holomorphic map, and have the property that the projections of $G$ to surface group factors have finitely generated kernel.
\end{remark}

\begin{remark}
The condition in Proposition \ref{propUnivFinPresHol} that the projection to orbisurfaces has finitely generated kernel is necessary, even under the assumption that the homomorphism is induced by a holomorphic map. To see this one can take a generic hyperplane section $C=H\cap (S_{g_1}\times S_{g_2})$ of a direct product of two Riemann surfaces. Then $C$ admits holomorphic maps onto both factors and by the Lefschetz Hyperplane Theorem the image of its fundamental group under the inclusion map is $\G_{g_1}\times \G_{g_2}$. This is in contrast to many of the group theoretic results in the previous sections, where the condition that the projection to factors has finitely generated kernel could be replaced by the assummption that the map is induced by a holomorphic map.
\end{remark}

\begin{corollary}
Let $X$ be a compact K\"ahler manifold and let $f=(f_1,\cdots,f_r): X \rightarrow S_{g_1,\mm_1} \times \cdots \times S_{g_r,\mm_r}$  be a holomorphic realisation of the universal homomorphism $\phi: G\rightarrow \pi_1^{orb} S_{g_1,\mm_1} \times \cdots \times \pi_1^{orb} S_{g_r,\mm_r}$ defined in Theorem \ref{thmDelCorSim}.

  Then the holomorphic projection $f_{ij}=(f_i,f_j): X \rightarrow S_{g_i,\mm_i} \times S_{g_j,\mm_j}$ is surjective for $1\leq i < j \leq r$.
\end{corollary}
\begin{proof}
This is an immediate consequence of applying Proposition \ref{propUnivFinPresHol} to Lemma \ref{lemUnivFinPres} and its proof.
\end{proof}

It is natural to ask if there is a generalisation of Proposition \ref{propUnivFinPresHol} to give surjective holomorphic maps onto products of $s$ factors. The examples constructed in Theorem \ref{thmExsGenClass} show that this is certainly false for general $r-1 \geq s\geq 3$ -- for instance consider Theorem \ref{thmExsGenClass} with $r=4$, $k=2$ and any choice of branched coverings satisfying all necessary conditions. More generally, we also note that all of the groups constructed in Theorem \ref{thmExsGenClass} are projective. Thus, we can use the Lefschetz Hyperplane Theorem to realise them as fundamental groups of compact projective surfaces. Hence, we can not even hope for holomorphic surjections onto $k$-tuples under the additional assumption that our groups are of type $\mathcal{F}_m$ and that $k\leq m$.

This leads us to the following variation on this question, to which the examples constructed in this work do not provide a negative answer.

\begin{question*}
Let $X$ be a compact K\"ahler manifold with $\pi_i X=0$ for $2\leq i \leq m-1$ and let $G=\pi_1 X$. Let $\phi: G\rightarrow  \pi_1^{orb} S_{g_1,\mm_1}\times \dots \times \pi_1^{orb} S_{g_r,\mm_r}$ be a homomorphism satisfying the conditions of Proposition \ref{propUnivFinPresHol} and let $f=(f_1,\dots,f_r):X\rightarrow  S_{g_1,\mm_1}\times \dots \times  S_{g_r,\mm_r}$ be a holomorphic map realising $\phi$. 

If the image $\phi(G)\leq \pi_1^{orb} S_{g_1,\mm_1}\times \dots \times \pi_1^{orb} S_{g_r,\mm_r}$ has finiteness type $\mathcal{F}_m$ with $m\geq 2$, does this imply that for $1\leq i_1 < \dots < i_k\leq r$ the holomorphic projection $(f_{i_1},\cdots,f_{i_k}): X \rightarrow S_{g_{i_1},\mm_{i_1}}\times \dots \times S_{g_{i_k},\mm_{i_k}}$ onto $k$ factors is surjective?
\end{question*}


Versions of the results of Sections \ref{secResCoabKGs} to \ref{secFinPropBetti} hold for the universal homomorphism of a K\"ahler group. This is because orbisurface fundamental groups have finite index surface subgroups and the universal homomorphism is induced by a holomorphic map. In particular, we obtain the following version of Theorem \ref{thmNewCoab}.

\begin{theorem}
 For every K\"ahler group $G$ there are $r\geq 0$, closed orientable hyperbolic orbisurfaces $S_{g_i,\mm_i}$ of genus $g_i\geq 2$ and a homomorphism $\phi: G \rightarrow \pi_1^{orb} S_{g_1,\mm_1}\times \dots \times \pi_1^{orb} S_{g_r,\mm_r}$ with the universal properties described in Theorem \ref{thmDelCorSim}. Its image $\overline{G}=\phi(G)\leq \pi_1^{orb} S_{g_1,\mm_1}\times \dots \times \pi_1^{orb} S_{g_r,\mm_r}$ is a finitely presented full subdirect product. 
 
 Let $k\geq 0$ and $m\geq 2$ such that $m > \frac{k}{2}$. If $\overline{G}$ is of type $\mathcal{F}_m$ then for every $1\leq i_1 < \dots < i_k \leq r$ the projection $p_{i_1,\dots,i_k}(\overline{G})\leq \pi_1^{orb} S_{g_{i_1},\mm_{i_1}}\times \dots \times \pi_1^{orb} S_{g_{i_k},\mm_{i_k}}$ has a finite index coabelian subgroup of even rank.
 \label{thmExistRestFinPropsMT}
\end{theorem}

\begin{proof}
  The assertion that $\overline{G}$ is finitely presented follows from Lemma \ref{lemUnivFinPres}. By Theorem \ref{thmDelCorSim} $\phi$ is induced by a holomorphic map. We can lift any such holomorphic map to a holomorphic map $f$ defining the restriction  $\phi|_{G_0}: G_0 \rightarrow \pi_1 R_{\g_1}\times \dots \times \pi_1 R_{\g_r}$, obtained by passing to finite index surface subgroups $\pi_1 R_{\g_i}\leq \pi_1^{orb} S_{g_i,\mm_i}$ and the finite index subgroup $G_0:= G \cap \phi^{-1}(\pi_1 R_{\g_1}\times \dots \times \pi_1 R_{\g_r})$. The result now follows from Corollary \ref{corFinPropsProjFactors} and Proposition \ref{propNewCoab}.   
\end{proof}

\begin{remark}
 Note that since $\overline{G}$ is finitely presented, the consequences of Theorems \ref{thmNewCoab} and \ref{thmExistRestFinPropsMT} always apply for $k=3$.
\end{remark}






