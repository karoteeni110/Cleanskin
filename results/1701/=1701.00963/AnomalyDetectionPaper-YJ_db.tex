%
% ewsn-full.tex
%

%
% NOTE
%
% ewsn-proc is based on sigplan-proc-varsize 
% The default of sigplan-proc-varsize is 9pt, indented paragraphs (ACM style)
% For EWSN or other 10pt conference, use the 10pt option
\documentclass[10pt,emptycopyrightspace]{ewsn-proc}

% TODO do we really need this?
% % hack to avoid the ugly ACM paragraph definition
% % => can't leave blank line after this
% (remove comment for this hack)
%\renewcommand{\paragraph}[1]{\vskip 6pt\noindent\textbf{#1 }}
\renewcommand{\paragraph}[1]{\noindent\textbf{#1 }}

\usepackage{balance}
\usepackage{comment}

%\documentclass[conference]{IEEEtran}

%\documentclass[10pt, conference, letterpaper]{IEEEtran}

\usepackage{cite}
%\usepackage[noadjust]{cite}


% *** MATH PACKAGES ***
%
\usepackage[cmex10]{amsmath}
% A popular package from the American Mathematical Society that provides
% many useful and powerful commands for dealing with mathematics. If using
% it, be sure to load this package with the cmex10 option to ensure that
% only type 1 fonts will utilized at all point sizes. Without this option,
% it is possible that some math symbols, particularly those within
% footnotes, will be rendered in bitmap form which will result in a
% document that can not be IEEE Xplore compliant!
%
% Also, note that the amsmath package sets \interdisplaylinepenalty to 10000
% thus preventing page breaks from occurring within multiline equations. Use:
%\interdisplaylinepenalty=2500
% after loading amsmath to restore such page breaks as IEEEtran.cls normally
% does. amsmath.sty is already installed on most LaTeX systems. The latest
% version and documentation can be obtained at:
% http://www.ctan.org/tex-archive/macros/latex/required/amslatex/math/



% *** SUBFIGURE PACKAGES ***
\usepackage[tight,footnotesize]{subfigure}
% subfigure.sty was written by Steven Douglas Cochran. This package makes it
% easy to put subfigures in your figures. e.g., "Figure 1a and 1b". For IEEE
% work, it is a good idea to load it with the tight package option to reduce
% the amount of white space around the subfigures. subfigure.sty is already
% installed on most LaTeX systems. The latest version and documentation can
% be obtained at:
% http://www.ctan.org/tex-archive/obsolete/macros/latex/contrib/subfigure/
% subfigure.sty has been superceeded by subfig.sty.



% *** PDF, URL AND HYPERLINK PACKAGES ***
%
\usepackage{url}
% url.sty was written by Donald Arseneau. It provides better support for
% handling and breaking URLs. url.sty is already installed on most LaTeX
% systems. The latest version can be obtained at:
% http://www.ctan.org/tex-archive/macros/latex/contrib/misc/
% Read the url.sty source comments for usage information. Basically,
% \url{my_url_here}.


% *** Do not adjust lengths that control margins, column widths, etc. ***
% *** Do not use packages that alter fonts (such as pslatex).         ***
% There should be no need to do such things with IEEEtran.cls V1.6 and later.
% (Unless specifically asked to do so by the journal or conference you plan
% to submit to, of course. )


% correct bad hyphenation here
%\hyphenation{op-tical net-works semi-conduc-tor}
%\usepackage{hyperref}

\usepackage{graphicx}
\usepackage{graphics}
\usepackage{amssymb}
\usepackage{amsmath}
%\usepackage{amsthm}
\usepackage{color}
\usepackage{courier}

\usepackage{multirow}
\usepackage{footnote}
\usepackage[bottom]{footmisc}

\usepackage{listings}
\lstset{language=Java}
\definecolor{dkgreen}{rgb}{0,0.6,0}
\definecolor{gray}{rgb}{0.5,0.5,0.5}
\definecolor{mauve}{rgb}{0.58,0,0.82}

\lstset{frame=tb,
	language=Java,
	aboveskip=3mm,
	belowskip=3mm,
	showstringspaces=false,
	columns=flexible,
	basicstyle={\small\ttfamily},
	numbers=none,
	numberstyle=\tiny\color{gray},
	keywordstyle=\color{blue},
	commentstyle=\color{dkgreen},
	otherkeywords={command, error_t, uint8_t},             % Add keywords here
	stringstyle=\color{mauve},
	breaklines=true,
	breakatwhitespace=true,
	tabsize=3
}

\usepackage[linesnumbered, ruled]{algorithm2e}
\SetKwRepeat{Do}{do}{while}%
%\renewcommand{\algorithmicrequire}{\textbf{Input:}}
%\renewcommand{\algorithmicensure}{\textbf{Output:}}

%reduce spacing in references
%\usepackage{setspace}
%\usepackage[square,sort,comma,numbers]{natbib}
%\setlength{\bibsep}{0.0pt}


% \theoremstyle{definition}
% \newtheorem{thm}{Theorem}
% \newtheorem{defn}[thm]{Definition}
% \newtheorem{lemma}[thm]{Lemma}
% \newtheorem{remark}{Remark}
% \newtheorem{proposition}[thm]{Proposition}



%check mark
\usepackage{tikz}
\def\checkmark{\tikz\fill[scale=0.4](0,.35) -- (.25,0) -- (1,.7) -- (.25,.15) -- cycle;} 

%for table caption small case
\usepackage[hang,small,bf]{caption}
\usepackage{tabulary}

\newcommand{\figref}[1]{\figurename~\ref{#1}}
\newcommand{\tblref}[1]{\tablename~\ref{#1}}

\newcommand{\nop}[1]{}
\newcommand{\tabincell}[2]{\begin{tabular}{@{}#1@{}}
	#2
\end{tabular}}

%%%%%%%%%%%% for double blind %%%%%%%%%%%%
\usepackage{lipsum}
%\setlength{\abovecaptionskip}{0.2cm} 
%\setlength{\belowcaptionskip}{-0.2cm} 
\usepackage{xcolor}
\usepackage{soul}
\sethlcolor{black}
\makeatletter
\newif\if@blind
\@blindtrue %use \@blindfalse on final version

%\@blindfalse

\if@blind \sethlcolor{black}\else
   \let\hl\relax
\fi
%%%%%%%%%%%% for double blind %%%%%%%%%%%%


\usepackage{epstopdf}

%\newcommand{\nop}[1]{}


%
% paper title
% can use linebreaks \\ within to get better formatting as desired

%\title{RADIUS: A Robust Approach to Detect Anomalous Link Quality Degradation in WSNs} 
%\title{RADIUS: A Robust Approach for Accurate Detection of Anomalous Link Quality Degradation in WSNs} 
%\title{RADIUS: A Robust Approach towards Accurate Detection of Anomalous Link Quality Degradation in WSNs} %YJ 24.09.2016
\title{RADIUS: A System for Detecting Anomalous Link Quality Degradation in Wireless Sensor Networks} %YJ 24.09.2016
	
%\author{
%	\IEEEauthorblockN{Songwei Fu\IEEEauthorrefmark{1}, Chia-Yen Shih\IEEEauthorrefmark{1}, Yuming Jiang\IEEEauthorrefmark{2}, Xintao Huan\IEEEauthorrefmark{1}, Pedro Jos\'{e} Marr\'{o}n\IEEEauthorrefmark{1}}
%	\IEEEauthorblockA{\IEEEauthorrefmark{1}Networked Embedded Systems, University of Duisburg-Essen, Germany}
%	%\\\{songwei.fu, chia-yen.shih, pjmarron\}@uni-due.de}
%	\IEEEauthorblockA{\IEEEauthorrefmark{2}Department of Telematics, Norwegian University of Science and Technology(NTNU), Norway}
%	%\\\{jiang\}@item.ntnu.no}
%}	




%
% NOTE
%
% The EWSN reviewing process is double blind: authors must not
% reveal their identities to the reviewers. Names and affiliations
% will only be added for the camera-ready version (see below)
%\numberofauthors{1}
%\author{
%\alignauthor Double Blind \\
%  \affaddr{do not reveal authors}
%}

\numberofauthors{6}

\author{
	% 1st. author
	\alignauthor
	Songwei Fu\\
	\affaddr{Networked Embedded Systems, University of Duisburg-Essen, Germany}\\
	%\email{AuthorEmail@gmail.com}
	% 2nd. author
	\alignauthor
	Chia-Yen Shih\\
	\affaddr{Networked Embedded Systems, University of Duisburg-Essen, Germany}\\
	%\email{AuthorEmail@gmail.com}
	% 3rd. author
	\alignauthor
	Yuming Jiang\\
	\affaddr{Department of Telematics, Norwegian University of Science and Technology(NTNU), Norway}\\
	%\email{AuthorEmail@gmail.com}
	\and  % use '\and' if you need 'another row' of author names
	% 4th. author
	\alignauthor
	Matteo Ceriotti\\
	\affaddr{Networked Embedded Systems, University of Duisburg-Essen, Germany}\\
	%\email{AuthorEmail@gmail.com}
	% 5th. author
	\alignauthor
	Xintao Huan\\
	\affaddr{Networked Embedded Systems, University of Duisburg-Essen, Germany}\\
	%\email{AuthorEmail@gmail.com}
	% 6th. author
	\alignauthor
	Pedro Jos\'{e} Marr\'{o}n\\
	\affaddr{Networked Embedded Systems, University of Duisburg-Essen, Germany}\\
	%\email{AuthorEmail@gmail.com}
	\and
	% 7th. author
%	\alignauthor
%	Author Name\\
%	\affaddr{This is Author School Name}\\
%	\email{AuthorEmail@gmail.com}
}


%\author[1]{Author A\thanks{A.A@university.edu}}
%\author[1]{Author B\thanks{B.B@university.edu}}
%\author[1]{Author C\thanks{C.C@university.edu}}
%\author[2]{Author D\thanks{D.D@university.edu}}
%\author[2]{Author E\thanks{E.E@university.edu}}
%\affil[1]{Department of Computer Science, \LaTeX\ University}
%\affil[2]{Department of Mechanical Engineering, \LaTeX\ University}

%\author{
%	\IEEEauthorblockN{Songwei Fu\IEEEauthorrefmark{1}, Yan Zhang\IEEEauthorrefmark{2}, Yuming Jiang\IEEEauthorrefmark{2}, Chengchen Hu\IEEEauthorrefmark{2}\IEEEauthorrefmark{3}, Chia-Yen Shih\IEEEauthorrefmark{1}, Pedro Jos\'{e} Marr\'{o}n\IEEEauthorrefmark{1}}
%	\IEEEauthorblockA{\IEEEauthorrefmark{1}Networked Embedded Systems, University of Duisburg-Essen, Germany}
%	%\\\{songwei.fu, chia-yen.shih, pjmarron\}@uni-due.de}
%	\IEEEauthorblockA{\IEEEauthorrefmark{2}Department of Telematics, Norwegian University of Science and Technology(NTNU), Norway}
%	%\\\{yanzhang, ymjiang\}@item.ntnu.no}
%	\IEEEauthorblockA{\IEEEauthorrefmark{3}Department of Computer Science and Technology, Xi'an Jiaotong University, China}
%}

\begin{document}
	
% make the title area
\maketitle




\begin{abstract}
% IEEEtran.cls defaults to using nonbold math in the Abstract.
% This preserves the distinction between vectors and scalars. However,
% if the conference you are submitting to favors bold math in the abstract,
% then you can use LaTeX's standard command \boldmath at the very start
% of the abstract to achieve this. Many IEEE journals/conferences frown on
% math in the abstract anyway.

%YJ 24.09.2016
%To ensure proper functioning of a Wireless Sensor Network (WSN), it is crucial that the network is able to detect anomalies in communication quality (e.g., RSSI) which may cause  performance degradation, so that the network can react accordingly. To this end, we propose RADIUS in this paper, which is a new lightweight threshold-based approach. In comparison with two popular thresholding approaches in the WSN literature, the performance of RADIUS is consistently more robust to the user-chosen parameter in them. In addition, with integrated techniques aiming at minimizing the detection error (caused by normal randomness of RSSI) in discriminating good links from weak links, RADIUS is able to achieve high detection accuracy under diverse link conditions. Specifically, in a prototype system deployed in an indoor testbed, the results show that RADIUS maintains a stable detection error of 6.13\% on average  in detecting link quality anomalies for all links across the network. %Old. For RADIUS as an approach. 

To ensure proper functioning of a Wireless Sensor Network (WSN), it is crucial that the network is able to detect anomalies in communication quality (e.g., RSSI), which may cause  performance degradation, so that the network can react accordingly. In this paper, we introduce RADIUS, a lightweight system for the purpose. The design of RADIUS is aimed at minimizing the detection error (caused by normal randomness of RSSI) in discriminating good links from weak links and at reaching high detection accuracy under diverse link conditions and dynamic environment changes. Central to the design is a threshold-based decision approach that has its foundation on the Bayes decision theory. In RADIUS, various techniques are developed to address challenges inherent in applying this approach. In addition, through extensive experiments, proper configuration of the parameters involved in these techniques is identified for an indoor environment. In a prototype implementation of the RADIUS system deployed in an indoor testbed, the results show that RADIUS is accurate in detecting anomalous link quality degradation for all links across the network, maintaining a stable error rate of 6.13\% on average. %The RADIUS system is implemented and evaluated on an indoor testbed and the results show that RADIUS is accurate in detecting link quality anomalies for all links across the network, maintaining a stable error rate of 6.13\% on average.



%In a prototype implementation of the RADIUS system deployed in an indoor testbed, the results show that RADIUS is accurate in detecting link quality anomalies for all links across the network, maintaining a stable error rate of 6.13\% on average.


%To ensure proper functioning of a Wireless Sensor Network (WSN), it is crucial that the network is able to detect anomalies in communication quality (e.g., RSSI) that may cause the performance degradation so that the network can react accordingly. In this paper, we propose RADIUS, a lightweight approach that employs various techniques for minimizing the detection error (caused by normal randomness of RSSI) in discriminating good links from weak links, reaching high detection accuracy under diverse link conditions as compared to two popular thresholding techniques in WSNs, and achieving robustness to environment changes. Through extensive experiments, a proper configuration of involved parameters is identified for indoor environments. In a prototype system deployed in an indoor testbed, the results show that RADIUS is accurate in detecting link quality anomalies for all links across the network, maintaining a stable detection error of 6.13\% on average. 

%%%%%%%%%%%%%%%%%%%%%%% previous version %%%%%%%%%%%%%%%%%%%%%%%%
%To ensure proper functioning of a Wireless Sensor Network (WSN), it is crucial to detect true anomalies in communication quality, not contributed by its normal randomness but caused by real signal attenuation that are significant enough to cause the performance degradation so that the network can react accordingly. In this paper, we propose RADIUS, a lightweight approach that provides minimized error in detecting link quality anomalies while being robust to different deployment areas and environment changes. 
%Specifically, RADIUS lays its foundation on Bayes decision theory. The achieved high accuracy is robust across the network as compared to two popular detection techniques in WSNs.In addition, RADIUS employs various supporting techniques that further improve the accuracy and assist in adapting the detection threshold to environment changes. Extensive experiments show that the employed supporting techniques significantly increase the system performance and the best configuration of involved parameters is identified for indoor environments. In a prototype system deployed in an indoor testbed, the results show that RADIUS is accurate in  detecting link quality anomalies for all links across the network, achieving a stable low detection error of 6.13\% on average. 



%%%%%%%%%%%%%%%%%%%%%%% previous previous version %%%%%%%%%%%%%%%%%%%%%%%%
%To ensure proper functioning of a Wireless Sensor Network (WSN), it is crucial to detect anomalies in communication quality so that the network can react accordingly. In this paper, we propose RADIUS, a lightweight approach that provides minimized error in detecting link quality anomalies while being robust to different parameter settings and environment changes. Specifically, RADIUS lays its foundation on Bayes decision theory (minimizing detection error), together with techniques that assist in adapting the detection threshold (achieving robustness).
%%motivated by a comparison to two popular detection techniques in WSNs,
%Extensive experiments have been conducted to investigate the effect of the involved parameters on the detection performance, identifying the best configuration for indoor environments. In a prototype system deployed in an indoor testbed, the results show that RADIUS is accurate in  detecting link quality anomalies for all links across the network, achieving a stable low detection error of 6.13\% on average. 



%%%%%%%%%%%%%%%%%%%%%%% SECON version %%%%%%%%%%%%%%%%%%%%%%%%
%To ensure proper functioning of a wireless sensor network (WSN), it is crucial that the network is able to detect anomalies in communication quality changes so that the network can react accordingly. To address this challenge, we propose a novel approach, called \textit{RADIUS}, which is simple, robust and adaptive. Specifically, \textit{RADIUS} is a threshold-based approach (simple), with foundation on Bayes decision theory (robust), together with techniques that assist in adapting the threshold (adaptive). In addition, extensive experiments have been conducted to investigate the effect of the involved parameters on the detection performance, which gives insights on the setting of them. Moreover, a system implementation of \textit{RADIUS} is introduced and its performance has been investigated on a testbed. The results show that \textit{RADIUS} is not only robust to dynamic changes but also accurate in detection, achieving an average detection error rate as low as 6.13\% on the testbed. 



\end{abstract}

%\input{introduction-YJ}
\section{Introduction}\label{sec:intro}

The performance of a Wireless Sensor Network (WSN) often deteriorates after in-situ deployment of the network \cite{1182885, 4408504, 6850017, 6661323}. Link quality degradation, due to, e.g., fading and interference, is one of the major reported causes behind such behavior, which may be significant enough to impact the link's performance, e.g., the packet delivery ratio. Detecting such anomalous degradation in link quality is crucial for an operational WSN to decide possible remedy actions such as tuning stack parameters \cite{Lin:atpc, 7164923}. In such way, the network can continuously maintain its performance and satisfy the user's requirements. 

%

%The performance of a Wireless Sensor Network (WSN) often deteriorates after in-situ deployment of the network \cite{1182885, 4408504, 6850017, 6661323}. Link quality degradation, due to, e.g., fading and interference, is one of the major reported causes behind such behavior, which may be significant enough to impact the link's performance (e.g., the packet delivery ratio). Hence, detecting anomalous degradation in link quality, is crucial for an operational WSN to maintain its performance and continuously satisfy the user's requirements. %To this end, we present a lightweight system, called RADIUS, in this paper, whose design is aimed at minimizing the detection error (caused by normal randomness of RSSI) in discriminating good links from weak links and at achieving high detection accuracy under diverse link conditions.

In resource constrained WSNs, detecting anomalous link quality degradation requires {\bf lightweight} solutions with low overheads in using memory, computation and communication resources. Resource-hungry centralized monitoring systems \cite{6661323, 1367278, 1267061} and/or machine learning-based detection techniques \cite{4085803, 4289308, 5356174} are hence hardly applicable to WSNs, due to large communication and/or computation overheads. In addition, a solution should be {\bf accurate} with a low error rate (false positive/negative rate) and be {\bf robust} with consistent performance under diverse link conditions and dynamic environment changes. However, in WSNs, due to the stochastic nature of link quality metrics, e.g., received signal strength indicator (RSSI) \cite{2893729}, it is challenging to distinguish between true link quality degradation and normal randomness. Data smoothing\cite{6199865} may only mitigate the problem. CDF-based \cite{4068315, 6199865} and Chebyshev inequality-based \cite{1689248, 1592596, 1515559} statistical techniques are lightweight and seem to be effective in making the distinguishing. However, our investigation, as to be shown later in this paper, reveals that it is difficult to optimize them to achieve both high detection accuracy and robustness for links which may experience diverse link conditions and dynamic environment changes. 

To meet these requirements, i.e., lightweight, accurate and robust, we have designed a system for detecting anomalous link quality degradation, called RADIUS. In addition to being lightweight, its design has also been aimed at minimizing the detection error (caused by normal randomness of RSSI) in discriminating good links from weak links and at being robust in maintaining the detection performance for different links and under dynamic environment changes. Central to the design is a threshold-based decision approach (for being lightweight) that has its foundation on the Bayes decision theory (for being accurate and robust). 

To the best of our knowledge, no prior work has investigated the applicability of Bayesian thresholding in detecting anomalous link quality degradation in WSNs. A possible reason is perhaps due to the various challenges inherent in applying the approach. To address these challenges, various techniques have been developed to identify the number of RSSI samples needed to achieve a ``good'' approximation of the mean and the standard deviation, to update the mean and standard deviation estimates, and to choose and update a ``proper'' setting for the \textit{a priori} probability, where the mean, the standard deviation and the \textit{a priori} probability are the three fundamental variables used in the Bayes formula. 

  
%To address these problems, we combine several supporting techniques with Bayesian thresholding in RADIUS for achieving an overall better system performance. To avoid high detection error rate due to insufficient training samples, a \textit{minimal training set estimation} technique is employed together with a sliding-window \textit{data smoothing} procedure to cope with the inherent randomness of the RSSI signal. In addition, \textit{self-adaptation} and \textit{system feedback-based adaptation} techniques are introduced to make RADIUS robust to environment changes. 

A prototype of the RADIUS system has been implemented and deployed in an indoor testbed. For proper configuration of the parameters involved in the various techniques in RADIUS, suggestions on their settings are given based on extensive experiments. In addition, we found that high detection accuracy can be achieved by RADIUS under diverse link conditions more robustly as compared to the CDF and Chebyshev thresholding techniques. Moreover, the overhead analysis and the detection results show that RADIUS not only has low overheads in memory, communication and computation, but also is accurate in detecting link quality anomalies for all links across the network, maintaining a stable error rate of 6.13\% on average. These are an indication of RADIUS in fulfilling the requirements. 


%The performance of a Wireless Sensor Network (WSN) often deteriorates after in-situ deployment of the network \cite{1182885, 4408504, 6850017, 6661323}. Link quality degradation, due to, e.g., fading and interference, is one of the major reported causes behind such behavior. Hence, detecting anomalous degradation in link quality, which may be significant enough to impact the link's performance (e.g., the packet delivery ratio), is crucial for an operational WSN to maintain its performance and continuously satisfy the user's requirements.

%%The performance of a Wireless Sensor Network (WSN) often degrades after in-situ deployment of the network \cite{1182885, 4408504, 6850017, 6661323}. Anomalous link quality degradation is one of the major reported causes, because the environment as well as fading and interference effects have a significant impact on radio links. The problem is exacerbated in WSNs where low power communication is often used to reduce energy consumption. Indeed, to maintain the performance of an operational WSN and continuously satisfy the user requirements, it is crucial to monitor the link quality of each link and detect anomalous quality degradation that may deteriorate the link performance, e.g., the packet delivery ratio (PDR). 

%Many WSN techniques can benefit from the detection of anomalous link quality degradation. Blacklisting techniques \cite{2187194} typically involve thresholds during the link estimation process, in which deciding the best link quality threshold to discriminate good links from weak links is critical. In addition, as an indicator of a good link turning into a weak link, anomalous link quality degradation can serve as a good trigger for taking remedy actions such as tuning stack parameters including transmission power \cite{Lin:atpc} or other layer parameters \cite{2185730,7164923}. In this paper, we focus on the detection of anomalous link quality degradation. 

  
%In resource constrained WSNs, detecting an anomalous link quality degradation requires lightweight solutions with minimum error rates and robust detection performance. First of all, considering the limited resources of sensor nodes, the detection system must have a \textbf{low overhead} in the use of memory, computation and communication resources. Therefore, both centralized monitoring systems \cite{6661323, 1367278, 1267061} and powerful machine learning-based detection techniques \cite{4085803, 4289308, 5356174} are hardly applicable to WSNs due to either large communication or computation overhead.  Furthermore, achieving a \textbf{minimal detection error rate} is challenging in WSNs due to the stochastic nature of link quality metrics, e.g., the received signal strength indicator (RSSI) \cite{2893729}. The difficulty resides in discriminating true link quality anomalies from normal randomness. Data smoothing\cite{6199865} may only mitigate the problem. CDF-based \cite{4068315, 6199865} and Chebyshev inequality-based \cite{1689248, 1592596, 1515559} statistical techniques are effective to identify thresholds of the monitored attributes with low overhead, however, such thresholds are typically not optimized to achieve a minimum detection error rate. Finally, the accuracy in detecting link quality anomalies must be \textbf{robust} across the whole network without being severely affected by either the diversity of link conditions
%the choice of the involved parameters 
%or the dynamics of the operational environment.



%Many existing network monitoring systems rely on active collection of network status \cite{6850017, 6661323, 1367278, 1267061}, introducing large communication overhead. On the other hand, powerful detection techniques such as data mining or machine learning \cite{4085803, 4289308, 5356174} own a good detection accuracy but also a high computational complexity. This makes them hardly applicable to WSNs where distributed anomaly detection schemes are required in order to minimize the communication overhead.


%Furthermore, achieving a \emph{minimal detection error} is challenging in WSNs due to the stochastic nature of typical %wireless 
%link quality metrics such as the received signal strength indicator (RSSI) \cite{2893729}. The difficulty resides in discriminating true link quality anomalies from normal randomness. Data smoothing\cite{6199865} may only mitigate the problem. CDF-based \cite{4068315, 6199865} and Chebyshev inequality-based \cite{1689248, 1592596, 1515559} statistical techniques are widely applied to WSNs because they can effectively identify thresholds of the monitored attributes with low overhead. However, such thresholds are typically not optimized to achieve a minimum detection error. 

%Finally, the accuracy in detecting link quality anomalies must be \emph{robust} across the whole network without being severely affected by either the diversity of link conditions
%%the choice of the involved parameters 
%or the dynamics of the operational environment. We show that the aforementioned CDF- and Chebyshev-based techniques heavily rely on the parameter choice. In other words, achieving high accuracy globally over the whole network with these two methods requires a fine-tuning process for each individual link, incurring a huge overhead for large networks. Moreover, during the system lifetime, environment changes typically force a new calibration of the system parameters to sustain a required detection accuracy.

%To meet the above-mentioned requirements, this paper proposes RADIUS, a novel approach to detect anomalies in (but not limited to) WSNs link quality, as observed through RSSI measurements. To achieve its goals, RADIUS bases on the Bayesian decision theory (\textit{Bayesian thresholding}, in particular) to minimize the error rate in distinguishing between good and weak links. As to be discussed later, the computation of the Bayes thresholds only relies on a user-defined parameter (\textit{a priori} probability of a good link) and the statistical measures (mean and standard deviation) of the RSSI values measured at a good link, incurring as low computational complexity as CDF-based or Chebyshev-based thresholding techniques. 

%%Under the assumption of a Gaussian channel, the complexity of the approach decreases significantly, maintaining a minimum detection error. 

%To the best of our knowledge, no prior work has investigated the applicability of Bayesian thresholding in detecting link quality anomalies in WSNs. We believe the reason is that there are several important problems that should be solved first before applying it. First, we need to know how many RSSI samples lead to a good approximation of the mean and standard deviation. Second, since the \textit{a priori} probability is a predefined parameter, we need to investigate the performance under a coarse parameter choice. Third, we need to update the mean, standard deviation as well as the \textit{a priori} probability to achieve robust detection performance. 


%To address these problems, we combine several supporting techniques with Bayesian thresholding in RADIUS for achieving an overall better system performance. To avoid high detection error rate due to insufficient training samples, a \textit{minimal training set estimation} technique is employed together with a sliding-window \textit{data smoothing} procedure to cope with the inherent randomness of the RSSI signal. In addition, \textit{self-adaptation} and \textit{system feedback-based adaptation} techniques are introduced to make RADIUS robust to environment changes. 


%We implement the RADIUS system and evaluate it in an indoor testbed. We show the high detection accuracy can be achieved under diverse link conditions as compared to CDF-based and Chebyshev-based thresholding techniques. Through extensive experiments, we identify a proper configuration of involved system parameters for indoor environments. In addition, we show that RADIUS is accurate in detecting link quality anomalies across the whole network, maintaining a stable error rate of 6.13\% on average.

%In a comparison with reference thresholding techniques, we demonstrate that the Bayesian thresholding can achieve minimal detection error without being sensitive to its parameters.
%In an extensive experimentation of RADIUS in an indoor testbed, we investigate the effect of each system parameter on the detection performance. The results show that while the high accuracy achieved by Bayesian thresholding is robust to the setting of its threshold parameter, employing the aforementioned supporting techniques increase the overall performance of RADIUS significantly. In addition, in an operational system, we show that RADIUS is accurate in detecting link quality anomalies across the whole network, achieving a stable low detection error of 6.13\% on average.

The rest of the paper is organized as follows. Section \ref{sec:related} discusses the related work. Section \ref{sec:system} presents the system design and motivates the adoption of Bayesian thresholding. Section \ref{sec:approach} introduces the key techniques used in RADIUS. Section \ref{sec:parameterChoice} analyzes the effect of the various involved parameters in these RADIUS techniques on the detection performance. Section \ref{sec:imp&eva} reports the details of our implementation, the corresponding system overheads and the overall performance evaluation in an operational system on an indoor testbed. Finally, Section \ref{sec:conclusion} concludes the paper.



% The performance of a wireless sensor network (WSN) often degrades after in-situ deployment of the network \cite{4408504, 6850017, 6661323}. Among various causes, anomalous link quality degradation is one of the most reported causes, because radio links are subject to environment, fading and interference. The problem is exacerbated in WSNs where low transmission power is often used to reduce the energy consumption. Indeed, to maintain the performance of WSNs, it is crucial to monitor the link quality of each link and detect anomalous link quality degradation that may deteriorate the performance of the link, such as the packet delivery ratio (PDR).

% In resource constrained WSNs, detecting anomalous link quality degradations is not as simple as one might imagine. A key challenge is to design a lightweight approach that provides robust detection performance of minimum detection error rate. 
% %i.e., the approach is optimized for accuracy while being robust so that the high accuracy can be achieved fairly for different deployments and it does degrade over time. 
% Although there have been many studies on the network performance diagnosis \cite{6850017, 6661323, 1367278, 1267061} and  anomaly detection techniques \cite{4085803, 4289308, 5356174, 4068315, 6199865, 1592596, 1689248, 1515559}, most of the existing approaches only partially tackle the challenge and do not provide all the necessary features at the same time. To achieve the optimal performance of detecting anomalous link quality degradation in sensor networks, the detection approach must have the following important features. %Those  

% (i) Low overhead. Due to the limited resources of sensor nodes, the detection system must not incur large memory, computation and communication overhead. Many existing network monitoring systems rely on active collection of network status \cite{6850017, 6661323, 1367278, 1267061}, introducing large communication overhead to the energy constrained sensor nodes. On the other hand, powerful detection techniques such data mining or machine learning-based techniques \cite{4085803, 4289308, 5356174} own a good detection accuracy with a cost of high computational complexity, hences are not applicable to sensor nodes if distributive anomaly detection scheme is employed for minimizing communication overhead.

% (ii) High accuracy. 
% %Due to the high cost of a wrong detection decision, it is critical to keep the detection error as low as possible. %Minimizing the detection error includes reducing a false positive rate (the ratio of legitimate behavior falsely identified as an anomaly) and a false negative rate (ratio of miss of capturing a real anomaly). 
% To achieve minimal detection error is challenging in WSNs due to the stochastic nature of a wireless link's quality\cite{1062741}. The difficulty is to distinguish true link quality anomalies from normal randomness. Smoothing\cite{6199865} may mitigate the randomness but does not solve the problem of minimizing the detection error. CDF-based \cite{4068315, 6199865} or Chebyshev inequality-based \cite{1592596, 1689248, 1515559} statistical techniques are widely applied to WSNs because they are effective to find thresholds of the monitored attributes with low overhead, however, their thresholds are normally not optimized to achieve the minimum detection error. 

% (iii) High robustness. The meaning of robustness is two fold. First, the high accuracy must hold for detecting anomalies across the whole network without incurring heavy overhead of tuning involved parameters. As to be discussed in Section \ref{sec:motivationBayes}, the literature-adopted  techniques heavily rely on the parameter choice and achieving high accuracy globally over the network must require a fine-tuning process for each individual link, incurring a huge overhead for large scale networks. Moreover, the detection performance must not degrade over time due to dynamic environment changes.

% This paper proposes RADIUS, a novel approach to detect anomalous degradations in (but not limited to) link quality described by RSSI (received signal strength indicator), which incurs low overhead yet provides robust detection performance of minimal error. To achieve this goal, we base RADIUS on the Bayes decision theory by employing \textit{Bayesian thresholding} technique as our core detection technique. Under the assumption of Gaussian channel, the complexity of Bayesian thresholding decreases significantly while the detection error is mathematically minimized. To the best of our knowledge, RADIUS is the first approach of applying Bayesian thresholding to detect link quality anomalies in WSNs. By comparing its performance with two statistical thresholding techniques, we discover a distinguishing property of Bayes threshold: its achieved minimal detection error is not sensitive to its parameter choice, contributing to both the accuracy and robustness of the system.

% In addition, we employ several supporting techniques in RADIUS to achieve a better system performance. To avoid high detection error due to insufficient training, a \textit{minimal training set estimation} technique is employed together with a sliding-window \textit{data smoothing} technique to copy with the normal randomness naturally existing in the RSSI signal. In addition, \textit{self-adaptation} and \textit{system feedback-based adaptation} techniques are introduced in RADIUS to enable its adaptation to dynamic environment changes. %Extensive experimental results show that the employed supporting techniques increase significantly the overall system performance. 

% To get the best performance out of RADIUS, we investigate the effect of each system parameter on the detection performance based on extensive experiments, giving insights on the setting of them for indoor environments. The results show that while the high accuracy achieved by Bayesian thresholding is robust to its parameter setting, employing the aforementioned supporting techniques increase the overall performance of RADIUS significantly with proper setting of the involved parameters. In addition, the performance of RADIUS is evaluated, based on its implementation on a testbed. The results show that RADIUS is accurate for detecting link quality anomalies of all links across the network, achieving a durable low detection error of 6.13\% on average.

%%%%% contributions revised from SECON version %%%%%%%%%%%%
%In summary, the contributions of this paper are several-fold: (1) A novel approach, called RADIUS, is proposed for anomalous communication quality degradation detection in WSNs, which is light weight, robust and accurate. Specifically, RADIUS lays its foundation on Bayes decision theory, together with a set of supporting techniques. (2) We investigate the effectiveness of Bayesian thresholding in detecting link quality anomaly. In addition, we substantiate the adoption of Bayesian thresholding by comparing its detection performance to two statistical thresholding techniques. To the best of our knowledge, we are the first one to apply Bayesian thresholding to detect anomalous link quality degradation. (3) The effect of each system parameter is detailedly investigated, based on which suggested settings of the involved parameters are provided. (4) The performance of RADIUS is evaluated on a WSN testbed, showing that RADIUS is robust to dynamic changes and achieves appealing performance. 

%%%%% contributions in SECON version %%%%%%%%%%%%
%In summary, the contributions of this paper are several-fold: (1) A novel approach, called \textit{RADIUS}, is proposed for anomalous communication quality degradation detection in WSNs, which is simple, robust, adaptive and accurate. Specifically, \textit{RADIUS} is a threshold-based approach, having foundation on Bayes decision theory, together with a smoothing technique to cope with the normal randomness in detection. In addition, \textit{RADIUS} adopts self-adaptation and system feedback-based adaptation to deal with normal dynamic changes in the system. (2) The effect of each involved parameter is detailedly investigated, based on which suggested settings of the involved parameters are provided. (3) A system implementation of \textit{RADIUS} is introduced, including the key functional modules as well as the interactions among them. (4) The performance of \textit{RADIUS} is evaluated on a WSN testbed, showing that \textit{RADIUS} is robust to dynamic changes and achieves appealing performance. (5) A comparison of the Bayes threshold technique with the two state-of-the-art threshold techniques is presented, which substantiates the adoption of the Bayes. 


% The rest of this paper is organized as follows. Sec. \ref{sec:related} reviews related work. Sec. \ref{sec:system} presents the system design and motivates the adoption of Bayesian thresholding. Sec. \ref{sec:approach} introduces the details of the RADIUS approach. Sec. \ref{sec:parameterChoice} analyzes the effect of system parameters on the detection performance. Sec. \ref{sec:imp&eva} introduces the implementation details, system overhead and its performance evaluation on an indoor testbed. Finally, Sec. \ref{sec:conclusion} concludes the paper.



%%%%%%%%%%%%%%%%%%%%%%%%%%%%%%%%%%% SECON version %%%%%%%%%%%%%%%%%%%%%%%%%%%%%%%%%%%%%%%%%%	
	
%The performance (e.g., packet delivery ratio) of a wireless sensor network (WSN) often degrades after in-situ deployment of the network \cite{6850017} \cite{4408504} \cite{6661323}. The reason for this is that the WSN may encounter various node failure and communication degradation problems that are not or cannot be detected during the deployment. Indeed, to ensure proper functioning of \textit{any} computer network, it is necessary to monitor the communication quality of each link and detect anomalous degradations or failures on the link so that the network can adapt accordingly. Specifically, it is crucial to detect anomalous link quality degradation events that may cause significant, negative impact on the user-experienced performance of the link, such as the packet delivery ratio (PDR) of the link falls below a user-defined value. 
%
%In WSNs, detecting anomalous link quality degradations is not as simple as one might imagine. This is mainly due to the special characteristics of WSNs. Specifically, limited by the processing capability, simple algorithms are preferred. In addition, due to the stochastic nature, a wireless link's quality can vary highly randomly. Moreover, constrained by the energy supply (e.g. only battery is available), there is a typical need of using minimal transmission power, which contributes additionally to the randomness of the received signal strength, the most direct and fundamental factor measuring the link quality. Essentially, the challenge is to develop a simple algorithm that can detect link quality degradations, which are potential causes of performance anomalies in the network, from the received link quality indicator signal, RSSI (received signal strength indicator), which is random by nature. In other words, the difficulty is to distinguish true causes of such network anomalies from normal randomness. Another part of the challenge is that the algorithm should be able to adapt to (normal) dynamic changes (e.g. due to environmental condition changes). 
%
%The objective of this work is to propose a novel approach to detect anomalous degradations in link quality described by RSSI, which is simple, robust and adaptive. We call the proposed approach \textit{RADIUS}. 
%
%Specifically, \textit{RADIUS} is a simple threshold-based approach. In the literature, threshold techniques have been adopted for similar purposes, owing to their simplicity in implementation. For example, in \cite{4068315}, a timeout threshold based on the \textit{x-th percentile} of the sample set of consecutively missing heartbeats was used for detecting node failures. In \cite{1592596}, one-side Chebyshev inequality with a chosen \textit{target false positive rate} is applied to detect performance problems in the network. In \cite{1515559}, the proposed approach uses a threshold value $k$ to detect abnormal traffic pattern if the short term mean of inter-arrival times of packets is $k$ \textit{times} of the standard deviation away from the long term mean. The threshold method in \cite{1515559} is essentially a special case of the Chebyshev threshold method, based on Chebyshev inequality with an implicit maximum false rate $1/k^2$. In all these threshold-based approaches, finding appropriate thresholds is crucial to achieving high detection accuracy. However, as to be discussed in Section \ref{sec:ThresholdCompare}, the detection performance based on the literature-adopted threshold techniques is so highly sensitive to the setting of the threshold value that it is difficult to conclude a ``good enough'' value from empirical study to ensure robust performance. Another consequence of such high sensitivity is that it is even more difficult to adapt these approaches to cope with dynamic changes, e.g. changes in humidity and temperature and changes in the deployment surrounding area, which are natural or normal in real networks. 
%
%
%To address the threshold sensitivity challenge and consequently make the proposed approach robust, we base \textit{RADIUS} on the Bayes decision theory, together with a sliding-window smoothing technique to copy with the normal randomness naturally existing in the RSSI signal. In addition, self-adaptation and system feedback-based adaptation techniques are introduced in \textit{RADIUS} to enable its adaptation and hence increased robustness to dynamic changes. 
%
%Owing to its robustness, an appealing property of \textit{RADIUS} is that the involved parameters can be easily set. This paper also presents results based on extensive experiments to investigate the effect of the involved parameters on the detection performance, which gives insights on the setting of them. In addition, the performance of \textit{RADIUS} is evaluated, based on its implementation on a testbed. The results show that \textit{RADIUS} is robust to different conditions and is accurate, achieving a detection error rate as low as 6.13\% on average on the testbed. 
%
%In summary, the contributions of this paper are several-fold: (1) A novel approach, called \textit{RADIUS}, is proposed for anomalous communication quality degradation detection in WSNs, which is simple, robust, adaptive and accurate. Specifically, \textit{RADIUS} is a threshold-based approach, having foundation on Bayes decision theory, together with a smoothing technique to cope with the normal randomness in detection. In addition, \textit{RADIUS} adopts self-adaptation and system feedback-based adaptation to deal with normal dynamic changes in the system. (2) The effect of each involved parameter is detailedly investigated, based on which suggested settings of the involved parameters are provided. (3) A system implementation of \textit{RADIUS} is introduced, including the key functional modules as well as the interactions among them. (4) The performance of \textit{RADIUS} is evaluated on a WSN testbed, showing that \textit{RADIUS} is robust to dynamic changes and achieves appealing performance. (5) A comparison of the Bayes threshold technique with the two state-of-the-art threshold techniques is presented, which substantiates the adoption of the Bayes. 
%
%
%The rest of this paper is organized as follows. Section \ref{sec:approach} introduces the \textit{RADIUS} approach. Section \ref{sec:parameterChoice} investigates the effect of each involved parameter on the detection performance of \textit{RADIUS}, and based on the investigation, gives suggestion on the setting of the parameter. Section \ref{sec:imp&eva} introduces an implementation of \textit{RADIUS} and its performance evaluation on a testbed WSN. Section \ref{sec:ThresholdCompare} compares the Bayes-based threshold technique with two other threshold techniques, which has motivated the adoption of Bayes theorem in \textit{RADIUS}. Finally, Section \ref{sec:conclusion} concludes the paper. 






\input{related-work}
\input{system}
\input{approach-YJ}
%\section{Analysis of the Parameter Space}\label{sec:parameterChoice} 
\section{Setting the RADIUS Parameters}\label{sec:parameterChoice} 

	
\begin{figure*}[t]
	\centering
	\includegraphics[width=1.0\linewidth, height = 5cm]{4-Find_ProbGood3}
	\vspace{-1.1cm}
	\caption{\textbf{Effect of the \textit{a priori} probability $P(H_g)$ on the error rates for 5 representative links. $P(H_g)$ varies in the range [$10^{-5}$, $1-10^{-5}$].} }
	\label{fig:EVA-ProbGood}
	\vspace{-0.4cm}
\end{figure*}
	
	
	
\begin{figure*}[t]
	\centering
	\begin{minipage}{.35\textwidth}
		\centering
		\includegraphics[width=1\linewidth, height=4cm]{1-Find_N_sigma3}
		\vspace{-0.75cm}
		\captionof{figure}{\textbf{Effect of the parameter $N_s$. The resultant $\sigma_s$ becomes more stable after $N_s = 250$.} }
		\vspace{-0.5cm}
		\label{fig:Nsigma}
	\end{minipage} \hfill
	\begin{minipage}{.6\textwidth}
		\centering
		%\vspace{-0.5cm}
		%\includegraphics[width=1\linewidth, height=4cm]{fig/2-Find_MeanError2}
		\includegraphics[width=1\linewidth, height=4cm]{2-Find_MeanError4}
		\vspace{-0.75cm}
		\captionof{figure}{\textbf{Effect of the parameter $E_{\mu}$. $E_{\mu} = 1$ dBm provides a good tradeoff between the detection accuracy and the training set size (i.e. training latency).}}
		\vspace{-0.5cm}
		\label{fig:ErrorMean}
	\end{minipage}
\end{figure*}

In the previous section, we presented the details of the Bayesian thresholding and the supporting techniques in RADIUS. We now study the impact of the aforementioned parameters required in each individual technique on the system performance. Specifically, we elaborate the effect of the Bayes threshold parameter $P(H_g)$ and then explore the parameter space of all the parameters involved in supporting techniques. Based on detailed analysis, we give insights on the best parameter setting for an indoor office environment.

For this, we performed extensive experiments, collecting real data traces from an indoor testbed, whose details are reported in Section \ref{sec:evaluation}. To capture different link conditions, we selected 8 sender-receiver pairs (either line-of-sight or non-line-of-sight) at different locations with various environment dynamics (e.g., human movements and obstacles). In each experiment, we simulated a good link turning into a weak link by decreasing the transmission power of the sender node from the maximum level gradually to the minimum with a packet sending rate of 5 Hz. The receiver node records the RSSI and PDR traces for more than 15 minutes. We repeated the experiment 10 times for each link. The minimum PDR that decides whether a link is a good link or a weak link is set to 80\% throughout the whole analysis.



\subsection{Bayesian Thresholding} \label{sec:ThresholdChoice}

According to Equation \ref{equ:rssiTHD}, calculating the Bayes threshold
%only 
requires a user-defined parameter: \textit{a priori} probability $P(H_g)$.
%if the underlying conditional distribution of the RSSI values is known from the training RSSI samples. 
Different from the general analysis depicted in Section \ref{sec:motivationBayes}, we present here the detailed analysis of the impacts of $P(H_g)$ on the false positive rate (FPR), false negative rate (FNR) and the total error rate. 
 
Figure \ref{fig:EVA-ProbGood} shows the change of the error rates with varying $P(H_g)$ ranging in [$10^{-5}$, $1-10^{-5}$] for 5 representative links out of the 8 analyzed links. We observe that the FPR always decreases with $P(H_g)$, while the FNR increases with $P(H_g)$. This is because a larger $P(H_g)$ indicates a higher weight on the FPR in the computation of the Bayes error (see Equation \ref{equ:error}). Hence,  reducing FPR is more effective than reducing FNR to keep the Bayes error rate low for a larger $P(H_g)$.  

Moreover, we can see in Figure \ref{fig:EVA-ProbGood} that the overall error rate mostly stays low regardless of the values of $P(H_g)$, except for the cases when the value of $P(H_g)$ is extremely close to 0 or 1. The results confirm that the system performance with the Bayes threshold is insensitive to the setting of $P(H_g)$ as long as extreme values are not considered. The reason for this is that the Bayesian thresholding approach always tries to balance between FPR and FNR for any $P(H_g)$ setting. 

From the figure, we can further see that a global $P(H_g)$ setting from a wide range (any value not close to 0 or 1) may not be the best setting for each individual link. However, it can provide for all different links near optimal detection accuracy at the same time. In other words, with a coarse global setting of $P(H_g)$ for all DAs, the Bayesian thresholding ensures RADIUS to deliver near optimal accuracy for different links under diverse link conditions without the need of tuning for each of them. For our case, we select the initial setting of $P(H_g)$ at 0.8 because we assume that the probability of the links being good is generally higher than that of being weak, in our deployment environment. Additionally, to avoid the significant increase of FNR caused by the over-adjustment due to the \textit{a priori} probability refinement, we limit the \textit{maximum} $P(H_g)$ to 0.99 for our deployment environment.




\subsection{Estimating the Minimal Training Set Size}\label{sec:evaParamTrainingSize}

As discussed in Section \ref{sec:minTrainingSet}, the first task of a DA before generating a Bayes threshold is to estimate the minimal training set size $N_{ts}$ for each link. A proper $N_{ts}$ needs to achieve a good tradeoff between detection accuracy and training latency. To apply Equation \ref{equ:minSampleSize}, the computation of $N_{ts}$ requires two parameters: (1) the number of first $N_s$ samples of RSSI for computing the standard deviation $\sigma_s$, and (2) the maximum error of the estimated mean  $E_{\mu}$. While the first $N_{s}$ samples of RSSI only give a quick indication, $N_{ts}$ is the resultant minimal training set size, from which the DA estimates the RSSI mean and standard deviation for computing Bayes thresholds. $N_{ts}$ is usually larger or at least equal to $N_{s}$.   

We first study the impact of $N_{s}$. A small $N_{s}$ may result in a partial view of the complete channel variation, while overly large $N_{s}$ may only increase the training delay. To understand the impact of $N_{s}$, we plot in Figure \ref{fig:Nsigma} the resultant standard deviation $\sigma_s$ for various values of $N_s$ based on the RSSI traces of all 8 links. We observe that the values of $\sigma_s$ initially have a larger variation and become more stable when $N_s$ is close to 250. The reason for this is that a small set of samples is insufficient to capture the overall temporal variations of RSSI, especially in an indoor environment where multi-path fading and interference are ubiquitous. Based on the result, we choose $N_s = 250$ for our indoor environment. %we conclude that, to find a good approximation of $\sigma_p$, we need to find the minimum value for $N_s$ with a converged $\sigma_s$, e.g., we select $N_s = 250$.
  
Then we focus on the impact of $E_{\mu}$. According to Equation \ref{equ:minSampleSize}, the choice of $E_{\mu}$ has a tradeoff: smaller $E_{\mu}$ indicates higher estimation accuracy of the RSSI mean and thus higher detection accuracy; a smaller $E_{\mu}$, however, may also increase the training set size significantly. By varying the estimated errors $E_{\mu}$, we plot the resultant training set size and error rates for 5 representative links in Figure \ref{fig:ErrorMean}. 
%For each link, we use the mean of 2000 samples in the RSSI trace as the true mean (we believe that 2000 samples are enough to give a nearly-true population mean for our environment, i.e., $E_{\mu} = 0$ dBm).  
The figure shows that the total error rate decreases significantly with a smaller $E_{\mu}$ at the expense of a rapidly increasing training set size. To balance between the detection error and the training time, we choose $E_{\mu} = 1$ dBm for our indoor environment, which causes only a slight increase of the detection error compared to that of $E_{\mu} = 0$ dBm while at the same time keeping $N_{ts}$ within the scale of a few hundred samples, achieving a good tradeoff between the training latency (several minutes with a sending frequency of 5 Hz) and detection accuracy. 

\begin{figure}[t]
	\centering
	%\vspace{-0.6cm}
	\includegraphics[width=1.0\linewidth, height = 4.5cm ]{3-Find_SlidingWindowSize_5}
	\vspace{-0.7cm}
	%\caption{Effect of data smoothing with different sliding window size $l$ for different $P(H_g)$. Error rate is close to lowest when $l = 3$ in all cases. }
	\caption{\textbf{Effect of data smoothing with different sliding window size $l$ for different $P(H_g)$. Error rate is close to lowest when $l=3$.} }
	%\caption{Effect of the sliding window size $l$ for different $P(H_g)$. As $l$ increases, FPR decreases while FNR increases. We choose $l = 3$.}
	\label{fig:EVA-slidingWindow}
	\vspace{-0.3cm}
\end{figure} 

\begin{figure}[t]
	\centering
	\includegraphics[width=1.0\linewidth, height = 6cm]{6-Find_UpdateWindowSize-WithPerson-nonLoS-UpDownUpDown-L1R1-2}
	\vspace{-1cm}
	\caption{\textbf{Effect of training set (TS) update with varying update window size $l_{update}$.}}
	\label{fig:EVA-updateWindow}
	\vspace{-0.7cm}
\end{figure} 

\subsection{Data Smoothing} \label{sec:ImpactDataSmoothing}

As mentioned in Section \ref{sec:dataSmoothing}, smoothing the noisy data during the detection phase requires a sliding window of size $l$ to reduce the detection error caused by the normal RSSI randomness. To see the impact of $l$, we demonstrate how the error rate changes with different values of $l$ (window size from 1 to 15) under two representative $P(H_g)$ values.

We observe from Figure~\ref{fig:EVA-slidingWindow} that, for both $P(H_g)$ settings, increasing $l$ reduces FPR but increases FNR, which causes the total error rate to first decrease and then increase with a larger $l$. The reason is that smoothing RSSI is effective to reduce false alarms. However, if $l$ keeps increasing, at some point, the real RSSI anomaly events are smoothed out, causing a significant increase in FNR. The impact of $l$ is also related to the setting of the minimal PDR ($PDR_{min}$) that defines a good link. In our case, as PDR is computed over a sliding window of 10 packets and $PDR_{min}$ is set to 80\%, a small sliding window $l$ is preferred to avoid the significant increase in FNR. For our case, we choose $l = 3$, at which the total error rate is close to the lowest for both $P(H_g)$ settings.  



\begin{figure}[t]
	\centering
	%\includegraphics[width=1.0\linewidth, height = 8cm]{fig/7-Find_AlarmNum_and_AdjustStep-WithPerson-nonLoS-L1R3-makeup2}
	\includegraphics[width=1.0\linewidth, height = 8cm]{7-Find_AlarmNum_and_AdjustStep-WithPerson-nonLoS-L1R3-makeup2-v2}
	\vspace{-1.5cm}
	\caption{\textbf{Effect of the \textit{a priori} refinement with varying $N_{alarm}$ and $\delta$.}}
	\label{fig:EVA-paramRefine}
	\vspace{-0.65cm}
\end{figure}

\subsection{Updating the Training Set} \label{sec:evaParamTrainingUpdate}

To be adaptive to varying environment conditions, RADIUS updates the training set as discussed in Section \ref{sec:trainingSetUpdate}, dynamically generating new thresholds. For this, the relevant parameter is the update window size $l_{update}$. We first show that the detection performance is enhanced with this updating technique, and then we discuss the impact of $l_{update}$.

In Figure~\ref{fig:EVA-updateWindow}(a), we present an RSSI trace with a valley of RSSI values (between 300 and 500 seconds) indicating an abnormal situation that causes the monitored PDR to fall below the expected performance as shown in Figure~\ref{fig:EVA-updateWindow}(b). By adapting the training set and consequently the threshold (see Figure~\ref{fig:EVA-updateWindow}(c)), we can see from Figure~\ref{fig:EVA-updateWindow}(d) that in this experiment, the detection error can be reduced of 3\%-4\% with updated thresholds, down from 18\% to 14\% approximately.

%\begin{figure}[t]
%	\centering
%	\includegraphics[width=1.0\linewidth, height = 7cm]{fig/6-Find_UpdateWindowSize-WithPerson-nonLoS-UpDownUpDown-L1R1-2}
%	\vspace{-1.1cm}
%	\caption{Effect of training set (TS) update with varying update window size $l_{update}$.}
%	\label{fig:EVA-updateWindow}
%	\vspace{-0.5cm}
%\end{figure} 


Furthermore, we also observe from Figure~\ref{fig:EVA-updateWindow}(d) that the impact of $l_{update}$ is not significant on the detection error
%, and therefore the parameter choice is more tolerate 
(we set $l_{update}$ to be 50 for our example deployment). However, a larger $l_{update}$ may require a longer time to fill up the window making the threshold update less responsive in some cases. With such setting of $l_{update}$, we observe the detection error can be reduced of 3\% to 8\% in all experiments. Considering that the total error rate in most of our experiments is less than 20\%, such amount of reduction in the error rate is significant. 






\subsection{Refinement of the A Priori Probability}\label{sec:EVA-paramRefine}


In addition to updating the training data set, one other situation that requires to generate a new threshold is when the detection accuracy degrades with an increasing number of false alarms, indicating the need of updating the \textit{a priori} probability. As described in Section \ref{sec:prioriRefinement}, we consider the maximum number of consecutive false alarms $N_{alarm}$ and the adjustment step $\delta$ of $P(H_g)$. We quantify the effects of these parameters in Figure \ref{fig:EVA-paramRefine}, where we compare the detection accuracy with and without the refinement of the \textit{a priori} probability. Specifically, we show the detection performance with varying $N_{alarm}$ and $\delta$. We use the suggested values in the above sections for the other parameters.

%\begin{figure}[t]
%	\centering
%	\includegraphics[width=1.0\linewidth, height = 8cm]{fig/7-Find_AlarmNum_and_AdjustStep-WithPerson-nonLoS-L1R3-makeup2}
%	\vspace{-1.1cm}
%	\caption{Effect of the \textit{a priori} probability refinement with varying $N_{alarm}$ and $\delta$.}
%	\label{fig:EVA-paramRefine}
%	\vspace{-0.8cm}
%\end{figure}


From Figure \ref{fig:EVA-paramRefine}, we can see that a smaller $N_{alarm}$ can reduce FPR but it may also cause a significant increase in FNR due to over-adjustment. On the other hand, a larger $N_{alarm}$ makes the system conservative on the \textit{a priori} probability refinement and hence the refinement less effective. The optimal choice of  $N_{alarm}$ falls at the location where the total error rate is lowest. In addition, the choice of the parameter $\delta$ needs to consider a tradeoff: larger $\delta$ indicates a more effective adjustment but a higher risk of over-adjustment. In our example, we choose $N_{alarm} = 5$ and $\delta = 0.003$. With such parameter settings, the analysis of all data traces shows that based on the accuracy improvement achieved by the training set updating technique, refining $P(H_g)$  can further reduce the error rate in a range from 2\% to 5\%. 
% on the basis of applying the training set update.  





\section{Implementation and Evaluation} \label{sec:imp&eva}

In the previous sections, we presented the RADIUS system design and analyzed the impact of its parameters on the detection performance. Based on these, we implemented the DA component of RADIUS for TelosB sensor platforms and the VCC for standard PCs. In this section, we detail our implementation and discuss the system overhead. At last, we show the evaluation results on the detection performance of the overall implemented system in an indoor testbed. 

\subsection{System Implementation} 

In this section, we first describe the implementation details of the two major RADIUS components: the DA and the VCC. We introduce the programming interface of the DA to show that it is easy to use for higher-layer services and applications, followed by the implementation details of the VCC and of the RADIUS IoT extension. 
%Then, we evaluate the implementation overhead to show that it is lightweight enough to be executed on actual sensor nodes. 

\begin{figure}[h]
	\vspace{-0.3cm}
	\begin{lstlisting}[frame=single]  % Start your code-block
	
	interface DetectionAgent {
	command void configureDA(Struct_Param parameters);
	command error_t start_Training();
	command error_t stop_Training();
	command error_t start_Detection();
	command error_t stop_Detection();
	command void update_RSSI(uint8_t rssi, uint8_t childId);
	}
	\end{lstlisting}
	\vspace{-0.55cm}
	\caption{\textbf{The programming interface for DA.}}
	\label{fig:TinyOSinterface}
	\vspace{-0.3cm}
\end{figure}



To ease the integration of RADIUS into higher-layer applications, we implemented the DA component as a module on TinyOS 2.1.2, which provides an interface \texttt{DetectionAgent} (see Figure \ref{fig:TinyOSinterface}). The \texttt{configureDA} command is used to configure DA with user-specified parameter settings (e.g., the initial $P(H_g)$) as provided in configuration messages sent by the VCC. The interface also provides control commands such as \texttt{start\_Training} or \texttt{start\_Detection} for executing the different phases in RADIUS. The command \texttt{update\_RSSI} is used to update the RSSI distribution of each link during both the training and the detection phases. With this programming interface, an application only needs to react to VCC's control messages and call the different commands accordingly. Note that RADIUS is not restricted to TelosB or TinyOS and can be easily adapted to other embedded devices with low-power radios.
% without the need to handle all specific details in RADIUS. 



\begin{figure}[t]
	%\vspace{-0.2cm}
	\centering
	\includegraphics[width=1\linewidth, height = 5.5cm]{Radius-GUI}
	\vspace{-0.9cm}
	\caption{\textbf{The Monitoring User Interface of VCC.}}
	\label{fig:GUI}
	\vspace{-0.7cm}
\end{figure}


\begin{figure}[t]
	%\vspace{-0.3cm}
	\centering
	\includegraphics[width=1\linewidth, height = 5.5cm]{IOT-interface}
	\vspace{-0.9cm}
	\caption{\textbf{The Android UI for RADIUS.}}
	\label{fig:IOT}
	\vspace{-0.7cm}
\end{figure}

The component VCC, running on a standard PC, is implemented in Java. It processes the alarms received from the network and produces diagnosis results reporting the relevant anomalies and their locations for the nodes that are experiencing high packet losses, which assists the system operator in identifying possible remedy actions. The VCC also includes a \textit{ Monitoring User Interface} (see Figure \ref{fig:GUI}), which provides a visualization of the packet delivery performance, detection status and the diagnosis results. Via this interface, the operator can monitor and control the RADIUS system.



We have also extended the VCC with an Internet-of-Things interface, which allows the VCC to connect with openHAB \cite{openHAB}, an open source smart home automation software. MQTT \cite{MQTT}, a lightweight messaging transport protocol, is used for the communication between openHAB and the VCC. With this extension, the user can remotely control the RADIUS system and monitor the detection results with an Android smart phone, as depicted in Figure \ref{fig:IOT}.



The current implementation of RADIUS is able to detect anomalies in link quality. Nevertheless, the programming interface (Figure \ref{fig:TinyOSinterface}) and the DA module can be easily extended to detect anomalies of other attributes, e.g., CRC error rate or packet overflow rate.
%, due to the modular implementation of the DA component. 
In addition, RADIUS currently works for static tree-based data collection applications but it can be also applied to other routing schemes. 
%E.g., before a node switches its parent, it informs the current parent node to stop monitoring for this node; after parent switch, the new parent node initiates a training phase. Once the thresholds are computed, it is ready to start the anomaly detection for this newly joined child node.




\subsection{System Overheads} \label{sec:overhead}
In this section, we analyze the memory, communication and computation overheads of our RADIUS implementation.

%\paragraph{Memory overhead.}

\paragraph{Memory overhead.} Detection Agents incur memory overhead on RAM (data) and ROM (program) of sensor nodes. As presented in Section \ref{sec:trainingSetUpdate}, we keep updating the training set to adapt the mean $\mu$ and standard deviation $\sigma$ of the density distribution of RSSI. To avoid increasing RAM usage during the update, we implemented this in a memory-friendly way, i.e., to compute $\mu$ and $\sigma$ with a single pass without storing the previous measurements of RSSI. To do so, we reformulate $\mu$  and $\sigma$ in the following way:
\setlength{\belowdisplayskip}{2pt} \setlength{\belowdisplayshortskip}{2pt}
\setlength{\abovedisplayskip}{2pt} \setlength{\abovedisplayshortskip}{2pt}
\begin{equation} \label{equ:stdComputation}
\mu = \frac{s}{n},   \quad  \sigma = \sqrt{\frac{1}{n-1}\big(q - \frac{s^2}{n}\big)}
\end{equation}
where $s$ and $q$ are defined as follows:
\begin{align}
s= \displaystyle \sum_i^n{x_i},\quad  q = \displaystyle \sum_i^n{x_i^2}
\end{align}
%\vspace{-0.1cm}
in which $x_i$ is the i-th RSSI reading. Instead of storing the entire training set, the DA then stores only 2 counters ($s$ and $q$) for each link to compute and update $\mu$ and $\sigma$. By doing so, the RAM consumption of RADIUS has a complexity of $O(mn)$, where $m$ is the number of links from direct child nodes and $n$ is the number of monitored attributes (e.g., RSSI, CRC error rate), remaining independent from the sample number.

To evaluate the RAM and ROM overhead, we compare the memory usage of a tree-based data collection application with and without the DA module. In the application, each node has two one-hop child nodes and therefore stores information about two links. The application alone consumes 3060 bytes in RAM and 25082 bytes in ROM while the application including the DA module consumes 3176 bytes in RAM and 31170 bytes in ROM. This indicates that the DA module consumes 116 bytes RAM (in comparison to 10 KB RAM in a TelosB device), and approximately 6 KB ROM (in comparison to 48 KB ROM in a TelosB device). %Such RAM overhead is small compared to 10 KB RAM in TelosB and the ROM overhead is acceptable compared to 48 KB ROM in TelosB. 

\paragraph{Communication overhead.} Due to its distributed architecture, the anomaly detection process alone incurs no communication overhead in RADIUS. It requires additional communication only if the DAs send alarms corresponding to detected anomalies, or when the VCC sends control messages. To reduce such overhead, the alarms with minimum information (2 bytes) about the detected anomaly are piggybacked on the application packets. On the other hand, the number of control messages delivered from the VCC to the DAs, based on our indoor testbed evaluation results, is negligible compared to the amount of received application packets.

\paragraph{Computation overhead.} The main computation overhead comes from the processing of the Bayesian thresholding. In RADIUS, the complexity of Bayesian thresholding involves the calculation of the mean, the standard deviation and the Bayes threshold according to Equation \ref{equ:rssiTHD}. Testing results show that the processing of a Bayes threshold takes about 10 ms, which is small compared to the normal packet inter-arrival time in typical data collection applications.


\begin{figure}[t]
	%\vspace{-0.4cm}
	\centering
	\includegraphics[width=1\linewidth, height=4cm]{floor-plan}
	\vspace{-0.85cm}
	\caption{\textbf{The indoor testbed.}}
	\label{fig:floor-plan}
	\vspace{-0.35cm}
\end{figure}

\subsection{Experimental Evaluation} \label{sec:evaluation}

We have evaluated the detection performance of our RADIUS implementation in an indoor testbed (Figure \ref{fig:floor-plan}) consisting of 12 TelosB motes, deployed in several offices of a university building. Each sensor node runs an application that collects environmental data and sends it to the sink every 2 seconds following a tree-based routing topology. We instrumented a Detection Agent on each sensor node and ran the VCC on a PC connected to the sink node (Base Station).


To configure the RADIUS system, we adopt the system parameter settings suggested in the previous analysis (Section \ref{sec:parameterChoice}). Table \ref{tab:parameter} summarizes the suggested parameter settings. After a training period of about 5 minutes, RADIUS starts the detection phase for a period of about 24 hours. During the experiment, we logged the received alarms, the RSSI traces and the PDR traces. Figure \ref{fig:EVA-1node} demonstrates our results of detecting link quality anomalies on one of the links.

\begin{table}[t]
	\centering
	%\vspace{-0.3cm}
	\caption{\textbf{The system parameter settings used in the evaluation.}}
	\vspace{-0.3cm}
        \footnotesize
	\begin{tabulary}{0.95\textwidth}{|l|l|l|}
		\hline
		\textbf{Techniques} & \textbf{Parameters} & \textbf{Settings} \\ 
		\hline
		\textbf{Bayesian} & initial $P(H_g)$ & 0.8\\
		\textbf{Thresholding} & maximum $P(H_g)$ & 0.99\\
		\hline
		\textbf{Training Set Size} & sample number $N_s$ to compute $\sigma_s$ & 250 \\
		\textbf{Estimation} & max. estimated error of mean $E_{\mu} $ & 1 dBm\\
		\hline
		\textbf{Data Smoothing}	& sliding window size $l$ & 3 \\
		\hline
		\textbf{Training Set Update}	& update window size $l_{update}$ & 50 \\ 
		\hline
		\textbf{\textit{A Priori} Probability} & max. alarm number $N_{alarm}$ & 5 \\ 
		%\hline			
		\textbf{Refinement} & adjustment step $\delta$ & 0.003 \\ 
		\hline
	\end{tabulary}
	\vspace{-0.6cm}
	\label{tab:parameter} 
\end{table}
%---------------------- 

\begin{figure}[t]
	\centering
	%\vspace{-0.1cm}
	\includegraphics[width=1.0\linewidth, height = 7.5cm]{8-Testbed-1node-24hours}
	\vspace{-0.7cm}
	\caption{\textbf{Experimental results for link quality anomaly detection of the link from node 11 to node 9. The  experiment was run for 24 hours, from 11:00 am till 11:00 am on the next day.}}
	\label{fig:EVA-1node}
	\vspace{-0.2cm}
\end{figure}


From the PDR trace depicted in Figure~\ref{fig:EVA-1node}(b), we can observe that the link frequently experienced high packet losses during the first 4 hours and the last 3 hours due to the bad channel quality caused by the students' movements crossing the communication link (see Figure~\ref{fig:EVA-1node}(a)). The received alarms that reported such anomalous link quality degradation are marked in red in Figure~\ref{fig:EVA-1node}(d). The results show a good detection accuracy. The overall FNR and FPR of detecting the anomalous RSSI degradation for this link over 24 hours are 5.1\% and 4\%, respectively. 

In addition, we can see from Figure~\ref{fig:EVA-1node}(e) that RADIUS can keep the error rate stable over the detection period. Figure~\ref{fig:EVA-1node}(c) clearly shows that the threshold is adaptive to RSSI variations due to the environment changes. From our analysis based on the logged refinement points (marked in Figure~\ref{fig:EVA-1node}(d)), RADIUS refines the \textit{a priori} probability at around 13:00 on the first day and 10:00 on the next day to reduce FPR and thus maintain the detection accuracy. 

To demonstrate that RADIUS can robustly achieve high detection accuracy for all the links across the entire deployed network, we plot in Figure \ref{fig:EVA-11nodes} the error rate for every link in the testbed. The figure shows that with a set of global parameter settings (listed in Table \ref{tab:parameter}), RADIUS achieves a low error rate for every link in the network (6.13\% on average). %This is achieved by the Bayesian thresholding which provides near optimal detection accuracy with the supporting hreshold adaptation techniques. 


%we can see that the system with a set of global parameter settings achieves a low error rate for every node in the network (6.13\% in average). %Note that node 6 has highest FPR and FNR among all nodes. This is because the normal RSSI trace of the link 6 - 8 highly overlaps its abnormal RSSI trace, making the detection more difficult. 

\begin{figure}[t]
	\centering
	%\vspace{-0.1cm}
	\includegraphics[width=1.0\linewidth]{9-Testbed-12nodes}
	\vspace{-0.7cm}
	\caption{\textbf{The error rates for every link in the network (Figure \ref{fig:floor-plan}).}}
	\label{fig:EVA-11nodes}
	\vspace{-0.55cm}
\end{figure}







%\input{discussion} 
\input{conclusion-YJ}

\balance
\bibliographystyle{abbrv}
%\bibliographystyle{bibtex/IEEEtran}
\bibliography{IEEEabrv,AnomalyDetectionPaper}


\end{document}


