\section{Introduction}\label{sec:intro}

The performance of a Wireless Sensor Network (WSN) often deteriorates after in-situ deployment of the network \cite{1182885, 4408504, 6850017, 6661323}. Link quality degradation, due to, e.g., fading and interference, is one of the major reported causes behind such behavior, which may be significant enough to impact the link's performance, e.g., the packet delivery ratio. Detecting such anomalous degradation in link quality is crucial for an operational WSN to decide possible remedy actions such as tuning stack parameters \cite{Lin:atpc, 7164923}. In such way, the network can continuously maintain its performance and satisfy the user's requirements. 

%

%The performance of a Wireless Sensor Network (WSN) often deteriorates after in-situ deployment of the network \cite{1182885, 4408504, 6850017, 6661323}. Link quality degradation, due to, e.g., fading and interference, is one of the major reported causes behind such behavior, which may be significant enough to impact the link's performance (e.g., the packet delivery ratio). Hence, detecting anomalous degradation in link quality, is crucial for an operational WSN to maintain its performance and continuously satisfy the user's requirements. %To this end, we present a lightweight system, called RADIUS, in this paper, whose design is aimed at minimizing the detection error (caused by normal randomness of RSSI) in discriminating good links from weak links and at achieving high detection accuracy under diverse link conditions.

In resource constrained WSNs, detecting anomalous link quality degradation requires {\bf lightweight} solutions with low overheads in using memory, computation and communication resources. Resource-hungry centralized monitoring systems \cite{6661323, 1367278, 1267061} and/or machine learning-based detection techniques \cite{4085803, 4289308, 5356174} are hence hardly applicable to WSNs, due to large communication and/or computation overheads. In addition, a solution should be {\bf accurate} with a low error rate (false positive/negative rate) and be {\bf robust} with consistent performance under diverse link conditions and dynamic environment changes. However, in WSNs, due to the stochastic nature of link quality metrics, e.g., received signal strength indicator (RSSI) \cite{2893729}, it is challenging to distinguish between true link quality degradation and normal randomness. Data smoothing\cite{6199865} may only mitigate the problem. CDF-based \cite{4068315, 6199865} and Chebyshev inequality-based \cite{1689248, 1592596, 1515559} statistical techniques are lightweight and seem to be effective in making the distinguishing. However, our investigation, as to be shown later in this paper, reveals that it is difficult to optimize them to achieve both high detection accuracy and robustness for links which may experience diverse link conditions and dynamic environment changes. 

To meet these requirements, i.e., lightweight, accurate and robust, we have designed a system for detecting anomalous link quality degradation, called RADIUS. In addition to being lightweight, its design has also been aimed at minimizing the detection error (caused by normal randomness of RSSI) in discriminating good links from weak links and at being robust in maintaining the detection performance for different links and under dynamic environment changes. Central to the design is a threshold-based decision approach (for being lightweight) that has its foundation on the Bayes decision theory (for being accurate and robust). 

To the best of our knowledge, no prior work has investigated the applicability of Bayesian thresholding in detecting anomalous link quality degradation in WSNs. A possible reason is perhaps due to the various challenges inherent in applying the approach. To address these challenges, various techniques have been developed to identify the number of RSSI samples needed to achieve a ``good'' approximation of the mean and the standard deviation, to update the mean and standard deviation estimates, and to choose and update a ``proper'' setting for the \textit{a priori} probability, where the mean, the standard deviation and the \textit{a priori} probability are the three fundamental variables used in the Bayes formula. 

  
%To address these problems, we combine several supporting techniques with Bayesian thresholding in RADIUS for achieving an overall better system performance. To avoid high detection error rate due to insufficient training samples, a \textit{minimal training set estimation} technique is employed together with a sliding-window \textit{data smoothing} procedure to cope with the inherent randomness of the RSSI signal. In addition, \textit{self-adaptation} and \textit{system feedback-based adaptation} techniques are introduced to make RADIUS robust to environment changes. 

A prototype of the RADIUS system has been implemented and deployed in an indoor testbed. For proper configuration of the parameters involved in the various techniques in RADIUS, suggestions on their settings are given based on extensive experiments. In addition, we found that high detection accuracy can be achieved by RADIUS under diverse link conditions more robustly as compared to the CDF and Chebyshev thresholding techniques. Moreover, the overhead analysis and the detection results show that RADIUS not only has low overheads in memory, communication and computation, but also is accurate in detecting link quality anomalies for all links across the network, maintaining a stable error rate of 6.13\% on average. These are an indication of RADIUS in fulfilling the requirements. 


%The performance of a Wireless Sensor Network (WSN) often deteriorates after in-situ deployment of the network \cite{1182885, 4408504, 6850017, 6661323}. Link quality degradation, due to, e.g., fading and interference, is one of the major reported causes behind such behavior. Hence, detecting anomalous degradation in link quality, which may be significant enough to impact the link's performance (e.g., the packet delivery ratio), is crucial for an operational WSN to maintain its performance and continuously satisfy the user's requirements.

%%The performance of a Wireless Sensor Network (WSN) often degrades after in-situ deployment of the network \cite{1182885, 4408504, 6850017, 6661323}. Anomalous link quality degradation is one of the major reported causes, because the environment as well as fading and interference effects have a significant impact on radio links. The problem is exacerbated in WSNs where low power communication is often used to reduce energy consumption. Indeed, to maintain the performance of an operational WSN and continuously satisfy the user requirements, it is crucial to monitor the link quality of each link and detect anomalous quality degradation that may deteriorate the link performance, e.g., the packet delivery ratio (PDR). 

%Many WSN techniques can benefit from the detection of anomalous link quality degradation. Blacklisting techniques \cite{2187194} typically involve thresholds during the link estimation process, in which deciding the best link quality threshold to discriminate good links from weak links is critical. In addition, as an indicator of a good link turning into a weak link, anomalous link quality degradation can serve as a good trigger for taking remedy actions such as tuning stack parameters including transmission power \cite{Lin:atpc} or other layer parameters \cite{2185730,7164923}. In this paper, we focus on the detection of anomalous link quality degradation. 

  
%In resource constrained WSNs, detecting an anomalous link quality degradation requires lightweight solutions with minimum error rates and robust detection performance. First of all, considering the limited resources of sensor nodes, the detection system must have a \textbf{low overhead} in the use of memory, computation and communication resources. Therefore, both centralized monitoring systems \cite{6661323, 1367278, 1267061} and powerful machine learning-based detection techniques \cite{4085803, 4289308, 5356174} are hardly applicable to WSNs due to either large communication or computation overhead.  Furthermore, achieving a \textbf{minimal detection error rate} is challenging in WSNs due to the stochastic nature of link quality metrics, e.g., the received signal strength indicator (RSSI) \cite{2893729}. The difficulty resides in discriminating true link quality anomalies from normal randomness. Data smoothing\cite{6199865} may only mitigate the problem. CDF-based \cite{4068315, 6199865} and Chebyshev inequality-based \cite{1689248, 1592596, 1515559} statistical techniques are effective to identify thresholds of the monitored attributes with low overhead, however, such thresholds are typically not optimized to achieve a minimum detection error rate. Finally, the accuracy in detecting link quality anomalies must be \textbf{robust} across the whole network without being severely affected by either the diversity of link conditions
%the choice of the involved parameters 
%or the dynamics of the operational environment.



%Many existing network monitoring systems rely on active collection of network status \cite{6850017, 6661323, 1367278, 1267061}, introducing large communication overhead. On the other hand, powerful detection techniques such as data mining or machine learning \cite{4085803, 4289308, 5356174} own a good detection accuracy but also a high computational complexity. This makes them hardly applicable to WSNs where distributed anomaly detection schemes are required in order to minimize the communication overhead.


%Furthermore, achieving a \emph{minimal detection error} is challenging in WSNs due to the stochastic nature of typical %wireless 
%link quality metrics such as the received signal strength indicator (RSSI) \cite{2893729}. The difficulty resides in discriminating true link quality anomalies from normal randomness. Data smoothing\cite{6199865} may only mitigate the problem. CDF-based \cite{4068315, 6199865} and Chebyshev inequality-based \cite{1689248, 1592596, 1515559} statistical techniques are widely applied to WSNs because they can effectively identify thresholds of the monitored attributes with low overhead. However, such thresholds are typically not optimized to achieve a minimum detection error. 

%Finally, the accuracy in detecting link quality anomalies must be \emph{robust} across the whole network without being severely affected by either the diversity of link conditions
%%the choice of the involved parameters 
%or the dynamics of the operational environment. We show that the aforementioned CDF- and Chebyshev-based techniques heavily rely on the parameter choice. In other words, achieving high accuracy globally over the whole network with these two methods requires a fine-tuning process for each individual link, incurring a huge overhead for large networks. Moreover, during the system lifetime, environment changes typically force a new calibration of the system parameters to sustain a required detection accuracy.

%To meet the above-mentioned requirements, this paper proposes RADIUS, a novel approach to detect anomalies in (but not limited to) WSNs link quality, as observed through RSSI measurements. To achieve its goals, RADIUS bases on the Bayesian decision theory (\textit{Bayesian thresholding}, in particular) to minimize the error rate in distinguishing between good and weak links. As to be discussed later, the computation of the Bayes thresholds only relies on a user-defined parameter (\textit{a priori} probability of a good link) and the statistical measures (mean and standard deviation) of the RSSI values measured at a good link, incurring as low computational complexity as CDF-based or Chebyshev-based thresholding techniques. 

%%Under the assumption of a Gaussian channel, the complexity of the approach decreases significantly, maintaining a minimum detection error. 

%To the best of our knowledge, no prior work has investigated the applicability of Bayesian thresholding in detecting link quality anomalies in WSNs. We believe the reason is that there are several important problems that should be solved first before applying it. First, we need to know how many RSSI samples lead to a good approximation of the mean and standard deviation. Second, since the \textit{a priori} probability is a predefined parameter, we need to investigate the performance under a coarse parameter choice. Third, we need to update the mean, standard deviation as well as the \textit{a priori} probability to achieve robust detection performance. 


%To address these problems, we combine several supporting techniques with Bayesian thresholding in RADIUS for achieving an overall better system performance. To avoid high detection error rate due to insufficient training samples, a \textit{minimal training set estimation} technique is employed together with a sliding-window \textit{data smoothing} procedure to cope with the inherent randomness of the RSSI signal. In addition, \textit{self-adaptation} and \textit{system feedback-based adaptation} techniques are introduced to make RADIUS robust to environment changes. 


%We implement the RADIUS system and evaluate it in an indoor testbed. We show the high detection accuracy can be achieved under diverse link conditions as compared to CDF-based and Chebyshev-based thresholding techniques. Through extensive experiments, we identify a proper configuration of involved system parameters for indoor environments. In addition, we show that RADIUS is accurate in detecting link quality anomalies across the whole network, maintaining a stable error rate of 6.13\% on average.

%In a comparison with reference thresholding techniques, we demonstrate that the Bayesian thresholding can achieve minimal detection error without being sensitive to its parameters.
%In an extensive experimentation of RADIUS in an indoor testbed, we investigate the effect of each system parameter on the detection performance. The results show that while the high accuracy achieved by Bayesian thresholding is robust to the setting of its threshold parameter, employing the aforementioned supporting techniques increase the overall performance of RADIUS significantly. In addition, in an operational system, we show that RADIUS is accurate in detecting link quality anomalies across the whole network, achieving a stable low detection error of 6.13\% on average.

The rest of the paper is organized as follows. Section \ref{sec:related} discusses the related work. Section \ref{sec:system} presents the system design and motivates the adoption of Bayesian thresholding. Section \ref{sec:approach} introduces the key techniques used in RADIUS. Section \ref{sec:parameterChoice} analyzes the effect of the various involved parameters in these RADIUS techniques on the detection performance. Section \ref{sec:imp&eva} reports the details of our implementation, the corresponding system overheads and the overall performance evaluation in an operational system on an indoor testbed. Finally, Section \ref{sec:conclusion} concludes the paper.



% The performance of a wireless sensor network (WSN) often degrades after in-situ deployment of the network \cite{4408504, 6850017, 6661323}. Among various causes, anomalous link quality degradation is one of the most reported causes, because radio links are subject to environment, fading and interference. The problem is exacerbated in WSNs where low transmission power is often used to reduce the energy consumption. Indeed, to maintain the performance of WSNs, it is crucial to monitor the link quality of each link and detect anomalous link quality degradation that may deteriorate the performance of the link, such as the packet delivery ratio (PDR).

% In resource constrained WSNs, detecting anomalous link quality degradations is not as simple as one might imagine. A key challenge is to design a lightweight approach that provides robust detection performance of minimum detection error rate. 
% %i.e., the approach is optimized for accuracy while being robust so that the high accuracy can be achieved fairly for different deployments and it does degrade over time. 
% Although there have been many studies on the network performance diagnosis \cite{6850017, 6661323, 1367278, 1267061} and  anomaly detection techniques \cite{4085803, 4289308, 5356174, 4068315, 6199865, 1592596, 1689248, 1515559}, most of the existing approaches only partially tackle the challenge and do not provide all the necessary features at the same time. To achieve the optimal performance of detecting anomalous link quality degradation in sensor networks, the detection approach must have the following important features. %Those  

% (i) Low overhead. Due to the limited resources of sensor nodes, the detection system must not incur large memory, computation and communication overhead. Many existing network monitoring systems rely on active collection of network status \cite{6850017, 6661323, 1367278, 1267061}, introducing large communication overhead to the energy constrained sensor nodes. On the other hand, powerful detection techniques such data mining or machine learning-based techniques \cite{4085803, 4289308, 5356174} own a good detection accuracy with a cost of high computational complexity, hences are not applicable to sensor nodes if distributive anomaly detection scheme is employed for minimizing communication overhead.

% (ii) High accuracy. 
% %Due to the high cost of a wrong detection decision, it is critical to keep the detection error as low as possible. %Minimizing the detection error includes reducing a false positive rate (the ratio of legitimate behavior falsely identified as an anomaly) and a false negative rate (ratio of miss of capturing a real anomaly). 
% To achieve minimal detection error is challenging in WSNs due to the stochastic nature of a wireless link's quality\cite{1062741}. The difficulty is to distinguish true link quality anomalies from normal randomness. Smoothing\cite{6199865} may mitigate the randomness but does not solve the problem of minimizing the detection error. CDF-based \cite{4068315, 6199865} or Chebyshev inequality-based \cite{1592596, 1689248, 1515559} statistical techniques are widely applied to WSNs because they are effective to find thresholds of the monitored attributes with low overhead, however, their thresholds are normally not optimized to achieve the minimum detection error. 

% (iii) High robustness. The meaning of robustness is two fold. First, the high accuracy must hold for detecting anomalies across the whole network without incurring heavy overhead of tuning involved parameters. As to be discussed in Section \ref{sec:motivationBayes}, the literature-adopted  techniques heavily rely on the parameter choice and achieving high accuracy globally over the network must require a fine-tuning process for each individual link, incurring a huge overhead for large scale networks. Moreover, the detection performance must not degrade over time due to dynamic environment changes.

% This paper proposes RADIUS, a novel approach to detect anomalous degradations in (but not limited to) link quality described by RSSI (received signal strength indicator), which incurs low overhead yet provides robust detection performance of minimal error. To achieve this goal, we base RADIUS on the Bayes decision theory by employing \textit{Bayesian thresholding} technique as our core detection technique. Under the assumption of Gaussian channel, the complexity of Bayesian thresholding decreases significantly while the detection error is mathematically minimized. To the best of our knowledge, RADIUS is the first approach of applying Bayesian thresholding to detect link quality anomalies in WSNs. By comparing its performance with two statistical thresholding techniques, we discover a distinguishing property of Bayes threshold: its achieved minimal detection error is not sensitive to its parameter choice, contributing to both the accuracy and robustness of the system.

% In addition, we employ several supporting techniques in RADIUS to achieve a better system performance. To avoid high detection error due to insufficient training, a \textit{minimal training set estimation} technique is employed together with a sliding-window \textit{data smoothing} technique to copy with the normal randomness naturally existing in the RSSI signal. In addition, \textit{self-adaptation} and \textit{system feedback-based adaptation} techniques are introduced in RADIUS to enable its adaptation to dynamic environment changes. %Extensive experimental results show that the employed supporting techniques increase significantly the overall system performance. 

% To get the best performance out of RADIUS, we investigate the effect of each system parameter on the detection performance based on extensive experiments, giving insights on the setting of them for indoor environments. The results show that while the high accuracy achieved by Bayesian thresholding is robust to its parameter setting, employing the aforementioned supporting techniques increase the overall performance of RADIUS significantly with proper setting of the involved parameters. In addition, the performance of RADIUS is evaluated, based on its implementation on a testbed. The results show that RADIUS is accurate for detecting link quality anomalies of all links across the network, achieving a durable low detection error of 6.13\% on average.

%%%%% contributions revised from SECON version %%%%%%%%%%%%
%In summary, the contributions of this paper are several-fold: (1) A novel approach, called RADIUS, is proposed for anomalous communication quality degradation detection in WSNs, which is light weight, robust and accurate. Specifically, RADIUS lays its foundation on Bayes decision theory, together with a set of supporting techniques. (2) We investigate the effectiveness of Bayesian thresholding in detecting link quality anomaly. In addition, we substantiate the adoption of Bayesian thresholding by comparing its detection performance to two statistical thresholding techniques. To the best of our knowledge, we are the first one to apply Bayesian thresholding to detect anomalous link quality degradation. (3) The effect of each system parameter is detailedly investigated, based on which suggested settings of the involved parameters are provided. (4) The performance of RADIUS is evaluated on a WSN testbed, showing that RADIUS is robust to dynamic changes and achieves appealing performance. 

%%%%% contributions in SECON version %%%%%%%%%%%%
%In summary, the contributions of this paper are several-fold: (1) A novel approach, called \textit{RADIUS}, is proposed for anomalous communication quality degradation detection in WSNs, which is simple, robust, adaptive and accurate. Specifically, \textit{RADIUS} is a threshold-based approach, having foundation on Bayes decision theory, together with a smoothing technique to cope with the normal randomness in detection. In addition, \textit{RADIUS} adopts self-adaptation and system feedback-based adaptation to deal with normal dynamic changes in the system. (2) The effect of each involved parameter is detailedly investigated, based on which suggested settings of the involved parameters are provided. (3) A system implementation of \textit{RADIUS} is introduced, including the key functional modules as well as the interactions among them. (4) The performance of \textit{RADIUS} is evaluated on a WSN testbed, showing that \textit{RADIUS} is robust to dynamic changes and achieves appealing performance. (5) A comparison of the Bayes threshold technique with the two state-of-the-art threshold techniques is presented, which substantiates the adoption of the Bayes. 


% The rest of this paper is organized as follows. Sec. \ref{sec:related} reviews related work. Sec. \ref{sec:system} presents the system design and motivates the adoption of Bayesian thresholding. Sec. \ref{sec:approach} introduces the details of the RADIUS approach. Sec. \ref{sec:parameterChoice} analyzes the effect of system parameters on the detection performance. Sec. \ref{sec:imp&eva} introduces the implementation details, system overhead and its performance evaluation on an indoor testbed. Finally, Sec. \ref{sec:conclusion} concludes the paper.



%%%%%%%%%%%%%%%%%%%%%%%%%%%%%%%%%%% SECON version %%%%%%%%%%%%%%%%%%%%%%%%%%%%%%%%%%%%%%%%%%	
	
%The performance (e.g., packet delivery ratio) of a wireless sensor network (WSN) often degrades after in-situ deployment of the network \cite{6850017} \cite{4408504} \cite{6661323}. The reason for this is that the WSN may encounter various node failure and communication degradation problems that are not or cannot be detected during the deployment. Indeed, to ensure proper functioning of \textit{any} computer network, it is necessary to monitor the communication quality of each link and detect anomalous degradations or failures on the link so that the network can adapt accordingly. Specifically, it is crucial to detect anomalous link quality degradation events that may cause significant, negative impact on the user-experienced performance of the link, such as the packet delivery ratio (PDR) of the link falls below a user-defined value. 
%
%In WSNs, detecting anomalous link quality degradations is not as simple as one might imagine. This is mainly due to the special characteristics of WSNs. Specifically, limited by the processing capability, simple algorithms are preferred. In addition, due to the stochastic nature, a wireless link's quality can vary highly randomly. Moreover, constrained by the energy supply (e.g. only battery is available), there is a typical need of using minimal transmission power, which contributes additionally to the randomness of the received signal strength, the most direct and fundamental factor measuring the link quality. Essentially, the challenge is to develop a simple algorithm that can detect link quality degradations, which are potential causes of performance anomalies in the network, from the received link quality indicator signal, RSSI (received signal strength indicator), which is random by nature. In other words, the difficulty is to distinguish true causes of such network anomalies from normal randomness. Another part of the challenge is that the algorithm should be able to adapt to (normal) dynamic changes (e.g. due to environmental condition changes). 
%
%The objective of this work is to propose a novel approach to detect anomalous degradations in link quality described by RSSI, which is simple, robust and adaptive. We call the proposed approach \textit{RADIUS}. 
%
%Specifically, \textit{RADIUS} is a simple threshold-based approach. In the literature, threshold techniques have been adopted for similar purposes, owing to their simplicity in implementation. For example, in \cite{4068315}, a timeout threshold based on the \textit{x-th percentile} of the sample set of consecutively missing heartbeats was used for detecting node failures. In \cite{1592596}, one-side Chebyshev inequality with a chosen \textit{target false positive rate} is applied to detect performance problems in the network. In \cite{1515559}, the proposed approach uses a threshold value $k$ to detect abnormal traffic pattern if the short term mean of inter-arrival times of packets is $k$ \textit{times} of the standard deviation away from the long term mean. The threshold method in \cite{1515559} is essentially a special case of the Chebyshev threshold method, based on Chebyshev inequality with an implicit maximum false rate $1/k^2$. In all these threshold-based approaches, finding appropriate thresholds is crucial to achieving high detection accuracy. However, as to be discussed in Section \ref{sec:ThresholdCompare}, the detection performance based on the literature-adopted threshold techniques is so highly sensitive to the setting of the threshold value that it is difficult to conclude a ``good enough'' value from empirical study to ensure robust performance. Another consequence of such high sensitivity is that it is even more difficult to adapt these approaches to cope with dynamic changes, e.g. changes in humidity and temperature and changes in the deployment surrounding area, which are natural or normal in real networks. 
%
%
%To address the threshold sensitivity challenge and consequently make the proposed approach robust, we base \textit{RADIUS} on the Bayes decision theory, together with a sliding-window smoothing technique to copy with the normal randomness naturally existing in the RSSI signal. In addition, self-adaptation and system feedback-based adaptation techniques are introduced in \textit{RADIUS} to enable its adaptation and hence increased robustness to dynamic changes. 
%
%Owing to its robustness, an appealing property of \textit{RADIUS} is that the involved parameters can be easily set. This paper also presents results based on extensive experiments to investigate the effect of the involved parameters on the detection performance, which gives insights on the setting of them. In addition, the performance of \textit{RADIUS} is evaluated, based on its implementation on a testbed. The results show that \textit{RADIUS} is robust to different conditions and is accurate, achieving a detection error rate as low as 6.13\% on average on the testbed. 
%
%In summary, the contributions of this paper are several-fold: (1) A novel approach, called \textit{RADIUS}, is proposed for anomalous communication quality degradation detection in WSNs, which is simple, robust, adaptive and accurate. Specifically, \textit{RADIUS} is a threshold-based approach, having foundation on Bayes decision theory, together with a smoothing technique to cope with the normal randomness in detection. In addition, \textit{RADIUS} adopts self-adaptation and system feedback-based adaptation to deal with normal dynamic changes in the system. (2) The effect of each involved parameter is detailedly investigated, based on which suggested settings of the involved parameters are provided. (3) A system implementation of \textit{RADIUS} is introduced, including the key functional modules as well as the interactions among them. (4) The performance of \textit{RADIUS} is evaluated on a WSN testbed, showing that \textit{RADIUS} is robust to dynamic changes and achieves appealing performance. (5) A comparison of the Bayes threshold technique with the two state-of-the-art threshold techniques is presented, which substantiates the adoption of the Bayes. 
%
%
%The rest of this paper is organized as follows. Section \ref{sec:approach} introduces the \textit{RADIUS} approach. Section \ref{sec:parameterChoice} investigates the effect of each involved parameter on the detection performance of \textit{RADIUS}, and based on the investigation, gives suggestion on the setting of the parameter. Section \ref{sec:imp&eva} introduces an implementation of \textit{RADIUS} and its performance evaluation on a testbed WSN. Section \ref{sec:ThresholdCompare} compares the Bayes-based threshold technique with two other threshold techniques, which has motivated the adoption of Bayes theorem in \textit{RADIUS}. Finally, Section \ref{sec:conclusion} concludes the paper. 





