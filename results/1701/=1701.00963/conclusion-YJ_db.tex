\section{Conclusion} \label{sec:conclusion}

This paper presents RADIUS, a system for detecting anomalous link quality degradations in low-power radio links. The RADIUS system is light-weight, accurate and robust to a diversity of link conditions and dynamic environment changes. To achieve this, RADIUS (1) lays its foundation on a Bayesian thresholding scheme, integrated with dedicated techniques for (2) minimal training set size estimation, (3) sliding-window data smoothing, (4) distribution self-adaptation, and (5) feedback-based threshold parameter adaptation. The comparison with two popular statistical approaches shows that RADIUS does not need fine-tuning of its threshold parameter to achieve near-optimal accuracy across the network. The impact of the system parameters is also investigated in detail, identifying the best configuration for an indoor environment. Moreover, we have implemented the RADIUS system and evaluated its performance on an indoor WSN testbed, showing that it can adapt to dynamic environment changes and achieve accurate detection over the entire network with an average error rate of 6.13\%.


%Many WSN techniques can benefit from the detection of anomalous link quality degradation. Blacklisting techniques \cite{2187194} typically involve thresholds during the link estimation process, in which deciding the best link quality threshold to discriminate good links from weak links is critical. In addition, as an indicator of a good link turning into a weak link, anomalous link quality degradation can serve as a good trigger for taking remedy actions such as tuning stack parameters including transmission power \cite{Lin:atpc} or other layer parameters \cite{2185730,7164923}. In this paper, we focus on the detection of anomalous link quality degradation. 


%%%%%% SECON text%%%%%%%%%
%In this paper, we propose \textit{RADIUS} for anomaly detection in WSNs, which, while being a simple threshold-based approach, is robust, adaptive and accurate. This is achieved by laying its (1) foundation on the Bayes theorem, integrated with (2) a sliding-window data smoothing technique, (3) a self-adaptation technique, and (4) a system feedback-based adaptation technique. Among them, (1) and (2) are the main contributors to RADIUS's robustness, while (3) and (4) make RADIUS adaptive. Through extensive experimental study, it is found that the performance of RADIUS is insensitive or is robust to the setting of most of the involved parameters, in the sense that there exist (broad) value regions of these parameters where the performance remains stable. This appealing property has led to easy parameter-setting of RADIUS. In addition, a system implementation and evaluation on a WSN testbed show that RADIUS can adapt to dynamic changes and achieve accurate detection with an average error rate as low as 6.13\% on the testbed. Moreover, the detection sensitivity, with respect to the change of the threshold parameter, of the Bayes threshold technique has been compared with those of the percentile-based and the Chebyshev inequality-based, two state-of-the-art threshold techniques, substantiating the adoption of the Bayes technique in RADIUS. 




