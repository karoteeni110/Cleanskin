\subsection{Related Work} \label{sec:survey}

The outlier analysis problem has been studied extensively in the
literature \cite{outlierbook,chandola,hawkins}.  Numerous algorithms
have been proposed in the literature for outlier detection of
conventional multidimensional data \cite{hd,lof,knorr,rama}. The key
methods, which are used  frequently for outlier analysis include
distance-based methods \cite{knorr,rama}, density-based methods
\cite{lof}, and subspace methods \cite{hd,keller,laz,muller,zimek}.
In distance-based methods, data points are declared outliers, when
they are situated far away from  the dense regions in the underlying
data. Typically, indexing or other summarization schemes may be used
in order to improve the efficiency of the approach. In density-based
methods \cite{lof},   data points with low local density with
respect to the remaining points are declared outliers. In addition,
a number of subspace methods \cite{hd,keller,laz,muller,zimek} have
been proposed recently, in which outliers are defined on the basis
of subspace behavior of the underlying data.

Most of the traditional multidimensional
methods \cite{chandola,outlierbook} can also be extended to text
data, though they are not particularly suited to the latter. Some
methods have been designed for outlier detection with matrix
factorization in network data sets \cite{tong}, that
are not applicable to text data. Text data is uniquely
difficult because of its sparse and high dimensional nature.  As a
result, many of the outliers detected using conventional methods may
simply correspond to noisy text segments. Therefore, careful
modeling is required with the use of matrix factorization methods.

Over the last decade, Non-negative Matrix Factorization (NMF) has
emerged as another important low rank approximation technique, where
the low-rank factor matrices are constrained to have only
non-negative elements. Lee and Seung \cite{Lee1999} introduced a
multiplicative update based low rank approximation with non-negative
factors to overcome the challenges of truncated SVD. Subsequent to
this work, NMF has received enormous attention and has been
successfully applied to a broad range of important problems in areas
including computer vision, community detection in
social networks, visualization, recommender systems bioinformatics,
etc. In spite of broad range of applications,
NMF's literature in text domain is scarce. Xu {\em et. al.}
\cite{Xu2003} experimented with NMF for document clustering instead
of SVD based Latent Semantic Indexing (LSI). Other than applications
of NMF in the  text domain, Gaussier and Goutte \cite{Gaussier2005}
established the equivalence between NMF and pLSA. Similarly, Ding
{\em et. al.} \cite{Ding2006}  explained the equivalence between NMF
and pLSI.

In this paper, we use an NMF approach for concise modelling of the
patterns, the background, and the anomalies in the underlying data.
It should be pointed out that NMF is similar to the generative
models of text such as pLSI and LDA \cite{Gaussier2005}
\cite{Ding2006} \cite{Singh2008}, though NMF often provides better
interpretability. Our important challenge is to model the outliers
along with the low rank space of the input matrix. We identified
$\ell_{1,2}$-norm as an appropriate approach for factorization in
outlier analysis. Recently, the researchers have used  $\ell_{2,1}$-norm in their models to solve various problems, though the
corresponding solution techniques are not easily generalizable to
the $\ell_{1,2}$-norm. Yang et.al., \cite{Yang2011}, under the
assumption that the class label of input data can be predicted by a
linear classifier, incorporate discriminative analysis and
$\ell_{2,1}$-norm minimization into a joint framework for
unsupervised feature selection problem. Similarly, Liu {\em et al}
\cite{Liu2009}, solve $\ell_{2,1}$-norm regularized regression model
for joint feature selection from multiple tasks. They also propose
to use Nesterov's method to solve the optimization problem with
non-smooth $\ell_{2,1}$-norm regularization. Also, Kong {\em et al}
\cite{Kong2011} propose a robust formulation of NMF using
$\ell_{2,1}$-norm loss function for data with noises.
\subsection{Our Contributions}
Text data is uniquely challenging to outlier detection both because
of its sparsity and high dimensional nature.  Given the relevant
literature for NMF and text outliers, we propose the first approach
to detect outliers in text data using non-negative matrix
factorization. We extend the fact that NMF is similar to pLSI and
LDA generative models and model the outliers using the
$\ell_{1,2}$-norm.  This particular formulation of NMF is
non-standard, and requires careful design of optimization methods to
solve the problem. We solve the resulting optimization problem using
block coordinate descent technique. We also present extensive
experimental results both on text and other kinds of market basket
data sets. We show significant improvements achieved by the approach
over other baseline methods.
