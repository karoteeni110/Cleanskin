Multisplit is a broadly useful parallel primitive that permutes its input data into contiguous \emph{buckets} or \emph{bins}, where the function that categorizes an element into a bucket is provided by the programmer.
Due to the lack of an efficient multisplit on GPUs, programmers often choose to implement multisplit with a sort.
One way is to first generate an auxiliary array of bucket IDs and then sort input data based on it.
In case smaller indexed buckets possess smaller valued keys, another way for multisplit is to directly sort input data.
Both methods are inefficient and require more work than necessary: the former requires more expensive data movements while the latter spends unnecessary effort in sorting elements within each bucket.
In this work, we provide a parallel model and multiple implementations for the multisplit problem. Our principal focus is multisplit for a small (up to 256) number of buckets.
We use warp-synchronous programming models and emphasize warp-wide communications to avoid branch divergence and reduce memory usage.
We also hierarchically reorder input elements to achieve better coalescing of global memory accesses.
On a GeForce GTX 1080 GPU, we can reach a peak throughput of 18.93 Gkeys/s (or 11.68 Gpairs/s) for a  key-only (or key-value) multisplit.
Finally, we demonstrate how multisplit can be used as a building block for radix sort. In our multisplit-based sort implementation, we achieve comparable performance to the fastest GPU sort routines, sorting 32-bit keys (and key-value pairs) with a throughput of 3.0 G~keys/s (and 2.1 Gpair/s).
