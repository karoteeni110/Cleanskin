
\documentclass{article}
%%%%%%%%%%%%%%%%%%%%%%%%%%%%%%%%%%%%%%%%%%%%%%%%%%%%%%%%%%%%%%%%%%%%%%%%%%%%%%%%%%%%%%%%%%%%%%%%%%%%%%%%%%%%%%%%%%%%%%%%%%%%%%%%%%%%%%%%%%%%%%%%%%%%%%%%%%%%%%%%%%%%%%%%%%%%%%%%%%%%%%%%%%%%%%%%%%%%%%%%%%%%%%%%%%%%%%%%%%%%%%%%%%%%%%%%%%%%%%%%%%%%%%%%%%%%

\ifdefined\COMSOC%\ifCLASSOPTIONcompsoc
  \usepackage[nocompress]{cite}%1,2,3,4 will not be compressed as 1-4
\else
  \usepackage{cite} % normal IEEE
\fi
\ifdefined\ACM %ACM has Option clash for package graphicx
	\usepackage{amsmath}
\else
	\usepackage[cmex10]{amsmath} %ACM conflicts with cmex10 option
	\usepackage{amsthm} %ACM conflicts with this on \proof
%\ifCLASSINFOpdf
	\usepackage[pdftex]{graphicx} %\else \usepackage[dvips]{graphicx} \fi
\fi

\ifdefined\META     \usepackage{stmaryrd}    \fi %for \rrbracket

\newcommand{\begproof}{\ifdefined\dbcol\begin{IEEEproof}\else\begin{proof}\fi}
\newcommand{\Endproof}{\ifdefined\dbcol\end{IEEEproof}\else\end{proof}\fi}
\newcommand{\metacom}[1]{\ifdefined\META\bluepure{$\blacktriangleright$}#1\bluepure{$\rrbracket$}\fi}
\newcommand{\metafoot}[1]{\ifdefined\META\footnote{\bluepure{$\blacktriangleright$}#1}\fi}

%\usepackage{mdwmath}
%\usepackage{mdwtab}
% Also highly recommended is Mark Wooding's extremely powerful MDW tools,
% especially mdwmath.sty and mdwtab.sty which are used to format equations
% and tables, respectively. The MDWtools set is already installed on most
% LaTeX systems.

\usepackage[font=small]{subfig} %font=scriptsize/small/footnotesize
%--------- INFOCOM'14 camera-ready version used below -----------
%\usepackage[caption=false,textfont=sf,font=footnotesize]{subfig}
%\usepackage[font=small]{caption}%footnotesize
	%Package caption Warning: Unsupported document class (or package) detected, (caption)  usage of the caption package is not recommended.
%--------- INFOCOM'14 camera-ready version used above -----------

\usepackage{hyperref} %for special characters and smart line breaking in URL

%%%%%%%%%%%%% MY OWN ADDED PACKAGES %%%%%%%%%%%%%%%%%

\usepackage{amssymb}
\usepackage{bbm} %\mathbbm 1 %\newcommand{\ind}{1\hspace{-2.3mm}{1}}
\usepackage{verbatim}
\usepackage{color}
\usepackage[normalem]{ulem}%{soul}
%\usepackage{algorithmic}
%\usepackage{algorithm}
\usepackage[ruled,lined,linesnumbered]{algorithm2e}
%\usepackage{parskip}
%\usepackage{fullpage} %1-inch margin
%\usepackage[margin=1.5cm]{geometry}
%\usepackage[left=1.5cm,right=1.5cm,top=2cm,bottom=2cm,nohead]{geometry}%,nofoot
%\usepackage{anysize}
%\marginsize{2cm}{2cm}{1cm}{1cm} %\marginsize{left}{right}{top}{bottom}
%%%%%%%%%%%%%%%%%%%%%%%%%%%%%%%%%

% fontsize illustration: http://tex.stackexchange.com/questions/24599/what-point-pt-font-size-are-large-etc
%Command             10pt    11pt    12pt
%\tiny               5       6       6
%\scriptsize         7       8       8
%\footnotesize       8       9       10
%\small              9       10      10.95
%\normalsize         10      10.95   12
%\large              12      12      14.4
%\Large              14.4    14.4    17.28
%\LARGE              17.28   17.28   20.74
%\huge               20.74   20.74   24.88
%\Huge               24.88   24.88   24.88

\newtheorem{thm}{Theorem}
\newtheorem{lem}{Lemma}
\newtheorem{prop}{Proposition}
\newtheorem{coro}{Corollary}
\newtheorem{defn}{Definition}
\newtheorem{hypo}{Hypothesis}
\newtheorem{conj}{Conjecture}
\newtheorem{ex}{Example}
\newcommand{\eref}[1]{Eq.~\eqref{#1}} %IEEE style manual suggests using (1) only
\newcommand{\fref}[1]{Fig.~\ref{#1}}
\newcommand{\tref}[1]{Table~\ref{#1}}
\newcommand{\sref}[1]{Section~\ref{#1}}
\newcommand{\thmref}[1]{Theorem~\ref{#1}}
\newcommand{\lref}[1]{Lemma~\ref{#1}}
\newcommand{\pref}[1]{Proposition~\ref{#1}}
\newcommand{\cref}[1]{Corollary~\ref{#1}}
\newcommand{\dref}[1]{Definition~\ref{#1}}
%\newcommand{\href}[1]{Hypothesis~\ref{#1}}
\newcommand{\jref}[1]{Conjecture~\ref{#1}}
\newcommand{\aref}[1]{Algorithm~\ref{#1}}
\newcommand{\pcref}[1]{Procedure~\ref{#1}}
\newcommand{\vect}[1]{\boldsymbol{#1}}
\newcommand{\veclong}[1]{\overrightarrow{#1}} %but the arrow is a bit too big
\newcommand{\sumiN}{\sum_{i=1}^N}
\newcommand{\sumin}{\sum_{i=1}^n}
\newcommand{\suml}{\sum_{l=1}^N}
\newcommand{\sumk}{\sum_{k=1}^N}
\newcommand{\prodi}{\prod_{i=1}^N}
\newcommand{\blue}[1]{{\color{blue}\dotuline{#1}}}
\newcommand{\bluepure}[1]{{\color{blue}{#1}}}
%\newcommand{\blue}[1]{#1}
\newcommand{\red}[1]{{\color{red}\dashuline{#1}}}
%\newcommand{\red}[1]{#1}
\newcommand{\tts}[1]{\tt\small{#1}}
\newcommand{\ttscr}[1]{\tt\scriptsize{#1}}
\newcommand{\ttfoot}[1]{\tt\footnotesize{#1}}
\newcommand{\its}[1]{\it\small{#1}}
\newcommand{\itscr}[1]{\it\scriptsize{#1}}
\newcommand{\itfoot}[1]{\it\footnotesize{#1}}
\newcommand{\eqv}{\Leftrightarrow}%\iff is quite long
\newcommand{\means}{\Rightarrow}
\newcommand{\eps}{\epsilon}
\newcommand{\ovl}{\overline}
\newcommand{\nn}{\nonumber}
\newcommand{\opd}{\operatorname{d}\!}
\renewcommand{\d}[2]{\frac{\opd#1}{\opd#2}} % for derivatives
\newcommand{\dd}[2]{\frac{\opd^2\!#1}{\opd#2^2}} % for double derivatives
\newcommand{\pd}[2]{\frac{\partial #1}{\partial #2}} % for partial derivatives
\newcommand{\pdd}[2]{\frac{\partial^2 #1}{\partial {#2}^2}} % for double partial derivatives
\newcommand{\pddd}[3]{\frac{\partial^2 #1}{\partial {#2} \partial {#3}}}
\newcommand{\inv}[1]{\frac{1}{#1}} %inverse
\DeclareMathOperator*{\argmax}{arg\,max}
\DeclareMathOperator*{\argmin}{arg\,min}
\DeclareMathOperator\erf{erf}
\DeclareMathOperator*{\E}{\mathbb{E}}% Expectation symbol; Using the starred version of \DeclareMathOperator makes sure subscripts goes _beneath_ the symbol in display mode.

%\floatname{algorithm}{Procedure}
%\renewcommand{\algorithmicrequire}{\textbf{Input:}}
%\renewcommand{\algorithmicensure}{\textbf{Output:}}

\hyphenation{op-tical net-works semi-conduc-tor}



% Macros for Scientific Word 4.0 documents saved with the LaTeX filter.
% Copyright (C) 2002 Mackichan Software, Inc.

\typeout{TCILATEX Macros for Scientific Word 4.0 <12 Mar 2002>.}
\typeout{NOTICE:  This macro file is NOT proprietary and may be
freely copied and distributed.}
%
\makeatletter

%%%%%%%%%%%%%%%%%%%%%
% FMTeXButton
% This is used for putting TeXButtons in the
% frontmatter of a document. Add a line like
% \QTagDef{FMTeXButton}{101}{} to the filter
% section of the cst being used. Also add a
% new section containing:
%     [f_101]
%     ALIAS=FMTexButton
%     TAG_TYPE=FIELD
%     TAG_LEADIN=TeX Button:
%
% It also works to put \defs in the preamble after
% the \input tcilatex
\def\FMTeXButton#1{#1}
%
%%%%%%%%%%%%%%%%%%%%%%
% macros for time
\newcount\@hour\newcount\@minute\chardef\@x10\chardef\@xv60
\def\tcitime{
\def\@time{%
  \@minute\time\@hour\@minute\divide\@hour\@xv
  \ifnum\@hour<\@x 0\fi\the\@hour:%
  \multiply\@hour\@xv\advance\@minute-\@hour
  \ifnum\@minute<\@x 0\fi\the\@minute
  }}%

%%%%%%%%%%%%%%%%%%%%%%
% macro for hyperref and msihyperref
%\@ifundefined{hyperref}{\def\hyperref#1#2#3#4{#2\ref{#4}#3}}{}

\def\x@hyperref#1#2#3{%
   % Turn off various catcodes before reading parameter 4
   \catcode`\~ = 12
   \catcode`\$ = 12
   \catcode`\_ = 12
   \catcode`\# = 12
   \catcode`\& = 12
   \y@hyperref{#1}{#2}{#3}%
}

\def\y@hyperref#1#2#3#4{%
   #2\ref{#4}#3
   \catcode`\~ = 13
   \catcode`\$ = 3
   \catcode`\_ = 8
   \catcode`\# = 6
   \catcode`\& = 4
}

\@ifundefined{hyperref}{\let\hyperref\x@hyperref}{}
\@ifundefined{msihyperref}{\let\msihyperref\x@hyperref}{}




% macro for external program call
\@ifundefined{qExtProgCall}{\def\qExtProgCall#1#2#3#4#5#6{\relax}}{}
%%%%%%%%%%%%%%%%%%%%%%
%
% macros for graphics
%
\def\FILENAME#1{#1}%
%
\def\QCTOpt[#1]#2{%
  \def\QCTOptB{#1}
  \def\QCTOptA{#2}
}
\def\QCTNOpt#1{%
  \def\QCTOptA{#1}
  \let\QCTOptB\empty
}
\def\Qct{%
  \@ifnextchar[{%
    \QCTOpt}{\QCTNOpt}
}
\def\QCBOpt[#1]#2{%
  \def\QCBOptB{#1}%
  \def\QCBOptA{#2}%
}
\def\QCBNOpt#1{%
  \def\QCBOptA{#1}%
  \let\QCBOptB\empty
}
\def\Qcb{%
  \@ifnextchar[{%
    \QCBOpt}{\QCBNOpt}%
}
\def\PrepCapArgs{%
  \ifx\QCBOptA\empty
    \ifx\QCTOptA\empty
      {}%
    \else
      \ifx\QCTOptB\empty
        {\QCTOptA}%
      \else
        [\QCTOptB]{\QCTOptA}%
      \fi
    \fi
  \else
    \ifx\QCBOptA\empty
      {}%
    \else
      \ifx\QCBOptB\empty
        {\QCBOptA}%
      \else
        [\QCBOptB]{\QCBOptA}%
      \fi
    \fi
  \fi
}
\newcount\GRAPHICSTYPE
%\GRAPHICSTYPE 0 is for TurboTeX
%\GRAPHICSTYPE 1 is for DVIWindo (PostScript)
%%%(removed)%\GRAPHICSTYPE 2 is for psfig (PostScript)
\GRAPHICSTYPE=\z@
\def\GRAPHICSPS#1{%
 \ifcase\GRAPHICSTYPE%\GRAPHICSTYPE=0
   \special{ps: #1}%
 \or%\GRAPHICSTYPE=1
   \special{language "PS", include "#1"}%
%%%\or%\GRAPHICSTYPE=2
%%%  #1%
 \fi
}%
%
\def\GRAPHICSHP#1{\special{include #1}}%
%
% \graffile{ body }                                  %#1
%          { contentswidth (scalar)  }               %#2
%          { contentsheight (scalar) }               %#3
%          { vertical shift when in-line (scalar) }  %#4

\def\graffile#1#2#3#4{%
%%% \ifnum\GRAPHICSTYPE=\tw@
%%%  %Following if using psfig
%%%  \@ifundefined{psfig}{\input psfig.tex}{}%
%%%  \psfig{file=#1, height=#3, width=#2}%
%%% \else
  %Following for all others
  % JCS - added BOXTHEFRAME, see below
    \bgroup
       \@inlabelfalse
       \leavevmode
       \@ifundefined{bbl@deactivate}{\def~{\string~}}{\activesoff}%
        \raise -#4 \BOXTHEFRAME{%
           \hbox to #2{\raise #3\hbox to #2{\null #1\hfil}}}%
    \egroup
}%
%
% A box for drafts
\def\draftbox#1#2#3#4{%
 \leavevmode\raise -#4 \hbox{%
  \frame{\rlap{\protect\tiny #1}\hbox to #2%
   {\vrule height#3 width\z@ depth\z@\hfil}%
  }%
 }%
}%
%
\newcount\@msidraft
\@msidraft=\z@
\let\nographics=\@msidraft
\newif\ifwasdraft
\wasdraftfalse

%  \GRAPHIC{ body }                                  %#1
%          { draft name }                            %#2
%          { contentswidth (scalar)  }               %#3
%          { contentsheight (scalar) }               %#4
%          { vertical shift when in-line (scalar) }  %#5
\def\GRAPHIC#1#2#3#4#5{%
   \ifnum\@msidraft=\@ne\draftbox{#2}{#3}{#4}{#5}%
   \else\graffile{#1}{#3}{#4}{#5}%
   \fi
}
%
\def\addtoLaTeXparams#1{%
    \edef\LaTeXparams{\LaTeXparams #1}}%
%
% JCS -  added a switch BoxFrame that can
% be set by including X in the frame params.
% If set a box is drawn around the frame.

\newif\ifBoxFrame \BoxFramefalse
\newif\ifOverFrame \OverFramefalse
\newif\ifUnderFrame \UnderFramefalse

\def\BOXTHEFRAME#1{%
   \hbox{%
      \ifBoxFrame
         \frame{#1}%
      \else
         {#1}%
      \fi
   }%
}


\def\doFRAMEparams#1{\BoxFramefalse\OverFramefalse\UnderFramefalse\readFRAMEparams#1\end}%
\def\readFRAMEparams#1{%
 \ifx#1\end%
  \let\next=\relax
  \else
  \ifx#1i\dispkind=\z@\fi
  \ifx#1d\dispkind=\@ne\fi
  \ifx#1f\dispkind=\tw@\fi
  \ifx#1t\addtoLaTeXparams{t}\fi
  \ifx#1b\addtoLaTeXparams{b}\fi
  \ifx#1p\addtoLaTeXparams{p}\fi
  \ifx#1h\addtoLaTeXparams{h}\fi
  \ifx#1X\BoxFrametrue\fi
  \ifx#1O\OverFrametrue\fi
  \ifx#1U\UnderFrametrue\fi
  \ifx#1w
    \ifnum\@msidraft=1\wasdrafttrue\else\wasdraftfalse\fi
    \@msidraft=\@ne
  \fi
  \let\next=\readFRAMEparams
  \fi
 \next
 }%
%
%Macro for In-line graphics object
%   \IFRAME{ contentswidth (scalar)  }               %#1
%          { contentsheight (scalar) }               %#2
%          { vertical shift when in-line (scalar) }  %#3
%          { draft name }                            %#4
%          { body }                                  %#5
%          { caption}                                %#6


\def\IFRAME#1#2#3#4#5#6{%
      \bgroup
      \let\QCTOptA\empty
      \let\QCTOptB\empty
      \let\QCBOptA\empty
      \let\QCBOptB\empty
      #6%
      \parindent=0pt
      \leftskip=0pt
      \rightskip=0pt
      \setbox0=\hbox{\QCBOptA}%
      \@tempdima=#1\relax
      \ifOverFrame
          % Do this later
          \typeout{This is not implemented yet}%
          \show\HELP
      \else
         \ifdim\wd0>\@tempdima
            \advance\@tempdima by \@tempdima
            \ifdim\wd0 >\@tempdima
               \setbox1 =\vbox{%
                  \unskip\hbox to \@tempdima{\hfill\GRAPHIC{#5}{#4}{#1}{#2}{#3}\hfill}%
                  \unskip\hbox to \@tempdima{\parbox[b]{\@tempdima}{\QCBOptA}}%
               }%
               \wd1=\@tempdima
            \else
               \textwidth=\wd0
               \setbox1 =\vbox{%
                 \noindent\hbox to \wd0{\hfill\GRAPHIC{#5}{#4}{#1}{#2}{#3}\hfill}\\%
                 \noindent\hbox{\QCBOptA}%
               }%
               \wd1=\wd0
            \fi
         \else
            \ifdim\wd0>0pt
              \hsize=\@tempdima
              \setbox1=\vbox{%
                \unskip\GRAPHIC{#5}{#4}{#1}{#2}{0pt}%
                \break
                \unskip\hbox to \@tempdima{\hfill \QCBOptA\hfill}%
              }%
              \wd1=\@tempdima
           \else
              \hsize=\@tempdima
              \setbox1=\vbox{%
                \unskip\GRAPHIC{#5}{#4}{#1}{#2}{0pt}%
              }%
              \wd1=\@tempdima
           \fi
         \fi
         \@tempdimb=\ht1
         %\advance\@tempdimb by \dp1
         \advance\@tempdimb by -#2
         \advance\@tempdimb by #3
         \leavevmode
         \raise -\@tempdimb \hbox{\box1}%
      \fi
      \egroup%
}%
%
%Macro for Display graphics object
%   \DFRAME{ contentswidth (scalar)  }               %#1
%          { contentsheight (scalar) }               %#2
%          { draft label }                           %#3
%          { name }                                  %#4
%          { caption}                                %#5
\def\DFRAME#1#2#3#4#5{%
  \hfil\break
  \bgroup
     \leftskip\@flushglue
     \rightskip\@flushglue
     \parindent\z@
     \parfillskip\z@skip
     \let\QCTOptA\empty
     \let\QCTOptB\empty
     \let\QCBOptA\empty
     \let\QCBOptB\empty
     \vbox\bgroup
        \ifOverFrame
           #5\QCTOptA\par
        \fi
        \GRAPHIC{#4}{#3}{#1}{#2}{\z@}%
        \ifUnderFrame
           \break#5\QCBOptA
        \fi
     \egroup
   \egroup
   \break
}%
%
%Macro for Floating graphic object
%   \FFRAME{ framedata f|i tbph x F|T }              %#1
%          { contentswidth (scalar)  }               %#2
%          { contentsheight (scalar) }               %#3
%          { caption }                               %#4
%          { label }                                 %#5
%          { draft name }                            %#6
%          { body }                                  %#7
\def\FFRAME#1#2#3#4#5#6#7{%
 %If float.sty loaded and float option is 'h', change to 'H'  (gp) 1998/09/05
  \@ifundefined{floatstyle}
    {%floatstyle undefined (and float.sty not present), no change
     \begin{figure}[#1]%
    }
    {%floatstyle DEFINED
     \ifx#1h%Only the h parameter, change to H
      \begin{figure}[H]%
     \else
      \begin{figure}[#1]%
     \fi
    }
  \let\QCTOptA\empty
  \let\QCTOptB\empty
  \let\QCBOptA\empty
  \let\QCBOptB\empty
  \ifOverFrame
    #4
    \ifx\QCTOptA\empty
    \else
      \ifx\QCTOptB\empty
        \caption{\QCTOptA}%
      \else
        \caption[\QCTOptB]{\QCTOptA}%
      \fi
    \fi
    \ifUnderFrame\else
      \label{#5}%
    \fi
  \else
    \UnderFrametrue%
  \fi
  \begin{center}\GRAPHIC{#7}{#6}{#2}{#3}{\z@}\end{center}%
  \ifUnderFrame
    #4
    \ifx\QCBOptA\empty
      \caption{}%
    \else
      \ifx\QCBOptB\empty
        \caption{\QCBOptA}%
      \else
        \caption[\QCBOptB]{\QCBOptA}%
      \fi
    \fi
    \label{#5}%
  \fi
  \end{figure}%
 }%
%
%
%    \FRAME{ framedata f|i tbph x F|T }              %#1
%          { contentswidth (scalar)  }               %#2
%          { contentsheight (scalar) }               %#3
%          { vertical shift when in-line (scalar) }  %#4
%          { caption }                               %#5
%          { label }                                 %#6
%          { name }                                  %#7
%          { body }                                  %#8
%
%    framedata is a string which can contain the following
%    characters: idftbphxFT
%    Their meaning is as follows:
%             i, d or f : in-line, display, or floating
%             t,b,p,h   : LaTeX floating placement options
%             x         : fit contents box to contents
%             F or T    : Figure or Table.
%                         Later this can expand
%                         to a more general float class.
%
%
\newcount\dispkind%

\def\makeactives{
  \catcode`\"=\active
  \catcode`\;=\active
  \catcode`\:=\active
  \catcode`\'=\active
  \catcode`\~=\active
}
\bgroup
   \makeactives
   \gdef\activesoff{%
      \def"{\string"}%
      \def;{\string;}%
      \def:{\string:}%
      \def'{\string'}%
      \def~{\string~}%
      %\bbl@deactivate{"}%
      %\bbl@deactivate{;}%
      %\bbl@deactivate{:}%
      %\bbl@deactivate{'}%
    }
\egroup

\def\FRAME#1#2#3#4#5#6#7#8{%
 \bgroup
 \ifnum\@msidraft=\@ne
   \wasdrafttrue
 \else
   \wasdraftfalse%
 \fi
 \def\LaTeXparams{}%
 \dispkind=\z@
 \def\LaTeXparams{}%
 \doFRAMEparams{#1}%
 \ifnum\dispkind=\z@\IFRAME{#2}{#3}{#4}{#7}{#8}{#5}\else
  \ifnum\dispkind=\@ne\DFRAME{#2}{#3}{#7}{#8}{#5}\else
   \ifnum\dispkind=\tw@
    \edef\@tempa{\noexpand\FFRAME{\LaTeXparams}}%
    \@tempa{#2}{#3}{#5}{#6}{#7}{#8}%
    \fi
   \fi
  \fi
  \ifwasdraft\@msidraft=1\else\@msidraft=0\fi{}%
  \egroup
 }%
%
% This macro added to let SW gobble a parameter that
% should not be passed on and expanded.

\def\TEXUX#1{"texux"}

%
% Macros for text attributes:
%
\def\BF#1{{\bf {#1}}}%
\def\NEG#1{\leavevmode\hbox{\rlap{\thinspace/}{$#1$}}}%
%
%%%%%%%%%%%%%%%%%%%%%%%%%%%%%%%%%%%%%%%%%%%%%%%%%%%%%%%%%%%%%%%%%%%%%%%%
%
%
% macros for user - defined functions
\def\limfunc#1{\mathop{\rm #1}}%
\def\func#1{\mathop{\rm #1}\nolimits}%
% macro for unit names
\def\unit#1{\mathord{\thinspace\rm #1}}%

%
% miscellaneous
\long\def\QQQ#1#2{%
     \long\expandafter\def\csname#1\endcsname{#2}}%
\@ifundefined{QTP}{\def\QTP#1{}}{}
\@ifundefined{QEXCLUDE}{\def\QEXCLUDE#1{}}{}
\@ifundefined{Qlb}{\def\Qlb#1{#1}}{}
\@ifundefined{Qlt}{\def\Qlt#1{#1}}{}
\def\QWE{}%
\long\def\QQA#1#2{}%
\def\QTR#1#2{{\csname#1\endcsname {#2}}}%
\long\def\TeXButton#1#2{#2}%
\long\def\QSubDoc#1#2{#2}%
\def\EXPAND#1[#2]#3{}%
\def\NOEXPAND#1[#2]#3{}%
\def\PROTECTED{}%
\def\LaTeXparent#1{}%
\def\ChildStyles#1{}%
\def\ChildDefaults#1{}%
\def\QTagDef#1#2#3{}%

% Constructs added with Scientific Notebook
\@ifundefined{correctchoice}{\def\correctchoice{\relax}}{}
\@ifundefined{HTML}{\def\HTML#1{\relax}}{}
\@ifundefined{TCIIcon}{\def\TCIIcon#1#2#3#4{\relax}}{}
\if@compatibility
  \typeout{Not defining UNICODE  U or CustomNote commands for LaTeX 2.09.}
\else
  \providecommand{\UNICODE}[2][]{\protect\rule{.1in}{.1in}}
  \providecommand{\U}[1]{\protect\rule{.1in}{.1in}}
  \providecommand{\CustomNote}[3][]{\marginpar{#3}}
\fi

\@ifundefined{lambdabar}{
      \def\lambdabar{\errmessage{You have used the lambdabar symbol.
                      This is available for typesetting only in RevTeX styles.}}
   }{}

%
% Macros for style editor docs
\@ifundefined{StyleEditBeginDoc}{\def\StyleEditBeginDoc{\relax}}{}
%
% Macros for footnotes
\def\QQfnmark#1{\footnotemark}
\def\QQfntext#1#2{\addtocounter{footnote}{#1}\footnotetext{#2}}
%
% Macros for indexing.
%
\@ifundefined{TCIMAKEINDEX}{}{\makeindex}%
%
% Attempts to avoid problems with other styles
\@ifundefined{abstract}{%
 \def\abstract{%
  \if@twocolumn
   \section*{Abstract (Not appropriate in this style!)}%
   \else \small
   \begin{center}{\bf Abstract\vspace{-.5em}\vspace{\z@}}\end{center}%
   \quotation
   \fi
  }%
 }{%
 }%
\@ifundefined{endabstract}{\def\endabstract
  {\if@twocolumn\else\endquotation\fi}}{}%
\@ifundefined{maketitle}{\def\maketitle#1{}}{}%
\@ifundefined{affiliation}{\def\affiliation#1{}}{}%
\@ifundefined{proof}{\def\proof{\noindent{\bfseries Proof. }}}{}%
\@ifundefined{endproof}{\def\endproof{\mbox{\ \rule{.1in}{.1in}}}}{}%
\@ifundefined{newfield}{\def\newfield#1#2{}}{}%
\@ifundefined{chapter}{\def\chapter#1{\par(Chapter head:)#1\par }%
 \newcount\c@chapter}{}%
\@ifundefined{part}{\def\part#1{\par(Part head:)#1\par }}{}%
\@ifundefined{section}{\def\section#1{\par(Section head:)#1\par }}{}%
\@ifundefined{subsection}{\def\subsection#1%
 {\par(Subsection head:)#1\par }}{}%
\@ifundefined{subsubsection}{\def\subsubsection#1%
 {\par(Subsubsection head:)#1\par }}{}%
\@ifundefined{paragraph}{\def\paragraph#1%
 {\par(Subsubsubsection head:)#1\par }}{}%
\@ifundefined{subparagraph}{\def\subparagraph#1%
 {\par(Subsubsubsubsection head:)#1\par }}{}%
%%%%%%%%%%%%%%%%%%%%%%%%%%%%%%%%%%%%%%%%%%%%%%%%%%%%%%%%%%%%%%%%%%%%%%%%
% These symbols are not recognized by LaTeX
\@ifundefined{therefore}{\def\therefore{}}{}%
\@ifundefined{backepsilon}{\def\backepsilon{}}{}%
\@ifundefined{yen}{\def\yen{\hbox{\rm\rlap=Y}}}{}%
\@ifundefined{registered}{%
   \def\registered{\relax\ifmmode{}\r@gistered
                    \else$\m@th\r@gistered$\fi}%
 \def\r@gistered{^{\ooalign
  {\hfil\raise.07ex\hbox{$\scriptstyle\rm\text{R}$}\hfil\crcr
  \mathhexbox20D}}}}{}%
\@ifundefined{Eth}{\def\Eth{}}{}%
\@ifundefined{eth}{\def\eth{}}{}%
\@ifundefined{Thorn}{\def\Thorn{}}{}%
\@ifundefined{thorn}{\def\thorn{}}{}%
% A macro to allow any symbol that requires math to appear in text
\def\TEXTsymbol#1{\mbox{$#1$}}%
\@ifundefined{degree}{\def\degree{{}^{\circ}}}{}%
%
% macros for T3TeX files
\newdimen\theight
\@ifundefined{Column}{\def\Column{%
 \vadjust{\setbox\z@=\hbox{\scriptsize\quad\quad tcol}%
  \theight=\ht\z@\advance\theight by \dp\z@\advance\theight by \lineskip
  \kern -\theight \vbox to \theight{%
   \rightline{\rlap{\box\z@}}%
   \vss
   }%
  }%
 }}{}%
%
\@ifundefined{qed}{\def\qed{%
 \ifhmode\unskip\nobreak\fi\ifmmode\ifinner\else\hskip5\p@\fi\fi
 \hbox{\hskip5\p@\vrule width4\p@ height6\p@ depth1.5\p@\hskip\p@}%
 }}{}%
%
\@ifundefined{cents}{\def\cents{\hbox{\rm\rlap c/}}}{}%
\@ifundefined{tciLaplace}{\def\tciLaplace{L}}{}%
\@ifundefined{tciFourier}{\def\tciFourier{F}}{}%
\@ifundefined{textcurrency}{\def\textcurrency{\hbox{\rm\rlap xo}}}{}%
\@ifundefined{texteuro}{\def\texteuro{\hbox{\rm\rlap C=}}}{}%
\@ifundefined{textfranc}{\def\textfranc{\hbox{\rm\rlap-F}}}{}%
\@ifundefined{textlira}{\def\textlira{\hbox{\rm\rlap L=}}}{}%
\@ifundefined{textpeseta}{\def\textpeseta{\hbox{\rm P\negthinspace s}}}{}%
%
\@ifundefined{miss}{\def\miss{\hbox{\vrule height2\p@ width 2\p@ depth\z@}}}{}%
%
\@ifundefined{vvert}{\def\vvert{\Vert}}{}%  %always translated to \left| or \right|
%
\@ifundefined{tcol}{\def\tcol#1{{\baselineskip=6\p@ \vcenter{#1}} \Column}}{}%
%
\@ifundefined{dB}{\def\dB{\hbox{{}}}}{}%        %dummy entry in column
\@ifundefined{mB}{\def\mB#1{\hbox{$#1$}}}{}%   %column entry
\@ifundefined{nB}{\def\nB#1{\hbox{#1}}}{}%     %column entry (not math)
%
\@ifundefined{note}{\def\note{$^{\dag}}}{}%
%
\def\newfmtname{LaTeX2e}
% No longer load latexsym.  This is now handled by SWP, which uses amsfonts if necessary
%
\ifx\fmtname\newfmtname
  \DeclareOldFontCommand{\rm}{\normalfont\rmfamily}{\mathrm}
  \DeclareOldFontCommand{\sf}{\normalfont\sffamily}{\mathsf}
  \DeclareOldFontCommand{\tt}{\normalfont\ttfamily}{\mathtt}
  \DeclareOldFontCommand{\bf}{\normalfont\bfseries}{\mathbf}
  \DeclareOldFontCommand{\it}{\normalfont\itshape}{\mathit}
  \DeclareOldFontCommand{\sl}{\normalfont\slshape}{\@nomath\sl}
  \DeclareOldFontCommand{\sc}{\normalfont\scshape}{\@nomath\sc}
\fi

%
% Greek bold macros
% Redefine all of the math symbols
% which might be bolded  - there are
% probably others to add to this list

\def\alpha{{\Greekmath 010B}}%
\def\beta{{\Greekmath 010C}}%
\def\gamma{{\Greekmath 010D}}%
\def\delta{{\Greekmath 010E}}%
\def\epsilon{{\Greekmath 010F}}%
\def\zeta{{\Greekmath 0110}}%
\def\eta{{\Greekmath 0111}}%
\def\theta{{\Greekmath 0112}}%
\def\iota{{\Greekmath 0113}}%
\def\kappa{{\Greekmath 0114}}%
\def\lambda{{\Greekmath 0115}}%
\def\mu{{\Greekmath 0116}}%
\def\nu{{\Greekmath 0117}}%
\def\xi{{\Greekmath 0118}}%
\def\pi{{\Greekmath 0119}}%
\def\rho{{\Greekmath 011A}}%
\def\sigma{{\Greekmath 011B}}%
\def\tau{{\Greekmath 011C}}%
\def\upsilon{{\Greekmath 011D}}%
\def\phi{{\Greekmath 011E}}%
\def\chi{{\Greekmath 011F}}%
\def\psi{{\Greekmath 0120}}%
\def\omega{{\Greekmath 0121}}%
\def\varepsilon{{\Greekmath 0122}}%
\def\vartheta{{\Greekmath 0123}}%
\def\varpi{{\Greekmath 0124}}%
\def\varrho{{\Greekmath 0125}}%
\def\varsigma{{\Greekmath 0126}}%
\def\varphi{{\Greekmath 0127}}%

\def\nabla{{\Greekmath 0272}}
\def\FindBoldGroup{%
   {\setbox0=\hbox{$\mathbf{x\global\edef\theboldgroup{\the\mathgroup}}$}}%
}

\def\Greekmath#1#2#3#4{%
    \if@compatibility
        \ifnum\mathgroup=\symbold
           \mathchoice{\mbox{\boldmath$\displaystyle\mathchar"#1#2#3#4$}}%
                      {\mbox{\boldmath$\textstyle\mathchar"#1#2#3#4$}}%
                      {\mbox{\boldmath$\scriptstyle\mathchar"#1#2#3#4$}}%
                      {\mbox{\boldmath$\scriptscriptstyle\mathchar"#1#2#3#4$}}%
        \else
           \mathchar"#1#2#3#4%
        \fi
    \else
        \FindBoldGroup
        \ifnum\mathgroup=\theboldgroup % For 2e
           \mathchoice{\mbox{\boldmath$\displaystyle\mathchar"#1#2#3#4$}}%
                      {\mbox{\boldmath$\textstyle\mathchar"#1#2#3#4$}}%
                      {\mbox{\boldmath$\scriptstyle\mathchar"#1#2#3#4$}}%
                      {\mbox{\boldmath$\scriptscriptstyle\mathchar"#1#2#3#4$}}%
        \else
           \mathchar"#1#2#3#4%
        \fi
      \fi}

\newif\ifGreekBold  \GreekBoldfalse
\let\SAVEPBF=\pbf
\def\pbf{\GreekBoldtrue\SAVEPBF}%
%

\@ifundefined{theorem}{\newtheorem{theorem}{Theorem}}{}
\@ifundefined{lemma}{\newtheorem{lemma}[theorem]{Lemma}}{}
\@ifundefined{corollary}{\newtheorem{corollary}[theorem]{Corollary}}{}
\@ifundefined{conjecture}{\newtheorem{conjecture}[theorem]{Conjecture}}{}
\@ifundefined{proposition}{\newtheorem{proposition}[theorem]{Proposition}}{}
\@ifundefined{axiom}{\newtheorem{axiom}{Axiom}}{}
\@ifundefined{remark}{\newtheorem{remark}{Remark}}{}
\@ifundefined{example}{\newtheorem{example}{Example}}{}
\@ifundefined{exercise}{\newtheorem{exercise}{Exercise}}{}
\@ifundefined{definition}{\newtheorem{definition}{Definition}}{}


\@ifundefined{mathletters}{%
  %\def\theequation{\arabic{equation}}
  \newcounter{equationnumber}
  \def\mathletters{%
     \addtocounter{equation}{1}
     \edef\@currentlabel{\theequation}%
     \setcounter{equationnumber}{\c@equation}
     \setcounter{equation}{0}%
     \edef\theequation{\@currentlabel\noexpand\alph{equation}}%
  }
  \def\endmathletters{%
     \setcounter{equation}{\value{equationnumber}}%
  }
}{}

%Logos
\@ifundefined{BibTeX}{%
    \def\BibTeX{{\rm B\kern-.05em{\sc i\kern-.025em b}\kern-.08em
                 T\kern-.1667em\lower.7ex\hbox{E}\kern-.125emX}}}{}%
\@ifundefined{AmS}%
    {\def\AmS{{\protect\usefont{OMS}{cmsy}{m}{n}%
                A\kern-.1667em\lower.5ex\hbox{M}\kern-.125emS}}}{}%
\@ifundefined{AmSTeX}{\def\AmSTeX{\protect\AmS-\protect\TeX\@}}{}%
%

% This macro is a fix to eqnarray
\def\@@eqncr{\let\@tempa\relax
    \ifcase\@eqcnt \def\@tempa{& & &}\or \def\@tempa{& &}%
      \else \def\@tempa{&}\fi
     \@tempa
     \if@eqnsw
        \iftag@
           \@taggnum
        \else
           \@eqnnum\stepcounter{equation}%
        \fi
     \fi
     \global\tag@false
     \global\@eqnswtrue
     \global\@eqcnt\z@\cr}


\def\TCItag{\@ifnextchar*{\@TCItagstar}{\@TCItag}}
\def\@TCItag#1{%
    \global\tag@true
    \global\def\@taggnum{(#1)}}
\def\@TCItagstar*#1{%
    \global\tag@true
    \global\def\@taggnum{#1}}
%
%%%%%%%%%%%%%%%%%%%%%%%%%%%%%%%%%%%%%%%%%%%%%%%%%%%%%%%%%%%%%%%%%%%%%
%
\def\QATOP#1#2{{#1 \atop #2}}%
\def\QTATOP#1#2{{\textstyle {#1 \atop #2}}}%
\def\QDATOP#1#2{{\displaystyle {#1 \atop #2}}}%
\def\QABOVE#1#2#3{{#2 \above#1 #3}}%
\def\QTABOVE#1#2#3{{\textstyle {#2 \above#1 #3}}}%
\def\QDABOVE#1#2#3{{\displaystyle {#2 \above#1 #3}}}%
\def\QOVERD#1#2#3#4{{#3 \overwithdelims#1#2 #4}}%
\def\QTOVERD#1#2#3#4{{\textstyle {#3 \overwithdelims#1#2 #4}}}%
\def\QDOVERD#1#2#3#4{{\displaystyle {#3 \overwithdelims#1#2 #4}}}%
\def\QATOPD#1#2#3#4{{#3 \atopwithdelims#1#2 #4}}%
\def\QTATOPD#1#2#3#4{{\textstyle {#3 \atopwithdelims#1#2 #4}}}%
\def\QDATOPD#1#2#3#4{{\displaystyle {#3 \atopwithdelims#1#2 #4}}}%
\def\QABOVED#1#2#3#4#5{{#4 \abovewithdelims#1#2#3 #5}}%
\def\QTABOVED#1#2#3#4#5{{\textstyle
   {#4 \abovewithdelims#1#2#3 #5}}}%
\def\QDABOVED#1#2#3#4#5{{\displaystyle
   {#4 \abovewithdelims#1#2#3 #5}}}%
%
% Macros for text size operators:
%
\def\tint{\mathop{\textstyle \int}}%
\def\tiint{\mathop{\textstyle \iint }}%
\def\tiiint{\mathop{\textstyle \iiint }}%
\def\tiiiint{\mathop{\textstyle \iiiint }}%
\def\tidotsint{\mathop{\textstyle \idotsint }}%
\def\toint{\mathop{\textstyle \oint}}%
\def\tsum{\mathop{\textstyle \sum }}%
\def\tprod{\mathop{\textstyle \prod }}%
\def\tbigcap{\mathop{\textstyle \bigcap }}%
\def\tbigwedge{\mathop{\textstyle \bigwedge }}%
\def\tbigoplus{\mathop{\textstyle \bigoplus }}%
\def\tbigodot{\mathop{\textstyle \bigodot }}%
\def\tbigsqcup{\mathop{\textstyle \bigsqcup }}%
\def\tcoprod{\mathop{\textstyle \coprod }}%
\def\tbigcup{\mathop{\textstyle \bigcup }}%
\def\tbigvee{\mathop{\textstyle \bigvee }}%
\def\tbigotimes{\mathop{\textstyle \bigotimes }}%
\def\tbiguplus{\mathop{\textstyle \biguplus }}%
%
%
%Macros for display size operators:
%
\def\dint{\mathop{\displaystyle \int}}%
\def\diint{\mathop{\displaystyle \iint}}%
\def\diiint{\mathop{\displaystyle \iiint}}%
\def\diiiint{\mathop{\displaystyle \iiiint }}%
\def\didotsint{\mathop{\displaystyle \idotsint }}%
\def\doint{\mathop{\displaystyle \oint}}%
\def\dsum{\mathop{\displaystyle \sum }}%
\def\dprod{\mathop{\displaystyle \prod }}%
\def\dbigcap{\mathop{\displaystyle \bigcap }}%
\def\dbigwedge{\mathop{\displaystyle \bigwedge }}%
\def\dbigoplus{\mathop{\displaystyle \bigoplus }}%
\def\dbigodot{\mathop{\displaystyle \bigodot }}%
\def\dbigsqcup{\mathop{\displaystyle \bigsqcup }}%
\def\dcoprod{\mathop{\displaystyle \coprod }}%
\def\dbigcup{\mathop{\displaystyle \bigcup }}%
\def\dbigvee{\mathop{\displaystyle \bigvee }}%
\def\dbigotimes{\mathop{\displaystyle \bigotimes }}%
\def\dbiguplus{\mathop{\displaystyle \biguplus }}%


\if@compatibility\else
  \RequirePackage{amsmath}
  \makeatother
  \endinput
\fi

%%%%%%%%%%%%%%%%%%%%%%%%%%%%%%%%%%%%%%%%%%%%%%%%%%%%%%%%%%%%%%%%%%%%%%%%%%
% NOTE: The rest of this file is read only if in LaTeX 2.09 compatibility
% mode. This section is used to define AMS-like constructs in the
% event they have not been defined.
%%%%%%%%%%%%%%%%%%%%%%%%%%%%%%%%%%%%%%%%%%%%%%%%%%%%%%%%%%%%%%%%%%%%%%%%%%
\typeout{TCILATEX defining AMS-like constructs in LaTeX 2.09 COMPATIBILITY MODE}
\def\ExitTCILatex{\makeatother\endinput}

\bgroup
\ifx\ds@amstex\relax
   \message{amstex already loaded}\aftergroup\ExitTCILatex
\else
   \@ifpackageloaded{amsmath}%
      {\message{amsmath already loaded}\aftergroup\ExitTCILatex}
      {}
   \@ifpackageloaded{amstex}%
      {\message{amstex already loaded}\aftergroup\ExitTCILatex}
      {}
   \@ifpackageloaded{amsgen}%
      {\message{amsgen already loaded}\aftergroup\ExitTCILatex}
      {}
\fi
\egroup


%%%%%%%%%%%%%%%%%%%%%%%%%%%%%%%%%%%%%%%%%%%%%%%%%%%%%%%%%%%%%%%%%%%%%%%%
%  Macros to define some AMS LaTeX constructs when
%  AMS LaTeX has not been loaded
%
% These macros are copied from the AMS-TeX package for doing
% multiple integrals.
%
\let\DOTSI\relax
\def\RIfM@{\relax\ifmmode}%
\def\FN@{\futurelet\next}%
\newcount\intno@
\def\iint{\DOTSI\intno@\tw@\FN@\ints@}%
\def\iiint{\DOTSI\intno@\thr@@\FN@\ints@}%
\def\iiiint{\DOTSI\intno@4 \FN@\ints@}%
\def\idotsint{\DOTSI\intno@\z@\FN@\ints@}%
\def\ints@{\findlimits@\ints@@}%
\newif\iflimtoken@
\newif\iflimits@
\def\findlimits@{\limtoken@true\ifx\next\limits\limits@true
 \else\ifx\next\nolimits\limits@false\else
 \limtoken@false\ifx\ilimits@\nolimits\limits@false\else
 \ifinner\limits@false\else\limits@true\fi\fi\fi\fi}%
\def\multint@{\int\ifnum\intno@=\z@\intdots@                          %1
 \else\intkern@\fi                                                    %2
 \ifnum\intno@>\tw@\int\intkern@\fi                                   %3
 \ifnum\intno@>\thr@@\int\intkern@\fi                                 %4
 \int}%                                                               %5
\def\multintlimits@{\intop\ifnum\intno@=\z@\intdots@\else\intkern@\fi
 \ifnum\intno@>\tw@\intop\intkern@\fi
 \ifnum\intno@>\thr@@\intop\intkern@\fi\intop}%
\def\intic@{%
    \mathchoice{\hskip.5em}{\hskip.4em}{\hskip.4em}{\hskip.4em}}%
\def\negintic@{\mathchoice
 {\hskip-.5em}{\hskip-.4em}{\hskip-.4em}{\hskip-.4em}}%
\def\ints@@{\iflimtoken@                                              %1
 \def\ints@@@{\iflimits@\negintic@
   \mathop{\intic@\multintlimits@}\limits                             %2
  \else\multint@\nolimits\fi                                          %3
  \eat@}%                                                             %4
 \else                                                                %5
 \def\ints@@@{\iflimits@\negintic@
  \mathop{\intic@\multintlimits@}\limits\else
  \multint@\nolimits\fi}\fi\ints@@@}%
\def\intkern@{\mathchoice{\!\!\!}{\!\!}{\!\!}{\!\!}}%
\def\plaincdots@{\mathinner{\cdotp\cdotp\cdotp}}%
\def\intdots@{\mathchoice{\plaincdots@}%
 {{\cdotp}\mkern1.5mu{\cdotp}\mkern1.5mu{\cdotp}}%
 {{\cdotp}\mkern1mu{\cdotp}\mkern1mu{\cdotp}}%
 {{\cdotp}\mkern1mu{\cdotp}\mkern1mu{\cdotp}}}%
%
%
%  These macros are for doing the AMS \text{} construct
%
\def\RIfM@{\relax\protect\ifmmode}
\def\text{\RIfM@\expandafter\text@\else\expandafter\mbox\fi}
\let\nfss@text\text
\def\text@#1{\mathchoice
   {\textdef@\displaystyle\f@size{#1}}%
   {\textdef@\textstyle\tf@size{\firstchoice@false #1}}%
   {\textdef@\textstyle\sf@size{\firstchoice@false #1}}%
   {\textdef@\textstyle \ssf@size{\firstchoice@false #1}}%
   \glb@settings}

\def\textdef@#1#2#3{\hbox{{%
                    \everymath{#1}%
                    \let\f@size#2\selectfont
                    #3}}}
\newif\iffirstchoice@
\firstchoice@true
%
%These are the AMS constructs for multiline limits.
%
\def\Let@{\relax\iffalse{\fi\let\\=\cr\iffalse}\fi}%
\def\vspace@{\def\vspace##1{\crcr\noalign{\vskip##1\relax}}}%
\def\multilimits@{\bgroup\vspace@\Let@
 \baselineskip\fontdimen10 \scriptfont\tw@
 \advance\baselineskip\fontdimen12 \scriptfont\tw@
 \lineskip\thr@@\fontdimen8 \scriptfont\thr@@
 \lineskiplimit\lineskip
 \vbox\bgroup\ialign\bgroup\hfil$\m@th\scriptstyle{##}$\hfil\crcr}%
\def\Sb{_\multilimits@}%
\def\endSb{\crcr\egroup\egroup\egroup}%
\def\Sp{^\multilimits@}%
\let\endSp\endSb
%
%
%These are AMS constructs for horizontal arrows
%
\newdimen\ex@
\ex@.2326ex
\def\rightarrowfill@#1{$#1\m@th\mathord-\mkern-6mu\cleaders
 \hbox{$#1\mkern-2mu\mathord-\mkern-2mu$}\hfill
 \mkern-6mu\mathord\rightarrow$}%
\def\leftarrowfill@#1{$#1\m@th\mathord\leftarrow\mkern-6mu\cleaders
 \hbox{$#1\mkern-2mu\mathord-\mkern-2mu$}\hfill\mkern-6mu\mathord-$}%
\def\leftrightarrowfill@#1{$#1\m@th\mathord\leftarrow
\mkern-6mu\cleaders
 \hbox{$#1\mkern-2mu\mathord-\mkern-2mu$}\hfill
 \mkern-6mu\mathord\rightarrow$}%
\def\overrightarrow{\mathpalette\overrightarrow@}%
\def\overrightarrow@#1#2{\vbox{\ialign{##\crcr\rightarrowfill@#1\crcr
 \noalign{\kern-\ex@\nointerlineskip}$\m@th\hfil#1#2\hfil$\crcr}}}%
\let\overarrow\overrightarrow
\def\overleftarrow{\mathpalette\overleftarrow@}%
\def\overleftarrow@#1#2{\vbox{\ialign{##\crcr\leftarrowfill@#1\crcr
 \noalign{\kern-\ex@\nointerlineskip}$\m@th\hfil#1#2\hfil$\crcr}}}%
\def\overleftrightarrow{\mathpalette\overleftrightarrow@}%
\def\overleftrightarrow@#1#2{\vbox{\ialign{##\crcr
   \leftrightarrowfill@#1\crcr
 \noalign{\kern-\ex@\nointerlineskip}$\m@th\hfil#1#2\hfil$\crcr}}}%
\def\underrightarrow{\mathpalette\underrightarrow@}%
\def\underrightarrow@#1#2{\vtop{\ialign{##\crcr$\m@th\hfil#1#2\hfil
  $\crcr\noalign{\nointerlineskip}\rightarrowfill@#1\crcr}}}%
\let\underarrow\underrightarrow
\def\underleftarrow{\mathpalette\underleftarrow@}%
\def\underleftarrow@#1#2{\vtop{\ialign{##\crcr$\m@th\hfil#1#2\hfil
  $\crcr\noalign{\nointerlineskip}\leftarrowfill@#1\crcr}}}%
\def\underleftrightarrow{\mathpalette\underleftrightarrow@}%
\def\underleftrightarrow@#1#2{\vtop{\ialign{##\crcr$\m@th
  \hfil#1#2\hfil$\crcr
 \noalign{\nointerlineskip}\leftrightarrowfill@#1\crcr}}}%
%%%%%%%%%%%%%%%%%%%%%

\def\qopnamewl@#1{\mathop{\operator@font#1}\nlimits@}
\let\nlimits@\displaylimits
\def\setboxz@h{\setbox\z@\hbox}


\def\varlim@#1#2{\mathop{\vtop{\ialign{##\crcr
 \hfil$#1\m@th\operator@font lim$\hfil\crcr
 \noalign{\nointerlineskip}#2#1\crcr
 \noalign{\nointerlineskip\kern-\ex@}\crcr}}}}

 \def\rightarrowfill@#1{\m@th\setboxz@h{$#1-$}\ht\z@\z@
  $#1\copy\z@\mkern-6mu\cleaders
  \hbox{$#1\mkern-2mu\box\z@\mkern-2mu$}\hfill
  \mkern-6mu\mathord\rightarrow$}
\def\leftarrowfill@#1{\m@th\setboxz@h{$#1-$}\ht\z@\z@
  $#1\mathord\leftarrow\mkern-6mu\cleaders
  \hbox{$#1\mkern-2mu\copy\z@\mkern-2mu$}\hfill
  \mkern-6mu\box\z@$}


\def\projlim{\qopnamewl@{proj\,lim}}
\def\injlim{\qopnamewl@{inj\,lim}}
\def\varinjlim{\mathpalette\varlim@\rightarrowfill@}
\def\varprojlim{\mathpalette\varlim@\leftarrowfill@}
\def\varliminf{\mathpalette\varliminf@{}}
\def\varliminf@#1{\mathop{\underline{\vrule\@depth.2\ex@\@width\z@
   \hbox{$#1\m@th\operator@font lim$}}}}
\def\varlimsup{\mathpalette\varlimsup@{}}
\def\varlimsup@#1{\mathop{\overline
  {\hbox{$#1\m@th\operator@font lim$}}}}

%
%Companion to stackrel
\def\stackunder#1#2{\mathrel{\mathop{#2}\limits_{#1}}}%
%
%
% These are AMS environments that will be defined to
% be verbatims if amstex has not actually been
% loaded
%
%
\begingroup \catcode `|=0 \catcode `[= 1
\catcode`]=2 \catcode `\{=12 \catcode `\}=12
\catcode`\\=12
|gdef|@alignverbatim#1\end{align}[#1|end[align]]
|gdef|@salignverbatim#1\end{align*}[#1|end[align*]]

|gdef|@alignatverbatim#1\end{alignat}[#1|end[alignat]]
|gdef|@salignatverbatim#1\end{alignat*}[#1|end[alignat*]]

|gdef|@xalignatverbatim#1\end{xalignat}[#1|end[xalignat]]
|gdef|@sxalignatverbatim#1\end{xalignat*}[#1|end[xalignat*]]

|gdef|@gatherverbatim#1\end{gather}[#1|end[gather]]
|gdef|@sgatherverbatim#1\end{gather*}[#1|end[gather*]]

|gdef|@gatherverbatim#1\end{gather}[#1|end[gather]]
|gdef|@sgatherverbatim#1\end{gather*}[#1|end[gather*]]


|gdef|@multilineverbatim#1\end{multiline}[#1|end[multiline]]
|gdef|@smultilineverbatim#1\end{multiline*}[#1|end[multiline*]]

|gdef|@arraxverbatim#1\end{arrax}[#1|end[arrax]]
|gdef|@sarraxverbatim#1\end{arrax*}[#1|end[arrax*]]

|gdef|@tabulaxverbatim#1\end{tabulax}[#1|end[tabulax]]
|gdef|@stabulaxverbatim#1\end{tabulax*}[#1|end[tabulax*]]


|endgroup



\def\align{\@verbatim \frenchspacing\@vobeyspaces \@alignverbatim
You are using the "align" environment in a style in which it is not defined.}
\let\endalign=\endtrivlist

\@namedef{align*}{\@verbatim\@salignverbatim
You are using the "align*" environment in a style in which it is not defined.}
\expandafter\let\csname endalign*\endcsname =\endtrivlist




\def\alignat{\@verbatim \frenchspacing\@vobeyspaces \@alignatverbatim
You are using the "alignat" environment in a style in which it is not defined.}
\let\endalignat=\endtrivlist

\@namedef{alignat*}{\@verbatim\@salignatverbatim
You are using the "alignat*" environment in a style in which it is not defined.}
\expandafter\let\csname endalignat*\endcsname =\endtrivlist




\def\xalignat{\@verbatim \frenchspacing\@vobeyspaces \@xalignatverbatim
You are using the "xalignat" environment in a style in which it is not defined.}
\let\endxalignat=\endtrivlist

\@namedef{xalignat*}{\@verbatim\@sxalignatverbatim
You are using the "xalignat*" environment in a style in which it is not defined.}
\expandafter\let\csname endxalignat*\endcsname =\endtrivlist




\def\gather{\@verbatim \frenchspacing\@vobeyspaces \@gatherverbatim
You are using the "gather" environment in a style in which it is not defined.}
\let\endgather=\endtrivlist

\@namedef{gather*}{\@verbatim\@sgatherverbatim
You are using the "gather*" environment in a style in which it is not defined.}
\expandafter\let\csname endgather*\endcsname =\endtrivlist


\def\multiline{\@verbatim \frenchspacing\@vobeyspaces \@multilineverbatim
You are using the "multiline" environment in a style in which it is not defined.}
\let\endmultiline=\endtrivlist

\@namedef{multiline*}{\@verbatim\@smultilineverbatim
You are using the "multiline*" environment in a style in which it is not defined.}
\expandafter\let\csname endmultiline*\endcsname =\endtrivlist


\def\arrax{\@verbatim \frenchspacing\@vobeyspaces \@arraxverbatim
You are using a type of "array" construct that is only allowed in AmS-LaTeX.}
\let\endarrax=\endtrivlist

\def\tabulax{\@verbatim \frenchspacing\@vobeyspaces \@tabulaxverbatim
You are using a type of "tabular" construct that is only allowed in AmS-LaTeX.}
\let\endtabulax=\endtrivlist


\@namedef{arrax*}{\@verbatim\@sarraxverbatim
You are using a type of "array*" construct that is only allowed in AmS-LaTeX.}
\expandafter\let\csname endarrax*\endcsname =\endtrivlist

\@namedef{tabulax*}{\@verbatim\@stabulaxverbatim
You are using a type of "tabular*" construct that is only allowed in AmS-LaTeX.}
\expandafter\let\csname endtabulax*\endcsname =\endtrivlist

% macro to simulate ams tag construct


% This macro is a fix to the equation environment
 \def\endequation{%
     \ifmmode\ifinner % FLEQN hack
      \iftag@
        \addtocounter{equation}{-1} % undo the increment made in the begin part
        $\hfil
           \displaywidth\linewidth\@taggnum\egroup \endtrivlist
        \global\tag@false
        \global\@ignoretrue
      \else
        $\hfil
           \displaywidth\linewidth\@eqnnum\egroup \endtrivlist
        \global\tag@false
        \global\@ignoretrue
      \fi
     \else
      \iftag@
        \addtocounter{equation}{-1} % undo the increment made in the begin part
        \eqno \hbox{\@taggnum}
        \global\tag@false%
        $$\global\@ignoretrue
      \else
        \eqno \hbox{\@eqnnum}% $$ BRACE MATCHING HACK
        $$\global\@ignoretrue
      \fi
     \fi\fi
 }

 \newif\iftag@ \tag@false

 \def\TCItag{\@ifnextchar*{\@TCItagstar}{\@TCItag}}
 \def\@TCItag#1{%
     \global\tag@true
     \global\def\@taggnum{(#1)}}
 \def\@TCItagstar*#1{%
     \global\tag@true
     \global\def\@taggnum{#1}}

  \@ifundefined{tag}{
     \def\tag{\@ifnextchar*{\@tagstar}{\@tag}}
     \def\@tag#1{%
         \global\tag@true
         \global\def\@taggnum{(#1)}}
     \def\@tagstar*#1{%
         \global\tag@true
         \global\def\@taggnum{#1}}
  }{}

\def\tfrac#1#2{{\textstyle {#1 \over #2}}}%
\def\dfrac#1#2{{\displaystyle {#1 \over #2}}}%
\def\binom#1#2{{#1 \choose #2}}%
\def\tbinom#1#2{{\textstyle {#1 \choose #2}}}%
\def\dbinom#1#2{{\displaystyle {#1 \choose #2}}}%

% Do not add anything to the end of this file.
% The last section of the file is loaded only if
% amstex has not been.
\makeatother
\endinput


\begin{document}
\title{Examples of renormalized SDEs}

%\date{}
%
%
%\title{More than one Author with different Affiliations}
\author{Y. Bruned, I. Chevyrev, and P. K. Friz}

\date{\today}


% [1]{Y. Brunded}
%\author[2]{I. Chevyrev}
%\author[2,3]{P.K. Friz}%\thanks{B.B@university.edu}}
%\author[3]{P. Gassiat}%\thanks{C.C@university.edu}}
%%\author[2]{Author D\thanks{D.D@university.edu}}
%%\author[2]{Author E\thanks{E.E@university.edu}}
%{\tiny
%\affil[1]{{Warwick University}}
%\affil[2]{{Institut f\"ur Mathematik, Technische Universit\"at Berlin}}
%\affil[3]{{Weierstra\ss --Institut f\"ur Angewandte Analysis und Stochastik, Berlin}}
%}


\maketitle

\begin{abstract}
We demonstrate two examples of stochastic processes whose lifts to geometric rough paths require a renormalisation procedure to obtain convergence in rough path topologies. Our first example involves a physical Brownian motion subject to a magnetic force which dominates over the friction forces in the small mass limit. Our second example involves a lead-lag process of discretised fractional Brownian motion with Hurst parameter $H \in (1/4,1/2)$, in which the stochastic area captures the quadratic variation of the process. In both examples, a renormalisation of the second iterated integral is needed to ensure convergence of the processes, and we comment on how this procedure mimics negative renormalisation arising in the study of singular SPDEs and regularity structures.
\end{abstract}



%\tableofcontents

\section{Introduction}

In recent years, the theory of regularity structures~\cite{Hairer14} has been proposed to give meaning to a wide class of singular SPDEs. A central feature of the theory is the notion of renormalisation, specifically ``negative renormalisation''~\cite{BHZ16}, which is required to obtain convergence of random models to a meaningful limit. It is well-known that this procedure is inherit to the problem since naive approximations of such equations typically fail to converge (with a number of notable exceptions, including a special variant of gKPZ~\cite{Hairer16}). The same feature thus naturally appears in other solution theories which have been proposed to solve such equations, including the theories of paracontrolled distributions~\cite{GIP15} and the Wilsonian renormalisation group~\cite{Kupiainen16}.

Viewing regularity structures as a multidimensional generalisation of the theory of rough paths~\cite{Lyons98}, it is natural to ask how renormalisation manifests itself in the latter. As solution theories to singular S(P)DEs, both share the common feature that one must give meaning to often analytically ill-posed higher order terms (iterated integrals) of distributions, which is typically done through some stochastic means. However, a key difference in applications of rough paths to SDEs and rough SPDEs~\cite{FrizVictoir10} is that one can usually give meaning to such terms as limits of iterated integrals of mollifications of the irregular noise without the need of renormalisation.

The purpose of this note is to demonstrate situations in rough paths theory which fall outside this usual setting and for which renormalisation is a necessary feature. Specifically, we construct two stochastic processes whose lifts to geometric rough paths fail to converge without a renormalisation procedure akin to the one encountered in the theory of regularity structures.

The first is a physical Brownian motion subject to a magnetic force which dominates over the friction forces in the small mass limit. This example builds on the work~\cite{FrizGassiat15}, where a similar situation was considering with a constant magnetic field.

The second is a lead-lag process of a discretized path, which we take to be fractional Brownian motion with Hurst parameter $H \in (1/4,1/2]$. The stochastic area of this lead-lag process captures the quadratic variation of the discretized path, and thus, as one can expect, the second iterated integral fails to converge as the mesh of the discretisation goes to zero (unless $H=1/2$). This example is motivated from a similar Hoff process considered for semi-martinagles in~\cite{Flint16}.

In both examples we demonstrate an explicit renormalisation procedure of the second iterated integral under which the processes converge in rough path topologies. These (diverging) counter-terms serve precisely the same re-centring role encountered in regularity structures (for a direct comparison, consider the renormalisation of PAM~\cite{Hairer14,GIP15} where only one diverging term needs to be considered). In turn, rough differential equations driven by the renormalised and unrenormalised rough paths are related to one another by the addition of diverging terms, which again mimics the situation encountered in singular SPDEs. We refer to the upcoming work~\cite{Bruned16} for a much more detailed study of this relation.

\medskip

{\bf Acknowledgements.} P.K.F. is partially supported by the European Research Council through CoG-683166 and DFG research unit FOR2402. I.C., affiliated to TU Berlin when this project was commenced, was supported by DFG research unit FOR2402.


\section{Magnetic field blow-up}\label{subsec:magnetic}

%This follows on the work of Friz, Gassiat and Lyons~\cite{FrizGassiat15}.
Consider a physical Brownian motion in a magnetic field with dynamics given by
\[
m \ddot{x} = -A\dot{x} + B\dot{x} + \xi, \; \; x(t) \in \R^d, 
\]
where $A$ is a symmetric matrix with strictly positive spectrum (representing friction), $B$ is an anti-symmetric matrix (representing the Lorentz force due to a magnetic field), and $\xi$ is an $\R^d$-valued white noise in time. We shall consider the situation that $A$ is constant whereas $B$ is a function of the mass $m$.

We rewrite these dynamics as
\begin{align*}
dX_t &= \frac{1}{m} P_t dt, \; \; X_0 = 0, \\
dP_t &= -\frac{1}{m} M P_t dt + dW_t, \; \; P_0 = 0,
\end{align*}
where $M = A - B$, and we have chosen the starting point as zero simply for convenience. We furthermore introduce the parameter $\varepsilon^2 = m$ and write $X^\varepsilon_t, P^\varepsilon_t$, and $M^\varepsilon = A-B^\varepsilon$ to denote the dependence on $\varepsilon$.

We are interested in the convergence of the processes $P^\varepsilon$ and $M^\varepsilon X^\varepsilon$ in rough path topologies. Let $G^2(\R^d)$ and $\g^2(\R^d)$ denote the step-$2$ free nilpotent Lie group and Lie algebra respectively. Let us also write $\g^2(\R^d) = \R^d \oplus \g^{(2)}(\R^d)$ for the decomposition of $\g^2(\R^d)$ into the first and second levels, where we identify $\g^{(2)}(\R^d)$ with the space of anti-symmetric $d\times d$ matrices. 

For every $\varepsilon > 0$, define the matrix
\[
C^\varepsilon = \int_0^\infty e^{-M^\varepsilon s}e^{-(M^{\varepsilon})^* s}ds,
\]
and the element
\[
v^\varepsilon = -\frac{1}{2}(M^\varepsilon C^\varepsilon - C^\varepsilon(M^\varepsilon)^*) \in \g^{(2)}(\R^d).
\]

%\begin{remark}\label{remark:timeComp}
%We canonically extend every $\R^d$-valued process $Z_t$ to an $\R^{1+d}$-valued process by introducing the time component $(t,Z_t)$. In this notation,

For any $v \in \g^{(2)}(\R^d)$, $p \in [1,3)$, and $p$-rough path $(Z_{s,t},\Z_{s,t}) \in G^2(\R^d)$ (where we ignore zeroth component $1$), we define the translated rough path $T_v(Z_{s,t}, \Z_{s,t})$ by
\begin{equation}\label{eq:TvDef}
T_v(Z_{s,t}, \Z_{s,t}) = (Z_{s,t}, \Z_{s,t} + (t-s)v^\varepsilon).
\end{equation}

%\begin{remark}\label{remark:timeComp}
%In the notation of~\cite{Bruned16}, $T_v$  We canonically extend every $\R^d$-valued process $Z_t$ to an $\R^{1+d}$-valued process by introducing the time component $(t,Z_t)$. In this notation,
%\end{remark}

Consider the $G^2(\R^d)$-valued processes
\begin{align*}
(P^\varepsilon_{s,t}, \Pbb^\varepsilon_{s,t}) &= \left(P^\varepsilon_{s,t}, \int_s^t P^\varepsilon_{s,r} \otimes \circ dP^\varepsilon_r\right), \\
(Z^\varepsilon_{s,t}, \Z^\varepsilon_{s,t}) &= \left(M^\varepsilon X^\varepsilon_{s,t}, \int_s^t M^\varepsilon X_{s,r} \otimes d(M^\varepsilon X^\varepsilon)_r \right),
\end{align*}
and the canonical lift of the Brownian motion $W$
\[
(W_{s,t}, \W_{s,t}) = \left(W_{s,t}, \int_s^t W_{s,r} \otimes \circ dW_r\right),
\]
where the integrals in the definition of $\Pbb^\varepsilon_{s,t}$ and $\W_{s,t}$ are in the Stratonovich sense.

The following proposition establishes the convergence of the ``renormalised'' paths $T_{v^\varepsilon}(P^\varepsilon_{s,t}, \Pbb^\varepsilon_{s,t})$ and $T_{v^\varepsilon}(Z^\varepsilon_{s,t}, \Z^\varepsilon_{s,t})$.

\begin{theorem}\label{thm:magneticConv}
Suppose that
\begin{equation}\label{eq:MBound}
\lim_{\varepsilon \rightarrow 0} |M^\varepsilon|\varepsilon^\kappa = 0 \; \textnormal{ for some } \kappa \in [0,1].
\end{equation}
Then for any $\alpha \in [0,1/2-\kappa/4)$ and $q < \infty$, it holds that $T_{v^\varepsilon}(P^\varepsilon, \Pbb^\varepsilon) \rightarrow (0,0)$ and $T_{v^\varepsilon}(Z^\varepsilon, \Z^\varepsilon) \rightarrow (W,\W)$ in $L^q$ and $\alpha$-H{\"o}lder topology as $\varepsilon \rightarrow 0$. More precisely, as $\varepsilon \rightarrow 0$, in $L^q$
\[
\sup_{s,t \in [0,T]} \frac{|P^\varepsilon_{s,t}|}{|t-s|^\alpha} + \sup_{s,t \in [0,T]} \frac{|\Pbb^\varepsilon_{s,t} + (t-s)v^\varepsilon|}{|t-s|^{2\alpha}} \rightarrow 0.
\]
and
\[
\sup_{s,t \in [0,T]} \frac{|Z^\varepsilon_{s,t} - W_{s,t}|}{|t-s|^\alpha} + \sup_{s,t \in [0,T]} \frac{|\Z^\varepsilon_{s,t} + (t-s)v^\varepsilon - \W_{s,t}|}{|t-s|^{2\alpha}} \rightarrow 0.
\]
\end{theorem}

The rest of the section is devoted to the proof of Theorem~\ref{thm:magneticConv} which builds on the proof of~\cite{FrizGassiat15} Theorem~1.

We set $Y^\varepsilon = P^\varepsilon/\varepsilon$ and obtain that
\begin{align*}
dY^\varepsilon_t &= -\varepsilon^2 M^\varepsilon Y^\varepsilon_t dt + \varepsilon^{-1}dW_t \\
dX^\varepsilon_t &= \varepsilon^{-1}Y^\varepsilon_t dt.
\end{align*}
For fixed $\varepsilon$, we introduce the Brownian motion $\tilde W^\varepsilon_\cdot = \varepsilon W_{\varepsilon^{-2} \cdot}$ and consider
\begin{align*}
d\tilde Y^\varepsilon_t &= -M^\varepsilon\tilde Y^\varepsilon_t dt + d\tilde W^\varepsilon_t.
%\\
%d\tilde X^\varepsilon_t &= \tilde Y^\varepsilon_t dt.
\end{align*}
Observe that we have the pathwise equalities
\begin{equation}\label{eq:pathwiseEq}
Y^\varepsilon_\cdot = \tilde Y^\varepsilon_{\varepsilon^{-2}\cdot},
%(Y^\varepsilon_\cdot, \varepsilon^{-1} X^\varepsilon_{\cdot}) = (\tilde Y^\varepsilon_{\varepsilon^{-2}\cdot}, \tilde X^\varepsilon_{\varepsilon^{-2}\cdot}),
\end{equation}
and since $Y^\varepsilon_0 = 0$, we have
\begin{equation}\label{eq:YSol}
\tilde Y^\varepsilon_t = \int_0^t e^{-M^\varepsilon (t-s)} d\tilde W^\varepsilon_s.
\end{equation}

%\begin{remark}
%Up to now, the only difference with the proof of Theorem 1 in Friz et al. is that $M^\varepsilon$ may depend on $\varepsilon$. If $M^\varepsilon$ is constant for $\varepsilon > 0$, then the law of $\tilde Y^\varepsilon$ does not depend on $\varepsilon$, but in the general case, it does.
%\end{remark}

The dependence of $M^\varepsilon$ on $\varepsilon$ is, by construction, only though $B^\varepsilon$, the anti-symmetric part of $M^\varepsilon$. In particular, since the symmetric part $A$ stays constant and has strictly positive spectrum, it follows that for some $\lambda > 0$, $\Real(\sigma(M^\varepsilon)) \subset (\lambda,\infty)$ for all $\varepsilon > 0$. In particular,
\begin{equation}\label{eq:expBound}
\sup_{\tau > 0} \sup_{\varepsilon > 0} \frac{|e^{-\tau M^\varepsilon}|}{e^{-\lambda \tau}} < \infty.
\end{equation}
We see then that
\[
\sup_{\varepsilon > 0} |C^\varepsilon| < \infty
\]
and
\begin{equation}\label{YtildeL2}
\sup_{\varepsilon > 0} \sup_{0 \leq t < \infty} \EEE{|\tilde Y^\varepsilon_t|^2} < \infty.
\end{equation}

\begin{lemma}\label{lem:YBounds}
There exists $C_1 > 0$ such that for all $\varepsilon \in (0,1]$ and $s,t \in [0,T]$
\[
\EEE{\left| Y^{\varepsilon}_{s,t} \right|^2}^{1/2} \leq C_1 \min\{\varepsilon^{-1}|t-s|^{1/2}, 1\}
\]
and
\[
\EEE{\left| \int_s^t Y^{\varepsilon}_r \otimes Y^{\varepsilon}_r dr - (t-s)C^\varepsilon \right|^2}^{1/2} \leq C_1 \min\{\varepsilon |t-s|^{1/2}, |t-s|\}.
\]
\end{lemma}

\begin{proof}
The first inequality is clear from~\eqref{eq:pathwiseEq},~\eqref{eq:YSol} and~\eqref{YtildeL2}. For the second, from~\eqref{eq:YSol}, we see that for every $r > 0$, $\tilde Y^{\varepsilon}_r$ has distribution $\NN(0, C^\varepsilon_r)$ where
\[
C^\varepsilon_r = \int_0^r e^{-M^\varepsilon (r-u)} e^{-(M^\varepsilon)^*(r-u)} du = \int_0^r e^{-M^\varepsilon u} e^{-(M^\varepsilon)^* u} du.
\]
Hence $Y^\varepsilon_r = \tilde Y_{\varepsilon^{-2}r}$ has distribution $\NN(0, C^\varepsilon_{\varepsilon^{-2} r})$. Thus
\[
\EEE{\int_s^t Y^{\varepsilon}_r \otimes Y^{\varepsilon}_r dr} = \int_s^t C^\varepsilon_{\varepsilon^{-2} r} dr = \int_s^t \int_0^{\varepsilon^{-2} r} e^{-M^\varepsilon u} e^{-(M^\varepsilon)^* u} du dr =: \mu_{s,t}^\varepsilon.
\]
Observe that from~\eqref{eq:expBound}
\begin{align*}
|\mu_{s,t}^\varepsilon - (t-s) C^\varepsilon| &\leq \int_s^t \int_{\varepsilon^{-2} r}^\infty |e^{-M^\varepsilon u} e^{-(M^\varepsilon)^* u}| du dr \\
&\leq C_2\int_s^t \int_{\varepsilon^{-2}r}^\infty  e^{-2\lambda u} dudr \\
&\leq C_3 \int_s^t e^{-2\lambda \varepsilon^{-2}r}dr \\
&\leq C_4 \min\{\varepsilon^2, |t-s|\} \\
&\leq C_4 \min\{\varepsilon |t-s|^{1/2}, |t-s|\}.
\end{align*}
We now claim that
\[
\EEE{\left| \int_s^t Y^{\varepsilon}_r \otimes Y^{\varepsilon}_r dr - \mu_{s,t}^\varepsilon \right|^2} \leq C_5 \min\{\varepsilon^2 |t-s|, |t-s|^2\},
\]
from which the conclusion follows. Indeed, by Fubini and Wick's formula
\begin{align*}
\EEE{\left(\int_s^t Y^{\varepsilon,i}_r Y^{\varepsilon,j}_r dr\right)^2}
&= \int_{[s,t]^2} \EEE{Y^{\varepsilon,i}_r Y^{\varepsilon,j}_r Y^{\varepsilon,i}_u Y^{\varepsilon,j}_u} dr du \\
&= \int_{[s,t]^2} \EEE{Y^{\varepsilon,i}_r Y^{\varepsilon,j}_r}\EEE{Y^{\varepsilon,i}_u Y^{\varepsilon,j}_u} dr du \\
&+ \int_{[s,t]^2}  \EEE{Y^{\varepsilon,i}_r Y^{\varepsilon,i}_u}\EEE{Y^{\varepsilon,j}_r Y^{\varepsilon,j}_u} dr du \\
&+ \int_{[s,t]^2}  \EEE{Y^{\varepsilon,i}_r Y^{\varepsilon,j}_u}\EEE{Y^{\varepsilon,j}_r Y^{\varepsilon,i}_u} dr du \\
%&\leq (\mu_{i,j}^\varepsilon)_{s,t}^2 + \int_{[s,t]^2} \EEE{Y^{\varepsilon,i}_u Y^{\varepsilon,j}_r}^2 + \EEE{Y^{\varepsilon,i}_r Y^{\varepsilon,j}_u}^2 dr du \\
&\leq (\mu_{i,j}^\varepsilon)_{s,t}^2 + 4\int_{[s,t]^2} \left|\EEE{Y^{\varepsilon}_u \otimes Y^{\varepsilon}_r}\right|^2 \mathbf{1}\{r \leq u\} dr du.
\end{align*}

Observe that for $r \leq u$
\[
\EEE{Y^\varepsilon_u \mid Y^\varepsilon_r} = e^{-\varepsilon^{-2} M^\varepsilon(u-r)} Y_r^\varepsilon.
\]
and so
\[
|\EEE{Y^\varepsilon_u\otimes Y^\varepsilon_r}|^2 \mathbf{1}\{r \leq u\}  \leq C_6 e^{-\varepsilon^{-2}2\lambda(u-r)} |C^\varepsilon_{\varepsilon^{-2}r}| \leq C_7 e^{-\varepsilon^{-2}2\lambda(u-r)}.
\]
Thus
\begin{align*}
\EEE{\left(\int_s^t Y^{\varepsilon,i}_r Y^{\varepsilon,j}_r dr - (\mu_{i,j}^\varepsilon)_{s,t}\right)^2}
&= \EEE{\left(\int_s^t Y^{\varepsilon,i}_r Y^{\varepsilon,j}_r dr\right)^2} - (\mu_{i,j}^\varepsilon)_{s,t}^2 \\
&\leq C_8 \int_s^t \int_r^t e^{-\varepsilon^{-2}2\lambda(u-r)} du dr \\
&\leq C_9 \min\{\varepsilon^2 |t-s|, |t-s|^2\}
\end{align*}
as claimed.
\end{proof}

\begin{lemma}\label{lem:PBounds}
There exists $C_{10} > 0$ such that for all $\varepsilon \in (0,1]$ and $s,t \in [0,T]$
\[
\norm{P^\varepsilon_{s,t}}_{L^2} \leq C_{10} \min\{\varepsilon, |t-s|^{1/2}\}
\]
and
\[
\norm{\Pbb^\varepsilon_{s,t} + (t-s)v^\varepsilon}_{L^2} \leq C_{10} |M^\varepsilon| \min\{\varepsilon |t-s|^{1/2}, |t-s|\}
\]
\end{lemma}

\begin{proof}
The first inequality is immediate from Lemma~\ref{lem:YBounds}. For the second, we have
\begin{align*}
\Pbb^\varepsilon_{s,t} &= \varepsilon^2 \int_s^t Y^\varepsilon_{s,r} \otimes \circ dY^\varepsilon_r \\
&= -\int_s^t Y^\varepsilon_{s,r} \otimes M^\varepsilon Y^\varepsilon_r dr + \varepsilon\int_s^t Y^\varepsilon_{s,r}\otimes dW_r + \frac{1}{2}(t-s)I.
\end{align*}
Since $Y^\varepsilon_{s,r} \otimes M^\varepsilon Y^\varepsilon_r = (Y^\varepsilon_{s,r} \otimes Y^\varepsilon_r)(M^\varepsilon)^*$ and we can directly verify that $v^\varepsilon = C^\varepsilon(M^\varepsilon)^* - \frac{1}{2}I$, we have
\[
\Pbb^\varepsilon_{s,t} + (t-s)v^\varepsilon = -\left(\int_s^t Y^\varepsilon_{s,r} \otimes Y^\varepsilon_r dr - (t-s)C^\varepsilon\right) (M^\varepsilon)^* + \varepsilon \int_s^t Y^\varepsilon_{s,r} \otimes dW_r.
\]
From Lemma~\ref{lem:YBounds}, we see that
\[
\norm{\varepsilon \int_s^t Y^\varepsilon_{s,r} \otimes dW_r}_{L^2} \leq C_{11}\min\{\varepsilon|t-s|^{1/2},|t-s|\}.
\]
Furthermore, by Fubini and Wick's formula, we can readily show
\[
\norm{\int_s^t Y^\varepsilon_s \otimes Y^\varepsilon_r dr}_{L^2} \leq C_{12} \min\{\varepsilon |t-s|^{1/2}, |t-s|\}.
\]
It now follows from Lemma~\ref{lem:YBounds} that
\[
\norm{\Pbb^\varepsilon_{s,t} + (t-s)v^\varepsilon}_{L^2} \leq  C_{13} |M^\varepsilon| \min\{\varepsilon |t-s|^{1/2}, |t-s|\}.
\]
\end{proof}

\begin{lemma}\label{lem:ZBounds}
There exists $C_{14} > 0$ such that for all $\varepsilon \in (0,1]$ and $s,t \in [0,T]$
\[
\EEE{|Z^\varepsilon_{s,t} - W_{s,t}|^2}^{1/2} \leq C_{14} \min\{\varepsilon, |t-s|^{1/2}\}
\]
and
\[
\EEE{\left|\Z^\varepsilon_{s,t} + (t-s)v^\varepsilon - \W_{s,t}\right|^2}^{1/2} \leq C_{14} |M^\varepsilon| \min\{\varepsilon |t-s|^{1/2}, |t-s|\}.
\]
\end{lemma}

\begin{proof}
The first inequality follows from $Z^\varepsilon_{s,t} = W_{s,t} - \varepsilon Y^\varepsilon_{s,t}$ and Lemma~\ref{lem:YBounds}. For the second, we have
\begin{align*}
\int_s^t Z^\varepsilon_{s,r} \otimes dZ^\varepsilon_r
&= \int_s^t Z^\varepsilon_{s,r} \otimes dW_r - \varepsilon\left( \int_s^t Z^\varepsilon_{r} \otimes dY^\varepsilon_r - Z^\varepsilon_s\otimes Y^\varepsilon_{s,t}\right) \\
%&= \int_s^t Z^\varepsilon_{s,r} \otimes dW_r - \varepsilon \left( Z^\varepsilon_t\otimes Y^\varepsilon_t - Z^\varepsilon_s\otimes Y^\varepsilon_s - \int_s^t dZ_r \otimes Y^\varepsilon_r - Z^\varepsilon_s\otimes (Y^\varepsilon_t - Y^\varepsilon_s)\right) \\
&= \int_s^t Z^\varepsilon_{s,r} \otimes dW_r - \varepsilon \left( Z^\varepsilon_t\otimes Y^\varepsilon_t - \int_s^t dZ_r \otimes Y^\varepsilon_r - Z^\varepsilon_s\otimes Y^\varepsilon_t\right) \\
&= \int_s^t Z^\varepsilon_{s,r} \otimes dW_r - \varepsilon Z_{s,t}^\varepsilon\otimes Y_t^\varepsilon + \int_s^t M^\varepsilon Y^\varepsilon_r \otimes Y^\varepsilon_r dr.
\end{align*}
We see that
\[
\norm{\int_s^t Z^\varepsilon_{s,r} \otimes dW_r - \int_s^t W_{s,r} \otimes dW_r}^2_{L^2} \leq C_{15}\min\{\varepsilon^2|t-s|, |t-s|^2\}.
\]
Furthermore, by Fubini and Wick's formula, we can readily show
\[
\norm{\varepsilon Z^\varepsilon_{s,t}\otimes Y^\varepsilon_t}^2_{L^2} = \norm{\int_s^t M^\varepsilon Y^\varepsilon_{r}\otimes Y^\varepsilon_t dr}^2_{L^2} \leq C_{16}|M^\varepsilon| \min\{\varepsilon^2 |t-s|, |t-s|^2\}.
\]
%(in fact, by Gaussian chaos, we also have $\norm{\varepsilon Z^\varepsilon_{s,t}\otimes Y^\varepsilon_t}^2_{L^2} \leq \varepsilon^2 |t-s|$).
Finally, by Lemma~\ref{lem:YBounds}
\[
\norm{\int_s^t M^\varepsilon Y^\varepsilon_r \otimes Y^\varepsilon_r dr - (t-s)M^\varepsilon C^\varepsilon|}_{L^2} \leq C_{17}|M^\varepsilon| \min \{\varepsilon |t-s|^{1/2}, |t-s|\}.
\]
It follows that
\[
\norm{\Z^\varepsilon_{s,r} - \W_{s,t} - (t-s)(M^\varepsilon C^\varepsilon - \frac{1}{2}I)}_{L^2} \leq C_{18}|M^\varepsilon| \min \{\varepsilon |t-s|^{1/2}, |t-s|\}.
\]
We can directly verify $v^\varepsilon = -M^\varepsilon C^\varepsilon + \frac{1}{2}I$, from which the conclusion follows.
\end{proof}

\begin{proof}[Proof of Theorem~\ref{thm:magneticConv}]
Observe that condition~\eqref{eq:MBound} implies that
\[
\lim_{\varepsilon \rightarrow 0}|M^\varepsilon|\varepsilon = 0.
\]
From Lemmas~\ref{lem:PBounds} and~\ref{lem:ZBounds}, along with Gaussian chaos, we thus obtain the pointwise convergence as $\varepsilon \rightarrow 0$ for any $q < \infty$ and $s,t \in [0,T]$ in $L^q$
\[
|P^\varepsilon_{s,t}| + |\Pbb^\varepsilon_{s,t} + (t-s)v^\varepsilon|^{1/2} \rightarrow 0
\]
and
\[
|Z^\varepsilon_{s,t} - W_{s,t}| + |\Z^\varepsilon_{s,t} + (t-s)v^\varepsilon - \W_{s,t}|^{1/2} \rightarrow 0.
\]
Furthermore, since $\min\{\varepsilon |t-s|^{1/2}, |t-s|\} \leq \varepsilon^{\kappa}|t-s|^{1-\kappa/2}$ for all $\kappa \in [0,1]$, condition~\eqref{eq:MBound}, Lemmas~\ref{lem:PBounds} and~\ref{lem:ZBounds}, and Gaussian chaos imply that for any $q < \infty$ there exists $C_q > 0$ such that for all $s,t \in [0,T]$ and
\begin{align*}
\sup_{\varepsilon \in (0,1]} \EEE{|P^\varepsilon_{s,t}|^q} &\leq C_q|t-s|^{q/2}, \\
\sup_{\varepsilon \in (0,1]} \EEE{|Z^\varepsilon_{s,t} - W_{s,t}|^q} &\leq C_q|t-s|^{q/2}
\end{align*}
and
\begin{align*}
\sup_{\varepsilon \in (0,1]} \EEE{|\Pbb^\varepsilon_{s,t} + (t-s)v^\varepsilon|^q} &\leq C_q|t-s|^{q(1-\kappa/2)}, \\
\sup_{\varepsilon \in (0,1]} \EEE{|\Z^\varepsilon_{s,t} + (t-s)v^\varepsilon - \W_{s,t}|^q} &\leq C_q|t-s|^{q(1-\kappa/2)}.
\end{align*}
Applying Theorem~A.13 of~\cite{FrizVictoir10} completes the proof.
\end{proof}


\section{Rough lead-lag process}\label{subsec:Hoff}

Consider a path $X : [0,1] \mapsto \R^d$. Let $n \geq 1$ be an integer and write for brevity $X^n_i = X_{i/n}$. Consider the piecewise linear path $\tilde X^n : [0,1] \mapsto \R^{2d}$ defined by
\begin{align*}
\tilde X^n_{2i/2n} &= (X^n_i, X^n_i), \\
\tilde X^n_{(2i+1)/2n} &= (X^n_i, X^n_{i+1}),
\end{align*}
and linear on the intervals $\left[\frac{2i}{2n}, \frac{2i+1}{2n}\right]$ and $\left[\frac{2i+1}{2n}, \frac{2i+2}{2n}\right]$ for all $i = 0,\ldots, n-1$. Note that this is a variant of the Hoff process considered in~\cite{Flint16}.

Denote by $\tilde \Xbf^n_{s,t} = \exp(\tilde X^n_{s,t} + \Abb^n_{s,t})$ the level-$2$ lift of $\tilde X^n$, where $\Abb^n_{s,t}$ is the $(2d) \times (2d)$ anti-symmetric L{\'e}vy area matrix given by
\[
\Abb^n_{s,t}
%= \left( \begin{array}{cc} \Abb^{b,b}_{s,t} & \Abb^{b,f}_{s,t}  \\ \Abb^{f,b}_{s,t} & \Abb^{f,f}_{s,t} \end{array} \right)
= \frac{1}{2}\left(\int_s^t \tilde X^n_{s,r} \otimes d\tilde X^n_r - \int_s^t \tilde X^n_{s,r} \otimes d\tilde X^n_r\right).
\]


Let $H \in (0,1)$ and consider a fractional Brownian motion $B^H$ with covariance $R(s,t) = \frac{1}{2}(t^{2H} + s^{2H} - |t-s|^{2H})$. Let $X : [0,1] \mapsto \R^d$ be $d$ independent copies of $B^H$.

Recall the definition of $T_v$ from~\eqref{eq:TvDef}. We are interested in the convergence in rough path topologies of $T_{\tilde v^n}(\tilde \Xbf^n)$ where $\tilde v^n \in \g^{(2)}(\R^{2d})$ is appropriately chosen. Define the (diagonal) $d\times d$ matrix
\[
v^n = \frac{1}{2}\EEE{\sum_{k=0}^{n-1} (X^n_{k+1}-X^n_k) \otimes (X^n_{k+1}-X^n_k)} = \frac{n^{1-2H}}{2} I,
\]
and the anti-symmetric $(2d) \times (2d)$ matrix
\[
\tilde v^n = \left( \begin{array}{cc} 0 & -v^n  \\ v^n  & 0 \end{array} \right) \in \g^{(2)}(\R^{2d}).
\]
Finally, consider the path $\tilde X = (X,X) : [0,1] \mapsto \R^{2d}$, its canonically defined L{\'e}vy area $\Abb$ (which exists for $1/4 < H \leq 1$), and its level-$2$ lift $\tilde \Xbf = \exp(\tilde X + \Abb)$. The following is the main result of this subsection.

\begin{theorem}\label{thm:HoffConv}
Suppose $1/4 < H \leq 1/2$. Then for all $\alpha \in [0, H)$ and $q < \infty$, it holds that $T_{\tilde v^n}(\tilde \Xbf^n) \rightarrow \tilde \Xbf$ in $L^q$ and $\alpha$-H{\"o}lder topology. More precisely, as $n \rightarrow \infty$, in $L^q$
\[
\sup_{s,t \in [0,T]} \frac{|\tilde X^n_{s,t} - \tilde X_{s,t}|}{|t-s|^\alpha} + \sup_{s,t \in [0,T]} \frac{|\Abb^n_{s,t} + (t-s)\tilde v^n - \Abb_{s,t}|}{|t-s|^{2\alpha}} \rightarrow 0.
\]
\end{theorem}

The rest of the section is devoted to the proof of Theorem~\ref{thm:HoffConv}. We first state two lemmas which are purely deterministic.

Let $Y^n : [0,1] \mapsto \R^{d}$ be the piecewise linear interpolation of $X$ over the partition $\left(0, \frac{1}{n}, \ldots, \frac{n-1}{n}, 1\right)$, let $\tilde Y^n = (Y^n,Y^n) : [0,1] \mapsto \R^{2d}$,
% be two identical copies of $Y$ (so that $\tilde Y^i_t = \tilde Y^{d+i}_{t} = Y^i_t$ for all $1 \leq i \leq d$),
and let $\Ybb^n$ be the L{\'e}vy area of $\tilde Y^n$.


\begin{lemma}\label{lem:XYDiff}
Let $s \in [\frac{m}{n},\frac{m+1}{n}]$ and $t \in [\frac{k}{n}, \frac{k+1}{n}]$ with $s < t$, and define
\begin{align*}
\Delta_1 &= n\left(\frac{m+1}{n} \wedge t - s \right) |X^n_{m+1} - X^n_m|, \\
\Delta_2 &= |X^n_k - X^n_{m+1}| \; \textnormal{ if $k > m$}, \; \; 0 \textnormal{ if  $k= m$} \\
\Delta_3 &= n\left(t-\frac{k}{n}\vee s\right)|X^n_{k+1} - X^n_k|.
\end{align*}
There exists a constant $C_1 > 0$ such that for all $n \geq 1$ and $0 \leq s < t \leq 1$, it holds that
\[
|\tilde X^n_{s,t} - \tilde Y^n_{s,t}| \leq C_1\left( \Delta_1 + \Delta_3 \right).
\]
and, if $k > m$, $|\Abb^n_{s,t} - \Abb^n_{\frac{m+1}{n}, \frac{k}{n}}|$ and $|\Ybb^n_{s,t} - \Ybb^n_{\frac{m+1}{n}, \frac{k}{n}}|$ are bounded above by
\[
C_1 \left(\Delta_1^2 + (\Delta_1+ \Delta_2 + \Delta_3)\Delta_3 + \Delta_1 \Delta_2 \right).
\]
\end{lemma}

\begin{proof}
Direct calculation and triangle inequality.
\end{proof}

The second part of the above lemma essentially allows us to work over the partition $\left(0, \frac{1}{n}, \ldots, \frac{n-1}{n}, 1\right)$, on which computations are easier.

\begin{lemma}\label{lem:partitionPoints}
Suppose $0 \leq m \leq k \leq n$.

1) For all pairs $1 \leq i,j \leq d$ and $d+1 \leq i,j \leq 2d$
\[
\left(\Abb^n_{\frac{m}{n}, \frac{k}{n}}\right)^{i,j} = \left(\Ybb^n_{\frac{m}{n}, \frac{k}{n}}\right)^{i,j}.
\]

2) For all $1 \leq i \leq d < j \leq 2d$
\[
\left(\Abb^n_{\frac{m}{n}, \frac{k}{n}}\right)^{i,j} = \left(\Ybb^n_{\frac{m}{n}, \frac{k}{n}}\right)^{i,j} - \frac{1}{2} \sum_{r=m}^{k-1} (X^{n,i}_{r+1}-X^{n,i}_r)(X^{n,j}_{r+1}-X^{n,j}_r)
\]
\end{lemma}

\begin{proof}
Denote $\tilde X^n = (M^n, N^n)$, so that $M^n$ is the lag component, and $N^n$ is the lead. The first equality is clear since $M^n$ and $N^n$ are simply reparametrisations of $Y^n$ over the interval $[\frac{m}{n},\frac{k}{n}]$. For the second, observe that
\[
\int_{m/n}^{k/n} M^{n,i}_{m/n, r} dN^{n,j}_r = \sum_{r=m}^{k-1} (X^{n,i}_{r}-X^{n,i}_{m}) (X^{n,j}_{r+1} - X^{n,j}_{r})
%= \sum_{r=m}^{k-1} X^{n,i}_{r}(X^{n,j}_{r+1} - X^{n,j}_{r}) - X^{n,i}_{m} (X^{m,j}_{k} - X^{n,j}_{m})
\]
and
\[
\int_{m/n}^{k/n} N^{n,j}_{m/n, r} dM^{n,i}_r = \sum_{r=m}^{k-1} (X^{n,j}_{r+1}-X^{n,j}_{m}) (X^{n,i}_{r+1} - X^{n,i}_{r}).
%= \sum_{r=m}^{k-1} X^j_{r+1}(X^i_{r+1} - X^i_{r}) - X^j_{m} (X^i_{k} - X^i_{m}).
\]
Remark now that the signature of $Y^n$ over $[\frac{m}{n},\frac{k}{n}]$ is
\[
e^{X_{m+1} - X_{m}} \ldots e^{X_k - X_{k-1}},
\]
so that a calculation with the CBH formula gives
\begin{align*}
\left(\Ybb^n_{m/n,k/n}\right)^{i,j}
%&= \frac{1}{2}\left( \int_{m/n}^{k/n} M^i_{m/n, r} dM^j_r - \int_{m/n}^{k/n} M^j_{m/n, r} dM^i_r \right) \\
&= \frac{1}{2}\sum_{r=m}^{k-1} (X^{n,i}_r-X^{n,i}_m)(X^{n,j}_{r+1}-X^{n,j}_r) - (X^{n,j}_r-X^{n,j}_m)(X^{n,i}_{r+1}-X^{n,i}_r).
%&= \frac{1}{2}\sum_{r=1}^k X^i_r X^j_{r+1} - X^j_{r}X^i_{r+1} - X^i_m(X^j_k-X^j_m) + X^j_m(X^i_k-X^i_m).
\end{align*}
Using the fact that
\begin{align*}
\left(\Abb^n_{m/n,k/n}\right)^{i,j} &= \frac{1}{2}\left( \int_{m/n}^{k/n} M^{n,i}_{m/n, r} dN^{n,j}_r - \int_{m/n}^{k/n} N^{n,j}_{m/n, r} dM^{n,i}_r \right),
%&= \frac{1}{2}\sum_{r=m}^{k-1} X^i_{r}(X^j_{r+1} - X^j_{r})  - X^j_{r+1}(X^i_{r+1} - X^i_{r}) -X^i_{m} (X^j_{k} - X^j_{m}) + X^j_{m} (X^i_{k} - X^i_{m}) \\
%&= \Ybb^{i,j}_{m/n,k/n} - \frac{1}{2} \sum_{r=m}^{k-1} (X^i_{r+1}-X^i_r)(X^j_{r+1}-X^j_r).
\end{align*}
the conclusion readily follows.
\end{proof}

We now return to the specific case that $X : [0,1] \mapsto \R^d$ is given by $d$-independent copies of a fractional Brownian motion with Hurst parameter $H \in (0,1)$. In particular, this implies that for all $s,t,a,b \in [0,1]$
\begin{equation}\label{eq:covariance}
\EEE{(X^i_t-X^i_s)(X^j_b-X^j_a)} = \delta_{i,j} \frac{1}{2}(|t-a|^{2H} + |s-b|^{2H} - |t-b|^{2H} - |s-a|^{2H}).
\end{equation}
Consider the $(2d) \times (2d)$ anti-symmetric matrix
\[
\Pbb^n_{s,t} = \Abb^n_{s,t} - \Ybb^n_{s,t} + (t-s)\tilde v^n.
\]

\begin{lemma}\label{lem:PControlPartition}
There exists $C_2 > 0$ such that for all $H \leq 1/2$, $n \geq 1$, and $0 \leq m \leq k \leq n$
\[
\norm{\Pbb^n_{m/n,k/n}}_{L^2} \leq C_2 \frac{(k-m)^{1/2}}{n^{2H}}.
\]
\end{lemma}

\begin{proof}
Denote $K = k-m$. By part (2) of Lemma~\ref{lem:partitionPoints}, we have
\[
\norms{\Pbb^n_{m/n,k/n}} \leq \sum_{i,j=1}^d \norms{\sum_{r=m}^{k-1} (X^{n,i}_{r+1} - X^{n,i}_r)(X^{n,j}_{r+1}- X^{n,j}_r) - \frac{K}{n}v^n_{i,j}}.
\]
Observe moreover that
\[
\frac{K}{n}v^n_{i,j} = \EEE{\sum_{r=m}^{k-1}(X^{n,i}_{r+1} - X^{n,i}_r)(X^{n,j}_{r+1}- X^{n,j}_r)} = \delta_{i,j} K n^{-2H},
\]
and that for all $r,\ell \in \{0,\ldots, n-1\}$
\[
\EEE{(X^{n,i}_{r+1} - X^{n,i}_r)(X^{n,i}_{\ell+1} - X^{n,i}_\ell)} = \frac{n^{-2H}}{2}(|r-\ell+1|^{2H} + |r-\ell-1)|^{2H} - 2|r-\ell|^{2H}).
\]
Then for all $i \neq j$, by independence of the components of $X$,
\begin{align*}
& \EEE{\left(\sum_{r=m}^{k-1}  (X^{n,i}_{r+1} - X^{n,i}_r)(X^{n,j}_{r+1}- X^{n,j}_r) \right)^2}
\\ &= \EEE{\sum_{r=m}^{k-1} \sum_{\ell=m}^{k-1} X^i_{r,r+1}X^i_{\ell,\ell+1} X^j_{r,r+1} X^j_{\ell,\ell+1}} \\
&= \sum_{r=m}^{k-1} \sum_{\ell=m}^{k-1} \EEE{(X^{n,i}_{r+1} - X^{n,i}_r)(X^{n,i}_{\ell+1} - X^{n,i}_{\ell})}^2 \\
&= \sum_{r=m}^{k-1} \sum_{\ell=m}^{k-1} \frac{n^{-4H}}{4}\left(|r-\ell+1|^{2H} + |r-\ell-1|^{2H} - 2|r-\ell|^{2H} \right)^2 \\
&= \frac{n^{-4H}}{4} \sum_{x=-K+1}^{K-1} (K-|x|)(|x+1|^{2H} + |x-1|^{2H} - 2x^{2H})^2 \\
&=: \psi(n,K).
\end{align*}
Likewise for $i=j$, by Wick's formula,
\begin{align*}
\EEE{\left(\sum_{r=m}^{k-1} (X^{n,i}_{r+1} - X^{n,i}_r)^2 \right)^2}
%&= \EEE{\sum_{r=m}^{k-1} \sum_{\ell=m}^{k-1} (X^i_{r,r+1}X^i_{\ell,\ell+1})^2} \\
%&= \sum_{r=m}^{k-1} \sum_{\ell=m}^{k-1} \EEE{(X^i_{r,r+1}X^i_{\ell,\ell+1})^2} \\
%&= \sum_{r=m}^{k-1} \EEE{(X^i_{r,r+1})^4} + 2\sum_{m \leq r < \ell \leq k-1} \\
&= \sum_{r=m}^{k-1} \sum_{\ell=m}^{k-1} \EEE{(X^{n,i}_{r+1} - X^{n,i}_{r})^2} \EEE{(X^{n,i}_{\ell+1} - X^{n,i}_{\ell})^2} \\
&+ 2\EEE{(X^{n,i}_{r+1} - X^{n,i}_{r})(X^{n,i}_{\ell+1} - X^{n,i}_{\ell})}^2  \\
%&= \sum_{r=m}^{k-1} \sum_{\ell=m}^{k-1} n^{-4H} + 2\frac{n^{-4H}}{4}\left(|r-\ell+1|^{2H} + |r-\ell-1|^{2H} + 2|r-\ell|^{2H} \right)^2 \\
&= K^2 n^{-4H} + 2\psi(n,K) \\
&= \left(\frac{K}{n}v^n_{i,i}\right)^2 + 2\psi(n,K).
\end{align*}
It hence follows that
\[
\norm{\Pbb^n_{m/n, k/n}}_{L^2}^2 \leq C_3\psi(n,K).
\]
The conclusion now follows since one can readily show that there exists $C_4 > 0$ such that for all $H \leq 1/2$, $n \geq 1$, and $0 \leq K \leq n$
\[
\psi(n,K) \leq C_4K n^{-4H}
\]
(in fact the inequality holds for all $H < 3/4$, though with a constant in general depending on $H$).
\end{proof}


\begin{lemma}\label{lem:YPBounds}
There exists $C_5 > 0$ such that for all $H \in \left(\frac{1}{4},\frac{1}{2}\right]$, $n \geq 1$ and $0 \leq s < t \leq 1$
\[
%\norm{\Ybb^n_{s,t}}_{L^2},
\norm{\Pbb^n_{s,t}}_{L^2} \leq C_5|t-s|^{2H}.
\]
\end{lemma}

\begin{proof}
Suppose $s \in [\frac{m}{n},\frac{m+1}{n}]$ and $t \in [\frac{k}{n}, \frac{k+1}{n}]$.
If $m=k$, then $\Abb^n_{s,t} = \Ybb^n_{s,t} = 0$ and $|t-s| < n^{-1}$, so that
\[
|\Pbb^n_{s,t}| = \left(t- s\right)|\tilde v^n| \leq |t-s|^{2H}.
\]
For the case $k > m$, following the notation of Lemma~\ref{lem:XYDiff}, note that $\EEE{\Delta_1^2}$ and $\EEE{\Delta_3^2}$ are bounded above by
\[
n^{-2H} \min\{n^2|t-s|^2,1\}.
\]
It readily follows that for all $\ell \in \{1,2,3\}$
\[
\EEE{\Delta^2_\ell} \leq |t-s|^{2H}.
\]
Moreover, we have
\[
\left(t-\frac{k}{n} + \frac{m+1}{n} - s\right)|\tilde v^n| \leq \min \{|t-s|, n^{-1} \} n^{1-2H} \leq |t-s|^{2H}.
\]
Hence Lemma~\ref{lem:XYDiff} implies that
\begin{multline*}
|\Pbb^n_{s,t} - \Pbb^n_{(m+1)/n, k/n}| \leq 2C_1\left(\Delta_1^2 + (\Delta_1+ \Delta_2 + \Delta_3)\Delta_3 + \Delta_1 \Delta_2 \right) \\ + \left(t-\frac{k}{n} + \frac{m+1}{n} - s\right)|\tilde v^n|,
\end{multline*}
and so by Gaussian chaos
\[
\norm{\Pbb^n_{s,t} - \Pbb^n_{(m+1)/n, k/n}}_{L^2} \leq C_6|t-s|^{2H}.
\]
The conclusion now follows from Lemma~\ref{lem:PControlPartition} since $(k-m-1)^{1/2}n^{-2H} \leq |t-s|^{2H}$ for all $H \geq 1/4$.
\end{proof}


\begin{proof}[Proof of Theorem~\ref{thm:HoffConv}]
Let $0 \leq s < t \leq 1$. We observe that as $n \rightarrow \infty$, it readily follows from Lemmas~\ref{lem:XYDiff} and~\ref{lem:PControlPartition} that in $L^q$
\[
|\tilde X^n_{s,t} - \tilde Y^n_{s,t}| \rightarrow 0
\]
and
\[
|\Abb^n_{s,t} + (t-s)\tilde v^n - \Ybb^n_{s,t}| \rightarrow 0.
\]
Furthermore, by Gaussian chaos, Lemma~\ref{lem:XYDiff} implies
\[
\sup_{n \geq 1} \EEE{|\tilde X^n_{s,t} - \tilde Y^n_{s,t}|^q} \leq C_q |t-s|^{qH},
\]
while Lemma~\ref{lem:YPBounds} implies
\[
\sup_{n \geq 1} \EEE{|\Abb^n_{s,t} + (t-s)\tilde v^n - \Ybb^n_{s,t}|^q} \leq C_q |t-s|^{2qH}.
\]
Applying Theorem~A.13 of~\cite{FrizVictoir10}, and the fact that $\exp(\tilde Y^n + \Ybb^n) \rightarrow \tilde \Xbf$ in $\alpha$-H{\"o}lder topology in $L^q$ (\cite{FrizVictoir10} Theorem~15.42), completes the proof.
\end{proof}


\bibliographystyle{plain}
\bibliography{AllRefs}

\end{document}
