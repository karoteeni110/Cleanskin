\usepackage[utf8]{inputenc}
\usepackage[innermargin=0.545in,outermargin=0.545in,top=0.545in,bottom=1in]{geometry}
\usepackage[english]{babel}
\usepackage[T1]{fontenc}
\usepackage{epsfig}
\usepackage{amsmath, amssymb, amsbsy}
\usepackage{mathdots}
\usepackage{xspace}
\usepackage[noend]{algpseudocode}
\usepackage{algorithm}
\usepackage{algorithmicx}
\usepackage{color}
\usepackage{cite}
\usepackage{booktabs}
\usepackage{verbatim}
\usepackage[colorlinks=true,citecolor=blue,linkcolor=blue,urlcolor=blue]{hyperref}
\usepackage{url}
\usepackage{lipsum}
\usepackage{tikz}
\usepackage{enumitem}
\usetikzlibrary{arrows,matrix,positioning}
\usepackage{colortbl}
\usepackage{flushend}

\newtheorem{theorem}{Theorem}
\newtheorem{lemma}[theorem]{Lemma}
\newtheorem{corollary}[theorem]{Corollary}
\newtheorem{proposition}[theorem]{Proposition}
\newtheorem{conjecture}[theorem]{Conjecture}
\newtheorem{question}[theorem]{Question}
\newtheorem{problem}[theorem]{Problem}
\newtheorem{example}[theorem]{Example}
\newtheorem{remark}[theorem]{Remark}
\newtheorem{definition}{Definition}

\usepackage[final,tracking=true,kerning=true,spacing=true,factor=1100,stretch=10,shrink=20]{microtype}

\algrenewcommand\alglinenumber[1]{{\scriptsize#1}}
\algrenewcommand\algorithmicrequire{\textbf{Input:}}
\algrenewcommand\algorithmicensure{\textbf{Output:}}
\newcommand{\Ifline}[2]{\State \textbf{if }#1{ \textbf{then} }#2}


% Setup reference system (i.e. vref/cref commands)
\usepackage{varioref}        % for \vref{:...} gives "As 1.2 on page 4"
% NOTE: Has to be late in loading order
\usepackage{fancyref}
\renewcommand{\fancyrefdefaultformat}{plain}
\makeatletter
\def\mkfancyprefix#1#2{%
\expandafter\def\csname fancyref#1labelprefix\endcsname{#1}%
% plain lowercase
\begingroup\def\x{\endgroup\frefformat{plain}}%
    \expandafter\x\csname fancyref#1labelprefix\endcsname
    {\MakeLowercase{#2}\fancyrefdefaultspacing##1}%
% plain uppercase
\begingroup\def\x{\endgroup\Frefformat{plain}}%
    \expandafter\x\csname fancyref#1labelprefix\endcsname
    {#2\fancyrefdefaultspacing##1}%
% vario lowercase
\begingroup\def\x{\endgroup\frefformat{vario}}%
    \expandafter\x\csname fancyref#1labelprefix\endcsname
    {\MakeLowercase{#2}\fancyrefdefaultspacing##1##3}%
% vario uppercase
\begingroup\def\x{\endgroup\Frefformat{vario}}%
    \expandafter\x\csname fancyref#1labelprefix\endcsname
    {#2\fancyrefdefaultspacing##1##3}%
}
\makeatother
\fancyrefchangeprefix{\fancyrefeqlabelprefix}{eqn}
\mkfancyprefix{ssec}{Section}
\mkfancyprefix{tbl}{Table}
\mkfancyprefix{thm}{Theorem}
\mkfancyprefix{lem}{Lemma}
\mkfancyprefix{cor}{Corollary}
\mkfancyprefix{prop}{Proposition}
\mkfancyprefix{prob}{Problem}
\mkfancyprefix{alg}{Algorithm}
\mkfancyprefix{inv}{Invariant}
\mkfancyprefix{ex}{Example}
\mkfancyprefix{line}{Line}
\mkfancyprefix{def}{Definition}
\mkfancyprefix{itm}{Item}
\mkfancyprefix{app}{Appendix}
\mkfancyprefix{rem}{Remark}
\newcommand{\cref}[1]{\Fref{#1}}

\def\ve#1{{\mathchoice{\mbox{\boldmath$\displaystyle #1$}}%
              {\mbox{\boldmath$\textstyle #1$}}%
              {\mbox{\boldmath$\scriptstyle #1$}}%
              {\mbox{\boldmath$\scriptscriptstyle #1$}}}}




%% Todos

\definecolor{sunset}{rgb}{1,0.5,.05}
\definecolor{marine}{rgb}{0,0,.7}
\definecolor{navy}{rgb}{0,0,.5}
\definecolor{forest}{rgb}{0,.6,0}
\definecolor{brown}{rgb}{0.59, 0.29, 0.0}

\newcommand{\todo}[1]{{\color{red}[#1]}}
\newcommand{\jsrn}[1]{{\color{sunset}[/jsrn: #1]}}
\newcommand{\pu}[1]{{\color{marine}[/pu: #1]}}
\newcommand{\pabe}[1]{{\color{forest}[/pabe: #1]}}
\newcommand{\alternative}[1]{{\color{brown}[/alternative: #1]}}
% \renewcommand{\todo}[1]{}
 \renewcommand{\jsrn}[1]{}
 \renewcommand{\pu}[1]{}
 \renewcommand{\pabe}[1]{}
 \renewcommand{\alternative}[1]{}


%%% Algebra definitions

\newcommand{\param}{\mu}
\newcommand{\mo}{{-1}}
\newcommand{\K}{\F}
\newcommand{\C}{\ensuremath{\cal C}}
\newcommand{\F}{\mathbb{F}}
\newcommand{\R}{{\cal{R}}}
\newcommand{\Q}{{\cal{Q}}}
\newcommand{\N}{\mathbb{N}}
\newcommand{\Z}{\mathbb{Z}}
\newcommand{\ZZ}{\mathbb{Z}}
\newcommand{\Qab}{\Q^{\mathrm{ab}}}
\newcommand{\M}{{\cal{M}}}
\newcommand{\MQ}{\Q^{n\times n}} %\mathrm{M}_n(\Q)}
\newcommand{\GLQ}{\mathrm{GL}_n(\Q)}
\newcommand{\OD}[1]{{\Delta(#1)}} % Orthogonality defect
\newcommand{\m}{\ensuremath{\ve{m}}}
\renewcommand{\u}{\ensuremath{\ve{u}}}
\renewcommand{\v}{\ensuremath{\ve{v}}}
\newcommand{\w}{\ve{w}}
\renewcommand{\vec}[1]{\ensuremath{\ve{#1}}}
\newcommand{\LPc}{\ensuremath{\mathrm{LP}}}
\newcommand{\LP}[2][\null]{\LPc_{#1}(#2)}
\newcommand{\LT}[1]{\ensuremath{\mathrm{LT}(#1)}}
\newcommand{\LC}[1]{\ensuremath{\mathrm{LC}(#1)}}
\newcommand{\blanksymb}{\ensuremath{\,\text{\textvisiblespace}}\,}
\newcommand{\word}[1]{\textnormal{#1}}
\newcommand\modop{\ \word{mod}\ }         % mod infix operator (less spacing than \mod)
\newcommand\characteristic{\word{char}\ }
\newcommand{\code}[1]{\textup{\textsf{#1}}}
\newcommand{\NN}{\mathbb{N}}
\newcommand{\weight}{\word{wt}}

\newcommand{\T}[1]{[#1]} % alternative subscripting

\newcommand{\Transp}{^\top}
\newcommand\vecVal{\psi}
\newcommand\Dieu{Dieudonn\'e\xspace}
\newcommand\Oapp{O^{\scriptscriptstyle \sim}\!}

\DeclareMathOperator{\rank}{rank}
\DeclareMathOperator{\diag}{diag}
\DeclareMathOperator{\maxdeg}{maxdeg}


% define proof environment as IEEEproof
% IEEEtran will redefine proof at the beginning of document to include a
% stupid warning, so the only way to override this is to redefine it again at
% beginning of document. As this command is run after IEEEtran, this
% at-beginning-of-document redefinition will occur *after* IEEEtran's.
\AtBeginDocument{\def\proof{\IEEEproof}\def\endproof{\endIEEEproof}}


%%%%%%%%%%%%%%%%%%%%%%%%%%%%%%%%%%%%%%%%%%%%%%%%%%%%%%%%%%%%%%%%%%%%%%%%%%%%%%
% Others
%%%%%%%%%%%%%%%%%%%%%%%%%%%%%%%%%%%%%%%%%%%%%%%%%%%%%%%%%%%%%%%%%%%%%%%%%%%%%%

\newcommand{\Fq}{\mathbb{F}_q}
\newcommand{\Fs}{\mathbb{F}_s}
\newcommand{\Fsi}[1]{\mathbb{F}_{s_{#1}}}
\newcommand{\evpolys}{\mathcal{V}_{k,t,h,\eta}}
\newcommand{\evpolysMultipleTwists}{\mathcal{M}_{k,\tVec,\linMapVec,\etaVec}}
\newcommand{\evpolysL}{\mathcal{M}_{\L}}

\newcommand{\Code}{\mathcal{C}}
\newcommand{\TRS}[3]{\Code_{#2}^{#3}(#1)}

\newcommand{\Cstar}{\TRS{\alphaVec,\eta}{k}{*}}
\newcommand{\Cplus}{\TRS{\eta,\ast}{k}{+}}
\newcommand{\Csingle}{\TRS{\alphaVec,t,h,\eta}{k}{}}
\newcommand{\Cmult}{\TRS{\alphaVec,\tVec,\linMapVec,\etaVec}{k}{}}
\newcommand{\CL}[1]{\TRS{\alphaVec,\L}{#1}}
\newcommand{\CgL}[1]{\TRS{\alphaVec,\betaVec,\L}{#1}}
\newcommand{\CLdual}[1]{\TRS{\alphaVec,-\L\rotate\transpose}{#1}}
\newcommand{\CgLdual}[1]{\TRS{\alphaVec,\betaVec,-\L\rotate\transpose}{#1}}

\newcommand{\twisted}{$(k,t,h,\eta)$-twisted }
\newcommand{\twistedMultipleTwists}{$(k,\tVec,\linMapVec,\etaVec)$-twisted }
\newcommand{\twistedC}{$(t,h,\eta)$-twisted }
\newcommand{\twistedCMultipleTwists}{$(\tVec,\linMapVec,\etaVec)$-twisted }
\newcommand{\CGRS}[1]{\mathcal{C}_{#1}^\mathrm{GRS}}
\newcommand{\Van}{\vec V}

\newcommand{\dual}{\bot}

\newcommand{\Iset}{\mathcal{I}}
\newcommand{\Af}{\Iset}
\newcommand{\Zmatrix}{\ve 0}
\newcommand{\closure}[1]{\langle #1 \rangle}

\newcommand{\A}{\ve A}
\newcommand{\G}{\ve G}
\renewcommand{\H}{\ve H}
\newcommand{\D}{\ve D}
\newcommand{\I}{\ve I}
\newcommand{\Ik}{\I_{k \times k}}
\newcommand{\B}{\ve B}
\renewcommand{\C}{\ve C}
\renewcommand{\L}{\ve L}

\newcommand{\x}{\ve x}
\renewcommand{\a}{\ve a}

\newcommand{\gi}[1]{g^{(#1)}}
\newcommand{\gVeci}[1]{{\ve g}^{(#1)}}
\newcommand{\sigmai}[1]{\sigma^{(#1)}}
\newcommand{\fpolyi}[1]{f^{(#1)}}
\newcommand{\Ti}[1]{T^{(#1)}}
\newcommand{\Atilde}{\ve{\tilde{A}}}

\newcommand{\ev}[2]{\mathrm{ev}_{#2}(#1)}

\newcommand{\EtaSet}{T}

\newcommand{\numTwists}{\ell}

\newcommand{\tVec}{{\ve t}}
\newcommand{\hVec}{{\ve h}}
\newcommand{\etaVec}{{\ve \eta}}

\newcommand{\Jset}{\mathcal{J}}
\newcommand{\Kset}{\mathcal{K}}
\newcommand{\xii}[1]{\xi^{(#1)}}
\newcommand{\mybox}{\Box}%\boxed{\phantom{\cdot}}}
\newcommand{\etai}[1]{\eta_{t_{#1}}^{-1}}
\newcommand{\transpose}{^\mathrm{T}}
\newcommand{\upsidedownop}{_\updownarrow}
\newcommand{\leftrightop}{_\leftrightarrow}
\newcommand{\rotate}{_\circ}


\newcommand{\linMap}{\lambda}
\newcommand{\linMapVec}{{\ve \lambda}}
\newcommand{\alphaVec}{{\ve \alpha}}
\newcommand{\betaVec}{{\ve \beta}}
\renewcommand{\b}{\ve b}
\newcommand{\g}{\ve g}
\newcommand{\etaSet}{\mathcal{H}}
\newcommand{\etaSetMDSnonGRS}{\mathcal{H}^\ast}
\newcommand{\AetaM}{\A^{(\etaVec)}}
\newcommand{\AetaScalarM}{\A^{(\eta)}}
\newcommand{\AetaScalarV}{\A^{(\eta)}}

\newcommand{\AtildeEta}{\widetilde{\A}^{(\etaVec)}}
\newcommand{\BetaM}{{\B^{(\etaVec)}}}
\newcommand{\BetaMj}[1]{{\B_{#1}^{(\etaVec)}}}
\newcommand{\AetaV}{A^{(\etaVec)}}
\renewcommand{\characteristic}{\mathrm{char}}

\newcommand{\pij}[1]{p^{(#1)}}
\newcommand{\numEtaBound}{\chi}
\newcommand{\J}{{\ve J}}
\newcommand{\PG}{\mathrm{PG}}

\renewcommand{\c}{\ve c}
\newcommand{\cVec}{\ve c}
\newcommand{\rVec}{\ve r}
\newcommand{\eVec}{\ve e}
\newcommand{\DecodeRSAlg}{\mathrm{DecodeRS}}
\newcommand{\DecodeCsingleAlg}{\mathrm{DecodeSingleTwistedRSCode}}
\newcommand{\wtH}{\mathrm{wt_H}}
\newcommand{\dH}{\mathrm{d_H}}
\newcommand{\tmax}{\tau}

\newcommand{\lcm}{\mathrm{lcm}}
\newcommand{\ord}{\mathrm{ord}}
\newcommand{\nmax}{{n_\mathrm{max}}}

\newcommand{\nonecell}{}%\cellcolor[gray]{0.8}}

\newcommand{\Startw}{$(*)$-Twisted }
\newcommand{\startw}{$(*)$-twisted }
\newcommand{\Plustw}{$(+)$-Twisted }
\newcommand{\plustw}{$(+)$-twisted }

\newcommand\Osoft{O^{\scriptscriptstyle \sim}\!}

\definecolor{myred}{rgb}{0.7,0,0}
\newcommand{\new}[1]{\textcolor{myred}{#1}}
