\documentclass[12pt]{amsart}
%\documentclass[11pt]{amsart}

\usepackage{amssymb,latexsym}
%\usepackage[hmargin = 0.75 in, vmargin = 0.75 in]{geometry}


\usepackage{enumerate}

\makeatletter

\@namedef{subjclassname@2010}{

  \textup{2010} Mathematics Subject Classification}


\makeatother
\newtheorem{thm}{Theorem}[section]
\newtheorem{Con}[thm]{Conjecture}
\newtheorem{Cor}{Corollary}
\newtheorem{Pro}{Proposition}
\newtheorem{cor}[thm]{Corollary}
\newtheorem{lem}[thm]{Lemma}
\newtheorem{pro}[thm]{Proposition}
\theoremstyle{definition}
\newtheorem{defin}{Definition}
\newtheorem{rem}[thm]{Remark}
\newtheorem{exa}[thm]{Example}
\newtheorem*{xrem}{Remark}
\numberwithin{equation}{section}


\newcommand{\X}{\mathbb{X}}
\newcommand{\Y}{\mathbb{Y}}
\newcommand{\ex}{\mathbb{E}}
\newcommand{\re}{\textup{Re}}
\newcommand{\im}{\textup{Im}}
\newcommand{\F}{\mathcal{F}}
\newcommand{\D}{\mathcal{D}}
\newcommand{\mb}{\mathbb}
\newcommand{\mc}{\mathcal}
\newcommand{\mbf}{\mathbf}
\newcommand{\e}{\varepsilon}
\newcommand{\ra}{\rightarrow}
\renewcommand{\bar}{\overline}
\newcommand{\sums}{\sideset{}{^\flat}\sum}

\frenchspacing

\textwidth=15.5cm

\textheight=23cm

\parindent=16pt

\oddsidemargin=0cm

\evensidemargin=0cm

\topmargin=-0.5cm



\begin{document}


\baselineskip=17pt



\title[Large odd order character sums]{Large odd order character sums and improvements of the P\'{o}lya-Vinogradov inequality}



\author{Youness Lamzouri}

\author{Alexander P.  Mangerel}


\address{Department of Mathematics and Statistics,
York University,
4700 Keele Street,
Toronto, ON,
M3J1P3
Canada}

\email{lamzouri@mathstat.yorku.ca}

\address{Department of Mathematics\\ University of Toronto\\
Toronto, Ontario, Canada}
\email{sacha.mangerel@mail.utoronto.ca}


\date{}


\begin{abstract}  
For a primitive Dirichlet character $\chi$ modulo $q$, we define $M(\chi)=\max_{t } |\sum_{n \leq t} \chi(n)|$. In this paper, we study this quantity for characters of a fixed odd order $g\geq 3$. Our main result provides a further improvement of the classical P\'{o}lya-Vinogradov inequality in this case. More specifically, we show that for any such character $\chi$ we have
$$M(\chi)\ll_{\varepsilon} \sqrt{q}(\log q)^{1-\delta_g}(\log\log q)^{-1/4+\varepsilon},$$
where $\delta_g := 1-\frac{g}{\pi}\sin(\pi/g)$. This 
improves upon the works of Granville and Soundararajan and of Goldmakher. Furthermore, assuming the Generalized Riemann hypothesis (GRH) we prove that
$$
M(\chi) \ll \sqrt{q} \left(\log_2 q\right)^{1-\delta_g} \left(\log_3 q\right)^{-\frac{1}{4}}\left(\log_4 q\right)^{O(1)},
$$
where $\log_j$ is the $j$-th iterated logarithm. We also show unconditionally that this bound is best possible (up to a power of $\log_4 q$). One of the key ingredients in the proof of the upper bounds is a new Hal\'asz-type inequality for logarithmic mean values of completely multiplicative functions, which might be of independent interest.
\end{abstract}

\subjclass[2010]{Primary 11L40.}

\thanks{The first author is partially supported by a Discovery Grant from the Natural Sciences and Engineering Research Council of Canada.}


\maketitle
\section{Introduction}
The study of Dirichlet characters and their sums has been a central topic in analytic number theory for a long time. Let $q\geq 2$ and  $\chi$ be a non-principal Dirichlet character modulo $q$. An important quantity associated to $\chi$ is 
$$M(\chi) := \max_{t \leq q} \left|\sum_{n \leq t} \chi(n) \right|.$$
The best-known upper bound for $M(\chi)$, obtained independently by P\'olya and Vinogradov in 1918, reads
\begin{equation}\label{PVORIG}
M(\chi)  \ll \sqrt{q} \log q.
\end{equation}
Though one can establish this inequality using only basic Fourier analysis, improving on it has proved to be a difficult problem, and resisted substantial progress for several decades. Conditionally on the Generalized Riemann Hypothesis (GRH),  Montgomery and Vaughan \cite{MV2} showed in 1977 that
\begin{equation}\label{MVCHARBOUND}
M(\chi)\ll \sqrt{q}\log\log q.
\end{equation}
This bound is best possible in view of an old result of Paley \cite{Pa} that there exists an infinite family of primitive quadratic characters $\chi \bmod q$ such that 
\begin{equation}\label{Paley}
M(\chi) \gg \sqrt{q} \log \log q.
\end{equation} 
Assuming GRH, Granville and Soundararajan \cite{GrSo2} extended Paley's result to characters of a fixed even order $2k\geq 4$.  The assumption of GRH was later removed by Goldmakher and Lamzouri \cite{GL2}, who obtained this result unconditionally, and subsequently Lamzouri \cite{LAM} obtained the optimal implicit constant in \eqref{Paley} for even order characters. 

The situation is quite different for odd order characters. In this case, Granville and Soundararajan \cite{GrSo2} proved the remarkable result that both the P\'{o}lya-Vinogradov and the Montgomery-Vaughan bounds can be improved. More specifically, if $g\geq 3$ is an odd integer, and $\chi$ is a primitive character of order $g$ and conductor $q$ then they showed that
%both conditionally and unconditionally, for certain sets of characters $\chi$. 
%Their argument builds on the above-mentioned work of Montgomery and Vaughan on exponential sums with multiplicative coefficients and incorporates some of their own innovations in the theory of mean values of multiplicative functions. To describe their result, we must recall that a character $\chi$ modulo $q$ is said to have \emph{order} $g$ if $g$ is minimal such that $\chi^g = \chi_0$, where $\chi_0$ is the principal character modulo $q$. For $g \geq 3$ let $\delta_g := 1-\frac{g}{\pi}\sin(\pi/g)$. If we allow $\chi$ to be non-principal and have \emph{odd} order $g \geq 3$, Granville and Soundararajan proved that,
\begin{equation} \label{GSUP}
M(\chi) \ll \sqrt{q} (\log Q)^{1-\frac{\delta_g}{2} + o(1)},
\end{equation}
where $\delta_g := 1-\frac{g}{\pi}\sin(\pi/g)$ and 
\begin{equation}\label{THEQ}
Q := \begin{cases}  q &\text{ unconditionally}, \\  \log q &\text{ on GRH}. \end{cases}
\end{equation}
By refining their method, Goldmakher \cite{GOLD} was able to obtain the improved bound
%better exponent $\delta_g$ in \eqref{GSUP}. Indeed, he showed that 
\begin{equation} \label{GOLDSUP}
M(\chi) \ll \sqrt{q} (\log Q)^{1-\delta_g + o(1)}.
\end{equation}
Our first result gives a further improvement of the P\'olya-Vinogradov inequality for $M(\chi)$ when $\chi$ has odd order $g\geq 3$. Here and throughout, we write $\log_k x = \log(\log_{k-1} x)$ to denote the $k$th iterated logarithm, where $\log_1 x=\log x$.
\begin{thm}\label{MCHIUP1}
Let $g \geq 3$ be a fixed odd integer, and let $\e > 0$ be small. Then, for any primitive Dirichlet character $\chi$ of order $g$ and conductor $q$  we have
$$
M(\chi) \ll_{\e} \sqrt{q} \left(\log q\right)^{1-\delta_g}(\log\log  q)^{-\frac{1}{4}+ \e}.
$$
\end{thm}
\noindent The occurrence of $\e$ in the exponent of $\log\log q$ in the upper bound is a consequence of the possible existence of Siegel zeros. In particular, if Siegel zeros do not exist then the $(\log\log q)^{\e}$ term can be replaced by $(\log _3 q)^{O(1)}.$



Assuming GRH, and using results of Granville and Soundararajan (see Theorem \ref{GrSoGRH} below), Goldmakher \cite{GOLD} also showed that the conditional bound in \eqref{GOLDSUP} is best possible. More precisely, for every $\varepsilon>0$ and odd integer $g\geq 3$, he proved the existence of an infinite family of primitive characters $\chi\bmod q$ of order $g$ such that 
\begin{equation}\label{GSDOWN}
M(\chi) \gg_{\varepsilon} \sqrt{q}(\log\log q)^{1-\delta_g -\varepsilon},
\end{equation}
 conditionally on the GRH. By modifying the argument of Granville and Soundararajan and using ideas of Paley \cite{Pa}, Goldmakher and Lamzouri \cite{GL1} proved this result unconditionally. 

It is natural to ask to what degree of precision we can determine the exact order of magnitude of the maximal values of $M(\chi)$ when $\chi$ has odd order $g\geq 3$; in particular, can we determine the optimal $(\log\log q)^{o(1)}$ contributions in the conditional part of \eqref{GOLDSUP}, and in \eqref{GSDOWN}. We make progress in this direction by showing that this term can be replaced by $(\log_3 q)^{-\frac{1}{4}}(\log _4q)^{O(1)}$ in both  \eqref{GOLDSUP} and \eqref{GSDOWN}. This allows us to conditionally determine the maximal values of $M(\chi)$, up to a power of $\log_4 q.$ %Wherever employed, we always assume that $x$ is sufficiently large so that this quantity is well-defined and positive. \\ 
%Finally, let $B_g(Q) := Q\exp\left\{-\frac{\log Q \log g}{\log_2 Q}\right\}$, for $Q > e$ and $g \geq 3$. \\
%that, on GRH, the upper bound corresponding to \eqref{PREC1} holds with the same amount of precision. By the same token, we are able to improve the unconditional upper bound as well.
\begin{thm}\label{MCHIUP2}
Assume GRH. Let $g \geq 3$ be a fixed odd integer. Then for any primitive Dirichlet character $\chi$ of order $g$ and conductor $q$ we have
\begin{equation}\label{PREC0}
M(\chi) \ll \sqrt{q} \left(\log_2 q\right)^{1-\delta_g}(\log_3 q)^{-\frac{1}{4}}(\log_4q)^{O(1)}.
\end{equation}
\end{thm} 

%We shall also unconditionally prove an omega result for $M(\chi)$ that matches the order of magnitude of  the conditional bound in \eqref{PREC0}, up to a power of $\log_4 q$.
\begin{thm} \label{MCHILOW}
Let $g\geq 3$ be a fixed odd integer. There are arbitrarily large $q$ and primitive Dirichlet characters $\chi$ modulo $q$ of order $g$ such that
\begin{equation} \label{PREC1}
M(\chi)  \gg \sqrt{q} \left(\log_2 q\right)^{1-\delta_g}  \left(\log_3 q\right)^{-\frac{1}{4}}\left(\log_4 q\right)^{O(1)}.
\end{equation}
\end{thm}
%We also note that the method in \cite{GOLD} actually allows one to prove the same upper bound on GRH.
%\begin{thm}[\cite{GOLD}, Section 8]
%If $\chi$ is any character of conductor $q$ and odd order $g \geq 3$, and $\psi$ is any even character modulo $m \leq $ then
%\begin{equation*}
%\mb{D}(\chi,\psi;\log q) \geq \delta_g \log_3 q + O\left(\frac{\log_3 q}{m^2} + \log m\right).
%\end{equation*}
%\end{thm} 
%\noindent Note that on GRH, this indeed matches the order of magnitude of $M(\chi)$ given by Theorem \ref{MCHIUP}, up to a power of $\log_4 q$. 

To obtain Theorem \ref{MCHILOW}, our argument relates $M(\chi)$ to the values of certain associated Dirichlet $L$-functions at $1$, and uses zero-density results and ideas from \cite{LAM} to construct characters $\chi$ for which these values are large. We shall discuss the different ingredients in the proofs of Theorems \ref{MCHIUP1}, \ref{MCHIUP2}  and \ref{MCHILOW} in detail in the next section. 
 
Recent progress on character sums was made possible by Granville and Soundararajan's discovery of a hidden structure among the characters $\chi$ having large $M(\chi)$. In particular, they show that $M(\chi)$ is large only when $\chi$ \emph{pretends} to be a character of small conductor and opposite parity. To define this notion of \emph{pretentiousness}, we need some notation.  Here and throughout we denote by $\mathcal{F}$ the class of completely multiplicative functions $f$ such that $|f(n)|\leq 1$ for all $n$. For $f, g\in \mc{F}$ we define
\begin{equation*}
\mb{D}(f,g;y) := \left(\sum_{p \leq y} \frac{1-\text{Re}(f(p)\bar{g(p)})}{p}\right)^{\frac{1}{2}},
\end{equation*}
which turns out to be a pseudo-metric on $\mc{F}$ (see \cite{GrSo2}). 
We say that $f$ \emph{pretends} to be $g$ (up to $y$) if there is a constant $0\leq \delta<1$ such that $\mb{D}(f,g;y)^2\leq \delta \log\log  y$. 

One of the key ingredients in the proof of \eqref{GSUP} is the following bound for logarithmic mean values of functions $f\in \mc{F}$ in terms of $\mb{D}(f,1;x)$ (see Lemma 4.3 of \cite{GrSo2}) 
\begin{equation}\label{BOUNDLOGDISTANCE}
\sum_{n\leq x} \frac{f(n)}{n}\ll (\log x) \exp\left(-\frac{1}{2}\mb{D}(f,1;x)^2\right).
\end{equation}
Note that the factor $1/2$ inside the exponential on the right hand side of \eqref{BOUNDLOGDISTANCE} is responsible for the weaker exponent $\delta_g/2$ in \eqref{GSUP}.

Goldmakher \cite{GOLD} realized that one can obtain the optimal exponent $\delta_g$ in \eqref{GOLDSUP} by replacing \eqref{BOUNDLOGDISTANCE} by a Hal\'{a}sz-type inequality for logarithmic mean values of multiplicative functions due to Montgomery and Vaughan \cite{MV}. Combining Theorem 2 of \cite{MV} with refinements of Tenenbaum (see Chapter III.4 of \cite{Te}) he deduced that (see Theorem 2.4 in \cite{GOLD})
\begin{equation} \label{HMTFIRST}
 \sum_{n \leq x} \frac{f(n)}{n} \ll (\log x)\exp\big(-\mc{M}(f;x,T)\big) + \frac{1}{\sqrt{T}},
\end{equation}
for all $f\in \mc{F}$ and $T\geq 1$, where 
$$
\mc{M}(f;x,T) := \min_{|t| \leq T} \mb{D}(f,n^{it};x)^2.
$$

Motivated by our investigation of character sums, we are interested in characterizing the functions $f\in \mc{F}$ that have a \emph{large} 
logarithmic mean, in the sense that
\begin{equation}\label{LARGELOG}
\sum_{n \leq x} \frac{f(n)}{n}\gg (\log x)^{\alpha},
\end{equation}
for some $0<\alpha\leq 1$.
Taking $T=1$ in \eqref{HMTFIRST} shows that this happens only when $f$ pretends to be $n^{it}$ for some $|t|\leq 1$. However, observe that
 $$\sum_{n \leq x} \frac{n^{it}}{n} = \frac{x^{it}-2^{it}}{it} +O(1)\asymp \min\left(\frac{1}{|t|}, \log x\right),$$
and hence $f(n)=n^{it}$ satisfies \eqref{LARGELOG} only when $|t|\ll (\log x)^{-\alpha}$. By refining the ideas of Montgomery and Vaughan \cite{MV} and Tenenbaum \cite{Te}, we prove the following result, which shows that this is essentially the only case.  
 %Theorem \ref{MCHIUP} constitutes an improvement of Theorem 3 from \cite{GOLD}. Our proof relies on an extension of an inequality of Hal\'{a}sz-type for logarithmic mean values of 1-bounded, completely multiplicative functions, which we state below. In \cite{GOLD}, a lower bound is given for $\mc{M}(\chi \bar{\psi}; \log q, T)$ with $T \geq 1$, as is the usual paradigm in applications of inequalities like \eqref{HMTFIRST}. However, it turns out to be advantageous in this case to be able to choose $T < 1$ (see Proposition \ref{AD} in the Appendix to this paper, where it is shown that this advantage is actually necessary to prove Theorem \ref{MCHIUP} with the method of Granville and Soundararajan).\\
%We define $\mathcal{S}(y)$ to be the set of positive integers whose primes factors are all less than or equal to  $y$ (these are also called \emph{$y$-friable} or \emph{$y$-smooth} integers). We prove the following.
\begin{thm}\label{LogarithmicMean}
Let $f\in \mathcal{F}$ and $x\geq 2$. Then, for any real number $0< T \leq 1$ we have
$$\sum_{n\leq x}\frac{f(n)}{n}\ll  (\log x) \exp\big(-\mc{M}(f;x, T)\big)+\frac{1}{T},
$$
where the implicit constant is absolute.
\end{thm}
Taking $T=c(\log x)^{-\alpha}$ in this result (where $c>0$ is a suitably small constant), we deduce that if $f\in \mc{F}$ satisfies \eqref{LARGELOG}, then $f$ pretends to be $n^{it}$ for some $|t|\ll (\log x)^{-\alpha}$. Theorem \ref{LogarithmicMean} will be one of the key ingredients in obtaining our superior bounds for $M(\chi)$ in Theorems \ref{MCHIUP1} and \ref{MCHIUP2}.


%%%%%%%%%%%%%%%%%%%%%%%%%%%%%%%%%%%%%%%%%%%%%%%%%%%%
\section{Detailed statement of results} 
To explain the key ideas in the proofs of Theorems \ref{MCHIUP1}, \ref{MCHIUP2} and \ref{MCHILOW}, we shall first sketch the argument of Granville and Soundararajan \cite{GrSo2}. Their starting point is P\'olya's Fourier expansion (see section 9.4 of \cite{MVbook}) for the character sum
$\sum_{n\leq t}\chi(n)$, which reads 
\begin{equation}\label{Polya}
\sum_{n\leq t}\chi(n)
=\frac{\tau(\chi)}{2\pi i}
	\sum_{1\leq |n|\leq N} \frac{\overline{\chi}(n)}{n}
		\left(1-e\left(-\frac{nt}{q}\right)\right)
		+O\left(1+\frac{q\log q}{N}\right),
\end{equation}
where $\chi$ is a primitive character modulo $q$,  $e(x) := e^{2\pi i x}$ and $\tau(\chi)$ is the Gauss sum 
$$
 \tau(\chi) := \sum_{n=1}^q \chi(n) e\Big(\frac{n}{q}\Big).
$$
Note that $|\tau(\chi)| = \sqrt{q}$ whenever $\chi$ is primitive.

Thus, in order to estimate $M(\chi)$, one needs to understand the size of the exponential sum
\begin{equation}\label{EXPCHARA}
\sum_{1\leq |n|\leq q} \frac{\chi(n)}{n}e(n\theta),
\end{equation}
for $\theta\in [0, 1]$. Montgomery and Vaughan \cite{MV2} showed that this sum is small if $\theta$ belongs to a \emph{minor arc}, i.e., $\theta$ can only be well-approximated by rationals with large denominators (compared to $q$). This leaves the more difficult case of $\theta$ lying in a \emph{major arc}. In this case, $\theta$ can be well-approximated by some rational $b/r$ with suitably small $r$ (compared to $q$). Granville and Soundararajan showed that in this case there is some large $N$ (depending on $\theta$, $b$, $r$ and $q$) such that we can approximate the sum 
\eqref{EXPCHARA} by 
%Let us sketch the upper bound method of Granville and Soundararajan (see \cite{GrSo2}) to show how the theory of mean values of multiplicative functions plays a role. The starting point is the observation, due to P\'{o}lya, that the map $S_{\chi} : [0,1] \ra \mb{C}$, given by
%\begin{equation*}
%S_{\chi}(t) := \sum_{n \leq tq} \chi(n), 
%\end{equation*}
%is $1$-periodic and of bounded variation. Suppose $t$ is such that $|S_{\chi}(t)| = M(\chi)$. One can then express $M(\chi)$ by its Fourier series the  dominant contribution of which (see \eqref{Polya} below) is
%\begin{equation*}
%\sqrt{q}\left|\sum_{1 \leq |n| \leq q} \frac{\chi(n)}{n} e(-nt)\right|.
%\end{equation*}
%The largest possible values of this sum occur when $t$ can be approximated by a rational $b/r$ with suitably small $r$ (compared to $q$). In this case, they show that there is some large $N$ (depending on $t$) such that we can approximate the above sum by
\begin{align*}
&\sum_{1\leq |n| \leq N} \frac{\chi(n)}{n}e(bn/r) = \sum_{a\bmod r} e(ab/r) \sum_{1\leq |n| \leq N \atop n \equiv a \bmod r} \frac{\chi(n)}{n} \\
&=\frac{1}{\phi(r)} \sum_{\psi \bmod r} \left(\sum_{a\bmod r}\bar{\psi}(a)e(ab/r)\right) \sum_{1\leq |n| \leq N} \frac{\chi(n)\bar{\psi}(n)}{n}.
\end{align*}
The bracketed term, a Gauss sum, is well understood; in particular it has norm $\leq \sqrt{r^{\ast}}$, where $r^{\ast}$ is the conductor of $\psi$ (see e.g., Theorem 9.7 of \cite{MVbook}).   
Thus, what remains to be determined in order to bound $M(\chi)$, is an upper bound for the sum
\begin{equation}\label{LOGMEANPN}
\sum_{1\leq |n| \leq N} \frac{\chi(n)\bar{\psi}(n)}{n}
\end{equation}
for each character $\psi$ modulo $r$. Furthermore, observe that if $\chi$ and $\psi$ have the same parity then this sum is exactly $0$; hence, we only need to consider the case when $\chi$ and $\psi$ have opposite parities.

Granville and Soundararajan's breakthrough stems from their discovery of a ``repulsion'' phenomenon between characters $\chi$ of odd order (which are necessarily of even parity), and characters $\psi$ of odd parity and small conductor. A consequence of this phenomenon is that the sum \eqref{LOGMEANPN} is small, allowing them to improve the P\'olya-Vinogradov inequality in this case. 
More specifically, they show that if $\chi$ is a primitive character of odd order $g\geq 3$ and $\psi$ is an odd primitive character of conductor $m\leq (\log y)^A$ then  
 \begin{equation}\label{LOWERBOUDISTANCE}
 \mb{D}(\chi,\psi;y)^2\geq (\delta_g+o(1))\log\log y
 \end{equation}
(see Lemma 3.2 of \cite{GrSo2}). Inserting this bound in \eqref{BOUNDLOGDISTANCE} allows them to bound the sum \eqref{LOGMEANPN}, from which they deduce the unconditional case of \eqref{GSUP}. The proof of the conditional part of \eqref{GSUP} (when $Q=\log q$) proceeds along the same lines, but uses an additional ingredient, namely the following approximation for the sum \eqref{EXPCHARA} (see Proposition 2.3 and Lemma 5.2 of \cite{GrSo2}) conditional on GRH:
\begin{equation}\label{APPROXFRIABLE}
\sum_{n\leq q}\frac{\chi(n)}{n} e(n\theta)= \sum_{\substack{n\leq q \\ n \in \mc{S}(y)}}\frac{\chi(n)}{n} e(n\theta)+O\left(y^{-1/6}(\log q)^2\right).
\end{equation}
Here, $\mc{S}(y)$ is the set of $y$-\emph{friable} integers (also known as $y$-\emph{smooth} integers), i.e., the set of positive integers $n$ whose prime factors are all less than or equal to $y$.

In \cite{GOLD}, Goldmakher showed that the bound \eqref{LOWERBOUDISTANCE} is best possible. Furthermore, in order to obtain the exponent $\delta_g$ in \eqref{GOLDSUP}, he used the inequality \eqref{HMTFIRST} to bound the sum \eqref{LOGMEANPN} in terms of $\mc{M}(\chi\bar{\psi};y, T)$. However, to ensure that this argument works, one needs to show that the lower bound \eqref{LOWERBOUDISTANCE} still persists if we twist $\chi\bar{\psi}$ by Archimedean characters $n^{it}$ for $|t|\leq T$. By a careful analysis of 
$\mc{M}(\chi\bar{\psi};y,T)$, Goldmakher (see Theorem 2.10 of \cite{GOLD}) proved that (under the same assumptions as \eqref{LOWERBOUDISTANCE})
\begin{equation}\label{LBDGOLD}
\mc{M}(\chi\bar{\psi};y,(\log y)^2)\geq (\delta_g+o(1))\log\log y.
\end{equation}
Thus, by combining this bound with \eqref{HMTFIRST} and following closely the argument in \cite{GrSo2}, he was able to obtain \eqref{GOLDSUP}. 

%Granville and Soundararajan were able to deduce a useful non-trivial bound for this sum that led to \eqref{GSUP} using ideas in multiplicative function theory. They also show that the lower bound \eqref{GSDOWN} can also be determined by an analysis of logarithmic means of multiplicative functions. \\
%It turns out that, based on the seminal work of Hal\'{a}sz and subsequent refinements due to Montgomery and later Tenenbaum ( for references), there is an appropriate measure for the size of these logarithmic means, defined as follow. For $T \geq 0$, $X \geq 1$, and $f$ a 1-bounded arithmetic function we thus define
%Thus, if $\chi(n)\bar{\psi}(n)$ has a complex argument that correlates with that of some Archimedean character $n \mapsto n^{it}$ with $|t| \leq T$ on primes, i.e., such that $\text{Re}(\chi(p)\bar{\psi}(p)p^{-it})$ is close to 1 on average, then the logarithmic mean value is large; otherwise, the mean is small.  \\
%More recently, Goldmakher \cite{GOLD} improved the exponent $1-\frac{\delta_g}{2}$ in \eqref{GSUP} to $1-\delta_g$ to match the exponent on $\log_2 q$ in \eqref{GSDOWN}. He achieves this by refining the original argument of Granville and Sound just described, and, crucially, by carefully and optimally bounding $\mc{M}(\chi\bar{\psi};\log q,\log^2 q)$ from below. \\

%that, on GRH, 
%\begin{equation*}
%M(\chi) \ll \sqrt{q} \log^{1-\delta_g} q \log_3^{-\frac{1}{4}} q \log_4^{O(1)} q
%\end{equation*}
%for all non-principal characters $\chi$ of odd degree, and unconditionally there exists an infinite family of characters $\chi$ modulo $q$ such that
%\begin{equation*}
%M(\chi) \gg \sqrt{q} \log^{1-\delta_g} q \log_3^{-\frac{1}{4}} q \log_4^{-O(1)} q.
%\end{equation*}
%We also derive a sharper unconditional upper bound by the same proof. \\
%Let us broadly give an account of the improvements and their sources in each of Theorems \ref{MCHIUP} and \ref{MCHILOW}. \\
In order to improve these results and establish Theorems \ref{MCHIUP1} and \ref{MCHIUP2}, the first step is to obtain more precise estimates for the quantity $\mc{M}(\chi\bar{\psi};y, T)$. We discover that there is a substantial difference between the sizes of $\mc{M}(\chi\bar{\psi}; y, T_1)$ and $\mc{M}(\chi\bar{\psi}; y, T_2)$ if $T_1$ is small and $T_2$ is large (a result that may be surprising in view of \eqref{LOWERBOUDISTANCE} and \eqref{LBDGOLD}). In fact, we prove that there is a large secondary term of size $(\log_2y)/k^2$ (where $k$ is the order of $\psi$) that appears in the estimate of $\mc{M}(\chi\bar{\psi}; y, T)$ when $T\leq (\log y)^{-c}$ (for some constant $c>0$), but disappears when $T\geq 1$. 
\begin{pro}\label{MINDIST}
Let $g\geq 3$ be a fixed odd integer, $\alpha\in (0, 1)$, and $\e>0$ be small. Let $\chi$ be a primitive character of order $g$ and conductor $q$. Let $\psi$ be an odd primitive character modulo $m$, with $m \leq (\log y)^{4\alpha/7}$. Put $k^{\ast} := k/(k,g)$. Then we have\begin{equation} \label{LowerBoundDistance2}
\mc{M}(\chi\bar{\psi}; y, (\log y)^{-\alpha}) \geq \left(\delta_g + \frac{\alpha \pi^2(1-\delta_g)}{4(gk^{\ast})^2}\right) \log_2 y - \beta \e \log m+O_{\alpha}\left(\log_2 m\right),
\end{equation}
where $\beta = 1$ if $m$ is an exceptional modulus and $\beta = 0$ otherwise.
\end{pro}
\begin{pro} \label{AD}
Assume GRH. Let $g\geq 3$ be a fixed odd integer. Let $N$ be large, and $y\leq (\log N)/10$. Let $\psi$ be an odd primitive character of conductor $m$ such that  $\exp\left(2\sqrt{\log_3 y}\right) \leq m\leq \exp\left(\sqrt{\log y}\right)$. Then, there exist at least $\sqrt{N}$  primitive characters $\chi$ of order $g$ and conductor $q\leq N$, such that for all $T\geq 1$ we have
\begin{equation*}
\mc{M}(\chi\bar{\psi}; y, T) \leq \delta_g \log_2 y + O\left(\log_2 m\right).
\end{equation*}
\end{pro}

The secondary term of size $\asymp (\log_2y)/k^2$ in the right hand side of \eqref{LowerBoundDistance2} is responsible for the additional saving of $(\log_2 Q)^{-1/4}$ (where $Q$ is defined in \eqref{THEQ}) in Theorems \ref{MCHIUP1} and \ref{MCHIUP2}; clearly, it does not appear in Proposition \ref{AD}, even in the range $m \ll (\log_2 y)^{\frac{1}{2}-\e}$ in which this secondary term is large.  Note that when $m$ is an exceptional modulus (see the precise definition in \eqref{EXCEPTIONALMODULUS} below), there is an additional term that appears when estimating $\mc{M}(\chi\bar{\psi}; y, (\log y)^{-\alpha})$  that has size $\log L(1, \chi_m)$, where $\chi_m$ is the exceptional character modulo $m$. In this case, the extra term $\varepsilon\log m$ on the right hand side of \eqref{LowerBoundDistance2} is due to Siegel's bound $L(1,\chi_m)\gg_{\e} m^{-\varepsilon}$. 



 
%However, recall that in Goldmakher's improvement of the Granville-Soundararajan bound for $M(\chi)$, the key ingredient is to use \eqref{HMTFIRST} instead of \eqref{LOWERBOUDISTANCE}, when bounding logarithmic mean values of Dirichlet characters. Hence, in order to use Goldmakher's refinement, we need to prove an analogue of Proposition \ref{UPPER} for $\mc{M}\big(\chi\bar{\psi};y,(\log y)^2\big)$. Unfortunately, it turns out that when twisting by $n^{it}$ for $|t|\geq 1$, the quantity $\mc{M}(\chi\bar{\psi};y,T)$ (for $T\geq 1$) becomes substantially smaller than $\mb{D}(\chi, \psi, y)^2$, and we lose the additional saving. 

To complete the proofs of Theorems \ref{MCHIUP1} and \ref{MCHIUP2}, we shall use our Theorem \ref{LogarithmicMean} to bound the sum \eqref{LOGMEANPN}, where we might choose $T=(\log y)^{-\alpha}$ to take advantage of Proposition \ref{MINDIST}. Note that in view of Proposition \ref{AD}, one loses the additional saving of $(\log_2 Q)^{-1/4}$ in Theorems \ref{MCHIUP1} and \ref{MCHIUP2} if one simply uses \eqref{HMTFIRST} with $T=(\log y)^2$, as in \cite{GOLD}. By using Theorem \ref{LogarithmicMean} and following the ideas in \cite{GrSo2}, we prove the following result, which is a refinement of Theorem 2.9 in \cite{GOLD}. This together with Proposition \ref{MINDIST} implies both Theorems \ref{MCHIUP1} and  \ref{MCHIUP2}.

\begin{thm}\label{COND}
Let $\chi$ be a primitive character modulo $q$, and let $Q$ be as in \eqref{THEQ}.  Of all primitive characters with conductor below $(\log Q)^{4/11}$, let $\xi$ modulo $m$ be that character for which $\mc{M}\left(\chi\bar{\xi};Q, (\log Q)^{-7/11}\right)$ is a minimum. Then we have
\begin{equation*}
M(\chi) \ll \Big(1-\chi(-1)\xi(-1)\Big)\frac{\sqrt{qm}}{\phi(m)} (\log Q) \exp\left(-\mc{M}\left(\chi\bar{\xi}; Q,(\log Q)^{-\frac{7}{11}}\right)\right) + \sqrt{q}\left(\log Q\right)^{\frac{9}{11} + o(1)}.
\end{equation*}
\end{thm}
Note that $\delta_g$ is decreasing as a function of $g$, so $1-\delta_g \geq 1-\delta_3\approx 0.827>9/11$ for all $g \geq 3$. Therefore, when $\chi$ is a primitive character of odd order $g\geq 3$ and conductor $q$, we get the better bound 
$ M(\chi)\ll \sqrt{q}\left(\log Q\right)^{\frac{9}{11} + o(1)}$, unless $\xi$ is odd and $\mc{M}\left(\chi\bar{\xi}; Q,(\log Q)^{-\frac{7}{11}}\right)$ is small.

%The upshot of Proposition \ref{AD} is that if $\psi$ is a primitive character with conductor $m$ that minimizes $\mc{M}(\chi\bar{\psi}; \log q,T)$ then $m$ may be of size $(\log_3 q)^{o(1)}$ and we do not deduce the additional savings by $(\log_3 q)^{-\frac{1}{4}}$ claimed in Theorem \ref{MCHIUP} on GRH by applying Proposition \ref{COND}. In particular, our Theorem \ref{LogarithmicMean} is necessary to deduce this savings.

%When $\phi(m) \geq k > \sqrt{\log_2Q}$ the lower bound in Proposition \ref{MINDIST} is $\delta_g \log_2 Q + O\left(\log_2 m\right)$. If the minimizing character has conductor $m \gg \sqrt{\log_2 Q}$ then
%similar upon specializing to the case of $m$ prime with $(g,m-1) = 1$, as mentioned above. 
%Indeed, note that $\mc{M}(\chi\bar{\psi};Q,(\log Q)^{-3/4})$ is decreasing with $k^{\ast}$, and is thus minimal for $\psi$ with conductor $m \leq \sqrt{\log Q}$ when $k^{\ast} = \phi(m) = m-1$, and $m$ is as large as possible in this range. 
We next discuss the ideas that go into the proof of Theorem \ref{MCHILOW}. 
%There are two main ingredients in its proof. 
To obtain \eqref{GSDOWN} under GRH, Goldmakher \cite{GOLD} used the following result from \cite{GrSo2}, which relates $M(\chi)$ to the distance between $\chi$ and any primitive character $\psi$ with small conductor and parity opposite to that of $\chi$. 
\begin{thm}[Theorem 2.5 of \cite{GrSo2}]\label{GrSoGRH}
Assume GRH. Let $\chi\bmod q$ and $\psi\bmod m$ be primitive characters such that $\chi(-1)=-\psi(-1)$. Then we have
$$
M(\chi) + \frac{\sqrt{qm}}{\phi(m)}\log_3q \gg \frac{\sqrt{qm}}{\phi(m)} (\log_2 q) \exp\left(-\mb{D}(\chi,\psi;\log q)^2\right).
$$
\end{thm}
\noindent Thus, it only remains to produce characters $\chi$ and $\psi$ which satisfy the assumptions of Theorem \ref{GrSoGRH}, and for which the lower bound \eqref{LOWERBOUDISTANCE} is attained when $y=\log q$.  Using the Eisenstein reciprocity law, Goldmakher (see Proposition 9.3 of \cite{GOLD}) proved that for any $\varepsilon>0$, there exists an odd primitive character $\psi$ modulo $m\ll_{\varepsilon}1$, and an infinite family of primitive characters $\chi\bmod q$ of order $g$ such that 
\begin{equation}\label{GOLDreciprocity}
\mb{D}(\chi,\psi;\log q)^2\leq (\delta_g+\varepsilon)\log_3q.
\end{equation}
To remove the assumption of GRH, Goldmakher and Lamzouri \cite{GL1} (see Theorem 1 of \cite{GL1}) used ideas of Paley \cite{Pa} to obtain a weaker version of Theorem \ref{GrSoGRH} unconditionally. Namely, they showed that if $\chi$ is odd and $\psi$ is even then
$$M(\chi) + \sqrt{q}\gg \frac{\sqrt{qm}}{\phi(m)}\left(\frac{\log_2 q}{\log_3 q}\right) \exp\left(-\mb{D}(\chi,\psi;\log q)^2\right).$$
Although this bound is enough to obtain \eqref{GSDOWN} unconditionally in view of \eqref{GOLDreciprocity}, it is not sufficient to yield the precise estimate in Theorem \ref{MCHILOW}, due to the loss of a factor of $\log_3 q$ over Theorem \ref{GrSoGRH}. \\
Using a completely different method, based on zero density estimates for Dirichlet $L$-functions, we recover the original bound of Granville and Soundararajan unconditionally for all characters $\chi$ modulo $q$ with $q\leq N$, except for a small \emph{exceptional} set of cardinality $\ll N^{\varepsilon}$. Our argument also gives a simple proof of Theorem \ref{GrSoGRH}, which exploits the natural properties of the values of Dirichlet $L$-functions at $1$, and avoids the difficult study of exponential sums with multiplicative functions (see Section 6 of \cite{GrSo2}). Note that the statement of Theorem \ref{GrSoGRH} trivially holds when $m > \log q$, since $\mb{D}(\chi,\psi;\log q)^2\ll \log_3 q$. We thus only need to consider the case $m\leq \log q$.
\begin{thm} \label{GL}
Let $\e > 0$ and let $N$ be large. Let $m \leq \log N$ be a positive integer and let $\psi$ be a primitive character modulo $m$.  Then, for all but at most $N^{\varepsilon}$ primitive characters $\chi$ modulo $q$ with   $q\leq N$ and such that $\chi(-1)=-\psi(-1)$ we have 
\begin{equation}\label{LowerBoundMchi}
M(\chi) + \sqrt{q} \gg_{\varepsilon} \frac{\sqrt{qm}}{\phi(m)} (\log_2 q) \exp\left(-\mb{D}(\chi,\psi;\log q)^2\right).
\end{equation}
Moreover, if we assume GRH, then \eqref{LowerBoundMchi} is valid for all primitive characters $\chi$ modulo $q$ with $q\leq N$, and the implicit constant in \eqref{LowerBoundMchi} is absolute.
\end{thm}
To complete the proof of Theorem \ref{MCHILOW}, we thus need to refine the estimate \eqref{GOLDreciprocity}, and this can be achieved using the same ideas as in the proof of Proposition \ref{MINDIST}. However, Goldmakher's proof of \eqref{GOLDreciprocity} only produces an infinite sequence of primitive characters $\chi$, and this is not enough to use in Theorem \ref{GL}, due to the possible existence of an exceptional set of characters for which \eqref{LowerBoundMchi} does not hold. To overcome this difficulty, we use the results of \cite{LAM} to prove the existence of \emph{many} primitive characters $\chi$ of order $g$ and conductor $q\leq N$ such that when $y \ll \log N$, $\mb{D}(\chi,\psi;y)$ is maximal.

\begin{pro} \label{UPPER}
Let $g\geq 3$ be a fixed odd integer. Let $N$ be large and $y\leq (\log N)/10$ be a real number. Let $m$ be a non-exceptional modulus such that $m\leq (\log y)^{4/7}$, and let $\psi$ be an odd primitive character of conductor $m$. Let $k$ be the order of $\psi$ and  put $k^{\ast}=k/(g,k)$.  Then, there exist at least $\sqrt{N}$ primitive characters $\chi$ of order $g$ and conductor $q\leq N$ such that
\begin{equation} \label{OPT}
\mb{D}(\chi,\psi;y)^2 =  \left(1-(1-\delta_g)\frac{\pi/gk^{\ast}}{\tan(\pi/gk^{\ast})}\right) \log_2y + O\left(\log_2 m\right).
\end{equation}
\end{pro}
%\begin{thm} \label{GL}
%For any $\psi \ (\text{mod } m)$ be a primitive odd character. Then
%\begin{equation*}
%M(\chi) + \sqrt{q} \gg \frac{\sqrt{qm} \log_2 q}{\phi(m) \log_3 q} \exp\left(-\mb{D}^2(\chi,\psi;\log q)\right).
%\end{equation*}
%\end{thm}
%While Theorem 1 of \cite{GL} proves an estimate of this type for any $q$ and any $\chi$ modulo $q$, their estimate is weaker by a factor of $\log_3 q$. In any case, the statement of Proposition \ref{GL} here suffices for our proof of Theorem \ref{MCHILOW}, and we therefore do not seek to prove our sharper inequality in the more general case.\\
%\noindent The second main ingredient that we use to prove Theorem \ref{MCHILOW} is the following more precise version of a result of Goldmakher (see Proposition 9.2 in \cite{GOLD}).
%\begin{pro} \label{UPPER}
%Let $g\geq 3$ be a fixed odd integer, and $\varepsilon>0$ be small. Let $\psi \pmod m$ be an odd primitive character of even order $k$, and $y$ be such that $m\leq \sqrt{\log y}$. Put $k^{\ast}=k/(g,k)$.  Then, for any primitive character $\chi \pmod q$ of order $g$ we have
%\begin{equation}\label{LowerBoundDistance}
%\mb{D}^2(\chi,\psi;y)\geq \left(\delta_g + \frac{1-\delta_g}{3(gk^{\ast})^2}\right) \log_2y-\beta \varepsilon \log m+ O_{\varepsilon}\left(\frac{\log_2 y}{(k^{\ast})^4}+\log_2 m\right),
%\end{equation}
%where $\beta=0$ if $m$ is a non-exceptional modulus, and $\beta=1$ if $m$ is exceptional.

%Moreover, if $m$ is non-exceptional and $y\leq (\log Q)/10$, then there exist at least  $\sqrt{Q}$ primitive characters $\chi$ of order $g$ and conductor $q\leq Q$ such that
%\begin{equation} \label{OPT}
%\mb{D}^2(\chi,\psi;y) = \left(\delta_g + \frac{1-\delta_g}{3(gk^{\ast})^2}\right) \log_2y + O\left(\frac{\log_2 y}{(k^{\ast})^4}+\log_2 m\right).
%\end{equation}
%\end{pro}
%\noindent Note that if we assume GRH and specialize to the case of $m$ prime with $(g,m-1) = 1$, as is the case in Proposition \ref{UPPER} then $k^{\ast} = k$ (since $k|m-1$), and we can choose $\psi$ to be the generator of the dual group modulo $m$, so that $k =m-1$ precisely. This gives an error term of the same quality in Proposition \ref{MINDIST} as in Proposition \ref{UPPER}. \\
%The plan of the paper is as follows. In the next section we prove Proposition \ref{GL}, and in Section 3 we prove Proposition \ref{UPPER}. In the Section 4 we prove Proposition \ref{MINDIST}, and in Section 5 we prove Theorem \ref{LogarithmicMean}. Finally, in Section 6, we prove Proposition \ref{COND}.

%%%%%%%%%%%%%%%%%%%%%%%%%%%%%%%%%%%%%%%%%%%%%

\section{A lower bound for $M(\chi)$: Proof of Theorem \ref{GL}}


The main ingredient in the proof of  Theorem \ref{GrSoGRH} of \cite{GrSo2} is the approximation \eqref{APPROXFRIABLE}, which is valid under the assumption of GRH. 
To avoid this assumption, we shall instead relate $M(\chi)$ to the values of certain Dirichlet $L$-functions at $s=1$, and then use the classical zero-density estimates for these $L$-functions. \begin{pro}\label{CharSumL1}
Let $q$ be large and $m \leq q/(\log q)^2$. Let $\chi\bmod q$ and $\psi\bmod m$ be primitive characters such that $\psi(-1)=-\chi(-1)$. Then we have
$$ M(\chi)+\sqrt{q} \gg \frac{\sqrt{qm}}{\phi(m)} \cdot \left|L\left(1, \chi\overline{\psi}\right)\right|.$$
\end{pro}
We first need the following lemma.
\begin{lem} \label{PVAPP}
Let $q$ be large and $m \leq q/(\log q)^2$. Let $\chi$ be a character modulo $q$ and $\psi$ be a character modulo $m$ such that $\chi \bar{\psi}$ is non-principal. Then
\begin{equation*}
L(1,\chi\bar{\psi}) = \sum_{n \leq q} \frac{\chi(n)\bar{\psi}(n)}{n} + O(1).
\end{equation*}
\end{lem}
\begin{proof}
Note that $\chi\bar{\psi}$ is a non-principal character of conductor at most $qm\leq (q/\log q)^2$. Therefore, using partial summation and the P\'{o}lya-Vinogradov inequality we obtain
\begin{align*}
\sum_{q < n \leq N} \frac{\chi(n)\bar{\psi}(n)}{n} &= \sum_{q<n\leq N} \frac{1}{n(n+1)}\left(\sum_{q < k \leq n} \chi\bar{\psi}(k)\right) +O(1) \ll 1,
\end{align*}
and the claim follows.
\end{proof}
\begin{proof}[Proof of Proposition \ref{CharSumL1}]
Taking $N=q$ in \eqref{Polya} gives
$$M(\chi)+\log q\gg \sqrt{q} \cdot \max_{\theta}\left|\sum_{1\leq |n|\leq q} \frac{\chi(n)}{n}
		\left(1-e\left(n \theta\right)\right)\right|.$$
Moreover,  we observe that
\begin{align*}
\sum_{b\bmod m} \psi(b) \sum_{1\leq |n|\leq q} \frac{\chi(n)}{n}\left(1-e\left(\frac{nb}{m}\right)\right)
&=-\sum_{1\leq |n|\leq q}\frac{\chi(n)}{n}\sum_{b\bmod m} \psi(b)e\left(\frac{nb}{m}\right)\\
&= -\tau(\psi)\sum_{1\leq |n|\leq q}\frac{\chi(n)\bar{\psi}(n)}{n},\\
\end{align*}
which follows from the identity
$$ \sum_{b\bmod m} \psi(b)e\left(\frac{nb}{m}\right)= \bar{\psi}(n)\tau(\psi).$$
Since $\chi$ and $\psi$ are primitive and $m\leq q/(\log q)^2$ then $\chi\bar{\psi}$ is non-principal. Therefore, by Lemma \ref{PVAPP} together with the fact that $\chi\bar{\psi}(-1)=-1$ we deduce that
$$ \sum_{1\leq |n|\leq q}\frac{\chi(n)\bar{\psi}(n)}{n}= 2 \sum_{1\leq n\leq q}\frac{\chi(n)\bar{\psi}(n)}{n}
= 2 L(1,\chi\bar{\psi})+O(1).$$
The result follows upon noting that
$$ \left|\sum_{b\bmod m} \psi(b) \sum_{1\leq |n|\leq q} \frac{\chi(n)}{n}\left(1-e\left(\frac{nb}{m}\right)\right)\right|\leq \phi(m) \cdot \max_{\theta}\left|\sum_{1\leq |n|\leq q} \frac{\chi(n)}{n}
		\left(1-e\left(n \theta\right)\right)\right|,
$$
and that $|\tau(\psi)| = \sqrt{m}$ by the primitivity of $\psi$.
\end{proof}

In order to complete the proof of Theorem \ref{GL}, we need to approximate $L(1,\chi\bar{\psi})$ by a short truncation of its Euler product. Using zero density estimates, we prove that this is possible for almost all primitive characters $\chi$. 
\begin{pro}\label{BoundExcep}
Fix $0<\varepsilon<1$ and let $A=100/\varepsilon$.   Let $N$ be large and $m \leq \log N$. Then for all but at most $N^{\varepsilon}$ primitive characters $\chi$ modulo $q \leq N$ we have
\begin{equation}\label{ShortApproxL1chi}
L(1,\chi\bar{\psi}) = \left(1+O\left(\frac{1}{\log N}\right)\right)\prod_{p \leq \log^A N} \left(1-\frac{\chi(p)\bar{\psi(p)}}{p}\right)^{-1} .
\end{equation}
for all primitive characters $\psi$ modulo $m$. Moreover, if we assume GRH, then \eqref{ShortApproxL1chi} is valid with $A=10$, for all primitive characters $\chi$ modulo $q\leq N$ and $\psi$ modulo $m$.
\end{pro}
In order to prove this proposition, we first need some preliminary results.
\begin{lem}\label{GoodLFunctions} Let $q$ be  large and $\chi$ be a non-principal character modulo $q$. Let $2\leq T\leq q^2$ and $X\geq 2$. 
Let $\frac{1}{2} \leq \sigma_0 < 1$ and suppose that the
rectangle $\{ s:  \sigma_0 <\textup{Re}(s) \leq 1, \ \
|\textup{Im}(s)| \leq T+3\}$ does not contain any zeros of $L(s,\chi)$.
Then we have
$$
\log L(1, \chi)= -\sum_{p\leq X} \log \left(1-\frac{\chi(p)}{p}\right)+O\left(\frac{\log X}{T}+\frac{\log q}{(1-\sigma_0)T}+ \frac{\log q \log T}{(1-\sigma_0)^2} X^{(\sigma_0-1)/2}\right).
$$
\end{lem}
\begin{proof}
Let $\alpha=1/\log X$. Then it follows from Perron's formula that
\begin{equation}\label{Perron}
\begin{aligned}
&\frac{1}{2\pi i} \int_{\alpha-iT}^{\alpha+iT} \log L(1+s, \chi) \frac{X^s}{s} ds\\
& = \sum_{n\leq X} \frac{\Lambda(n)}{n\log n}\chi(n)+ O\left(\sum_{n=1}^{\infty} \frac{\Lambda(n)}{n^{1+\alpha}\log n}\min\left(1, \frac{1}{T\log|X/n|}\right)\right)\\
&= \sum_{n\leq X} \frac{\Lambda(n)}{n\log n}\chi(n)+ O\left(\frac{\log X}{T}+ \frac{1}{X}\right),
\end{aligned}
\end{equation}
by a standard estimation of the error term. Moreover, we observe that 
\begin{align*}
 \sum_{n\leq X} \frac{\Lambda(n)}{n\log n}\chi(n)&= -\sum_{p\leq X} \log \left(1-\frac{\chi(p)}{p}\right)+O\left(\sum_{k=2}^{\infty}\sum_{p^k>X} \frac{1}{k p^k}\right)\\
 &=-\sum_{p \leq X} \log\left(1-\frac{\chi(p)}{p}\right) + O\left(X^{-\frac{1}{2}}\right).
\end{align*}
%Since $\sum_{p > \sqrt{X}} p^{-2} \ll X^{-1/2}$ and $p^{(k-2)/4} \geq \log^2 k$ for each $k \geq 3$ and each sufficiently large $p$, the error term in \eqref{ERRSIMPLE} is bounded above by
%\begin{align*}
%\sum_{k \geq 2} \sum_{p^k > X} \frac{1}{kp^k} 
%X^{-\frac{1}{2}} + \sum_{k \geq 2} \frac{1}{k \log^2 k} \sum_{p > X^{\frac{1}{k}}} \frac{1}{p^2} \cdot p^{\frac{3(2-k)}{4}} &\ll X^{-\frac{1}{2}} + X^{-\frac{1}{4}} \sum_{k \geq 2} \frac{1}{k\log^2 k} \sum_{p} \frac{1}{p^2} \\ &\ll X^{-\frac{1}{4}}.
%\end{align*}
%As such,
%\begin{align*}
%\sum_{n \leq X} \frac{\Lambda(n)}{n\log n} \chi(n) = -\sum_{p \leq X} \log\left(1-\frac{\chi(p)}{p}\right) + O\left(X^{-\frac{1}{4}}\right).
 %&= -\sum_{p\leq X} \log \left(1-\frac{\chi(p)}{p}\right) +O\left(\frac{1}{\sqrt{X}} + \sum_{k \geq 3} \frac{1}{k\log^2 k} X^{-\frac{3(k-2)}{4k}}\sum_{p > X^{\frac{1}{k}}} \frac{1}{p^2}\right) \\
 %&= -\sum_{p\leq X} \log \left(1-\frac{\chi(p)}{p}\right) +O\left(X^{-\frac{1}{4}}\right).
%\end{align*}
We now move the contour in \eqref{Perron} to the line $\re(s)=\sigma_1-1$, where $\sigma_1=(1+\sigma_0)/2$. We encounter a simple pole at $s=0$ that leaves a residue of $\log L(1, \chi)$. Furthermore, it follows from Lemma 8.1 of \cite{GrSo0} that for $\sigma\geq \sigma_1$ and $|t|\leq T$ we have 
$$\log L(\sigma+it, \chi)\ll \frac{\log q}{\sigma-\sigma_0}\ll\frac{\log q}{1-\sigma_0} .$$Therefore, we deduce that
$$ \frac{1}{2\pi i} \int_{\alpha-iT}^{\alpha+iT} \log L(1+s, \chi) \frac{X^s}{s} ds= \log L(1, \chi)+\mathcal{E}, $$
where 
\begin{align*}
\mathcal{E}&= \frac{1}{2\pi i} \left(\int_{\alpha-iT}^{\sigma_1-1-iT}+\int_{\sigma_1-1-iT}^{\sigma_1-1+iT}+ \int_{\sigma_1-1+iT}^{\alpha+iT}\right)\log L(1+s, \chi) \frac{X^s}{s} ds\\
&\ll \frac{\log q}{(1-\sigma_0)T}+ \frac{\log q \log T}{(1-\sigma_0)^2} X^{(\sigma_0-1)/2}.
\end{align*}
Since $\sigma_0 \geq 1/2$, combining the above estimates completes the proof.
\end{proof}
\begin{lem}\label{InducedFixed}
Let $\xi\bmod q$ and $\psi \bmod m$ be primitive characters. Then, there is a unique primitive character $\chi$ such that $\chi\psi$ is induced by $\xi$ if $m \mid q$, and no such character exists if $m\nmid q$.
\end{lem}

\begin{proof}
Suppose that $\chi \psi$ is induced by $\xi$, where $\chi$ is a primitive character of conductor $\ell$. Then we must have $q=[\ell, m]$, and hence 
there is no such character $\chi$ if $m\nmid q$. 

Now, suppose that $m\mid q$, and let $m=p_1^{a_1}\cdots p_k^{a_k}$ be its prime factorization. We construct $\chi$ in this case as follows. Since $q=[\ell, m]$, then we have 
$q=q_0\cdot p_1^{b_1}\cdots p_k^{b_k}$ where $(q_0, m)=1$ and $b_j\geq a_j$ for all $1\leq j\leq k$, and $\ell=q_0\cdot p_1^{c_1}\cdots p_k^{c_k}$ where $c_j=b_j$ if $b_j>a_j$ and $0\leq c_j\leq a_j$ if $b_j=a_j$. 
Now, since $\xi$ is primitive then $\xi= \tilde{\xi}\cdot\xi_1\cdots\xi_k$ where $\tilde{\xi}$ is a primitive character modulo $q_0$ and $\xi_j$ is a primitive character modulo $p_j^{b_j}$ for $1\leq j\leq k$. Similarly, we have $\psi=\psi_1\cdots\psi_k$ and $\chi=\tilde{\chi}\cdot \chi_1\cdots\chi_k$ where $\tilde{\chi}$ is a primitive character modulo $q_0$ and $\psi_j, \chi_j$ are primitive characters modulo $p_j^{a_j}$ and $p_j^{c_j}$ respectively. Moreover, since $\xi$ induces $\chi\psi$ then we must have $\tilde{\chi}=\tilde{\xi}$, and $\xi_j$ induces $\chi_j\psi_j$ for all $1\leq j\leq k$. But this implies that $\chi_j(n)=\xi_j(n)\bar{\psi_j(n)}$ for all $n$ such that $p_j\nmid n$, and hence we deduce that there is only one choice for $\chi_j$ since it is  primitive. Since this holds for all $1\leq j\leq k$, the character $\chi$ is unique.
\end{proof}


\begin{proof}[Proof of Proposition \ref{BoundExcep}]
By Bombieri's classical zero-density estimate (see Theorem 20 of \cite{Bo}), we know that there are at most $N^{6(1-\sigma)}(\log N)^B$ primitive characters $\xi$ with conductor $q\leq N\log N$ and such that $L(s, \xi)$ has a zero in the rectangle $\{s: \sigma \leq \re(s) \leq 1, |\im(s)|\leq N\}$, where $B$ is an absolute constant. Let $\xi_1, \cdots, \xi_L$ be these characters with $\sigma=1-\varepsilon/20$. Then, it follows from the above argument that $L\ll N^{\varepsilon/2}$. 
 
Recall that if $\xi$ is a primitive character that induces $\tilde{\xi}$, then $L(s, \xi)$ and $L(s, \tilde{\xi})$ have the same zeros in the half-plane $\re(s)>0$.   For a primitive character $\psi$ modulo $m$, let $\mc{E}_{\psi}$ denote the set of primitive characters $\chi$ modulo $q$ with $q\leq N$ and such that $\chi \bar{\psi}$ is induced by one of the characters $\xi_j$ for $1\leq j\leq L$. Let $\mathcal{E}_m$ be the union over all primitive characters $\psi$ modulo $m$ of the sets $\mathcal{E}_{\psi}$. Then, it follows from Lemma \ref{InducedFixed} that
$$
\left|\mc{E}_m\right| \leq \sum_{\psi \bmod m \atop \psi \textup { primitive}} |\mc{E}_{\psi}| \leq L\phi(m) \ll N^{\varepsilon}.
$$
Let $X=(\log N)^{A}$ where $A=100/\varepsilon$.  If $\chi$ is a primitive character with conductor $q\leq N$ and such that $\chi\notin\mc{E}_m$ then it follows from Lemma \ref{GoodLFunctions} with $T=X$ that for all primitive characters $\psi$ modulo $m$ we have
$$
\log L(1,\chi\bar{\psi})= -\sum_{p\leq X} \log\left(1-\frac{\chi(p)\bar{\psi}(p)}{p}\right) + O\left(\frac{1}{\log N}\right),
$$
which implies \eqref{ShortApproxL1chi}.
Finally, if we assume GRH, then this estimate is valid for all primitive characters $\chi$ modulo $q\leq N$ and $\psi$ modulo $m$ with $X=(\log N)^{10}$ by Lemma \ref{GoodLFunctions}.
\end{proof}
We can now prove Theorem \ref{GL}.
\begin{proof}[Proof of Theorem \ref{GL}]
Combining Propositions \ref{CharSumL1} and \ref{BoundExcep} we deduce that for all but at most $N^{\varepsilon}$ primitive characters $\chi$ modulo $q$ with $N^{\varepsilon/3} \leq q \leq N$ we have
\begin{equation}\label{LowerBoundMCHI}
M(\chi) + \sqrt{q}  \gg \frac{\sqrt{qm}}{\phi(m)} \prod_{p \leq \log^A N} \left(1-\frac{\chi(p)\bar{\psi(p)}}{p}\right)^{-1} 
\end{equation}
with $A=100/\varepsilon$. The first part of the theorem follows, upon noting that
$$ \prod_{p \leq \log^A N} \left(1-\frac{\chi(p)\bar{\psi(p)}}{p}\right)^{-1}  \gg_{\varepsilon} (\log_2 q) \cdot \exp\big(-\mb{D}(\chi,\psi;\log q)^2\big).
$$
The second part follows along the same lines, since if we assume GRH then \eqref{LowerBoundMCHI} holds with $A=10$ for all primitive characters $\chi$ with conductor $q\leq N$.
\end{proof}

%%%%%%%%%%%%%%%%%%%%%%%%%%%%%%%%%%%%%%%%%%%%%%%%%%%
\section{Estimates for the distance $\mb{D}(\chi,\psi;y)$: Proofs of Proposition \ref{UPPER} and Theorem \ref{MCHILOW}}
We shall first prove a lower bound for $\mb{D}(\chi,\psi;y)^2$, which is a refined version of \eqref{LOWERBOUDISTANCE}, that shows that Proposition  \ref{UPPER} is best possible. This will also be the main ingredient in the proof of Proposition \ref{MINDIST}. 

\begin{pro} \label{UPPER2}
Let $g\geq 3$ be a fixed odd integer, and $\varepsilon>0$ be small. Let $\psi$ be an odd primitive character of conductor $m$ and order $k$, and $y$ be such that $m\leq (\log y)^{4/7}$. Put $k^{\ast}=k/(g,k)$.  Then, for any primitive character $\chi \pmod q$ of order $g$ we have
$$
\mb{D}(\chi,\psi;y)^2\geq \left(1-(1-\delta_g)\frac{\pi/gk^{\ast}}{\tan(\pi/gk^{\ast})}\right) \log_2y-\beta \varepsilon \log m+ O\left(\log_2 m\right),
$$
where $\beta=0$ if $m$ is a non-exceptional modulus, and $\beta=1$ if $m$ is exceptional.
\end{pro}
We say that $m\geq 1$ is an {\it exceptional} modulus if there exists a Dirichlet character $\chi_m$ and a complex number $s$ such that  $L(s,\chi_m)=0$ and 
\begin{equation}\label{EXCEPTIONALMODULUS}
\text{Re}(s)\geq 1-\frac{c}{\log (m(\text{Im}(s)+2))}
\end{equation} 
for some sufficiently small constant $c>0$. 
One expects that there are no such moduli, but what is known unconditionally is that if $m$ is exceptional, then there is only one {\it exceptional} character $\chi_m$ modulo $m$, which is quadratic, and for which $L(s,\chi_m)$ has a unique zero in the region \eqref{EXCEPTIONALMODULUS} which is real and simple (this zero is called a Siegel zero).  


For $g \geq 3$, we let $\mu_g$ denote the set of $g$-th roots of unity. Then, we observe that
\begin{align}
\mb{D}(\chi,\psi;y)^2 &= \log\log y- \sum_{p \leq y} \frac{\text{Re}(\chi(p)\bar{\psi}(p))}{p} +O(1)  \nonumber\\
&\geq  \log \log  y - \sum_{\ell\bmod k} \max_{z \in \mu_g \cup \{0\}} \text{Re}\left(z \cdot e\left(-\frac{\ell}{k}\right)\right)\sum_{p \leq y \atop \psi(p)= e\left(\frac{\ell}{k}\right)} \frac{1}{p} +O(1).\label{DIST}
\end{align}
Proposition \ref{UPPER2} follows from this inequality together with Proposition \ref{PreciseSumMax} below, which provides an asymptotic formula for the sum on the right hand side of \eqref{DIST}. To establish Proposition \ref{UPPER}, we need an additional ingredient, namely that there exist many primitive characters $\chi$ whose values we can control at the small primes $p\leq c \log q$ so that the inequality in \eqref{DIST} is sharp for $y\leq c\log q$. This is proven in Lemma \ref{VEC} below. 
\begin{pro}\label{PreciseSumMax}
Let $g\geq 3$ be a fixed odd integer, and $\varepsilon>0$ be small. Let $\psi$ be an odd primitive character of conductor $m$ and order $k$, and $y$ be such that $m\leq (\log y)^{4/7}$. Put $k^{\ast}=k/(g,k)$. Then 
\begin{equation}\label{AsymptoticMaxSumRE}
\begin{aligned}
&\sum_{\ell\bmod k} \max_{z \in \mu_g \cup \{0\}} \textup{Re}\left(z \cdot e\left(-\frac{\ell}{k}\right)\right)\sum_{p \leq y \atop \psi(p)= e\left(\frac{\ell}{k}\right)} \frac{1}{p} \\
&= (1-\delta_g)\frac{\pi/gk^{\ast}}{\tan(\pi/gk^{\ast})}\log_2 y + \theta \varepsilon \log m+ O\left(\log_2 m\right),
\end{aligned}
\end{equation}
where $\theta=0$ if $m$ in a non-exceptional modulus, and $|\theta|\leq 1$ if $m$ is exceptional. 
\end{pro}

%We will be choosing our vector $\mbf{z}$ to maximize the sum in the following lemma.
We first record the following lemma, which is a special case of Lemma 8.3 of \cite{GOLD}. 
\begin{lem} \label{MAX}
Let $g,k$ and $k^{\ast}$ be as in Proposition \ref{PreciseSumMax}. Then
%\geq 3$ be odd and let $k \geq 2$ be even. Put $k^{\ast} := k/(g,k)$. Then
\begin{equation*}
\frac{1}{k}\sum_{\ell\bmod k}\max_{z \in \mu_g \cup \{0\}} \textup{Re}\left(z \cdot e\left(-\frac{\ell}{k}\right)\right) = (1-\delta_g)\frac{\pi/gk^{\ast}}{\tan(\pi/gk^{\ast})}. 
%+ O\left(\frac{1}{(gk^{\ast})^4}\right)\right).
\end{equation*}
\end{lem}
\begin{proof}
This is Lemma 8.4 of \cite{GOLD} (see also Lemma \ref{MAXGold} below) with $\theta=0$.
%we have
%\begin{align*}
%\frac{1}{k}\sum_{\ell\bmod k}\max_{z \in \mu_g \cup \{0\}} \text{Re}\left(z \cdot e\left(-\frac{\ell}{k}\right)\right) &= \frac{\sin(\pi/g)}{k^{\ast}\tan(\pi/gk^{\ast})} = (1-\delta_g)\frac{\pi/gk^{\ast}}{\tan(\pi/gk^{\ast})} \\
%&\leq (1-\delta_g)\left(1-\frac{3 \pi^2}{10(gk^{\ast})^2}\right),
%+ O\left(\frac{1}{(gk^{\ast})^4}\right)\right),
%\end{align*}
%since $gk^{\ast} \geq 6$ and $y/\tan y \leq 1-\frac{3}{10}y^2$ whenever $0 \leq y \leq \frac{\pi}{6}$.
\end{proof}

In view of this lemma, our next task is to estimate the inner sum in the left hand side of \eqref{AsymptoticMaxSumRE}. Since $\psi$ is periodic modulo $m$ we have
\begin{equation} \label{ALLAS}
\sum_{p \leq y \atop \psi(p) = e\left(\frac{\ell}{k}\right)}\frac{1}{p} = \sum_{a \bmod m \atop \psi(a) = e\left(\frac{\ell}{k}\right)} \sum_{p \leq y \atop p\equiv a \bmod m} \frac{1}{p}.
\end{equation}

%Our main object of study is thus
%\begin{equation}\sum_{l(k)} \text{Re}\left(z_le\left(-\frac{l}{k}\right)\right)\sum_{p \leq y \atop \psi(p) = e\left(\frac{l}{k}\right)}\frac{1}{p} \label{MAINSUM}\end{equation}
%We shall estimate 
%\begin{equation} \label{MAIN}
%\sum_{p \leq y \atop \psi(p) = e\left(\frac{l}{k}\right)} \frac{1}{p},
%\end{equation}
%for each $l$ modulo $k$. 

In what follows we shall need estimates of Mertens type for sums of reciprocals of primes from specific arithmetic progressions $a$ modulo $m$ that are uniform in a range of the modulus $m$. Results of this type were established by Languasco and Zaccagnini \cite{LaZ}.  
\begin{lem} [Theorem 2 and Corollary 3 of \cite{LaZ}] \label{LOGLZ}
Let $x \geq 3$.  Then, uniformly in $m \leq \log x$ and reduced residue classes $a$ modulo $m$, we have
$$
-\sum_{p \leq x \atop p \equiv a \bmod m} \log\left(1-\frac{1}{p}\right) = \frac{1}{\phi(m)} \log_2 x -  C_m(a) + O\left(\frac{(\log\log x)^{16/5}}{(\log x)^{3/5}}\right),
$$
where $C_m(a)$ is defined in \eqref{MERTENSCONSTANT} below.
\end{lem}

%Results of this type depend on the existence of exceptional characters modulo $m$. For the range $m \leq (\log \log y)^A$ we shall be able to eschew such technicalities. 



We shall refer to $C_m(a)$ as the \emph{Mertens constant} of the residue class $a$ modulo $m$. Given $m \geq 2$ and $(a, m)=1$, this quantity is defined by
%As described in the introduction of \cite{LaZ}, Williams showed that for $m$ prime
\begin{equation}\label{MERTENSCONSTANT}
C_m(a):= \frac{1}{\phi(m)}\sum_{\chi\neq \chi_0 \bmod m} \overline{\chi}(a) \cdot \log \frac{K(1, \chi)}{L(1, \chi)}-\frac{1}{\phi(m)}\left(\gamma + \log(\phi(m)/m)\right),
\end{equation}
where, for each non-principal character $\chi$ modulo $q$,
$$K(s, \chi):= \sum_{n=1}^{\infty} \frac{k_{\chi}(n)}{n^s} $$ 
is an absolutely convergent Dirichlet series for $\re(s)>0$, and $k_{\chi}(n)$ is a completely multiplicative function defined as
\begin{equation}\label{KChi}
k_{\chi}(p):= p\left(1-\left(1-\frac{\chi(p)}{p}\right)\left(1-\frac 1p\right)^{-\chi(p)}\right).
\end{equation}
Moreover, it follows from the definition of $k_{\chi}(p)$ and Taylor expansion that
\begin{equation}\label{BoundK}
|k_{\chi}(p)|\ll \frac{1}{p}.
\end{equation}

In order to study the asymptotic behaviour of the sum in \eqref{ALLAS}, it will be crucial to have an upper bound for the average of $|C_m(a)|$. 
\begin{lem}\label{AverageMertens}
Fix $\varepsilon>0$, and let $m\geq 3$. Then, we have
$$
\sum_{\substack{a \bmod m\\ (a,m)=1}} \left|C_m(a)\right| \leq  \begin{cases} O(\log_2m), & \text{ if } m \text{ is a non-exceptional modulus},\\
\varepsilon \log m + O(\log_2m), &  \text{ if } m \text{ is exceptional}.\end{cases}
$$
\end{lem}

\begin{proof}
First, since $\phi(m)\gg m/\log_2m$ then 
$$
C_m(a)=\frac{1}{\phi(m)}\sum_{\chi\neq \chi_0 \bmod m} \overline{\chi}(a) \cdot \log \frac{K(1, \chi)}{L(1, \chi)}+O\left(\frac{\log_3 m}{\phi(m)}\right).
$$
Let $\chi$ be a non-principal character modulo $m$. By \eqref{BoundK} we have
\begin{align*}
\log K(1, \chi)&= -\sum_{p\leq x} \log\left(1-\frac{k_{\chi}(p)}{p}\right) +O\left(\sum_{p>x}\frac{|k_{\chi}(p)|}{p}\right)\\
&= -\sum_{p\leq x} \log\left(1-\frac{k_{\chi}(p)}{p}\right) +O\left(\frac 1x\right).
\end{align*}
Furthermore, it follows from \eqref{KChi} that
$$-\log\left(1-\frac{k_{\chi}(p)}{p}\right)+\log \left(1-\frac{\chi(p)}{p}\right)= \chi(p)\log \left(1-\frac{1}{p}\right).$$
If $\chi$ is a non-exceptional character, then $L(\sigma+it, \chi)$ does not vanish when
$$ \sigma\geq 1-\frac{c}{\log(m(|t|+2))},$$
for some positive constant $c$. 
Therefore, taking $T=m^2$, $\sigma_0= 1- c/(4\log m)$ and $X=\exp((\log m)^{3})$ in Lemma \ref{GoodLFunctions} we obtain
\begin{equation}\label{LongApproxL1}
\log L(1,\chi)= -\sum_{p\leq X}\log\left(1-
\frac{\chi(p)}{p}\right) +O\left(\frac{1}{m}\right).
\end{equation}
 We first consider the case when $m$ is a non-exceptional modulus. Using the above estimates together with the orthogonality of characters we conclude that
\begin{equation}\label{OrthogonalityCM}
\begin{aligned}
C_m(a)&= \frac{1}{\phi(m)}\sum_{\chi\neq \chi_0 \bmod m} \overline{\chi}(a)\sum_{p\leq X}\chi(p)\log \left(1-\frac{1}{p}\right) +O\left(\frac{\log_3 m}{\phi(m)}\right)\\
&= \sum_{\substack{p\leq X\\ p\equiv a \bmod m}}\log \left(1-\frac{1}{p}\right)-\frac{1}{\phi(m)} \sum_{\substack{p\leq X \\ p\nmid m}}\log \left(1-\frac{1}{p}\right)+O\left(\frac{\log_3 m}{\phi(m)}\right).\\
\end{aligned}
\end{equation}
Thus, we deduce in this case that
\begin{align*}
\sum_{\substack{a \bmod m\\ (a,m)=1}} \left|C_m(a)\right|
&\leq 
- \sum_{\substack{a \bmod m\\ (a,m)=1}} \sum_{\substack{p\leq X\\ p\equiv a \bmod m}}\log \left(1-\frac{1}{p}\right)
- \sum_{p\leq X}\log \left(1-\frac{1}{p}\right) +O(\log_3m)\\
& \ll \log_2 m.
\end{align*}
Now, suppose that $m$ is an exceptional modulus, and let $\chi_m$ be the exceptional character modulo $m$. The approximation \eqref{LongApproxL1} is valid for all non-principal characters $\chi\neq \chi_m$ modulo $m$. Furthermore, for $\chi=\chi_m$ we have Siegel's bound (see Theorem 11.4 in \cite{MVbook})
$$ \log L(1, \chi_m)\geq -\varepsilon \log m +O_{\varepsilon}(1),$$
and hence, instead of \eqref{LongApproxL1} we use that
$$ \left|\log L(1, \chi_m) + \sum_{p\leq X}\log\left(1-
\frac{\chi_m(p)}{p}\right)\right|\leq \varepsilon\log m+ O(\log_2 m).$$
Thus, similarly to \eqref{OrthogonalityCM} we obtain in this case that
$$
|C_m(a)| \leq -\sum_{\substack{p\leq X\\ p\equiv a \bmod m}}\log \left(1-\frac{1}{p}\right)+ \frac{\varepsilon\log m}{\phi(m)}+O\left(\frac{\log_2 m}{\phi(m)}\right).$$
Summing over all reduced residue classes $a$ modulo $m$ gives the desired bound.
\end{proof}
Proposition \ref{PreciseSumMax} now follows readily.
\begin{proof}[Proof of Proposition \ref{PreciseSumMax}]
First, note that for each fixed $\ell$ modulo $k$, there are exactly $\phi(m)/k$ residue classes $a$ modulo $m$ such that $(a,m)=1$ and $\psi(a) = e\left(\frac{\ell}{k}\right)$. This follows from the simple fact that the number of such residue classes equals the size of the kernel of $\psi$, and by basic group theory this is $|\left(\mb{Z}/m\mb{Z}\right)^{\ast}|/|\text{Im}(\psi)| = \phi(m)/k$. Thus, we deduce from \eqref{ALLAS} and Lemma \ref{LOGLZ} that
\begin{align*}
\sum_{p \leq y \atop \psi(p) = e\left(\frac{\ell}{k}\right)} \frac{1}{p} &= \sum_{a\bmod m \atop \psi(a) = e\left(\frac{\ell}{k}\right)} \left(\frac{\log_2 y}{\phi(m)} - C_m(a) + \sum_{p \leq y \atop p \equiv a(m)} \left(\log\left(1-\frac{1}{p}\right)+ \frac{1}{p}\right) + O\left(\frac{1}{(\log y)^{4/7}}\right)\right)\\
&= \frac{\log_2 y}{k} - \sum_{a \bmod m \atop \psi(a) = e\left(\frac{\ell}{k}\right)} C_m(a) + \sum_{p \leq y \atop \psi(p) =e\left(\frac{\ell}{k}\right)} \left(\log\left(1-\frac{1}{p}\right) + \frac{1}{p}\right) + O\left(\frac{\phi(m)}{k(\log y)^{4/7}}\right).
\end{align*}
Summing over $\ell$ modulo $k$, and using Lemma \ref{MAX},  we get
\begin{align*}
&\sum_{\ell\bmod k} \max_{z \in \mu_g \cup \{0\}} \textup{Re}\left(z \cdot e\left(-\frac{\ell}{k}\right)\right)\sum_{p \leq y \atop \psi(p)= e\left(\frac{\ell}{k}\right)} \frac{1}{p} \\
&= (1-\delta_g)\frac{\pi/gk^{\ast}}{\tan(\pi/gk^{\ast})}\log_2 y + \theta \sum_{\substack{a \bmod m\\ (a,m)=1}} |C_m(a)|+ O\left(1\right),
\end{align*}
for some complex number $|\theta|\leq 1$. Appealing to Lemma \ref{AverageMertens} completes the proof. 
\end{proof}
%We specialize to $k = \phi(m)$ (e.g., $\psi$ is the generator of the dual group modulo $m$, which is cyclic for $m$ prime), so that the third error term is negligible. 
%Putting $y = C \log q$ as above and inserting this last expression into \eqref{DIST} gives precisely
%\begin{equation*}\mb{D}^2(\chi,\psi;\log q) = \left(\delta_g+ \frac{1-\delta_g}{6(gk)^2}\right)\log_3 q + O_g\left(\frac{\log_3 q}{k^4} + \log_2 m\right),\end{equation*}
%which completes the proof of Proposition \ref{UPPER} upon choosing $k = m-1$.

Let $\psi$ be any odd character modulo $m$, with even order $k$. In choosing characters $\chi$ of order $g$ and conductor $q\leq N$ that maximize the distance $\mb{D}(\chi,\psi;y)$  with $y\leq (\log N)/10$, we will need to be able to choose the values of $\chi$ at the ``small'' primes $p \leq y$. Using Eisenstein's reciprocity law and the Chinese Remainder Theorem, Goldmakher \cite{GOLD} proved the existence of such characters.

\begin{lem}[Proposition 9.3 of \cite{GOLD}] \label{FIX}
Let $g \geq 3$ be fixed, and $y$ be large. Let $\{z_p\}_p$ be a sequence of complex numbers such that $z_p\in \mu_g \cup \{0\}$ for each prime $p$. There exists a positive integer $q$ such that $$g\prod_{\substack{p\leq y\\ p\nmid g}} p\leq q\leq  2g\prod_{\substack{p\leq y\\ p\nmid g}}p,$$ and a primitive Dirichlet character $\chi$ of order $g$ and conductor $q$ such that $\chi(p) = z_p$ for all $p \leq y$ with $p\nmid g$.
\end{lem}
However, in order to prove Theorem \ref{MCHILOW} we need to find ``many'' such characters, since we must avoid those in the exceptional set of Theorem \ref{GL}, which has size at most $N^{\varepsilon}$.  To this end we prove
\begin{lem} \label{VEC}
Let $N$ be large. Let $g \geq 3$ be fixed. Let $2 \leq y \leq (\log N)/10$, and put $\mbf{z} = (z_p)_{p \leq y} \in (\mu_g \cup \{0\})^{\pi(y)}$. There are  
$$\gg \frac{N^{3/4}}{g^{2\pi(y)+2} \log^2 N}$$ primitive Dirichlet characters $\chi$ of order $g$ and conductor $q\leq N$ such that $\chi(p) = z_p$ for each $p \leq y$ such that $p\nmid g$.
\end{lem}
The special case $\mbf{z}=\mbf{1}=(1, 1, \dots, 1)$ was proved by the first author in Lemma 2.3 of \cite{LAM}, but the proof there does not appear to generalize to all $\mbf{z}\in (\mu_g \cup \{0\})^{\pi(y)}$. However, we will show that one can combine the special case $\mbf{z}=\mbf{1}$ with Lemma \ref{FIX} in order to obtain the general case in Lemma \ref{VEC}.    

\begin{proof}[Proof of Lemma \ref{VEC}]
Let $S_{\mbf{z},g}(N)$ be the set of all characters $\chi$ of order $g$ and conductor $q\leq N$ such that $\chi(p) = z_p$ for all $p\leq y$ with $p\nmid g$. By Lemma \ref{FIX}, there exists $\ell$ and a primitive Dirichlet character $\xi$ of order $g$ and conductor $\ell$ such that $\xi(p) = z_p$ for all $p \leq y$ with $p\nmid g$. Moreover, one has
$$\log \ell =\sum_{p\leq y}\log p+O_g(1)= y(1+o(1)),$$
by the prime number theorem, and hence $\ell\leq N^{1/8}$ by our assumption on $y$. 

On the other hand, Lemma 2.3 of \cite{LAM} implies that there are
$$\gg  \frac{N^{3/4}}{g^{2\pi(y)+2} \log^2 N}$$
primitive Dirichlet characters $\psi_{n}$ of order $g$ and conductor $n$, such that $n=q_1q_2$ where $N^{3/8}<q_1<q_2<2N^{3/8}$ are primes with $p_1\equiv p_2\equiv 1\bmod g$, and such that $\psi_{n}(p)=1$ for all primes $p\leq y$. Now, for any such $n$ we have $(\ell, n)=1$ since $\ell \leq N^{1/8}$, and hence $\psi_{n}\xi$ is a primitive character of order $g$ and conductor $n \ell \leq  N$. Finally observe that $\psi_{n}\xi(p) = z_p$ for each $p \leq y$ such that $p\nmid g$. Thus we deduce that $\psi_{n}\xi\in S_{\mbf{z},g}(N)$ for every character $\psi_{n}$, completing the proof.
 \end{proof}
 We finish this section by proving Proposition \ref{UPPER} and Theorem \ref{MCHILOW}.
 \begin{proof}[Proof of Proposition \ref{UPPER}]
 Let $m$ be a non-exceptional modulus, and $\psi$ be an odd primitive character modulo $m$ with order $k$.  For each $0\leq \ell \leq k-1$, suppose that the maximum of $\textup{Re}\left(ze\left(-\frac{\ell}{k}\right)\right)$ for $z \in (\mu_g \cup \{0\})^{\pi(y)}$ is attained when $z=z_{\ell}$. Then, it follows from Lemma \ref{VEC} that there are at least $\sqrt{N}$ primitive characters $\chi$ of order $g$ and conductor $q\leq N$ such that
$$\sum_{p \leq y} \text{Re}\frac{\chi(p)\bar{\psi}(p)}{p}=\sum_{\ell\bmod k} \text{Re}\left(z_{\ell} \cdot e\left(-\frac{\ell}{k}\right)\right)\sum_{p \leq y \atop \psi(p)= e\left(\frac{\ell}{k}\right)} \frac{1}{p}+O_{g}(1).$$
The desired result then follows from \eqref{DIST} and Proposition \ref{PreciseSumMax}. 
\end{proof}

\begin{proof}[Proof of Theorem \ref{MCHILOW}]
Let $N$ be sufficiently large, and let $y=(\log N)/10$. Let $m$ be a prime number that is also a non-exceptional modulus, such that $\sqrt{\log_3 N}\leq m \leq 2\sqrt{\log_3N}$. One can make such a choice since it is known that there is at most one exceptional prime modulus between $x$ and $2x$ for any $x\geq 2$ (see Chapter 14 of \cite{Da} for a reference). Let $\psi$ be a primitive character modulo $m$ of order $k=\phi(m)=m-1$. Note that such a character is necessarily odd. By Proposition \ref{UPPER}, there are at least $\sqrt{N}/2$ primitive characters of order $g$ and conductor $N^{1/3}\leq q\leq N$ such that 
\begin{equation*}
\mb{D}(\chi,\psi;y)^2 = \left(1-(1-\delta_g)\frac{\pi/gk^{\ast}}{\tan(\pi/gk^{\ast})}\right) \log_2y +O(\log_2 m)= \delta_g \log_3 q + O\left(\log_5 q\right),
\end{equation*}
since $gk^{\ast}\geq k$ and $t/\tan(t)=1+O(t^2)$.
Thus, since $\mb{D}(\chi,\psi;\log q)^2=\mb{D}(\chi,\psi;y)^2+O(1)$, then it follows from Theorem \ref{GL} (with $\varepsilon=1/4$) that there are at least $\sqrt{N}/3$ primitive characters of order $g$ and conductor $N^{1/3}\leq q\leq N$ such that
$$ 
M(\chi) \gg \frac{\sqrt{qm}}{\phi(m)} (\log_2 q)^{1-\delta_g} (\log_4 q)^{O(1)}\gg \sqrt{q} (\log_2 q)^{1-\delta_g} (\log_3 q)^{-\frac14}(\log_4 q)^{O(1)} .
$$
\end{proof}
%%%%%%%%%%%%%%%%%%%%%%%%%%%%%%%%%
\section{Estimates for $\mc{M}(\chi\bar{\psi}; y, T)$: Proofs of Propositions \ref{MINDIST} and \ref{AD}}

\subsection{Lower bounds for $\mc{M}(\chi\bar{\psi}; y, T)$ for small twists $T$: Proof of Proposition \ref{MINDIST}}
Let $\chi$ be a primitive character modulo $q$ of odd order $g\geq 3$, and $\psi$ be an odd primitive character of conductor $m$ and order $k$. Let $y\geq \exp(m^{7/(4\alpha)})$ be a real number, and put $z=\exp\left((\log y)^{\alpha}\right)$. Since $m\leq(\log z)^{4/7}$, then it follows from 
Proposition \ref{UPPER2} that for all $x\geq z$ we have
\begin{equation}\label{ProUPPER2}
\begin{aligned}
\mb{D}(\chi, \psi; x)^2 &\geq \left(1-(1-\delta_g)\frac{\pi/gk^{\ast}}{\tan(\pi/gk^{\ast})}\right) \log_2x-\beta \varepsilon \log m+ O\left(\log_2 m\right)\\
&\geq \left(\delta_g+ \frac{\pi^2(1-\delta_g)}{4 (gk^{\ast})^2}\right) \log_2x-\beta \varepsilon \log m+ O\left(\log_2 m\right).
\end{aligned}
\end{equation}
since $g k^{\ast}\geq 6$, and $u/\tan(u)\leq 1-u^2/4$ for $0\leq u\leq \pi/6$. 

Let $t$ be a real number such that $|t|\leq (\log y)^{-\alpha}$. First, if $|t|\leq (\log y)^{-1}$ , then since $p^{-it}=1+O(|t|\log p)$ we obtain
\begin{equation}\label{SMALLT}
\mb{D}(\chi\bar{\psi}, n^{it}; y)^2= \mb{D}(\chi, \psi; y)^2+O\left(|t| \sum_{p \leq y} \frac{\log p}{p}\right)=\mb{D}(\chi, \psi; y)^2+O(1),
\end{equation}
and hence the desired lower bound for $\mb{D}(\chi\bar{\psi}, n^{it}; y)^2$ follows from \eqref{ProUPPER2}. Thus, we can and will assume throughout this subsection that $|t|> (\log y)^{-1}$. 
Furthermore, since $|t| \leq (\log y)^{-\alpha}=1/\log z$, then similarly to \eqref{SMALLT} one has
 $$
\mb{D}(\chi\bar{\psi}, n^{it}; y)^2
=\mb{D}(\chi, \psi; z)^2+
\sum_{z < p \leq y} \frac{1-\text{Re}(\chi(p)\bar{\psi}(p)p^{-it})}{p} +O(1).
$$
Therefore, in view of  \eqref{ProUPPER2}, it is enough to prove the following result in order to deduce  Proposition \ref{MINDIST}.
\begin{pro}\label{MEDIUMPRIMES} Let $\chi$, $\psi$, $y$, $z$ and $t$ be as above. Then we have
 $$\sum_{z < p \leq y} \frac{1-\textup{Re}(\chi(p)\bar{\psi}(p)p^{-it})}{p}\geq \delta_g \log\left(\frac{\log y}{\log z}\right)+O(1).$$
\end{pro}
To establish this result, we will follow the arguments in Section 8 of \cite{GOLD}. We shall need the following lemmas.
\begin{lem}[Lemma 8.3 of \cite{GOLD}]\label{MAXGold}
Let $g\geq 3$ be odd, $k\geq 2$ be even, and $\theta\in \mathbb{R}$. Put $k^{\ast}=k/(g, k)$. Then we have
$$
\frac{1}{k}\sum_{\ell \bmod k} \max_{z\in \mu_g\cup \{0\}}\textup{Re}\left(z \cdot e\left(\theta - \frac{\ell}{k}\right)\right) = \frac{\sin(\pi/g)}{ k^{\ast}\tan(\pi/gk^{\ast})} F_{gk^{\ast}}\left(-gk^{\ast} \theta\right),
$$
where $F_n(u) := \cos(2\pi\{u\}/n) + \tan(\pi/n)\sin(2\pi \{u\}/n)$, and $\{u\}$ is the fractional part of $u$.
\end{lem}

\begin{lem}\label{IntegralFN}
Let $T>1$ and $n\geq 3$ be a positive integer. Then
\begin{equation}
\int_1^T \frac{F_{n}(u)}{u} du= \frac{n}{\pi} \tan\left(\frac{\pi}{n}\right) \log T + O(1), \label{UPINT}
\end{equation}
and 
\begin{equation}
\int_{1/T}^1 \frac{F_{n}(u)}{u} du= \log T+ O(1). \label{LOWINT}
\end{equation}
In particular, for any $0<A<B$ we have
\begin{equation}\label{LOWUPINT}
\int_{A}^B \frac{F_{n}(u)}{u} du\leq  \frac{n}{\pi} \tan\left(\frac{\pi}{n}\right)\log(B/A)+ O(1),
\end{equation}
and the constants in the $O(1)$ error terms are absolute. 
\end{lem}
\begin{proof}
We first prove \eqref{UPINT}. Since $F_{n}$ is bounded and $1$-periodic, we have
\begin{align*}
\int_1^T \frac{F_{n}(u)}{u} du 
&= \sum_{1 \leq j \leq T} \int_0^1 \frac{F_{n}(u)}{u+j} du +O(1) = \sum_{1 \leq j \leq T} \frac{1}{j} \int_0^1 F_{n}(u) du + O(1) \\
&= \frac{n}{\pi} \tan\left(\frac{\pi}{n}\right) \log T + O(1). 
\end{align*}
The second estimate \eqref{LOWINT} follows from observing that for $u\in [0, 1)$ and $n\geq 3$ we have
$$ F_n(u)= 1+ O\left(\frac{u^2}{n^2}+\tan\left(\frac{\pi}{n}\right)\frac{u}{n}\right)=1+O(u).$$
Finally, to prove \eqref{LOWUPINT} we consider the three cases $1\leq A<B$, $A<1<B$, and $A<B\leq 1$. The first case follows from \eqref{UPINT}, and the third follows from \eqref{LOWINT} upon using the inequality $\tan(\pi/n)\ge \pi/n$.  Finally, in the second case we have
$$ \int_A^B \frac{F_{n}(u)}{u} du = \int_A^1 \frac{F_{n}(u)}{u} du +\int_1^B\frac{F_{n}(u)}{u} du=\frac{n}{\pi} \tan\left(\frac{\pi}{n}\right)\log B-\log A+ O(1),$$
which implies the result since $\tan(\pi/n)\ge \pi/n$ and $-\log A>0.$
\end{proof}
\begin{proof}[Proof of Proposition \ref{MEDIUMPRIMES}]
Let $x_0=z$, and $\delta>0$ be a small parameter to be chosen. For each positive integer $r \leq R := \left\lfloor\log(y/z)/\log(1+\delta)\right\rfloor$, set $x_r := (1+\delta)^r z$. We consider the sum
$$  S= \sum_{z < p \leq y}\frac{\text{Re}(\chi(p)\bar{\psi}(p)p^{-it})}{p}=\sum_{0 \leq r \leq R-1} \sum_{x_r < p \leq x_{r+1}} \frac{\text{Re}(\chi(p)\bar{\psi}(p)p^{-it})}{p} + O\left(\delta\right).
$$
Write $\theta_r := -\frac{t \log x_r}{2\pi}$, and note that if $p \in (x_r,x_{r+1}]$ then $$|p^{-it} - e(\theta_r)| \ll |t| \log(1+\delta) \ll \delta|t|,$$ so that
\begin{equation}\label{SPLIT}
S= \sum_{0 \leq r \leq R-1} \sum_{x_r < p \leq x_{r+1}} \frac{\text{Re}(e(\theta_r)\chi(p)\bar{\psi}(p))}{p}
 +O\left(\delta\right).
\end{equation}
For each $0 \leq r \leq R-1$, we define
$$
S_r:=\sum_{x_r < p \leq x_{r+1}} \frac{\text{Re}(e(\theta_r)\chi(p)\bar{\psi}(p))}{p}
\leq  \sum_{\ell \bmod k} \max_{z \in \mu_p \cup \{0\}} \text{Re}\left(ze\left(\theta_r-\frac{\ell}{k}\right)\right)\sum_{a \bmod m \atop \psi(a) = e\left(\frac{\ell}{k}\right)} \sum_{x_r < p \leq x_{r+1} \atop p \equiv a \bmod m} \frac{1}{p}.
$$
%We now invoke Theorem 2 of \cite{LaZ}. Thus, as $m$ is non-exceptional, provided that $m \leq \exp\left(\log^{2/5} x_0 \log_2^{\frac{1}{5}} x_0\right)$ then there is some $B > 0$ such that for each $r$ we have
%On GRH, we have
Note that $m  \leq (\log x_r)^{4/7}$ for each $0 \leq r \leq R$.  Thus, by the Siegel-Walfisz theorem (see Corollary 11.19 in \cite{MVbook}), there is a positive constant $b$ such that 
\begin{equation*}
\sum_{x_r < p \leq x_{r+1} \atop p \equiv a \bmod m} \log p = \frac{x_{r+1}-x_r}{\phi(m)} +O\left(x_r \exp\left(-b\sqrt{\log x_r}\right)\right),
%\log^2 x_r}{\sqrt{x_r}}\right)\right).
%\log\left(\frac{\log x_{r+1}}{\log x_r}\right) + O\left(\phi(m)^{-1}L(x_r)^{-B}\right);
\end{equation*}
for all $0 \leq r \leq R-1$. Moreover, for $x_r < p \leq x_{r+1} = (1+\delta)x_r$, we have
$$
\frac{1}{p} = \frac{\log p}{x_r\log x_r}\left(1+\frac{p\log p-x_r\log x_r}{x_r\log x_r}\right)^{-1}= \big(1+O(\delta)\big) \frac{\log p}{x_r\log x_r}.
$$
Thus, combining these two statements, we get
\begin{equation*}
\sum_{x_r < p \leq x_{r+1} \atop p \equiv a \bmod m} \frac{1}{p} = \frac{1+O(\delta)}{x_r\log x_r} \sum_{x_r < p \leq x_{r+1} \atop p \equiv a \bmod m} \log p = \big(1+O(\delta)\big)\frac{\delta}{\phi(m) \log x_r} +O\left(\exp\left(-b\sqrt{\log z}\right)\right),
\end{equation*}
and upon summing over $a$ modulo $m$ such that $\psi(a) = e\left(\frac{\ell}{k}\right)$, of which there are $\phi(m)/k$ (as remarked in Section 3), we see that
\begin{align*}
S_r &\leq \big(1+O(\delta)\big)\frac{\delta}{k\log x_r}\sum_{\ell \bmod k} \max_{z \in \mu_p \cup \{0\}} \text{Re}\left(ze\left(\theta_r-\frac{\ell}{k}\right)\right) + O\left(\phi(m)\exp\left(-b\sqrt{\log z}\right)\right)\\
& \leq \big(1+O(\delta)\big)\frac{\delta \sin(\pi/g)}{ k^{\ast}\tan(\pi/gk^{\ast})}\frac{F_{gk^{\ast}}(-gk^{\ast} \theta_r)}{\log x_r}+ O\left(\phi(m)\exp\left(-b\sqrt{\log z}\right)\right)
\end{align*} 
by Lemma \ref{MAXGold}. Summing over $0 \leq r \leq R-1$ this yields
\begin{equation}\label{SPLIT2}
\sum_{0 \leq r \leq R-1} S_r \leq\big(1+O(\delta)\big)\frac{\delta \sin(\pi/g)}{ k^{\ast}\tan(\pi/gk^{\ast})} \sum_{0 \leq r \leq R-1} \frac{F_{gk^{\ast}}\left(\frac{t gk^{\ast}}{2\pi} \log x_r\right)}{\log x_r} +  O\left(\exp\left(-(\log y)^{\frac{\alpha}{4}}\right)\right),
\end{equation} 
since $\phi(m) R\ll (\log y)^3,$ and $z=\exp\left((\log y)^{\alpha}\right).$ 

Recall that for $n\ge 3$, $F_n(u)= \cos(2\pi\{u\}/n) + \tan(\pi/n)\sin(2\pi \{u\}/n)$ is bounded, periodic with period $1$, and continuous on $\mathbb{R}$ (since $\lim_{u\to 1^{-}} F_n(u)=F_n(0)$). Moreover, $F_n$ is continuously differentiable on the interval $(0,1)$, and $F_n'(u) =O(1/n)$ uniformly in $u\in \mathbb{R}\setminus\mathbb{Z}$. It follows from  these facts, together with the mean value theorem, that $|F_n(a)-F_n(b)|=O(|a-b|/n)$ for all $a, b\in \mathbb{R}$ such that $|a-b|<1$, where the constant in the $O$ is absolute. This shows that for all $u\in [\log x_r, \log x_{r+1})$ we have
$$ F_{gk^{\ast}}\left(\frac{t gk^{\ast}}{2\pi} u\right)= F_{gk^{\ast}}\left(\frac{t gk^{\ast}}{2\pi} \log x_r\right) +O\big(\delta|t|\big).$$
Furthermore, we note that
$$\int_{\log x_r}^{\log x_{r+1}}\frac{du}{u}=\big(1+O(\delta)\big)\frac{\delta}{\log x_r}.$$
Combining these two facts, we obtain
\begin{align*}
&\frac{\delta}{\log x_r}F_{gk^{\ast}}\left(\frac{t gk^{\ast}}{2\pi} \log x_r\right)
=\big(1+O(\delta)\big)\int_{\log x_r}^{\log x_{r+1}}F_{gk^{\ast}}\left(\frac{t gk^{\ast}}{2\pi} \log x_r\right)\frac{du}{u}\\
&= \big(1+O(\delta)\big)\int_{\log x_r}^{\log x_{r+1}}F_{gk^{\ast}}\left(\frac{t gk^{\ast}}{2\pi} u\right)\frac{du}{u} +O\left(\delta |t| \int_{\log x_r}^{\log x_{r+1}}\frac{du}{u} \right).
\end{align*}
Summing over $0\leq r\leq R-1$, we get
$$
 \delta\sum_{0\leq r\leq R-1}\frac{F_{gk^{\ast}}\left(\frac{t gk^{\ast}}{2\pi} \log x_r\right)}{\log x_r}
 = \big(1+O(\delta)\big)\int_{\log z}^{\log y}F_{gk^{\ast}}\left(\frac{t gk^{\ast}}{2\pi} u\right)\frac{du}{u} +O\left(\delta\right),
$$
since $\int_{\log x_R}^{\log y} dt/t\ll \delta$. 

We now estimate the integral in the main term above. One can easily check that for $n\geq 3$, $F_n$ is an even function. Making the change of variable $v := \frac{gk^{\ast}|t|}{2\pi} u$, and setting $A := \frac{gk^{\ast}|t|}{2\pi} \log z$ and $B := \frac{gk^{\ast}|t|}{2\pi} \log y$, we get
\begin{equation*}
\int_{\log z}^{\log y} F_{gk^{\ast}}\left(\frac{t gk^{\ast}}{2\pi} u\right)\frac{du}{u} = \int_A^B F_{gk^{\ast}}(v) \frac{dv}{v}\leq \frac{gk^{\ast}}{\pi} \tan\left(\frac{\pi}{gk^{\ast}}\right)\log\left(\frac{\log y}{\log z}\right)+ O(1),
\end{equation*}
by Lemma \ref{IntegralFN}. Combining the above estimates with \eqref{SPLIT} and \eqref{SPLIT2} we obtain
$$ S\leq \big(1+O(\delta)\big) \frac{g}{\pi} \sin\left(\frac{\pi}{g}\right)\log\left(\frac{\log y}{\log z}\right)+ O(1)\leq (1-\delta_g)\log\left(\frac{\log y}{\log z}\right)+O(\delta\log_2 y).$$
Choosing $\delta=(\log_2y)^{-1}$ completes the proof of Proposition \ref{MEDIUMPRIMES}. Proposition \ref{MINDIST} follows upon combining this result with \eqref{ProUPPER2}.
\end{proof}

\subsection{Estimating $\mc{M}(\chi\bar{\psi}; y, T)$ for large twists $T$} In this subsection, we prove the following result which implies Proposition \ref{AD}. 
\begin{pro} \label{AD2}
Assume GRH.  Let $g\geq 3$ be a fixed odd integer. Let $N$ be large and $y\leq (\log N)/10$. Let $\psi$ be an odd primitive character of conductor $m$ such that  $\exp\left(2\sqrt{\log_3 y}\right) \leq m\leq \exp\left(\sqrt{\log y}\right)$. Then, there exist at least $\sqrt{N}$  primitive characters $\chi$ of order $g$ and conductor $q\leq N$ such that 
\begin{equation*}
\mb{D}^2(\chi\bar{\psi}, n^i;y) = \delta_g \log_2 y + O\left(\log_2 m\right).
\end{equation*}
\end{pro}
\begin{proof}
We follow the proof of Proposition \ref{MEDIUMPRIMES} in such a way that we achieve equality in all steps. Since the arguments here are similar to those in that proof, we omit some of the details. \\
Let $z := \exp\left((\log m)^2\right)$ and $y \geq z$. Let $\delta>0$ be a small parameter to be chosen and put $R := \left\lfloor \log(y/z)/\log(1+\delta)\right \rfloor$ as before. Set $x_0=z$ and $x_r := (1+\delta)^r x_0$. Then, as $\sum_{p \leq z} \frac{1}{p} \ll \log_2 m$, it suffices to find at least $\sqrt{N}$ primitive characters $\chi$ of order $g$ and conductor $q\leq N$ such that 
\begin{equation} \label{REST}
\sum_{z < p \leq y} \frac{\text{Re}(\chi(p)\bar{\psi}(p)p^{-i})}{p} = (1-\delta_g) \log(\log y/\log z) + O(1).
\end{equation}
Let $\theta_r := -\frac{\log x_r}{2\pi}$, for each $0 \leq r \leq R-1$. As in the proof of Proposition \ref{MEDIUMPRIMES}, when $x_r < p \leq x_{r+1}$ we approximate $p^i$ by $x_r^i$, for each $0 \leq r \leq R-1$. Let $k$ be the order of $\psi$, and for each $r$ let $\{z_{r,\ell}\}_{\ell} \in (\mu_g \cup \{0\})^k$ be chosen so as to maximize the sum
\begin{equation*}
\sum_{\ell \bmod k} \text{Re}\left(z_{r,\ell} \cdot e\left(\theta_r-\frac{\ell}{k}\right)\right)\sum_{a \bmod m \atop \psi(a) =e(\ell/k)} \sum_{x_r < p \leq x_{r+1} \atop p \equiv a \bmod m} \frac{1}{p}.
\end{equation*}
By Lemma \ref{VEC} there are at least $\sqrt{N}$ primitive characters $\chi$ of order $g$ and conductor $q\leq N$ such that $\chi(p) = z_{r,\ell}$ whenever  $x_r < p \leq x_{r+1}$, $\psi(p) = e\left(\ell/k\right)$ and $p\nmid g$. For such characters, it follows that
\begin{align*}
&\sum_{z < p \leq y} \frac{\text{Re}(\chi(p)\bar{\psi}(p)p^{-i})}{p} = \sum_{0 \leq r \leq R-1} \sum_{x_r < p \leq x_{r+1}} \frac{\text{Re}(\chi(p)\bar{\psi(p)} x_r^{-i})}{p} + O\left(\delta\right) \\
&= \sum_{0 \leq r \leq R-1}\sum_{\ell \bmod k} \text{Re}\left(z_{r,\ell} \cdot e\left(\theta_r-\frac{\ell}{k}\right)\right)\sum_{a \bmod m \atop \psi(a) =e(\ell/k)} \sum_{x_r < p \leq x_{r+1} \atop p \equiv a \bmod m} \frac{1}{p} + O_{g}\left(1\right).
\end{align*}
Let 
$$S_r:= \sum_{\ell \bmod k} \text{Re}\left(z_{r,\ell} \cdot\left(\theta_r-\frac{\ell}{k}\right)\right)\sum_{a \bmod m \atop \psi(a) =e(\ell/k)} \sum_{x_r < p \leq x_{r+1} \atop p \equiv a \bmod m} \frac{1}{p}.$$
To estimate the inner sum, we use the following asymptotic formula, which is valid under the assumption of GRH:
\begin{equation*}
\sum_{x_r < p \leq x_{r+1} \atop p \equiv a \bmod m} \log p = \frac{x_{r+1}-x_r}{\phi(m)} + O\left(x_{r}^{1/2}\log^2 x_r\right).
\end{equation*}
This yields
\begin{equation*}
\sum_{x_r < p \leq x_{r+1} \atop p \equiv a \bmod m} \frac{1}{p} = \frac{\delta}{\phi(m) \log x_r} \left(1+O\left(\delta\right)\right) + O\left(x_{r}^{-2/5}\right).
\end{equation*}
Using this estimate and proceeding exactly as in the proof of 
Proposition \ref{MEDIUMPRIMES}, we obtain that 
$$
\sum_{0 \leq r \leq R-1} S_r =\left(1+O\left(\delta\right)\right)\frac{\sin(\pi/g)}{k^{\ast}\tan(\pi/gk^{\ast})}\int_{\log z}^{\log y} \frac{F_{gk^{\ast}}\left(\frac{gk^{\ast}}{2\pi} u\right)}{u} du + O\left(\mathcal{E} \right),
$$
where 
$$
\mathcal{E}\ll \delta + z^{-2/5} \sum_{0 \leq r \leq R-1} (1+\delta)^{-2r/5} \ll \delta  + z^{-2/5}\delta^{-1}.
$$
Here, note that if we transform the integral as we did in the proof of Proposition \ref{MEDIUMPRIMES}, i.e., with $v := \frac{gk^{\ast} u}{2\pi}$ then the bounds of integration, $A := \frac{gk^{\ast} \log z}{2\pi}$ and $B:= \frac{gk^{\ast} \log y}{2\pi}$ are both larger than $1$. Thus, applying Lemma \ref{IntegralFN}, we get
\begin{align*}
\int_{\log z}^{\log y} \frac{F_{gk^{\ast}}\left(\frac{gk^{\ast}}{2\pi} u\right)}{u} du &= \int_A^B \frac{F_{gk^{\ast}}(v)}{v} dv = \int_1^B \frac{F_{gk^{\ast}}(v)}{v} dv - \int_1^A \frac{F_{gk^{\ast}}(v)}{v} dv \\
&= \frac{gk^{\ast}}{\pi}\tan(\pi/gk^{\ast}) \log(B/A) + O(1).
\end{align*}
Inserting this into our estimate for $\sum_r S_r$, we get
\begin{equation*}
\sum_{0 \leq r \leq R-1} S_r = (1-\delta_g)\log(\log y/\log z) + O\left(1+\delta \log_2 y + z^{-\frac{2}{5}} \delta^{-1} \right).
\end{equation*}
Choosing $\delta = (\log_2 y)^{-1}$ as before, and noting that  $z\geq (\log_2 y)^4$ yields \eqref{REST} for $y$ sufficiently large. This completes the proof of Proposition \ref{AD2}. Proposition \ref{AD} follows as well. 
\end{proof}


%%%%%%%%%%%%%%%%%%%%%%%%%%%%%%%%%%%%%%%%%%%%%%%%%%%%%%%%%%%%%%%%%

\section{Logarithmic mean values of completely multiplicative functions: proof of Theorem \ref{LogarithmicMean}}
The key ingredient to the proof of Theorem \ref{LogarithmicMean} is the following generalization of Theorem 2 of  \cite{MV}. 
\begin{thm}\label{MontgomeryVaughan}
Let $f\in \mathcal{F}$ and $x\geq 2$. Then, for any $0<T\leq 1$ we have
$$\sum_{n\leq x} \frac{f(n)}{n}\ll \frac{1}{\log x} \int_{1/\log x}^1 \frac{H_{T}(\alpha)}{\alpha} d\alpha,
$$
where 
$$H_{T}(\alpha)=\left(\sum_{k=-\infty}^{\infty} \max_{s\in \mathcal{A}_{k, T}(\alpha)} \left|\frac{F(1+s)}{s}\right|^2\right)^{1/2}.$$
and 
$$\mathcal{A}_{k,T}(\alpha)=\{s=\sigma+it: \alpha\leq \sigma\leq 1, |t-kT|\leq T/2\}.
$$
\end{thm}
Montgomery and Vaughan \cite{MV} established this result for $T=1$, and a straightforward generalization of their proof allows one to obtain Theorem \ref{MontgomeryVaughan} for any $0<T\leq 1$. For the sake of completeness we will include a full sketch of the necessary modifications to obtain this result. The only different treatment occurs when bounding the integrals on the left hand side of \eqref{TheModificationMV} below.  

\begin{lem}\label{MVModification}
Let $0<\alpha, T\leq 1$. Then we have
\begin{equation}\label{TheModificationMV}
\int_{-\infty}^{\infty}\left|\frac{F'(1+\alpha+it)}{\alpha+it}\right|^2 dt+ \int_{-\infty}^{\infty}\left|\frac{F(1+\alpha+it)}{(\alpha+it)^2}\right|^2 dt \ll \frac{H_{T}(\alpha)^2}{\alpha}.
\end{equation}
\end{lem}
\begin{proof}
First, we have
\begin{align*}
\int_{-\infty}^{\infty}\left|\frac{F'(1+\alpha+it)}{\alpha+it}\right|^2 dt
&=\sum_{k=-\infty}^{\infty}\int_{kT-T/2}^{kT+T/2}\left|\frac{F'(1+\alpha+it)}{\alpha+it}\right|^2dt\\
&\leq \sum_{k=-\infty}^{\infty}\max_{|t-kT|\leq T/2}\left|\frac{F(1+\alpha+it)}{\alpha+it}\right|^2 \int_{kT-T/2}^{kT+T/2}\left|\frac{F'(1+\alpha+it)}{F(1+\alpha+it}\right|^2dt.
\end{align*}
To bound the integral on the right hand side of this inequality, we appeal to a result of Montgomery (see Lemma 6.1 of \cite{Te}) which states that if $\sum_{n\geq 1}a_n n^{-s}$ and $\sum_{n\geq 1}b_n n^{-s}$ are two Dirichlet series which are absolutely convergent for $\re(s)>1$ and satisfy $|a_n|\leq b_n$ for all $n\geq 1$, then we have 
\begin{equation}\label{MontgomeryInequality}
\int_{-u}^{u}\left|\sum_{n=1}^{\infty}\frac{a_n}{n^{\sigma+it}}\right|^2dt \leq 3 \int_{-u}^{u}\left|\sum_{n=1}^{\infty}\frac{b_n}{n^{\sigma+it}}\right|^2 dt,
\end{equation}
for any real numbers $u\geq 0$ and $\sigma>1$. This implies that
\begin{align*}
\int_{kT-T/2}^{kT+T/2}\left|\frac{F'(1+\alpha+it)}{F(1+\alpha+it)}\right|^2dt 
&=\int_{-T/2}^{T/2} \left| \sum_{n=1}^{\infty} \frac{\Lambda(n) f(n)}{n^{1+\alpha+ikT+it}}\right|^2dt 
\ll \int_{-T/2}^{T/2} \left|\frac{\zeta'(1+\alpha+it)}{\zeta(1+\alpha+it)}\right|^2 dt\\
& 
\ll \int_{-T/2}^{T/2} \frac{1}{|\alpha+it|^2}dt\leq \int_{-\infty}^{\infty} \frac{1}{\alpha^2+t^2}dt \ll \frac{1}{\alpha}.
\end{align*}
Hence, we deduce that 
$$ \int_{-\infty}^{\infty}\left|\frac{F'(1+\alpha+it)}{\alpha+it}\right|^2 dt \ll \frac{H_{T}(\alpha)^2}{\alpha}.$$
To complete the proof, note that
$$\int_{-\infty}^{\infty}\left|\frac{F(1+\alpha+it)}{(\alpha+it)^2}\right|^2 dt \leq \sum_{k=-\infty}^{\infty}\max_{|t-kT|\leq T/2}\left|\frac{F(1+\alpha+it)}{\alpha+it}\right|^2 \int_{kT-T/2}^{kT+T/2}\frac{1}{|\alpha+it|^2}dt \ll \frac{H_{T}(\alpha)^2}{\alpha}.$$
\end{proof}
\begin{proof}[Proof of Theorem \ref{MontgomeryVaughan}]
Let
$$S(x)=\sum_{n\leq x} \frac{f(n)}{n}.$$
From the Euler product, $|F(2)|>0$, so $H_{T}(\alpha)\gg 1$. Thus, it is enough to prove the statement for $x\geq x_0$, where $x_0$ is a suitably large constant. Moreover, observe that $\int_{1/\log x}^1H_{T}(\alpha)\alpha^{-1} d\alpha$ is strictly increasing as a function of $x$, and $|S(x)\log x|$ is strictly increasing for $x\in [n, n+1)$, for all $n\geq 1$. Hence it is enough to prove the result for $x\in \mathcal{B}$ where
$$\mathcal{B}=\{ x\geq x_0: |S(y)\log y| < |S(x)\log x| \text{ for all } y<x\}.$$
Montgomery and Vaughan proved that for $x\in \mathcal{B}$ we have (see equations (7) and (8) of \cite{MV})
$$
|S(x)|\log x \ll \int_e^x \frac{|S(u)|}{u}du+ \frac{1}{\log x} \left|\sum_{n\leq x} \frac{f(n)}{n}(\log n)\log\left(\frac xn\right)\right| +  \frac{1}{\log x} \left|\sum_{n\leq x} \frac{f(n)}{n}\log^2\left(\frac xn\right)\right|.
$$
Integrating the first integral by parts, we get
\begin{equation}\label{IntegrationPartS} \int_e^x \frac{|S(u)|}{u}du \ll \frac{J(x)}{\log x} +\int_{e}^x \frac{J(u)}{u(\log u)^2} du,
\end{equation}
where
$$J(u):= \int_{e}^u \frac{|S(t)|\log t}{t} dt \ll (\log u)^{1/2} \left(\int_e^u \frac{|S(t)|^2(\log t)^2}{t} dt\right)^{1/2},
$$
by the Cauchy-Schwarz inequality. Using Parseval's Theorem, Montgomery and Vaughan proved that (see equation (14) of \cite{MV}) 
$$\int_e^u \frac{|S(t)|^2(\log t)^2}{t} dt \ll \int_{-\infty}^{\infty}\left|\frac{F'(1+\beta+it)}{\beta+it}\right|^2 dt+ \int_{-\infty}^{\infty}\left|\frac{F(1+\beta+it)}{(\beta+it)^2}\right|^2 dt,$$
where $\beta=2/\log u$. Appealing to Lemma \ref{MVModification} and making the change of variable $\alpha=1/\log u$ in the integral of the right hand side of \eqref{IntegrationPartS} we deduce that
\begin{equation}\label{OneIntegralTerm}
\int_e^x \frac{|S(u)|}{u}du \ll H_{T}\left(\frac{2}{\log x}\right)+ \int_{1/\log x}^1\frac{H_{T}(2\alpha)}{\alpha} d\alpha.
\end{equation}
Since $H_{T}(\alpha)$ is decreasing as a function of $\alpha$, we have
\begin{equation}\label{OneIntegralTerm2}
H_{T}\left(\frac{2}{\log x}\right)\ll \int_{1/\log x}^{2/\log x} \frac{H_{T}(\alpha)}{\alpha} d\alpha \leq \int_{1/\log x}^1\frac{H_{T}(\alpha)}{\alpha} d\alpha.
\end{equation}
Combining \eqref{OneIntegralTerm} and \eqref{OneIntegralTerm2} we get
$$\int_e^x \frac{|S(u)|}{u}du\ll \int_{1/\log x}^1\frac{H_{T}(\alpha)}{\alpha} d\alpha.$$
Furthermore, Montgomery and Vaughan proved that (see pages 207-208 of \cite{MV})
$$ \sum_{n\leq x} \frac{f(n)}{n}(\log n)\log\left(\frac xn\right) \ll \left(\frac{1}{\beta} \int_{-\infty}^{\infty}\left|\frac{F'(1+\beta+it)}{\beta+it}\right|^2 dt\right)^{1/2}
$$
and 
$$  \sum_{n\leq x} \frac{f(n)}{n}\log^2\left(\frac xn\right)\ll \left(\frac{1}{\beta} \int_{-\infty}^{\infty}\left|\frac{F(1+\beta+it)}{(\beta+it)^2}\right|^2 dt\right)^{1/2},
$$
where $\beta=2/\log x$. Combining these bounds with Lemma \ref{MVModification} and equation \eqref{OneIntegralTerm2} completes the proof.
\end{proof}
In order to derive Theorem \ref{LogarithmicMean} from Theorem \ref{MontgomeryVaughan}, we need to bound $H_{T}(\alpha)$, and hence to bound $|F(1+s)|$ for $\re(s)\geq \alpha$. 
Tenenbaum (see Section III.4 of \cite{Te}) proved that for all $y, T\geq 2$, and $1/\log y\leq \alpha\leq 1$, we have 
\begin{equation}\label{Tenenbaum}
\max_{|t|\leq T}|F(1+\alpha+it)|\ll (\log y) \exp\big(-\mathcal{M}(f; y, T)\big).
\end{equation}
However, this bound does not hold for all $T>0$ and $1/\log y\leq \alpha\leq 1$. Indeed, taking $f$ to be the M\"obius function $\mu$, $\alpha=1/2$,  $y$ large and $T=1/\log y$ shows that $\max_{|t|\leq T}|F(1+\alpha+it)|\geq |\zeta(3/2)|^{-1}$, while
$$ \mathcal{M}(f; y, T)= \min_{|t|\leq 1/\log y} \sum_{p\leq y} \frac{1+ \re(p^{-it})}{p}=
2\sum_{p\leq y} \frac{1}{p}+O(1)=2\log\log y+O(1),$$ 
and hence the right side of \eqref{Tenenbaum} is $\ll 1/(\log y)$. Nevertheless, using Tenenbaum's ideas, we show that \eqref{Tenenbaum} is valid whenever $T\geq \alpha$. 

\begin{lem}\label{MaxFDistance}
Let $y\geq 2$ and $f \in \mathcal{F}$ such that $f(p)=0$ for $p>y$. Let $F(s)$ be its corresponding Dirichlet series.  Then, for all real numbers $0< \alpha\leq 1$ and $T\geq \alpha$ we have 
$$
\max_{|t|\leq T} \big|F(1+\alpha+it)|\ll (\log y)\exp\big(-\mathcal{M}(f; y, T)\big).
$$
\end{lem}
\begin{proof} Note that
\begin{equation}\label{MFYT}
\mathcal{M}(f; y, T)= \log_2 y- \max_{|t|\leq T} \re \sum_{p\leq y} \frac{f(p)}{p^{1+it}}+O(1).
\end{equation}
We first remark that the result is trivial if $\alpha\leq 1/\log y$, since in this case we have 
$$
\log |F(1+\alpha+it)|= \re \sum_{p\leq y} \frac{f(p)}{p^{1+\alpha+it}}+O(1)= \re \sum_{p\leq y} \frac{f(p)}{p^{1+it}}+O(1),
$$
which follows from the fact that $|p^{\alpha}-1|\ll \alpha\log p$.\\
Now, suppose that $\alpha\geq 1/\log y$ and put $A=\exp(1/\alpha)$. Then we have
$$ \log |F(1+\alpha+it)| =\re \sum_{p\leq y} \frac{f(p)}{p^{1+\alpha+it}}+O(1)= \re \sum_{p\leq A} \frac{f(p)}{p^{1+\alpha+it}}+O(1)=\re \sum_{p\leq A} \frac{f(p)}{p^{1+it}}+O(1),$$
since  $\sum_{p>A} p^{-1-\alpha}\ll 1$ by the prime number theorem. Furthermore, for any $|\beta|\leq \alpha/2$ we have 
$$\sum_{p\leq A} \frac{f(p)}{p^{1+i(t+\beta)}}=\sum_{p\leq A} \frac{f(p)}{p^{1+it}}+O(1),$$
and hence
$$ \max_{|t|\leq T} \big|F(1+\alpha+it)|\ll \max_{|t|\leq (T-\alpha/2)} \exp\left(\re\sum_{p\leq A} \frac{f(p)}{p^{1+it}}\right).
$$
Now, let $|t|\leq T-\alpha/2$ be a real number. Then, we have
\begin{align*}
\int_{t-\alpha/2}^{t+\alpha/2}  \re\left(\sum_{p\leq y} \frac{f(p)}{p^{1+iu}}\right) du
&= \re \sum_{p\leq y}\frac{f(p)}{p^{1+it}}\left(\frac{p^{i\alpha/2}-p^{-i\alpha/2}}{i\log p}\right)\\
&=\alpha \left(\re\sum_{p\leq A} \frac{f(p)}{p^{1+it}}\right) + O\left(\alpha + \sum_{p>A} \frac{1}{p\log p}\right).
\end{align*}
Since $\sum_{p>A} (p\log p)^{-1}\ll \alpha$ by the prime number theorem, we deduce that
$$\re\sum_{p\leq A}\frac{f(p)}{p^{1+it}}=\frac{1}{\alpha} \int_{t-\alpha/2}^{t+\alpha/2} \re\left(\sum_{p\leq y} \frac{f(p)}{p^{1+iu}}\right) du +O(1)\leq \max_{|t|\leq T} \re \sum_{p\leq y} \frac{f(p)}{p^{1+it}}+O(1).$$
Appealing to \eqref{MFYT} completes the proof.
\end{proof}

%\begin{lem}\label{BoundH}
%Let $\delta> 0$.  Let $f \in \mathcal{F}$ be such that $f(p)=0$ for $p>y$, and $F(s)$ be its corresponding Dirichlet series. Then we have  
%$$ \alpha \cdot M_{\delta}(\alpha)\ll \begin{cases} (\alpha\delta)^{-1/2} & \text{ if } \alpha >\delta,\\ \delta^{-1}+ \log y \cdot \exp\left(-\mathcal{M}(f; y, \delta\log_2y)\right) & \text{ if } 0<\alpha \leq  \delta.
 %\end{cases}
% $$
%\end{lem}
We finish this section by proving a slightly stronger form of Theorem \ref{LogarithmicMean}, which we shall need to prove Theorems \ref{MCHIUP1} and \ref{MCHIUP2}. One can also show that the following result follows from Theorem \ref{LogarithmicMean}, so it is in fact equivalent to it. 
\begin{thm}\label{LogarithmicMean2}
Let $f\in \mathcal{F}$ and $x, y\geq 2$ be real numbers. Then, for any real number $0< T\leq 1$ we have
$$\sum_{\substack{n\leq x\\ n\in \mathcal{S}(y)}}\frac{f(n)}{n}\ll (\log y) \cdot \exp\big(-\mc{M}(f;y,T)\big)+\frac{1}{T},
$$
where the implicit constant is absolute, and $\mc{S}(y)$ is the set of $y$-friable numbers.
\end{thm}

\begin{proof}
First, observe that the result is trivial if $T\leq 1/\log x$, since we have in this case
$$\sum_{\substack{n\leq x\\ n\in \mathcal{S}(y)}}\frac{f(n)}{n}\ll \sum_{n\leq x}\frac{1}{n}\ll \log x\ll \frac{1}{T}.$$
Now assume that $1/\log x<T\leq 1$. Let $g$ be the completely multiplicative function such that $g(p)=f(p)$ for $p\leq y$ and $g(p) = 0$ otherwise, and let $G$ be its corresponding Dirichlet series. Then, it follows from Theorem \ref{MontgomeryVaughan} that
\begin{equation}\label{SmoothFG}
\sum_{\substack{n\leq x\\ n\in \mathcal{S}(y)}} \frac{f(n)}{n}=
\sum_{n\leq x}\frac{g(n)}{n}\ll \frac{1}{\log x}\int_{1/\log x}^1 \frac{H_{T}(\alpha)}{\alpha} d\alpha, 
\end{equation}
where 
$$
H_{T}(\alpha)=\left(\sum_{k=-\infty}^{\infty} \max_{s\in \mathcal{A}_{k, T}(\alpha)} \left|\frac{G(1+s)}{s}\right|^2\right)^{1/2}.
$$
First, observe that if $|t-kT|\leq T/2$ and $k\neq 0$ then $|t|\asymp |k|T$.  Moreover, uniformly for all $t\in \mathbb{R}$, we have 
\begin{equation}\label{TrivialBoundF}
|G(1+\sigma+it)|\leq \zeta(1+\sigma)\ll \frac{1}{\sigma}.
\end{equation}
We will first bound $H_{T}(\alpha)$ when $\alpha>T$.  Using \eqref{TrivialBoundF} we obtain in this case
\begin{equation}\label{BigAlpha}
\alpha^2 \cdot H_{T}(\alpha)^2\ll \sum_{k=-\infty}^{\infty} \max_{|t-kT|\leq T/2}\frac{1}{\alpha^2 +t^2}\ll \sum_{|k|> \alpha/T}\frac{1}{k^2T^2}+ \sum_{|k|\leq \alpha/T} \frac{1}{\alpha^2}\ll \frac{1}{\alpha T}.
\end{equation}
Now, suppose that $0<\alpha\leq T$. To bound $H_{T}(\alpha)$ in this case, we first use \eqref{TrivialBoundF} for $|k|\geq 1$. This gives
$$
H_{T}(\alpha)^2 
\ll \frac{1}{\alpha^2}\sum_{|k|\geq 1} \frac{1}{k^2T^2}+\frac{1}{\alpha^2}\max_{s\in \mathcal{A}_{0, T}(\alpha)} |G(1+s)|^2\ll \frac{1}{(\alpha T)^2}+ \frac{1}{\alpha^2}\max_{s\in \mathcal{A}_{0, T}(\alpha)} |G(1+s)|^2.
$$
Furthermore, by  \eqref{TrivialBoundF} and Lemma \ref{MaxFDistance} we have
\begin{align*}
\max_{s\in \mathcal{A}_{0, T}(\alpha)} |G(1+s)|
&\ll \max_{\substack{|t|\leq T\\ \sigma\geq T}} |G(1+\sigma+it)|+ \max_{\substack{|t|\leq T\\ \alpha \leq \sigma\leq T}} |G(1+\sigma+it)|\\
&\ll \frac{1}{T}+ (\log y)\exp\big(-\mathcal{M}(g; y, T)\big).
\end{align*}
Since $\mathcal{M}(g; y, T)=\mathcal{M}(f; y, T)$ we deduce that for $0<\alpha\leq T$ we have
\begin{equation}\label{SmallAlpha}
H_{T}(\alpha)^2\ll \frac{1}{(\alpha T)^2}+\frac{(\log y)^2}{\alpha^2}\exp\left(-2\mathcal{M}(f; y, T)\right).
\end{equation}
Using \eqref{BigAlpha} when $T<\alpha\leq 1$ and \eqref{SmallAlpha} when $1/\log x\leq \alpha\leq T$ we get
\begin{align*}
\int_{1/\log x}^1 \frac{H_{T}(\alpha)}{\alpha} d\alpha 
&\ll \left(\frac{1}{T}+ \log y \cdot \exp\left(-\mathcal{M}(f; y, T)\right)\right)\int_{1/\log x}^{T} \frac{1}{\alpha^2}d\alpha+  \frac{1}{T^{1/2}}\int_{T}^1\frac{1}{\alpha^{5/2}}d\alpha\\
&\ll \frac{\log x}{T}+ (\log x)(\log y) \exp\left(-\mathcal{M}(f; y, T)\right). 
\end{align*}
Inserting this bound in \eqref{SmoothFG} yields the result.

\end{proof}
%%%%%%%%%%%%%%%%%%%%%%%%%%%%%%%%%%%%%%%%%%%%%
\section{Proofs of Theorems \ref{COND}, \ref{MCHIUP1} and \ref{MCHIUP2}}
%Let us now trace the argument that allows us to apply Theorem \ref{LogarithmicMean} to prove Theorem \ref{COND}. 

To prove Theorem \ref{COND}, the general strategy we use is that of \cite{GrSo2} (with the refinements from \cite{GOLD}), and it will be clear where we shall make use of Theorem \ref{LogarithmicMean2}. We will consider the conditional (on GRH) and unconditional results simultaneously, setting $y := \log^{12} q$ if we are assuming GRH, and setting $y := q$ otherwise. We recall here that $y=Q$ in the unconditional case, and $y = Q^{12}$ on GRH, so that in all cases we have $\log y \asymp \log Q$.\\
 When $\chi$ is primitive and $\alpha \in \mb{R}$, we have
\begin{equation*}
\sum_{n \leq q} \frac{\chi(n)}{n}e(n\alpha) = \sum_{n \leq q \atop n \in \mc{S}(y)}\frac{\chi(n)}{n}e(n\alpha) + O(1);
\end{equation*}
on GRH, this follows from \eqref{APPROXFRIABLE}, and unconditionally this statement is trivial.
Inserting this estimate in P\'olya's Fourier expansion \eqref{Polya} gives
\begin{equation*}
M(\chi)\ll \sqrt{q}\left( \max_{\alpha \in [0,1]} \left|\sum_{1\leq |n| \leq q \atop n \in \mc{S}(y)} \frac{\chi(n)}{n}\big(1-e(n\alpha)\big)\right| + 1\right).
\end{equation*}
Therefore, to prove Theorem \ref{COND} it suffices to show that for all $\alpha\in [0, 1]$ we have 
\begin{equation}\label{BoundExponentialSum}
\sum_{1\leq |n| \leq q \atop n \in \mc{S}(y)}\frac{\chi(n)}{n}e\left(n\alpha\right) \ll \Big(1-\chi(-1)\xi(-1)\Big)\frac{\sqrt{m}}{\phi(m)}(\log Q ) e^{-\mc{M}(\chi\bar{\xi};Q, (\log Q)^{-7/11})}+ \left(\log Q\right)^{\frac{9}{11} + o(1)}.
\end{equation}
Let $\alpha\in [0, 1]$ and $R:= (\log Q)^{5}$. By Dirichlet's theorem on Diophantine approximation, there exists a rational approximation $|\alpha - b/r| \leq 1/rR$, with $1\leq r\leq R$ and $(b, r)=1$. 
Let $M := (\log Q)^{4/11}$. We shall distinguish between two cases. If $r\leq M$, we say that $\alpha$ lies on a \emph{major} arc, and if $M<r\leq R$ we say that $\alpha$ lies on a \emph{minor} arc. In the latter case, we shall use Corollary 2.2 of \cite{GOLD}, which is a consequence of the work of Montgomery and Vaughan \cite{MV2}. Indeed, this shows that
\begin{equation*}
\sum_{1\leq |n| \leq q \atop n \in \mc{S}(y)} \frac{\chi(n)}{n}e\left(n\alpha\right) \ll \frac{(\log M)^{5/2}}{\sqrt{M}}\log y +\log R+\log_2 y \ll \left(\log Q\right)^{\frac{9}{11} + o(1)}.
\end{equation*}
We now handle the more difficult case of $\alpha$ lying on a major arc. First, it follows from Lemma 4.1 of \cite{GOLD} (which is a refinement of Lemma 6.2 of \cite{GrSo2}) that for $N := \min\{q,|r\alpha - b|^{-1}\}$, we have
\begin{equation}\label{DIVSUM}
\begin{aligned}
\sum_{1\leq |n| \leq q \atop n \in \mc{S}(y)} \frac{\chi(n)}{n}e(n\alpha) &= \sum_{1\leq |n| \leq N \atop n \in \mc{S}(y)} \frac{\chi(n)}{n}e\left(\frac{nb}{r}\right) + O\left(\frac{(\log R)^{3/2}}{\sqrt{R}}(\log y)^2 +\log R+\log_2 y\right)\\
&= \sum_{1\leq |n| \leq N \atop n \in \mc{S}(y)} \frac{\chi(n)}{n}e\left(\frac{nb}{r}\right) + O\left(\log_2 Q\right).
\end{aligned}
\end{equation}
We first assume that $b\neq 0$. In this case we can use an identity of Granville and Soundararajan (see Proposition 2.3 of \cite{GOLD}) which asserts that
\begin{equation}\label{GrSoIdentity}
\begin{aligned}
&\sum_{1\leq |n| \leq N \atop n \in \mc{S}(y)} \frac{\chi(n)}{n}e\left(\frac{nb}{r}\right)\\
&=\Big(1-\chi(-1)\psi(-1)\Big)\sum_{d\mid r \atop d \in \mc{S}(y)} \frac{\chi(d)}{d}\cdot \frac{1}{\phi(r/d)} \sum_{\psi \bmod r/d}\tau(\psi) \bar{\psi}(b) 
\left(\sum_{n \leq N/d \atop n \in \mc{S}(y)} \frac{\chi(n)\bar{\psi}(n)}{n}\right).
\end{aligned}
\end{equation}
To bound the inner sum above, we appeal to Theorem \ref{LogarithmicMean2} with $T=(\log Q)^{-7/11}$. This implies that
$$
\sum_{n \leq N/d \atop n \in \mc{S}(y)} \frac{\chi(n)\bar{\psi}(n)}{n}\ll (\log y)\cdot \exp\left(-\mc{M}(\chi\bar{\psi};y, (\log Q)^{-7/11})\right)+(\log Q)^{7/11}.
$$
Moreover, in the conditional case  $y = Q^{12}$, and thus we have
$$ \mc{M}(\chi\bar{\psi};y, (\log Q)^{-7/11})\geq \mc{M}(\chi\bar{\psi};Q, (\log Q)^{-7/11}) +O(1).$$
Therefore, we get
\begin{equation}\label{Thm1.3}
\sum_{n \leq N/d \atop n \in \mc{S}(y)} \frac{\chi(n)\bar{\psi}(n)}{n}\ll (\log Q)\cdot \exp\left(-\mc{M}(\chi\bar{\psi};Q, (\log Q)^{-7/11})\right)+(\log Q)^{7/11}.
\end{equation}
We now order the primitive characters $\psi\pmod \ell$ for $\ell\leq M$ (including the trivial character $\psi$ which equals $1$ for all integers)  as $\{\psi_k\}_k$, where
\begin{equation*}
\mc{M}(\chi\bar{\psi_k};Q, (\log Q)^{-7/11}) \leq \mc{M}(\chi\bar{\psi_{k+1}};Q, (\log Q)^{-7/11}), 
\end{equation*}
for all $k \geq 1$. Note that $\psi_1=\xi$, in the notation of Theorem \ref{COND}. Furthermore, by a slight variation of Lemma 3.1 of \cite{BGS} we have  
$$
 \mc{M}\left(\chi\bar{\psi_k};Q, (\log Q)^{-7/11}\right) \geq 
\left(1-\frac{1}{\sqrt{k}}\right)\log_2 Q+O\left(\sqrt{\log_2 Q}\right).
$$
Therefore, if $\psi \pmod \ell$ is induced by $\psi_{k}$, then
\begin{equation}\label{Repulsive}
\begin{aligned}
\mc{M}\left(\chi\bar{\psi};Q, (\log Q)^{-7/11}\right) &\geq \mc{M}\left(\chi\bar{\psi_k};Q, (\log Q)^{-7/11}\right) +O\left(\sum_{p \mid \ell} \frac{1}{p}\right)\\
 &\geq 
\left(1-\frac{1}{\sqrt{k}}+o(1)\right)\log_2 Q,
\end{aligned}
\end{equation}
since $\sum_{p\mid \ell}1/p\ll \log_2 \ell\ll \log_3 Q$. 
Inserting this bound in \eqref{Thm1.3}, we deduce that the contribution of all characters $\psi$ that are induced by some $\psi_k$ with $k\geq 3$ to \eqref{GrSoIdentity} is
$$
\ll (\log Q)^{7/11}\sum_{d\mid r} \frac{1}{d\phi(r/d)} \sum_{\psi \bmod r/d}|\tau(\psi)|\ll  (\log Q)^{7/11}\sum_{d\mid r} \frac{\sqrt{r}}{d^{3/2}}\ll (\log Q)^{9/11},
$$
since $1/\sqrt{3}<7/11$, $|\tau(\psi)|\leq \sqrt{r/d}$, and $r\leq (\log Q)^{4/11}.$ Moreover, 
observe that there is at most one character $\psi \pmod{r/d}$ such that $\psi$ is induced by $\psi_2$. Using \eqref{Repulsive}, we deduce that the contribution of these characters to  \eqref{GrSoIdentity} is
$$ \ll (\log Q)^{1/\sqrt{2}+o(1)}\sum_{d\mid r} \frac{1}{d}\cdot\frac{\sqrt{r/d}}{\phi(r/d)} \ll (\log Q)^{1/\sqrt{2}+o(1)}\log r \ll(\log Q)^{1/\sqrt{2}+o(1)}. $$
Thus, it now remains to estimate the contribution of the characters $\psi\bmod r/d$ that are induced by $\xi$, recalling that $\xi$ has conductor $m$. If $m\nmid r$, there are no such characters $\psi$ and the theorem follows in this case. If $m\mid r$ and $\psi\bmod r/d$ is induced by $\xi$, then we must have $d \mid (r/m)$. Furthermore, by Lemma 4.1 of \cite{GrSo2} we have 
$$\tau(\psi)= \mu\left(\frac{r}{dm}\right)\xi\left(\frac{r}{dm}\right)\tau(\xi).$$
Therefore, the contribution of these characters to \eqref{GrSoIdentity} is
\begin{equation}\label{INDUCED}
\Big(1-\chi(-1)\xi(-1)\Big)\bar{\xi}(b)\tau(\xi)\sum_{d\mid (r/m) \atop d \in \mc{S}(y)} \frac{\chi(d)}{d}\cdot \frac{1}{\phi(r/d)}  \mu\left(\frac{r}{dm}\right)\xi\left(\frac{r}{dm}\right)
\sum_{\substack{n \leq N/d\\ (n, r/d)=1\\ n \in \mc{S}(y)}} \frac{\chi(n)\bar{\xi}(n)}{n}.
\end{equation}
%Let $m_k$ denote the conductor of $\psi_k$. Following the argument in Section 6 of \cite{GOLD}, for each $d | r$ we let $\mc{K}_d := \{k : m_k | r/d\}$. Thus, the set of characters modulo $r/d$ are induced by the collection $\{\psi_k\}_{k \in \mc{K}_d}$. Writing $\chi_0^{(s)}$ to denote the primitive character modulo $s$, we 
%\begin{equation*}
%a(d) = \sum_{k \in \mc{K}_d} \tau\left(\psi_k\chi_0^{(r/d)}\right)\bar{\psi_k}(b)\chi_0^{(r/d)}(b)\left(\sum_{n \leq N/d \atop n \in \mc{S}(Q), (n,r/d) = 1} \frac{\chi(n) \bar{\psi}_k(n)}{n}\right).
%\end{equation*}
%Note that by Theorem 9.10 in \cite{MVbook},
%\begin{equation*}
%\tau\left(\psi_k \chi_0^{(m_k)}\right) = \mu\left(\frac{r}{dm_k}\right)\chi_0^{(m_k)}\left(\frac{r}{dm_k}\right)\psi\left(\frac{r}{dm_k}\right)\tau(\psi_k),
%\end{equation*}
Furthermore, it follows from Lemma 4.4 of \cite{GrSo2} that
\begin{align*}
\sum_{\substack{n \leq N/d\\ (n, r/d)=1\\ n \in \mc{S}(y)}} \frac{\chi(n)\bar{\xi}(n)}{n} &= \sum_{\substack{n \leq N\\ (n, r/d)=1\\ n \in \mc{S}(y)}} \frac{\chi(n)\bar{\xi}(n)}{n} +O(\log d)\\
&= \prod_{p \mid  \frac rd} \left(1-\frac{\chi(p)\bar{\xi}(p)}{p}\right) \sum_{n \leq N \atop n \in \mc{S}(y)} \frac{\chi(n)\bar{\xi}(n)}{n} + O\left((\log_2 Q)^2\right).
\end{align*}
Thus, in view of Theorem \ref{LogarithmicMean2}, we deduce that \eqref{INDUCED} is 
\begin{equation}\label{INDUCED2}
\begin{aligned} 
\ll &\Big(1-\chi(-1)\xi(-1)\Big)\sqrt{m}\left((\log Q)e^{-\mc{M}\left(\chi\bar{\xi};Q, (\log Q)^{-7/11}\right)}+(\log Q)^{7/11}\right)\\
& \times \sum_{\substack{d\mid (r/m)\\ (r/(dm), m)=1}} \frac{1}{d\phi(r/d)}  \mu^2\left(\frac{r}{dm}\right) \prod_{p \mid \frac{r}{dm}} \left(1+\frac{1}{p}\right).
\end{aligned}
\end{equation}
Finally, by a change of variables $a= r/(md)$, we obtain
\begin{align*}
\sum_{\substack{d\mid (r/m)\\ (r/(dm), m)=1}} \frac{1}{d\phi(r/d)}  \mu^2\left(\frac{r}{dm}\right) \prod_{p \mid  \frac{r}{dm}} \left(1+\frac{1}{p}\right)
&= \frac{m}{r\phi(m)}\sum_{\substack{a\mid (r/m)\\ (a, m)=1}} \frac{a}{\phi(a)}  \mu^2\left(a\right) \prod_{p \mid  a} \left(1+\frac{1}{p}\right)\\
&\leq \frac{1}{\phi(m)}\cdot \frac{1}{r/m} \prod_{p \mid (r/m)} \left(1+\frac{p+1}{p-1}\right)\leq \frac{4}{\phi(m)}, 
\end{align*}
since $2p/(p-1)\leq p$ for all primes $p\geq 3$. Combining this bound with  \eqref{INDUCED2}, it follows that the contribution of the characters $\psi$ that are induced by $\xi$ to \eqref{GrSoIdentity} is
$$\ll \Big(1-\chi(-1)\xi(-1)\Big)\frac{\sqrt{m}}{\phi(m)}(\log Q) e^{-\mc{M}\left(\chi\bar{\xi};Q, (\log Q)^{-7/11}\right)}+(\log Q)^{7/11}. $$

It thus remains to consider when $b = 0$, and hence $r = 1$. First, if $\xi$ is identically $1$ (so $m=1$), then a trivial application of Theorem \ref{LogarithmicMean2} shows that in this case
\begin{equation*}
\sum_{1\leq |n| \leq N \atop n \in \mc{S}(y)} \frac{\chi(n)}{n} \ll \Big(1-\chi(-1)\Big)\frac{\sqrt{m}}{\phi(m)} (\log Q)e^{-\mc{M}\left(\chi; Q,(\log Q)^{-7/11}\right)} + (\log Q)^{7/11}.
\end{equation*}
On the other hand, if $\xi$ is not the trivial character, then it follows from \eqref{Repulsive} that
$$
\mc{M}(\chi; Q,(\log Q)^{-7/11}) \geq \left(1-\frac{1}{\sqrt{2}}+o(1)\right)\log_2 Q,
$$
and hence by Theorem \ref{LogarithmicMean2} we get
$$ 
\sum_{1\leq |n| \leq N \atop n \in \mc{S}(y)} \frac{\chi(n)}{n}\ll (\log Q)^{1/\sqrt{2}+o(1)},
$$
which completes the proof of \eqref{BoundExponentialSum}. Theorem \ref{COND} follows as well.

We end this section by deducing Theorems \ref{MCHIUP1} and \ref{MCHIUP2} from Theorem \ref{COND} and Proposition \ref{MINDIST}. We shall prove both results simultaneously, by setting $Q:=\log q$ on GRH and $Q:=q$ unconditionally. 
\begin{proof}[Proof of Theorems \ref{MCHIUP1} and \ref{MCHIUP2}]
Let  $\xi$ be the character of conductor $m \leq (\log Q)^{4/11}$ that minimizes $\mc{M}\left(\chi\bar{\psi}; Q,(\log Q)^{-7/11}\right)$. If $\xi$ is even, then it follows from Theorem \ref{COND} that
$$ M(\chi)\ll \sqrt{q} (\log Q)^{9/11+o(1)},$$
which trivially implies the result in this case since $1-\delta_g>9/11$, for all $g\geq 3$.
Now, suppose that $\xi$ is odd and let $k$ be its order. We also let $\beta=1$ if $m$ is an exceptional modulus, and $\beta=0$ otherwise. Then, combining Theorem \ref{COND} and Proposition \ref{MINDIST} (with $\alpha=7/11$) we obtain
\begin{equation}\label{FINALBOUND}
\begin{aligned}
M(\chi) &\ll \frac{\sqrt{qm}}{\phi(m)} (\log Q)^{1-\delta_g} \exp\left(-\frac{c_1(1-\delta_g)}{(gk^{\ast})^2}\log_2 Q+ \beta \e \log m+  O\left(\log_2 m\right)\right) \\
&\ll \sqrt{q}(\log Q)^{1-\delta_g} \exp\left(-\left(\frac{1}{2}-\beta \e\right) \log m- \frac{c_1(1-\delta_g)}{g^2m^2}\log_2 Q+ c_2 \log_2 m\right),
\end{aligned}
\end{equation}
for some  positive constants $c_1, c_2$, since $\phi(m)\gg m/\log_2 m$.  One can easily check that the expression inside the exponential is maximal when $m \asymp \sqrt{\log_2 Q}$, and its maximum equals 
$$ -\left(\frac{1}{4}-\frac{\beta \e}{2}\right) \log_3 Q+ O\left(\log_4 Q\right).$$
Inserting this estimate in \eqref{FINALBOUND} completes the proof.
%note that for $\phi(m_0) \gg \sqrt{\log_2 Q}$ the exponential is bounded, and the bound is decreasing with $m_0$; in the converse case the exponential has the dominant contribution). In this case, the upper bound in Theorem \ref{MCHIUP} holds. \\ On the other hand, when $\psi_0$ is exceptional there is an additional term $\e \log m_0$ in the exponential, which leads to the additional $\left(\log_2 Q\right)^{\e}$ unconditionally.
\end{proof}



\end{document} 

%It will be apparent from the proof of Proposition \ref{UPPER} that those $\chi$ for which $\mb{D}(\chi,\psi;\log q)$ is maximal occurs when $
%\begin{pro}
%Let $m \leq \log^A x$ be a fixed prime and $Q \leq x\log^{-C} x$, for $A,C > 0$. Then there is some $B=B(A,C) > 0$ such that for infinitely many $q \leq Q/m$,
%\begin{equation*}
%\frac{1}{\phi(m)} \sum_{\psi (m)} \sup_{t \leq q} \left(\sum_{n \leq t} \chi(n)\bar{\psi(n)}\Lambda(n)\right) \ll x^{\frac{1}{2}}\log^{B}(x).
%\end{equation*}
%\end{pro}
%\begin{proof}
%By Uchiyama's refinement of the Barban-Davenport-Halberstam theorem, we have
%\begin{equation*}
%Q^{-1}\sum_{q \leq Q} \frac{1}{\phi(q)}\sum_{\chi (q) \atop \chi \neq \chi_0} \sup_{t \leq x} \left|\sum_{n \leq t} \chi(n) \Lambda(n)\right|^2 \ll x\log^3 x.
%\end{equation*}
%Observe that $\gg Q/m$ of the moduli $q' \leq Q$ can be expressed as $qm$, where $(q,m) = 1$.  As such, we see that
%\begin{equation*}
%(Q/m)^{-1} \sum_{q \leq Q/m} \frac{1}{\phi(q)\phi(m)} \mathop{\sum_{\chi (q)} \sum_{\psi(m)}}_{\chi \bar{\psi} \text{ non-principal}} \sup_{t \leq x} \left|\sum_{n \leq t} \chi(n)\bar{\psi}(n) \Lambda(n)\right|^2 \ll x\log^{3+A} x.
%\end{equation*}
%Thus, by Markov's inequality it follows that for all but $Q/\log x$ of the moduli $q \leq Q$, we have 
%\begin{equation*}
%\frac{1}{\phi(q)\phi(m)} \mathop{\sum_{\chi (q)} \sum_{\psi(m)}}_{\chi \bar{\psi} \text{ non-principal}} \sup_{t \leq x} \left|\sum_{n \leq t} \chi(n)\bar{\psi}(n) \Lambda(n)\right|^2 \ll x\log^{4+A+C} x
%\end{equation*}
%Select some such $q \gg x/\log^{C+A} x$. Then we can choose some $\chi$ modulo $q$ for some such $q$ such that
%\begin{equation*}
%\frac{1}{\phi(m)} \sum_{\psi (m) \atop \chi\bar{\psi} \text{ non-principal}} \left(\sup_{t \leq x} \left|\sum_{n \leq t} \chi(n)\bar{\psi}(n)\Lambda(n)\right|\right)^2 \ll x\log^{4+A+C} x.
%\end{equation*}
%By Cauchy-Schwarz, this means that
%\begin{equation*}
%\frac{1}{\phi(m)} \sum_{\psi (m) \atop \chi\bar{\psi} \text{ non-principal}} \sup_{t \leq x} \left|\sum_{n \leq t} \chi(n)\bar{\psi}(n)\Lambda(n)\right| \ll x^{\frac{1}{2}}\log^{\frac{1}{2}(4+A+C)} x.
%\end{equation*}
%This completes the proof.
%\end{proof}
%\begin{cor}
%There are infinitely many $q$ such that for all $q^{\frac{1}{2}+\delta} < t \leq q$ we have
%\begin{equation*}
%\frac{1}{\phi(m)} \sum_{\psi (m)} \left|\sum_{n \leq t} \chi(p)\bar{\psi(p)}\right| \ll t^{1-\delta}\log^{O(1)} t.
%\end{equation*}
%\end{cor}
%\begin{proof}
%By the proof of the previous proposition we may select $q \gg x/\log^C x$ such that if $t > q^{\frac{1}{2}+\delta} \gg x^{\frac{1}{2}+\delta-\epsilon}$, we have
%\begin{equation*}
%\frac{1}{\phi(m)}\sum_{\psi (m) \atop \chi\bar{\psi} \text{ non-principal}} \sup_{t \leq q} \left|\sum_{n \leq t} \chi(n)\bar{\psi}(n)\Lambda(n)\right| \ll x^{\frac{1}{2}} \log^{O(1)} x \ll t^{1-\delta} \log^{O(1)} t
%\end{equation*}
%Moreover, we of course have
%\begin{equation*}
%\sum_{n \leq t} \chi(n)\bar{\psi(n)} \Lambda(n) = \sum_{p \leq t} \chi(p)\bar{\psi}(p) \log p + O(\sqrt{t} \log t),
%\end{equation*}
%so that by partial summation
%\begin{align*}
%&\sum_{p \leq t} \chi(p)\bar{\psi(p)} \ll \int_1^t \frac{du}{u\log^2 u} \left|\sum_{p \leq u} \chi(p) \bar{\psi(p)} \log p\right| + O(t^{1-\delta}\log^{-1} t) \\
%&\ll t^{1-\delta}\log^{-1} t + \int_{t^{\frac{1}{2}}}^{t^{1-\delta}} \frac{du}{\log^2 u} \ll t^{1-\delta}\log^{-1} t.
%\end{align*}
%This completes the proof.
%\end{proof}
%\begin{lem} \label{SMOOTH}
%Let $\delta > 0$, and $2 \leq y \leq q$. Let $\chi$ be a character modulo $q$ such that for $m$ prime, $m \leq x^{\delta/2}$,
%\begin{equation*}
%\frac{1}{\phi(m)} \sum_{\psi (m)} \left|\sum_{p \leq t} \chi(p)\bar{\psi}(p)\right| \ll t^{1-\delta}
%\end{equation*}
%for all $y < t \leq q$. Then for any $b$ modulo $m$,
%\begin{equation*}
%\sum_{n \leq q} \frac{\chi(n)}{n}e\left(\frac{bn}{m}\right) = \sum_{n \leq q \atop P^+(n) \leq y} \frac{\chi(n)}{n}e\left(\frac{bn}{m}\right) + O\left(\frac{\log q}{y^{\delta/2}}\right).
%\end{equation*}
%\end{lem}
%\begin{proof}
%We begin by showing that $\sum_{p \leq t} \chi(p)e\left(\frac{pc}{m}\right) \ll m^{\frac{1}{2}}t^{1-\delta}$ for each $t \leq q$ and each $(c,m) = 1$. To see this we observe that
%\begin{align*}
%\sum_{p \leq t} \chi(p)e\left(\frac{pc}{m}\right) &= \frac{1}{\phi(m)}\sum_{\psi (m)} \sum_{a(m)} e\left(\frac{ac}{m}\right) \bar{\psi}(a)\sum_{p \leq t \atop (p,m) = 1} \chi(p)\bar{\psi}(p) + O\left(\log m\right) \\
%&= \frac{1}{\phi(m)}\sum_{\psi (m)} \tau(\bar{\psi}) \sum_{p \leq t \atop (p,m) = 1} \chi(p)\bar{\psi}(p) + O\left(\log m\right).
%\end{align*}
%By assumption, we thus have
%\begin{equation*}
%\sum_{p \leq t} \chi(p)e\left(\frac{pc}{m}\right) \leq \frac{1}{\phi(m)}\sum_{\psi(m)} |\tau(\psi)| \left|\sum_{p \leq t \atop (p,m) = 1} \chi(p)\bar{\psi}(p)\right| \ll \sqrt{m}t^{1-\delta}.
%\end{equation*}
%Next, we show that
%\begin{equation*}
%\sum_{n \leq q} \chi(n) e\left(\frac{bn}{m}\right) = \sum_{n \leq q \atop P^+(n) \leq y} \chi(n)e\left(\frac{bn}{m}\right) + O\left(\frac{q}{y^{\delta/2}}\right).
%\end{equation*}
%The lemma then follows by partial summation. For the claim, we note that
%\begin{align*}
%&\sum_{n \leq q} \chi(n) e\left(\frac{bn}{m}\right) - \sum_{n \leq q \atop P^+(n) \leq y} \chi(n)e\left(\frac{bn}{m}\right) \\
%&= \sum_{k \leq q/y} \chi(k) \sum_{\max\{P^+(k) - 1, y\} < p \leq q/k} \chi(p)e\left(\frac{pkb}{m}\right) \ll  q^{1-\delta/2}\sum_{k \leq q/y} k^{-1+\delta/2} \\
%&\ll qy^{-\delta/2},
%\end{align*}
%as claimed.
%\end{proof}





%\\
%&= \log(\log y/\log x_0) - \text{Re}\sum_{l (k)} z_l e\left(-\frac{\beta \log p}{2\pi} -\frac{l}{k}\right) \sum_{a (m) \atop \psi(a) = e\left(\frac{l}{k}\right)} \sum_{x_0 < p \leq y \atop p \equiv a (m)} \frac{1}{p},
%\end{align*}
%where we have chosen $z_l$ such that $z_le\left(-
%Inserting a factor of $\zeta\left(1+\frac{1}{3\log_2 q}\right)$, we get
%\begin{equation*}

%\begin{align*}
%&\sum_{b (m)} \psi(b)\sum_{n \leq q \atop P^+(n) \leq \log^{2/\delta} q} \frac{\chi(n)}{n}e\left(\frac{bn}{m}\right) = \sum_{n \leq q \atop P^+(n) \leq \log^{2/\delta} q} \frac{\chi(n)}{n}\sum_{b(m)} \psi(n) e\left(\frac{bn}{m}\right) \\
%&= \tau(\psi)\sum_{n \leq q \atop P^+(n) \leq \log^{2/\delta} q} \frac{\chi(n)\bar{\psi}(n)}{n}.
%\end{align*}
%Thus, choosing $b$ such that $\left|\sum_{n \leq q \atop P^+(n) \leq \log^{2/\delta} q} \frac{\chi(n)}{n}e\left(\frac{bn}{m}\right)\right|$ exceeds the average (over $b$ modulo $m$), we have
%\begin{equation*}
%M(\chi) + \sqrt{q} \gg \frac{\sqrt{qm}}{\phi(m)}\left(\left|\sum_{P^+(n) \leq \log^{2/\delta}q} \frac{\chi(n)\bar{\psi}(n)}{n} \right| - \left|\sum_{n > q \atop P^+(n) \leq \log^{2/\delta}q} \frac{\chi(n)\bar{\psi}(n)}{n}\right|\right).
%\end{equation*}
%Bounding the second sum trivially and using Rankin's trick, for $\eta > 0$ to be chosen we have
%\begin{align*}
%&\left|\sum_{n > q \atop P^+(n) \leq \log^{2/\delta}q} \frac{\chi(n)\bar{\psi}(n)}{n}\right| \leq q^{-\eta} \exp\left(\sum_{p \leq \log^{2/\delta} q} \frac{1}{p} - \sum_{p \leq \log^{2/\delta} q} \left(\frac{1}{p}-\frac{1}{p^{1-\eta}}\right)\right) \\
%&\ll_{\delta} q^{-\eta} \log_2 q\exp\left(\eta \sum_{p \leq \log^{2/\delta} q} \frac{\log p}{p}\right) \ll q^{-\eta} \log_2 q \log^{2\eta/\delta} q.
%\end{align*}
%Choosing $\eta := \frac{\delta}{2 \log_2 q}$ suffices to give
%\begin{equation*}
%M(\chi) + \sqrt{q} \gg \frac{\sqrt{qm}}{\phi(m)} \left|\sum_{P^+(n) \leq \log^{2/\delta}q} \frac{\chi(n)\bar{\psi}(n)}{n}\right|,
%\end{equation*}
%and by a basic Euler product argument, this is easily seen to give
%\begin{equation*}
%M(\chi) + \sqrt{q} \gg_{\delta} \frac{\sqrt{qm}}{\phi(m)} \log_2 q \exp\left(-\mb{D}^2(\chi,\psi;\log q)\right),
%\end{equation*}
%as claimed.
%\end{proof}


%\begin{lem}
%Suppose $Q \geq 2$, $q \leq Q$, $\sigma_0 \in (1/2,1)$ and $A > 0$. Suppose that $\chi$ modulo $q$ is $(Q,A,\sigma_0)$-good. Then there is some $B = B(A,\sigma_0) > 0$ such that
%\begin{equation*}
%\log L(1,\chi) = \sum_{p \leq \log^B q} \frac{\chi(p)}{p} + O(1).
%\end{equation*}
%\end{lem}
%\begin{lem}






%we can choose any $a$ such that $\psi(a) = z^{\frac{\phi(m)}{k}\left(l+u\frac{k}{\phi(m)}\right)}$, for $z$ a primitive $\phi(m)$th root of unity and some $0 \leq u \leq \frac{\phi(m)}{k}-1$.   
%\begin{equation}
%\sum_{l(k)} z_l e\left(-\frac{l}{k}\right) = (g,k) \frac{\sin(\pi/g)}{\tan(\pi(g,k)/gk)}, \label{IDEN}
%\end{equation}
%and by basic group theory, for any $l (k)$ the number of different residue classes $a$ modulo $m$ for which $\psi(a) = e\left(\frac{l}{k}\right)$ is $\frac{\phi(m)}{k}$. Proposition 9.2 in \cite{GOLD} estimates \eqref{MAINSUM}, but the statement seems not to be efficient. Indeed, he estimates 
%\begin{equation*}
%\sum_{p \leq y \atop \psi(p) = e\left(\frac{l}{k}\right)} \frac{1}{p} = \sum_{a(m) \atop \psi(a) = e\left(\frac{l}{k}\right)} \sum_{p \leq y \atop p \equiv a(m)} \frac{1}{p}
%\end{equation*}


%\begin{proof}
%Let $Q,T \geq 3$. As in Lemma 1 of \cite{LaZ}, there exists a constant $c_1 > 0$ in the box 
%\begin{equation*}
%\mc{B}(Q,T) := \left\{s = \sg + i\tau : |\tau| \leq T, 1-\frac{c_1}{\log Q + (\log T)^{\frac{2}{3}} (\log_2 T)^{\frac{1}{3}}} < \sg \leq 1\right\}
%\end{equation*}
%there is at most one value $\rho:= \beta + i\gamma$ such that for some $m \leq Q$ and a some character $\chi$ modulo $m$ (necessarily quadratic), $L(\rho,\chi) = 0$. In this case, we say that $m$ is the \emph{exceptional} modulus if $m$ is the modulus of that $\chi$ for which $L(\beta + i\gamma,\chi) = 0$.  Now define $\mc{I}_{\beta}(x) := \int_x^{\infty} \frac{dt}{t\log t} e^{-(1-\beta) \log t}$, where $\beta = \text{Re}(\rho)$, and let 
%\begin{equation*}
%G(x;m,\beta) := \begin{cases} e^{-I_{\beta}(x)} &\text{ if $m$ is exceptional} \\ 1 &\text{ otherwise}.
%\end{equation*}
%Then by Theorem 2 of \cite{LaZ}, we have for each $m \leq R(x)^A$ and $(a,m) = 1$, 
%\begin{equation*}
%\prod_{p \leq x \atop p \equiv a(m)} \left(1-\frac{1}{p}\right) = \frac{C(m,a)}{(\log x)^{\frac{1}{\phi(m)}}} G(x;m,\beta) \left(1+O\left(L(x)^{-B}\right).
%\end{equation*}
%Taking logarithms and rearranging, we have
%\begin{equation*}
%\sum_{p \leq x \atop p \equiv a(m)} \frac{1}{p} = \frac{1}{\phi(m)} \log_2 x - \log C(m,a) + \log G(x;m,\beta) + \sum_{p \leq x \atop p \equiv a (m)} \left(\frac{1}{p} + \log(1-1/p)\right) + O\left((x)
%
%It follows from Theorem 2 and Corollary 3 of \cite{LaZ} that, uniformly for $m\leq \log x$ we have 
%$$ \prod_{\substack{p\leq x\\ p\equiv a \bmod m}}\left(1-\frac{1}{p}\right)= \frac{C(m, a)}{(\log x)^{1/\varphi(m)}}\left(1+O\left(\frac{1}{(\log x)^{3/5-\epsilon}}\right)\right)
%$$
%The range of this estimate increases to $m \leq \exp\left(\log^{2/5} x \log^{-1/5} x\right)$ when $m$ is confined to the set of non-exceptional moduli. 
%Therefore, we deduce that in the same range for $m$ we have
%\begin{equation} \label{MERT}
%\sum_{\substack{p\leq x\\ p\equiv a \bmod m}} \frac{1}{p}= \frac{1}{\varphi(m)} \log_2 x+ \log C(m, a)+ \sum_{\substack{p\leq x\\ p\equiv a \bmod m}}\left(\log\left(1-\frac1p\right)+\frac1p\right)+ O\left(\frac{1}{(\log x)^{3/5-\epsilon}}\right).
%\end{equation}

\begin{lem}\label{ApproximationL} Let $m$ a positive integer that is not an exceptional discriminant, and put $X=\exp\left((\log m)^{3}\right)$.  Then, for any non-principal character modulo $m$ we have
$$
\log L(1, \chi)= -\sum_{p\leq X} \log \left(1-\frac{\chi(p)}{p}\right)+O\left(\frac 1m\right).
$$
\end{lem}
\begin{proof}
Let $\alpha=1/\log X$. Then it follows from Perron's formula that
\begin{equation}\label{Perron}
\begin{aligned}
\frac{1}{2\pi i} \int_{\alpha-iT}^{\alpha+iT} \log L(1+s, \chi) \frac{X^s}{s} ds &= \sum_{n\leq X} \frac{\Lambda(n)}{n\log n}\chi(n)+ O\left(\sum_{n=1}^{\infty} \frac{\Lambda(n)}{n^{1+\alpha}\log n}\min\left(1, \frac{1}{T\log|X/n|}\right)\right)\\
&= \sum_{n\leq X} \frac{\Lambda(n)}{n\log n}\chi(n)+ O\left(\frac{\log X}{T}+ \frac{1}{X}\right),
\end{aligned}
\end{equation}
by a standard estimation of the error term. Moreover, we observe that 
\begin{align*}
 \sum_{n\leq X} \frac{\Lambda(n)}{n\log n}\chi(n)&= -\sum_{p\leq X} \log \left(1-\frac{\chi(p)}{p}\right)+O\left(\sum_{k=2}^{\infty}\sum_{p^k>X} \frac{1}{k p^k}\right)\\
 &= -\sum_{p\leq X} \log \left(1-\frac{\chi(p)}{p}\right) +O\left(\frac{1}{X}\right).
\end{align*}
Since $m$ is a non-exceptional discriminant, there exists a positive constant $c$ such that $L(\sigma+it, \chi)$ does not have a zero in the region
$$\sigma\geq 1- \beta(t), \text{ where } \beta(t)=\frac{c}{\log(m(|t|+2))}.$$
We now move the contour in \eqref{Perron} to the curve $\{\sigma + it : \sigma = -\beta(t)\}$. We encounter a simple pole at $s=0$ which leaves a residue of $\log L(1, \chi)$. Therefore, we deduce that
$$ \frac{1}{2\pi i} \int_{\alpha-iT}^{\alpha+iT} \log L(1+s, \chi) \frac{X^s}{s} ds= \log L(1, \chi)+\mathcal{E}, $$
where 
\begin{align*}
\mathcal{E}&= \frac{1}{2\pi i} \left(\int_{\alpha-iT}^{-\beta(t)-iT}+\int_{-\beta(t)-iT}^{-\beta(t)+iT}+ \int_{-\beta(t)+iT}^{\alpha+iT}\right)\log L(1+s, \chi) \frac{X^s}{s} ds\\
&\ll \frac{(\log m T)^2}{T}+ (\log m T)^3 X^{-\beta(t)},
\end{align*}
where the bounds here are standard (and follow from the results in \cite{GrSo} for example).
Choosing $T=m^2$ completes the proof.
\end{proof}

\begin{proof}
Let $0<\sigma\leq \delta$, and recall that the Fourier transform of $k(t)= e^{-\sigma|t|}$, is 
$$ \hat{k}(\omega)=\int_{-\infty}^{\infty} k(t) e^{-i\omega t} dt= \frac{2\sigma}{\sigma^2+\omega^2}.
$$
Let $A=\delta (\log_2y)/2$. Then, by the Fourier inversion formula we have 
$$ k(t)= \frac{1}{\pi}\int_{-\infty}^{\infty}\frac{\sigma}{\sigma^2+\omega^2}e^{i\omega t} d\omega= \frac{1}{\pi}\int_{-A}^{A}\frac{\sigma}{\sigma^2+\omega^2}e^{i\omega t} d\omega+O\left(\frac{1}{\log_2y}\right).$$
Thus, we get
$$ \frac{1}{p^{\sigma}}=k(\log p)=k(-\log p)=\frac{1}{\pi}\int_{-A}^{A}\frac{\sigma}{\sigma^2+\omega^2}\frac{1}{p^{i\omega}} d\omega+O\left(\frac{1}{\log_2y}\right).$$ 
Multiplying both sides of the inequality by $f(p)/p^{1+it}$ and summing over $p\leq y$ gives
$$ \sum_{p\leq y}\frac{f(p)}{p^{1+\sigma+it}}= \frac{1}{\pi}\int_{-A}^{A}\frac{\sigma}{\sigma^2+\omega^2}\left(\sum_{p\leq y} \frac{f(p)}{p^{1+i(t+\omega)}}\right) d\omega +O(1).
$$
Taking the real parts of both sides, and noting that  $\sigma/(\sigma^2+\omega^2)\geq 0$ and $|t|\leq \delta$, we obtain 
\begin{equation}\label{MaxKernel}
\begin{aligned}
\max_{|t|\leq \delta}\left(\re\sum_{p\leq y}\frac{f(p)}{p^{1+\sigma+it}}\right)
&\leq 
\max_{|\tau|\leq 2A} \left(\re \sum_{p\leq y} \frac{f(p)}{p^{1+i\tau}}\right)\cdot  \frac{1}{\pi}\int_{-A}^{A}\frac{\sigma}{\sigma^2+\omega^2} d\omega +O(1)\\
&\leq \left(\log_2 y-\mathcal{M}(f; y, \delta\log_2y)\right)\left(\frac{1}{\pi} \int_{-A}^A \frac{\sigma}{\sigma^2 + \omega^2} d\omega\right) +O(1),
\end{aligned}
\end{equation}
Clearly, $\log_2 y \geq \mathcal{M}(f; y, \delta\log_2y)$, and 
$$ \frac{1}{\pi}\int_{-A}^{A}\frac{\sigma}{\sigma^2+\omega^2} d\omega\leq \frac{1}{\pi}\int_{-\infty}^{\infty}\frac{\sigma}{\sigma^2+\omega^2} d\omega=1.$$
Taking the exponential of each side of  \eqref{MaxKernel} and noting that 
$$|F(1+\sigma+it)|\asymp \exp\left(\re\sum_{p\leq x}\frac{f(p)}{p^{1+\sigma+it}}\right),
$$
completes the proof.
\end{proof}




