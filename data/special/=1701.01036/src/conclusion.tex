\begin{section}{Conclusion}
Despite the great success of neural style transfer, the rationale behind neural style transfer was far from crystal. The vital ``trick'' for style transfer is to match the Gram matrices of the features in a layer of a CNN. Nevertheless, subsequent literatures about neural style transfer just directly improves upon it without investigating it in depth. In this paper, we present a timely explanation and interpretation for it. First, we theoretically prove that matching the Gram matrices is equivalent to a specific Maximum Mean Discrepancy (MMD) process. Thus, the style information in neural style transfer is intrinsically represented by the  distributions of activations in a CNN, and the style transfer can be achieved by distribution alignment. Moreover, we exploit several other distribution alignment methods, and find that these methods all yield promising transfer results. Thus, we justify the claim that neural style transfer is essentially a special domain adaptation problem both theoretically and empirically. We believe this interpretation provide a new lens to re-examine the style transfer problem, and will inspire more exciting works in this research area.



\end{section}