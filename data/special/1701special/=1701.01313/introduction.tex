% !TeX root = main.tex

\section{introduction}
Let $\Sigma$ be a closed surface of genus $g$. It is a well-known fact that the size of a collection of non-homotopic simple closed curves on $\Sigma$ is bounded by $3g-3$.
Such a collection induces an action of $\pi_1(\Sigma)$ on a dual tree.
Sageev \cite{Sag95} showed how a general collection of curves gives rise to an action on a CAT(0) cube complex. This motivates the following definition.
Let $d\in\N$. A collection $\ptrn{S}$ of homotopy classes of essential curves on $\Sigma$ is called a \emph{$d$-pattern} if any pairwise intersecting set of lifts of them to the universal cover $\uc{\Sigma}$ of $\Sigma$ has cardinality at most $d$.
Applying Sageev's construction to a $d$-pattern yields a CAT(0) cube complex of dimension at most $d$.



%Let $\Sigma$ be a compact surface,  and let $d\in\N$. 
%This definition is motivated by the well-known construction of dual CAT(0) cube complexes by Sageev \cite{Sag95}.


%We remark that for $d=1$ this amounts to saying that $\ptrn{S}$ is a collection of disjoint simple closed curves and arcs. 
%In this case, one obtains a dual tree.
%It is a well-known fact that for $d=1$ the size of $\ptrn{S}$ is bounded by $3g-3+p$ where $g$ and $p$ are the genus and number of boundary components of $\Sigma$ respectively (whenever this number is positive). 
Thus, one is naturally led to ask the following question.

\begin{question}\label{curves on surfaces?}
Is there a bound $D=D(\Sigma,d)$ on the possible size of $\ptrn{S}$?
\end{question}

Similarly one can define $d$-patterns for collections of subsurfaces in 3-manifolds, and ask a similar question.
Let us note, that for $d=1$, this question was answered by Kneser \cite{Kne29} for collections of subspheres in 3-manifolds, and by Haken \cite{Hak61} and Milnor \cite{Mil62} for general subsurfaces. In \cite{BeLa16}, we answered both question affirmatively for $d=2$.

Dunwoody \cite{Dun85} defined the notion of patterns (which we consider as \emph{$1$-patterns}) on general finite 2-dimensional simplicial complexes. 
As in the case of $1$-patterns on surfaces (and 3-manifolds), $1$-patterns on simplicial complexes give rise to dual trees when lifted to the universal cover.
This fact was used in his paper to study actions of finitely presented groups by introducing \emph{resolutions} and studying their properties. 
In particular, Dunwoody proved that the size of a pattern on a finite 2-dimensional simplicial complex is bounded above by a bound which depends only on the simplicial complex. 
This result is a crucial step in the proof of accessibility, and moreover provides an easy combinatorial proof of the aforementioned bounds on $1$-patterns on surfaces and 3-manifolds.

In \cite{BeLa16}, we introduced the notion of $d$-patterns on 2-dimensional simplicial complexes and resolutions of actions on CAT(0) cube complexes. We will review these definitions in Section \ref{tracks and patterns}.
%We remark that for $d=2$, the bound for curves on surfaces can also be obtained by an Euler characteristic argument.

%In \todo{ref to Dunwoody}, Dunwoody studied group actions on trees using resolutions. 
%The resolution of a group action on a tree is a tree obtained by pulling back the mid-edges of the tree to a simply connected complex on which the group acts freely.
%He used these construction to prove the accessibility of finitely presented groups, and to give simple proofs of theorems of Kneser and Haken-Milnor providing bounds on the sizes of collections of disjoint essential embedded surfaces in 3-manifolds.
%
%In \todo{ref to 2d paper}, the authors introduced a similar notions for group actions on cube complexes.
%Similar to Dunwoody's result one obtains bounds on the number of tracks in the resolution of a 2-dimensional cube complex is bounded by a number that depends only on the group. As a corollary, similar bounds on collections of curves on surfaces (and embedded surfaces in 3-manifolds) with the property that any pairwise intersecting collection of lifts of the curves to the universal is of size at most two.

In this paper we extend the main result of \cite{BeLa16} to arbitrary $d$.

\begin{theorem}\label{main result}
	Let $\simp{K}$ be a finite 2-dimensional simplicial complex, and let $d\in\N$. Then there exists a constant $C=C(\simp{K},d)$ such that any $d$-pattern on $\simp{K}$ has at most $C$ parallelism classes of tracks.
\end{theorem}

As corollary we derive the following theorem, which answers Question \ref{curves on surfaces?}.

\begin{theorem}\label{thm: bound on subsurfaces}
	Let $\Sigma$ be a compact surface, and let $d\in\N$. There exists a constant $C=C(K,d)$ such that any $d$-pattern of curves and arcs on $\Sigma$ has at most $C=C(K,d)$ different homotopy classes.
\end{theorem}

Similarly for $3$-manifolds, we have the following.

\begin{theorem}\label{thm: bound on curves}
	Let $\mfld M$ be a compact irreducible, boundary-irreducible 3-manifold, and let $d\in\N$. There exists a constant $C=C(M,d)$, such that if $\ptrn S$ is a collection of non-homotopic, $\pi_1$-injective, 2-sided, embedded subsurfaces, such that the size of a pairwise intersecting collection of lifts to $\uc{\mfld M}$ is at most $d$, then $|\ptrn S|\le C$.
\end{theorem}

%\begin{theorem} \label{thm: bound on submanifolds} \todo{rewrite}
%Let $\mfld M$ be a compact surface or 3-manifold manifold, and let $d\in\N$. There exists a constant $C$, depending only on $\mfld M$ and on $d$, such that if $\ptrn S$ is a collection of non-homotopic, $\pi_1$-injective, co-dimension-1, 2-sided, embedded sub-manifolds, such that the size of a pairwise intersecting collection of lifts to $\uc{\mfld M}$ is at most $d$, then $|\ptrn S|\le C$.\todo{I copied it from the previous paper, but it seems to be missing some assumptions.}
%\end{theorem}

For the proofs of Theorem \ref{thm: bound on subsurfaces} and Theorem \ref{thm: bound on curves}  from Theorem \ref{main result}, we refer to Section 5 in~\cite{BeLa16}.

Dunwoody's bound on patterns was extensively used in the literature to study accessibility of group actions on trees.
In this paper, we focus on generalizing acylindrical accessibility for CAT(0) cube complexes.

Let $G$ be a group, $\mathcal{C}$ be a collection of subgroups of $G$ which is closed under conjugation and subgroups, and $k$ be a natural number. 
We say that the group $G$ acts $(k,\mathcal{C})$-\emph{acylindrically} on a tree if the stabilizer of any segment of $k$ edges in the tree belongs to the collection $\mathcal{C}$. 
Similarly one can define $(k,\mathcal{C})$-\emph{acylindricity on hyperplanes} for actions on cube complexes by requiring that the common stabilizer of any chain of $k$ halfspaces belongs to $\mathcal{C}$.
This notion should not be confused with acylindrical actions (and weak acylindrical actions) on metric spaces, see Bowditch \cite{Bow08}, even though the two are related by recent work of Genevois \cite{Gen16}.


In \cite{Sel97}, Sela proved that for any finitely generated group $G$ and $k$, any reduced $(k,\{1\})$-acylindrical action of $G$ on a tree has a bounded quotient, or equivalently, there is a bound on the number of orbits of edges.
In \cite{Del99}, Delzant proved a similar result for finitely presented groups using Dunwoody's bounds on resolutions. He showed that if $G$ is finitely presented and does not split non-trivially over a subgroup in $\mathcal{C}$, then there is a bound that depends on $G$ and $k$ on the number of edge-orbits of $(k,\mathcal{C})$-acylindrical actions of $G$ on a tree.

Since Theorem \ref{main result} applies more generally to cubulations which come from patterns, following Delzant's proof, we are able to prove the following theorem.

\begin{theorem}[Acylindrical accessibility for CAT(0) cube complexes]\label{acylindrical accessibility}
	Let $G$ be a finitely presented group, let $\mathcal{C}$ be a family of subgroups of $G$ which is closed under conjugation, commensurability, and subgroups, and let $d\in\N$. There exists $D=D(d,G)$ such that if $G$ does not act essentially on a $d$-dimensional CAT(0) cube complex with hyperplanes stabilizers in $\mathcal{C}$, then any $(k,\mathcal{C})$-acylindrical on hyperplanes essential action on a $d$-dimensional CAT(0) cube complex has at most $k\cdot D$ hyperplanes.
\end{theorem}

%\begin{proof}
%	Let $\simp{K}$ be a presentation complex for $G$, i.e $\pi_1 (\simp{K})=G$.
%	Let $\CC{X}$ be a $d$-dimensional CAT(0) cube complex on which $G$ acts $(k,\mathcal{C})$-acylindrically. 
%	Pullback the hyperplanes of $\CC{X}$ to get a $d$-pattern $\ptrn{P}$ on $\simp{K}$. 
%	By the pigeon hole principle, if $\ptrn{P}$ has more than $kC$ tracks, then there are $k+1$ which belong to the same parallelism class. Let $\trk{t}$ be a track in this parallelism class. By assumption, the stabilizers of $\trk{t}$ is in $\mathcal {C}$, and this track alone gives a $d$-pattern on $G$, which induces a $G$ action on a $d$-dimensional CAT(0) cube complex whose hyperplane stabilizers are in $\mathcal {C}$. Contradicting the assumption on $G$.	
%\end{proof}

The following Corollary follows from Theorem \ref{acylindrical accessibility} and item \ref{one end implies no CCC over finite} of Proposition \ref{cube complexes to trees}.

\begin{corollary}\label{one ended acyl accessibility}
	Let $G$ be a finitely presented one-ended group, then for all $d$ there exists a constant $C=C(d,G)$ such that every $(k,\mathcal{F})$-acylindrical on hyperplanes action on a $d$-dimensional CAT(0) cube complex has at most $k\cdot C$ hyperplanes, where $\mathcal {F}$ is the collection of all finite subgroups.
\end{corollary}

As an application we prove the following on embeddings of finitely presented one-ended groups into hyperbolic Coxeter groups.

\begin{corollary}
	Let $G$ be a finitely presented one-ended group, and let $d\in\N$. Then there exists $D=D(G,d)$ such that for any embedding of $G$ into a hyperbolic right-angled Coxeter group $W_\Gamma$ on a graph $\Gamma$ with clique number at most $d$, there exists a subgraph $\Gamma'$ with at most $D$ vertices such that the image of $G$ is in a conjugate of the special parabolic subgroup $W_{\Gamma'}\le W_{\Gamma}$.
\end{corollary}

\begin{proof}
	Without loss of generality let $\Gamma$ be such that $G$ does not embed into a conjugate of a proper special subgroup. 
	The embedding induces an action of $G$ on the Davis complex $\CC{X}$ of $W_\Gamma$.
	Each hyperplane of $\CC{X}$ has a corresponding vertex $v$ in $\Gamma$, and the stabilizer of the hyperplane is a conjugate of the special subgroup $W_{\Link(v,\Gamma)}$. 
	By the hyperbolicity of $W_\Gamma$, the stars of any two vertices at distance 2 intersect in a clique. Hence, the common stabilizer of any two adjacent hyperplanes in $A(\Gamma)$ is finite. Thus, the common stabilizer in $G$ is finite. 
	This shows that the action of $G$ on $\CC{X}$ is (2,$\mathcal{F}$)-acylindrical on hyperplanes. 
	By Corollary \ref{one ended acyl accessibility} we obtain the desired conclusion.
\end{proof}


We note that the bounds obtained in Theorem \ref{main result} are probably far from being sharp, since they depend in part on Ramsey's theorem. 
Thus, we did not bother computing them.
However, one may ask what are the effective bounds. 
In particular, even though our bound in Theorem \ref{thm: bound on subsurfaces} depends linearly on the genus of the surface, the question of finding the optimal dependence on $d$ remains open. 

A priori, Question \ref{curves on surfaces?} may appear related to the bounds obtained in  Aougab and Gaster \cite{AoGa15}  or Przytycki \cite{Prz15} on sets of curves with bounded intersections. 
However, we would like to point out that these problem are of fundamentally different nature. 
For example, while there are only finitely many mapping class group orbits of sets of curves with at most $k$ intersections, there are infinitely many orbits of $d$-patterns for any $d\ge 2$. 
%\todo{add discussion about bounds on counting curves on surfaces, Przytycki et al.}