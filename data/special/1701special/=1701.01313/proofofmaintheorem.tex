% !TeX root = main.tex
\section{Proof of the main theorem}
We follow the proof of Theorem A' in \cite{BeLa16}.

\begin{proof}[Proof of Theorem \ref{main result}]
Let $\usimp{K}$ be the universal cover of $\simp{K}$ and $\uptrn{P}$ the pattern on $\usimp{K}$ associated to $\ptrn{P}$. Since $\ptrn P$ is a $d$-pattern, the \CCC $\CC{X}$ is a $d$ dimensional cube complex.

For a vertex $\usimpv{x}$ in $\usimp{K}$ call $\CCv{\bar{x}}$ the corresponding vertex in $\CC{X}$. Similarly the halfspaces corresponding to a track $\utrk{t}$ in $\uptrn{P}$ are called $\hs{h}_{\utrk{t}}$ and $\comp{\hs{h}}_{\utrk{t}}$. A triangle in $\CC{X}$ is a triplets of vertices $(\CCv{\bar{x}}, \CCv{\bar{y}}, \CCv{\bar{z}})$  coming from a triangle $(\usimpv{x},\usimpv{y},\usimpv{z})$ of $\usimp{K}$.

Two tracks $\utrk{t}$ and $\utrk{t}'$ of $\uptrn P$ are \emph{\adjP}  if they cross an edge $[\usimpv{x},\usimp{y}]$ such that $\hs{h}_{\utrk{t}}$ and $\hs{h}_{\utrk{t}'}$ are \adjP in one of the oriented interval defined by $\CCv{\bar{x}}$ and $\CCv{\bar{y}}$.

Note that if two halfspaces $(\hs{h}, \hs{k})$ in $\CC{X}$ are not parallel but intersect an interval $[\CCv{\bar{v}}, \CCv{\bar{w}}]$ in which they are \adjP, then:
\begin{enumerate}
\item  either there exists some triangle $(\CCv{\bar{x}}, \CCv{\bar{y}}, \CCv{\bar{z}})$ such that $(\hs{h}, \hs{k})$ is \adjP in  $[\CCv{\bar{x}}, \CCv{\bar{y}}]$ but is separated by the midpoint of $(\CCv{\bar{x}}, \CCv{\bar{y}}, \CCv{\bar{z}})$,
\item or  there exists some triangle $(\CCv{\bar{x}}, \CCv{\bar{y}}, \CCv{\bar{z}})$ such that $(\hs{h}, \hs{k})$ is \adjP in  $[\CCv{\bar{x}}, \CCv{\bar{y}}]$, intersects $[\CCv{\bar{x}}, \CCv{\bar{z}}]$ but is not \adjP in it.
\end{enumerate}

If there are no parallel tracks in $\ptrn P$, a halfspace $\hs h$ in $\CC{X}$  belongs to one of the following categories that can be bounded.

\begin{enumerate}
\item \label{enum1} The halfspace $\hs h$ is associated to a track belonging to an edge of $K$ which is not in a triangle. Two tracks of this form on the same edge are parallel, therefore on each edge there is at most one track $\trk{t}$, associated to two halfspaces $\hs{h}_{\trk{t}}$ and $\comp{\hs{h}}_{\trk{t}}$.
\item \label{enum2} The halfspace $\hs h$ belongs to an interval $[\CCv{\bar{x}}, \CCv{\bar{y}}]$ and is maximal in it. For each directed interval there are at most $d$ maximal halfspaces, and thus at most $2 d$ per edge.  Note that this case contains the previous one.
\item  \label{enum3} There exist some halfspace $\hs k$ and some triangle $(\CCv{\bar{x}}, \CCv{\bar{y}}, \CCv{\bar{z}})$ such that $(\hs{h}, \hs{k})$ is \adjP in  $[\CCv{\bar{x}}, \CCv{\bar{y}}]$ but is separated by the midpoint of $(\CCv{\bar{x}}, \CCv{\bar{y}}, \CCv{\bar{z}})$. By lemma \ref{lemma1} each triangle and directed interval $\Int{I}$ defined by an edge of the triangle, there is a bound $C_1$ of pairs of \adjP halfspaces in $\Int{I}$ separated by the midpoint of the triangle. There are $6$ directed intervals associated to each triangle.
\item \label{enum4} There exist some halfspace $\hs k$ and some triangle $(\CCv{\bar{x}}, \CCv{\bar{y}}, \CCv{\bar{z}})$ such that $(\hs{h},\hs{k})$ is \adjP in  $[\CCv{\bar{x}}, \CCv{\bar{y}}]$, intersects $[\CCv{\bar{x}}, \CCv{\bar{z}}]$ but is not \adjP in it. By lemma \ref{lemma2}, for each triangle and each pair of intervals $[\CCv{\bar{x}}, \CCv{\bar{y}}]$ and $[\CCv{\bar{x}}, \CCv{\bar{z}}]$ there is a bound $C_2$ of pairs of halfspaces   that intersect both $[\CCv{\bar{x}}, \CCv{\bar{y}}]$ and $[\CCv{\bar{x}}, \CCv{\bar{z}}]$, \adjP in the first one but not the second one. There are $6$ choices of such a pair of intervals per triangle.
\item \label{enum5} There exist some halfspace $\hs k$ and some triangle $(\CCv{\bar{x}}, \CCv{\bar{y}}, \CCv{\bar{z}})$ such that $(\hs{h},\hs{k})$ is \adjP in  $[\CCv{\bar{y}}, \CCv{\bar{x}}]$, intersects $[\CCv{\bar{z}}, \CCv{\bar{x}}]$ but is not \adjP in it. By lemma \ref{lemma2}, given a triangle and a pair of intervals $[\CCv{\bar{x}}, \CCv{\bar{y}}]$ and $[\CCv{\bar{x}}, \CCv{\bar{z}}]$ there are no pair of halfspaces  that intersects both $[\CCv{\bar{y}}, \CCv{\bar{x}}]$ and $[\CCv{\bar{y}}, \CCv{\bar{x}}]$, \adjP in the first one but not the second one.
\end{enumerate}


 If we denote by $E$ and $T$ the number of edges and triangles in $\simp K$, then there are at most $2dE +(6C_1+6C_2)T$ non parallel halfspaces  in $\CC{X}$.
\end{proof}