%auto-ignore
\usepackage{fullpage}
\usepackage{amsfonts, amssymb, amsmath, amsthm}
\usepackage{latexsym}
\usepackage[tracking=smallcaps]{microtype}	% for an arXiv submission, set \pdfoutput=1 near the top so it uses pdflatex
\usepackage{url}
\usepackage{color}
\definecolor{DarkGray}{rgb}{0.1,0.1,0.5}
\usepackage[colorlinks=true,breaklinks, linkcolor=black,citecolor=black,urlcolor=DarkGray]{hyperref}	% linkcolor=blue,citecolor=blue,urlcolor=blue %DarkGray

\usepackage{graphicx}
\usepackage[tight, TABBOTCAP]{subfigure}
%\usepackage{cancel}
%\usepackage{multirow}
%\usepackage[small]{caption}	%% caption package is useful for allowing line breaks within figure captions

%\usepackage{import}
%\usepackage{longtable}
%\usepackage{booktabs}
%\usepackage{ltxtable}
%\usepackage{rotating}	%% rotating package defines sideways environment

%\usepackage{calc}	%% The calc package reimplements the LaTeX commands \setcounter, ..., so that these commands accept an infix notation expression.

%\let\oldbibliography\thebibliography
%\renewcommand{\thebibliography}[1]{%
%  \oldbibliography{#1}%
%  \setlength{\itemsep}{.25pt}%
%}

%\usepackage[letterpaper]{geometry}		\geometry{includefoot,verbose,nohead,tmargin=1in,bmargin=.75in,lmargin=1.5in,rmargin=1in}
%\setlength{\parindent}{0.25in} \setlength{\parskip}{6pt}	\renewcommand{\baselinestretch}{1.66}	% stretch things out
%\def\ssp{\def\baselinestretch{1.0}\large\normalsize}	% single-space references

\def\place #1#2#3{\mspace{#2}\makebox[0pt]{\raisebox{#3}{#1}}\mspace{-#2}}	% the second argument should be in mu, and the third argument in pt

%%    Q-circuit version 2
%    Copyright (C) 2004  Steve Flammia & Bryan Eastin
%    Last modified on: 9/16/2011
%
%    This program is free software; you can redistribute it and/or modify
%    it under the terms of the GNU General Public License as published by
%    the Free Software Foundation; either version 2 of the License, or
%    (at your option) any later version.
%
%    This program is distributed in the hope that it will be useful,
%    but WITHOUT ANY WARRANTY; without even the implied warranty of
%    MERCHANTABILITY or FITNESS FOR A PARTICULAR PURPOSE.  See the
%    GNU General Public License for more details.
%
%    You should have received a copy of the GNU General Public License
%    along with this program; if not, write to the Free Software
%    Foundation, Inc., 59 Temple Place, Suite 330, Boston, MA  02111-1307  USA

% Thanks to the Xy-pic guys, Kristoffer H Rose, Ross Moore, and Daniel Müllner,
% for their help in making Qcircuit work with Xy-pic version 3.8.  
% Thanks also to Dave Clader, Andrew Childs, Rafael Possignolo, Tyson Williams,
% Sergio Boixo, Cris Moore, Jonas Anderson, and Stephan Mertens for helping us test 
% and/or develop the new version.

\usepackage{xy}
\xyoption{matrix}
\xyoption{frame}
\xyoption{arrow}
\xyoption{arc}

\usepackage{ifpdf}
\ifpdf
\else
\PackageWarningNoLine{Qcircuit}{Qcircuit is loading in Postscript mode.  The Xy-pic options ps and dvips will be loaded.  If you wish to use other Postscript drivers for Xy-pic, you must modify the code in Qcircuit.tex}
%    The following options load the drivers most commonly required to
%    get proper Postscript output from Xy-pic.  Should these fail to work,
%    try replacing the following two lines with some of the other options
%    given in the Xy-pic reference manual.
\xyoption{ps}
\xyoption{dvips}
\fi

% The following resets Xy-pic matrix alignment to the pre-3.8 default, as
% required by Qcircuit.
\entrymodifiers={!C\entrybox}

\newcommand{\bra}[1]{{\left\langle{#1}\right\vert}}
\newcommand{\ket}[1]{{\left\vert{#1}\right\rangle}}
    % Defines Dirac notation. %7/5/07 added extra braces so that the commands will work in subscripts.
\newcommand{\qw}[1][-1]{\ar @{-} [0,#1]}
    % Defines a wire that connects horizontally.  By default it connects to the object on the left of the current object.
    % WARNING: Wire commands must appear after the gate in any given entry.
\newcommand{\qwx}[1][-1]{\ar @{-} [#1,0]}
    % Defines a wire that connects vertically.  By default it connects to the object above the current object.
    % WARNING: Wire commands must appear after the gate in any given entry.
\newcommand{\cw}[1][-1]{\ar @{=} [0,#1]}
    % Defines a classical wire that connects horizontally.  By default it connects to the object on the left of the current object.
    % WARNING: Wire commands must appear after the gate in any given entry.
\newcommand{\cwx}[1][-1]{\ar @{=} [#1,0]}
    % Defines a classical wire that connects vertically.  By default it connects to the object above the current object.
    % WARNING: Wire commands must appear after the gate in any given entry.
\newcommand{\gate}[1]{*+<.6em>{#1} \POS ="i","i"+UR;"i"+UL **\dir{-};"i"+DL **\dir{-};"i"+DR **\dir{-};"i"+UR **\dir{-},"i" \qw}
    % Boxes the argument, making a gate.
\newcommand{\meter}{*=<1.8em,1.4em>{\xy ="j","j"-<.778em,.322em>;{"j"+<.778em,-.322em> \ellipse ur,_{}},"j"-<0em,.4em>;p+<.5em,.9em> **\dir{-},"j"+<2.2em,2.2em>*{},"j"-<2.2em,2.2em>*{} \endxy} \POS ="i","i"+UR;"i"+UL **\dir{-};"i"+DL **\dir{-};"i"+DR **\dir{-};"i"+UR **\dir{-},"i" \qw}
    % Inserts a measurement meter.
    % In case you're wondering, the constants .778em and .322em specify
    % one quarter of a circle with radius 1.1em.
    % The points added at + and - <2.2em,2.2em> are there to strech the
    % canvas, ensuring that the size is unaffected by erratic spacing issues
    % with the arc.
\newcommand{\measure}[1]{*+[F-:<.9em>]{#1} \qw}
    % Inserts a measurement bubble with user defined text.
\newcommand{\measuretab}[1]{*{\xy*+<.6em>{#1}="e";"e"+UL;"e"+UR **\dir{-};"e"+DR **\dir{-};"e"+DL **\dir{-};"e"+LC-<.5em,0em> **\dir{-};"e"+UL **\dir{-} \endxy} \qw}
    % Inserts a measurement tab with user defined text.
\newcommand{\measureD}[1]{*{\xy*+=<0em,.1em>{#1}="e";"e"+UR+<0em,.25em>;"e"+UL+<-.5em,.25em> **\dir{-};"e"+DL+<-.5em,-.25em> **\dir{-};"e"+DR+<0em,-.25em> **\dir{-};{"e"+UR+<0em,.25em>\ellipse^{}};"e"+C:,+(0,1)*{} \endxy} \qw}
    % Inserts a D-shaped measurement gate with user defined text.
\newcommand{\multimeasure}[2]{*+<1em,.9em>{\hphantom{#2}} \qw \POS[0,0].[#1,0];p !C *{#2},p \drop\frm<.9em>{-}}
    % Draws a multiple qubit measurement bubble starting at the current position and spanning #1 additional gates below.
    % #2 gives the label for the gate.
    % You must use an argument of the same width as #2 in \ghost for the wires to connect properly on the lower lines.
\newcommand{\multimeasureD}[2]{*+<1em,.9em>{\hphantom{#2}} \POS [0,0]="i",[0,0].[#1,0]="e",!C *{#2},"e"+UR-<.8em,0em>;"e"+UL **\dir{-};"e"+DL **\dir{-};"e"+DR+<-.8em,0em> **\dir{-};{"e"+DR+<0em,.8em>\ellipse^{}};"e"+UR+<0em,-.8em> **\dir{-};{"e"+UR-<.8em,0em>\ellipse^{}},"i" \qw}
    % Draws a multiple qubit D-shaped measurement gate starting at the current position and spanning #1 additional gates below.
    % #2 gives the label for the gate.
    % You must use an argument of the same width as #2 in \ghost for the wires to connect properly on the lower lines.
\newcommand{\control}{*!<0em,.025em>-=-<.2em>{\bullet}}
    % Inserts an unconnected control.
\newcommand{\controlo}{*+<.01em>{\xy -<.095em>*\xycircle<.19em>{} \endxy}}
    % Inserts a unconnected control-on-0.
\newcommand{\ctrl}[1]{\control \qwx[#1] \qw}
    % Inserts a control and connects it to the object #1 wires below.
\newcommand{\ctrlo}[1]{\controlo \qwx[#1] \qw}
    % Inserts a control-on-0 and connects it to the object #1 wires below.
\newcommand{\targ}{*+<.02em,.02em>{\xy ="i","i"-<.39em,0em>;"i"+<.39em,0em> **\dir{-}, "i"-<0em,.39em>;"i"+<0em,.39em> **\dir{-},"i"*\xycircle<.4em>{} \endxy} \qw}
    % Inserts a CNOT target.
\newcommand{\qswap}{*=<0em>{\times} \qw}
    % Inserts half a swap gate.
    % Must be connected to the other swap with \qwx.
\newcommand{\multigate}[2]{*+<1em,.9em>{\hphantom{#2}} \POS [0,0]="i",[0,0].[#1,0]="e",!C *{#2},"e"+UR;"e"+UL **\dir{-};"e"+DL **\dir{-};"e"+DR **\dir{-};"e"+UR **\dir{-},"i" \qw}
    % Draws a multiple qubit gate starting at the current position and spanning #1 additional gates below.
    % #2 gives the label for the gate.
    % You must use an argument of the same width as #2 in \ghost for the wires to connect properly on the lower lines.
\newcommand{\ghost}[1]{*+<1em,.9em>{\hphantom{#1}} \qw}
    % Leaves space for \multigate on wires other than the one on which \multigate appears.  Without this command wires will cross your gate.
    % #1 should match the second argument in the corresponding \multigate.
\newcommand{\push}[1]{*{#1}}
    % Inserts #1, overriding the default that causes entries to have zero size.  This command takes the place of a gate.
    % Like a gate, it must precede any wire commands.
    % \push is useful for forcing columns apart.
    % NOTE: It might be useful to know that a gate is about 1.3 times the height of its contents.  I.e. \gate{M} is 1.3em tall.
    % WARNING: \push must appear before any wire commands and may not appear in an entry with a gate or label.
\newcommand{\gategroup}[6]{\POS"#1,#2"."#3,#2"."#1,#4"."#3,#4"!C*+<#5>\frm{#6}}
    % Constructs a box or bracket enclosing the square block spanning rows #1-#3 and columns=#2-#4.
    % The block is given a margin #5/2, so #5 should be a valid length.
    % #6 can take the following arguments -- or . or _\} or ^\} or \{ or \} or _) or ^) or ( or ) where the first two options yield dashed and
    % dotted boxes respectively, and the last eight options yield bottom, top, left, and right braces of the curly or normal variety.  See the Xy-pic reference manual for more options.
    % \gategroup can appear at the end of any gate entry, but it's good form to pick either the last entry or one of the corner gates.
    % BUG: \gategroup uses the four corner gates to determine the size of the bounding box.  Other gates may stick out of that box.  See \prop.

\newcommand{\rstick}[1]{*!L!<-.5em,0em>=<0em>{#1}}
    % Centers the left side of #1 in the cell.  Intended for lining up wire labels.  Note that non-gates have default size zero.
\newcommand{\lstick}[1]{*!R!<.5em,0em>=<0em>{#1}}
    % Centers the right side of #1 in the cell.  Intended for lining up wire labels.  Note that non-gates have default size zero.
\newcommand{\ustick}[1]{*!D!<0em,-.5em>=<0em>{#1}}
    % Centers the bottom of #1 in the cell.  Intended for lining up wire labels.  Note that non-gates have default size zero.
\newcommand{\dstick}[1]{*!U!<0em,.5em>=<0em>{#1}}
    % Centers the top of #1 in the cell.  Intended for lining up wire labels.  Note that non-gates have default size zero.
\newcommand{\Qcircuit}{\xymatrix @*=<0em>}
    % Defines \Qcircuit as an \xymatrix with entries of default size 0em.
\newcommand{\link}[2]{\ar @{-} [#1,#2]}
    % Draws a wire or connecting line to the element #1 rows down and #2 columns forward.
\newcommand{\pureghost}[1]{*+<1em,.9em>{\hphantom{#1}}}
    % Same as \ghost except it omits the wire leading to the left. 

%\renewcommand{\measureD}[1]{*{\xy*+=+<.5em>{\vphantom{\rule{0em}{.1em}#1}}*\cir{r_l};p\save*!R{#1} \restore\save+UC;+UC-<.15em,0em>*!R{\hphantom{#1}}+L **\dir{-} \restore\save+DC;+DC-<.15em,0em>*!R{\hphantom{#1}}+L **\dir{-} \restore\POS+UC-<.1em,0em>*!R{\hphantom{#1}}+L;+DC-<.15em,0em>*!R{\hphantom{#1}}+L **\dir{-} \endxy} \qw}
\newcommand{\redcontrol}{*!<0em,.025em>-=-{\color{red}\bullet\color{black}}}
\newcommand{\redqwx}[1][-1]{\color{red}\ar @{-} [#1,0]\color{black}}
\newcommand{\redctrl}[1]{\redcontrol \redqwx[#1] \qw}
\newcommand{\redtarg}{*!<0em,.019em>=<.79em,.68em>{\color{red}\xy {<0em,0em>*{} \ar @{ - } +<.4em,0em> \ar @{ - } -<.4em,0em> \ar @{ - } +<0em,.36em> \ar @{ - } -<0em,.36em>},<0em,-.019em>*+<.8em>\frm{o}\endxy} \color{black}\qw}

\newcommand{\bra}[1]{{\langle#1|}}
\newcommand{\ket}[1]{{|#1\rangle}}
\newcommand{\braket}[2]{{\langle#1|#2\rangle}}
\newcommand{\ketbra}[2]{{\ket{#1}\!\bra{#2}}}
\newcommand{\lbra}[1]{{\bra{\overline{#1}}}}
\newcommand{\lket}[1]{{\ket{\overline{#1}}}}
\newcommand{\abs}[1]{{\lvert #1\rvert}}	% since the delimiters do not scale, it might be a good idea to add a dummy {} at the end, so \abs{big expression}^2 has the superscript at a low height
\newcommand{\bigabs}[1]{{\big\lvert #1\big\rvert}}
\newcommand{\Bigabs}[1]{{\Big\lvert #1\Big\rvert}}

\newcommand{\norm}[1]{{\| #1 \|}}
\newcommand{\bignorm}[1]{{\big\| #1 \big\|}}
\newcommand{\Bignorm}[1]{{\Big\| #1 \Big\|}}
\newcommand{\Biggnorm}[1]{{\Bigg\| #1 \Bigg\|}}
%\newcommand{\trnorm}[1]{{\norm{#1}_{\mathrm{tr}}}}
\newcommand{\trnorm}[1]{{\| #1 \|_{\mathrm{tr}}}}
\newcommand{\bigtrnorm}[1]{{\bigl\| #1 \bigr\|_{\mathrm{tr}}}}	% by not calling \bignorm, the subscript height is independent of the argument
\newcommand{\Bigtrnorm}[1]{{\Bigl\| #1 \Bigr\|_{\mathrm{tr}}}}
\newcommand{\Biggtrnorm}[1]{{\Biggl\| #1 \Biggr\|_{\mathrm{tr}}}}

\newcommand{\eps}{{\epsilon}}
\newcommand{\binomial}[2]{\ensuremath{\left(\begin{smallmatrix}#1 \\ #2 \end{smallmatrix}\right)}}
\newcommand{\fastmatrix}[1]{\left(\begin{smallmatrix}#1\end{smallmatrix}\right)}
\newcommand{\smatrx}[1]{\ensuremath{\left(\begin{smallmatrix}#1\end{smallmatrix}\right)}}
\newcommand{\matrx}[1]{\ensuremath{\left(\begin{matrix}#1\end{matrix}\right)}}
\DeclareMathOperator{\Ex}{\operatorname{E}}
\DeclareMathOperator{\Tr}{\operatorname{Tr}}

\def\tensor {\otimes}
\def\adjoint{\dagger} %{*}

\def\A {{\mathcal A}}
\def\B {{\mathcal B}}
\def\C {{\bf C}}
\def\D {{\mathcal D}}
\def\E {{\mathcal E}}
\def\F {{\mathcal F}}
\def\G {{\mathcal G}}
\def\H {{\mathcal H}}
\let\Lstroke\L	\def\L {{\mathcal L}}		
\def\N {{\bf N}}
\def\cP {{\mathcal P}}
\def\R {{\bf R}}
\def\S {{\mathcal S}}
\def\U {{\mathcal U}}
\def\V {{\mathcal V}}

%% Complexity classes: 
\renewcommand{\P}{\ensuremath{\mathsf{P}}}%{{\mathcal{NP}}}
\newcommand{\NP}{\ensuremath{\mathsf{NP}}}%{{\mathcal{NP}}}
\newcommand{\IP}{\ensuremath{\mathsf{IP}}}%{{\mathcal{NP}}}
\newcommand{\PSPACE}{\ensuremath{\mathsf{PSPACE}}}%{{\mathcal{NP}}}
\newcommand{\BQP}{\ensuremath{\mathsf{BQP}}}%{{\mathcal{BQP}}}
\newcommand{\EXP}{\ensuremath{\mathsf{EXP}}}%{{\mathcal{NP}}}
\newcommand{\NEXP}{\ensuremath{\mathsf{NEXP}}}%{{\mathcal{NP}}}
\newcommand{\QIP}{\ensuremath{\mathsf{QIP}}}%{{\mathcal{NP}}}
\newcommand{\QMIP}{\ensuremath{\mathsf{QMIP}}}%{{\mathcal{NP}}}
\newcommand{\MIP}{\ensuremath{\mathsf{MIP}}}%{{\mathcal{NP}}}

\DeclareMathOperator{\Span}{\operatorname{Span}}
\DeclareMathOperator{\Range}{\operatorname{Range}}
\DeclareMathOperator{\Kernel}{\operatorname{Ker}}
\DeclareMathOperator{\poly}{\operatorname{poly}}
\DeclareMathOperator{\qpoly}{\operatorname{qpoly}}
\DeclareMathOperator{\rank}{\operatorname{rank}}
\newcommand{\identity}{\ensuremath{\boldsymbol{1}}} %\mathbb{I}
\newcommand{\Id}{\identity} 
\DeclareMathOperator{\CNOT}{\operatorname{CNOT}}
\DeclareMathOperator{\SWAP}{\operatorname{SWAP}}

%\newtheorem*{maintheorem}{Main Theorem}

\newcommand{\hugelpar}[1]{\left(\vbox to #1{}\right.}
\newcommand{\hugerpar}[1]{\left.\vbox to #1{}\right)}
\newcounter{sprows}
\newcounter{spcols}
\newlength{\spheight}
\newlength{\spraise}
\newcommand{\spleft}[2][0pt]{\multirow{\value{sprows}}{*}{%
	\vbox to \spraise{\vss\hbox{$#2 \hugelpar{\spheight}\hskip -#1$}\vss}}}
\newcommand{\spright}[2][0pt]{\multirow{\value{sprows}}{*}{%
	\vbox to \spraise{\vss\hbox{\hskip -#1 $\hugerpar{\spheight} #2$}\vss}}}

\newcommand{\comment}[1]{\emph{\color{blue}Comment:\color{black} #1}} % use for simply removing comments
\newlength{\commentslength}
\newcommand{\comments}[1]{
\hspace{-2\parindent}
\addtolength{\commentslength}{-\commentslength}
\addtolength{\commentslength}{\linewidth}
\addtolength{\commentslength}{-\parindent}
\fcolorbox{blue}{white}{\smallskip\begin{minipage}[c]{\commentslength}
\emph{Comments:}\begin{itemize}#1\end{itemize}\end{minipage}}\bigskip
}
%\renewcommand{\comment}[1]{}\renewcommand{\comments}[1]{}
\newcommand{\rem}[1]{}

%\numberwithin{equation}{section} % makes Eq. numbers (section.number)

\newtheorem{theorem}{Theorem}[section]
\newtheorem{lemma}[theorem]{Lemma}
\newtheorem{corollary}[theorem]{Corollary}
\newtheorem{claim}[theorem]{Claim}
\newtheorem{fact}[theorem]{Fact}
\newtheorem{proposition}[theorem]{Proposition}
\newtheorem{conjecture}[theorem]{Conjecture}

%\theoremstyle{definition}
\newtheorem{definition}[theorem]{Definition}
\newtheorem{condition}[theorem]{Condition}
%\theoremstyle{remark}
\newtheorem{remark}[theorem]{Remark}
%% some font options include \itshape (preferred), \slshape (same as italic, but with broader spacing), \bfseries (bold), \normalfont
%\newtheoremstyle{definition}{}{}{\normalfont}{}{\itshape}{.}{ }{}
%\theoremstyle{definition}
\newtheorem{example}[theorem]{Example}

\newfont{\subsubsecfnt}{ptmri8t at 11pt}
\renewcommand{\subparagraph}[1]{\smallskip{\subsubsecfnt #1.}}

%% The first versions below hyperlink the whole reference, while the second versions only hyperlink the number
\newcommand{\eqnref}[1]{\hyperref[#1]{{(\ref*{#1})}}}
\newcommand{\thmref}[1]{\hyperref[#1]{{Theorem~\ref*{#1}}}}
\newcommand{\lemref}[1]{\hyperref[#1]{{Lemma~\ref*{#1}}}}
\newcommand{\corref}[1]{\hyperref[#1]{{Corollary~\ref*{#1}}}}
\newcommand{\defref}[1]{\hyperref[#1]{{Definition~\ref*{#1}}}}
\newcommand{\secref}[1]{\hyperref[#1]{{Section~\ref*{#1}}}}
\newcommand{\figref}[1]{\hyperref[#1]{{Figure~\ref*{#1}}}}
\newcommand{\tabref}[1]{\hyperref[#1]{{Table~\ref*{#1}}}}
\newcommand{\remref}[1]{\hyperref[#1]{{Remark~\ref*{#1}}}}
\newcommand{\appref}[1]{\hyperref[#1]{{Appendix~\ref*{#1}}}}
\newcommand{\claimref}[1]{\hyperref[#1]{{Claim~\ref*{#1}}}}
\newcommand{\factref}[1]{\hyperref[#1]{{Fact~\ref*{#1}}}}
\newcommand{\propref}[1]{\hyperref[#1]{{Proposition~\ref*{#1}}}}
\newcommand{\exampleref}[1]{\hyperref[#1]{{Example~\ref*{#1}}}}
\newcommand{\conjref}[1]{\hyperref[#1]{{Conjecture~\ref*{#1}}}}

%
%\newcommand{\eqnref}[1]{{(\hyperref[#1]{\ref*{#1}})}}
%\newcommand{\thmref}[1]{{Theorem~\hyperref[#1]{\ref*{#1}}}}
%\newcommand{\lemref}[1]{{Lemma~\hyperref[#1]{\ref*{#1}}}}
%\newcommand{\corref}[1]{{Corollary~\hyperref[#1]{\ref*{#1}}}}
%\newcommand{\defref}[1]{{Definition~\hyperref[#1]{\ref*{#1}}}}
%\newcommand{\secref}[1]{{Section~\hyperref[#1]{\ref*{#1}}}}
%\newcommand{\figref}[1]{{Figure~\hyperref[#1]{\ref*{#1}}}}
%\newcommand{\tabref}[1]{{Table~\hyperref[#1]{\ref*{#1}}}}
%\newcommand{\remref}[1]{{Remark~\hyperref[#1]{\ref*{#1}}}}
%\newcommand{\appref}[1]{{Appendix~\hyperref[#1]{\ref*{#1}}}}
%\newcommand{\claimref}[1]{{Claim~\hyperref[#1]{\ref*{#1}}}}
%\newcommand{\propref}[1]{{Proposition~\hyperref[#1]{\ref*{#1}}}}
%\newcommand{\exampleref}[1]{{Example~\hyperref[#1]{\ref*{#1}}}}
%\newcommand{\conjref}[1]{{Conjecture~\hyperref[#1]{\ref*{#1}}}}

\allowdisplaybreaks[1]
%\sloppy

%% Paper-specific macros:
\newcommand{\ADV} {\mathrm{Adv}}
\newcommand{\ADVpm} {\mathrm{Adv}^{\pm}}
\def\CZ {C\!Z}	% control-Z gate
\DeclareMathOperator{\abst}{\operatorname{abs}}
%\newcommand{\B}{B}	% \{0,1\}	{{\bf Z}_2}

\DeclareMathOperator{\depth}{\operatorname{depth}}
\DeclareMathOperator{\AND}{\ensuremath{\operatorname{AND}}}
\DeclareMathOperator{\OR}{\ensuremath{\operatorname{OR}}}
\DeclareMathOperator{\MAJ}{{\operatorname{MAJ}_3}}
\DeclareMathOperator{\EQUAL}{{\operatorname{EQUAL}}}
\DeclareMathOperator{\EXACT}{{\operatorname{EXACT}}}

\def\COLOR{}
\ifdefined\COLOR
\newcommand{\Alice}[1] {{\color{red} {#1}}}
\newcommand{\Bob}[1] {{\color{blue} {#1}}}
\else
\newcommand{\Alice}[1] {{\color{black} {#1}}}
\newcommand{\Bob}[1] {{\color{black} {#1}}}
\fi

\newcommand{\EPRstate}{{\mathrm{EPR}}}

\DeclareMathOperator{\ima}{Im}

\usepackage{array}