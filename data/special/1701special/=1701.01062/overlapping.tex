\documentclass[preprintnumbers,11pt,onecolumn]{article}
\pdfoutput=1	% needed for the arXiv to properly compile with pdflatex

\input{header}

\renewcommand{\comment}[1]{}
\renewcommand{\comments}[1]{}

\begin{document}
\def\compilefullpaper{}

\title{Overlapping qubits}
\author{Rui Chao$^1$ \and Ben W. Reichardt$^1$ \and Chris Sutherland$^1$ \and Thomas Vidick$^2$}
\date{}


\maketitle
\footnotetext[1]{University of Southern California}
\footnotetext[2]{Department of Computing and Mathematical Sciences, California Institute of Technology}


%\begin{figure}\center
  %\missingfigure[figheight=.10\textheight, figwidth=\textwidth]{Graphical Abstract}
%  \includegraphics[height=.15\textheight]{graphical_abstract-crop}
%  \caption{Scheme of analyses involving the core structural connectivity matrix.\label{fig:process-illustration}}
%\end{figure}

Isolating the common brain connectivity network from a population is a main problem in current neuroscience~\cite{Bullmore2009,Gong2009,Wassermann2016}. Recent evidence suggests that there's a common and densely connected brain connectome across humans~\cite{Bassett2013}. In this work we present a new approach for selecting these common connections, combining recent topological hypotheses~\cite{Bassett2013}  and  current methods~\cite{Gong2009,Wassermann2016}.

Finding the common brain connectome across subjects has the potential to increase our understanding of the relationship between function and structure in the brain. This relationship is one of the main open questions in neuroscience~\cite{Bullmore2009,Donahue2016}. Moreover, knowledge about the most common connections in a population will facilitate clinical and cognitive Diffusion MRI analyses by reducing the number of surveyed connections, increasing the statistical power of those analyses. Finding the common connectome will also allow us to increase our knowledge about the brain structure by comparing core networks across different populations.

We formalize the problem of selecting the common connections combining graph theory and statistics. Then, we prove that the problem is \NP-Hard and propose a polynomial-time algorithm to find approximate solutions. To do this, we develop an exact polynomial-time algorithm for a relaxed version of the problem and prove the algorithm's correctness and complexity.

Currently, the most used algorithm to extract a population's core structural connectivity network (CSNC)~\cite{Gong2009} uses an statistical approach: first, compute a connectivity matrix for each subject; then, analize each connection separately with a hypothesis test, using as null hypothesis that that edge is not present in the population; finally, construct a binary graph with the edges for which the null hypothesis was rejected. The main problem of Gong et al.'s~\cite{Gong2009} algorithm is that the resulting graph can be a set of disconnected subgraphs. Moreover, recent studies have shown that the brain has a \emph{core} network tightly connected and a sparsely connected \emph{outer} one~\cite{Bassett2013}. In other words, this approach ignores the resulting network's topology. Performing statistical analyses in a feature set chosen by hypothesis testing incurs in the double dipping problem~\cite{Kriegeskorte2009}.

A newer approach to solve the CSNC problem, designed by Wassermann et al.~\cite{Wassermann2016}, uses graph theory to get a connected CSCN: first, compute a binary connectivity graph for each subject using a threshold;  for each possible connection compute the ``cost'' of including or excluding it from the common graph by evaluating in how many subjects that connection is present; finally, construct the binary graph with all the edges that is ``cheaper'' to include than to exclude and connect the resulting graph if it's disconnected, using the minimum possible cost. This algorithm guarantees that the resulting graph is connected, but the connection binarization discards significant information for the resulting common network. In other words, it discards information of the probability of each connection being in the brain. This is problematic because the resulting graph may include edges for which tractography assigned a very low existence probability across subjects. Also, the outer part of the brain, the connections which do not result in the core network, should also be sparsely connected~\cite{Bassett2013}, which this algorithm does not enforce.

In this work we propose, for the first time, a polynomial-time algorithm to obtain the CSCN of a population  addressing the issues listed above. Our algorithm combines the recent graph-theoretical approach~\cite{Wassermann2016} with the statistical awareness of the most popular one~\cite{Gong2009}. We start by formalizing the problem, which allow us to prove that it's \NP-Hard. Then, we propose a first algorithm that solves a relaxed version of the problem in an exact way, giving the best possible core graph for our formalization. Then, we adapt it to guarantee a connected result, agreeing with recent evidence on structural connectivity network topology \cite[e.g.]{Bassett2013}. Finally, we validate our approach using 300 subjects from the HCP database and comparing the performance of the networks obtained by our new approach, Wassermann et al.'s~\cite{Wassermann2016} and Gong et al.'s~\cite{Gong2009} predicting connectivity values from handedness in the core network.


\section{What is a qubit?  When are qubits in tensor product?}

As explained in the introduction, we take a basis-independent, operator-centric view of what it means to have a qubit, or multiple independent qubits, in an a priori unstructured Hilbert space~$\H$.  The following definition formalizes these notions.  Notation: Let $[n] = \{1, 2, \ldots, n\}$, and $I = \big(\begin{smallmatrix}1&0\\0&1\end{smallmatrix}\big)$, $\sigma^x = \big(\begin{smallmatrix}0&1\\1&0\end{smallmatrix}\big)$, $\sigma^y = \big(\begin{smallmatrix}0&-i\\i&0\end{smallmatrix}\big)$ and $\sigma^z = \big(\begin{smallmatrix}1&0\\0&-1\end{smallmatrix}\big)$ be the Pauli matrices.  The commutator is $[S, T] = S T - T S$, and the anticommutator is $\{S, T\} = S T + T S$.  When we write, e.g., ``$S_j$ for $S \in \{X, Z\}$" we mean the set $\{X_j, Z_j\}$, i.e., the letter~$S$ is meant to be directly replaced by $X$ or~$Z$.  

\begin{definition}
A \emph{qubit} in a Hilbert space $\H$ is a pair of anti-commuting reflections $(X, Z)$ on~$\H$.  The \emph{overlap} between two qubits $(X_1, Z_1)$ and $(X_2, Z_2)$ is given by $\max_{S, T \in \{X, Z\}} \norm{[S_1, T_2]}$.  The qubits are in \emph{tensor product} if they have overlap~$0$; in this case we also say that the qubits are \emph{independent}.  
\end{definition}

The following simple lemma ties this definition to the more usual one of a qubit as defined by a factorization $\H \simeq \C^2 \otimes \H'$.  The lemma is a special case of \thmref{t:whatismanyqubits} below.  

\begin{lemma} \label{t:whatisaqubit}
Let $X$ and $Z$ be reflections (Hermitian operators that square to the identity) on a separable Hilbert space~$\H$ such that $X$ and $Z$ anti-commute: $\{X, Z\} = 0$.  Then there exists a separable space $\H'$ such that $\H$ is isomorphic to $\C^2\otimes \H'$, and up to a unitary change of basis the reflections $X, Z$ are the standard Pauli operators: 
\begin{equation*}
X = \sigma^x \otimes \identity_{\H'}, \qquad Z = \sigma^z \otimes \identity_{\H'}
 \enspace .
\qedhere
\end{equation*}
\end{lemma}

The following theorem justifies our definition of two qubits being in ``tensor product'' when their overlap is $0$, or equivalently when the associated reflections pairwise commute.  

\begin{theorem} \label{t:whatismanyqubits}
Suppose that $X_1, Z_1, \ldots, X_n, Z_n$ are reflections on~$\H$ such that for all~$j$, $\{X_j, Z_j\} = 0$ and furthermore for all $i \neq j$ and $S,T\in\{X,Z\}$, $S_i$ and $T_j$ pairwise commute, $[S_i, T_j] = 0$.  
Then there exists a separable space $\H''$ such that $\H$ is isomorphic to $(\C^2)^{\otimes n} \otimes \H''$, and up to a unitary change of basis the reflections $X_j, Z_j$ are the standard Pauli operators on $n$ qubits: 
\begin{equation*}
\begin{split}
X_1 &= \sigma^x \otimes I^{\otimes (n-1)} \otimes \identity_{\H''} \\
Z_1 &= \sigma^z \otimes I^{\otimes (n-1)} \otimes \identity_{\H''}
\end{split}
\qquad\quad \cdots \qquad\quad
\begin{split}
X_n &= I^{\otimes (n-1)} \otimes \sigma^x \otimes \identity_{\H''} \\
Z_n &= I^{\otimes (n-1)} \otimes \sigma^z \otimes \identity_{\H''}
 \enspace . 
\end{split}
\end{equation*}
\end{theorem}

\begin{proof}
Let $X = X_1$, $Z = Z_1$.  As $Z^2 = \identity$, $\Pi_\pm = \tfrac12 (\identity \pm Z)$ are projections, with $\Pi_+ + \Pi_- = \identity$, $\Pi_+ - \Pi_- = Z$ and $\Pi_+ \Pi_- = \Pi_- \Pi_+ = 0$.  Multiplying both sides of $\{X, Z\} = 0$ by $\Pi_\pm$ yields $\Pi_\pm X \Pi_\pm = 0$, i.e., $X = \Pi_+ X \Pi_- + \Pi_- X \Pi_+$.  Then $X^2 = \identity$ implies that $\Pi_\pm X \Pi_\mp X \Pi_\pm = \Pi_\pm$; and comparing the ranks of both sides gives $\mathrm{Rank}(\Pi_\mp) \geq \mathrm{Rank}(\Pi_\pm)$, i.e., $\mathrm{Rank}(\Pi_+) = \mathrm{Rank}(\Pi_-)$.  

Let $\ket{u_1^\pm}, \ket{u_2^\pm}, \ldots$ be an orthonormal basis for $\mathrm{Range}(\Pi_\pm)$.  Let $S = \sum_j (\ketbra{u_j^+}{u_j^-} + \ketbra{u_j^-}{u_j^+})$.  Then $S = S^\adjoint$, $S^2 = \identity$ and $S \Pi_\pm = \Pi_\mp S$.  Let $U = \Pi_+ X \Pi_- S + \Pi_-$.  $U$ is unitary: $U U^\adjoint = U^\adjoint U = \identity$.  Furthermore, $U^\adjoint Z U = Z$, and $U^\adjoint X U = S$.  Relabeling the basis elements $\ket{0, j} = \ket{u_j^+}$, $\ket{1, j} = \ket{u_j^-}$, we obtain $U^\adjoint Z U = \sigma^z \otimes \identity$ and $U^\adjoint X U = \sigma^x \otimes \identity$, as desired.  

Now consider $X_2$.  In the above basis, it can be expanded as $I \otimes A + \sum_{\beta \in \{x,y,z\}} \sigma^\beta \otimes B_\beta$, but the commutation relationships $[X_2, X_1] = [X_2, Z_1] = 0$ imply that each $B_\beta = 0$.  Similarly, all the reflections $Z_2, \ldots, X_n, Z_n$ act trivially on the first $\C^2$ register.  Inductively repeating the above argument for $X_1$ and~$Z_1$ gives the theorem.  
\end{proof}

Registers that are in tensor product are independent of each other, in the sense that for a quantum state $\ket \psi \in \H' \otimes \H''$, a quantum operation on $\H'$ cannot affect the reduced density matrix $\Tr_{\H'} \ketbra \psi \psi$ in the other register.  It should be noted, though, that a qubit can simultaneously have maximal overlap with many other mutually independent qubits.  For example, for $n$ odd, $X = (\sigma^x)^{\otimes n}$ and $Z = (\sigma^z)^{\otimes n}$ are anti-commuting reflections, defining a qubit, such that for every $j \in [n]$, $\norm{[X, \sigma^z_j]} = \norm{[Z, \sigma^x_j]} = 2$.  (Similarly, in $(\C^2)^{\otimes n}$, for a Haar random unitary~$U$, $\norm{[U \sigma^\alpha_1 U^\dagger, \sigma^\beta_j]}$ will be concentrated around the maximal value of~$2$.)  Thus the norm of the reflections' commutator is not a ``monogamous" measure of qubit overlap.  


\section{Packing qubits} \label{s:packing}

How many pairwise $\eps$-overlapping qubits can be packed into $2^n$ dimensions?  Formally, in $2^n$ dimensions, we wish to place $2 m$ reflections $(X_1, Z_1), \ldots, (X_m, Z_m)$ such that each pair $(X_j,Z_j)$ defines a qubit, so that $\{ X_j, Z_j \} = 0$, and operators with different indices nearly commute: $\norm{[S_i, T_j]} \leq \epsilon$ for $i \neq j$ and $S, T\in \{X,Z\}$. How large can~$m$ be?  

One's intuition might be pulled in either of two directions.  From the perspective of information theory, Nayak's private information retrieval bound $m \leq n / (1 - H(p))$~\cite{Nayak99privateinformationretrieval} suggests that packing $\omega(n)$ qubits into $2^n$ dimensions is unlikely to be possible.  However, a formal connection between our problem and private information retrieval is not obvious: the existence of $m$ pairs of approximately commuting qubit operators does not imply that there exists a family of $2^m$ states that could be used to encode $m$ bits with a good probability of recovery. 

From a geometric perspective the problem can be viewed as one of packing subspaces.  Each reflection $R_j$ is about a certain subspace, projected to by $\tfrac12(I + R_j)$. As explained in the previous section, the anticommutation condition implies that $X_j$ and~$Z_j$ correspond to subspaces with all principal angles $\pi/4$, while the approximate commutation condition $\norm{[S_i, T_j]} \leq \epsilon$ translates into the corresponding subspaces making principal angles close to~$0$ or~$\pi/2$. By analogy to the problem of packing nearly orthogonal unit vectors\footnote{For vector packing upper bounds on~$m$, see, e.g., \cite{KabatjanskiiLevenstein78vectorpacking}, \cite[Lemma~9.1]{Alon03extremal1}, \cite{Tao13vectorpacking}.} one might guess that as long as $\epsilon$ is not required to go to $0$ too fast with $n$, $m$ can be exponential in~$n$.  

The results in this section demonstrate that the geometric intuition is more accurate.  \thmref{t:manynearlycommutingprojections} shows that for sufficiently small $\epsilon$ (inverse linear in~$n$), no more than $m \leq n$ $\epsilon$-overlapping qubits can fit in $2^n$ dimensions.  In contrast, \thmref{t:qubitpacking} shows that as long as $\epsilon = \Omega(1)$, $m$ can be exponential in~$n$; more generally $m = \omega(n)$ for any $\epsilon = \omega(\sqrt{(\log n) / n})$.  For the range of overlaps $1 / n \lesssim \epsilon \lesssim \sqrt{(\log n) / n}$, we do not know whether strictly more than $n$ qubits can be packed into $2^n$ dimensions.  


\subsection{Lower bound: packing exponentially many qubits in $2^n$ dimensions} \label{s:packingqubitslowerbound}

We give a randomized construction that packs $m = e^{\Theta(n \epsilon^2)}$ qubits into $2^n$ dimensions.  This beats the trivial $m = n$ for $\epsilon = \Omega(\sqrt{(\log n) / n})$, and is exponential in~$n$ for constant $\epsilon > 0$.  

\begin{theorem} \label{t:qubitpacking}
There exist $2^n$-dimensional reflections $X_1, Z_1, \ldots, X_m, Z_m$, for $m = e^{\Omega(n \epsilon^2)}$, such that $\{ X_j, Z_j \} = 0$ and $\norm{[S_i, T_j]} = O(\epsilon)$ for all $i \neq j$ and $S,T\in\{X,Z\}$.
\end{theorem}

\begin{proof}
By the Johnson-Lindenstrauss Lemma~\cite{JohnsonLindenstrauss84, DasguptaGupta03JohnsonLindenstrauss}, $e^{n \epsilon^2 / 4}$ unit vectors can be chosen in~$\R^{2 n}$ so that for any pair $\ket u, \ket v$, $\abs{\braket u v} \leq \epsilon$.  Collecting these vectors in triples, we obtain $m = \tfrac13 e^{n \epsilon^2 / 4}$ three-dimensional subspaces with the angles between any two in the range $[\tfrac\pi2 - O(\epsilon), \tfrac\pi2]$. Let $\{ \ket{e_j}, \ket{f_j}, \ket{g_j} \}$, for $j \in [m]$, be orthonormal bases for the subspaces.  

Let $C_1, \ldots, C_{2 n}$ denote a $2^n$-dimensional representation of the Clifford algebra, i.e., Hermitian matrices that satisfy $\{C_i, C_j\} = 2 \delta_{ij} \Id$.  For each $j \in [m]$, let 
\begin{align*}
E_j &= \sum_k \braket{k}{e_j} \, C_k &
F_j &= \sum_k \braket{k}{f_j} \, C_k &
G_j &= \sum_k \braket{k}{g_j} \, C_k
 \enspace .
\end{align*}
Then it is easy to check that for distinct $S, T \in \{E, F, G\}$, $\{S_j, T_j\} = 0$ and $\norm{\{S_i, T_j\}} = O(\epsilon)$ for $i \neq j$.  Let $X_j = i E_j F_j$ and $Z_j = i E_j G_j$; these matrices are Hermitian, square to $\identity$, and anti-commute.  Moreover, for $i \neq j$ and $S,T \in \{X, Z\}$, we have $\norm{[S_i, T_j]} = O(\epsilon)$.  
\end{proof}

\appref{s:qubitpackingprooftwo} gives an alternative proof of \thmref{t:qubitpacking} using the exterior algebra.  


\subsection{Upper bound: Separating overlapping qubit operators} \label{s:packingqubitsupperbound}

We provide two different methods for creating independent qubits from partially overlapping qubits.  The first argument, given in \secref{s:blockdiagonalization}, performs a careful analysis of a sequential block-diagonalization procedure.  The second argument, in \secref{s:swapnorm}, is simpler but requires the introduction of a larger Hilbert space in which to define the approximating operators.  


\subsubsection{Separating nearly commuting projections} \label{s:blockdiagonalization}

We first consider the case of separating projections that nearly commute pairwise.  

\begin{theorem} \label{t:manynearlycommutingprojections}
Let $P_1, \ldots, P_n$ be projections on a finite-dimensional Hilbert space such that for some $\epsilon \leq \tfrac{1}{32 n}$, 
\begin{equation*}
\norm{[P_i, P_j]} \leq \epsilon \qquad \text{for all $i, j$.}
\end{equation*}
Then there exist projections $Q_1, \ldots, Q_n$ with, for all $i, j$, 
\begin{align*}
[Q_i, Q_j] &= 0 \\
\norm{P_i - Q_i} &\leq 8 n \epsilon
 \enspace .
\end{align*}
\end{theorem}

The bound in \thmref{t:manynearlycommutingprojections} is nearly tight; see \lemref{t:movementlowerbound} below.  

The proof of the theorem is constructive.  It uses two basic operations, that we analyze with two lemmas.  First we block-diagonalize operators with respect to a projection~$Q$ so that they commute with~$Q$.  The first lemma bounds how block-diagonalizing two operators affects their commutator.  

\begin{lemma} \label{t:blockdiagonalizedcommutator}
Let $Q$ be a projection, and for operators $P_i$, $i = 1, 2$, let $P_i' = Q P_i Q + (\identity-Q) P_i (\identity-Q)$.  Then $[Q, P_i'] = 0$, $\norm{P_i' - P_i} = \norm{[Q, P_i]}$, and 
\begin{equation*}\begin{split}
\norm{[P_1', P_2']} &\leq \norm{[P_1, P_2]} + 2 \norm{[Q, P_1]} \cdot \norm{[Q, P_2]}
 \enspace .
\end{split}\end{equation*}
\end{lemma}

\begin{proof}
Work in a basis in which $Q$ is diagonal: $Q = \fastmatrix{\identity & 0\\0 & 0}$.  Then $P_i = \big(\begin{smallmatrix}A_i & B_i \\ C_i & D_i\end{smallmatrix}\big)$ and $P_i' = \big(\begin{smallmatrix}A_i & 0 \\ 0 & D_i\end{smallmatrix}\big)$.  As $[Q, P_i] = \big(\begin{smallmatrix}0 & B_i \\ -C_i & 0\end{smallmatrix}\big)$, $\norm{P_i' - P_i} = \max\{ \norm{B_i}, \norm{C_i} \} = \norm{[Q, P_i]}$.  We also compute 
\begin{align*}
[P_1, P_2] &= \fastmatrix{
[A_1, A_2] + B_1 C_2 - B_2 C_1
& A_1 B_2 + B_1 D_2 - A_2 B_1 - B_2 D_1 \\
C_1 A_2 + D_1 C_2 - C_2 A_1 - D_2 C_1 
& [D_1, D_2] + C_1 B_2 - C_2 B_1
}
 \enspace .
\end{align*}
Each diagonal block in $[P_1, P_2]$ above, $Q [P_1, P_2] Q$ and $(\identity - Q) [P_1, P_2] (\identity - Q)$, must have norm at most $\norm{[P_1, P_2]}$.  The claimed bound for $\norm{[P_1', P_2']} = \max\{ \norm{[A_1, A_2]}, \norm{[D_1, D_2]} \}$ follows.  
\end{proof}

When one block-diagonalizes a projection, the result might not be a projection.  The second basic operation consists in rounding the eigenvalues to the closest integer, $0$ or $1$.  The second lemma bounds how this affects the commutator with another operator.  

\begin{lemma} \label{t:perturbedcommutator}
Let $Q$ be a projection and $Q'$ Hermitian with $[Q, Q'] = 0$ and $\norm{Q - Q'} < 1/2$.  Then for any Hermitian~$P$, 
\begin{equation*}
\norm{[Q, P]} \leq \frac{\norm{[Q', P]}}{1 - 2 \norm{Q - Q'}}
 \enspace .
\end{equation*}
\end{lemma}

\noindent
This bound can be much stronger than the trivial $\norm{[Q, P]} \leq \norm{[Q', P]} + 2 \norm{P} \norm{Q - Q'}$.\footnote{For $P \succeq 0$, trivially $\norm{[Q, P]} \leq \norm{[Q', P]} + \norm{[Q - Q', P - \tfrac{\norm{P}}{2} \identity]} \leq \norm{[Q', P]} + \norm{P} \norm{Q - Q'}$, but \lemref{t:perturbedcommutator} is still stronger.}  It follows by substituting $A = \Big(\begin{smallmatrix}0 & P (2 Q - \identity) \\ (2Q - \identity) P & 0\end{smallmatrix}\Big)$, $B = \Big(\begin{smallmatrix}0 & (2 Q - \identity) P \\ P (2 Q - \identity) & 0\end{smallmatrix}\Big)$ and $\Gamma = \abs{2 Q' - \identity} \oplus \abs{2 Q' - \identity}$ into the following theorem, and using $\abs{2 Q' - \identity} (2 Q - \identity) = (2 Q - \identity) \abs{2 Q' - \identity} = 2 Q' - \identity$.  

\begin{theorem}[{\cite[Theorem~1]{BhatiaDavisKittaneh91perturbcommutator}}] \label{t:bhatiaroundingeigenvalues}
If $A$ and $B$ are Hermitian, and $\Gamma \succ 0$, then 
\begin{equation*}
\norm{A - B} \leq \norm{\Gamma^{-1}} \cdot \norm{A \Gamma - \Gamma B}
 \enspace .
\end{equation*}
\end{theorem}

\begin{proof}[Proof of \thmref{t:manynearlycommutingprojections}]
We proceed inductively.  The induction hypothesis is that we have defined $Q_1, \ldots, Q_k, P_{k+1}^{(k)}, \ldots, P_{n}^{(k)}$ such that 
\begin{itemize}
\item $0 \preceq P_j^{(k)} \preceq \identity$, $\norm{P_j^{(k)} - P_j} \leq \delta_k$, $\norm{[P_i^{(k)}, P_j^{(k)}]} \leq \epsilon_k$.  
\item $Q_1, \ldots, Q_k$ are projections, commuting with each other and all $P_j^{(k)}$, with $\norm{P_k - Q_k} \leq 2 \delta_{k-1}$.  
\end{itemize}
For the base case, $\delta_0 = 0$ and $\epsilon_0 = \epsilon$.  

In the induction step, we let $Q_{k+1}$ be the projection formed by rounding $P_{k+1}^{(k)}$'s eigenvalues to $0$ or~$1$, and define $P_{k+2}^{(k+1)}, \ldots, P_n^{(k+1)}$ by block-diagonalizing the $P_j^{(k)}$ operators with respect to $Q_{k+1}$: 
\begin{equation*}
P_j^{(k+1)} = Q_{k+1} P_j^{(k)} Q_{k+1} + (\identity-Q_{k+1}) P_j^{(k)} (\identity-Q_{k+1})
 \enspace .
\end{equation*}
Indeed, then $\norm{Q_{k+1} - P_{k+1}} \leq \norm{P_{k+1}^{(k)} - P_{k+1}} + \norm{Q_{k+1} - P_{k+1}^{(k)}} \leq 2 \delta_k$.  Also, $0 \preceq P_j^{(k+1)} \preceq \identity$.  Using \lemref{t:blockdiagonalizedcommutator}, we compute 
\begin{align*}
\norm{P_j^{(k+1)} - P_j}
&\leq \norm{P_j^{(k)} - P_j} + \norm{P_j^{(k+1)} - P_j^{(k)}} \\
&\leq \delta_k + \norm{[Q_{k+1}, P_j^{(k)}]} \\
\norm{[P_i^{(k+1)}, P_j^{(k+1)}]}
&\leq \norm{[P_i^{(k)}, P_j^{(k)}]} + 2 \norm{[Q_{k+1}, P_i^{(k)}]} \cdot \norm{[Q_{k+1}, P_j^{(k)}]}
 \enspace .
\end{align*}
Thus we may take $\delta_{k+1} = \delta_k + \max_j \norm{[Q_{k+1}, P_j^{(k)}]}$ and $\epsilon_{k+1} = \epsilon_k + 2 \max_j \norm{[Q_{k+1}, P_j^{(k)}]}{}^2$.  It remains to bound $\max_j \norm{[Q_{k+1}, P_j^{(k)}]}$.  

The naive bound $\norm{[Q_{k+1}, P_j^{(k)}]} \leq \norm{[P_{k+1}^{(k)}, P_j^{(k)}]} + 2 \norm{Q_{k+1} - P_{k+1}^{(k)}} \leq \epsilon_k + 2 \delta_k$ is no good, as it allows the errors to grow exponentially with~$k$.  Instead, applying \lemref{t:perturbedcommutator} gives 
\begin{equation*}
\bignorm{[Q_{k+1}, P_j^{(k)}]}
\leq \frac{\epsilon_k}{1 - 2 \delta_k}
 \enspace .
\end{equation*}
Provided that all $\epsilon_k \leq 2 \epsilon$ and $\delta_k \leq 1/4$, $(1 - 2 \delta_k)^{-1} \leq 2$, and we obtain the recursions 
\begin{equation*}\begin{split}
\delta_{k+1} &\leq \delta_k + 2 \epsilon_k \leq \delta_k + 4 \epsilon \\
\epsilon_{k+1} &\leq \epsilon_k + 8 \epsilon_k^2 \leq \epsilon_k + 32 \epsilon^2
 \enspace .
\end{split}\end{equation*}
Thus $\delta_{k+1} \leq 4 (k+1) \epsilon$ and $\epsilon_{k+1} \leq \epsilon + 32 k \epsilon^2$.  Given $\epsilon \leq \tfrac{1}{32 n}$, indeed $\epsilon_k \leq 2 \epsilon$ and $\delta_k \leq 1/4$.  
\end{proof}


\subsubsection{Separating partially overlapping qubits}

The following theorem is an extension of \thmref{t:manynearlycommutingprojections} which allows us to separate $\eps$-overlapping qubits.   

\begin{theorem} \label{t:manynearlyindependentqubits}
Let $X_1, Z_1, \ldots, X_n, Z_n$ be Hermitian matrices each having eigenvalues in the range $[-1, -1+\epsilon] \cup [1-\epsilon, 1]$, and satisfying $\norm{\{X_j, Z_j\}} \leq \epsilon$ and $\norm{[S_i, T_j]} \leq \epsilon$ for all $i \neq j$ and $S, T \in \{X, Z\}$.  Assume $\epsilon / (1-\epsilon)^2 \leq \tfrac{1}{64 n}$.  
Then there exist reflections $X_1', Z_1', \ldots, X_n', Z_n'$ with $\{X_j', Z_j'\} = 0$, and $[S_i', T_j'] = 0$ and $\norm{S_j' - S_j} \leq 4 n \epsilon / (1-\epsilon)^2 + \epsilon$ for all $i \neq j$ and $S, T \in \{X, Z\}$.  
\end{theorem}

\begin{proof}
Let $\H$ be the finite-dimensional Hilbert space on which the matrices act.  Introduce $n$ additional qubits, and on $(\C^2)^{\otimes n} \otimes \H$, define 
\begin{align*}
R_{2j-1}' &= \sigma_j^x \otimes X_j \\
R_{2j}' &= \sigma_j^z \otimes Z_j
 \enspace ,
\end{align*}
for $j = 1, \ldots, n$, where $\sigma_j^x$ and $\sigma_j^z$ are the standard Pauli operators acting on the $j$th added qubit.  

For Pauli operators $\sigma$ and~$\tau$, 
\begin{equation*}
[\sigma \otimes A, \tau \otimes B] =  \begin{cases} 
(\sigma \tau) \otimes [A, B] & \text{if $[\sigma, \tau] = 0$} \\
(\sigma \tau) \otimes \{A, B\} & \text{if $\{\sigma, \tau\} = 0$ \enspace .}
\end{cases}
\end{equation*}
Thus for all $i, j$, 
\begin{equation*}
\norm{[R_i', R_j']} \leq \epsilon
 \enspace .
\end{equation*}

Define reflections $R_1, \ldots, R_{2n}$ by rounding to $\pm 1$ the eigenvalues of each of $R_1', \ldots, R_{2n}'$.  The operators $R_j$ still have the form $(\text{Pauli}) \otimes (\text{Reflection})$.  By \thmref{t:bhatiaroundingeigenvalues}, 
\begin{equation*}
\norm{[R_i, R_j]} \leq \frac{1}{(1 - \epsilon)^2} \epsilon
 \enspace .
\end{equation*}
Define projections $P_1, \ldots, P_{2n}$ by $P_j = \tfrac12 (\identity + R_j)$.  Then 
\begin{equation*}\begin{split}
\norm{[P_i, P_j]}
&= \tfrac14 \norm{[R_i, R_j]} \\
&\leq \frac14 \frac{1}{(1-\epsilon)^2} \epsilon
 \enspace .
\end{split}\end{equation*}

Applying \thmref{t:manynearlycommutingprojections} for separating projections yields projections $Q_1, \ldots, Q_{2n}$ satisfying $[Q_i, Q_j] = 0$ and 
\begin{equation*}
\norm{Q_j - P_j} \leq 8 \cdot (2n) \cdot \frac14 \frac{1}{(1-\epsilon)^2} \epsilon = \frac{4 n \epsilon}{(1 - \epsilon)^2}
 \enspace ,
\end{equation*}
provided that $\epsilon / (1-\epsilon)^2 \leq 1/(64n)$.  

We claim that the reflections $2 Q_{2j-1} - \identity$ and $2 Q_{2j} - \identity$ still have the form $\sigma_j^x \otimes X_j'$ and $\sigma_j^z \otimes Z_j'$, resepectively, for reflections $X_j'$ and~$Z_j'$ on $\H$.  Indeed, the proof of the projections separation theorem, \thmref{t:manynearlycommutingprojections}, involved two basic operations: 
\begin{enumerate}
\item Block-diagonalizing an operator~$A$ with respect to a reflection~$R$: 
\begin{align*}
A
&\rightarrow \tfrac12 (\identity + R) A \tfrac12 (\identity + R) + \tfrac12 (\identity - R) A \tfrac12 (\identity - R) \\
&= \frac12 (A + R A R)
 \enspace .
\end{align*}
\item Rounding the eigenvalues of a Hermitian operator~$A$ to $\pm 1$.  
\end{enumerate}
Observe that if $A = \sigma \otimes A'$ for a Pauli~$\sigma$, and $R = \tau \otimes R'$ for a Pauli~$\tau$, then both of these basic operations result in an operator $\sigma \otimes A''$, for the same Pauli~$\sigma$.  

Thus indeed $\{X_j', Z_j'\} = 0$ and $[S_i', T_j'] = 0$ for $i \neq j$ and $S, T \in \{X, Z\}$.  Also $\norm{Q_j - P_j} \leq 4 n \epsilon / (1-\epsilon)^2$ implies 
\begin{align*}
\norm{S_j' - S_j} 
&\leq 2 \norm{Q_j - P_j} + \norm{R_j' - R_j} \\
&\leq \frac{8 n \epsilon}{(1-\epsilon)^2} + \epsilon
 \enspace . \qedhere
\end{align*} 
\end{proof}

Since \thmref{t:manynearlyindependentqubits} yields $n$ qubits in tensor product, the dimension of the ambient space~$\H$ must be at least $2^n$.  Rephrasing this, we obtain: 

\begin{corollary}
In $2^n$ dimensions, at most $n$ qubits can be placed with pairwise ``overlaps" $\norm{[S_i, T_j]} \leq \epsilon$, if $\epsilon / (1-\epsilon)^2 \leq 1/(64 n)$.  
\end{corollary}


\subsubsection{SWAP-based argument} \label{s:swapnorm}

If we are willing to work in a larger space, then there is a simpler argument for moving overlapping qubits into tensor product.  Instead of repeatedly block-diagonalizing operators and rounding their eigenvalues to $\pm 1$, as in \thmref{t:manynearlyindependentqubits}, we can swap in fresh qubits to enforce a tensor-product structure.  We will show: 

\begin{theorem} \label{t:swappingmanynearlyindependentqubits}
Let $X_1, Z_1, \ldots, X_n, Z_n$ be reflections on~$\H$, satisfying $\{X_j, Z_j\} = 0$ and $\norm{[S_i, T_j]} \leq \epsilon$ for all $i \neq j$ and $S, T \in \{X, Z\}$.  Extend these operators by the identity to act on $\H \otimes (\C^2)^{\otimes n}$.  

Then there exist reflections $X_1', Z_1', \ldots, X_n', Z_n'$ on $\H \otimes (\C^2)^{\otimes n}$, with $\{X_j', Z_j'\} = 0$, $[S_i', T_j'] = 0$ and $\norm{S_j' - S_j} \leq 2 n \epsilon$.  
\end{theorem}

\begin{proof}
For $j \in [n]$, let $\S_j = \frac12 \big( \identity \otimes \identity + X_j \otimes \sigma^x_j + Z_j \otimes \sigma^z_j + i (X_j Z_j) \otimes \sigma^y_j \big)$.  Acting on $\H \otimes (\C^2)^{\otimes n}$, $\S_j$ swaps the $j$th added $\C^2$ register with the qubit defined by $X_j, Z_j$.  

For $T \in \{X, Z\}$ and $i \in \{ 1, \ldots, j \}$ define 
\begin{equation*}
T_j^{(i)} = (\S_1 \cdots \S_{i-1}) \, T_j \, (\S_{i-1} \cdots \S_1)
 \enspace .
\end{equation*}
Let $T_j' = T_j^{(j)} = (\S_1 \cdots \S_{j-1}) T_j (\S_{j-1} \cdots \S_1)$.  

Then for $i < j$, $\norm{[S_i', T_j']} = \norm{[S_i, \S_i \cdots \S_{j-1} T_j \S_{j-1} \cdots \S_i]}$.  This is $0$, since for any operator~$A$ that is the identity on the $i$th added $\C^2$ register, $[S_i, \S_i A \S_i] = 0$.  

Furthermore, 
\begin{align*}
\norm{T_j' - T_j}
&\leq \sum_{i=1}^{j-1} \norm{T_j^{(i+1)} - T_j^{(i)}} \\
&= \sum_{i=1}^{j-1} \norm{\S_i T_j \S_i - T_j} \\
&= \sum_{i=1}^{j-1} \norm{[\S_i, T_j]} \\
&\leq \frac12 \sum_{i=1}^{j-1} \big( \norm{[X_i, T_j]} + \norm{[Z_i, T_j]} + \norm{[X_i Z_i, T_j]} \big) \\
&\leq 2 \epsilon (j-1)
 \enspace . \qedhere
\end{align*}
\end{proof}

Since \thmref{t:swappingmanynearlyindependentqubits} works in the larger space $\H \otimes (\C^2)^{\otimes n}$, unlike \thmref{t:manynearlyindependentqubits} it does not give an upper bound on the number of nearly independent qubits that can be packed into~$\H$.  


\subsubsection{Lower bound: Sometimes $\Omega(n \epsilon)$ movement is necessary}

\thmref{t:manynearlyindependentqubits} shows that $n$ qubits with pairwise ``overlaps" at most $\epsilon$ can be separated into tensor product by moving each qubit $O(n \epsilon)$ in operator norm.  Is the loss of a factor of~$n$ necessary?  The following example shows that our bound is essentially tight.  

\begin{lemma} \label{t:movementlowerbound}
For any integer $n$, and any $\epsilon \in [0, \pi / n^2]$, there exist $2 n$ qubits $X_1, Z_1, \ldots, X_{2n}, Z_{2n}$ in $(\C^2)^{\otimes (2n)}$ such that $\norm{[S_i, T_j]} \leq \epsilon$ for all $i \neq j$ and $S,T\in\{X,Z\}$ but such that for any \emph{independent} qubits $X_1', Z_1', \ldots, X_{2n}', Z_{2n}'$ (with $[S_i', T_j'] = 0$ for $i \neq j$), 
\begin{equation*}
\max_{\substack{1 \leq j \leq 2n \\ S \in \{X, Z\}}} \bignorm{S_j - S_j'} \geq \frac{n \epsilon}{2 \pi}
 \enspace .
\end{equation*}
\end{lemma}

\begin{proof}
Construct qubits $X_j, Z_j$ as the standard qubits, except with the second~$n$ qubit operators perturbed by the Hamiltonian 
\begin{equation*}
H = \tfrac14 (\sigma^z_1 + \cdots + \sigma^z_n) (\sigma^z_{n+1} + \cdots + \sigma^z_{2n})
 \enspace .
\end{equation*}
That is, $X_j = \sigma^x_j$, $Z_j = \sigma^z_j$ for $j \leq n$, and $X_j = e^{i \epsilon H} \sigma^x_j  e^{-i \epsilon H}$, $Z_j = e^{i \epsilon H} \sigma^z_j  e^{-i \epsilon H} = \sigma^z_j$ for $j > n$.  Then if $j, k \leq n$ or $j, k > n$, the operators for qubits $j$ and~$k$ commute.  If $j \leq n < k$, then the operators for qubits~$j$ and~$k$ commute, except for $X_j$ and $X_k$.  We compute $\norm{[X_j, X_k]} = \norm{X_j X_k X_j - X_k} = \norm{ e^{-i \epsilon H} \sigma^x_j e^{i \epsilon H} \sigma^x_k e^{-i \epsilon H} \sigma^x_j e^{i \epsilon H} - \sigma^x_k } = \norm{ e^{i \epsilon \sigma^z_j \sigma^z_k} - \identity } = \abs{e^{i \epsilon} - 1} \leq \epsilon$.  

Let $X'_1, \ldots, X'_{2n}$ be any pairwise commuting reflections.  Let $J = \{1, \ldots, n\}$, $K = \{n+1, \ldots, 2n\}$.  Let $X_J = \prod_{j \in J} X_j$, $X_K = \prod_{k \in K} X_k$.  Similarly define $X_J', X_K'$ and $\sigma^x_J, \sigma^x_K$.  Thus $X_J = \sigma^x_J$, $X_K = e^{i \epsilon H} \sigma^x_K e^{-i \epsilon H}$.  In order to lower-bound $\max_j \norm{X_j - X_j'}$, we study $\norm{[X_J, X_K]} = \norm{(X_J X_K)^2 - \identity}$.  

On one hand, since the $X'_j$ operators commute, $(X_J' X_K')^2 = \identity$.  By triangle inequalities, and using $\norm{X_j} = \norm{X_j'} = 1$ for all~$j$, $\norm{X_J X_K - X_J' X_K'} \leq \sum_j \norm{X_j - X_j'}$, and hence 
\begin{equation} \label{e:movementlowerbound1}
\norm{(X_J X_K)^2 - \identity} \leq 2 \sum_j \norm{X_j' - X_j} \leq 4 n \cdot \max_j \norm{X_j' - X_j}
 \enspace .  
\end{equation}
On the other hand, 
\begin{align*}
(X_J X_K)^2
&= \sigma^x_J \big( e^{i \epsilon H} \sigma^x_K \, e^{-i \epsilon H} \big) \sigma^x_J \big( e^{i \epsilon H} \sigma^x_K \, e^{-i \epsilon H} \big) \\
&= e^{-i \epsilon H} \sigma^x_K \, e^{2 i \epsilon H} \sigma^x_K \, e^{-i \epsilon H} \\
&= e^{-4 i \epsilon H}
 \enspace .
\end{align*}
Since $\norm{H} = n^2/4$, provided that $n^2 \epsilon \leq \pi$ it holds that 
\begin{equation} \label{e:movementlowerbound2}
\bignorm{(X_J X_K)^2 - \identity} = \abs{e^{i n^2 \epsilon} - 1} \geq \frac{2}{\pi} \cdot n^2 \epsilon
 \enspace .
\end{equation}
Combining the bounds~\eqnref{e:movementlowerbound1} and~\eqnref{e:movementlowerbound2} gives $\frac{2}{\pi} n^2 \epsilon \leq \norm{(X_J X_K)^2 - \identity} \leq 4 n \cdot \max_j \norm{X_j' - X_j}$, or $\max_j \norm{X_j' - X_j} \geq n \epsilon / (2 \pi)$.  
\end{proof}


%auto-ignore
\ifx\compilefullpaper\undefined  
\documentclass[11pt]{article}
\input{header}
\begin{document}
%\tableofcontents
\fi


\section{State-dependent qubit separation} \label{s:statedependent}

A problem with both \thmref{t:manynearlyindependentqubits} and \thmref{t:swappingmanynearlyindependentqubits} is that they might be difficult to apply to real experimental systems.  This is because it is difficult to establish the assumption of qubits nearly in tensor product, $\norm{[S_i, T_j]} \leq \epsilon$ for $i \neq j$ and $S, T\in\{X,Z\}$.  In addition to the operators, a physical system involves an underlying state~$\ket \psi$.  The operators can be understood only in terms of their effects on $\ket \psi$.  Consider for example a Hilbert space that splits as $\H \oplus \H'$, where $\ket \psi$ is supported only on $\H$ and available operators leave $\H$ invariant.  Then there is no experimental way to fathom the operators' behavior, e.g., their commutation relationships, on~$\H'$.  Theorems~\ref{t:manynearlyindependentqubits} and~\ref{t:swappingmanynearlyindependentqubits} cannot be applied.  This example might not seem so troubling, because we can simply restrict everything to~$\H$; but it becomes more problematic if $\ket \psi$, say, has nonzero but very small support on~$\H'$.  

We would like qubit-separation theorems that have experimentally accessible assumptions.  In particular, the theorems' assumptions should be stated relative to the system's state~$\ket \psi$.  For example, in Theorems~\ref{t:manynearlyindependentqubits} and~\ref{t:swappingmanynearlyindependentqubits} we might loosen the assumption $\norm{[S_i, T_j]} \leq \epsilon$ for $i \neq j$ to be only $\norm{[S_i, T_j] \ket \psi} \leq \epsilon$.  Naturally, the conclusions will have to be correspondingly weakened.  In the above example with $\H \oplus \H'$, if the reflections are far from commuting on~$\H'$ then we cannot hope to find nearby commuting operators, $\norm{S_j' - S_j} \approx 0$; but perhaps we can get $\norm{(S_j' - S_j) \ket \psi} \approx 0$.  

In order to extend our results to experimental systems we proceed in three steps.  

\begin{enumerate}
\item 
First, in \secref{s:statedependentcommutationprotocol} below, we give a protocol that can be used to test if two reflections, $S$ and~$T$, are close to commuting on a state~$\ket \psi$: $[S, T] \ket \psi \approx 0$.  The protocol is very simple: measure $S$, measure $T$, then measure $S$ again.  If $S$ and~$T$ commute on~$\ket \psi$, then the two $S$ measurements will give the same result; and, intuitively, when they do not commute measuring $T$ will disturb the state and make it less likely to get the same $S$ result.  

\item 
However, in \secref{s:statedependentseparationcounterexample}, we show that the condition $[S_i, T_j] \ket \psi \approx 0$ for operators on different qubits is not sufficient to establish that there are nearby independent qubits $X_1', Z_1', \ldots, X_n', Z_n'$.  In fact, we give an explicit construction of a state~$\ket \psi$ and $n$ qubit operators $X_1, Z_1, \ldots, X_n, Z_n$ in $< n^2$ dimensions such that for $i \neq j$, $[S_i, T_j] \ket \psi = 0$ precisely.  Since $n^2 \leq 2^n$ for $n \geq 4$, the dimension of the space is not sufficient to fit $n$ independent qubits.  

(We also show why the basic induction argument used to prove \thmref{t:manynearlyindependentqubits} fails when errors are measured relative to a state~$\ket \psi$.  The errors accumulate too rapidly, leading to an exponential dependence on~$n$, instead of polynomial.)  

\item 
We remedy this problem in \secref{s:statedependentnqubitprotocol} with a more advanced testing protocol.  Intuitively, the improved protocol tests not just pairwise commutation relationships, such as $S_i T_j \ket \psi \approx T_j S_i \ket \psi$, but also higher-order relationships such as $S_i T_j U_k \ket \psi \approx U_k T_j S_i \ket \psi$.  The protocol is still quite simple, though.  Basically, measure all the qubit operators in order (either $X_1, Z_1, X_2, Z_2, \ldots$ or $Z_1, X_1, Z_2, X_2, \ldots$), then go back and measure a random qubit operator ($Z_j$ or $X_j$, respectively), and verify that the measurement result is unchanged.  We show that if the protocol accepts with probability $1 - \epsilon$, then the qubit operators ``simulate'' $n$ independent qubit operators in a certain sense.  In particular, as a corollary, the system's dimension must be at least $(1 - O(n^2 \epsilon)) 2^n$.  

The dimension bound is not fully satisfactory.  A $2^n$ lower bound would be preferable.  However, speculatively, the simulation statement might be strong enough to form the foundation for an analysis that the system can be used as an $n$-qubit quantum computer.  Such an extension is nontrivial, though, and we leave it to future work.  
\end{enumerate}


\medskip %
\subsection{Protocol for testing state-dependent commutation} \label{s:statedependentcommutationprotocol}

We present a protocol that can be used to test whether two reflections approximately commute on a given state.  

\medskip %

\begin{theorem} \label{t:statedependentcommutationprotocol}
Let $S$ and~$T$ be reflections, acting on a state~$\ket \psi$.  Consider the following protocol: 
\begin{enumerate}
\item Measure $S$.  
\item Measure $T$, but ignore the result.    
\item Measure $S$ again.  Accept if the result is unchanged.  
\end{enumerate}
Then the probability of accepting is given by 
\begin{equation*}
\Pr[\mathrm{accept}] = 1 - \tfrac18 \bignorm{[S, T] \ket \psi}{}^2
 \enspace .
\end{equation*}
\end{theorem}


\begin{proof}
For $a, b \in \{0, 1\}$, let $S_a = \tfrac12 (\Id + (-1)^a S)$ and $T_b = \tfrac12 (\Id + (-1)^b T)$.  Then since $[S, T_0] = -[S, T_1] = \tfrac12 [S, T]$, 
\begin{align*}
\norm{[S, T] \ket \Psi}^2
&= 2 \big( \norm{ [S, T_0] \ket \psi }^2 + \norm{ [S, T_1] \ket \psi }^2 \big) \\
&= \sum_{a, b} \bignorm{S_a [S, T_b] \ket \psi}^2 
 \enspace , 
\intertext{where we have used $\norm{\ket \phi}^2 = \norm{S_0 \ket \phi}^2 + \norm{S_1\ket \phi}^2$ for any $\ket \phi$.  Then from $S_a S = S S_a = (-1)^a S_a$, we find $S_a [S, T_b] = S_a [S, T_b] (S_0 + S_1) = 2 (-1)^a S_a T S_{\bar a}$, so }
\norm{[S, T] \ket \psi}^2
&= 8 \sum_{a, b} \bignorm{S_a T_b S_{\bar a} \ket \psi}^2 \\
&= 8 \, (1 - \Pr[\mathrm{accept}])
 \enspace .  \qedhere
\end{align*}
\end{proof}


\subsection{Qubits that commute on a state need not be close to independent qubits}
\label{s:statedependentseparationcounterexample}

In the projection separating argument of \thmref{t:manynearlycommutingprojections}, the key observation was that for projections $P$, $Q$, $R$ with $\norm{[P,Q]}, \norm{[P,R]} \leq \delta$ and $\norm{[Q,R]} \leq \epsilon$, if $Q$ and $R$ are both block-diagonalized with respect to~$P$ then the results still nearly commute:  
\begin{equation*}
\left\Vert \big[ PQP + (\identity-P)Q(\identity-P), PRP + (\identity-P)R(\identity-P) \big] \right\Vert \leq \epsilon + 2 \delta^2
 \enspace .
\end{equation*}
The quadratic dependence on $\delta$ meant that errors did not accumulate badly through the induction.  

Here is a counterexample showing that errors \emph{can} accumulate badly in block diagonalization if we measure errors relative to a state $\ket \psi$, using $\norm{[P,Q] \ket \psi}$.  Define $P$, $Q$, $R$ and $\ket \psi$ as 
\begin{equation}
P = \fastmatrix{
1 & 0 & 0 & \delta \\
0 & 1/2 & 1/2 & 0 \\
0 & 1/2 & 1/2 & 0 \\
\delta & 0 & 0 & 0
}
\qquad
Q = \fastmatrix{
1&0&0&0\\
0&0&0&0\\
0&0&1&0\\
0&0&0&0
}
\qquad
R = \fastmatrix{
1&0&0&0\\
0&0&0&0\\
0&0&1/2&1/2\\
0&0&1/2&1/2
}
\qquad 
\ket \psi = \fastmatrix{1\\0\\0\\0}
 \enspace .
\end{equation}
Then $P$, $Q$ and $R$ are projections (up to second order in $\delta$ for~$P$), with $\norm{[P,Q] \ket \psi}, \norm{[P,R] \ket \psi} = O(\delta)$, $[Q,R] \ket \psi = 0$, and yet 
\begin{equation*}
\left\Vert \big[ PQP + (\identity-P)Q(\identity-P), PRP + (\identity-P)R(\identity-P) \big] \ket \psi \right\Vert = \Omega(\delta)
 \enspace .
\end{equation*}
The idea is that $Q$ and $R$ commute on the first two dimensions, and are far from commuting on the last two dimensions; but this property is broken by the block diagonalization.  

This example suggests that in a simple induction argument, starting with projections $P_1, \ldots, P_n$ having pairwise commutators $\norm{[P_i, P_j] \ket \psi} \sim \epsilon$, after block-diagonalizing with respect to $P_1$, the errors can grow to $\sim 2 \epsilon$, then to $\sim 4 \epsilon$ after block-diagonalizing with respect to the new $P_2$, and so on; the errors potentially grow exponentially.  

In fact, it is not only our \emph{proof} of Theorems~\ref{t:manynearlycommutingprojections} and~\ref{t:manynearlyindependentqubits} that fails when errors are measured relative to a state~$\ket \psi$.  The theorems themselves fail, as shown by the following construction. 

\begin{lemma} \label{t:kcommute}
For any $n$ and $k \in [n]$, there exists a space $\H$ of dimension at most $1 + \sum_{j=0}^k {n \choose j}$, a vector $\ket \psi \in \H$ and $n$ qubits $X_j, Z_j$ such that 
\begin{equation*}
S^{(1)}_{j_1} \cdots S^{(k)}_{j_k} \ket \psi = S_{j_{\sigma(1)}}^{\sigma(1)} \cdots S_{j_{\sigma(k)}}^{\sigma(k)} \ket \psi
\end{equation*}
for all distinct indices $j_1, \ldots, j_k \in [n]$, $S^{(1)}, \ldots, S^{(k)} \in \{X, Z\}$, and permutations $\sigma$ of $[k]$.  
\end{lemma}

In particular, for $k = 2$, the lemma places $n$ qubits in $O(n^2)$ dimensions---for example, four qubits in~$12$ dimensions---such that $[S_i, T_j] \ket \psi = 0$ for all $i \neq j$ and $S,T\in\{X,Z\}$.    

\begin{proof}
Let us begin by explaining the $n = 4$, $k = 2$ special case of the construction.  Define $\H$ to have orthonormal basis $\ket{0000}, \ket{1000}, \ldots, \ket{0001}, \ket{1100}, \ldots, \ket{0011}, \ket d$, i.e., all $n$-bit strings of Hamming weight at most~$k$, together with an additional vector~$\ket d$.  Let~$\ket \psi = \ket{0000}$, and consider the following operators for the first qubit:
\vspace{.4cm}
\begin{gather*}
X_1 = \hspace{.5cm}
\left(
\makebox(95,43){\hspace{-.7cm}\raisebox{-1in}{$
\begin{smallmatrix}
&\rotatebox{85}{\tiny 0000}&
\rotatebox{85}{\tiny 1000}&
\rotatebox{85}{\tiny 0100}&
\rotatebox{85}{\tiny 0010}&
\rotatebox{85}{\tiny 0001}&
\rotatebox{85}{\tiny 1100}&
\rotatebox{85}{\tiny 1010}&
\rotatebox{85}{\tiny 1001}&
\rotatebox{85}{\tiny 0110}&
\rotatebox{85}{\tiny 0101}&
\rotatebox{85}{\tiny 0011}&
\rotatebox{85}{\tiny $d$} \\
\rotatebox{0}{\tiny 0000\;\;\;}&0&1& & & & & & & & & &  \\
\rotatebox{0}{\tiny 1000\;\;\;}&1&0& & & & & & & & & &  \\
\rotatebox{0}{\tiny 0100\;\;\;}& & &0& & &1& & & & & &  \\
\rotatebox{0}{\tiny 0010\;\;\;}& & & &0& & &1& & & & &  \\
\rotatebox{0}{\tiny 0001\;\;\;}& & & & &0& & &1& & & &  \\
\rotatebox{0}{\tiny 1100\;\;\;}& & &1& & &0& & & & & &  \\
\rotatebox{0}{\tiny 1010\;\;\;}& & & &1& & &0& & & & &  \\
\rotatebox{0}{\tiny 1001\;\;\;}& & & & &1& & &0& & & &  \\
\rotatebox{0}{\tiny 0110\;\;\;}& & & & & & & & &0&1& &  \\
\rotatebox{0}{\tiny 0101\;\;\;}& & & & & & & & &1&0& &  \\
\rotatebox{0}{\tiny 0011\;\;\;}& & & & & & & & & & &0&1 \\
\rotatebox{0}{\tiny $d$}& & & & & & & & & & &1&0 
\end{smallmatrix}
$}}
\right)
\place{\Huge 0}{-45mu}{10pt}
\place{\Huge 0}{-110mu}{-35pt}
\qquad\qquad
\def\minusone{\hspace{-.4cm}\text{--}1\hspace{-.2cm}}
Z_1 = \hspace{.5cm}
\left(
\makebox(95,43){\hspace{-.7cm}\raisebox{-1in}{$
\begin{smallmatrix}
&\rotatebox{85}{\tiny 0000}&
\rotatebox{85}{\tiny 1000}&
\rotatebox{85}{\tiny 0100}&
\rotatebox{85}{\tiny 0010}&
\rotatebox{85}{\tiny 0001}&
\rotatebox{85}{\tiny 1100}&
\rotatebox{85}{\tiny 1010}&
\rotatebox{85}{\tiny 1001}&
\rotatebox{85}{\tiny 0110}&
\rotatebox{85}{\tiny 0101}&
\rotatebox{85}{\tiny 0011}&
\rotatebox{85}{\tiny $d$} \\
\rotatebox{0}{\tiny 0000\;\;\;}&1& & & & & & & & & & &  \\
\rotatebox{0}{\tiny 1000\;\;\;}& &\minusone& & & & & & & & & &  \\
\rotatebox{0}{\tiny 0100\;\;\;}& & &1& & & & & & & & &  \\
\rotatebox{0}{\tiny 0010\;\;\;}& & & &1& & & & & & & &  \\
\rotatebox{0}{\tiny 0001\;\;\;}& & & & &1& & & & & & &  \\
\rotatebox{0}{\tiny 1100\;\;\;}& & & & & &\minusone& & & & & &  \\
\rotatebox{0}{\tiny 1010\;\;\;}& & & & & & &\minusone& & & & &  \\
\rotatebox{0}{\tiny 1001\;\;\;}& & & & & & & &\minusone& & & &  \\
\rotatebox{0}{\tiny 0110\;\;\;}& & & & & & & & &1& & &  \\
\rotatebox{0}{\tiny 0101\;\;\;}& & & & & & & & & &\minusone& &  \\
\rotatebox{0}{\tiny 0011\;\;\;}& & & & & & & & & & &1&  \\
\rotatebox{0}{\tiny $d$}& & & & & & & & & & & &\minusone 
\end{smallmatrix}
$}}
\right)
\place{\Huge 0}{-55mu}{10pt}
\place{\Huge 0}{-130mu}{-25pt}
 \enspace .
\end{gather*}
Unspecified matrix entries are $0$.  $X_2$ and $Z_2$ can be obtained from $X_1$ and $Z_1$ by switching the first and second bits in each basis element, leaving $\ket d$ alone; and similarly for $X_3, Z_3$ and $X_4, Z_4$.  Then $P_i^2 = \identity$, $\{X_i, Z_i\} = 0$ and $[P_i, Q_j] \ket \psi = 0$, for $i \neq j$ and $P, Q \in \{X, Z\}$.  

The idea behind this construction is that $X_j, Z_j$ act largely as the Pauli operators $\sigma^x_j, \sigma^z_j$.  However, we have truncated the standard basis $\ket{0000}, \ldots, \ket{1111}$ for $(\C^2)^{\otimes 4}$ to include only strings of Hamming weight $\leq 2$.  Since applying $\sigma^x_1$ to $\ket{0110}, \ket{0101}$ and $\ket{0011}$ would give strings of Hamming weight~$3$, we instead pair these dimensions up arbitrarily for~$X_1$, and define $Z_1$ on them to make it anti-commute with $X_1$.  The extra dimension $\ket d$ is needed to make the total dimension even.  

It is straightforward to generalize the example: by truncating strings at Hamming weight~$k$ the same construction places $n$ qubits in $\sum_{j=0}^k \big(\begin{smallmatrix}n\\j\end{smallmatrix}\big)$ dimensions (or one more if this dimension is odd), such that any combination of up to $k$ qubit operators commute on $\ket \psi = \ket{0^n}$, e.g., if $k \geq 3$, $X_1 X_2 X_3 \ket \psi = X_3 X_2 X_1 \ket \psi = \ket{1110^{n-3}}$.  
\end{proof}


\subsection{Protocol to test for $n$ independent qubits} \label{s:statedependentnqubitprotocol}

The problem with the protocol in \thmref{t:statedependentcommutationprotocol} is that it only tests commutation between pairs of operators on the state $\ket \psi$: $[S, T] \ket \psi \approx 0$.  \lemref{t:kcommute} shows that $n$ qubits in only $O(n^2)$ dimensions can pass this test on every pair.  The lemma furthermore suggests that any test involving qubit operator sequences of length $o(n)$ can be satisfied in dimension $2^{o(n)}$.  Therefore, we need a protocol that has at least~$n$ steps.  

\figref{f:statedependentnqubitprotocol} gives our testing protocol.  We argue that if the protocol accepts with high probability, then the $n$ overlapping qubits $X_j, Z_j$ are nearly equivalent to $n$ independent qubits $\hat X_j, \hat Z_j$ in an enlarged space $\H' = \H \otimes (\C^2)^{\otimes 2 n}$.  

\begin{figure}
{ \noindent \hrulefill \\
\centering \textbf{Protocol to test for $n$ independent qubits} \\ } \smallskip

Let $\ket \psi \in \H$ be a state.  Let $X_1, Z_1, \ldots, X_n, Z_n$ be qubit operators on~$\H$, i.e., reflections satisfying $\{X_j, Z_j\} = 0$ for all~$j$.  
\begin{enumerate}
\item With equal probabilities $1/2$, measure the reflections in order, either $X_1, Z_1, X_2, Z_2, \ldots$, or $Z_1, X_1, Z_2, X_2, \ldots$.  
\item Pick a uniformly random index $j \in [n]$.  If $Z$ went second in step $(1)$, then measure~$Z_j$; and if $X$ went second, then measure $X_j$.  Accept if the result is unchanged from the operator's previous measurement.  Otherwise reject.  
\end{enumerate}
\vspace{-1\baselineskip}
\hrulefill
\caption{Protocol to test for $n$ independent qubits.} \label{f:statedependentnqubitprotocol}
\end{figure}

\begin{theorem} \label{t:statedependentnqubitprotocol}
Consider the protocol of \figref{f:statedependentnqubitprotocol}.  Assume the probability it accepts is at least $1 - \epsilon$.  

Let $\ket{\EPRstate} = \tfrac{1}{\sqrt 2}(\ket{00} + \ket{11})$.  Let $\ket{\Psi_0} = \ket \psi \otimes \ket{\EPRstate}^{\otimes n} \in \H' = \H \otimes (\C^2)^{\otimes 2 n}$, and let $\ket \Psi$ be obtained from $\ket{\Psi_0}$ by swapping each qubit $X_j, Z_j$ with the first half of one of the EPR states, in order $j = 1, \ldots, n$.  (See \figref{f:addneprstates}.)  Then there exist $n$ independent qubits, given by $\hat X_1, \hat Z_1, \ldots, \hat X_n, \hat Z_n$, on $\H'$ such that for any sequence of qubit operators $U_{j_1}, \ldots, U_{j_k}$, where $U_j$ acts on the $X_j, Z_j$ qubit and $\norm{U_j} \leq 1$, 
\begin{equation} \label{e:statedependentnqubitprotocol}
\bignorm{ U_{j_1} \cdots U_{j_k} \ket \Psi - \hat U_{j_1} \cdots \hat U_{j_k} \ket \Psi } = O(k n \sqrt \epsilon)
 \enspace .
\end{equation}
Here $\hat U_j$ is the same operator as $U_j$, except acting on the $\hat X_j, \hat Z_j$ qubit.  That is, if $U_j$ has Pauli expansion $U_j = \alpha_j \identity + \beta_j X_j + \gamma_j Z_j + \delta_j (i X_j Z_j)$ for scalars $\alpha_j, \beta_j, \gamma_j, \delta_j$, then $\hat U_j = \alpha_j \identity + \beta_j \hat X_j + \gamma_j \hat Z_j + \delta_j (i \hat X_j \hat Z_j)$.  
\end{theorem}
 
Observe that if the $X_j, Z_j$ qubits are independent of each other, then the measurements on different qubits commute, and so the protocol accepts with probability one.  In that case, there is nothing to show.  In general, however, measuring qubits $j+1, \ldots, n$ can disturb the last measurement on qubit~$j$.  

The EPR state appears in the conclusion of \thmref{t:statedependentnqubitprotocol} even though it is not used in the testing protocol.  Essentially this is because of the following two properties of $\ket{\EPRstate}$: 
\begin{enumerate}
\item Depolarizing a qubit, i.e., replacing it with the maximally mixed state, is equivalent to swapping it with the first qubit of a fresh EPR state then tracing out the EPR state's registers.  
\item For any $2 \times 2$ matrix $M$, $(I \otimes M) \ket{\EPRstate} = (M^T \otimes I) \ket{\EPRstate}$.  
\end{enumerate}
The second property is key in our analysis for algebraically manipulating operators to show approximate commutation.  To see how, consider for example a state $\ket \phi$ that involves four qubits, labeled $1, 2, 1', 2'$, where the $j'$ qubits do not overlap with any others.  If $\ket \phi$ is close to an EPR state on qubits $(1,1')$ and $(2,2')$, then operators on qubits~$1$ and~$2$ necessarily nearly commute on~$\ket \phi$: 
\begin{align*}
U_1 V_2 \ket \phi
&\approx U_1 V_{2'}^T \ket \phi 
= V_{2'}^T U_1 \ket \phi \\
&\approx V_{2'}^T U_{1'}^T \ket \phi 
= U_{1'}^T V_{2'}^T \ket \phi \\
&\approx U_{1'}^T V_2 \ket \phi 
= V_2 U_{1'}^T \ket \phi \\
&\approx V_2 U_1 \ket \phi
 \enspace .
\end{align*}
The trick is to pull operators from one side of an approximate EPR state to the other, commute them there, then pull them back.  

\begin{figure}
\centering
\includegraphics[scale=.07]{images/psiwitheprstates.png}
\caption{The state $\ket{\Psi_0}$ is given by $\ket \psi \otimes \ket{\EPRstate}^{\otimes n}$, where the EPR states are on qubits $1'$ and $1''$, $2'$ and $2''$, and so on.  To get $\ket \Psi$, swap qubit~$j'$ with the qubit in~$\H$ defined by $X_j, Z_j$, for $j = 1, \ldots, n$.  Observe that starting from $\ket \psi$ and depolarizing the $X_j, Z_j$ qubits, for $j = 1, \ldots, n$, is equivalent to tracing out all $j'$ and~$j''$ qubits from $\ketbra \Psi \Psi$.} \label{f:addneprstates}
\end{figure}

\begin{proof}[Proof of \thmref{t:statedependentnqubitprotocol}]
To analyze the protocol, we relate it to a separate protocol that is based on swapping qubits with halves of EPR states.  Observe that measuring either $X_i$ then $Z_i$, or $Z_i$ then $X_i$, and discarding the second measurement result, is equivalent to depolarizing the qubit.  Depolarizing a qubit is equivalent to swapping it with one half of $\ket{\EPRstate}$ and tracing out the original EPR state's registers.  Therefore, the protocol of \figref{f:statedependentnqubitprotocol} accepts with the same probability as the following protocol: 
\begin{enumerate}
\item Append to the system $n$ EPR states, on qubits labeled $1', 1'', \ldots, n', n''$.  Thus the system is in the state $\ket{\Psi_0} = \ket \psi \otimes \ket{\EPRstate}^{\otimes n} \in \H \otimes (\C^2_{1'} \otimes \C^2_{1''}) \otimes \cdots \otimes (\C^2_{n'} \otimes \C^2_{n''})$; see \figref{f:addneprstates}.  
\item For $i$ from $1$ up to $n$, swap the qubit defined by $X_i, Z_i$ with the new qubit~$i'$.  
\item Pick a uniformly random index~$j \in [n]$.  With equal probabilities $1/2$, measure either $X_j$ and~$\sigma^x_{j''}$, or $Z_j$ and $\sigma^z_{j''}$.  Accept if the measurement results are the same, both $+1$ or both~$-1$.  
\end{enumerate}
Indeed, for $\alpha \in \{x, z\}$, measuring $\sigma^\alpha_{j''}$ at the end of the protocol is equivalent to measuring $\sigma^\alpha_{j'}$ at the start, which is also equivalent to measuring just after swapping with the $X_j, Z_j$ qubit.  

If the protocol accepts with probability $1 - \epsilon$, then for probabilities $\epsilon_j$ satisfying $\epsilon = \tfrac{1}{n} \sum_j \epsilon_j$, we have $\min\!\big\{ \norm{\tfrac12 (\identity + X_j \otimes \sigma^x_{j''}) \ket \Psi}{}^2, \norm{\tfrac12 (\identity + Z_j \otimes \sigma^z_{j''}) \ket \Psi}{}^2 \big\} \geq 1 - 2  \epsilon_j$, where $\ket \Psi$ is the state after the swap gates in step~(2).  In particular, 
\begin{equation*}
\max\Big\{ \bignorm{X_j \otimes \sigma^x_{j''} \ket \Psi - \ket \Psi}, \bignorm{Z_j \otimes \sigma^z_{j''} \ket \Psi - \ket \Psi} \Big\} \leq 2 \sqrt{2 \epsilon_j}
 \enspace .
\end{equation*}
This implies that for any one-qubit operator $U_j$ acting on the $X_j, Z_j$ qubit, $U_j \ket \Psi \approx U_{j''}^T \ket \Psi$, where $U_{j''}$ is the same operator, but acting on the $j''$ qubit.  More precisely, if $U_j = \alpha_j \identity + \beta_j X_j + \gamma_j Z_j + \delta_j (i X_j Z_j)$ for complex scalars $\alpha_j, \beta_j, \gamma_j, \delta_j$, then $U_{j''}^T = \alpha_j \identity + \beta_j \sigma^x_{j''} + \gamma_j \sigma^z_{j''} - \delta_j \sigma^y_{j''}$; and, since $\max\{ \abs{\alpha_j}, \abs{\beta_j}, \abs{\gamma_j}, \abs{\delta_j} \} \leq \norm{U_j}$, 
\begin{align*}
\bignorm{(U_j - U_{j''}^T) \ket \Psi} 
&\leq (\abs{\beta_j} + \abs{\gamma_j} + 2 \abs{\delta_j}) \cdot 2 \sqrt{2 \epsilon_j} \\
&\leq 4 \norm{U_j} \cdot 2 \sqrt{2  \epsilon_j}
 \enspace .
\end{align*}
For each~$i$, let $\S_i$ be the operator on that swaps the $X_i, Z_i$ qubit with the new qubit~$i'$: $\S_i = \frac12 \big( \identity + X_i \otimes \sigma^x_{i'} + Z_i \otimes \sigma^z_{i'} + i (X_i Z_i) \otimes \sigma^y_{i'} \big)$.  
For $i \leq j$, let $\S_{i,j} = \S_i \S_{i+1} \ldots \S_j$ and $\S_{j,i} = \S_j \S_{j-1} \ldots \S_i$.  Thus $\ket \Psi = \S_{n,1} \ket{\Psi_0}$.  

Let $\hat P_i = \S_{n,i+1} P_i \S_{i+1,n} = \S_{n,i} \sigma^P_{i'} \S_{i,n} = \S_{n,1} \sigma^P_{i'} \S_{1,n}$.  As $[\sigma^P_{i'}, \sigma^Q_{j'}] = 0$ for $i \neq j$ and $P, Q \in \{X, Z\}$, so too $[\hat P_i, \hat Q_j] = 0$.  

Observe that 
\begin{equation} \label{e:switch1}
\hat U_j \ket \Psi = U_{j''}^T \ket \Psi
 \enspace ,
\end{equation}
 since 
\begin{align*}
\hat U_j \S_{n,1} \ket{\Psi_0} 
&= (\S_{n,1} U_{j'} \S_{1,n}) \S_{n,1} \ket{\Psi_0} \\
&= \S_{n,1} U_{j'} \ket{\Psi_0} \\
&= \S_{n,1} U_{j''}^T \ket{\Psi_0}
 \enspace ,
\end{align*}
where the last equality is because $\ket{\Psi_0}$ includes an EPR state between qubits $j'$ and~$j''$.  It follows that for any unitary $U$ acting only on the $X_j, Z_j$ qubit,
\begin{equation} \label{e:normbound0}
\bignorm{(U_j - \hat U_j) \ket \Psi} \leq 8 \sqrt{2 \epsilon_j}
 \enspace .
\end{equation}
Now consider a sequence of operators $U_{j_1}, \ldots, U_{j_k}$, where $U_j$ acts on the $X_j, Z_j$ qubit and $\norm{U_j} \leq 1$.  Then iterating $\hat U_j \ket \Psi = U_{j''}^T \ket \Psi$ gives 
\begin{align*}
\hat U_{j_1} \cdots \hat U_{j_k} \ket \Psi 
&= \hat U_{j_1} \cdots \hat U_{j_{k-1}} U_{j_k''}^T \ket \Psi \\
&= U_{j_k''}^T \hat U_{j_1} \cdots \hat U_{j_{k-1}} \ket \Psi \\
&= \cdots \\
&= U_{j_k''}^T \cdots U_{j_1''}^T \ket \Psi 
 \enspace .
\intertext{To continue, iterate on $U_j \ket \Psi \approx U_{j''}^T \ket \Psi$: }
&\approx U_{j_1} U_{j_k''}^T \cdots U_{j_2''}^T \ket \Psi \\
&\approx \cdots \\
&\approx U_{j_1} \cdots U_{j_k} \ket \Psi
 \enspace .
\end{align*}
The overall error satisfies 
\begin{equation*}
\bignorm{ U_{j_1} \cdots U_{j_k} \ket \Psi - \hat U_{j_1} \cdots \hat U_{j_k} \ket \Psi } 
\leq k \cdot 4 \max \norm{U_{j_\ell}} \cdot 2 \sqrt{2 \epsilon_{j_\ell}}
= O(k \sqrt{n \epsilon})
 \enspace .  \qedhere
\end{equation*}
\end{proof}

In \thmref{t:statedependentnqubitprotocol}, the definition of $\ket \Psi$ requires adding to $\H$ an additional ancilla register $(\C^2)^{\otimes 2n}$.  It is therefore not clear that the theorem should imply an exponential lower bound on the dimension of~$\H$.  In fact, though, it does lower-bound $\dim \H$: 

\begin{corollary} \label{t:statedependentnqubitdimensionlowerbound}
If the protocol in \figref{f:statedependentnqubitprotocol} accepts with probability at least $1 - \epsilon$, then 
\begin{equation*}
\dim \H \geq \big( 1 - O(n^2 \epsilon) \big) \, 2^n
 \enspace .
\end{equation*}
\end{corollary}

\begin{proof}
For $(a, b) \in \{0,1\}^n \times \{0,1\}^n$ let 
\begin{equation*}
\ket{\Psi_{a,b}} 
= \big( X_n^{a_n} Z_n^{b_n} \big) \cdots \big( X_1^{a_1} Z_1^{b_1} \big) \ket \Psi
 \enspace .
\end{equation*}

\begin{claim}
The $\ket{\Psi_{a,b}}$ satisfy 
$\dim \Span \{ \ket{\Psi_{a,b}} \} \geq \big( 1 - O(n^2 \epsilon) \big) 4^n$.  
\end{claim}

\begin{proof}
Let $B = \sum_{a, b} \ketbra{\Psi_{a,b}}{a, b}$.  
Adopt the notation from the proof of \thmref{t:statedependentnqubitprotocol}.  
For $k \in \{0, \ldots, n\}$ define $\ket{\hat \Psi^{(k)}_{a,b}}$ similarly to $\ket{\Psi_{a,b}}$, except using the operators $\hat X_j$ and $\hat Z_j$ in place of $X_j$ and~$Z_j$ for $j \leq k$. Thus $\ket{\hat \Psi^{(0)}_{a,b}} = \ket{ \Psi_{a,b}}$.  Let $\ket{\hat \Psi_{a,b}} = \ket{ \hat \Psi^{(n)}_{a,b}}$ and define $\hat{B}$ as $B$ using the $\ket{\hat \Psi_{a,b}}$ instead of $\ket{\Psi_{a,b}}$.  Using the triangle inequality and $\norm{X_j}, \norm{Z_j} \leq 1$,
\begin{align}
\bignorm{ \ket{\hat \Psi_{a,b}} - \ket{\Psi_{a,b}} } 
&\leq \sum_{k=1}^n \bignorm{\ket{\hat \Psi^{(k)}_{a,b}} - \ket{\hat \Psi^{(k-1)}_{a,b}}} \notag\\
&\leq \sum_{k=1}^n \Bignorm{\big( \hat X_k^{a_k} \hat Z_k^{b_k} - X_k^{a_k} Z_k^{b_k} \big) \Big( \prod_{j < k} \hat X_j^{a_j} \hat Z_j^{b_j} \Big) \ket{\Psi}}
 \enspace . \label{e:normbound2a}
\end{align}
By Eq.~\eqnref{e:switch1} from the proof of \thmref{t:statedependentnqubitprotocol}, $\hat P_j \ket \Psi = P_{j''}^T \ket \Psi$, where $P_{j''}$ acts only on the $j''$ ancilla qubit and therefore commutes with all $Q_k$ and $\hat Q_k$.  Thus for any~$k \in [n]$, 
\begin{align*}
\big( \hat X_k^{a_k} \hat Z_k^{b_k} - X_k^{a_k} Z_k^{b_k} \big) \Big( \prod_{j < k} \hat X_j^{a_j} \hat Z_j^{b_j} \Big)\ket \Psi
&=\Big( \prod_{j < k} \big(X_{j''}^{a_j} Z_{j''}^{b_j} \big)^T \Big) \big(\hat X_k^{a_k} \hat Z_k^{b_k} - X_k^{a_k} Z_k^{b_k} \big) \ket \Psi
 \enspace .
\end{align*}
Thus starting from Eq.~\eqnref{e:normbound2a} and applying~\eqnref{e:normbound0}, we obtain the bound
\begin{equation} \label{e:normbound2}
\bignorm{ \ket{\hat \Psi_{a,b}} - \ket{\Psi_{a,b}}} 
\leq \sum_{k=1}^n 8 \sqrt{2 \epsilon_k}
  \enspace .
\end{equation}
Moreover, the $\ket{\hat \Psi_{a,b}}$ vectors are orthonormal: 
\begin{align*}
\braket{\hat \Psi_{a,b}}{\hat \Psi_{c,d}}
&= \bra{\Psi_0} \S_{1,n} \prod_{j=1}^n \big( \hat Z_j^{b_j} \hat X_j^{a_j + c_j} \hat Z_j^{d_j} \big) \S_{n,1} \ket{\Psi_0} \\
&= (-1)^{(a + c) \cdot b} \bra{\EPRstate}^{\otimes n} \prod_{j=1}^n \big( (\sigma^x_{j'})^{a_j+c_j} (\sigma^z_{j'})^{b_j+d_j} \big) \ket{\EPRstate}^{\otimes n} \\
&= \delta_{a,c} \delta_{b,d}
 \enspace .
\end{align*}
Therefore $\hat B$ is an isometry.  Its singular values are $1$ with multiplicity $4^n$.  Let $\lambda_1 \geq \cdots \geq \lambda_{4^n} \geq 0$ be the singular values of~$B$.  (Some $\lambda_i$ may be zero.)  Then, relating the singular values of $B$ and~$\hat B$ to the Frobenius norm of their difference, 
\begin{align*}
\sum_i \abs{\lambda_i - 1}^2 
&\leq \norm{B - \hat B}{}_F^2 \\
&= \sum_{a,b} \bignorm{\ket{\Psi_{a,b}} - \ket{\hat \Psi_{a,b}}}^2 \\
&\leq 4^n \cdot 128 \cdot n^2 \epsilon
 \enspace ,
\end{align*}
where the last bound is by Eq.~\eqnref{e:normbound2} and $\sum_k \epsilon_k = n \epsilon$.  Since the left-hand side is at least $4^n - \rank(B)$, we obtain $\rank(B) \geq \big( 1 - O(n^2 \epsilon) \big) 4^n$.  
\end{proof}

Let $\ket \Psi$ have Schmidt decomposition $\ket \Psi = \sum_{i = 1}^{d} \sqrt{p_i} \ket{u_i} \otimes \ket{v_i}$ across the partition $\H$, $(\C^2)^{\otimes 2 n}$.  Extend the set $\{\ket{u_1}, \ldots, \ket{u_d}\}$, if necessary, to form an orthonormal basis for~$\H$.  The vectors $\ket{\Psi_{a,b}}$ are obtained from $\ket \Psi$ by applying operators $X_j, Z_j$ supported only on~$\H$.  Therefore, they lie in the span of $\{ \ket{u_i} \otimes \ket{v_j} : i \in [\dim \H], j \in [d] \}$.  In particular, $\dim \Span \{\ket{\Psi_{a,b}}\} \leq d \dim \H \leq (\dim \H)^2$, as desired.  
\end{proof}

\begin{remark}
In \thmref{t:manynearlyindependentqubits}, different qubits overlapping by $\epsilon = O(1/n)$ already implies $\dim \H \geq 2^n$.  In contrast, in \corref{t:statedependentnqubitdimensionlowerbound}, $\epsilon$ must be exponentially small before $\dim \H \geq 2^n$ is required.  Is this polynomial versus exponential separation a consequence of loose analysis, an inherent drawback of the protocol in \figref{f:statedependentnqubitprotocol}, or an inherent property of any efficient state-dependent qubit testing protocol?  

\newcommand\restrict[1]{\raisebox{-.5ex}{$|$}{}_{#1}^{}}

The following example suggests at least that our analysis is not too loose.   Let $\H = \Span\{ \ket x : x \neq 0^n, 1^n \} \subset (\C^2)^{\otimes n}$.  Define $n$ qubits by $Z_j = \sigma^z_j \restrict{\H}$ and $X_j = \sigma^x_j \restrict{\H} + \sigma^x_j (\ketbra{1^n}{0^n} + \ketbra{0^n}{1^n}) \sigma^x_j$.  That is, while $\sigma^x_j$ maps the basis states $\sigma^x_j \ket{0^n}$ and $\sigma^x_j \ket{1^n}$ outside of~$\H$, $X_j$ instead maps them to each other.  Even though $\dim \H = 2^n - 2 < 2^n$, it seems that these $n$ qubits can pass our testing protocol with probability $1 - 1/\exp(n)$.\footnote{A natural generalization of this construction removes all strings of Hamming weight $< t$ or $> n-t$, with $Z_j = \sigma^z_j \restrict{\H}$ and $X_j \ket x = \sigma^x_j \ket x$ except $X_j \ket x = \ket{\overline x}$ when $\sigma^x_j \ket x$ would cross the boundary. We omit the details.}  
\end{remark}


\ifx\compilefullpaper\undefined  
%\addcontentsline{toc}{section}{References}
\bibliographystyle{alpha-eprint}
\bibliography{q}

\end{document}
\fi


\subsection*{Acknowledgements}

We would like to thank Greg Kuperberg for helpful comments, particularly regarding the proof of \thmref{t:qubitpacking}.  
R.C., B.R.~and C.S.~supported by NSF grant CCF-1254119 and ARO grant W911NF-12-1-0541.  T.V.~supported by NSF CAREER grant CCF-1553477, an AFOSR YIP award, and the IQIM, an NSF Physics Frontiers Center (NFS Grant PHY-1125565) with support of the Gordon and Betty Moore Foundation (GBMF-12500028).  


\bibliographystyle{alpha-eprint}
\bibliography{q}


\appendix


\section{Qubit packing using the exterior algebra} \label{s:qubitpackingprooftwo}

An alternative proof of \thmref{t:qubitpacking} was suggested to the authors by Greg Kuperberg~\cite{kuperberg14personal}.  The rough idea is to begin by packing nearly orthogonal unit vectors in $\R^n$, then define qubits using fermion creation and annihilation operators on the $2^n$-dimensional exterior algebra.   

\begin{proof}[Proof of \thmref{t:qubitpacking}]
By the Johnson-Lindenstrauss Lemma~\cite{JohnsonLindenstrauss84, DasguptaGupta03JohnsonLindenstrauss}, $e^{n \epsilon^2 / 8}$ unit vectors can be chosen in~$\R^n$ so that for any pair $\ket u, \ket v$, $\abs{\braket u v} \leq \epsilon$.  Pairing these vectors up arbitrarily, we obtain $m = \tfrac12 e^{n \epsilon^2 / 8}$ two-dimensional planes the angles between any two of which are in the range $(\tfrac\pi2 - \epsilon, \tfrac\pi2]$.  

If $\ket 1, \ldots, \ket n$ is a basis for $\R^n$, let $\Lambda(\R^n)$ be the $2^n$-dimensional exterior algebra, with basis $\ket{i_1} \wedge \ket{i_2} \wedge \ldots \wedge \ket{i_k}$ for $i_1, \ldots, i_k \in [n]$ and $k = 0, 1, \ldots, n$.  For a unit vector $\ket v \in \R^n$ and $\ket w \in \Lambda(\R^n)$, define the fermion creation and annihilation operators 
\begin{equation*}\begin{split}
a_v^\dagger \ket w &= \ket v \wedge \ket w \\
a_v \ket w &= (\bra v \otimes \identity) \ket w
 \enspace .
\end{split}\end{equation*}
Observe that this definition is basis independent, in the sense that for any unitary $R$ on $\R^n$, 
\begin{equation*}\begin{split}
a_{R v}^\dagger \hat R \ket w &= \hat R a_v^\dagger \ket w \\
a_{R v} \hat R \ket w &= \hat R a_v \ket w
 \enspace ,
\end{split}\end{equation*}
where $\hat R (\ket{v_1} \wedge \cdots \wedge \ket{v_k}) = (R \ket{v_1}) \wedge \cdots \wedge (R \ket{v_k})$.  

If we choose a basis for $\R^n$ beginning with $\ket v$, then $a_v^\dagger a_v$ projects onto those basis terms in $\Lambda(\R^n)$ that include $\ket v$, while $a_v a_v^\dagger$ projects onto the complementary set of basis terms.  Thus $a_v^\dagger a_v + a_v a_v^\dagger = \identity$, while also $a_v^2 = (a_v^\dagger)^2 = 0$.  Furthermore, if $\ket u$ is a unit vector perpendicular to $\ket v$, then the anticommutators satisfy $\{a_v, a_u\} = \{a_v^\dagger, a_u^\dagger\} = 0$, as $\ket u \wedge \ket v = - \ket v \wedge \ket u$, while if $\ket w$ has $k$ terms, 
\begin{align*}
a_u a_v^\dagger \ket w 
&= (\bra u \otimes \identity)(\ket v \wedge \ket w) \\
&= (-1)^k (\bra u \otimes \identity \ket w \wedge \ket v \\
&= -a_v^\dagger a_u \ket w
 \enspace .
\end{align*}
Thus $\{a_u, a_v^\dagger\} = 0$.  

Now for each of the $m$ pairwise nearly orthogonal planes, let $\{ \ket{u_j}, \ket{v_j} \}$ constitute an orthonormal basis.  Define 
\begin{equation}\begin{split}
X_j &= (-a_{u_j} + a_{u_j}^\dagger)(a_{v_j} + a_{v_j}^\dagger) \\
Z_j &= 2 a_{v_j} a_{v_j}^\dagger - \identity = a_{v_j} a_{v_j}^\dagger - a_{v_j}^\dagger a_{v_j}
 \enspace .
\end{split}\end{equation}
To understand this construction, observe that for orthonormal vectors $\ket u, \ket v \in \R^n$, and any $\ket w \in \Lambda(\R^n)$ with $a_u \ket w = a_v \ket w = 0$, the operators $a_u, a_u^\dagger, a_v, a_v^\dagger$ fix the subspace spanned by $\ket w, \ket v \wedge \ket w, \ket u \wedge \ket w, \ket u \wedge \ket v \wedge \ket w$.  In this basis, 
\begin{equation*}
a_u = \fastmatrix{0&0&1&0\\0&0&0&1\\0&0&0&0\\0&0&0&0}
\qquad 
a_v = \fastmatrix{0&1&0&0\\0&0&0&0\\0&0&0&-1\\0&0&0&0}
 \enspace .
\end{equation*}
Hence, 
\begin{gather*}
(-a_u + a_u^\dagger)(a_v + a_v^\dagger) = \fastmatrix{0&0&0&1\\0&0&1&0\\0&1&0&0\\1&0&0&0}
\qquad
2 a_v a_v^\dagger - \identity = \fastmatrix{1&0&0&0\\0&-1&0&0\\0&0&1&0\\0&0&0&-1}
 \enspace .
\end{gather*}
The former matrix is $\sigma_X \otimes \sigma_X$, and the latter matrix is $I \otimes \sigma_Z$, where $\sigma_X, \sigma_Z$ are the standard Pauli operators.  In particular, observe that $X_j^2 = Z_j^2 = \identity$, $X_j Z_j = -Z_j X_j$.  

The above construction satisfies that if $\ket{u_1}, \ket{v_1}, \ket{u_2}, \ket{v_2}$ are pairwise orthogonal, then $[X_1, X_2] = [X_1, Z_2] = [Z_1, X_2] = [Z_1, Z_2] = 0$.  The reason we use two vectors to define each $X_j, Z_j$ (instead of just taking $X = a_u + a_u^\dagger$, $Z = 2 a_u a_u^\dagger - \identity$) is to obtain the above commutation relationships.  Since $X_1, Z_1$ are each quadratic in $a_{u_1}, a_{u_1}^\dagger, a_{v_1}, a_{v_1}^\dagger$, terms involving only $a_{u_2}, a_{u_2}^\dagger, a_{v_2}, a_{v_2}^\dagger$ commute past them.  

Next, for \emph{nearly} orthogonal planes we will show that the commutator norm $\norm{[S_i, T_j]} = O(\epsilon)$, for $i\neq j$ and $S,T\in\{X,Z\}$.  

If $\ket u, \ket v$ are orthonormal, and $\ket t = \epsilon \ket u + \sqrt{1-\epsilon^2} \ket v$, then 
\begin{equation*}
a_t = \epsilon a_u + \sqrt{1 - \epsilon^2} a_v = \fastmatrix{0&\sqrt{1-\epsilon^2}&\epsilon&0\\0&0&0&\epsilon\\0&0&0&-\sqrt{1-\epsilon^2}\\0&0&0&0}
\end{equation*}
satisfies $\{a_t, a_u\} = 0$, $\{a_t, a_u^\dagger\} = \epsilon \identity$.  In general, 
\begin{align*}
\{a_t, a_u\} &= 0 \\
\{a_t, a_u^\dagger\} &= \braket{u}{t} \identity
 \enspace .
\end{align*}

It follows that if $\abs{\braket{u_1}{u_2}}, \abs{\braket{u_1}{v_2}}, \abs{\braket{v_1}{u_2}}, \abs{\braket{v_1}{v_2}} \leq \epsilon$, then $\norm{[S_1, T_2]} = O(\epsilon)$ for $S,T\in\{X,Z\}$.  Indeed, 
\begin{align*}
X_1 a_{u_2}
&= (-a_{u_1} + a_{u_1}^\dagger)(a_{v_1} + a_{v_1}^\dagger) a_{u_2} \\
&= -(-a_{u_1} + a_{u_1}^\dagger) \big[ a_{u_2} (a_{v_1} + a_{v_1}^\dagger) - \braket{u_2}{v_1} \identity \big] \\
&= \big[ a_{u_2} (-a_{u_1} + a_{u_1}^\dagger) - \braket{u_2}{u_1} \identity \big] (a_{v_1} + a_{v_1}^\dagger) + \braket{u_2}{v_1} (-a_{u_1} + a_{u_1}^\dagger) \\
&= a_{u_2} X_1 - \braket{u_2}{u_1} (a_{v_1} + a_{v_1}^\dagger) + \braket{u_2}{v_1} (-a_{u_1} + a_{u_1}^\dagger)
 \enspace ,
\end{align*}
implying $\norm{[X_1, a_{u_2}]} \leq \abs{\braket{u_2}{u_1}} + \abs{\braket{u_2}{v_1}} \leq 2 \epsilon$.  Similarly, 
\begin{align*}
Z_1 a_{u_2} 
&= (2 a_{u_1} a_{u_1}^\dagger - \identity) a_{u_2} \\
&= 2 a_{u_1} (\braket{u_1}{u_2} \identity - a_{u_2} a_{u_1}^\dagger) - a_{u_2} \\
&= a_{u_2} Z_1 + 2 \abs{\braket{u_1}{u_2}} a_{u_1}
 \enspace ,
\end{align*}
implying $\norm{[Z_1, a_{u_2}]} \leq 2 \abs{\braket{u_1}{u_2}} \leq 2 \epsilon$.  Thus $\norm{[S_1, T_2]} \leq c \, \epsilon$ for a fairly small constant~$c$.  
\end{proof}


\end{document}