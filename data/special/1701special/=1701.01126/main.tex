%!TEX program = pdflatex
%
% File naaclhlt2016.tex
%

\documentclass[11pt,letterpaper]{article}
\usepackage{coling2016}
\usepackage{times}
\usepackage{latexsym}

\usepackage{amsmath}
\usepackage{amssymb,amsmath,epsfig}
\usepackage{bbm}
\usepackage{cprotect}
\usepackage{tikz}
\usepackage{tikz-qtree}
\usepackage{graphics}
\usepackage{graphicx}
\usepackage{grffile}
\usepackage[position=b]{subcaption}

\usepackage{color}
\usepackage{tablefootnote}
\usepackage{calc}
\usepackage{array}
\usepackage[export]{adjustbox}
\usepackage{url}
%\makesavenoteenv{tabular}
%\makesavenoteenv{table}
%\naaclfinalcopy % Uncomment this line for the final submission
%\def\naaclpaperid{688} %  Enter the naacl Paper ID here

% To expand the titlebox for more authors, uncomment
% below and set accordingly.
% \addtolength\titlebox{.5in}    

\newcommand\BibTeX{B{\sc ib}\TeX}



% A

\newcommandx\A[2][1=]{
\ifthenelse{\equal{#1}{}}
{\hspace{-1mm}(\textbf{A\ref{#2}})\hspace{-1mm}}
{\hspace{-1mm}(\textbf{A\ref{#1}--\ref{#2}})\hspace{-1mm}}
}

% B

\newcommandx\B[2][1=]{
\ifthenelse{\equal{#1}{}}
{\hspace{-1mm}(\textbf{S})\hspace{-1mm}}
{\hspace{-1mm}(\textbf{S\ref{#1}--\ref{#2}})\hspace{-1mm}}
}
%\newcommandx\B[2][1=]{
%\ifthenelse{\equal{#1}{}}
%{\hspace{-1mm}(\textbf{S\ref{#2}})\hspace{-1mm}}
%{\hspace{-1mm}(\textbf{S\ref{#1}--\ref{#2}})\hspace{-1mm}}
%}
\newcommand{\bd}{c}
\newcommand{\bias}[2]{\beta_{#2} \langle #1 \rangle} 
\newcommand{\biasfilt}[2]{\bar{\beta}_{#2} \langle #1 \rangle} 
\newcommand{\binset}[1]{\mathsf{I}_{#1}}
\newcommand{\Binsp}[1]{\mathsf{B}_{#1}}
\newcommand{\bmf}[1]{\mathbb{F}(#1)}


% C

\newcommand{\cat}{\mathsf{Cat}}
\newcommand{\chunk}[3]{{#1}_{#2}^{#3}}

% D 

\newcommand{\DDelta}[3]{\Delta_{#1}\langle #2\rangle(#3)}
\newcommand{\dlim}{\stackrel{\mathcal D}{\longrightarrow}}

% E

\newcommand{\E}{\mathbb{E}}
\newcommand{\ed}{g}
\newcommand{\Efd}{\mathcal{E}}
\newcommand{\enoch}[3]{E_{#1,#2}^{#3}}
\newcommand{\epart}[2]{\xi_{#1}^{#2}}
\newcommand{\eqdef}{\vcentcolon=}
\newcommand{\Esp}{\mathsf{E}}
\newcommandx{\eve}[3][1=]{\ifthenelse{\equal{#1}{}}{E_{#2}^{#3}}{E_{#1,#2}^{#3}}}

% F

\newcommandx{\filt}[1][1=]{\ifthenelse{\equal{#1}{}}{\filtsymb}{\filtsymb \langle #1 \rangle}}
\newcommand{\filtpart}[1][1=]{\ifthenelse{\equal{#1}{}}{\filtsymb_\N}{\filtsymb_\N \langle #1 \rangle}}
\newcommand{\filtsymb}{\bar{\eta}}
\newcommandx{\filtvariance}[3][1=,3=]{\bar{\sigma}^{#1}_{#3} \langle #2 \rangle}

% G

\newcommandx{\gen}[1][1=]{\ifthenelse{\equal{#1}{1}}{G}{G'}} 
\newcommand{\genkernel}{\kernel{K}}

% H

\newcommand{\hk}{\kernel{M}}


% I

\newcommand{\init}{\chi}
\newcommand{\ind}[2]{I_{#1}^{#2}}
\newcommand{\intvect}[2]{\llbracket #1, #2 \rrbracket}

% K 

\newcommand{\kernel}[1]{\mathbf{#1}}

% L

\newcommand{\lag}{\lambda}
\newcommand{\lagtime}[2]{#1(#2)}
%\newcommand{\lagtime}[2]{\langle #1 \rangle_{#2}}
\newcommandx{\likeli}[3][1=]{\pi_{#1} \langle #2 \rangle(#3)}
\newcommand{\limitfunc}[1]{\pi \langle #1 \rangle}

% M

\newcommand{\md}{m}
\newcommand{\mdlow}{\ushort{\varepsilon}}
\newcommand{\mdup}{\bar{\varepsilon}}
%\newcommand{\mdr}{\mathcal{M}}
\newcommand{\mdr}{\mathbb{M}}
\newcommand{\me}{\mathrm{e}}
\newcommand{\mk}{\kernel{M}}
\newcommand{\mklow}{\ushort{\varepsilon}}
\newcommand{\mkup}{\bar{\varepsilon}}
\newcommand{\mumeas}[2]{\mu_{#1} \langle #2 \rangle}

% N

\newcommand{\N}{N}
\newcommand{\nset}{\mathbb{N}}
\newcommand{\nsetpos}{\mathbb{N}^\ast}

% O

\newcommand{\1}{\mathbbm{1}}
\newcommand{\ordo}{\mathcal{O}}

% P 

\newcommand{\p}{p}
\newcommand{\partfd}[1]{\mathcal{F}_{#1}}
\newcommand{\per}{\zeta}
\newcommand{\perblock}{\bar{\zeta}}
\newcommand{\plim}{\stackrel{\prob}{\longrightarrow}}
\newcommandx{\pot}[1][1=]{\ifthenelse{\equal{#1}{}}{g}{g \langle #1 \rangle}}
\newcommand{\potlow}{\ushort{\delta}}
\newcommand{\potup}{\bar{\delta}}
\newcommand{\predsymb}{\eta}
\newcommandx{\pred}[1][1=]{\ifthenelse{\equal{#1}{}}{\predsymb}{\predsymb \langle #1 \rangle}}
\newcommand{\predpart}[1][1=]{\ifthenelse{\equal{#1}{}}{\predsymb_\N}{\predsymb_\N \langle #1 \rangle}}
\newcommand{\prob}{\mathbb{P}}
\newcommand{\probmeas}[1]{\mathbb{M}(#1)}
\newcommand{\probdoeblin}[2]{\mu_{#1} \langle #2 \rangle}
\newcommand{\rmd}{\mathrm{d}}

% R

\newcommand{\rate}{\rho}
\newcommand{\refm}{\nu}
\newcommand{\rset}{\mathbb{R}}
\newcommand{\rsetpos}{\mathbb{R}^\ast_+}

% S

% T

\newcommand{\tbw}{\emph{(To be written.)}}
\newcommand{\term}[3][]{\upsilon_{#2,#3} \langle #1 \rangle}
%\newcommand{\thickhline}{%
%    \noalign {\ifnum 0=`}\fi \hrule height 2pt
%    \futurelet \reserved@a \@xhline
%}
%\newcolumntype{w}{@{\hskip\tabcolsep\vrule width 2pt\hskip\tabcolsep}}
%\makeatother

% U 

\newcommandx{\uk}[1][1=]{\ifthenelse{\equal{#1}{}}{\kernel{Q}}{\kernel{Q} \langle #1 \rangle}}
\newcommand{\unitstr}[2]{1_{#1}}
\newcommand{\unpredsymbol}{\gamma}
\newcommandx{\unpred}[1][1=]{\ifthenelse{\equal{#1}{}}{\unpredsymb}{\unpredsymb \langle #1 \rangle}}
\newcommand{\unpredpart}[1][1=]{\ifthenelse{\equal{#1}{}}{\unpredsymbol_\N}{\unpredsymbol_\N \langle #1 \rangle}}

% V 

\newcommandx{\varest}[3][1=,3=]{\ifthenelse{\equal{#3}{}}{\sigma^{#1}_\N \langle #2 \rangle}{\sigma^{#1}_{\N, #3} \langle #2 \rangle}}
\newcommandx{\varestfilt}[3][1=,3=]{\ifthenelse{\equal{#3}{}}{\bar{\sigma}^{#1}_\N \langle #2 \rangle}{\bar{\sigma}^{#1}_{\N, #3} \langle #2 \rangle}}
\newcommandx{\variance}[3][1=,3=]{\sigma^{#1}_{#3} \langle #2 \rangle}


% W 

\newcommand{\wgt}[2]{\omega_{#1}^{#2}}
\newcommand{\wgtsum}[1]{\Omega_{#1}}

% X 

\newcommand{\Xsp}{\mathsf{X}}
\newcommand{\Xfd}{\mathcal{X}}

% Y

\newcommand{\Ysp}{\mathsf{Y}}
\newcommand{\Yfd}{\mathcal{Y}}

% Z

\newcommand{\zerostr}[1]{0_{#1}}
\newcommand{\Zsp}{\mathsf{Z}}
\newcommand{\Zfd}{\mathcal{Z}}
\newcommand{\zset}{\mathbb{Z}}

% Hypotheses

\newcounter{hypA}
\newenvironment{hypA}{\refstepcounter{hypA}\begin{itemize}
  \item[({\bf A\arabic{hypA}})]}{\end{itemize}}
%\newenvironment{hypA}{\begin{sf}\refstepcounter{hypA}\begin{itemize}
%  \item[({\bf A\arabic{hypA}})]}{\end{itemize}\end{sf}}

\newcounter{hypB}
%\newenvironment{hypB}{\refstepcounter{hypB}\begin{itemize}
%  \item[({\bf S\arabic{hypB}})]}{\end{itemize}}
\newenvironment{hypB}{\refstepcounter{hypB}\begin{itemize}
  \item[({\bf S})]}{\end{itemize}}



\title{Textual Entailment with Structured Attentions and Composition}

% Author information can be set in various styles:
% For several authors from the same institution:
% \author{Author 1 \and ... \and Author n \\
%         Address line \\ ... \\ Address line}
% if the names do not fit well on one line use
%         Author 1 \\ {\bf Author 2} \\ ... \\ {\bf Author n} \\
% For authors from different institutions:
% \author{Author 1 \\ Address line \\  ... \\ Address line
%         \And  ... \And
%         Author n \\ Address line \\ ... \\ Address line}
% To start a seperate ``row'' of authors use \AND, as in
% \author{Author 1 \\ Address line \\  ... \\ Address line
%         \AND
%         Author 2 \\ Address line \\ ... \\ Address line \And
%         Author 3 \\ Address line \\ ... \\ Address line}
% If the title and author information does not fit in the area allocated,
% place \setlength\titlebox{<new height>} right after
% at the top, where <new height> can be something larger than 2.25in
\author{Kai Zhao \and Liang Huang \and Mingbo Ma \\
School of Electrical Engineering and Computer Science \\ 
Oregon State University \\ Corvallis, Oregon, USA \\
{\tt \{kzhao.hf, lianghuang.sh, cosmmb\}@gmail.com}}

\date{}

\begin{document}

\maketitle

\vspace{-0.35in}
\begin{abstract}
Deep learning techniques are increasingly popular 
in the textual entailment task, % is witnessing a burgeoning
%interest in leveraging the generalization power of 
%deep learning techniques
overcoming the fragility of traditional discrete models with hard alignments and logics.
In particular, the recently proposed attention models 
\cite{rocktaschel2015reasoning,wang2015learning} achieves state-of-the-art accuracy by
computing soft word alignments between %words
the premise and hypothesis sentences.
However, there remains a major limitation:
this line of work completely ignores syntax and recursion,
%outperforming traditional models with sparse features. % on a large dataset.
%However, syntactic trees, 
which is helpful in many traditional efforts.
%is completely ignored in this line of work.
We show that it is beneficial to extend the attention model 
to tree nodes between premise and hypothesis.
More importantly, this subtree-level
attention reveals information about entailment relation.
We study the recursive composition of this subtree-level entailment relation,
%and propose to combine the attention calculation and the entailment
%composition,
which can be viewed as a soft version of 
the Natural Logic framework
\cite{maccartney2009extended}.
Experiments show that our structured attention and
entailment composition model can correctly identify and infer
entailment relations from the bottom up,
and bring significant improvements in accuracy.
\end{abstract}


\section{Introduction}
\label{sec:intro}
%!TEX root = /Users/audrey/Dropbox/PhD/MOMAB/ArXiv/Latex/paper.tex

\section{Introduction}
\label{sec:intro}

Multi-objective optimization (MOO)~\cite{Coello2007} is a topic of great importance for real-world applications. Indeed, optimization problems are characterized by a number of conflicting, even contradictory, performance measures relevant to the task at hand. For example, when deciding on the healthcare treatment to follow for a given sick patient, a trade-off must be made between the efficiency of the treatment to heal the sickness, the side effects of the treatment, and the treatment cost. MOO is often tackled by combining the objective into a single measure (a.k.a.~scalarization). Such approaches are said to be \emph{a priori}, as the preferences over the objectives is defined before carrying out the optimization itself. The challenge lies in the determination of the appropriate scalarization function to use and its parameterization. Another way to conduct MOO consists in learning the optimal trade-offs (the so-called Pareto-optimal set). Once the optimization is completed, techniques from the field of multi-criteria decision-making are applied to help the user to select the final solution from the Pareto-optimal set. These \emph{a posteriori} techniques may require a huge number of evaluations to have a reliable estimation of the objective values over all potential solutions. Indeed, the Pareto-optimal set can be quite large, encompassing a majority, if not all, of the potential solutions. In this work, we tackle the MOO problem where the scalarization function \emph{exists} a priori, but might be unknown, in which case a user can act as a black box for articulating preferences. Integrating the user to the learning loop, she can provide feedback by selecting her preferred choice given a set of options -- the scalarization function lying in her head.

More specifically, we consider problems where outcomes are stochastic and costly to evaluate (e.g., involving a human in the loop). The challenge is therefore to identify the best solutions given random observations sampled from different (unknown) density distributions. We formulate this problem as multi-objective bandits, where we aim at finding the solution that maximizes the preference function while maximizing the performance of the solutions evaluated during the optimization. The Thompson sampling (TS)~\cite{Thompson1933} technique is a typical approach for bandits problems, where potential solutions are tried based on a Bayesian posterior over their expected outcome. Here we consider TS from multivariate normal (MVN) priors for multi-objective bandits.
% Let the \emph{right choice} denote the option that maximize the preference function -- the option that the user would select given that she had knowledge of the Pareto-optimal set. A learning algorithm for the multi-objective bandits setting aims at learning good-enough estimations of the available options to allow the user to make the right choices and its performance depends on the robustness of the preference function to the quality of estimations. We therefore need a measure for characterizing the quality of estimations required in order for the option maximizing the preference function to remain unchanged. For that purpose, we introduce the concept of preference radius providing the tolerance range over objective value estimations, such that the user preference would remain the same as if the Pareto-optimal set was known. We use this concept for providing a theoretical analysis of TS from MVN priors.
We introduce the concept of preference radius providing the tolerance range over objective value estimations, such that the \emph{best option} given the preference function remains unchanged. We use this concept for providing a theoretical analysis of TS from MVN priors.
%
Finally, we perform some empirical experiments to support the theoretical results and also highlight the importance of tackling multi-objective bandits problems as such instead of scalarizing those under the traditional bandit setting. 

% The original contributions of the paper consist in:
% \begin{itemize}
%     \item providing a general formulation of the MOO under the a priori multi-objective bandits setting;
%     \item proposing the preference radius to characterize the robustness of the preference function to the estimations quality;
%     \item proposing a theoretical analysis of the TS algorithm from MVN priors;
%     \item showing with empirical experiments that multi-objective bandits cannot simply be brought back to single-objective bandits.
% \end{itemize}



\section{Structured Attentions \& Entailment Composition}
\label{sec:model}
\input{attention}

\iffalse
\section{Structured Tree Entailment}
\label{sec:entailment}
\input{entailment}
\fi

\section{Review: Recursive Tree Meaning Representations}
\label{sec:treelstm}
\input{treelstm}

\section{Empirical Evaluations}
\label{sec:exp}
\input{exp}

\iffalse
\section{Discussion}
\label{sec:disc}
%!TEX root = main.tex

Although many attention-based
models, including our model, 
achieve superior 
results in the textual entailment task,
we can still see the limitations
for this approach.

Despite those sentence pairs that require
more common knowledge to find the entailment relations,
we are more interested in sentences that are difficult
because they involve interesting linguistic properties.

Consider the following two pairs of sentences
that are difficult for current attention-based models:
\begin{enumerate}
\item \begin{itemize}
\item Premise: The boy loves the girl.
\item Hypothesis: The girl loves the boy.
\end{itemize}
Here the only difference between the two sentences
is the order/structure of the words. To handle this problem
the attention-based models should take the reordering
into consideration.
\item 
\begin{itemize}
\item Premise: A stuffed animal on the couch.
\item Hypothesis: An animal on the couch.
\end{itemize}
In this example, every hypothesis word occurs in the premise sentence,
but it is difficult to learn that ``a stuffed animal'' 
is not ``an animal''.
\end{enumerate}


\fi

\section{Conclusion}
\label{sec:conclusion}
We have presented an approach to model the composition
of the entailment relation following the tree structure for the sentence entailment task. We adapted the attention model
for tree structures. Experiments show that
our model bring significant improvements in accuracy,
and is easy to interpret.

\section*{Acknowledgments}
We thank the anonymous reviewers for helpful comments.
We are also grateful to James Cross, Dezhong Deng, and Lemao Liu for suggestions.
This project was supported in part
by NSF IIS-1656051, DARPA FA8750-13-2-0041
(DEFT), and a Google Faculty Research Award.

\bibliography{entailment}
\bibliographystyle{acl}


\end{document}
