We presented for the first time a polynomial algorithm to extract the core structural connectivity network of a population combining a graph-theoretical approach with statistic relevance of the connections, observing the recent evidence of the structural network topology.

Our results show that our algorithm outperforms, in the prediction experiment, the most used technique~\cite{Gong2009} as well as latest approaches~\cite{Wassermann2016}. In Table~\ref{table:number_of_features} we can see that our algorithm preserves, in average, more connections correlated with the handedness of the subjects. We can also see that despite being less stable than Wassermann et al.'s it is stabler than Gong et al.'s. Finally, Fig.~\ref{fig:prediction_performance} shows that, in the handedness prediction experiment, our method outperforms  Gong et al.'s and Wassermann et al's: the number of cases with lower AIC and MSE is larger in our case. Hence, our CSCN is better as linear model relating connectivity with handedness in terms of model fitting and prediction.


In terms of theoretical contributions, we formalized the problem, proved its difficulty and gave a novel algorithm for dealing with it. We then validated our approach by showing its power as feature selector for getting connections related to handedness with 300 real subjects' data. The experiment shows our method performs better than the currently available. Moreover, our method avoids the double dipping problem by not choosing the feature set with hypothesis testing.