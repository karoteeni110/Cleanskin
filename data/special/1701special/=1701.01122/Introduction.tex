\begin{abstract}

We investigate the connection between the star formation rate (SFR) of galaxies
and their central black hole accretion rate (BHAR) using the \eagle
cosmological hydrodynamical simulation.  We find, in striking concurrence with
recent observational studies, that the \av{SFR}--BHAR relation for an AGN
selected sample produces a relatively flat trend, whilst the \av{BHAR}--SFR
relation for a SFR selected sample yields an approximately linear trend.  These
trends remain consistent with their instantaneous equivalents even when both
SFR and BHAR are time-averaged over a period of 100~Myr.  There is no universal
relationship between the two growth rates. Instead, SFR and BHAR evolve through
distinct paths that depend strongly on the mass of the host dark matter halo.
The galaxies hosted by haloes of mass \M{200} $\lesssim 10^{11.5}$\Msol grow
steadily, yet black holes (BHs) in these systems hardly grow, yielding a lack
of correlation between SFR and BHAR. As haloes grow through the mass range
$10^{11.5} \lesssim$ \M{200} $\lesssim 10^{12.5 }$\Msol BHs undergo a rapid
phase of non-linear growth. These systems yield a highly non-linear correlation
between the SFR and BHAR, which are non-causally connected via the mass of the
host halo.  In massive haloes (\M{200} $\gtrsim 10^{12.5}$\Msol) both SFR and
BHAR decline on average with a roughly constant scaling of SFR/BHAR $\sim
10^{3}$.  Given the complexity of the full SFR--BHAR plane built from multiple
behaviours, and from the large dynamic range of BHARs, we find the primary
driver of the different observed trends in the \av{SFR}--BHAR and
\av{BHAR}--SFR relationships are due to sampling considerably different regions
of this plane.  

\end{abstract}

\begin{keywords}
galaxies: active -- galaxies: evolution
\end{keywords}

\section{Introduction}
\label{sect:introduction}

Substantial effort has been dedicated both observationally and theoretically to
identifying the link between the growth of galaxies and their central
supermassive black holes (BHs). However, the nature of this relationship
remains poorly understood.  Indirect evidence of a causal connection has been
suggested empirically based on the \textit{integrated} properties of galaxies
and their BH counterparts. For example, galaxy bulge mass (\M{*, bulge}) and
the mass of the central BH (\M{BH}) exhibit a tight, approximately linear
correlation for bulge masses in excess of \M{*, bulge} $\sim 10^{10}$\Msol
\citep[e.g,][]{Magorrian1998,Kormendy2013,McConnellandMa2013,Scott2013}.
However, at lower bulge mass, a steeper trend has been advocated
\citep[e.g,][]{Scott2013,Greene2016}.  Additionally, the cosmic star formation
rate (SFR) and black hole accretion rate (BHAR) densities broadly trace one
another through time
\citep[e.g,][]{Heckman2004,Aird2010,MadauandDickinson2014}.

A simple interpretation for these global relationships is that the growth rates
that build these properties (i.e. the SFR of the galaxy and accretion rate of
the BH) are proportional throughout their evolution, thus growing the two
components in concert. More complex evolutionary scenarios have also been
proposed. For example, a simple time-averaged relationship built from a common
fuel reservoir of cold gas \citep{AlexanderandHickox2012,Hickox2014}, a rapid
build up of galaxy and BH mass via merger induced starburst/quasar activity
\citep[e.g,][]{Sanders1988,DiMatteo2005,Hopkins2008} or a mutual dependence on
the mass or potential of the dark matter halo
\citep{BoothandSchaye2010,BoothandSchaye2011,Bower2017}.  In these scenarios
the SFR and BHAR do not necessarily trace each other directly and the observed
correlations may only appear in massive galaxies due to an averaging of very
different histories.  Furthermore, \citet{Peng2007} and \citet{Jahnke2011} go
as far as to suggest there is no causal connection of any kind, with correlations
only appearing as result of a random walk. 

To test these scenarios, numerous observational studies have attempted to
identify a direct link between the intrinsic growth rates of galaxies and their
central BHs. Studies that investigate the mean SFR (\av{SFR}) as a function of
BHAR consistently find no evidence for a correlation for moderate-luminosity
sources \citep[\LXray $\lesssim$ $10^{44}$~erg~s$^{-1}$;
e.g,][]{Lutz2010,Harrison2012,Page2012,Mullaney2012b,Rosario2012,Stanley2015,Azadi2015}.
For high-luminosity sources (\LXray > $10^{44}$~erg~s) however, there has been
significant disagreement as to if this relation becomes positively correlated
\citep[e.g.][]{Lutz2010}, negatively correlated \citep[e.g.][]{Page2012} or
continues to remain uncorrelated
\citep[e.g,][]{Harrison2012,Rosario2012,Stanley2015,Azadi2015}.  These
disparities between various works at the high-luminosity end are likely due to
small number statistics and sample variance \citep{Harrison2012}, and indeed,
recent studies using large sample sizes confirm the extension of a flat trend
to higher luminosities \citep{Stanley2015,Azadi2015}.

A flat trend for the \av{SFR}--BHAR relation could potentially be interpreted
as revealing an absence of a connection between SFR and BHAR. However, studies
that have investigated the mean BHAR (\av{BHAR}) as a function of SFR
consistently find a \textit{positive} relationship
\citep[e.g,][]{Rafferty2011,Symeonidis2011,Mullaney2012a,Chen2013,Delvecchio2015}.
Within the paradigm of a linear \M{BH}--\M{bulge,*} relation due to a universal
co-evolution of BH and galaxy growth, both approaches are expected to produce a
consistent, similarly linear result (see \cref{sect:motivation} for a
derivation of why this is).  \citet{Hickox2014} proposes a potential solution,
suggesting that SFR and BHAR are connected \textit{on average} over a period of
100~Myr, with a linear scaling. This relationship disappears when measured
instantaneously owing to the rapid variability timescale of AGN, with respect
to that of galactic star formation.  

From a theoretical perspective, many simulations have focussed on the growth of
BHs in galaxy mergers \citep[e.g,][]{DiMatteo2005,Hopkins2005}. Whilst both
star formation and BH accretion are typically enhanced during the merger
proper, the extent of the connection between SFR and BHAR pre- and post-merger
event remains unclear. \citet{Neistein2014} demonstrate through the use of a
semi-analytical model that the observed correlations between galaxies and
their central BHs can be reproduced when BH growth occurs only during merger
induced starbursts. This could explain the lack of a correlation between growth
rates in low-luminosity systems whilst allowing for mutual enhancement during
the merger events themselves.  \citet{Thacker2014} investigate the impact of
various feedback models on the SFR--BHAR parameter space in a set of equal mass
merger simulations. They find a complex evolution for individual systems, even
when averaged over 20~Myrs. Any correlation found is strongly dependant on the
feedback model chosen, with the post-merger phase showing the strongest
evidence for a positive connection. Using a high-resolution hydrodynamical
merger suite, \citet{Volonteri2015a} find BHAR and galaxy-wide SFR to be
typically temporally uncorrelated. They suggest in \citet{Volonteri2015b} that
the observed discrepancy between the \av{SFR}--BHAR and \av{BHAR}--SFR
relations seen observationally are a result of sampling two different
projections of the full bi-variate SFR--BHAR distribution whose build up is
constructed from different behaviours between SFR and BHAR before, during and
after the merger event. 

It is now possible to extend these investigations to within a full cosmological
context. Using the semi-analytical code \galform, \citet{Gutcke2015} find a
negative SFR--AGN luminosity correlation at low AGN luminosities, this then
transitions to a strong positive correlation at high AGN luminosities.  In the
cosmological hydrodynamical simulation \illustris, \citet{Sijacki2015} find a
single trend in the SFR--BHAR relationship embedded in a large scatter,
particularly in BHAR.  Cosmological hydrodynamical simulations have the
advantage of probing the entire galaxy population within a self consistent
variety of environments with a diverse range of accretion and merger histories.
Here we investigate to what extent galaxy and BH growth rates are connected
within the \dquotes{Evolution and Assembly of GaLaxies and their Environment}
\citep[\eagle,][]{Schaye2015,Crain2015} \footnote{\url{www.eaglesim.org}}
\footnote{Galaxy and halo catalogues of the simulation suite are publicly
available at \url{http://www.eaglesim.org/database.php} \citep{McAlpine2015}.}
simulation.  Adopting physical prescriptions for radiative cooling, star
formation, stellar mass loss, BH accretion, BH mergers and both stellar and AGN
feedback, \eagle reproduces many observed properties of galaxies, BHs and the
intergalacic medium with unprecedented fidelity \citep[e.g,][]{Schaye2015,
Furlong2015a, Furlong2015b, Trayford2015, Schaller2015a, Lagos2015,
Rahmati2015, Bahe2016, Crain2016, RosasGuevara2016, Segers2016, Trayford2016}. 

The paper is organised as follows.  In \cref{sect:simulationsandsubgrid} we
provide a brief overview of the \eagle simulation suite, including the subgrid
model prescription and simulation output.  The results are presented in
\cref{sect:results}.  We examine the \eagle predictions of the \av{SFR}--BHAR
relationship for an AGN selected sample and the \av{BHAR}--SFR relationship for
a SFR selected sample in \cref{sect:observations}, finding good agreement to
recent observational findings.  To investigate why these trends might be
different, we explore the effect of time-averaging each growth rate and examine
potential sampling biases in \cref{sect:understanding}.  \cref{sect:to_halo}
examines the influence of the host dark matter halo on both SFR and BHAR,
finding that each exhibits a strong connection.  Finally in
\cref{sect:discussion}, we discuss the changing relationship between SFR and
BHAR as the halo grows and in \cref{sect:conclusions}, we present our
conclusions.

