%!TEX root = ../wbi.tex
\section{Conclusions}
\label{sec:conclusions}

In this paper we presented a software abstraction layer to simplify the development of whole-body controllers.
While there are already some whole-body control software libraries, they already define the controller structure and leave to the user only the possibility to specify objectives and constraints.

On the other hand the proposed library leaves complete freedom to the control designer by exposing all the information needed. It does not make any assumptions on the controller structure.
The whole-body abstraction library presents also the following advantages:
\begin{itemize}
    \item it decouples the writing of the controller from a particular robot implementation
    \item it decouples the writing of the controller from a specific dynamic library implementation
    \item it allows more concise and clear code as it represents uniquely the code needed to implement the mathematical formulation of the controller. All the implementation details are left to the library
    \item it allows to benchmark the controller on different platforms or with different implementations.
\end{itemize}
Furthermore, the possibility to expose the functionality at an higher level than C++ facilitates the writing of controllers as the results on the iCub robot clearly prove.

We voluntarily did not consider some aspects as they are out of the scope of the present contribution. 
Nevertheless they must be taken into account when a controller is implemented and used on the real system.
In particular the following details should be considered:
\begin{itemize}
    \item how are controllers run on the platform? Do they run as threads?
    \item how are controllers configured and initialized?
    \item how is communication with other software performed? For example, how are desired values provided to the controller, coming from a planner or higher-level control loop?
\end{itemize}
By not considering these details in the abstraction library, we render the library portable to different systems.
Indeed, the actual control law is not concerned by the previously listed implementation details.

While the more complex demos have been achieved by directly executing the Simulink model connected to the robot, we recognize the need to automatically generate self-contained C++ code.
The advantage is twofold.
On one side the autogenerated code is in general more optimized than the code directly executed in Simulink, even if less optimized than ad-hoc C++ code.
On the other side, this would remove the requirement of having a Simulink installation on the computers controlling the robot.
