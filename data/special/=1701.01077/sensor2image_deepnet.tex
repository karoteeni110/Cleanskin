
%% bare_conf.tex
%% V1.4b
%% 2015/08/26
%% by Michael Shell
%% See:
%% http://www.michaelshell.org/
%% for current contact information.
%%
%% This is a skeleton file demonstrating the use of IEEEtran.cls
%% (requires IEEEtran.cls version 1.8b or later) with an IEEE
%% conference paper.
%%
%% Support sites:
%% http://www.michaelshell.org/tex/ieeetran/
%% http://www.ctan.org/pkg/ieeetran
%% and
%% http://www.ieee.org/

%%*************************************************************************
%% Legal Notice:
%% This code is offered as-is without any warranty either expressed or
%% implied; without even the implied warranty of MERCHANTABILITY or
%% FITNESS FOR A PARTICULAR PURPOSE! 
%% User assumes all risk.
%% In no event shall the IEEE or any contributor to this code be liable for
%% any damages or losses, including, but not limited to, incidental,
%% consequential, or any other damages, resulting from the use or misuse
%% of any information contained here.
%%
%% All comments are the opinions of their respective authors and are not
%% necessarily endorsed by the IEEE.
%%
%% This work is distributed under the LaTeX Project Public License (LPPL)
%% ( http://www.latex-project.org/ ) version 1.3, and may be freely used,
%% distributed and modified. A copy of the LPPL, version 1.3, is included
%% in the base LaTeX documentation of all distributions of LaTeX released
%% 2003/12/01 or later.
%% Retain all contribution notices and credits.
%% ** Modified files should be clearly indicated as such, including  **
%% ** renaming them and changing author support contact information. **
%%*************************************************************************


% *** Authors should verify (and, if needed, correct) their LaTeX system  ***
% *** with the testflow diagnostic prior to trusting their LaTeX platform ***
% *** with production work. The IEEE's font choices and paper sizes can   ***
% *** trigger bugs that do not appear when using other class files.       ***                          ***
% The testflow support page is at:
% http://www.michaelshell.org/tex/testflow/



\documentclass[conference]{IEEEtran}
% Some Computer Society conferences also require the compsoc mode option,
% but others use the standard conference format.
%
% If IEEEtran.cls has not been installed into the LaTeX system files,
% manually specify the path to it like:
% \documentclass[conference]{../sty/IEEEtran}

\usepackage{graphicx}
\DeclareGraphicsExtensions{.pdf,.jpg,.png,.svg}
\graphicspath{{figures/}}

\usepackage{footmisc}
\makeatletter
\let\old@ps@headings\ps@headings
\let\old@ps@IEEEtitlepagestyle\ps@IEEEtitlepagestyle
\def\confheader#1{%
	% for all pages except the first
	\def\ps@headings{%
		\old@ps@headings%
		\def\@oddhead{\strut\hfill#1\hfill\strut}%
		\def\@evenhead{\strut\hfill#1\hfill\strut}%
	}%
	% for the first page
	\def\ps@IEEEtitlepagestyle{%
		\old@ps@IEEEtitlepagestyle%
		\def\@oddhead{\strut\hfill#1\hfill\strut}%
		\def\@evenhead{\strut\hfill#1\hfill\strut}%
	}%
	\ps@headings%
}
\makeatother

\confheader{%
	Published in: 2017 International Joint Conference on Neural Networks (IJCNN),
	14-19 May 2017, Anchorage, AK, USA
}
% Some very useful LaTeX packages include:
% (uncomment the ones you want to load)


% *** MISC UTILITY PACKAGES ***
%
%\usepackage{ifpdf}
% Heiko Oberdiek's ifpdf.sty is very useful if you need conditional
% compilation based on whether the output is pdf or dvi.
% usage:
% \ifpdf
%   % pdf code
% \else
%   % dvi code
% \fi
% The latest version of ifpdf.sty can be obtained from:
% http://www.ctan.org/pkg/ifpdf
% Also, note that IEEEtran.cls V1.7 and later provides a builtin
% \ifCLASSINFOpdf conditional that works the same way.
% When switching from latex to pdflatex and vice-versa, the compiler may
% have to be run twice to clear warning/error messages.






% *** CITATION PACKAGES ***
%
%\usepackage{cite}
% cite.sty was written by Donald Arseneau
% V1.6 and later of IEEEtran pre-defines the format of the cite.sty package
% \cite{} output to follow that of the IEEE. Loading the cite package will
% result in citation numbers being automatically sorted and properly
% "compressed/ranged". e.g., [1], [9], [2], [7], [5], [6] without using
% cite.sty will become [1], [2], [5]--[7], [9] using cite.sty. cite.sty's
% \cite will automatically add leading space, if needed. Use cite.sty's
% noadjust option (cite.sty V3.8 and later) if you want to turn this off
% such as if a citation ever needs to be enclosed in parenthesis.
% cite.sty is already installed on most LaTeX systems. Be sure and use
% version 5.0 (2009-03-20) and later if using hyperref.sty.
% The latest version can be obtained at:
% http://www.ctan.org/pkg/cite
% The documentation is contained in the cite.sty file itself.






% *** GRAPHICS RELATED PACKAGES ***
%
\ifCLASSINFOpdf
  % \usepackage[pdftex]{graphicx}
  % declare the path(s) where your graphic files are
  % \graphicspath{{../pdf/}{../jpeg/}}
  % and their extensions so you won't have to specify these with
  % every instance of \includegraphics
  % \DeclareGraphicsExtensions{.pdf,.jpeg,.png}
\else
  % or other class option (dvipsone, dvipdf, if not using dvips). graphicx
  % will default to the driver specified in the system graphics.cfg if no
  % driver is specified.
  % \usepackage[dvips]{graphicx}
  % declare the path(s) where your graphic files are
  % \graphicspath{{../eps/}}
  % and their extensions so you won't have to specify these with
  % every instance of \includegraphics
  % \DeclareGraphicsExtensions{.eps}
\fi
% graphicx was written by David Carlisle and Sebastian Rahtz. It is
% required if you want graphics, photos, etc. graphicx.sty is already
% installed on most LaTeX systems. The latest version and documentation
% can be obtained at: 
% http://www.ctan.org/pkg/graphicx
% Another good source of documentation is "Using Imported Graphics in
% LaTeX2e" by Keith Reckdahl which can be found at:
% http://www.ctan.org/pkg/epslatex
%
% latex, and pdflatex in dvi mode, support graphics in encapsulated
% postscript (.eps) format. pdflatex in pdf mode supports graphics
% in .pdf, .jpeg, .png and .mps (metapost) formats. Users should ensure
% that all non-photo figures use a vector format (.eps, .pdf, .mps) and
% not a bitmapped formats (.jpeg, .png). The IEEE frowns on bitmapped formats
% which can result in "jaggedy"/blurry rendering of lines and letters as
% well as large increases in file sizes.
%
% You can find documentation about the pdfTeX application at:
% http://www.tug.org/applications/pdftex





% *** MATH PACKAGES ***
%
%\usepackage{amsmath}
% A popular package from the American Mathematical Society that provides
% many useful and powerful commands for dealing with mathematics.
%
% Note that the amsmath package sets \interdisplaylinepenalty to 10000
% thus preventing page breaks from occurring within multiline equations. Use:
%\interdisplaylinepenalty=2500
% after loading amsmath to restore such page breaks as IEEEtran.cls normally
% does. amsmath.sty is already installed on most LaTeX systems. The latest
% version and documentation can be obtained at:
% http://www.ctan.org/pkg/amsmath





% *** SPECIALIZED LIST PACKAGES ***
%
%\usepackage{algorithmic}
% algorithmic.sty was written by Peter Williams and Rogerio Brito.
% This package provides an algorithmic environment fo describing algorithms.
% You can use the algorithmic environment in-text or within a figure
% environment to provide for a floating algorithm. Do NOT use the algorithm
% floating environment provided by algorithm.sty (by the same authors) or
% algorithm2e.sty (by Christophe Fiorio) as the IEEE does not use dedicated
% algorithm float types and packages that provide these will not provide
% correct IEEE style captions. The latest version and documentation of
% algorithmic.sty can be obtained at:
% http://www.ctan.org/pkg/algorithms
% Also of interest may be the (relatively newer and more customizable)
% algorithmicx.sty package by Szasz Janos:
% http://www.ctan.org/pkg/algorithmicx




% *** ALIGNMENT PACKAGES ***
%
%\usepackage{array}
% Frank Mittelbach's and David Carlisle's array.sty patches and improves
% the standard LaTeX2e array and tabular environments to provide better
% appearance and additional user controls. As the default LaTeX2e table
% generation code is lacking to the point of almost being broken with
% respect to the quality of the end results, all users are strongly
% advised to use an enhanced (at the very least that provided by array.sty)
% set of table tools. array.sty is already installed on most systems. The
% latest version and documentation can be obtained at:
% http://www.ctan.org/pkg/array


% IEEEtran contains the IEEEeqnarray family of commands that can be used to
% generate multiline equations as well as matrices, tables, etc., of high
% quality.




% *** SUBFIGURE PACKAGES ***
%\ifCLASSOPTIONcompsoc
%  \usepackage[caption=false,font=normalsize,labelfont=sf,textfont=sf]{subfig}
%\else
%  \usepackage[caption=false,font=footnotesize]{subfig}
%\fi
% subfig.sty, written by Steven Douglas Cochran, is the modern replacement
% for subfigure.sty, the latter of which is no longer maintained and is
% incompatible with some LaTeX packages including fixltx2e. However,
% subfig.sty requires and automatically loads Axel Sommerfeldt's caption.sty
% which will override IEEEtran.cls' handling of captions and this will result
% in non-IEEE style figure/table captions. To prevent this problem, be sure
% and invoke subfig.sty's "caption=false" package option (available since
% subfig.sty version 1.3, 2005/06/28) as this is will preserve IEEEtran.cls
% handling of captions.
% Note that the Computer Society format requires a larger sans serif font
% than the serif footnote size font used in traditional IEEE formatting
% and thus the need to invoke different subfig.sty package options depending
% on whether compsoc mode has been enabled.
%
% The latest version and documentation of subfig.sty can be obtained at:
% http://www.ctan.org/pkg/subfig




% *** FLOAT PACKAGES ***
%
%\usepackage{fixltx2e}
% fixltx2e, the successor to the earlier fix2col.sty, was written by
% Frank Mittelbach and David Carlisle. This package corrects a few problems
% in the LaTeX2e kernel, the most notable of which is that in current
% LaTeX2e releases, the ordering of single and double column floats is not
% guaranteed to be preserved. Thus, an unpatched LaTeX2e can allow a
% single column figure to be placed prior to an earlier double column
% figure.
% Be aware that LaTeX2e kernels dated 2015 and later have fixltx2e.sty's
% corrections already built into the system in which case a warning will
% be issued if an attempt is made to load fixltx2e.sty as it is no longer
% needed.
% The latest version and documentation can be found at:
% http://www.ctan.org/pkg/fixltx2e


%\usepackage{stfloats}
% stfloats.sty was written by Sigitas Tolusis. This package gives LaTeX2e
% the ability to do double column floats at the bottom of the page as well
% as the top. (e.g., "\begin{figure*}[!b]" is not normally possible in
% LaTeX2e). It also provides a command:
%\fnbelowfloat
% to enable the placement of footnotes below bottom floats (the standard
% LaTeX2e kernel puts them above bottom floats). This is an invasive package
% which rewrites many portions of the LaTeX2e float routines. It may not work
% with other packages that modify the LaTeX2e float routines. The latest
% version and documentation can be obtained at:
% http://www.ctan.org/pkg/stfloats
% Do not use the stfloats baselinefloat ability as the IEEE does not allow
% \baselineskip to stretch. Authors submitting work to the IEEE should note
% that the IEEE rarely uses double column equations and that authors should try
% to avoid such use. Do not be tempted to use the cuted.sty or midfloat.sty
% packages (also by Sigitas Tolusis) as the IEEE does not format its papers in
% such ways.
% Do not attempt to use stfloats with fixltx2e as they are incompatible.
% Instead, use Morten Hogholm'a dblfloatfix which combines the features
% of both fixltx2e and stfloats:
%
% \usepackage{dblfloatfix}
% The latest version can be found at:
% http://www.ctan.org/pkg/dblfloatfix




% *** PDF, URL AND HYPERLINK PACKAGES ***
%
%\usepackage{url}
% url.sty was written by Donald Arseneau. It provides better support for
% handling and breaking URLs. url.sty is already installed on most LaTeX
% systems. The latest version and documentation can be obtained at:
% http://www.ctan.org/pkg/url
% Basically, \url{my_url_here}.




% *** Do not adjust lengths that control margins, column widths, etc. ***
% *** Do not use packages that alter fonts (such as pslatex).         ***
% There should be no need to do such things with IEEEtran.cls V1.6 and later.
% (Unless specifically asked to do so by the journal or conference you plan
% to submit to, of course. )


% correct bad hyphenation here
\hyphenation{op-tical net-works semi-conduc-tor}


\begin{document}
%
% paper title
% Titles are generally capitalized except for words such as a, an, and, as,
% at, but, by, for, in, nor, of, on, or, the, to and up, which are usually
% not capitalized unless they are the first or last word of the title.
% Linebreaks \\ can be used within to get better formatting as desired.
% Do not put math or special symbols in the title.
\title{Transforming Sensor Data to the Image Domain for Deep Learning - an Application to Footstep Detection}


% author names and affiliations
% use a multiple column layout for up to three different
% affiliations

%\author{\IEEEauthorblockN{Author 1\IEEEauthorrefmark{1}\textsuperscript{\footnotemark{1}},
%Author 2\IEEEauthorrefmark{1}\textsuperscript{\footnotemark{1}},
%Author 3\IEEEauthorrefmark{2}, \\
%Author 4\IEEEauthorrefmark{3} and 
%Author 5\IEEEauthorrefmark{4}}
%\IEEEauthorblockA{\IEEEauthorrefmark{1}Affiliation 1}
%\IEEEauthorblockA{\IEEEauthorrefmark{2}Affiliation 2}
%\IEEEauthorblockA{\IEEEauthorrefmark{3}Affiliation 3}
%\IEEEauthorblockA{\IEEEauthorrefmark{4}Affiliation 4}}

%\author{\IEEEauthorblockN{Monit Shah Singh\IEEEauthorrefmark{1}\textsuperscript{\footnotemark{1}},
%Vinaychandran Pondenkandath\IEEEauthorrefmark{2}\textsuperscript{\footnotemark{1}},
%Bo Zhou\IEEEauthorrefmark{3}, \\
%Paul Lukowicz\IEEEauthorrefmark{4} and 
%Marcus Liwicki\IEEEauthorrefmark{5}}
%\IEEEauthorblockA{\IEEEauthorrefmark{1}TU Kaiserslautern, Germany,\\
%$Monit\_Shah.Singh@dfki.de$}
%\IEEEauthorblockA{\IEEEauthorrefmark{2}MindGarage, University of Kaiserslautern, Germany and DIVA group, University of Fribourg,
%$vinaychandran.pondenkandath@unifr.ch$}
%\IEEEauthorblockA{\IEEEauthorrefmark{3}DFKI Kaiserslautern, Germany,
%$Bo.Zhou@dfki.de$}
%\IEEEauthorblockA{\IEEEauthorrefmark{4}DFKI and 
%TU Kaiserslautern, Germany,
%$Paul.Lukowicz@dfki.de$}
%\IEEEauthorblockA{\IEEEauthorrefmark{5}MindGarage, University of Kaiserslautern, Germany and DIVA group, University of Fribourg`,
%$liwicki@cs.uni\mbox{-}kl.de$}}

\author{\IEEEauthorblockN{Monit Shah Singh\textsuperscript{\footnotemark{1}}\IEEEauthorrefmark{1},
		Vinaychandran Pondenkandath\textsuperscript{\footnotemark{1}}\IEEEauthorrefmark{2}\IEEEauthorrefmark{4},
		Bo Zhou\IEEEauthorrefmark{3}, \\
		Paul Lukowicz\IEEEauthorrefmark{1}\IEEEauthorrefmark{3} and 
		Marcus Liwicki\IEEEauthorrefmark{2}\IEEEauthorrefmark{4}}
	\IEEEauthorblockA{\IEEEauthorrefmark{1}TU Kaiserslautern, Germany}
	\IEEEauthorblockA{\IEEEauthorrefmark{2}MindGarage, TU Kaiserslautern, Germany}
	\IEEEauthorblockA{\IEEEauthorrefmark{3}DFKI, Kaiserslautern, Germany}
	\IEEEauthorblockA{\IEEEauthorrefmark{4}DIVA, University of Fribourg, Switzerland}
	\IEEEauthorblockA{$Monit\_Shah.Singh@dfki.de$, $vinaychandran.pondenkandath@unifr.ch$} \IEEEauthorblockA{$Bo.Zhou@dfki.de$, $Paul.Lukowicz@dfki.de$, $liwicki@cs.uni\mbox{-}kl.de$}
}

% conference papers do not typically use \thanks and this command
% is locked out in conference mode. If really needed, such as for
% the acknowledgment of grants, issue a \IEEEoverridecommandlockouts
% after \documentclass

% for over three affiliations, or if they all won't fit within the width
% of the page, use this alternative format:
% 
%\author{\IEEEauthorblockN{Michael Shell\IEEEauthorrefmark{1},
%Homer Simpson\IEEEauthorrefmark{2},
%James Kirk\IEEEauthorrefmark{3}, 
%Montgomery Scott\IEEEauthorrefmark{3} and
%Eldon Tyrell\IEEEauthorrefmark{4}}
%\IEEEauthorblockA{\IEEEauthorrefmark{1}School of Electrical and Computer Engineering\\
%Georgia Institute of Technology,
%Atlanta, Georgia 30332--0250\\ Email: see http://www.michaelshell.org/contact.html}
%\IEEEauthorblockA{\IEEEauthorrefmark{2}Twentieth Century Fox, Springfield, USA\\
%Email: homer@thesimpsons.com}
%\IEEEauthorblockA{\IEEEauthorrefmark{3}Starfleet Academy, San Francisco, California 96678-2391\\
%Telephone: (800) 555--1212, Fax: (888) 555--1212}
%\IEEEauthorblockA{\IEEEauthorrefmark{4}Tyrell Inc., 123 Replicant Street, Los Angeles, California 90210--4321}}




% use for special paper notices
%\IEEEspecialpapernotice{(Invited Paper)}

\IEEEoverridecommandlockouts
\IEEEpubid{\makebox[\columnwidth]{
		\copyright2017 IEEE \hfill} \hspace{\columnsep}\makebox[\columnwidth]{ }}
\IEEEpubid{\makebox[\columnwidth]{10.1109/IJCNN.2017.7966182~
		\copyright2017 IEEE \hfill} \hspace{\columnsep}\makebox[\columnwidth]{ }}


% make the title area
\maketitle

% As a general rule, do not put math, special symbols or citations
% in the abstract
%\begin{abstract}
%In this paper we propose to shift the problem domain towards image data in order to make deep Convolutional Neural Networks (CNN) applicable on raw sensor data. While CNN have become the state of the art in object recognition in various domains, they are still premature on other data. Especially for specific sensory data they have not been applicable yet as not much labelled training data is available.

%In this paper we introduce the idea of shifting the problem domain towards image data and using transfer learning for interpreting this data. In particular, we investigate data from pressure sensors, convert it into heat-map images. Then we transfer from pre-trained image-recognition CNNs Our novel algorithm clearly outperforms the state-of-the-art in ... (And put the numbers)
%\end{abstract}

\begin{abstract}
Convolutional Neural Networks (CNNs) have become the state-of-the-art in various computer vision tasks, but they are still premature for most sensor data, especially in pervasive and wearable computing. A major reason for this is the limited amount of annotated training data. In this paper, we propose the idea of leveraging the discriminative power of pre-trained deep CNNs on 2-dimensional sensor data by transforming the sensor modality to the visual domain. By three proposed strategies, 2D sensor output is converted into pressure distribution imageries. Then we utilize a pre-trained CNN for transfer learning on the converted imagery data. We evaluate our method on a gait dataset of floor
surface pressure mapping. We obtain a classification accuracy of 87.66\%, which outperforms the conventional machine learning methods by over 10\%. 
\end{abstract}



% no keywords


\footnotetext[1]{These two authors contributed equally to this work.}

% For peer review papers, you can put extra information on the cover
% page as needed:
% \ifCLASSOPTIONpeerreview
% \begin{center} \bfseries EDICS Category: 3-BBND \end{center}
% \fi
%
% For peerreview papers, this IEEEtran command inserts a page break and
% creates the second title. It will be ignored for other modes.
\IEEEpeerreviewmaketitle

\begin{figure*}
	\centering
		\includegraphics[width=15cm]{./figures/step_frames.pdf}
	\caption{The step images obtained after modality transformation of pressure sensor data (a) all frames in a sequence of single step (walking direction is upwards), (b) average of all the frames in a sequence.}
	\label{fig:step_frames}
\end{figure*}





\section{Introduction}
 A \textit{K\"ahler group} is a group that can be realised as the fundamental group of a closed K\"ahler manifold. The question of which finitely presented groups are K\"ahler was first raised by Serre in the 1950s and has driven a field of very active research since. While numerous strong constraints have been proved and examples of K\"ahler groups with a variety of different properties have been constructed, the question remains wide open. For a general background on K\"ahler groups see \cite{ABCKT-95}, for a more recent overview see \cite{Bur-10}. 

 While a general answer seems out of reach for the moment, it is fruitful to consider Serre's question in the context of more specific classes of groups. For instance, it has been shown that if the fundamental group of a compact 3-manifold without boundary is K\"ahler then it is finite \cite{DimSuc-09} (see also \cite{BisMjSes-12} and \cite{Kot-12-II}) and that a K\"ahler group with non-trivial first L2-Betti number is commensurable to a \textit{surface group} (i.e. the fundamental group of a closed Riemann surface) \cite{Gro-89}. Delzant and Py showed that if a K\"ahler group acts geometrically on a locally finite CAT(0) cube complex, then it is commensurable to a direct product of finitely many surface groups and a free abelian group \cite{DelPy-16}. 
 
 
 More generally, a close connection between K\"ahler groups acting on CAT(0) cube complexes and subgroups of direct products of surface groups has been observed starting with the work of Delzant and Gromov on cuts in K\"ahler groups \cite{DelGro-05} (see also \cite{Py-13, DelPy-16}). This led Delzant and Gromov to pose the question of which K\"ahler groups are subgroups of direct products of surface groups? Following the work of Bridson, Howie, Miller and Short \cite{BriHowMilSho-02, BriHowMilSho-09}, one knows that this question is intimately related to the question of finding K\"ahler groups which are not of \textit{finiteness type} $\mathcal{F}_r$ for some $r$, i.e. do not admit a classifying space with finite $r$-skeleton: any subgroup of a direct product of $k$ surface groups which is $\mathcal{F}_k$ is virtually a direct product of surface groups and finitely generated free groups.
 
 The first examples of K\"ahler subgroups of direct products of surface groups which are of type  $\mathcal{F}_{r-1}$ but not $\mathcal{F}_r$ ($r\geq 3$) were constructed by Dimca, Papadima and Suciu \cite{DimPapSuc-09-II}. Their class of examples has since been extended by Biswas, Mj and Pancholi \cite{BisMjPan-14} and by the author \cite{Llo-16-II}. All of these examples arise as kernels of surjective homomorphisms of the form $\pi_1S_{g_1}\times \cdots \times \pi_1 S_{g_r}\rightarrow \ZZ^2$ where $r\geq 3$ and $S_{g_i}$ is a closed Riemann surface of genus $g_i\geq 2$, $1\leq i \leq r$. Recently, examples of K\"ahler groups that are of type $\mathcal{F}_{r-1}$ but not of type $\mathcal{F}_r$, and which are not commensurable to any subgroup of a direct product of surface groups have been constructed by Bridson and the author \cite{BriLlo-16}. 
 
 This paper consists of three parts. In the first part (Section 2) we develop a new construction method for K\"ahler groups. The groups obtained from this method arise as fundamental groups of fibres of holomorphic maps onto higher-dimensional complex tori. In the second and third part we address Delzant and Gromov's question. In the second part (Sections 3 -- 5) we apply our construction method to provide K\"ahler subgroups of direct products of surface groups that are not commensurable with any of the previous examples. These arise as kernels of a surjective homomorphism onto $\ZZ^{2k}$ and are \textit{irreducible}, i.e. do not decompose as direct product of two nontrivial groups (even virtually). The examples constructed in this work significantly extend the range of irreducible full subdirect K\"ahler subgroups of direct products of surface groups: all previous examples of such K\"ahler subgroups of a product of $r$ surface groups are either virtually a product of surface groups and a free abelian group, or of type $\mathcal{F}_{r-1}$, but not of type $\mathcal{F}_r$. Here we produce irreducible examples of type $\mathcal{F}_k$ and not of type $\mathcal{F}_{k+1}$ for all $2\leq k \leq r-1$, hence covering the full range of possible finiteness properties \cite{BriHowMilSho-02, BriHowMilSho-09}. In the third part (Sections 6 -- 9) we give new constraints of K\"ahler subgroups of direct products of surface groups. In particular, we show that if a full subdirect product of $r$ surface group is K\"ahler of type $\mathcal{F}_k$ with $k> \frac{r}{2}$ then it is virtually the kernel of an epimorphism from the product of surface groups onto a free abelian group of even rank.
 


One says that a surjective holomorphic map $h: X\rightarrow Y$ between compact complex manifolds has \textit{isolated singularities} if the critical locus of $h$ intersects each fibre (preimage of a point) in a discrete subset. The key result in our construction method is Theorem \ref{thmFiltVerGen}, a special case of which is:
 
 \begin{theorem}
 Let $X$ be a compact complex manifold of dimension $n+k$ and let $Y$ be a complex torus of dimension $k$. Let $h:X\rightarrow Y$ be a surjective holomorphic map with connected smooth generic fibre $H$. Assume that there is a filtration
 \[
  \left\{0\right\} \subset Y^0\subset Y^1 \subset \cdots \subset Y^{k-1}\subset Y^k=Y
 \]
of $Y$ by complex subtori $Y^l$ of dimension $l$ such that the projections
\[
h_l=\pi_l\circ h: X\rightarrow Y/Y^{k-l}
\]
have isolated singularities, where $\pi_l: Y\rightarrow Y/Y^{k-l}$ is the holomorphic quotient homomorphism.

If $n=\mathrm{dim}H\geq 2$, then the map $h$ induces a short exact sequence 
\[
 1 \rightarrow \pi_1 H \rightarrow \pi_1 X \rightarrow \pi_1 Y= \ZZ^{2k}\rightarrow 1.
\]
Furthermore, we obtain that $\pi_i(X,H)=0$ for $2\leq i \leq \mathrm{dim}H$.
\label{thmFiltVer}
\end{theorem} 

Theorem \ref{thmFiltVer} and Theorem \ref{thmFiltVerGen} are generalisations of \cite[Theorem C]{DimPapSuc-09-II} and \cite[Theorem 2.2]{BriLlo-16}. We expect that our methods can be applied to construct interesting new classes of K\"ahler groups. Indeed we provide a first application in this work, by constructing new classes of K\"ahler subgroups of direct products of surface groups.

\begin{notation*}
Throughout this article $S_{g}$ will denote a closed hyperbolic surface of genus $g\geq 2$ and $\Gamma_{g}=\pi_1 S_{g}$ its fundamental group.
\end{notation*}

 \begin{theorem}
  Let $r\geq 3$ and $r-2\geq k \geq 1$ be integers and let $E$ be an elliptic curve (i.e. a complex torus of dimension one). For $1\leq i \leq r$ let $\alpha_i: S_{\g_i}\rightarrow E$ be a branched cover of $E$, with $\g_i\geq 2$ and assume that $\alpha_i$ is surjective on fundamental groups for $1\leq i \leq r$. Then there is a surjective holomorphic map 
  \[
   h: S_{\g_1}\times \cdots \times S_{\g_r}\rightarrow E^{\times k}
  \] 
  with smooth generic fibre $\overline{H}$ such that the restriction of $h$ to each factor $S_{\g_i}$ factors through $\alpha_i$. The map $h$ induces a short exact sequence
  \[
   1\rightarrow \pi_1 \overline{H}\rightarrow  \pi_1 S_{\g_1}\times \cdots \times \pi_1 S_{\g_r}\stackrel{h_{\ast}}{\rightarrow} \pi_1 E^{\times k} \cong \ZZ^{2k}\rightarrow 1
  \]
  and the group $\pi_1 \overline{H}$ is K\"ahler of type $\mathcal{F}_{r-k}$ but not of type $\mathcal{F}_{r-k+1}$. Furthermore, $\pi_1 \overline{H}$ is irreducible.
  \label{thmIntroA}
 \end{theorem}  

Here we use the notation $E^{\times k}=\underbrace{E\times \dots \times E}_{\mbox{$k$ times}}$ for the cartesian product of $k$ copies of $E$. The coabelian subgroups of direct products of surface groups form an important subclass of the class of all subgroups of direct products of surface groups. Indeed, in the case of three factors any finitely presented full subdirect subgroup of $D=\pi_1 S_{\g_1}\times \pi_1 S_{\g_2}\times \pi_1 S_{\g_3}$ is virtually \textit{coabelian}, i.e. contains the derived subgroup $\left[D_0,D_0\right]$ of some $D_0\leq D$ of finite index; with more factors any full subdirect subgroup is virtually conilpotent \cite{BriHowMilSho-13}. We will give a more detailed discussion of subgroups of direct products of surface groups in Section \ref{secNotProd}. 
 
We will see that the examples we construct to prove Theorem \ref{thmIntroA} are in some sense the generic class of examples with the property that the image of each factor in $E^{\times k}$ is an elliptic curve. As a consequence of Theorem \ref{thmIntroA} and its proof we obtain that there are indeed K\"ahler groups covering the full range of possible finiteness properties of irreducible full subdirect products of surface groups.
 
 \begin{theorem}
  For every $r \geq 3$, $\g_1,\cdots,\g_r\geq 2$ and $r-1\geq m\geq 2$, there is a K\"ahler subgroup $G\leq \Gamma_{\g_1}\times \cdots \times \Gamma_{\g_r}$ which is an irreducible full subdirect product of type $\mathcal{F}_m$ but not of type $\mathcal{F}_{m+1}$.
  \label{corthmIntroA}
 \end{theorem}

We will see that a modification of the construction used to prove Theorem \ref{thmIntroA} provides a second class of examples (see Theorem \ref{thmExtendedRange}). As a consequence we see that Theorem \ref{corthmIntroA} can also be proved by considering only holomorphic maps to a product of two elliptic curves. The reduction in dimension of the complex torus will come at the cost of loosing the genericity of the examples in Theorem \ref{thmIntroA}.

Conversely, we address the question of finding constraints on K\"ahler subgroups of direct products of surface groups, or more generally on K\"ahler groups that admit homomorphisms to direct products of surface groups. 

\begin{theorem}
\label{thmNewCoab}
 Let $G=\pi_1 X$ with $X$ compact K\"ahler and let $\phi: G \rightarrow \overline{G}$ be a surjective homomorphisms onto a subgroup $\overline{G}\leq \G_{g_1}\times \dots \times \G_{g_r}$. Assume that $\phi$ has finitely generated kernel and that $\overline{G}$ is full and of type $\mathcal{F}_m$ for $m\geq 2$.
 
 Then, after reordering factors, there is $s\geq 0$ such that, for any $k< 2m$ and any $1\leq i_1 < \dots < i_k\leq s$, the projection $p_{i_1,\dots,i_k}(\overline{G})\leq \G_{g_{i_1}}\times \dots \times \G_{g_{i_k}}$ is virtually coabelian of even rank. Furthermore, the center $\mathrm{Z}(\overline{G})= \overline{G} \cap \left(\G_{g_{s+1}}\times \dots \times \G_{g_r}\right)\leq p_{s+1,\dots,r}(\overline{G})\cong \ZZ^{r-s}$ is a finite index subgroup. 
\end{theorem}

Combining Theorem \ref{thmNewCoab} with a study of the first Betti number of coabelian subdirect products of groups in Section \ref{secFinPropBetti}, allows us to show that there are non-K\"ahler subgroups of direct products of surface groups with interesting properties.

\begin{corollary}
\label{corNewCoabExs}
 Let $G=\ker \psi$ for $\psi: \G_{g_1}\times \dots\times \G_{g_r}\rightarrow \ZZ^{2l+1}$ an epimorphism. Then $\ker \psi$ is not K\"ahler.
\end{corollary}



\begin{corollary}
 For $r\geq 6$ and $g_1,\dots,g_r\geq 2$ there is a non-K\"ahler full subdirect product $G\leq \G_{g_1}\times \dots \times \G_{g_r}$ with even first Betti number.
 \label{corExEvenB1Intro}
\end{corollary}










\subsection*{Structure:} 


This work is structured as follows: In Section \ref{secMainThm} we prove Theorem \ref{thmFiltVerGen}. In Sections \ref{secExamples} and \ref{secExCombined} we construct large new classes of K\"ahler subgroups of direct products of surface groups, which we use to prove Thereoms \ref{thmIntroA} and \ref{corthmIntroA}. In Section \ref{secNotProd} we show that these examples are irreducible, i.e. are not virtually direct products, and derive there precise finiteness properties. In Section \ref{secResCoabKGs} we study homomorphisms from K\"ahler groups to direct products of surface groups and show Theorem \ref{thmNewCoab} and Corollary \ref{corNewCoabExs}. In Section \ref{secSESCoab} we study the first Betti number of coabelian subgroups of direct products of groups and prove Corollary \ref{corExEvenB1Intro}. We apply the results of Section \ref{secFinPropBetti} to obtain additional constraints on homomorphisms from K\"ahler groups to direct products of surface groups. In Section \ref{secConsGens} we consider the universal homomorphism from a K\"ahler group to a direct product of Riemann orbisurfaces and we explain why our constraints apply to it.

 
\begin{acknowledgements*}
I am very grateful to my PhD advisor Martin Bridson for his generous support and the many very helpful discussions we had about the contents of this paper, and to Simon Donaldson for inspiring conversations about topics related to the contents of this paper.

This article is a fundamentally rewritten and extended version of arXiv:1701.01163v2. Some of the new material contained in this work is based on results from the authors PhD thesis. 
\end{acknowledgements*}
\begin{figure}
	\centering
	\includegraphics[width=8cm]{./figures/imagenet.pdf}
	\caption{Examples of easily visually interpretable images, (a) Magnetic Resonance Imaging (MRI) scan, (b) X-ray scan, (c) Feline, (d) Canine }
	\label{fig:imagenet}
\end{figure}
\input{Pressure_sensor_data}
\input{Modality_transformation}
\begin{figure}
	\centering
		\includegraphics[width=8cm]{./figures/max_avg_frame.pdf}
	\caption{Schematic diagram of max and average(of sequence) frame classification pipeline.}
	\label{fig:ann_pipeline}
\end{figure}

\input{Inception-v3}

\input{Evaluation}
\section{Discussion}
\label{sect:discussion}

\begin{figure}
\includegraphics[width=\columnwidth]{plots/AvSFRvsBHAR}

\caption{Each line shown here equates the median trends from the top two panels
of \cref{fig:avHistory_vs_hm} to give the 100~Myr average SFR as a function of
the 100~Myr average BHAR (in equal spacings of halo mass). Region \emph{A}
(shaded blue) corresponds to galaxies hosted by haloes with \M{200}
$\lesssim$\M{crit}. Galaxies in this regime increase their SFR with increasing
halo mass, while BHARs remain negligible on average. As haloes reach
$\sim$\M{crit} in region \emph{B} (shaded green), SFRs continue to rise,
however the BH growth increases by many orders of magnitude over this narrow
halo mass range. For haloes in excess of $\gtrsim$ \M{crit} shown in region
\emph{C} (shaded red), we see a reduction for both SFR and BHAR on average,
yielding a approximately constant scaling between the two growth rates (compare
to dashed green line which shows the linear relation \BHAR/\SFR = $10^{-3}$).}

\label{fig:sfr_vs_bhar_av} \end{figure}

Throughout this investigation we have consistently found no evidence supporting
a simple underlying relationship between the rate of a galaxy's star formation
and the accretion rate of its central BH. Instead, a mutual dependence of each
property upon the mass of the host halo yields a more complex connection. It is
interesting to examine, then, how the relation between the SFR and BHAR evolves
for individual objects. In the following discussion, we will provide a physical
interpretation based on the \citet{Bower2017} (hereafter B16) model for BH
growth \citep[for a similar interpretation on the importance of SN feedback to
BH growth see][]{Dubois2015,Habouzit2016}.  However, we stress the simulation
results are themselves independent of any physical interpretation.  

\cref{fig:sfr_vs_bhar_av} equates the median trends of the SFR and BHAR
histories shown in \cref{fig:avHistory_vs_hm}. This specifies the 100~Myr
average SFR as a function of the 100~Myr average BHAR in equal spacings of halo
mass.  Three distinct trends between SFR and BHAR emerge as the halo evolves:
the \emph{stellar feedback regulated} phase (shaded blue), the \emph{non-linear
BH growth} phase (shaded green) and the \emph{AGN feedback regulated} phase
(shaded red).

\begin{itemize}

\item \emph{Region A - The stellar feedback regulated phase}: From the time of
their seeding until they are hosted by haloes of mass \M{200} $\sim
10^{11.5}$\Msol the BH accretion rates are negligible (\BHAR $\leq
10^{-6}$\Msolyr on average).  By contrast, SFRs increase steadily with halo
mass. This behaviour produces the uncorrelated (yet causally connected)
$\sim$vertical trend in region \emph{A}, creating an imbalance of growth within
these systems. As a result, BHs remain close to their seed mass whilst the
halo/galaxy continues to grow around them (see the low-mass region of
\cref{fig:bhm_vs_hm}).

B16 interpret galaxies in this regime as being in a state of regulatory
equilibrium. Energy injected by stars heats the ISM within the stellar vicinity,
ejecting it, and causing it to rise buoyantly in the halo. This in turn creates
an outflow of material balancing the freshly sourced fuel from the cosmic web,
and as such prevents large gas densities from building up within the inner
regions of these low-mass galaxies.  Such low densities, coupled with the
relatively low mass BHs living within these galaxies (BHAR $\propto
M_{\mathrm{BH}}^{2}$), ensures that BHs fail to grow substantially. 

\item \emph{Region B - The non-linear BH growth phase}: Both galaxies and BHs
grow through the halo mass range \M{200} $\sim 10^{11.5} - 10^{12.0}$\Msol.
However, whereas the SFRs continue to increase steadily with increasing halo
mass, BHs rapidly transition to a non-linear phase of growth. This creates a
highly non-linear \textit{indirect} correlation between SFR and BHAR, connected
through the host halo mass.  

The physical interpretation posited by B16 is that haloes that grow to the
transition mass, \M{crit}, have become sufficiently massive to stall the
regulatory outflow. Due to (what is now) the halos' hot coronae, heated gas
ejected by stellar feedback loses the capability to rise buoyantly and
therefore returns to the galaxy centre.  Densities in the central regions of
the galaxy are no longer kept low and a \squotes{switch} to non-linear BH
growth is triggered.  

\item \emph{Region C - The AGN feedback regulated phase}: For haloes with
masses above \M{200} $\sim 10^{12}$\Msol SFRs and BHARs both decline on
average, correlated with an approximately linear trend (compare to green dashed
line, see also bottom panel of \cref{fig:avHistory_vs_hm}). 

B16 argue that BHs in these haloes have become sufficiently massive (through
their rapid non-linear growth) to efficiently regulate the gas inflow onto the
galaxy themselves via AGN feedback.  This again creates an equilibrium state,
for which a fluctuating low level of (specific) BH accretion is maintained,
keeping the outer halo hot and evaporating much of the new cold material trying
to enter the system from the intergalactic medium.

\end{itemize}

Galaxies and their central BHs within the \eagle simulation transition through
multiple stages of growth as their host dark matter halo evolves, creating
three distinct behaviours between SFR and BHAR. This is a stark contrast to a
simple model where SFR and BHAR correlate globally via a linear relation, on
average and for all halo masses. Whilst the underlying trend is only revealed
when each growth rate is time-averaged (given the inherent noise of
instantaneous growth rates), we only find an approximately linear correlation
for the most massive systems (\M{200} $\gtrsim 10^{12.5}$\Msol).

In this paper we have emphasised the role of the halo and how its interaction
with both SFR and BHAR shapes the growth rate relationship.  However,
additional factors may also contribute to the form this relationship takes. For
example, \citet{Volonteri2015b} find using a suite of isolated merger
simulations at fixed halo mass, that alternate behaviours between SFR and BHAR
before, during and after the merger proper collectively contribute to form a
complex two-dimensional plane.  Additionally, \citet{Pontzen2017} reveal the
particular importance differing merger histories can have on significantly
altering the growth rate history of both that of the galaxy and the central BH.
However, the global influence of mergers upon galaxy and BH growth rates in a
full cosmological context remains open for debate, and will be the subject of a
future paper. 


\input{Conclusion}









% conference papers do not normally have an appendix


% use section* for acknowledgment
\section*{Acknowledgment}

%The acknowledgement is left out for blind review.
This research was partially supported by the Rheinland-Pfalz Foundation for Innovation (RLP), grant HiMigiac, HisDoc III project funded by the Swiss National Science Foundation with the grant number 205120-169618 and the iMuSciCA project funded by the EU (GA 731861). The authors would also like to thank all the experiment participants.
\par
The authors would like to thank Muhammad Zeshan Afzal and Akansha Bhardwaj for their valuable comments and the German Research Center for Artificial Intelligence (DFKI) and Insiders Technologies GmbH for providing the computational resources.





% trigger a \newpage just before the given reference
% number - used to balance the columns on the last page
% adjust value as needed - may need to be readjusted if
% the document is modified later
%\IEEEtriggeratref{8}
% The "triggered" command can be changed if desired:
%\IEEEtriggercmd{\enlargethispage{-5in}}

% references section

% can use a bibliography generated by BibTeX as a .bbl file
% BibTeX documentation can be easily obtained at:
% http://mirror.ctan.org/biblio/bibtex/contrib/doc/
% The IEEEtran BibTeX style support page is at:
% http://www.michaelshell.org/tex/ieeetran/bibtex/
%\bibliographystyle{IEEEtran}
% argument is your BibTeX string definitions and bibliography database(s)
%\bibliography{IEEEabrv,../bib/paper}
%
% <OR> manually copy in the resultant .bbl file
% set second argument of \begin to the number of references
% (used to reserve space for the reference number labels box)
\bibliographystyle{IEEEtran}
\bibliography{deepsensor_ref}

% that's all folks
\end{document}


