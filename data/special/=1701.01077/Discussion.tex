\section{Discussion}
We see that with the use of deep neural networks we achieve considerably better recognition results. The accuracy obtained with this method in case of average frames outperforms the baseline reported in our previous work which uses the conventional feature set for the same task \cite{blind}. It is noteworthy, because, the average frames are significantly more lossy than the data available to the conventional methods. When we use our method on all frames in the sequence of a step, we obtain an accuracy with an increase of over 10\% when compared to the accuracy of 76.9\% achieved by wavelet transformation. These results are directly comparable because both the methods evaluate a single step at a time for the person identification task.

Thus, we suggest three different approaches for generating the visual representations: max-frame, average-frame, and all frames in a sequence. While these approaches produce satisfactory results, finding a good representation for any other given sensor data can still be challenging if the data is of a very different nature. However, it is suggested to try these approaches even if the visual representations seem to be not convincing. A network transferred from an easily visually interpretable domain can still be able to distinguish the classes very well.

When visualizing the activations of the first and second convolutional layers (See Fig. \ref{fig:max_conv_vis}), we see a difference in the activations for the maximum and average frames in the first convolutional layers, despite both the input images being relatively similar. For example, it can be seen in \ref{fig:max_conv_vis} (b) and (e) [\emph{yellow}] that the shape of the foot is clearly distinguishable between the two images.

There can be many areas in which this approach can be implemented. Crowd-movement data generated from various sources can be used to model the traffic distribution over a geographical location; pollution particulate matter concentration over time can be visualized and considered as a time-series for prediction of air quality. Such distributed numerical data can be visualized on a geographical map or globe. The considered patterns can be assigned labels associated with some events of the world. For example, a pattern generated from the crowd-movement data in a city can signify the busiest parts of the city at any given time. In this case, a pre-trained CNN can be used to classify different types of city parts.

%In the second paragraph it would be nice to put a bit of analysis, something like:When investigating the activations in the final feature layer of the Inception... we found that ... neurons specialized ...
%Or: The features of the Inception... pre-trained on ImageNet data for the task of identifying objects has been found to be a good set for our task as well. Looking deeper into the extracted features, we see that ...


%it would be great if you could add some specific exmples. As I have already indicated in the mail, crowd-movement data could generate patterns which might be useful as well.